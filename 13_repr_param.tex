\chapter[Représentation paramétrique des courbes \life \eng]{Représentations paramétriques des courbes\\ \life \eng}\label{cha[ReprParam}

\compileTHEO{

Les deux premières sections de ce chapitre sont nécessaires en autre
pour le calcul vectoriel et l'étude des systèmes d'équations
différentielles que nous ferons prochainement.  Nous verrons comment
calculer l'intégrale de fonctions et champs de vecteurs le long de
courbes paramétriques dans le chapitre sur le calcul vectoriel.   Les
solutions d'un système d'équations différentielles sont des courbes
paramétriques.

Les dernières sections de ce chapitre offrent plus d'information sur
la représentation paramétrique des courbes et peuvent servir de
matériel optionnel pour les étudiants en génie.

\section{Définition}
 \

\begin{defn} \index{Représentation paramétrique d'une courbe}
Soit $I \subset \RR$ un sous-intervalle de la droite réelle.  De plus,
soit $\displaystyle \left\{ \phi_1, \phi_2, \ldots, \phi_n \right\}$
un ensemble de $n$ fonctions continues à valeurs réelles définies sur
$I$.

L'ensemble des points
\[
\Gamma = \left\{ (\phi_1(t), \phi_2(t), \ldots, \phi_n(t)) : t \in I\right\}
\]
définit une courbe (figure~\ref{PARAM1}) dans l'espace $\RR^n$.
Les $n$ fonctions $\phi_1$, $\phi_2$, \ldots, $\phi_n$ définissent une
fonction $\phi: \RR \rightarrow \RR^n$ par 
\begin{equation}\label{courbeD1}
\phi(t) = (\phi_1(t), \phi_2(t), \ldots, \phi_n(t))
\end{equation}
pour $t\in \RR$.  L'image de $\phi$ est la courbe $\Gamma$.  La
fonction $\phi$ est une {\bfseries représentation paramétrique} de
la courbe $\Gamma$.

Puisque les élements de $\RR^n$ sont bien souvent représentés par des
matrices de dimensions \nm{n}{1}, une matrice avec $n$ ligne et une
colonne, nous définissons aussi une représentation paramétrique
$\phi: \RR \rightarrow \RR^n$ par
\begin{equation}\label{courbeD2}
\phi(t) = \begin{pmatrix}
  \phi_1(t) \\ \phi_2(t) \\ \vdots \\ \phi_n(t)
\end{pmatrix}
\end{equation}
pour $t\in \RR$.

(\ref{courbeD1}) est généralement associée à la représentation
géométrique de la courbe alors (\ref{courbeD2}) associée à la
représentation algébrique.  Le contexte détermine quelle forme est
utilisée.
\end{defn}

\PDFfig{13_repr_param/param1}{Une courbe du plan donnée par une
représentation paramétrique}{Une courbe dans le plan donnée par une
représentation paramétrique $\phi(t) = (\phi_1(t), \phi_2(t))$ pour
$t\in \RR$. Nous indiquons à l'aide d'une flèche la direction du
déplacement le long de la courbe lorsque $t$ augmente.}{PARAM1}

\begin{egg}
Une représentation paramétrique du cercle $\Gamma$ de rayon $1$ centré
à l'origine est donnée par les fonctions trigonométriques
$\cos(\theta)$ et $\sin(\theta)$.  Plus précisément, une
représentation paramétrique de cette courbe est donnée par
\[
(x,y) = (\cos(\theta), \sin(\theta))
\]
pour $\theta \in \RR$.

\PDFgraph{13_repr_param/param2}

Lorsque $\theta$ augmente, nous parcourons le cercle unité dans le sens
contraire aux aiguilles d'une montre.  Puisque $\cos(\theta)$ et
$\sin(\theta)$ sont des fonctions périodiques de période $2\pi$, nous
parcourons la courbe $\Gamma$ au complet à chaque fois que $\theta$
augmente de $2\pi$.
\end{egg}

\begin{egg}
Une autre représentation paramétrique du cercle $\Gamma$ de rayon
$1$ centré à l'origine est donnée par
\[
(x,y) = (\cos(2\theta), \sin(2\theta))
\]
pour $\theta \in \RR$.  Lorsque $\theta$ augmente, nous parcourons le
cercle unité dans le sens contraire aux aiguilles d'une
montre. Puisque $\cos(2\theta)$ et $\sin(2\theta)$ sont des fonctions
périodiques de période $\pi$, nous parcourons la courbe $\Gamma$ au
complet à chaque fois que $\theta$ augmente de $\pi$.  Cette
représentation permet de parcourir la courbe $\Gamma$ deux fois plus
rapidement que la représentation donnée à l'exemple précédent.
\end{egg}

Les deux exemples précédents montrent qu'une courbe $\Gamma$ a plus
qu'une représentation paramétrique.

\begin{egg}
Quelle est la courbe $\Gamma$ décrite par la représentation
paramétrique
\[
(x,y) = (\sec(\theta) , \tan(\theta))
\]
pour $-\pi/2 < \theta < \pi/2$?

Puisque $\tan^2(\theta) +1 = \csc^2(\theta)$, nous obtenons que
$y^2 + 1 = x^2$.  La courbe $\Gamma$ fait donc partie de la courbe
décrite par $x^2-y^2=1$.  L'équation $x^2-y^2=1$ est l'équation d'une
hyperbole dont l'axe est la droite horizontale $y=0$ et les sommets
sont aux points $(1,0)$ et $(-1,0)$.
\PDFgraph{13_repr_param/param3}

Par contre, la représentation paramétrique ne représente pas toute
l'hyperbole.  Nous avons $x = \sec(\theta) >0$ pour
$-\pi/2<\theta<\pi/2$,
$\displaystyle \lim_{\theta \to \pi/2} \sec(\theta) = 
\lim_{\theta \to \pi/2} \sec(\theta) =  +\infty$, 
$\displaystyle \lim_{\theta \to -\pi/2} \tan(\theta) = +\infty$,
$\displaystyle \lim_{\theta \to -\pi/2} \tan(\theta) =  -\infty$
et $(\sec(0), \tan(0)) = (1,0)$.  La courbe $\Gamma$ est donc la
branche de l'hyperbole à droite de l'axe des $y$ que nous parcourons de
bas en haut lorsque $\theta$ augmente.
\end{egg}

\begin{egg}
Quelle est la courbe $\Gamma$ décrite par la représentation
paramétrique
\[
(x,y) = (2t-1, t^2-1)
\]
pour $t\in \RR$?

De l'équation $x= 2t-1$, nous trouvons $t = (x+1)/2$.  Si nous substituons
cette expression pour $t$ dans l'équation $y=t^2-1$, nous obtenons
$y = (x+1)^2/4 -1$.  Cette dernière équation donne
\[
(x+1)^2 = 4(y+1) \; .
\]
C'est l'équation d'une parabole convexe dont l'axe est la droite
verticale $x=-1$ et le sommet est au point $(-1,-1)$.
\PDFgraph{13_repr_param/param4}

Puisque $x=2t-1$ augmente de $-\infty$ à $+\infty$ lorsque $t$
augmente de $-\infty$ à $+\infty$, la courbe est parcourue de gauche à
droite.
\end{egg}

\section{Droite tangente à une courbe}\label{DTcourbe}

Supposons qu'une courbe $\Gamma$ possède la représentation
paramétrique
\[
\VEC{x} = \phi(t)
\]
ou, plus explicitement,
\[
(x_1,x_2, \ldots, x_n) = (\phi_1(t), \phi_2(t), \ldots, \phi_n(t))
\]
pour $t \in ]a,b[ \subset \RR$, où les fonctions
$\phi_i:]a,b[ \rightarrow \RR$ sont différentiables.  Nous ne demandons
pas seulement que les fonctions $\phi_i$ soient continues mais
qu'elles soient aussi différentiables.

Pour trouver la direction de la droite tangente à la courbe $\Gamma$
au point
\[
\phi(t_0) = (\phi_1(t_0), \phi_2(t_0), \ldots, \phi_n(t_0)) \; ,
\]
nous considérons la sécante qui passe par les points
\[
\phi(t_0) = (\phi_1(t_0), \phi_2(t_0), \ldots, \phi_n(t_0))
\ \text{et} \ 
\phi(t_0+h) = (\phi_1(t_0+h), \phi_2(t_0+h), \ldots, \phi_n(t_0+h))
\]
où $h$ est petit. Le vecteur
\begin{align*}
\VEC{v}_h &= \frac{1}{h} \big(\phi(t_0+h)-\phi(t_0)\big) \\
&= \frac{1}{h}\big(\phi_1(t_0+h) - \phi_1(t_0),
\phi_2(t_0+h) - \phi_2(t_0), \ldots , \phi_n(t_0+h)-\phi_n(t_0)\big) \\
&= \left(\displaystyle \frac{\phi_1(t_0+h) - \phi_1(t_0)}{h},
\displaystyle \frac{\phi_2(t_0+h) - \phi_2(t_0)}{h}, \ldots ,
\displaystyle \frac{\phi_n(t_0+h) - \phi_n(t_0)}{h}\right)
\end{align*}
est parallèle à la sécante qui passe par les points $\phi(t_0)$ et
$\phi(t_0+h)$.

Lorsque $h$ tend vers $0$, le vecteur $\VEC{v}_h$ approche un vecteur
parallèle à la droite tangente à $\Gamma$ au point $\phi(t_0)$.  Or,
cette limite est
\begin{align*}
\lim_{h\rightarrow 0}\VEC{v}_h &=
\left(\displaystyle \lim_{h\rightarrow 0}\frac{\phi_1(t_0+h) - \phi_1(t_0)}{h},
\displaystyle \lim_{h\rightarrow 0}\frac{\phi_2(t_0+h) - \phi_2(t_0)}{h},
 \ldots ,
\displaystyle \lim_{h\rightarrow 0}\frac{\phi_n(t_0+h)
- \phi_n(t_0)}{h}\right) \\
&= (\phi_1'(t_0), \phi_2'(t_0), \ldots, \phi_n'(t_0)) \; ,
\end{align*}

Nous avons donc le résultat suivant.

\begin{prop}
Soit $\Gamma$, une courbe qui possède la représentation paramétrique
\[
(x_1,x_2, \ldots, x_n) =
\phi(t) = (\phi_1(t), \phi_2(t), \ldots, \phi_n(t))
\]
pour $t \in ]a,b[ \subset \RR$, où les fonctions
$\phi_i:]a,b[ \rightarrow \RR$ sont différentiables.
Le vecteur
\[
\phi'(t_0) = (\phi_1'(t_0), \phi_2'(t_0), \ldots, \phi_n'(t_0))
\]
est parallèle à la droite tangente à la courbe $\Gamma$ au point
$\phi(t_0)$ (figure~\ref{PARAM5}).  De plus, le vecteur $\phi'(t_0)$
pointe dans la direction associée à $t$ croissant.

Lorsque les éléments de $\RR^n$ sont représentés par des matrices de
dimension \nm{n}{1}, nous avons
\[
\phi(t) = \begin{pmatrix}
  \phi_1(t) \\ \phi_2(t) \\ \vdots \\ \phi_n(t)
\end{pmatrix}
\qquad \text{et} \qquad
\phi'(t) = \begin{pmatrix}
  \phi_1'(t) \\ \phi_2'(t) \\ \vdots \\ \phi_n'(t)
\end{pmatrix}
\]
Le contexte détermine quelle formulation doit être utilisée.
\end{prop}

\PDFfig{13_repr_param/param5}{tangente à une courbe donnée par une
représentation paramétrique}{La courbe $\Gamma$ donnée par la
représentation paramétrique $(x_1,x_2)=(\phi_1(t),\phi_2(t))$ pour
$t\in \RR$.  La sécante qui passe par $\phi(t_0)$ et $\phi(t_0+h)$, et
la tangente à la courbe au point $\phi(t_0)$ sont aussi incluses.  Le
vecteur $\VEC{v}_h$ est parallèle à la sécante.}{PARAM5}

Ainsi, la droite tangente à la courbe $\Gamma$ au point $\phi(t_0)$
possède la représentation paramétrique
\begin{align*}
(x_1,x_2, \ldots, x_n) &= \phi(t_0) + \alpha\, \phi'(t_0) \\
&= (\phi_1(t_0), \phi_2(t_0), \ldots, \phi_n(t_0))
+ \alpha (\phi_1'(t_0), \phi_2'(t_0), \ldots, \phi_n'(t_0)) \\
&= (\phi_1(t_0) + \alpha \phi_1'(t_0), \phi_2(t_0)+ \alpha
\phi_2'(t_0),  \ldots, \phi_n(t_0) + \alpha \phi_n'(t_0))
\end{align*}
pour $\alpha \in \RR$.

\begin{rmk}
Soit $\Gamma$ une courbe dans $\RR^2$ (i.e. $n=2$ dans la discussion
précédente) qui possède la représentation paramétrique
\[
(x,y) = (\phi_1(t), \phi_2(t))
\]
pour $t \in ]a,b[ \subset \RR$, où les fonctions
$\phi_1:]a,b[ \rightarrow \RR$ et $\phi_2:]a,b[ \rightarrow \RR$
sont différentiables.

Si $\phi_1'(t_0) = 0$ alors la tangente à la courbe $\Gamma$ au point
$(\phi_1(t_0), \phi_2(t_0))$ est verticale car la tangente est
parallèle au vecteur
$(\phi_1'(t_0), \phi_2'(t_0)) = (0,\phi_2'(t_0))$.

Si $\phi_2'(t_0) = 0$ alors la tangente à la courbe $\Gamma$ au point
$(\phi_1(t_0), \phi_2(t_0))$ est horizontale car la tangente est
parallèle au vecteur $(\phi_1'(t_0), \phi_2'(t_0)) = (\phi_1'(t_0),0)$.
\end{rmk}

\begin{rmk}
Soit $\Gamma$ une courbe dans $\RR^2$ qui possède la représentation
paramétrique
\[
(x,y) = (\phi_1(t), \phi_2(t))
\]
pour $t \in ]a,b[ \subset \RR$, où les fonctions
$\phi_1:]a,b[ \rightarrow \RR$ et $\phi_2:]a,b[ \rightarrow \RR$
sont différentiables.

Il est possible de démontrer que si $\phi_1'(t_0)\neq 0$ alors nous pouvons
représenter $\Gamma$ près du point $(\phi_1(t_0),\phi_2(t_0))$ par une
expression de la forme $y=f(x)$ où $f$ est aussi différentiable.  En
effet, si $\phi_1'(t_0) \neq 0$ alors $\phi_1$ est strictement
croissant ou décroissant au voisinage de $t_0$.  Ce qui implique que
$\phi_1$ est localement inversible au voisinage de $t_0$.  Nous avons
donc $t=\phi_1^{-1}(x)$ pour $x$ près de $\phi_1(t_0)$.  Si nous
substituons cette expression dans
$y=\phi_2(t)$, nous trouvons $y=f(x) \equiv \phi_2(\phi_1^{-1}(x))$ pour
$x$ près de $\phi_1(t_0)$.  Cette situation est illustrée à la
figure~\ref{PARAM6}.  Nous avons que $y=f(x)$ représente la courbe $\Gamma$
près du point $(\phi_1(t_1),\phi_2(t_1))$.
\end{rmk}

\PDFfig{13_repr_param/param6}{Une courbe du plan donnée par une
représentation paramétrique et une fonction}{La courbe $\Gamma$ donnée
par la représentation paramétrique $(x,y)=(\phi_1(t),\phi_2(t))$ pour
$t\in \RR$.  Pour $x$ près de $\phi_1(t_0)$, nous pouvons représenter la
courbe $\Gamma$ par $y = f(x)\equiv \phi_2(\phi_1^{-1}(x))$.}{PARAM6} 

\begin{egg}
Trouvons l'équation de la droite tangente à la courbe $\Gamma$ au point
$(0,-\pi)$ si $\Gamma$ possède la représentation paramétrique
\[
(x,y) = (t\sin(t), t \cos(t))
\]
pour $t\in \RR$.

Il faut trouver $t$ tel que $t\sin(t) = 0$ et $t\cos(t) = -\pi$.  De
la première équation, nous trouvons $t=n\pi$ pour $n \in \ZZ$.  Parmi ces
valeurs, seul $t=\pi$ satisfait la deuxième équation.  Nous avons donc que
$(0,\pi)$ est donné par $t=\pi$ dans la représentation paramétrique.

Posons $\phi_1(t) = t\sin(t)$ et $\phi_2(t) = t\cos(t)$, la
représentation paramétrique de la droite tangente à $\Gamma$ au point
$(\phi_1(\pi),\phi_2(\pi)) = (0 , -\pi)$ est
\begin{align*}
(x,y) &= \left(\phi_1(\pi)+\alpha\phi_1'(\pi),
\phi_2(\pi)+\alpha\phi_2'(\pi) \right) \\
&= \left( \alpha\left( \sin(t) + t\cos(t) \right)\big|_{t=\pi} , -\pi +
\alpha\left(\cos(t) - t \sin(t)\right)\big|_{t=\pi} \right)
= (-\alpha\,\pi , -\pi - \alpha)
\end{align*}
pour $\alpha \in \RR$.  Si nous éliminons $\alpha$ des deux équations
$x=-\alpha\,\pi$ et $y= -\pi - \alpha$, nous trouvons l'équation suivante
pour la droite tangente.
\[
y = \frac{1}{\pi}\,x - \pi \ .
\]
\end{egg}

\begin{egg}[\theory]
Traçons la courbe qui possède la représentation paramétrique
$x = t^3 - 3 t^2$ et $y = t^3-3t$ pour $t \in \RR$.

Trouvons les points de la courbe où la tangente est
horizontale ou verticale.  La tangente est horizontal si
$y'= 3t^2-3=0$.  Donc $t=\pm 1$.  Nous obtenons le point
$(-2,-2)$ pour $t=1$ et le point $(-4,2)$ pour $t=-1$.
La tangente est verticale si $x'= 3t^2 -6t =0$.  Donc $t=0$ ou
$t=2$.  Nous obtenons le point $(0,0)$ pour $t=0$ et le point $(-4,2)$
pour $t=2$.

La courbe passe au moins deux fois par le point $(-4,2)$.  De
plus, $x'(t)$ et $y'(t)$ ne changent pas de signes lorsque $t$ est dans un
des intervalles $]-\infty, -1[$, $]-1,0[$, $]0,1[$, $]1,2[$ et
$]2,\infty[$.

Nous pouvons déterminer la pente de la tangente à la courbe au point
$(x(t),y(t))$ à l'aide du rapport $y'(t)/x'(t)$ si $x'(t) \neq 0$.
Par contre, pour déterminer la courbure, nous utilisons la
représentation locale $y=f(x)$ de la courbe car $f''(x)$ va nous
permettre de déterminer si la courbe est concave ou convexe.

Il découle du théorème de la dérivée de fonctions composées que
\[
\dydx{f}{x}(x(t)) = \dydx{y}{x}(x(t)) = \dydx{y}{t}(t) \bigg/ \dydx{x}{t}(t)
= \frac{3t^2-3}{3t^2-6t} = \frac{t^2-1}{t^2-2t}
\]
pour $t$ dans un des intervalles mentionnés précédemment dans lesquels
$x'(t)\neq 0$.  De plus,
\begin{align*}
\dydxn{f}{x}{2}(x(t)) &= \dydxn{y}{x}{2}(x(t)) =
\dfdx{\left( \dydx{y}{x}(x(t)) \right)}{t}\bigg/\dydx{x}{t}(t)
= \frac{1}{3t^2-6t}\;
\dfdx{\left( \frac{t^2-1}{t^2-2t} \right)}{t} \\
&= \frac{-2(t^2-t+1)}{3(t^2-2t)^3}
\end{align*}
pour tout $t$ appartenant à un des intervalles mentionnés
précédemment.  Puisque $t^2-t+1 >0$ pour tout $t$, nous avons que
$\displaystyle \dydxn{y}{x}{2}(x(t))$ peut changer de signe seulement
lorsque cette dérivée seconde n'est pas définie; c'est-à-dire, lorsque
$t=0$ ou $2$.  Donc $\displaystyle \dydxn{y}{x}{2}(x(t))$ ne change
pas de signe lorsque $t$ est dans un des intervalles considérés ci-dessus.

Le tableau suivant donne le signe des dérivées dans les
différents intervalles.
\[
\begin{array}{c|c|c|c}
t & \displaystyle \dydx{y}{x} & \displaystyle \dydxn{y}{x}{2} &
\text{description de } y=f(x) \\[0.7em]
\hline
]-\infty,-1[ & + & - & \text{croissante et concave} \\
\hline
]-1,0[ & - & - & \text{décroissante et concave} \\
\hline
]0,1[ & + & + & \text{croissante et convexe} \\
\hline
]1,2[ & - & + & \text{décroissante et convexe} \\
\hline
]2,\infty[ & + & - & \text{croissante et concave}
\end{array}
\]

Pour faciliter le travail de tracer la courbe, il est souvent utile de
trouver les points où la courbe coupe l'axe des $x$ et l'axe des $y$
quand cela est possible.  La courbe coupe l'axe des $y$
(i.e. $x(t)=t^3-3t^2=0$) lorsque $t=0$ et $3$.  Nous obtenons les deux
points $(0,0)$ et $(0,18)$.  La courbe coupe l'axe des $x$
(i.e. $y(t)=t^3-3t=0$) lorsque $t=0$ et $\pm \sqrt{3}$.  Nous obtenons
les trois points $(0,0)$ et $(-3\sqrt{3}-9,0)$ et $(3\sqrt{3} -9,0)$.

Le dessin de la courbe est donné ci-dessous.
\PDFgraph{13_repr_param/param7}
\end{egg}

\begin{egg}[\theory]
Traçons la courbe qui possède la représentation paramétrique
$x = \sin(2t)$ et $y = \sin(t)$ pour $t \in \RR$.

Puisque cette représentation paramétrique est périodique de période
$2\pi$, nous pouvons assumer que $0\leq t \leq 2\pi$.

Trouvons les points de la courbe où la tangente est
horizontale ou verticale.  La tangente est horizontal si
$y'= \cos(t)= 0$; c'est-à-dire, si $t= \pi/2$ ou $3\pi/2$.  Nous obtenons
les points suivants.
\[
\begin{array}{l|l}
t & (x,y) \\
\hline
\pi/2 & (0,1) \\
3\pi/2 & (0,-1)
\end{array}
\]
La tangente est verticale si $x'= 2\cos(2t) = 0$; c'est-à-dire, si
$t=\pi/4$, $3\pi/4$, $5\pi/4$ ou $7\pi/4$.  Nous obtenons les points
suivants.
\[
\begin{array}{l|l}
t & (x,y) \\
\hline
\rule{0em}{1.2em} \pi/4 & (1, \sqrt{2}/2) \\
3\pi/4 & (-1, \sqrt{2}/2) \\
5\pi/4 & (1, -\sqrt{2}/2) \\
7\pi/4 & (-1, -\sqrt{2}/2)
\end{array}
\]
Nous avons que la courbe passe trois fois par le point $(0,0)$, lorsque
$t=0$, $\pi$ et $2\pi$.  De plus, $x'(t)$ et $y'(t)$ ne changent pas
de signes lorsque $t$ est dans un des intervalles de la forme
$]n\pi/4 , (n+1)\pi/4[$. 

Pour déterminer la courbure, nous utilisons la représentation locale
$y=f(x)$ de la courbe car $f''(x)$ va nous permettre de déterminer
si la courbe est concave ou convexe.

Il découle du théorème de la dérivée de fonctions composées que
\[
\dydx{f}{x}(x(t)) = \dydx{y}{x}(x(t)) = \dydx{y}{t}(t) \bigg/ \dydx{x}{t}(t)
= \frac{\cos(t)}{2\cos(2t)}
\]
pour $t$ dans un des intervalles mentionnés précédemment.  De plus,
\begin{align*}
\dydxn{f}{x}{2}(x(t)) &= \dydxn{y}{x}{2}(x(t)) =
\dfdx{\left( \dydx{y}{x}(x(t)) \right)}{t}\bigg/\dydx{x}{t}
= \frac{1}{2\cos(2t)}\;
\dfdx{\left( \frac{\cos(t)}{2\cos(2t)} \right)}{t} \\
&= \frac{(2\cos^2(t)+1)\sin(t)}{4\cos^3(2t)}
\end{align*}
pour tout $t$ appartenant à un des intervalles mentionnés
précédemment.  Pour simplifier le numérateur, nous avons utilisé les
identités $\cos(2t) = \cos^2(t)-\sin^2(t)$ et $\sin(2t) = 2\cos(t)\sin(t)$.
Notons que $\displaystyle \dydxn{y}{x}{2}(x(t))$ ne change pas de
signe lorsque $t$ est dans un des intervalles considérés ci-dessus
car $\displaystyle \dydxn{y}{x}{2}(x(t)) =0$ seulement pour $t=0$,
$\pi$ ou $2\pi$; et $\displaystyle \dydxn{y}{x}{2}(x(t))$ n'est pas
définie seulement pour $t= pi/4$, $3\pi/4$, $5\pi/4$ ou $7\pi/4$.

Le tableau suivant donne le signe des dérivées dans les différents
intervalles.
\[
\begin{array}{c|c|c|c}
t & \displaystyle \dydx{y}{x} & \displaystyle \dydxn{y}{x}{2} &
\text{description de } y = f(x) \\[0.7em]
\hline
]0,\pi/4[ & + & + & \text{croissante et convexe} \\
\hline
]\pi/4,\pi/2[ & - & - & \text{décroissante et concave} \\
\hline
]\pi/2,3\pi/4[ & + & - & \text{croissante et concave} \\
\hline
]3\pi/4,\pi[ & - & + & \text{décroissante et convexe} \\
\hline
]\pi,5\pi/4[ & - & - & \text{décroissante et concave} \\
\hline
]5\pi/4,3\pi/2[ & + & + & \text{croissante et convexe} \\
\hline
]3\pi/2,7\pi/4[ & - & + & \text{décroissante et convexe} \\
\hline
]7\pi/4,2\pi[ & + & - & \text{croissante et concave}
\end{array}
\]

Pour faciliter le travail de tracer la courbe, il est souvent utile de
trouver les points où la courbe coupe l'axe des $x$ et l'axe des $y$
(quand cela est possible).  La courbe coupe l'axe des $y$
(i.e. $x(t)=\sin(2t)=0$) lorsque $t=0$, $pi/2$, $\pi$ et $3\pi/2$.  Nous
obtenons seulement trois points, $(0,0)$, $(0,1)$ et $(0,-1)$.  La
courbe coupe l'axe des $x$ (i.e. $y(t)=\sin(t)=0$) lorsque $t=0$,
$\pi$ et $2\pi$.  Dans les trois cas, nous obtenons l'origine.

Le dessin de la courbe est donné ci-dessous.
\PDFgraph{13_repr_param/param9}
\end{egg}

\section{Longueur d'une courbe \eng}

Soit $\phi:[a,b]\rightarrow \RR^2$ une représentation paramétrique
d'une courbe $\Gamma$ dans le plan.  Nous pouvons estimer la longueur de la
courbe $\Gamma$ en approchant cette courbe par une courbe formée de
petits segments de droite (figure~\ref{PARAM8}).
La somme de la longueur de chacun des petits segments va donner une
approximation de la longueur de la courbe qui ira en s'améliorant si
nous prenons des segments de plus en plus petits.

\PDFfig{13_repr_param/param8}{Une courbe dans le plan que nous
approchons avec des segments de droites}{Une courbe dans le
plan que nous approchons avec des segments de droites.}{PARAM8}

Soit $k$ une entier positif.  Posons $\Delta t = (b-a)/k$ et
$t_j = a + j\,\Delta t$ pour $j=0$, $1$, $2$, \ldots, $k$.  Nous avons que
$t_0=a$ et $t_k = b$.  L'intervalle $[a,b]$ est partagé en
$k$ sous-intervalles de la forme $[t_j,t_{j+1}]$ pour $j=0$, $1$, $2$,
\ldots, $k-1$.  La longueur du segment de droite qui joint les points
\[
\phi(t_j) = \left( \phi_1(t_j) , \phi_2(t_j) \right)
\quad \text{et} \quad
\phi(t_{j+1}) = \left( \phi_1(t_{j+1}) , \phi_2(t_{j+1}) \right)
\]
est
\[
\sqrt{ (\phi_1(t_{j+1})-\phi_1(t_j))^2 + (\phi_2(t_{j+1})-\phi_2(t_j))^2  }
\]
grâce au théorème de Pythagore.  Ainsi, la longueur de la courbe
formée de tous les petits segments de droite pour $j=0$, $1$, \ldots,
$k-1$ est
\begin{equation}\label{sumSegm}
L_k = \sum_{j=0}^{k-1} \sqrt{ (\phi_1(t_{j+1})-\phi_1(t_j))^2 +
(\phi_2(t_{j+1})-\phi_2(t_j))^2  } \; ,
\end{equation}

Si nous utilisons le théorème de la moyenne, nous pouvons trouver $\alpha_j$ et
$\beta_j$ entre $x_j$ et $x_{j+1}$ tels que
\begin{align*}
\phi_1(t_{j+1})-\phi_1(t_j) = \phi_1'(\alpha_j) ( t_{j+1} - t_j) =
\phi_1'(\alpha_j) \Delta t
\intertext{et}
\phi_2(t_{j+1})-\phi_2(t_j) = \phi_2'(\beta_j) ( t_{j+1} - t_j) =
\phi_2'(\beta_j) \Delta t \; .
\end{align*}

Ainsi, La longueur du segment de droite qui joint les points
$\phi(t_j)$ et $\phi(t_{j+1})$ est
\begin{align*}
\sqrt{ (\phi_1(t_{j+1})-\phi_1(t_j))^2 + (\phi_2(t_{j+1})-\phi_2(t_j))^2  }
&= \sqrt{ (\phi_1'(\alpha_j) \Delta t)^2 +
(\phi_2'(\beta_j) \Delta t)^2 } \\
&= \sqrt{ (\phi_1'(\alpha_j))^2 + (\phi_2'(\beta_j))^2 } \,
\Delta t \; .
\end{align*}
Si nous substituons cette expression dans (\ref{sumSegm}) nous
obtenons la
formule suivante pour la longueur de la courbe formée de tous les
petits segments de droite pour $j=0$, $1$, \ldots, $k-1$.
\begin{equation}\label{sumSegm2}
L_k = \sum_{j=0}^{k-1} \sqrt{ (\phi_1'(\alpha_j))^2
+ (\phi_2'(\beta_j))^2 }\, \Delta t \; .
\end{equation}
Si $\alpha_j$ était égale à $\beta_j$, nous aurions une somme de
Riemann mais $\alpha_j$ n'est généralement pas égal à $\beta_j$.
De façon {\bfseries non rigoureuse}, nous allons quand même procéder
comme si nous avions une somme de Riemann car $\alpha_j$ et $\beta_j$
sont dans l'intervalle $[t_j, t_{j+1}]$ qui devient de plus en plus petit
lorsque $k \rightarrow \infty$. Ainsi, si $k \rightarrow \infty$, nous
obtenons le résultat suivant.

\begin{prop}
La longueur $L$ d'une courbe $\Gamma$ qui possède la représentation
paramétrique $\phi:[a,b]\rightarrow \RR^2$ est
\begin{equation}\label{arcLength}
L = \int_a^b \sqrt{ (\phi_1'(t))^2 + (\phi_2'(t))^2} \dx{t} \; .
\end{equation}
\end{prop}

\begin{rmk}[\theory]
Avec la définition de l'intégrale de Riemann
donnée à la section~\ref{Riemann_int}, nous pourrions démontrer
rigoureusement que la formule (\ref{arcLength}) est vrai.
\end{rmk}

Il est possible de démontrer à l'aide de la règle de substitution pour
l'intégration que la formule (\ref{arcLength}) est indépendante de la
représentation paramétrique de la courbe $\Gamma$ qui est utilisées.

\begin{rmk}
Lorsque nous calculons la longueur d'une courbe décrite par une
représentation paramétrique, il faut vérifier que celle-ci ne parcoure
pas une section de la courbe plus d'une fois.  Par exemple, les deux
représentations paramétriques suivantes tracent le cercle unité.
\[
  \phi(t) = \left( \cos(t) , \sin(t) \right)
\]
pour $0\leq t \leq 2\pi$ et
\[
  \psi(t) = \left( \cos(2t) , \sin(2t) \right)
\]
pour $0\leq t \leq 2\pi$.  Cependant,
\[
\int_0^{2\pi} \sqrt{ (\phi_1'(t))^2 + (\phi_2'(t))^2} \dx{t} = 2\pi
\]
et
\[
\int_0^{2\pi} \sqrt{ (\psi_1'(t))^2 + (\psi_2'(t))^2} \dx{t} = 4\pi
\; .
\]
La représentation paramétrique $\psi$ parcoure deux fois le cercle
unité.  Ce qui fait que l'intégrale donne deux fois la circonférence
du cercle unité.
\end{rmk}

\begin{egg}
Calculons la longueur de la courbe donnée par la représentation
paramétrique $x=t^3$ et $y=t^2$ pour $0\leq t \leq 4$.

Cette longueur est
\[
L = \int_0^4 \sqrt{ (x'(t))^2 + (y'(t))^2} \dx{t}
= \int_0^4 \sqrt{ (3t^2)^2 + (2t)^2} \dx{t}
= \int_0^4 t\, \sqrt{ 9t^2 + 4} \dx{t}
\]
Si $u = 9t^2 + 4$, alors $\dx{u} = 18 t \dx{t}$, $u=4$ lorsque
$t=0$ et $u=148$ lorsque $t=4$.  Ainsi, 
\[
L = \frac{1}{18} \int_4^{148} u^{1/2} \dx{u}
= \frac{1}{27} u^{3/2} \bigg|_4^{148} =
\frac{8}{27} \left( 37^{3/2} - 1 \right) \; .
\]
\end{egg}

Dans le cas particulier où la courbe $\Gamma$ peut être représentée
par le graphe d'une fonction $f$ sur une intervalle $[a,b]$, nous pouvons
utiliser la représentation $\phi:[a,b]\rightarrow \RR^2$ définie par
$\phi_1(x)=x$ et $\phi_2(x) = f(x)$ pour obtenir le résultat suivant
directement de (\ref{arcLength}).

\begin{prop}
La longueur $L$ d'une courbe $\Gamma$ qui peut être représentée par le
graphe d'une fonction $f$ sur un intervalle $[a,b]$ est
\begin{equation}\label{arcLength2}
L = \int_a^b \sqrt{ 1 + (f'(t))^2} \dx{x} \; .
\end{equation}
\end{prop}

\begin{egg}
Calculons la longueur de la courbe donnée par le graphe de
$f(x) = \ln(\sin(x))$ pour $\pi/6 \leq x \leq \pi/3$.

Puisque $f'(x) = \cos(x)/\sin(x)  = \cot(x)$, nous obtenons la longueur
\begin{align*}
L&= \int_{\pi/6}^{\pi/3} \sqrt{ 1 + (f'(x))^2}\dx{x}
= \int_{\pi/6}^{\pi/3} \sqrt{ 1 + \cot^2(x)}\dx{x}
= \int_{\pi/6}^{\pi/3} \sqrt{ \csc^2{x} }\dx{x}
= \int_{\pi/6}^{\pi/3} \csc{x}\dx{x} \\
&= - \ln(\csc(x)+\cot(x)) \bigg|_{\pi/6}^{\pi/3}
= -\ln\left(\frac{3}{\sqrt{3}}\right) + \ln(2+\sqrt{3})
\end{align*}
car $\csc(x) = 1/\cos(x) >0$ pour $\pi/6 \leq x \leq \pi/3$.
\end{egg}

\begin{egg}
Calculons la longueur de la courbe définie par $y = (x-1)^2$ entre les
points $(1.0)$ et $(2,1)$.

La courbe dont nous voulons calculer la longueur est la partie de la
parabole $y= (x-1)^2$ pour $1\leq x \leq 2$.  Si nous posons
$f(x) = (x-1)^2$, alors $f'(x) = 2(x-1)$.  La longueur de la courbe est
\begin{align*}
L &= \int_1^2 \sqrt{1+(f'(x))^2} \dx{x}
= \int_1^2 \sqrt{1+4(x-1)^2} \dx{x} \\
&= \left( \frac{1}{2}\,(x-1)\sqrt{1+4(x-1)^2} + \frac{1}{4}
\ln\left| 2(x-1) + \sqrt{1+4(x-1)^2} \right| \right)\bigg|_{x=1}^2 \\
&= \frac{1}{2}\, \sqrt{5} + \frac{1}{4}\, \ln\left| 2 + \sqrt{5}\right| \; .
\end{align*}
Le calcul de l'intégrale précédente demande plusieurs substitutions et
une intégration par parties.  Nous donnons les détails dans le
paragraphe qui suit.

Pour évaluer l'intégrale indéfinie
\[
\int \sqrt{1+4(x-1)^2} \dx{x} \; ,
\]
nous posons $\displaystyle x = 1 + \frac{1}{2}\, \tan(\theta)$ pour
$-\pi/2<\theta<\pi/2$.  Nous avons que
\[
\sqrt{1+4(x-1)^2} = \sqrt{1+\tan^2(\theta)} = \sec(\theta)
\]
car $\sec(\theta)>0$ pour $-\pi/2<\theta<\pi/2$.  De plus,
$\displaystyle \dx{x} = \frac{1}{2} \,\sec^2(\theta) \dx{\theta}$.
Ainsi,
\begin{align*}
\int \sqrt{1+4(x-1)^2} \dx{x} &= \int \sec(\theta)
\left(\frac{1}{2}\, \sec^2(\theta) \right) \dx{\theta} \\
&= \frac{1}{2}\, \int \sec^3(\theta) \dx{\theta} \; .
\end{align*}
À l'exemple~\ref{intSRQTxdpu}, nous avons montré que
\[
\int \sec^3(\theta) \dx{\theta} = \frac{1}{2} \sec(\theta) \tan(\theta)
+ \frac{1}{2} \ln|\sec(\theta)+\tan(\theta)| + C \ .
\]
Ainsi,
\begin{align*}
\int \sqrt{1+4(x-1)^2} \dx{x}
&= \frac{1}{2}\, \int \sec^3(\theta) \dx{\theta} \\
&= \frac{1}{4}\,\tan(\theta)\sec(\theta)
+ \frac{1}{4}\, \ln \big| \sec(\theta) + \tan(\theta) \big| + C \; .
\end{align*}
Puisque $\displaystyle \tan(\theta) = 2(x-1)$ et
$\displaystyle \sec(\theta) = \sqrt{ 1 + 4(x-1)^2}$ pour
$-\pi/2<\theta<\pi/2$, nous obtenons
\begin{align*}
\int \sqrt{1+4(x-1)^2} \dx{x} & = \frac{1}{2}\,(x - 1)\sqrt{ 1 + 4(x-1)^2} \\
& \quad + \frac{1}{4}\, \ln \big|2(x - 1) + \sqrt{ 1 + 4(x-1)^2} \big| + C \; .
\end{align*}
\end{egg}

% \begin{egg}
% Calculons la longueur de la courbe définie par
% $y = (x^2+2)^{3/2}/3$ pour $0\leq x \leq 1$.
% \end{egg}

\begin{egg}
Calculons la longueur de la courbe définie par
$y = \ln(1-x^2)$ pour $0\leq x \leq 1/2$.

Si nous posons $f(x) = \ln(1-x^2)$, alors
$\displaystyle f'(x) = \frac{-2x}{1-x^2}$.  La longueur de la courbe
est
\begin{align*}
L &= \int_0^{1/2} \sqrt{1+(f'(x))^2} \dx{x}
= \int_0^{1/2} \sqrt{1+\frac{4x^2}{(1-x^2)^2}} \dx{x}
= \int_0^{1/2} \sqrt{\frac{x^4+2x^2+1}{(1-x^2)^2}} \dx{x} \\
&= \int_0^{1/2} \sqrt{\frac{(x^2+1)^2}{(1-x^2)^2}} \dx{x}
= \int_0^{1/2} \frac{x^2+1}{1-x^2} \dx{x}
= \int_0^{1/2} \left( -1 + \frac{2}{1-x^2}\right) \dx{x} \\
&= \int_0^{1/2} \left( -1 -\frac{1}{x-1} + \frac{1}{x+1}\right) \dx{x}
= \left( -x - \ln|x-1| + \ln(x+1) \right)\bigg|_0^{1/2} \\
&= -\frac{1}{2} + \ln(3) \ .
\end{align*}
Le calcul de l'intégrale précédente a demandé une intégration par
fractions partielles.
\end{egg}

\section{Aire d'une surface \eng}

Soit $\Gamma$ une courbe dans le plan possédant la représentation
paramétrique
\[
\phi(t) = \left( \phi_1(t) , \phi_2(t) \right)
\]
pour $a \leq t \leq b$.  La rotation autour d'un axe de cette courbe
$\Gamma$ produit une surface $S$ dans l'espace
(figure~\ref{surfArea}).  Il est possible de calculer l'aire de cette
surface.

\PDFfig{13_repr_param/param10}{Une surface produite par la rotation d'une
courbe autour d'un axe}{Une surface produite par la rotation de la
courbe $\Gamma$ autour d'un axe horizontal $y=c$.}{surfArea}

Avant de développer la formule pour calculer l'aire de la surface $S$
que nous retrouvons à la figure~\ref{surfArea}, considérons 
la section horizontal d'un cône qui est donnée à la
figure~\ref{surfCone}.  L'aire de cette section horizontal
est $\displaystyle 2\pi\,\left(\frac{R_1+R_2}{2}\right)\,L$.

\PDFfig{13_repr_param/param11}{Une section horizontal d'un cône dont
l'axe est vertical}{Une section horizontal d'un cône dont l'axe est
vertical.}{surfCone}

Soit $k$ une entier positif.  Posons $\Delta t = (b-a)/k$ et
$t_j = a + j\,\Delta t$ pour $j=0$, $1$, $2$, \ldots, $k$.  Nous avons que
$t_0=a$ et $t_k = b$.  L'intervalle $[a,b]$ est partagé en
$k$ sous-intervalles de la forme $[t_j,t_{j+1}]$ pour $j=0$, $1$, $2$,
\ldots, $k-1$.

La section de la surface $S$ entre $t_j$ et $t_{j+1}$
(figure~\ref{surfSect}) représente approximativement une section d'un
cône comme celui que nous retrouvons à la figure~\ref{surfCone}.
L'aire de cette section est donc approximativement
\[
2\pi \, \left(\frac{ |\phi_2(t_j)-c| + |\phi_2(t_{j+1})-c|}{2}\right)
\sqrt{\left(\phi_1(t_{j+1})-\phi_1(t_j)\right)^2 +
\left( \phi_2(t_{j+1}) - \phi_2(t_j)\right)^2} \; .
\]

\PDFfig{13_repr_param/param12}{Une section d'un surface produite par la
rotation d'une courbe d'un axe horizontal}{Une section de la surface
$S$ représenté à la figure~\ref{surfArea} pour $t_j \leq t \leq t_{j+1}$
est approximativement une section d'un cône comme celle que
nous retrouvons à la figure~\ref{surfCone}.}{surfSect} 

L'aire total $A$ de la surface $S$ est donc donné approximativement
par
\begin{align*}
A &\approx \sum_{j=0}^{k-1}
2\pi \, \left(\frac{ |\phi_2(t_j)-c| + |\phi_2(t_{j+1})-c|}{2}\right) \\
&\qquad\qquad \times \sqrt{\left(\phi_1(t_{j+1})-\phi_1(t_j)\right)^2 +
\left( \phi_2(t_{j+1}) - \phi_2(t_j)\right)^2} \; .
\end{align*}
D'après le théorème de la moyenne, ils existent $\alpha_j$ et
$\beta_j$ entre $t_j$ et $t_{j+1}$ tels que
\begin{align*}
\phi_1(t_{j+1})-\phi_1(t_j) &= \phi_1'(\alpha_j)(t_{j+1}-t_j) =
\phi_1'(\alpha_j)\, \Delta t
\intertext{et}
\phi_2(t_{j+1})-\phi_2(t_j) &= \phi_2'(\beta_j)(t_{j+1}-t_j) =
\phi_2'(\beta_j)\, \Delta t \; .
\end{align*}
Donc
\begin{align*}
\sqrt{\left(\phi_1(t_{j+1})-\phi_1(t_j)\right)^2 +
\left( \phi_2(t_{j+1}) - \phi_2(t_j)\right)^2}
&= \sqrt{\left(\phi_1'(\alpha_j)\, \Delta t \right)^2 +
\left( \phi_2'(\beta_j)\, \Delta t \right)^2} \\
&= \sqrt{\left(\phi_1'(\alpha_j)\right)^2 +
\left( \phi_2'(\beta_j)\right)^2} \; \Delta t
\end{align*}
pour $j=0$, $1$, $2$, \ldots  De plus, le théorème des valeurs
intermédiaires donne $\gamma_j$ entre $t_j$ et $t_{j+1}$ tel que
\[
\left(\frac{ |\phi_2(t_j)-c| + |\phi_2(t_{j+1})-c|}{2}\right)
= \left|\phi_2(\gamma_j) - c\right| \; .
\]
L'aire $A$ de la surface $S$ est donc approximativement
\[
A \approx \sum_{j=0}^{k-1}
2\pi \, \left|\phi_2(\gamma_j) - c\right|\,
\sqrt{\left(\phi_1'(\alpha_j)\right)^2 +
\left( \phi_2'(\beta_j)\right)^2} \; \Delta t \; .
\]
Si $\alpha_j$, $\beta_j$ et $\gamma_j$ étaient égaux, nous
aurions une somme de Riemann mais $\alpha_j$, $\beta_j$ et
$\gamma_j$ ne sont généralement pas égaux.  De façon
{\bfseries non rigoureuse},
nous allons quand même procéder comme si nous avions une somme de Riemann car
$\alpha_j$, $\beta_j$ et $\gamma_j$ sont dans l'intervalle
$[t_j,t_{j+1}]$ qui devient de plus en plus petit lorsque
$k \rightarrow \infty$.   Ainsi, si $k \rightarrow \infty$, nous obtenons
le résultat suivant.

\begin{prop}
Soit $S$ une surface produite par la rotation autour d'un axe $y=c$
d'une courbe $\Gamma$ possédant la représentation paramétrique
\[
  \phi(t) = \left( \phi_1(t) , \phi_2(t) \right)
\]
pour $a \leq t \leq b$.  L'aire total $A$ de la surface $S$ est
\begin{equation} \label{AirSurfRot}
A = 2\pi\, \int_a^b \left| \phi_2(t)-c \right| \, 
\sqrt{\left(\phi_1'(t)\right)^2 + \left( \phi_2'(t)\right)^2} \dx{t} \; .
\end{equation}
Si $S$ est une surface produite par la rotation autour d'un axe $x=c$
de la courbe $\Gamma$, alors l'aire total $A$ de la surface $S$ est
\[
A = 2\pi\, \int_a^b \left| \phi_1(t)- c \right| \, 
\sqrt{\left(\phi_1'(t)\right)^2 +
\left( \phi_2'(t)\right)^2} \dx{t} \; .
\]
\end{prop}

\begin{rmk}
Comme pour le calcul de la longueur d'une courbe, lorsque que nous
calculons l'aire d'une surface produite par la rotation d'une courbe
autour d'un axe, il faut vérifier que la représentation paramétrique
de la courbe ne parcoure pas une section de la courbe plus d'une
fois.
\end{rmk}

Si la courbe $\Gamma$ est donnée par le graphe d'une fonction
$f:[a,b]\rightarrow \RR$, en prenant la représentation paramétrique
\[
\phi(x) = \left( \phi_1(x) , \phi_2(x) \right)
= \left( x , f(x) \right)
\]
pour $a \leq x \leq b$, nous obtenons le résultat suivant.

\begin{prop}
Soit $\Gamma$ une courbe représentée par le graphe d'une fonction
différentiable $f$ sur l'intervalle $[a,b]$.  L'aire $A$ de la surface
$S$ produite par la rotation de $\Gamma$ autour de l'axe $y=c$ est
\begin{equation} \label{AirSurfRotClass}
A = 2\pi\, \int_a^b \left| f(x)-c \right| \, 
\sqrt{1 + \left( f'(x) \right)^2} \dx{x} \; .
\end{equation}
\end{prop}

\begin{egg}
Calculons l'aire $A$ de la surface $S$ obtenue par la rotation autour
de l'axe des $x$ de la courbe $\Gamma$ ayant la représentation
paramétrique
\[
\phi(t) = \left( \phi_1(t) , \phi_2(t) \right)
= \left( t^3 , t^2 \right)
\]
pour $0 \leq t \leq 1$.

Posons $x=t^3$ et $y=t^2$.  La première équation donne $t= x^{1/3}$.
Si nous substituons cette expression dans la deuxième équation, nous
obtenons $y = x^{2/3}$.   La courbe $\Gamma$ est donc le graphe de
la fonction $y = f(x) = x^{2/3}$ pour $0\leq x \leq 1$.  Nous pourrions
donc utiliser (\ref{AirSurfRotClass}) pour calculer l'aire de la
surface $S$.  Nous laissons au lecteur le soin de faire ce calcul.
Nous calculons l'aire de la surface $S$ à l'aide de
(\ref{AirSurfRot}).

Puisque $\phi_1'(t) = 3t^2$ et $\phi_2'(t) = 2t$, nous avons
\[
A = 2\pi \,\int_0^1 t^2 \, \sqrt{(3t^2)^2 + (2t)^2} \dx{t}
= 2\pi \,\int_0^1 t^3 \, \sqrt{9t^2 + 4} \dx{t} \; .
\]
Si $u=9 t^2 +4$, alors $u=4$ pour $t=0$, $u=13$ pour $t=1$ et
$\displaystyle \dx{u} = 18 t \dx{t}$.  Ainsi,
\begin{align*}
A &= 2\pi \,\int_0^1 t^3 \, \sqrt{9t^2 + 4} \dx{t}
= \frac{\pi}{9} \, \int_0^1 t^2 \, \left(9t^2 + 4\right)^{1/2} \;
18 t \dx{t} \\
&= \frac{\pi}{9} \, \int_4^{13} \frac{u-4}{9} \, u^{1/2} \dx{u}
= \frac{\pi}{81} \, \int_4^{13} \left(u^{3/2}-4u^{1/2}\right) \dx{u} \\
&= \frac{\pi}{81} \,\left( \frac{2}{5} u^{5/2}- \frac{8}{3} u^{3/2}
\right)\bigg|_{u=4}^{13}
= \frac{\pi\left(494\sqrt{13} + 128\right)}{1215}
\approx 4.9364177 \; .
\end{align*}
\end{egg}

\section{Coordonnées polaires\index{Coordonnées polaires} \theory}

Chaque point $P$ du plan est uniquement déterminé par ses coordonnées
cartésiennes $(x,y)$ où $x$ est le déplacement horizontal et $y$ est
le déplacement vertical pour se rentre de l'origine $O$ au point $P$.

Le système de coordonnées polaires est une autre façon d'identifier
les points du plan.  Ce système de coordonnées est très utile lorsque
nous travaillons avec un objet qui exécute un mouvement circulaire.

Dans le système de coordonnées polaires, le point $P$ est représenté
par le couple $(r, \theta)$ où $r$ est la distance entre le point $P$
et l'origine $O$ et $\theta$ est l'angle entre l'axe des $x$ et la
droite $\overline{OP}$ mesuré dans le sens contraire aux aiguilles
d'une montre (figure~\ref{POLAR1}).

\PDFfig{13_repr_param/polar1}{Les coordonnées polaires}{Relation entre
les coordonnées polaires et les coordonnées cartésiennes d'un point
$P$.}{POLAR1}

Contrairement aux coordonnées cartésiennes, les coordonnées polaires
d'un point du plan ne sont pas uniques.  Si $(r,\theta)$ représente un
point $P$, alors $(r,\theta+2n\pi)$ avec $n\in \ZZ$ représentent aussi
le point $P$. L'origine est encore plus problématique car $(0,\theta)$
représente l'origine quel que soit $\theta$.

Si $(r,\theta)$ sont les coordonnées polaires d'un point $P$, alors
les coordonnées cartésiennes $(x,y)$ de ce point sont données par
\[
x = r\cos(\theta) \quad \text{et} \quad y = r\sin(\theta) \; .
\]
Si $(x,y) \neq (0,0)$ sont les coordonnées cartésiennes d'un point
$P$, alors les coordonnées polaires $(r,\theta)$ de ce point (modulo
un multiple de $2\pi$ pour $\theta$) sont donnée par
\begin{align*}
r &= \sqrt{x^2+y^2} \quad ,\\
\theta &= \begin{cases}
\arctan\left(y/x\right) & \quad \text{si} \quad x>0 \\
\arctan\left(y/x\right) + \pi & \quad \text{si} \quad x<0 \\
\pi/2 & \quad \text{si} \quad x=0 \quad \text{et} \quad y>0 \\
-\pi/2 & \quad \text{si} \quad x=0 \quad \text{et} \quad y<0
\end{cases}\; .
\end{align*}
Puisque
$\displaystyle -\frac{\pi}{2} < \arctan\left(\frac{y}{x}\right) <
\frac{\pi}{2}$ pour tout $x\neq 0$ et $y$, nous avons que $\theta$ donnée
par la formule précédente satisfait
$\displaystyle -\frac{\pi}{2} \leq \theta < \frac{3\pi}{2}$.

\begin{rmk}
Il est permit d'avoir des coordonnées polaires de la forme
$(r,\theta)$ où $r<0$.  Dans ce cas, il est sous-entendu que
$(r, \theta) = (|r|, \theta +\pi)$.  C'est-à-dire que $(r,\theta)$
avec $r<0$ est la réflexion par rapport à l'origine du point
$(|r|,\theta)$.
\end{rmk}

\begin{egg}
Quelles sont les coordonnées cartésiennes du point $P$ dont les
coordonnées polaires sont\\ $(3, -\pi/3)$?

Nous avons $x = 3 \cos(-\pi/3) = 3/2$ et $y = 3\sin(-\pi/3) = -3\sqrt{3}/2$.
Les coordonnées cartésiennes de $P$ sont donc $(3/2, -3\sqrt{3}/2)$.
\end{egg}

\begin{egg}
Quelles sont les coordonnées polaires du point $P$ dont les
coordonnées cartésiennes sont\\ $(-1, -\sqrt{3})$?

Nous avons $\displaystyle r = \sqrt{ (-1)^2 + (-\sqrt{3})^2} = 2$. Puisque
$x<0$, nous avons
\[
\theta = \arctan\left(\frac{-\sqrt{3}}{-1}\right) + \pi
= \arctan\left(\sqrt{3}\right) + \pi = \frac{\pi}{3} + \pi =
\frac{4\pi}{3} \; .
\]
Les coordonnées polaires de $P$ sont donc $(2, 4\pi/3)$.
\end{egg}

\subsection{Équations en coordonnées polaires}

Commençons notre apprentissage des courbes décrites par des équations
en coordonnées polaires par l'étude de quelques équations
élémentaires; c'est-à-dire, $r$ constant ou $\theta$ constant.

\begin{egg}
Quelle est la courbe décrite par l'équation $r=3$ et celle décrite par
$r=-3$?

Comme aucune contrainte n'est imposée sur l'angle $\theta$ dans
l'équation $r=3$, l'angle $\theta$ est libre.  En d'autres mots,
l'équation $r=3$ est vrai quelle que soit la valeur de $\theta$.  La
courbe représentée par cette équation est le cercle de rayon $3$
centré à l'origine; c'est l'ensemble de tous les points dont les
coordonnées cartésiennes $(x,y)$ satisfont $\sqrt{x^2+y^2} = 3$.
Lorsque $\theta$ augmente, nous parcourons le cercle de rayon $3$ dans
le sens contraire aux aiguilles d'une montre.

L'équation $r=-3$ décrit aussi le cercle de rayon $3$ centré à
l'origine car c'est la courbe qui est la réflexion par rapport à
l'origine du cercle $r=3$.
\end{egg}

\begin{egg}
Quelle est la courbe décrite par l'équation $\theta = \pi/4$?

Comme aucune contrainte n'est imposée sur le rayon $r$ dans l'équation
$\theta = \pi/4$, le rayon $r$ est libre.  En d'autres mots,
l'équation $\theta = \pi/4$ est vrai quelle que soit la valeur de
$r \in \RR$.  La courbe représentée par cette équation est la droite
de pente $\pi/4$ qui passe par l'origine;  c'est-à-dire, la droite
$y=x$ au complet car nous acceptons les valeurs négatives de $r$. Lorsque
$r$ augmente, $x$ et $y$ vont de $-\infty$ à $+\infty$.
\end{egg}

\begin{egg}
Traçons la courbe $\Gamma$ décrite par l'équation $r = 2\sin(\theta)$
et trouvons l'équation cartésienne qui décrit cette courbe.

Si nous multiplions l'équation $r=2\sin(\theta)$ par $r$, nous obtenons
$r^2= 2 r \sin(\theta)$.  Or $r^2 = x^2 + y^2$ et $y = r\sin(\theta)$.
Nous avons donc $x^2+y^2 = 2y$.  Nous obtenons l'équation en coordonnées
cartésiennes
\[
1 = x^2+y^2 -2y + 1 = x^2 + (y-1)^2 \; .
\]
$\Gamma$ est le cercle de rayon $1$ centré au point $(0,1)$.

Le graphe de $r$ en fonction de $\theta$ ainsi que le dessin de la
courbe $\Gamma$ sont donnés ci-dessous.  Les points suivants
appartiennent à la courbe $\Gamma$.
\[
\begin{array}{*{9}{c|}c}
\theta & 0 & \pi/4 & \pi/2 & 3\pi/4 & \pi & 5\pi/4 & 3\pi/2 & 7\pi/2 &
2\pi \\
\hline
\rule{0pt}{1.1em} r & 0 & \sqrt{2} & 2 & \sqrt{2} & 0 & -\sqrt{2} & -2 &
-\sqrt{2} & 0
\end{array}
\]
Ces points apparaissent sur la courbe $\Gamma$ dans le dessin
ci-dessous.  Nous parcourons la courbe $\Gamma$ deux fois dans le
sens contraire aux aiguilles d'une montre lorsque $\theta$ varie de
$0$ à $2\pi$.
\PDFgraph{13_repr_param/polar2}
\label{eggPOLAR2}
\end{egg}

\begin{egg}
Traçons la courbe $\Gamma$ décrite par l'équation $r = 2 + \cos(\theta)$.

Le graphe de $r$ en fonction de $\theta$ ainsi que le dessin de la
courbe $\Gamma$ sont donnés ci-dessous.  Les points suivants
appartiennent à la courbe $\Gamma$.
\[
\begin{array}{*{9}{c|}c}
\theta & 0 & \pi/4 & \pi/2 & 3\pi/4 & \pi & 5\pi/4 & 3\pi/2 & 7\pi/2 &
2\pi \\
\hline
\rule{0pt}{1.8em} r & 3 &
\displaystyle \frac{4+\sqrt{2}}{2} \approx 2.7071 & 2 &
\displaystyle \frac{4-\sqrt{2}}{2} \approx 1.29289
& 1 & \displaystyle \frac{4-\sqrt{2}}{2} & 2 &
\displaystyle \frac{4+\sqrt{2}}{2} & 0
\end{array}
\]
Ces points apparaissent sur la courbe $\Gamma$ dans le dessin
ci-dessous.  Nous parcourons la courbe $\Gamma$ une fois dans le sens
contraire aux aiguilles d'une montre lorsque $\theta$ varie de $0$ à
$2\pi$.
\PDFgraph{13_repr_param/polar3}
\label{eggPOLAR3}
\end{egg}

\begin{egg}
Traçons la courbe $\Gamma$ décrite par l'équation
$r = \sin(2\theta)$.

Le graphe de $r$ en fonction de $\theta$ ainsi que le dessin de la
courbe $\Gamma$ sont donnés ci-dessous.  Les points suivants
appartiennent à la courbe $\Gamma$.
\[
\begin{array}{*{9}{c|}c}
\theta & 0 & \pi/4 & \pi/2 & 3\pi/4 & \pi & 5\pi/4 & 3\pi/2 & 7\pi/2 &
2\pi \\
\hline
\rule{0pt}{1.5em} r & 0 & 1 & 0 & -1 & 0 & 1 & 0 & -1 & 0
\end{array}
\]
Ces points apparaissent sur la courbe $\Gamma$ dans le dessin
ci-dessous.  Nous parcourons la courbe $\Gamma$ en suivant l'ordre des
intervalles I, II, III, \ldots, VIII lorsque $\theta$ varie de $0$ à
$2\pi$.
\PDFgraph{13_repr_param/polar4}
\label{eggPOLAR4}
\end{egg}

\begin{rmk}
Les courbes que nous avons obtenues aux exemples~\ref{eggPOLAR2},
\ref{eggPOLAR3} et \ref{eggPOLAR4} ont certaines symétries.  La courbe
de l'exemple~\ref{eggPOLAR2} est inchangée par une réflexion par
rapport à l'axe des $y$.  La courbe de l'exemple~\ref{eggPOLAR3} est
inchangée par une réflexion par rapport à l'axe des $x$.  La courbe de
l'exemple~\ref{eggPOLAR4} est inchangée par une réflexion par rapport
à l'axe des $x$, à l'axe des $y$, à la droite $x=y$, à la droite
$y=-x$. et à l'origine.  Elle est de plus inchangée par une rotation
autour de l'origine de $\pi/2$, $\pi$ ou $3\pi/2$.  En fait, nous
disons que la courbe de l'exemple~\ref{eggPOLAR4} à la symétrie d'un
carré car les opérations qui transforment un carré en lui même vont aussi
transformer cette courbe en elle même.

Résumons comment déterminer les principales symétries d'une courbe
donnée en coordonnées polaires.  Soit $\Gamma$ une courbe donnée par
l'équation en coordonnées polaires $f(r,\theta)=0$.

\begin{itemize}
\item La courbe $\Gamma$ est inchangée par une réflexion par rapport à
l'axe des $x$ si $f(r,\theta) = f(r,-\theta) = 0$ (ou
$f(r,\theta) = f(-r,\pi-\theta) = 0$). 
\item La courbe $\Gamma$ est inchangée par une réflexion par rapport à
l'axe des $y$ si $f(r,\theta) = f(r,\pi-\theta) = 0$ (ou
$f(r,\theta) = f(-r,-\theta) = 0$).
\item La courbe $\Gamma$ est inchangée par une réflexion par rapport à
l'origine (cela correspond aussi à une rotation par $\pi$) si 
$f(r,\theta) = f(-r,\theta) = 0$ (ou $f(r,\theta) = f(r,\pi+\theta) = 0$).
\item La courbe $\Gamma$ est inchangée par une réflexion par rapport à
la droite $\theta = \pi/4$ (i.e. $y=x$) si
$\displaystyle f(r,\theta) = f\left(r,\frac{\pi}{2} -\theta\right) = 0$ (ou
$\displaystyle f(r,\theta) = f\left(-r,-\frac{\pi}{2}-\theta\right)$).
\item La courbe $\Gamma$ est inchangée par une réflexion par rapport à
la droite $\theta = -\pi/4$ (i.e. $y=-x$) si 
$\displaystyle f(r,\theta) = f\left(r,-\frac{\pi}{2} -\theta\right) = 0$
(ou
$\displaystyle f(r,\theta) = f\left(-r,\frac{\pi}{2}-\theta\right)$).
\item La courbe $\Gamma$ est inchangée par une rotation centrée à
l'origine de $\pi/2$ dans le sens contraire aux aiguilles d'une montre
si
$\displaystyle f(r,\theta) = f\left(r,\frac{\pi}{2}+\theta\right) = 0$
(ou
$\displaystyle f(r,\theta) = f\left(-r,-\frac{\pi}{2}+\theta\right)$).
\end{itemize}

Les conditions entre parenthèses dans la liste précédente sont données
car ils sont parfois plus simples à vérifier que les conditions
standards.
\end{rmk}

\subsection[Longueur d'une courbe]{Longueur d'une courbe donnée par
  une équation en coordonnées polaires}

Supposons qu'un courbe $\Gamma$ soit décrite par l'équation en
coordonnées polaires $f(r,\theta)=0$.  Supposons de plus que nous
puissions résoudre cette équation pour $r$ en fonction de $\theta$.  Dans
ce cas, une représentation paramétrique pour la courbe $\Gamma$ est
donnée par
\[
x=\phi_1(\theta) = r(\theta)\,\cos(\theta) \quad \text{et} \quad
y= \phi_2(\theta) = r(\theta)\,\sin(\theta) \quad
\]
pour $a\leq \theta \leq b$.  Puisque
\[
\phi_1'(\theta) = r'(\theta)\,\cos(\theta) - r(\theta)\,\sin(\theta)
\quad \text{et} \quad
\phi_2'(\theta) = r'(\theta)\,\sin(\theta) + r(\theta)\,\cos(\theta) \; ,
\]
nous avons
\begin{align*}
\left(\phi_1'(\theta)\right)^2 + \left(\phi_2'(\theta)\right)^2
&= \left( r'(\theta)\,\cos(\theta) - r(\theta)\,\sin(\theta)\right)^2
+ \left(r'(\theta)\,\sin(\theta) + r(\theta)\,\cos(\theta)\right)^2 \\
&= \left(r'(\theta)\right)^2\,\cos^2(\theta)
-2\,r'(\theta)\,r(\theta)\,\sin(\theta)\,\cos(\theta)
+\left(r(\theta)\right)^2\,\sin^2(\theta)\\
&\quad +\left(r'(\theta)\right)^2\,\sin^2(\theta)
+2\,r'(\theta)\,r(\theta)\,\sin(\theta)\,\cos(\theta)
+\left(r(\theta)\right)^2\,\cos^2(\theta) \\
&= \left(r'(\theta)\right)^2 + \left(r(\theta)\right)^2 \; ,
\end{align*}
Ainsi, la formule (\ref{arcLength}) pour calculer la longueur d'une
courbe devient
\[
L = \int_a^b \sqrt{\left(\phi_1'(\theta)\right)^2
+ \left(\phi_2'(\theta)\right)^2} \dx{\theta}
= \int_a^b
\sqrt{\left(r'(\theta)\right)^2 + \left(r(\theta)\right)^2}
\dx{\theta} \; .
\]
Nous avons donc le résultat suivant.

\begin{prop}
Soit $\Gamma$ une courbe donnée par une équation en coordonnées
polaire $r = r(\theta)$ pour $a \leq \theta \leq b$.  La longueur $L$
de cette courbe est donnée pas
\begin{equation} \label{arclengthPolar}
L = \int_a^b
\sqrt{\left(r'(\theta)\right)^2 + \left(r(\theta)\right)^2}
\dx{\theta} \; .
\end{equation}
\end{prop}

\begin{egg}
Traçons la courbe donnée par l'équation en coordonnées polaires
$r=\theta$ pour $0\leq \theta \leq 2\pi$, et calculons la longueur de
cette courbe.

Le graphe de $r$ en fonction de $\theta$ ainsi que le dessin de la
courbe sont donnés ci-dessous.
\PDFgraph{13_repr_param/polar5}
C'est une spirale.  Puisque $r(\theta) = \theta$ et
$r'(\theta)= 1$, la longueur $L$ de cette courbe est
\[
L = \int_0^{2\pi} \sqrt{1 + \theta^2} \dx{\theta} \; .
\]
Nous avons montré à l'aide d'une substitution trigonométrique à
l'exemple~\ref{intSRQTxdpu} que
\[
\int \sqrt{\theta^2+1} \dx{x} = \frac{1}{2}\, \theta\sqrt{1+\theta^2}
+ \frac{1}{2} \ln\left|\theta + \sqrt{1+\theta^2}\right| + C\; .
\]
Ainsi,
\[
L = \left( \frac{1}{2}\, \theta\sqrt{1+\theta^2}
+ \frac{1}{2} \ln\left|\theta + \sqrt{1+\theta^2}\right|\right)\bigg|_0^{2\pi}
= \pi\,\sqrt{1+4\pi^2} +
\frac{1}{2} \ln\left|2\pi+\sqrt{1+4\pi^2}\right| \; .
\]
\end{egg}

\subsection[Aire d'une région]{Aire d'une région bornée par une courbe  définie à l'aide d'une équation en coordonnées polaires}

Supposons que nous voulions calculer l'aire de la région $R$
représenté à la figure~\ref{POLAR6}.  La région $R$ est bornée par les
droites $\theta=\alpha$ et $\theta = \beta$, et la courbe
$r=r(\theta)$ pour $\alpha \leq \theta \leq \beta$. 

Soit $k$ une entier positif.  Posons $\Delta \theta =(\beta-\alpha)/k$
et $\theta_j = \alpha + j\,\Delta \theta$ pour $j=0$, $1$, $2$,
\ldots, $k$.  Nous avons que $\theta_0=\alpha$ et $\theta_k = \beta$.
L'intervalle $[\alpha,\beta]$ est partagé en $k$ sous-intervalles de la
forme $[\theta_j,\theta_{j+1}]$ pour $j=0$, $1$, $2$, \ldots, $k-1$.

Pour $j=0$, $1$, $2$, \ldots, $k-1$, nous choisissons $\theta_j^\ast$ entre
$\theta_j$ et $\theta_{j+1}$.  L'aire de la région bornée par les
droites $\theta=\theta_j$ et $\theta=\theta_{j+1}$, et la courbe
$r=r(\theta)$ pour $\theta_j \leq \theta \leq \theta_{j+1}$, est
approximativement l'aire $A_j$ de la région $R_j$ bornée par les
droites $\theta=\theta_j$ et $\theta=\theta_{j+1}$, et l'arc de cercle
$r=r(\theta_j^\ast)$ pour $\theta_j \leq \theta \leq \theta_{j+1}$.
La région $R_j$ est une pointe de tarte dont l'aire est
\[
A_j = \frac{1}{2}\,r^2(\theta_j^\ast) \Delta \theta_j^\ast
\]
car $R_j$ représente $\Delta \theta_j^\ast/2\pi$ du disque de rayon
$r(\theta_j^\ast)$ dont l'aire est $\displaystyle \pi r^2(\theta_j^\ast)$.

Ainsi, l'aire $A$ de la région $R$ est approximativement la somme de
l'aire de chacune des régions $R_j$.
\begin{equation} \label{airePolaireAppr}
A \approx \sum_{j=0}^{k-1} A_j = \sum_{j=0}^{k-1}
\frac{1}{2}\,r^2(\theta_j^\ast) \Delta \theta_j^\ast \; .
\end{equation}
La somme précédente représente une somme de Riemann pour l'intégrale
\[
\int_\alpha^\beta \frac{1}{2}\, r^2(\theta) \dx{\theta} \; .
\]
Si $k$ tend vers plus l'infini en (\ref{airePolaireAppr}), nous obtenons
le résultat suivant.

\begin{prop}
Soit $R$ une région bornée par les droites $\theta=\alpha$ et
$\theta = \beta$, et la courbe $r=r(\theta)$ pour
$\alpha \leq \theta \leq \beta$.  L'aire $A$ de $R$ est
\[
A = \int_\alpha^\beta \frac{1}{2}\, r^2(\theta) \dx{\theta} \; .
\]
\end{prop}

\PDFfig{13_repr_param/polar6}{Une région du plan bornée par une courbe
définie en coordonnées polaires}{La région $R$ du plan est bornée par
les droites $\theta=\alpha$ et $\theta = \beta$, et la courbe
$r=r(\theta)$ pour $\alpha \leq \theta \leq \beta$}{POLAR6} 

\begin{egg}
Dessinons la région $R$ bornée par la courbe $r=e^\theta$
pour $-\pi/2 \leq \theta \leq \pi/2$ et l'axe des $y$, et calculons
l'aire de cette région.

Le dessin de la région $R$ est donné ci-dessous.
\PDFgraph{13_repr_param/polar7}

L'aire de $R$ est
\[
A = \frac{1}{2} \int_{-\pi/2}^{\pi/2} \left( e^\theta \right)^2
\dx{\theta}
= \frac{1}{2} \int_{-\pi/2}^{\pi/2} e^{2\theta} \dx{\theta}
= \frac{1}{4} \, e^{2\theta} \bigg|_{-\pi/2}^{\pi/2}
= \frac{1}{4} \left( e^\pi-e^{-\pi} \right) = \frac{1}{2} \sinh(\pi) \; .
\]
\end{egg}

\begin{egg}
Dessinons la région $R$ bornée par la courbe
$r=\sin(3\theta)$, et calculons l'aire de cette région.

Le graphe de $r$ en fonction de $\theta$ ainsi que le dessin de la
région $R$ sont donnés ci-dessous.
\PDFgraph{13_repr_param/polar8}
Nous parcourons la contour de $R$ deux fois lorsque $\theta$ augment de $0$ à
$2\pi$.  De plus, nous remarquons que la courbe $r=\sin(3\theta)$ est
invariante pour la rotation de $2\pi/3$ car
\[
r=\sin\left(3\theta\right) \Rightarrow
\sin\left( 3\left(\theta+\frac{2\pi}{3}\right)\right)
= \sin\left(3\theta + 2\pi\right) = \sin\left(3\theta\right) = r \; .
\]
Il suffit donc de calculer l'aire de la partie de la région $R$ dans
le premier quadrant et de multiplier le résultat par $3$ pour trouver
l'aire $A$ de $R$.
\begin{align*}
A &= \frac{3}{2} \int_0^{\pi/3} \sin^2(3\theta) \dx{\theta}
= \frac{3}{4} \int_0^{\pi/3} \left( 1 - \cos(6\theta)\right) \dx{\theta} \\
&= \frac{3}{4} \left(\theta - \frac{1}{6} \, \sin(6\theta)\right)
\bigg|_0^{\pi/3}
= \frac{3}{4} \left( \frac{\pi}{3} \right) = \frac{\pi}{4} \ .
\end{align*}
\end{egg}

\begin{egg}
Dessinons de la région $R$ qui est à l'intérieure de la courbe
$r=1-\sin(\theta)$ et à l'extérieure du cercle $r=1$, et calculons l'aire
de cette région.

Le graphe de $r$ en fonction de $\theta$ ainsi que le dessin de la
région $R$ sont donnés ci-dessous.
\PDFgraph{13_repr_param/polar9}
Pour obtenir l'aire $A$ de la région $R$, il suffit de soustraire
l'aire de la partie du disque de rayon $1$ sous l'axe des $x$ de
l'aire de la région bornée par la courbe $r=1-\sin(\theta)$ pour
$\pi \leq \theta \leq 2\pi$ et l'axe des $x$.  C'est-à-dire,
\begin{align*}
A &= \frac{1}{2} \int_\pi^{2\pi} \left(1-\sin(\theta)\right)^2 \dx{\theta}
- \frac{1}{2} \int_\pi^{2\pi} 1^2 \dx{\theta}
= \frac{1}{2} \int_\pi^{2\pi} \left(1-2\sin(\theta)+\sin^2(\theta)\right)
\dx{\theta} - \frac{\pi}{2} \\
&= \frac{1}{2} \int_\pi^{2\pi} \left(1-2\sin(\theta)+
\frac{1}{2}\left(1-\cos(2\theta)\right)\right) \dx{\theta} - \frac{\pi}{2} \\
&= \frac{1}{2} \int_\pi^{2\pi} \left(\frac{3}{2} - 2\sin(\theta) -
\frac{1}{2}\cos(2\theta)\right) \dx{\theta} - \frac{\pi}{2}
= \frac{1}{2}\left( \frac{3}{2}\,\theta + 2\cos(\theta) - \frac{1}{4}
\sin(2\theta)\right)\bigg|_\pi^{2\pi}  - \frac{\pi}{2} \\
&= \frac{1}{2} \left(\frac{3\pi}{2} + 4 \right) - \frac{\pi}{2}
= \frac{\pi}{4} + 2 \; .
\end{align*}
\end{egg}

}  % End of theory

\section{Exercices}

\subsection{Droite tangente à une courbe}

\begin{question}
Considérons la courbe ayant la représentation paramétrique
$x(t) = t^2 -2t$ et $y(t) = t^2 - 1$ pour $t \in \RR$.

\subQ{a} Quels sont les points où cette courbe coupe l'axe des $x$ et
l'axe des $y$?\\
\subQ{b} Déterminez les points de la courbe où la tangente est
horizontal et ceux où la tangente est vertical.\\
\subQ{c} Trouvez l'équation de la droite tangente à la courbe au point
$(3,8)$.\\
\subQ{d} Trouvez les points de la courbe où la tangente à la
courbe en ces points passe par le point $(2,7)$.
\label{13Q1}
\end{question}

\subsection{Longueur d'une courbe}

\begin{question}[\eng]
Quelle est la longueur de la courbe donnée par la représentation
paramétrique $x=t^2$, $y=1+t^3$ qui va du point $(0,1)$ au point
$(4,9)$?
\label{13Q2}
\end{question}

\begin{question}[\eng]
Quelle est la longueur de la courbe donnée par la représentation
paramétrique $x = e^t \sin{t}$, $y = e^t \cos{t}$ pour
$0 \leq t \leq \pi/2$?
\label{13Q3}
\end{question}

\begin{question}[\eng]
Quelle est la longueur de la courbe $y = 1 + 2 x^{3/2}$ pour
$1 \leq x \leq 3$?
\label{13Q4}
\end{question}

\begin{question}[\eng]
Quelle est la longueur de la courbe $y = \ln|\cos(x)|$ pour
$0 \leq x \leq \pi/4$?
\label{13Q5}
\end{question}

\begin{question}[\eng]
Considérons la région $R$ bornée par l'axe des $x$, la courbe
$y = x^{1/3}$ et la droite $x = 1$.  Calculez le périmètre de cette
région avec une précision de $0.001$\,.
\label{13Q6}
\end{question}


%%% Local Variables: 
%%% mode: latex
%%% TeX-master: "notes"
%%% End: 

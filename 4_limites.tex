\chapter[Limite et fonctions continues]{Limites et fonctions continues}
\label{chapLimites}

\compileTHEO{

La première partie du chapitre est consacrée à l'étude de la
{\em limite d'une fonction} en un point et à l'infini.  La définition
de limite d'une fonction en un point est introduite de façon
intuitive.  Par la suite, nous donnons une définition rigoureuse de
limite.  La limite de fonctions en un point sera
aussi l'outil qui nous permettra de définir la
{\em dérivée d'une fonction} au chapitre suivant. 

La deuxième partie du chapitre présente la définition d'une
{\em fonction continue} avec quelques unes des propriétés des
fonctions continues.  Nous verrons d'autres propriétés des fonctions
continues aux prochains chapitres.

\section{Limites}\label{oneD_LimCont}

L'exemple suivant introduit le concept de {\em limite d'une fonction}
en un point.

\begin{egg}
Si $g(x) = x^2+1$, vérifions que $g(x)$ approche $2$ lorsque $x$
approche $1$.  Dans les tableaux suivants, nous évaluons $g$ a chacun des
termes de la suite de nombres
\[
\left\{ x_n\right\}_{n=1}^\infty = \left\{ x_1, x_2, x_3, \ldots\right\}
\]
qui tend vers $1$.

Si $x_n = 1 +1/n$ pour $n =1$, $2$, $3$, \ldots
\[
\begin{array}{l|c|c|c|c|c|c|c|c|c}
n & 1 & 2 & 3 & 4 & \ldots & 100 & \ldots & 10000 & \ldots \\
\hline
\rule[-1em]{0ex}{3em} \displaystyle x_n = 1 + \frac{1}{n} &
\displaystyle 2 & \displaystyle \frac{3}{2} &
\displaystyle \frac{4}{3} & \displaystyle \frac{5}{4} &
\ldots & \displaystyle \frac{101}{100} &
\ldots & \displaystyle \frac{10001}{10000} & \ldots \\
\hline
g(x_n) & 5 & 3.25 & 2.\overline{7} & 2.5625 &
\ldots & 2.0201 & \ldots & 2.00020001 & \ldots
\end{array}
\]

Si $x_n = 1 - 1/n^2$ pour $n=0$, $1$, $2$, \ldots
\[
\begin{array}{l|c|c|c|c|c|c|c|c}
n & 1 & 2 & 3 & \ldots & 100 & \ldots & 10000 & \ldots \\
\hline
\rule[-1em]{0ex}{3em} \displaystyle x_n = 1 - \frac{1}{n^2} &
\displaystyle 0 & \displaystyle \frac{3}{4} &
\displaystyle \frac{8}{9} & \ldots & \displaystyle \frac{9999}{10000} &
\ldots & \displaystyle \frac{99999999}{100000000} & \ldots \\
\hline
g(x_n) & 1 & 1.5625\ldots & 1.7901\ldots &
\ldots & 1.9998\ldots & \ldots & 1.99999998\ldots & \ldots
\end{array}
\]
Dans les deux cas, $g(x_n)$ approche la valeur $2$ lorsque
$n \to \infty$ et donc lorsque $x_n$ approche $1$; c'est-à-dire,  
\[
\lim_{n\rightarrow \infty} g(x_n) = 2
\quad \text{pour} \quad
\lim_{n\rightarrow \infty} x_n = 1
\]
dans les deux cas.

Nous pouvons vérifier algébriquement (et donc rigoureusement) que
$g(x_n)$ avec $x_n = 1 + 1/n$ approche $2$ lorsque $n \to \infty$.
Nous avons
\[
g(x_n) = \left(1+\frac{1}{n}\right)^2 + 1
= 2 + \frac{2}{n} + \frac{1}{n^2}
\]
qui approche $2$ lorsque $n \to \infty$ car $1/n$ et $1/n^2$ tendent
vers $0$ lorsque $n \to \infty$.

Comme nous l'avons fait pour $x_n = 1 + 1/n$, nous pouvons vérifier
algébriquement que $g(x_n)$ avec $x_n = 1 - 1/n^2$ approche $2$ lorsque
$n \to \infty$.  Nous avons
\[
g(x_n) = \left(1-\frac{1}{n^2}\right)^2 + 1
= 2 - \frac{2}{n^2} + \frac{1}{n^4}
\]
qui approche $2$ lorsque $n \to \infty$ car $1/n^2$
et $1/n^4$ approchent $0$ lorsque $n \to \infty$.

Quelque soit la suite de nombres
$\displaystyle \left\{x_n\right\}_{n=1}^\infty$
qui tend vers $1$ que nous choisissons, le résultat sera toujours une
suite $\displaystyle \left\{g(x_n)\right\}_{n=1}^\infty$ qui tend vers
la valeur $2$.  En d'autres mots,
\[
\lim_{n\rightarrow \infty} g(x_n) = 2 \quad
\text{pour toute suite} \quad \{x_n\}_{n=1}^\infty \quad
\text{qui satisfait} \quad
\lim_{n\rightarrow \infty} x_n = 1 \; .
\]
Pour résumer ce dernier énoncé, nous écrivons
\[
\lim_{x\rightarrow 1} g(x) = 2 \; .
\]

À partir du graphe de $g$ que nous retrouvons à la figure~\ref{DER3},
nous constatons que $g(x_n)$ approche la valeur $2$ pour la suite qui
$\displaystyle \left\{x_n\right\}_{n=1}^\infty$ approche $1$.
\label{egg_cont1}
\end{egg}

\MATHfig{4_limites/der3}{8cm}{Graphe de $g(x) = x^2 + 1$}
{Graphe de $g(x) = x^2 + 1$ pour $x$ près de $1$.}{DER3}

\begin{focus}{\dfn} \index{Limite d'une fonction}
Soit $f$ une fonction définie pour $x$ près de $c$ (il n'est pas
nécessaire que $f$ soit définie à $c$).  Nous écrivons
\[
\lim_{x\rightarrow c} f(x) = C
\]
s'il existe un {\em unique nombre} $C$ tel que
\[
\lim_{n\rightarrow \infty} f(x_n) = C
\]
{\em quelle que soit la suite}
$\displaystyle \left\{x_n\right\}_{n=1}^\infty$ de
{\em nombres différents de $c$} qui tend vers $c$.   Nous écrivons aussi
$f(x) \to C$ lorsque $x \to c$.  Nous disons que
{\bfseries $f(x)$ converge (ou tend) vers la valeur $C$
lorsque $x$ converge (ou tend) vers $c$}.  Nous disons aussi que
{\bfseries $C$ est la limite de $f$ au point $c$}.
\label{def1_conv}
\end{focus}

Il est aussi nécessaire à l'occasion de considérer seulement les suites
$\displaystyle \left\{x_n\right\}_{n=1}^\infty$ qui approchent $c$ par la
droite ou la gauche.

\begin{focus}{\dfn}
Si, dans la définition de la limite d'une fonction $f$ en un point
$c$, nous considérons seulement les suites
$\displaystyle \left\{x_n\right\}_{n=1}^\infty$ avec {\em $x_n < c$}, nous
écrivons
\[
\lim_{x\rightarrow c^-} f(x) = C
\]
ou $f(x) \to C$ lorsque $x \to c^-$,
et nous disons que {\bfseries $f(x)$ converge (ou tend) vers la valeur $C$
lorsque $x$ converge (ou tend) par la gauche vers $c$}.  Nous disons aussi
que {\bfseries $C$ est la limite à gauche de $f$ au point $c$}.
\index{Limite d'une fonction!à gauche}

De même, si, dans la définition de la limite d'une fonction $f$ en un
point $c$, nous considérons seulement les suites
$\displaystyle \left\{x_n\right\}_{n=1}^\infty$ avec {\em $x_n > c$}, nous
écrivons
\[
\lim_{x\rightarrow c^+} f(x) = C
\]
ou $f(x) \to C$ lorsque $x \to c^+$,
et nous disons que {\bfseries $f(x)$ converge (ou tend) vers la valeur $C$
lorsque $x$ converge (ou tend) par la droite vers $c$}.  nous disons aussi que
{\bfseries $C$ est la limite à droite de $f$ au point $c$}.
\index{Limite d'une fonction!à droite}
\end{focus}

\begin{egg}
Soit $g(x) = \sin(x)/x$. Quelle est la limite de $g(x)$ lorsque $x$
approche $0$?  En d'autres mots, quelle est la valeur de
$\displaystyle \lim_{x\rightarrow 0} \frac{\sin(x)}{x}$ ?

Dans le tableau suivant, nous évaluons $g$ à chacun des
termes de la suite $\displaystyle \left\{x_n\right\}_{n=1}^\infty$ qui
converge vers $0$ définie par $x_n = 1/n$ pour $n=1$, $2$, $3$,
\ldots\  Toute autre suite qui tend vers $0$ aurait pu être utilisée.
\[
\begin{array}{c|c|l}
\hline
\rule[-0.7em]{0ex}{2em} n & x_n = 1/n & g(x_n) \\
\hline
1 & 1 & 0.8414709848\ldots \\
2 & 1/2 & 0.9588510772\ldots \\
3 & 1/3 & 0.9815840903\ldots \\
4 & 1/4 & 0.9896158370\ldots \\
\vdots & \vdots & \vdots \\
100 & 1/100 & 0.9999833334\ldots \\
\vdots & \vdots & \vdots \\
10000 & 1/10000 & 0.9999999983\ldots \\
\vdots & \vdots & \vdots \\
\hline
\end{array}
\]
La suite $\displaystyle \left\{g(x_n)\right\}_{n=1}^\infty$
tend vers $1$.  Nous allons montrer plus loin que cela est vrai quelle que
soit la suite $\displaystyle \left\{x_n\right\}_{n=1}^\infty$ qui tend
vers $0$.  Nous pourrons donc dire que
$\displaystyle \lim_{x\rightarrow 0} \frac{\sin(x)}{x} = 1$.
Comme nous pouvons constater à partir du graphe de $g$ donné à la
figure~\ref{DER4}, $g(x)$ approche la valeur $1$ lorsque $x$ approche
$0$.  Par contre, la fonction $g$ n'est pas définie à l'origine.
\label{E_SINXX}
\end{egg}

\MATHfig{4_limites/der4}{8cm}{Graphe de $g(x) = \sin(x)/x$}
{Graphe de $g(x) = \sin(x)/x$ pour $x\neq 0$.}{DER4}

\begin{egg}
Soit $g(x) = \sin(1/x)$.  Est-ce que $g(x)$ approche une valeur quelconque
lorsque $x$ approche l'origine?  Si oui, quelle est cette valeur?

Le tableau suivant donne les valeurs de $g$ pour les termes de la
suite $\displaystyle \left\{x_n\right\}_{n=1}^\infty$ où
$\displaystyle x_n = \frac{1}{2n \pi}$ pour $n=1$, $2$, $3$, \ldots.
La suite $\displaystyle \left\{g(x_n)\right\}_{n=1}^\infty$ tend
vers $0$ car $g(x_n) = 0$ pour tout $n$.
\[
\begin{array}{l|l|l|l|l|l|l|l|l|l}
n & 1 & 2 & 3 & 4 & \ldots & 100 & \ldots & 10000 & \ldots \\
\hline
\rule[-1em]{0ex}{3em} \displaystyle x_n = \frac{1}{2n \pi} &
\displaystyle \frac{1}{2\pi} & \displaystyle \frac{1}{4\pi} &
\displaystyle \frac{1}{6\pi} & \displaystyle \frac{1}{8\pi} &
\ldots & \displaystyle \frac{1}{200\pi} &
\ldots & \displaystyle \frac{1}{20000\pi} & \ldots \\
\hline
g(x_n) & 0 & 0 & 0 & 0 & \ldots & 0 & \ldots & 0 & \ldots
\end{array}
\]

Il semble que $g(x)$ approche $0$ lorsque $x$ approche $0$.  Essayons
maintenant avec la suite $\displaystyle \left\{x_n\right\}_{n=1}^\infty$
où $x_n = 2/((4n+1)\pi)$ pour $n=1$, $2$, $3$, \ldots  La suite
$\displaystyle \left\{x_n\right\}_{n=1}^\infty$ tend vers $0$.  Par
contre, le tableau suivant semble indiquer que
$\displaystyle \left\{g(x_n)\right\}_{n=1}^\infty$
approche $1$ lorsque $x_n$ approche $0$.
\[
\begin{array}{l|l|l|l|l|l|l|l|l|l}
n & 1 & 2 & 3 & 4 & \ldots & 100 & \ldots & 10000 & \ldots \\
\hline
\rule[-1em]{0ex}{3em} \displaystyle x_n = \frac{2}{(4n+1)\pi} &
\displaystyle \frac{2}{5\pi} & \displaystyle \frac{2}{9\pi} &
\displaystyle \frac{2}{13\pi} & \displaystyle \frac{2}{17\pi} &
\ldots & \displaystyle \frac{2}{4001\pi} &
\ldots & \displaystyle \frac{2}{40001\pi} & \ldots \\
\hline
g(x_n) & 1 & 1 & 1 & 1 & \ldots & 1 & \ldots & 1 & \ldots
\end{array}
\]

En fait, pour tout nombre $\alpha$ entre $-1$ et $1$ inclusivement, nous
pourrions trouver une suite $\displaystyle \left\{x_n\right\}_{n=1}^\infty$
qui tend vers $0$ et telle que $g(x_n)$ tend vers $\alpha$ lorsque
$n\to \infty$.  Il y a aussi des suites
$\displaystyle \left\{x_n\right\}_{n=1}^\infty$ qui
tendent vers $0$ et telles que $g(x_n)$ n'approche aucune valeur fixe
lorsque $n \to \infty$ mais se promène entre $-1$ et $1$.

Le graphe de $g$ que nous retrouvons à la figure~\ref{DER5} montre bien
que $g(x)$ n'approche pas une valeur unique lorsque $x$ approche $0$.
Le graphe de $g$ oscille entre $-1$ et $1$ de plus en plus rapidement
lorsque $x$ approche $0$.

Donc, dans la définition de $\displaystyle \lim_{x\rightarrow c} f(x) = C$,
il est très important que la suite
$\displaystyle \left\{f(x_n)\right\}_{n=1}^\infty$ tende vers une
{\em valeur unique} $C$ {\em quelle que soit la suite}
$\displaystyle \left\{x_n\right\}_{n=1}^\infty$ qui tend vers $c$.
\end{egg}

\MATHfig{4_limites/der5}{8cm}{Graphe de $g(x) = \sin(1/x)$}
{Graphe de $g(x) = \sin(1/x)$ pour $x >0$.}{DER5}

La définition précédente pour
\[
\lim_{x\rightarrow c} f(x) = C \; ,
\]
où nous utilisons les suites $\displaystyle \left\{x_n\right\}_{n=1}^\infty$
qui tendent vers $c$ pour déterminer la valeur possible de $C$, est
très utile pour prédire numériquement la valeur possible $C$ de la
limite. Cependant, il est généralement impossible d'utiliser cette
définition pour prouver que la limite est bien la valeur $C$ suggérée.
Pour prouver cela, il faudrait vérifier que {\em toutes} les suites
$\displaystyle \left\{x_n\right\}_{n=1}^\infty$ qui tendent vers $c$
donnent des suites $\displaystyle \left\{f(x_n)\right\}_{n=1}^\infty$
qui tendent vers $C$.  Ceci est évidemment impossible.

\begin{egg}
Est-ce que la limite suivante existe?
\[
\lim_{x\to 1} \frac{|x-1|}{x-1}
\]

Nous considérons deux cas: $x$ converge vers $1$ avec $x>1$ et $x$
converge vers $1$ avec $x<1$.  Pour $x>1$, nous avons que $|x-1| = x-1$.
Ainsi,
\[
\lim_{x\to 1^+} \frac{|x-1|}{x-1} = \lim_{x\to 1^+} \frac{x-1}{x-1}
= \lim_{x\to 1^+} 1 = 1
\]
Pour $x<1$, nous avons que $|x-1| = 1 - x$.  Ainsi,
\[
\lim_{x\to 1^-} \frac{|x-1|}{x-1} = \lim_{x\to 1^-} \frac{1-x}{x-1}
= \lim_{x\to 1^-} -1 = -1
\]
Puisque
\[
\lim_{x\to 1^-} \frac{|x-1|}{x-1} \neq \lim_{x\to 1^+}
\frac{|x-1|}{x-1} \ ,
\]
nous devons conclure que la limite n'existe pas.
\end{egg}

\begin{egg}[\theory][Suite de l'exemple~\ref{egg_cont1}]
Vérifions de façon algébrique (et donc rigoureusement) que
$g(x) = x^2 +1$ satisfait $\displaystyle \lim_{x\rightarrow 1} g(x) = 2$.

Remarquons que toute suite
$\displaystyle \left\{x_n\right\}_{n=1}^\infty$
qui approche $1$ peut s'écrire $x_n = 1 + r_n$ où la suite
$\displaystyle \left\{r_n\right\}_{n=1}^\infty$ approche $0$ lorsque
$n \to \infty$.  C'est certainement le cas pour les
suites $\displaystyle \left\{ 1 + \frac{1}{n} \right\}_{n=1}^\infty$
et $\displaystyle \left\{1+\frac{1}{n^2} \right\}_{n=1}^\infty$
que nous avons utilisées précédemment.  Ainsi,
\[
g(x_n) = g(1+r_n) = (1+r_n)^2 + 1 = 2 + 2r_n + r_n^2 \to 2
\]
lorsque $n \to \infty$ car $r_n$ et $r_n^2$ approchent $0$ lorsque
$n \to \infty$.  Donc $\displaystyle \lim_{n\rightarrow \infty} g(x_n) = 2$.
\label{egg_contcont1}
\end{egg}

Malheureusement, la méthode algébrique utilisée à
l'exemple~\ref{egg_contcont1} est restreinte aux fonctions algébriques
(simples).

\subsection{Epsilon et delta \theory}

La définition de
\[
\lim_{x\rightarrow c} f(x) = C \; ,
\]
que nous allons donner ci-dessous est équivalente à la définition de
limite d'une fonction en un point que nous avons donnée précédemment mais
ne fait pas appelle aux suites.  Cette nouvelle définition est souvent
appelée la définition en termes de $\epsilon$ et $\delta$ de la limite
d'un fonction en un point.  C'est cette définition qui est généralement
utilisée pour démontrer rigoureusement que la limite d'une fonction
$f$ en un point $c$ est une valeur $C$.

\begin{focus}{\dfn} \index{Limite d'une fonction!en un point}
Soit $f$ une fonction définie pour $x \neq c$ (il n'est pas
nécessaire que $f$ soit définie à $x=c$).  Nous écrivons
\[
\lim_{x\rightarrow c} f(x) = C
\]
si, {\em quel que soit le petit nombre $\epsilon >0$} qui est donné,
nous pouvons trouver un nombre $\delta >0$ (qui peut dépendre de
$\epsilon$) tel que
\[
| f(x) - C | < \epsilon \quad \text{si} \quad |x-c|<\delta,\ x\neq c \; .
\]
Nous disons que {\bfseries $f(x)$ converge (ou tend) vers la valeur $C$
lorsque $x$ converge vers $c$}.
\label{def2_conv}
\end{focus}

\PDFfig{4_limites/epsdelt}{Définition avec $\epsilon$ et
$\delta$ de la limite d'une fonction en un point}{Pour $\epsilon$
donné, le graphe nous donne une valeur possible de $\delta$ pour que
$f(x) = \sin(x)/x$ soit entre $1-\epsilon$ et $1+\epsilon$ quel que
soit $x \neq 0$ entre $-\delta$ et $\delta$.}{EPSDELT}

Nous illustrons cette nouvelle formulation de la limite d'une fonction
en un point à l'aide de la fonction $f(x) = \sin(x)/x$ pour $x=0$.
Quelque soit la valeur de $\epsilon$, nous remarquons à partir du graphe
de $f$ à la figure~\ref{EPSDELT} qu'il est toujours possible de trouver
$\delta>0$ pour que $f(x)$ soit entre $1-\epsilon$ et $1+\epsilon$
($C=1$ dans l'énoncé de la définition) si $x$ est entre $-\delta$ et
$\delta$ ($c=0$ dans l'énoncé de la définition) avec $x \neq 0$.  Le
graphe de $f$ pour $-\delta < x < \delta$ et $x\neq 0$ est
complètement à l'intérieure de la boite définie par
$-\delta < x < \delta$ et $1-\epsilon < y < 1+\epsilon$. 
Plus $\epsilon$ sera petit, plus nous devrons prendre $\delta$ petit.

Nous présentons à la figure~\ref{EPSDELT2} le graphe d'une fonction
qui n'a pas de limite au point $c$.  Il n'existe pas de $\delta$ tel
que le graphe de $f$ pour $c-\delta < x < c+\delta$ avec $x \neq c$
soit complètement à l'intérieure de la boite définie par 
$c-\delta < x < c+\delta$ et $C-\epsilon < y < C+\epsilon$.  Il
n'existe donc pas de $\delta$ pour satisfaire la définition
précédente.

Dans la définition de limite, $\epsilon$ prend n'importe laquelle des
valeurs positives.  Il {\em ne suffit pas} de trouver un $\epsilon$
pour lequel nous pouvons trouver un $\delta$ tel que $f(x)$ soit dans
l'intervalle $]C-\epsilon, C+\epsilon[$ si $x \neq c$ est dans
l'intervalle $]c-\delta , c+\delta[$.  Il {\em faut} que pour chaque
valeur $\epsilon>0$ nous puisons trouver un $\delta$ (qui peut varier si
$\epsilon$ varie) tel que $f(x)$ soit dans l'intervalle
$]C-\epsilon, C+\epsilon[$ si $x \neq c$ est dans l'intervalle
$]c-\delta , c+\delta[$.

\PDFfig{4_limites/epsdelt2}{Une fonction qui n'a pas de limite en un point}
{Pour $\epsilon$ donné, il est impossible de trouver $\delta$ pour que
$f(x)$ soit entre $C-\epsilon$ et $C+\epsilon$ quel que soit $x \neq c$
entre $c-\delta$ et $c+\delta$.}{EPSDELT2}

\begin{egg}
Montrons à l'aide de la dernière définition de limite d'une fonction
en un point que
$\displaystyle \lim_{x\rightarrow 2} \frac{1}{x} = \frac{1}{2}$.

Soit $\epsilon >0$ quelconque mais fixe.  Puisque nous cherchons la
limite lorsque $x$ approche $2$, nous pouvons supposer que $x>1$.
Ainsi,
\[ 
\left|\frac{1}{x} - \frac{1}{2}\right| = \left| \frac{2-x}{2x} \right|
= \frac{|2-x|}{|2x|} < \frac{|2-x|}{2}
\]
pour $x>1$.  Si nous prenons $\delta = \min\{1, 2\epsilon\}$, alors
\[
\left|\frac{1}{x} - \frac{1}{2}\right| < \frac{|2-x|}{2}
< \frac{\delta}{2} \leq \epsilon
\]
pour $|x-2|<\delta$.
\end{egg}

\begin{focus}{\prp}
Les définitions de convergence données aux définitions~\ref{def1_conv}
et \ref{def2_conv} sont équivalente.
\end{focus}

\begin{proof}
\subQ{i} Supposons que $\displaystyle \lim_{x\rightarrow c} f(x) = C$
selon la définition~\ref{def1_conv}.  Montrons que la
définition~\ref{def2_conv} est satisfaite.

La preuve est par contradiction.  Supposons que la
définition~\ref{def2_conv} ne soit pas satisfaite.  Cela implique
qu'il existe un $\epsilon >0$ tel que pour tout $\delta$ nous pouvons
trouver au moins un point $x_\delta$ tel que
$\displaystyle \left|x_\delta -c\right| < \delta$ et
$\displaystyle \left|f(x_\delta) -C\right| \geq \epsilon$.

Si nous prenons $\delta = 1/n$ pour $n$ un entier positif, nous
obtenons une
suite $\displaystyle \left\{ x_{1/n} \right\}_{n=1}^\infty$ telle que
$\displaystyle \left|x_{1/n} - c \right| < \frac{1}{n}$ et
$\displaystyle \left|f(x_{1/n}) -C\right| \geq \epsilon$ pour tout $n$.
Nous avons donc $\displaystyle \lim_{n\rightarrow \infty} x_{1/n} = c$ mais
la suite $\displaystyle \left\{ f(x_{1/n}) \right\}_{n=1}^\infty$ ne
tend pas vers $C$.  Ce qui contredit la définition~\ref{def1_conv}.

\subQ{ii} Supposons maintenant que
$\displaystyle \lim_{x\rightarrow c} f(x) = C$ selon la
définition~\ref{def2_conv}.  Montrons que la
définition~\ref{def1_conv} est satisfaite.

Supposons que $\displaystyle \left\{ x_n \right\}_{n=1}^\infty$ soit
une suite qui tend vers $c$.  Soit $\epsilon >0$.  Selon la
définition~\ref{def2_conv}, il existe $\delta >0$ tel que
$\displaystyle \left|f(x) -C\right| < \epsilon$ si
$\displaystyle \left|x -c\right| < \delta$.  Puisque
$\displaystyle \lim_{n\rightarrow \infty} x_n = c$, il existe $N>0$
tel que $\displaystyle \left|x_n - c \right| < \delta$ pour $n>N$.
Ainsi,
$\displaystyle \left|f(x_n) -C\right| < \epsilon$ pour $n>N$.  Puisque
$\epsilon$ est arbitraire, nous avons
$\displaystyle \lim_{n\rightarrow \infty} f(x_n) = C$.

Comme le résultat du paragraphe précédent est vrai pour toute suite
$\displaystyle \left\{ x_n \right\}_{n=1}^\infty$ qui tend vers $c$,
nous avons que $\displaystyle \lim_{x\rightarrow c} f(x) = C$ selon la
définition~\ref{def1_conv}.
\end{proof}

\subsection{Règles pour évaluer les limites}

Le théorème suivant est parfois très utile pour évaluer la limite
d'une fonction en un point.  Nous avons déjà vu une version de ce théorème
(théorème~\ref{gendarmeS}) pour les suites.  La présente version
en est une conséquence.

\begin{focus}[][des gendarmes ou sandwich]{\thm}
\index{Théorème des gendarmes}\index{Théorème sandwich}
Soit $f$, $g$ et $h$; trois fonctions telles que
$f(x) \leq g(x) \leq h(x)$ pour $x$ près de $c$.  Si
\[
\lim_{x\rightarrow c} f(x) = \lim_{x\rightarrow c} h(x) = L \ ,
\]
alors
\[
\lim_{x\rightarrow c} g(x) = L \; .
\]
\label{gendarmeF}
\end{focus}

\begin{egg}[\theory]
La méthode algébrique de l'exemple~\ref{egg_contcont1} n'est pas utile
pour démontrer rigoureusement que
\begin{align} \label{sinxsx}
\lim_{x\rightarrow 0} \frac{\sin(x)}{x} = 1 \; .
\end{align}
Il faut utiliser une autre approche.  Le théorème précédent nous
permet de démontrer rigoureusement (\ref{sinxsx}).

Considérons le dessin à la figure~\ref{SINXX}.  Nous avons
\[
\sin(x) = |\sgm{CE}| < |\sgm{AC}|
< \text{longueur de l'arc de cercle}\ AC = x \; .
\]
Donc
\begin{equation}\label{inequ1}
\frac{\sin(x)}{x} < 1 \; .
\end{equation}
De plus,
\begin{align*}
x &= \text{longueur de l'arc de cercle}\ AC \\
&< |\sgm{AD}| + |\sgm{DC}| < |\sgm{AD}| + |\sgm{DB}|
= |\sgm{AB}| = \frac{|\sgm{AB}|}{|\sgm{OA}|}
= \tan(x) = \frac{\sin(x)}{\cos(x)} \; .
\end{align*}
Donc
\begin{equation}\label{inequ2}
\cos(x) < \frac{\sin(x)}{x} \; .
\end{equation}

Nous déduisons de (\ref{inequ1}) et (\ref{inequ2}) que
\[
\cos(x) < \frac{\sin(x)}{x} < 1  \; .
\]
Il est facile de vérifier à partir de la définition du cosinus que
\[
\lim_{x\rightarrow 0} \cos(x) = 1
\]
Par exemple, pour toute suite
$\displaystyle \left\{x_n\right\}_{n=1}^\infty$ qui tend vers $0$,
nous avons que $\cos(x_n)$ tend vers $\cos(0) = 1$. Ainsi, nous
obtenons que
$g(x) = \sin(x)/x$ approche $1$ lorsque $x$ approche $0$ grâce au
théorème des gendarmes.
\label{R_SINXX}
\end{egg}

Comme nous venons de voir, il n'est pas toujours facile de démontrer
rigoureusement qu'une fonction à une limite en un point.

\PDFfig{4_limites/sin_x_x}{$\displaystyle \lim_{x\rightarrow 0}\sin(x)/x = 1$}
{La figure utilisée pour démontrer rigoureusement que
$\displaystyle \lim_{x\rightarrow 0} \sin(x)/x = 1$.}{SINXX}

Comme pour les limites de suites, nous avons les propriétés suivantes.

\begin{focus}{\thm} \label{lim_funct_scd}
Supposons que $\displaystyle \lim_{x\rightarrow c} f(x) = A$ et
$\displaystyle \lim_{x\rightarrow c} g(x) = B$.
\begin{enumerate}
\item $\displaystyle \lim_{x\rightarrow c}\; (f(x) + g(x) ) = A + B$.
\item Si $k$ est un nombre réel, alors
$\displaystyle \lim_{x\rightarrow c}\; (k\; f(x)) = kA$.
\item $\displaystyle \lim_{x\rightarrow c}\; (f(x)g(x)) = AB$.
\item Si $B \neq 0$ (donc $g(x) \neq 0$ pour $x$ près de $c$),
alors $\displaystyle \lim_{x\rightarrow c} \frac{f(x)}{g(x)} = \frac{A}{B}$.
\end{enumerate}
\end{focus}

Les quatre conclusions du théorème précédent sont souvent énoncées de
la façon suivante.
\begin{align*}
\lim_{x\rightarrow c} (f(x)+g(x)) &= \lim_{x\rightarrow c} f(x) +
\lim_{x\rightarrow c} g(x) \; , \\
\lim_{x\rightarrow c} kf(x) &= k \lim_{x\rightarrow c} f(x) \; , \\
\lim_{x\rightarrow c} f(x)g(x) &= \left(\lim_{x\rightarrow c} f(x)\right) 
\left(\lim_{x\rightarrow c} g(x)\right)  \\
\intertext{et}
\lim_{x\rightarrow c} \frac{f(x)}{g(x)} &=
\frac{\displaystyle \lim_{x\rightarrow c} f(x)}
{\displaystyle \lim_{x\rightarrow c} g(x)}
\end{align*}
respectivement.

\section{Fonctions continues}

Une propriété que possèdent certaines fonctions est la
{\bfseries continuité}.  La fonction $f$ est continue au point $x=c$
si $f$ est définie au point $x=c$ et $f(x)$ approche la valeur $f(c)$
lorsque $x$ approche $c$.

Si nous utilisons la définition de limite introduite à la section
précédente pour définir la continuité d'une fonction au point $x=c$,
nous obtenons l'énoncé suivant.

\begin{focus}{\dfn} \index{Fonction continue!en un point}
Soit $f$ une fonction à valeurs réelles définie près d'un point $c$ et
au point $c$.  La fonction $f$ est {\bfseries continue au point $c$}
si
\[
\lim_{x\rightarrow c} f(x) = f(c)  \; .
\]
\end{focus}

En d'autres mots, $f$ est continue au point $c$ si $f(c)$ existe et
$f(x_n)$ approche $f(c)$ quelle que soit la suite
$\displaystyle \left\{x_n\right\}_{n=1}^\infty$ qui tend vers $c$.

La figure~\ref{DER2} contient le graphe d'une fonction $f$ qui
n'est pas continue au point $x=c$ car $f(x)$ approche la valeur $B$
lorsque $x>c$ approche $c$ et $f(x)$ approche la valeur $A\neq B$
lorsque $x<c$ approche $c$.  Comme $f$ est une fonction, elle ne peut
pas prendre deux valeurs distinctes, $A$ et $B$, lorsque $x=c$.  Dans
le dessin à la figure~\ref{DER2}, le cercle plein à l'extrémité gauche de la
courbe supérieure (à $x=c$) et le cercle vide à l'extrémité droite de
la courbe inférieure (aussi à $x=c$) indiquent que $f(c)=B$.

\PDFfig{4_limites/der2}{Une fonction discontinue}
{La fonction $f$ n'est pas continue au point $x=c$}{DER2}

Pour qu'une fonction ne soit pas continue en un point $x=c$, il faut
que le graphe de la fonction soit représenté par une courbe brisée à
$x=c$.

\begin{egg}
À l'exemple~\ref{egg_cont1}, nous avons montré que la fonction
$g(x) = x^2+1$ est continue au point $x=1$ car
$\displaystyle \lim_{x\rightarrow 1} g(x) = 2$ et $g(1)=2$. 
Donc
\[
\displaystyle \lim_{x\rightarrow 1} g(x) = g(1) \; .
\]
\end{egg}

\begin{egg}
La fonction $g(x) = \sin(x)/x$ que nous avons étudié à
l'exemple~\ref{E_SINXX} n'est pas continue au point $x=0$ car $g(x)$
n'est pas définie pour $x=0$. Par contre, si nous définissons
\[
f(x) = \begin{cases}
\displaystyle \frac{\sin(x)}{x} & \qquad \text{pour} \; x\neq 0 \\
1 & \qquad \text{pour} \; x = 0
\end{cases}
\]
alors $f(x) = g(x)$ pour $x\neq 0$ et $f$ est continue au point $x=0$
car
\[
\lim_{x \rightarrow 0} f(x) = \lim_{x\rightarrow 0} \frac{\sin(x)}{x}
= 1 = f(0) \; .
\]
Le graphe de $f$ est le graphe à la figure~\ref{DER4} où le point
$(0,1)$ est maintenant représenté par un cercle plein.
\end{egg}

\begin{focus}{\dfn} \index{Fonction continue!sur un intervalle}
Si une fonction $f$ est continue en tout point d'un intervalle
(e.g.\ $]a,b[$, $[a,b]$, $[a,b[$ or $]a,b]$) nous disons que la fonction
$f$ est {\bfseries continue cet intervalle}.
\end{focus}

Il est toujours préférable (si c'est possible) d'utiliser des
fonctions continues pour décrire les phénomènes physiques,
biologiques, et autres.  La raison pour cette préférence est qu'une
petite variation de l'argument d'une fonction continue aura très peu
d'effet (en général) sur la valeur retournée par la fonction.  Cela
est nécessaire si nous voulons faire des prédictions à long terme ou si
nous voulons estimer les valeurs à fournir à la fonction pour obtenir le
résultat escompté.  Nous illustrons ce dernier type de problèmes dans
l'exemple suivant.

\begin{egg}[\life]
La concentration d'un contaminant dans l'eau d'une rivière est
déterminée par la fonction
$\displaystyle f(t) = \frac{0.7 t}{1+t}$ où la variable $t$ est mesurée
en années.  À quel moment aurons-nous une concentration du contaminant
entre $0.1$ et $0.2$ pourcent?

Il faut trouver $t$ tel que
\[
  0.1 < f(t) = \frac{0.7t}{1+t} < 0.2 \quad .
\]
Il découle de la première inégalité que
\[
  0.1 < \frac{0.7t}{1+t} \Rightarrow 0.1 + 0.1 t < 0.7 t
  \Rightarrow 0.1 < 0.6 t \Rightarrow \frac{1}{6} < t
\]
et de la deuxième inégalité que
\[
  \frac{0.7t}{1+t} < 0.2 \Rightarrow 0.7 t < 0.2 + 0.2 t
  \Rightarrow 0.5 t < 0.2 \Rightarrow t < \frac{2}{5} \quad .
\]
Nous aurons donc une concentration entre $0.1$ et $0.2$ pourcent pour
$t$ entre $1/6 = 0.1\overline{6}$ de l'année (environ $2$ mois) et
$2/5 = 0.4$ de l'année (un peu moins de $5$ mois).

Si nous considérons la même question pour une concentration entre $0.13$
et $0.17$ pourcent, il faut trouver $t$ tel que
\[
  0.13 < f(t) = \frac{0.7t}{1+t} < 0.17 \quad .
\]
Puisque
\[
  0.13 < \frac{0.7t}{1+t} \Rightarrow 0.13 + 0.13 t < 0.7 t
  \Rightarrow 0.13 < 0.57 t \Rightarrow \frac{13}{57} < t
\]
et
\[
  \frac{0.7t}{1+t} < 0.17 \Rightarrow 0.7 t < 0.17 + 0.17 t
  \Rightarrow 0.53 t < 0.17 \Rightarrow t < \frac{17}{53} \ ,
\]
nous aurons une concentration entre $0.13$ et $0.17$ pourcent pour $t$
entre $13/57 =0.228\ldots$ de l'année et $17/53 = 0.320\ldots$ de l'année.

Si considère la même question pour une concentration entre $0.14$
et $0.16$ pourcent, nous trouvons $1/4 = 0.25 < t < 8/27 = 0.\overline{296}$,
Pour une concentration enter $0.145$ et $0.155$ pourcent, nous trouvons
$29/111 = 0.\overline{261} < t < 31/109 = 0.284\ldots$, etc.  À la
limite, nous avons un concentration de $0.15$ pourcent pour
$t = 3/11 = 0.\overline{27}$ de l'année.   C'est la valeur de $t$
lorsque $f(t) = 0.15$.

Naturellement, cet exemple n'est pas réaliste car la fonction $f$ est
donnée explicitement.  Dans une expérience réelle, la fonction $f$ est
inconnue.  Nous avons peut être quelques résultats expérimentaux 
qui nous donnent que la concentration du contaminant varie de $0.1$ à
$0.2$ pourcent entre $t=0.1\overline{6}$ et $t = 0.4$, de $0.13$ à
$0.17$ pourcent entre $t =0.228\ldots$ et $t = 0.320\ldots$, 
de $0.14$ à $0.16$ pourcent entre $t = 0.25$ et $t = 0.\overline{296}$,
de $0.145$ à $0.155$ pourcent entre $t = 0.\overline{261}$ et
$t = 0.284\ldots$, etc.  Si nous assumons que la fonction qui donne la
densité du contaminant en fonction du temps est une fonction continue,
nous pouvons alors conclure que la concentration du contaminant est
possiblement de $0.15$ pourcent pour $t=0.27$; une valeur qui se
trouve entre $t = 0.\overline{261}$ et $t = 0.284\ldots$.  Si nous
n'assumons pas que la fonction soit continue, nous ne pouvons rien
conclure.
\end{egg}

\subsection{Epsilon et delta \theory}

La définition en termes de $\epsilon$ et $\delta$ de la limite d'une
fonction à un point nous donne une définition de la continuité qui est
équivalente à celle que nous venons de donner.

\begin{focus}{\dfn} \index{Fonction continue!en un point}
Soit $f$ une fonction à valeurs réelles définie près d'un point $c$ et
au point $c$.  La fonction $f$ est continue au point $x=c$ si,
{\em quel que soit le petit nombre $\epsilon >0$} qui est donné, nous
pouvons trouver un nombre $\delta >0$ (qui peut dépendre de $\epsilon$)
tel que
\[
|f(x) - f(c)| < \epsilon \quad \text{si} \quad |x-c|\leq \delta \; .
\]
\end{focus}

\begin{rmk}
Nous illustrons à la figure~\ref{EPSDELTC} la deuxième formulation de la
définition d'une fonction continue $f$ en un point $c$.  Quel que soit
le nombre $\epsilon$, nous pouvons toujours trouver $\delta$ tel que le
graphe de la fonction $f$ entre $c-\delta$ et $c+\delta$,
{\em incluant $(c,f(c))$}, soit dans la boite définie par
$c-\delta < x < c+\delta$ et $f(c)-\epsilon < y < f(c)+\epsilon$.
\end{rmk}

\PDFfig{4_limites/epsdeltC}{Définition en termes de $\epsilon$ et $\delta$ de
la continuité d'une fonction à un point}{Pour $\epsilon$ donné, il est
possible de trouver $\delta$ pour que $f(x)$ (incluant $f(c)$) soit entre
$f(c)-\epsilon$ et $f(c)+\epsilon$ quel que soit $x$ entre $c-\delta$ et
$c+\delta$.}{EPSDELTC}

\section{Quelques propriétés des fonctions continues}

Les résultats suivants découlent de la définition d'une fonction continue.

\begin{focus}{\prp}
\begin{enumerate}
\item Les fonctions polynomiales, trigonométriques, exponentielles et
  logarithmiques sont des fonctions continues sur leur domaine.
\item Il découle du théorème~\ref{lim_funct_scd} que le produit et la
  somme de fonctions continues donnent une nouvelle fonction continue.
  De même, le quotient de deux fonctions continues donne une nouvelle
  fonction continue sauf aux points où le dénominateur est nul.
\item La composition de deux fonctions continues donne une nouvelle
  fonction continue.
\end{enumerate}
\end{focus}

Si $\displaystyle \{ x_n\}_{n=0}^\infty$ est une suite qui converge
vers un point $c$ du domaine d'une fonction continue $f$, alors
la suite $\displaystyle \{ f(x_n)\}_{n=0}^\infty$ converge vers le
point $f(c)$.   En d'autres mots,
\[
\lim_{n\to \infty} f(x_n) = f\left(\lim_{n\to \infty} x_n\right)
\]
Puisque la définition de limite d'une fonction en un point $c$ fait
appel aux suites $\displaystyle \{ x_n\}_{n=0}^\infty$ qui converge
vers $c$, nous obtenons le résultat suivant qui est très utile pour
calculer des limites.

\begin{focus}{\prp} \label{interLim}
Soit $f$ une fonction continue et $g$ une fonction dont l'image est un
sous-ensemble du domaine de $f$.  Nous pouvons donc considérer la
composition $f\circ g$.   Si $\displaystyle \lim_{x\to r} g(x) = s$
est un élément du domaine de $f$, alors
$\displaystyle \lim_{x\to r} f(g(x)) = f(s)$.  En d'autres mots,
\[
\lim_{x \to r} f(g(x)) = f\left(\lim_{x\to r} g(x)\right)
\]
\end{focus}

\begin{egg}
Calculons les limites suivantes si elles existent.
\begin{center}
\begin{tabular}{*{2}{l@{\hspace{0.6em}}l@{\hspace*{2.6em}}}l@{\hspace{0.6em}}l}
\subQ{a} & $\displaystyle \lim_{x\to 2} \sin\left(\frac{x^2-4}{x-2}\right)$ &
\subQ{b} & $\displaystyle \lim_{x\to 4} \frac{x-4}{2 - \sqrt{x}}$ &
\subQ{c} & $\displaystyle \lim_{x\to 1} \frac{x-1}{2 - \sqrt{x+3}}$
\end{tabular}
\end{center}

\subQ{a}
Notons que
\[
\lim_{x\to 2} \frac{x^2-4}{x-2}
= \lim_{x\to 2} \frac{(x-2)(x+2)}{x-2}
= \lim_{x\to 2} (x+2) = 4 \; .
\]
Puisque $\sin(x)$ est continue en $x = 4$, nous pouvons utiliser la
proposition~\ref{interLim} pour conclure que
\[
\lim_{x\to 2} \sin\left(\frac{x^2-4}{x-2}\right)
= \sin\left(\lim_{x\to 2} \; \frac{x^2-4}{x-2}\right)
= \sin(4) \ .
\]

\subQ{b} Remarquons que
\[
\frac{x-4}{2 - \sqrt{x}} = \frac{(\sqrt{x} - 2)(\sqrt{x}+2)}{2 - \sqrt{x}}
= -(\sqrt{x}+2)
\]
pour $x\neq 4$.  Ainsi,
\[
\lim_{x\to 4} \frac{x-4}{2 - \sqrt{x}}
= \lim_{x\to 4} -(\sqrt{x}+2) = -(\sqrt{4}+2) = -4
\]
Nous avons utilisé le fait que $\sqrt{x} - 2$ est une fonction continue pour
calculer la dernière limite.

\subQ{c} Pour évaluer cette limite, nous éliminons premièrement la racine
carrée au dénominateur.
\begin{align*}
\frac{x-1}{2 - \sqrt{x+3}}
&= \frac{x-1}{2 - \sqrt{x+3}}
\left( \frac{2 + \sqrt{x+3}}{2 + \sqrt{x+3}} \right)
= \frac{(x-1)(2 + \sqrt{x+3})}{4 - (x+3)} \\
&= \frac{(x-1)(2 + \sqrt{x+3})}{1 - x}
= -(2 + \sqrt{x+3})
\end{align*}
pour $x \neq 1$.  Ainsi,
\[
\lim_{x\to 1} \frac{x-1}{2 - \sqrt{x+3}}
= \lim_{x\to 1}  -(2 + \sqrt{x+3}) =  -(2 + \sqrt{4}) = -4
\]
Nous avons utilisé le fait que $2 + \sqrt{x+3}$ est une fonction continue pour
calculer la dernière limite.
\end{egg}

\begin{egg}[\theory]
Soit la fonction
\[
h(x) = \arctan(x) + \arctan\left(\frac{1}{x}\right) \quad , \quad
x\neq 0 \; .
\]
Montrons que
\[
h(x) = \begin{cases}
\pi/2 & \quad \text{si} \quad x>0 \\
-\pi/2 & \quad \text{si} \quad x<0
\end{cases} \; .
\]

Premièrement, montrons que
\[
\sin(h(x)) = \begin{cases}
1 & \quad \text{si} \quad x>0 \\
-1 & \quad \text{si} \quad x<0
\end{cases} \; .
\]
Posons $\theta_1 = \arctan(x)$ et $\theta_2 = \arctan(1/x)$.  Nous avons 
$-\pi/2 < \theta_1, \theta_2 < \pi/2$ et
\[
\sin(h(x)) = \sin(\theta_1+\theta_2) = \sin(\theta_1)\cos(\theta_2)
+ \sin(\theta_2)\cos(\theta_1) \; .
\]
Nous déduisons des deux dessins à la figure~\ref{arcXX1} que
\[
\sin(\theta_1) = \frac{x}{\sqrt{x^2+1}} \ ,
\ \cos(\theta_1) = \frac{1}{\sqrt{x^2+1}} \ ,
\ \sin(\theta_2) = \frac{1}{\sqrt{x^2+1}}
\quad \text{et} \quad
\cos(\theta_2) = \frac{x}{\sqrt{x^2+1}}
\]
pour $x>0$.  Donc
\[
\sin(h(x)) = \left(\frac{x}{\sqrt{x^2+1}}\right)
\left(\frac{x}{\sqrt{x^2+1}}\right)
+ \left(\frac{1}{\sqrt{x^2+1}}\right) \left(\frac{1}{\sqrt{x^2+1}}\right) = 1
\]
pour tout $x>0$.
De même, nous déduisons des deux dessins à la figure~\ref{arcXX2} que
\[
\sin(\theta_1) = \frac{x}{\sqrt{x^2+1}} \ ,
\ \cos(\theta_1) = \frac{1}{\sqrt{x^2+1}} \ ,
\ \sin(\theta_2) = \frac{-1}{\sqrt{x^2+1}}
\quad\text{et}\quad
\cos(\theta_2) = \frac{-x}{\sqrt{x^2+1}}
\]
pour $x<0$.  Donc
\[
\sin(h(x)) = \left(\frac{x}{\sqrt{x^2+1}}\right)
\left(\frac{-x}{\sqrt{x^2+1}}\right)
+ \left(\frac{-1}{\sqrt{x^2+1}}\right) \left(\frac{1}{\sqrt{x^2+1}}\right) = -1
\]
pour tout $x<0$.

Pour compléter le problème, remarquons que
$h(1) = 2\arctan(1) = \pi/2$.  Comme $\sin(h(x))$ est constant pour
$x>0$ et $h$ est une fonction continue pour $x>0$, $h(x)$ doit être
constant pour $x>0$ et ainsi $h(x) = \pi/2$ pour $x>0$.  De même,
$h(-1) = 2\arctan(-1) = -\pi/2$.  Comme $\sin(h(x))$ est constant pour
$x<0$ et $h$ est continue pour $x<0$, $h(x)$ doit être constant pour
$x<0$ et ainsi $h(x) = -\pi/2$ pour $x>0$.
\end{egg}

\PDFfig{4_limites/arcXX1}{Représentations graphiques de
$\tan(\theta_1)=x$ et $\tan(\theta_2)=1/x$ pour $x>0$}{Représentations
graphiques de $\tan(\theta_1)=x$ et $\tan(\theta_2)=1/x$ pour $x>0$
et $0< \theta_1, \theta_2 < \pi/2$.}{arcXX1} 

\PDFfig{4_limites/arcXX2}{Représentations graphiques de
$\tan(\theta_1)=x$ et $\tan(\theta_2)=1/x$ pour $x<0$}{Représentations
graphiques de $\tan(\theta_1)=x$ et $\tan(\theta_2)=1/x$ pour $x<0$
et $-\pi/2 < \theta_1, \theta_2 < 0$.   Notons que
$\tan(\theta_i)<0$, $\sin(\theta_i)<0$ et $\cos(\theta_i)>0$ pour
$-\pi/2 < \theta_1, \theta_2 < 0$.}{arcXX2}

Les fonctions continues possèdent une propriété très importante que
nous utiliserons lors de l'étude des fonctions au prochain chapitre.

\begin{focus}[][Théorème des valeurs intermédiaires]{\thm}
\index{Théorème des valeurs intermédiaires}
Soit $f$ une fonction continue sur l'intervalle $[a,b]$ et soit $m$
une valeur entre $f(a)$ et $f(b)$.  Il existe au moins une valeur $c$
telle que  $a\leq c \leq b$ et $f(c) = m$ (figure~\ref{CONTFCT}).
\label{TVI}
\end{focus}

\PDFfig{4_limites/contfct}{Théorème des valeurs intermédiaires}
{Illustration du théorème des valeurs intermédiaires.}{CONTFCT}

\section{Limites à l'infini et limites infinies}

Lorsque nous étudions une fonction, il est souvent intéressant de savoir
comment se comporte la fonction pour de très grandes valeurs de son
domaine.

\begin{egg}
Comme nous pouvons le constater à partir du graphe de la fonction
$\displaystyle g(x) = 5 + e^{(2 - x/10)}$ que l'on retrouve à la
figure~\ref{LIM1}, $g(x)$ semble approcher la valeur $5$ lorsque $x$
devient de plus en plus grand.  En effet, $g(x) \approx 5$ pour $x$
très grand.
\end{egg}

\MATHfig{4_limites/lim_inf1}{8cm}{Graphe de
$\displaystyle g(x) = 5 + e^{(2-x/10)}$}{Le graphe
de $\displaystyle g(x) = 5 + e^{(2-x/10)}$.  Nous avons que $g(x)$
approche $5$ lorsque $x$ tend vers plus l'infini.}{LIM1}

Commençons par donner un sens mathématique à l'énoncé \lgm $f(x)$
approche une certaine valeur lorsque $x$ tend vers plus l'infini\rgm.
Pour se faire, nous utilisons la définition~\ref{def_of_lim_at_inf} de
limite à l'infini pour les suites.  Nous obtenons la définition suivante
de limite d'une fonction $f$ lorsque $x$ tend vers plus ou moins
l'infini.

\begin{focus}{\dfn} \index{Limite d'une fonction!à l'infini}
Si la définition~\ref{def1_conv} est satisfaite lorsque $c$ est
remplacé par $\infty$, nous disons que {\bfseries $f(x)$ converge (ou tend)
vers la constante $C$ lorsque $x$ converge (ou tend) vers plus
l'infini} et nous écrivons
\[
\lim_{x\rightarrow \infty} f(x) = C \; .
\]
De même, si la définition~\ref{def1_conv} est satisfaite lorsque $c$
est remplacé par $-\infty$, nous disons que {\bfseries $f(x)$ converge (ou
tend) vers la constante $C$ lorsque $x$ converge (ou tend) vers moins
l'infini} et nous écrivons
\[
\lim_{x\rightarrow -\infty} f(x) = C \; .
\]
\label{def_of_lim_of_f_at_inf}
\end{focus}

\begin{egg}
Revenons à l'exemple précédent avec
$\displaystyle g(x) = 5 + e^{\left(2 - x/10\right)}$.

La suite $\displaystyle \left\{ x_n \right\}_{n=1}^\infty$ où
$x_n = n^2$ pour $n=1$, $2$, $3$, \ldots est une suite qui tend vers
plus l'infini car c'est une suite croissante de nombres sans borne
supérieure.  Nous donnons, dans le tableau~\ref{TAB_LIM_INF}, les valeurs
de $g(x_n)$ pour quelques uns des termes de la suite
$\displaystyle \left\{ x_n\right\}_{n=1}^\infty$.  Il est clair que
$g(x_n)$ approche $5$ lorsque $n$ (et donc $x_n$) devient de plus en
plus grand.

\begin{table}
\begin{center}
\begin{tabular}{c|c|c}
\hline
$n$ & $x_n = n^2$ & $g(x_n)$ \\ 
\hline
$1$ & $1$ & $11.68589444228\ldots$ \\ 
$2$ & $4$ & $9.95303242440\ldots$ \\ 
$3$ & $9$ & $8.00416602395\ldots$ \\ 
$4$ & $16$ & $6.49182469764\ldots$ \\ 
$5$ & $25$ & $5.60653065971\ldots$ \\ 
$\vdots$ & $\vdots$ & $\vdots$ \\ 
$11$ & $121$ & $5.00004107956\ldots$ \\ 
$12$ & $144$ & $5.00000411859\ldots$ \\ 
\hline
\end{tabular}
\qquad
\begin{tabular}{c|c|c}
\hline
$n$ & $x_n = n^2$ & $g(x_n)$ \\ 
\hline
$13$ & $169$ & $5.00000033807\ldots$ \\ 
$14$ & $196$ & $5.00000002272\ldots$ \\ 
$15$ & $225$ & $5.00000000125\ldots$ \\ 
$16$ & $256$ & $5.00000000006\ldots$ \\ 
$\vdots$ & $\vdots$ & $\vdots$ \\
$22$ & $484$ & $5.00000000000\ldots$ \\ 
$23$ & $529$ & $5.00000000000\ldots$ \\ 
$24$ & $576$ & $5.00000000000\ldots$ \\ 
\hline
\end{tabular}
\caption[Limite à l'infini d'une fonction]{$g(x_n)$ pour quelques
  valeurs de $n$. \label{TAB_LIM_INF}}
\end{center}
\end{table}

Puisque pour toute suite
$\displaystyle \left\{ x_n \right\}_{n=1}^\infty$
qui tend vers plus l'infini, nous pourrions montrer que $g(x_n)$ approche
$5$ lorsque $n$ devient de plus en plus grand, nous avons que
$\displaystyle \lim_{x\rightarrow \infty} g(x) = 5$.  Ainsi, $g(x)$
tend vers $5$ lorsque $x$ tend vers plus l'infini.

La droite $y=5$ est appelée une {\bfseries asymptote horizontale} pour
la fonction $g$.\index{Asymptote horizontale}

Nous pourrions aussi résonner à partir du graphe de $y=e^x$ pour montrer
que $\displaystyle \lim_{x\rightarrow \infty} g(x) = 5$.  En effet,
puisque $\displaystyle 2- x/10$ devient de plus en plus petit
(de plus en plus \lgm négatif\rgm) lorsque $x$ devient de plus en plus
grand, nous pouvons conclure à partir du graphe de $y=e^x$ que
$\displaystyle e^{\left(2- x/10\right)}$ tend vers $0$ lorsque
$x$ tend vers plus l'infini.  Donc
$\displaystyle 5+e^{\left(2- x/10\right)}$ tend vers $5$ lorsque
$x$ tend vers plus l'infini.
\label{asympt5}
\end{egg}

Les propositions \ref{suite1} et \ref{suite2} ne sont pas limitées à
la suite $\displaystyle \{n\}_{n=1}^\infty$.

\begin{focus}{\prp} \label{limite1}
Soit $r$ un nombre réel positif.  Alors
\[
\lim_{x\rightarrow \infty} r^x=
\begin{cases}
\infty &\qquad \text{pour $r>1$} \\
1 &\qquad \text{pour $r=1$} \\
0 &\qquad \text{pour $0 < r < 1$}
\end{cases} \; .
\]
\end{focus}

\begin{focus}{\prp} \label{limite2}
Soit $r$ un nombre réel.  Alors
\[
\lim_{x\rightarrow \infty} \frac{1}{x^r}=
\begin{cases}
0 &\qquad \text{pour $r>0$} \\
1 &\qquad \text{pour $r=0$} \\
\infty &\qquad \text{pour $r<0$}
\end{cases} \; .
\]
\end{focus}

\begin{rmk}
Dans les deux propositions précédentes, nous faisons référence à la
convergence d'une fonction vers plus l'infini lorsque $x$ tend vers
plus l'infini.  La définition~\ref{def_of_inf_lim_of_f} ci-dessous
explique ce que cela veut dire.  C'est deux propositions 
seront extrêmement utiles pour calculer des limites.
\end{rmk}

\begin{egg}
Évaluons les limites suivantes si elles existent.
\begin{center}
\begin{tabular}{*{2}{l@{\hspace{0.6em}}l@{\hspace*{2.6em}}}l@{\hspace{0.6em}}l}
\subQ{a} & $\displaystyle \lim_{x\to \infty} \frac{x^2 + x  +1}{5x^2 + 1}$ &
\subQ{b} & $\displaystyle \lim_{x\to \infty} \frac{x-4}{x^{3/2} + 2x +1}$ &
\subQ{c} & $\displaystyle \lim_{x\to \infty} \frac{\sqrt{4x^4+5}}
{\left(x^8 + x^2 + 1\right)^{1/4}}$
\end{tabular}
\end{center}

\subQ{a} Si nous divisons le numérateur et dénominateur par $x^2$, alors
\[
\lim_{x\to \infty} \frac{x^2 + x  +3}{5x^2 + 1}
= \lim_{x\to \infty} \frac{1 + (1/x)  + (3/x^2)}{5 + (1/x^2)}
= \frac{\displaystyle 1 + \lim_{x\to \infty}(1/x)  + 3\lim_{x\to \infty}(1/x^2)}
{\displaystyle 5 + \lim_{x\to \infty}(1/x^2)}
= \frac{1}{5}
\]
grâce à la proposition~\ref{limite2} qui nous donne
$\displaystyle \lim_{x\to \infty}(1/x) = \lim_{x\to \infty}(1/x^2) = 0$.

\subQ{b} Si nous divisons le numérateur et dénominateur par $x^{3/2}$, alors
\begin{align*}
\lim_{x\to \infty} \frac{x-4}{x^{3/2} + 2x +1}
&= \lim_{x\to \infty} \frac{(1/x^{1/2})-(4/x^{3/2})}
{1 + (2/x^{1/2}) +(1/x^{3/2})}\\
&= \frac{\displaystyle \lim_{x\to \infty} (1/x^{1/2})
 - 4 \lim_{x\to \infty} (1/x^{3/2})}
{\displaystyle 1 + 2 \lim_{x\to \infty}(1/x^{1/2})
 + \lim_{x\to \infty} (1/x^{3/2})}
= \frac{0}{1} = 0
\end{align*}
grâce encore à la proposition~\ref{limite2} qui nous donne
$\displaystyle \lim_{x\to \infty}(1/x^{1/2}) = \lim_{x\to \infty}(1/x^{3/2}) = 0$.

\subQ{c} Si nous factorisons $x^4$ à l'extérieur de la racine carrée et
$x^8$ à l'extérieur de la racine quatrième, alors
\[
\frac{\sqrt{4x^4+5}}{\left(x^8 + x^2 + 1\right)^{1/4}}
= \frac{x^2 \sqrt{4 + (5/x^4)}}{x^2\left(1 + (1/x^6) + (1/x^8)\right)^{1/4}}
= \frac{\sqrt{4 + (5/x^4)}}{\left(1 + (1/x^6) + (1/x^8)\right)^{1/4}} \ .
\]
Ainsi,
\begin{align*}
\lim_{x\to \infty} \frac{\sqrt{4x^4+5}}{ \left(x^8 + x^2 + 1\right)^{1/4}}
&= \lim_{x\to \infty}
\frac{\sqrt{4 + (5/x^4)}}{\left(1 + (1/x^6) + (1/x^8)\right)^{1/4}}\\
&= \frac{\sqrt{\displaystyle 4 + \lim_{x\to \infty} (5/x^4)}}
{\left(\displaystyle 1 + \lim_{x\to \infty} (1/x^6)
+ \lim_{x\to \infty} (1/x^8)\right)^{1/4}}
= \frac{\sqrt{4}}{1^{1/4}} = 2
\end{align*}
car $\displaystyle \lim_{x\to \infty}(1/x^r) = 0$ pour $r>0$ comme nous
avons énoncé à la proposition~\ref{limite2} .
\end{egg}

\begin{egg}
Déterminons si la limite suivante existe
\[
\lim_{x\rightarrow \infty} \left(\sqrt{x+1}-\sqrt{x}\right)
\]
et, si elle existe, qu'elle est cette limite.

Remarquons que
\[
\sqrt{x+1}-\sqrt{x} = \left
(\sqrt{x+1}-\sqrt{x}\right)
\left(\frac{\sqrt{x+1}+\sqrt{x}}{\sqrt{x+1}+\sqrt{x}}\right)
= \frac{1}{\sqrt{x+1}+\sqrt{x}} \ .
\]
Puisque $\sqrt{x+1}+\sqrt{x}$ croît sans borne supérieure lorsque $x$
augmente, nous avons
\[
\lim_{x\rightarrow \infty} \left(\sqrt{x+1}-\sqrt{x}\right)
= \lim_{x\rightarrow \infty} \frac{1}{\sqrt{x+1}+\sqrt{x}} = 0 \ .
\]
\label{conjinfty}
\end{egg}

Nous avons
\[
\lim_{n\rightarrow \infty} f(x_n) = \infty
\]
si la suite $\displaystyle \left\{ f(x_n) \right\}_{n=1}^\infty$
satisfait la définition~\ref{def_of_lim_at_inf} de convergence vers
plus l'infini.  De même,
\[
\lim_{n\rightarrow \infty} f(x_n) = -\infty
\]
si la suite $\displaystyle \left\{ f(x_n) \right\}_{n=1}^\infty$
satisfait la définition~\ref{def_of_lim_at_inf} de convergence vers
moins l'infini.  Nous pouvons ainsi donner la définition suivante.

\begin{focus}{\dfn} \index{Limite infinie d'une fonction}
Soit $f$ une fonction définie pour $x$ près de $c$.  Nous écrivons
\[
\lim_{x\rightarrow c} f(x) = +\infty  \quad \text{(resp. $-\infty$)}
\]
si
\[
\lim_{n\rightarrow \infty} f(x_n) = +\infty  \quad \text{(resp. $-\infty$)} 
\]
{\em quelle que soit la suite}
$\displaystyle \left\{x_n\right\}_{n=1}^\infty$ de
{\em nombres différents de $c$} qui tend vers $c$.
Nous disons que {\bfseries $f(x)$ converge (ou tend) vers plus l'infini
(resp. moins l'infini) lorsque $x$ converge (ou tend) vers $c$}.

Il n'est pas nécessaire que $f$ soit définie à $c$, et nous pouvons avoir
$c = +\infty$ ou $-\infty$.
\label{def_of_inf_lim_of_f}
\end{focus}

\begin{egg}
Si $\displaystyle f(x) = \frac{2}{(x-3)^2}$, que se passe-t-il lorsque
$x$ approche $3$?  Remarquons que $f(x)$ n'est pas définie pour
$x=3$.

Une façon de montrer que
\[
\lim_{x\rightarrow 3} \frac{2}{(x-3)^2} = \infty
\]
serait de montrer que, quelle que soit la suite
$\displaystyle \left\{x_n\right\}_{n=1}^\infty$ qui tend vers $3$, la
suite
$\displaystyle \left\{\frac{2}{(x_n-3)^2}\right\}_{n=1}^\infty$
tend vers plus l'infini.   Nous ne ferons pas cela mais le lecteur
peut vérifier avec une suite
$\displaystyle \left\{x_n\right\}_{n=1}^\infty$
de son choix qui tend vers $3$ que
$\displaystyle \left\{\frac{2}{(x_n-3)^2}\right\}_{n=1}^\infty$
tend vers plus l'infini.

Nous utilisons une autre approche (un peu moins rigoureuse) pour nous
convaincre que
\[
\lim_{x\rightarrow 3} \frac{2}{(x-3)^2} = \infty \; .
\]
Si $x$ est très près de $3$ avec $x>3$ alors $x-3$ est très près de
zéro avec $x-3>0$.  Donc $(x-3)^2$ est encore plus près de zéro que
$x-3$ peut l'être si $0<x-3<1$.  Ainsi, $2/(x-3)^2 >0$ est très grand
car nous divisons par un très petit nombre.   Plus $x-3>0$ sera petit, plus
$2/(x-3)^2 >0$ sera grand.  Un raisonnement semblable
pour $x$ très près de $3$ avec $x<3$, montre que $2/(x-3)^2>0$ devient
aussi de plus en plus grand lorsque $3-x>0$ devient de plus en plus
petit.

Si nous résonnons à partir du graphe de $f$ près de $3$ que nous retrouvons
à la figure~\ref{LIM4}, nous pouvons conclure que la fonction $f$ croît
sans borne supérieure lorsque $x$ tend vers $3$.  La droite $x=3$ est
appelée une
{\bfseries asymptote verticale}\index{Asymptote verticale} pour $f$. 
\end{egg}

\MATHfig{4_limites/lim_inf4}{8cm}{Asymptote verticale pour
$f(x) = 2/(x-3)^2$}{Le graphe de $f(x) = 2/(x-3)^2$.  $f$ croit sans
limite supérieure lorsque $x$ approche $3$.  La droite $x=3$ est une
asymptote verticale.}{LIM4}

\begin{egg}
Si $\displaystyle f(x) = \frac{2}{x^2-9}$, que se passe-t-il lorsque
$x$ approche $3$?  Remarquons que $f(x)$ n'est pas définie pour
$x=3$.

Si $x$ est très près de $3$ avec $x>3$ alors $x^2-9$ est très près de
zéro avec $x^2-9>0$.  Donc $2/(x^2-9)>0$ est très grand car nous divisons
par un très petit nombre.  Si $x$ est très près de $3$ avec $x<3$
alors $x^2-9$ est très près de zéro avec $x^2-9<0$.  Donc
$2/(x^2-9)<0$ est très petit car nous divisons par un nombre négatif qui
est très petit en valeur absolue.

Le graphe de $f$ près de $3$ que nous retrouvons à la figure~\ref{LIM3}
indique que la fonction $f$ croît sans borne supérieure lorsque $x$
approche $3$ avec $x>3$ et la fonction $f$ décroît sans borne
inférieure lorsque $x$ approche $3$ avec $x<3$.  La fonction
$2/(x^2-9)$ ne tend pas vers plus l'infini ou moins l'infini lorsque
$x$ tend vers $3$.
\end{egg}

\MATHfig{4_limites/lim_inf3}{8cm}{Asymptote verticale pour
$f(x) = 2/(x^2-9)$}{Le graphe de $f(x) = 2/(x^2-9)$.  $f$ croît sans
limite supérieure lorsque $x$ approche $3$ avec $x>3$ et $f$ décroît
sans limite inférieure lorsque $x$ approche $3$ avec $x<3$.}{LIM3}

Si $f$ est une fonction qui n'est pas définie en un point $c$,
l'exemple précédent suggère de considérer le comportement de cette
fonction lorsque nous nous approchons de $c$ avec des valeurs plus petites
que $c$ (i.e. par la gauche) ou des valeurs plus grandes que $c$ (i.e.
par la droite).

\begin{focus}{\dfn} \index{Limite infinie d'une fonction!à droite}
\index{Limite infinie d'une fonction!à gauche}
Soit $f$ une fonction définie pour $x>c$.  Nous écrivons
\[
\lim_{x\rightarrow c^+} f(x) = +\infty  \quad \text{(resp. $-\infty$)}
\]
si
\[
\lim_{n\rightarrow \infty} f(x_n) = +\infty  \quad \text{(resp. $-\infty$)} 
\]
{\em quelle que soit la suite}
$\displaystyle \left\{x_n\right\}_{n=1}^\infty$ avec {\em $x_n >c$}
qui tend vers $c$.
Nous disons que {\bfseries $f(x)$ converge (ou tend) vers plus l'infini
(resp. moins l'infini) lorsque $x$ converge par la droite vers $c$}.
De même, nous écrivons
\[
\lim_{x\rightarrow c^-} f(x) = \infty \quad \text{(resp. $-\infty$)}.
\]
si
\[
\lim_{n\rightarrow \infty} f(x_n) = +\infty  \quad \text{(resp. $-\infty$)} 
\]
{\em quelle que soit la suite}
$\displaystyle \left\{x_n\right\}_{n=1}^\infty$ avec {\em $x_n <c$}
qui tend vers $c$.
Nous disons que {\bfseries $f(x)$ converge (ou tend) vers plus l'infini
(resp. moins l'infini) lorsque $x$ converge par la gauche vers $c$}
(figure~\ref{LIM_INFINI}).
\end{focus}

\PDFfig{4_limites/lim_infini}{Limites à droite et à gauche}
{Dans le graphe à de gauche
$\displaystyle \lim_{x\rightarrow c^-} f(x) = \infty$
alors que dans le graphe à de droite 
$\displaystyle \lim_{x\rightarrow c^+} f(x) = -\infty$.}{LIM_INFINI}

\begin{egg}
À l'exemple précédent, nous avons
\[
\lim_{x\rightarrow 3^-} \frac{2}{x^2-9} = -\infty \qquad
\text{et} \qquad 
\lim_{x\leftarrow 3^+} \frac{2}{x^2-9} = \infty \; .
\]
La droite $x=3$ est appelée une {\bfseries asymptote verticale} pour $f$.
\index{Asymptote verticale}
\end{egg}

\subsection{Des définitions plus pratiques \theory}

Dans les exemples précédents, nous nous sommes inspiré du graphe de la
fonction pour déterminer les limites à l'infini et les limites
infinies en un point.  Ce n'est pas une approche rigoureuse et elle
dépend de notre capacité à tracer le graphe de la fonction.  Il nous
faut donc une définition de limite à l'infini et une définition de
limite infinie en un point qui soient en accord avec la
définition~\ref{def2_conv} de limite d'une fonction en un point en
termes de $\epsilon$ et $\delta$.

La définition~\ref{def2_conv} de la limite d'une fonction en un point
(équivalent à celle donné à la définition~\ref{def1_conv}) ne faisait
pas appelle aux suites.  Nous pouvons faire de même pour la définition de la
limite d'une fonction à l'infini
(définition~\ref{def_of_lim_of_f_at_inf}) et la définition de la limite
infinie d'une fonction en un point (définition~\ref{def_of_inf_lim_of_f}).

Ce sont ces définitions qui nous permettent de déterminer
rigoureusement si une limite à l'infini existe et qu'elle est cette
limite, ou si une fonction à une limite infinie en un point.

\begin{focus}{\dfn} \index{Limite d'une fonction!à l'infini}
Soit $f$ une fonction définie pour $x$ positif.  Nous écrivons
\[
\lim_{x\rightarrow \infty} f(x) = C
\]
et nous disons que $f(x)$ dent vers $C$ lorsque $x$ dent vers plus l'infini
si la condition suivante est satisfaite.  Il existe une
{\em unique constante $C$} telle que, pour toute valeur $\epsilon>0$,
nous pouvons trouver une constante $M>0$ (qui peut dépendre de $\epsilon$)
pour laquelle $f(x)$ est dans l'intervalle $]C-\epsilon,C+\epsilon[$
si $x>M$ (figure~\ref{LIM_INFINI1}).

De façons semblables, nous pouvons définir la limite vers moins l'infini.
\label{epM_def_of_lim_at_inf}
\end{focus}

\PDFfig{4_limites/lim_infini1}{Asymptote horizontale}
{La fonction $f$ tend vers $C$ lorsque $x$ tend vers plus l'infini.
Pour $\epsilon>0$ donné, nous pouvons voir à partir du graphe de $f$ qu'il
existe $M>0$ tel que $f(x)$ est entre $C-\epsilon$ et $C+\epsilon$ si
$x>M$.}{LIM_INFINI1}

\begin{focus}{\dfn} \index{Limite infinie d'une fonction}
Soit $f$ une fonction définie pour $x$ près d'un point $c$.  Nous écrivons
\[
\lim_{x\rightarrow c} f(x) = +\infty
\]
et nous disons que $f(x)$ tend vers plus l'infini lorsque $x$ tend vers $c$
si la condition suivante est satisfaite.
{\em Pour toute constante $M$}, il existe une constante $\delta>0$
(qui peut dépendre de $M$) pour laquelle $f(x) > M$ lorsque $x$ est
dans l'intervalle $]c-\delta,c+\delta[$ (figure~\ref{LIM_INFINI0}).

De façons semblables, nous pouvons définir la convergence vers moins
l'infini, et les convergences à droite et à gauche vers plus ou moins
l'infini.
\end{focus}

\PDFfig{4_limites/lim_infini0}{Asymptote verticale}
{La fonction $f$ tend vers plus l'infini lorsque $x$ tend vers
$c$.  Pour $M>0$ donné, nous pouvons voir à partir du graphe de $f$ qu'il
existe $\delta>0$ tel que $f(x)>M$ si $x$ est entre $c-\delta$ et
$c+\delta$.}{LIM_INFINI0}

\begin{egg}
Revenons à $\displaystyle f(x) = \frac{2}{x^2-9}$ que nous avons étudié
précédemment.  Montrons à l'aide des définitions précédentes que 
\[
\lim_{x\rightarrow c^-} f(x) = -\infty \quad \text{and} \quad
\lim_{x\rightarrow c^+} f(x) = +\infty \; .
\]

Prenons un très grand nombre $M$ (e.g.\ $M=10^6$).  Existe-t-il un
nombre $\delta>0$ tel que
\[
f(x)=\frac{2}{x^2-9} > M
\]
lorsque $3<x<3+\delta$?  Cherchons les valeurs de $x>3$ telles que
$\displaystyle \frac{2}{x^2-9} > M$; c'est-à-dire, telles que
$x^2 < 9 + 2/M$.  Il faut donc que
\[
3<x < \sqrt{9+2/M} = 3 + \left(\sqrt{9+2/M} - 3\right) \; .
\]
Si $\delta = \sqrt{9+2/M} - 3$, alors $\delta > 0$ car
$M>0$.  De plus, $f(x)>M$ pour $3<x<3+\delta$.  Le lecteur est invité à
calculer $\delta$ pour une valeur $M$ de son choix.  Nous venons de
montrer que quel que soit $M>0$, nous pouvons toujours trouver un nombre
$\delta>0$ tel que $f(x)>M$ pour $3<x<3+\delta$.  Ce qui prouve bien
que $f(x)$ dent vers $+\infty$ lorsque $x$ tend par la droite vers
$3$.

De même, prenons un très petit nombre $M$ (e.g.\ $M=-10^8$).
Existe-t-il un nombre $\delta>0$ tel que
\[
f(x)=\frac{2}{x^2-9} > M
\]
lorsque $3-\delta <x <3$?  Cherchons les valeurs de $x$ entre $0$ et
$3$ telles que $\displaystyle \frac{2}{x^2-9} < M$; c'est-à-dire,
telles que $x^2 > 9 + 2/M$.  Il faut donc que
\[
3> x > \sqrt{9+2/M} = 3 - (3 - \sqrt{9+2/M})
\]
où nous assumons que $M<-2/9$ pour que $9+2/M$ soit positif.
Si $\delta = 3 - \sqrt{9+2/M}$, alors $\delta > 0$ car $M < -2/9$.
De plus, $f(x)<M$ pour $3-\delta<x<3$.  Nous venons de montrer que
quel que soit $M<0$, nous pouvons toujours trouver un nombre
$\delta>0$ tel que $f(x)<M$ pour $3>x>3-\delta$.  Ce qui prouve bien
que $f(x)$ tend vers $-\infty$ lorsque $x$ tend par la gauche vers $3$. 
\end{egg}

\begin{egg}
Revenons au calcul de la limite
\[
\lim_{x\rightarrow \infty} \left(\sqrt{x+1}-\sqrt{x}\right)
\]
que nous avons étudié à l'exemple~\ref{conjinfty}.

Démontrons que
\[
\lim_{x\rightarrow \infty} \left(\sqrt{x+1}-\sqrt{x}\right)
= \lim_{x\rightarrow \infty} \frac{1}{\sqrt{x+1}+\sqrt{x}} = 0
\]
à partir de notre définition de limite à l'infini.  Soit
$\epsilon > 0$ quelconque mais fixe.  Si nous prenons 
$\displaystyle M = \frac{1}{4\epsilon^2}$, nous obtenons
\[
\left| \frac{1}{\sqrt{x+1}+\sqrt{x}} - 0 \right|
= \frac{1}{\sqrt{x+1}+\sqrt{x}} < \frac{1}{2\sqrt{x}}
< \frac{1}{2\sqrt{1/(4\epsilon^2)}} = \epsilon
\]
pour $x>M$.
\end{egg}

Pour terminer, nous considérons le cas où nous avons une limite infinie à
l'infini.

\begin{focus}{\dfn} \index{Limite infinie à l'infini}
Soit $f$ une fonction définie pour $x$ positif.  Nous écrivons
\[
\lim_{x\rightarrow +\infty} f(x) = +\infty
\]
et nous disons que $f(x)$ tend vers plus l'infini lorsque $x$ tend vers
plus l'infini si {\em pour toute constante $M>0$} il existe une constante
$m>0$ (qui peut dépendre de $M$) pour laquelle $f(x) > M$ lorsque $x>m$.  De
même, nous écrivons
\[
\lim_{x\rightarrow +\infty} f(x) = -\infty
\]
et nous disons que $f(x)$ tend vers moins l'infini lorsque $x$ tend vers
plus l'infini si {\em pour toute constante $M<0$} il existe une constante
$m>0$ (qui peut dépendre de $M$) pour laquelle $f(x) < M$ lorsque $x>m$.

De façons semblables, nous pouvons définir la convergence de $f(x)$ vers plus
l'infini ou moins l'infini lorsque $x$ converge vers moins l'infini.
\end{focus}

\subsection{Comportement asymptotique semblable \life}

\begin{focus}{\dfn}
\index{Comportement asymptotique semblable!au voisinage d'un point}
Soit $f$ et $g$ deux fonctions telles que $f(x)$ et $g(x)$ tendent
vers l'infini lorsque $x$ tend vers $c$.  Nous disons que $f$ et $g$ ont un
{\bfseries comportement asymptotique semblable} lorsque $x$ tend vers
$c$ si
\[
\lim_{x\rightarrow c} \frac{f(x)}{g(x)} = 1 \; .
\]
\end{focus}

Si $f$ et $g$ ont un comportement asymptotique semblable près de $c$,
alors $f(x) \approx g(x)$ pour $x$ très près de $c$.  Ainsi, les fonctions
$f$ et $g$ ont des graphes semblables pour $x$ près de $c$.

De même,

\begin{focus}{\dfn} \index{Comportement asymptotique semblable!à l'infinité}
Deux fonctions $f$ et $g$ ont un
{\bfseries comportement asymptotique semblable} lorsque $x$ tend vers
plus l'infini (resp. moins l'infini) si
\[
\lim_{x\rightarrow +\infty} \frac{f(x)}{g(x)} = 1 \qquad
(\text{resp.}\quad
\lim_{x\rightarrow -\infty} \frac{f(x)}{g(x)} = 1 \;) \; .
\]
\end{focus}

\begin{egg}
À l'exemple~\ref{asympt5}, nous avons montré que
$\displaystyle \lim_{x\rightarrow +\infty} g(x) = 5$ 
où $\displaystyle g(x)= 5 + e^{2-x/10}$.  Si nous posons $f(x) = 5$ pour
tout $x$, alors
\[
\lim_{x\rightarrow +\infty} \frac{f(x)}{g(x)} = 1 \; .
\]
Donc $g$ et $f$ ont des graphes semblables lorsque $x$ est très grand.
\end{egg}

Nous reviendrons sur l'étude du comportement asymptotique des
fonctions à la section~\ref{asympt_comp}.

}  % End of theory

\section{Exercices}

\subsection{Limites}

\begin{question}
Utilisez une approche numérique et une approche graphique
(c'est-à-dire que vous devez utiliser un logiciel ou une calculatrice
graphique pour tracer le graphe) pour trouver la valeur de la limite
suivante si elle existe.
\[
\lim_{x\rightarrow 0}\frac{e^{2x} - 1}{x} \ .
\]
\label{4Q1}
\end{question}

\begin{question}
Le graphe de la fonction $p$ est donné ci-dessous.
\MATHgraph{4_limites/discount1}{8cm}
Évaluez graphiquement
$\displaystyle \lim_{t\rightarrow 1^-} p(t)$,
$\displaystyle \lim_{t\rightarrow 1^+} p(t)$ et
$\displaystyle \lim_{t\rightarrow 1} p(t)$ si cela est possible.
\label{4Q2}
\end{question}

\begin{question}
À l'aide de suites de valeurs numériques, estimez la valeur des limites
suivantes.
\begin{center}
\begin{tabular}{*{1}{l@{\hspace{1em}}l@{\hspace{3.5em}}}l@{\hspace{1em}}l}
\subQ{a} & $\displaystyle \lim_{x\rightarrow 0} \frac{1-\cos(x)}{x}$ &
\subQ{b} & $\displaystyle \lim_{x\rightarrow 0} \frac{\ln(1-x)}{x}$
\end{tabular}
\end{center}
\label{4Q3}
\end{question}

\begin{question}
Nous savons que $\displaystyle \lim_{x\rightarrow 0^+} \sqrt{x} = 0$.
Pour quelles valeurs de $x$ aurons-nous que $\sqrt{x} < 0.1$?  Que 
$\sqrt{x} < 0.01$?  Est-ce que $\sqrt{x}$ approche rapidement $0$
lorsque $x>0$ approche $0$?
\label{4Q4}
\end{question}

\begin{question}
Le graphe de la fonction $f$ est donné ci-dessous.
\PDFgraph{4_limites/discount2}
Calculez les limites à droite et à gauche aux points $x=1$ et $x=2$.
Est-ce que la limite de la fonction existe aux points $x=1$ et $x=2$?
\label{4Q5}
\end{question}

\begin{question}
Soit $v(t) = 1 + t^2$.  Évaluez la limite
$\displaystyle \lim_{t\rightarrow 0} v(t)$.  Soit $\alpha$ la valeur de cette
limite.
\begin{enumerate}
\item Pour quelles valeurs de $t$ aurons-nous que $|v(t) - \alpha| < 1$?
\item Pour quelles valeurs de $t$ aurons-nous que $|v(t) - \alpha| < 0.5$?
\item Pour quelles valeurs de $t$ aurons-nous que $|v(t) - \alpha| < 0.01$?
\end{enumerate}
\label{4Q6}
\end{question}

\begin{question}[\eng \life]
Évaluez si possible la limite $\displaystyle \lim_{x\to 0} x^2\sin(1/x)$.
\label{4Q7}
\end{question}

\begin{question}[\eng]
La fonction de Heaviside est définie par
\[
H(x) =
\begin{cases}
0 & \quad \text{si} \quad x < 0 \\
1 & \quad \text{si} \quad x \geq 0
\end{cases} \ .
\]
Est-ce que $\displaystyle \lim_{x\to 0} H(x) = 1$?  Justifiez votre
réponse.
\label{4Q8}
\end{question}

\subsection{Fonctions continues}

\begin{question}
Déterminez si les limites suivantes existent.  Évaluez la limite
quand elle existe.
\begin{center}
\begin{tabular}{*{2}{l@{\hspace{0.5em}}l@{\hspace{3em}}}l@{\hspace{0.5em}}l}
\subQ{a} & $\displaystyle \lim_{t\rightarrow 0} \frac{1+t+t^2}{1+t}$ &
\subQ{b} & $\displaystyle \lim_{x\rightarrow 0} \frac{e^x}{1+x}$ &
\subQ{c} & $\displaystyle \lim_{x\to 2} \frac{(x-2)}{2+\sqrt{2x^2-4}}$ \\[0.9em]
\subQ{d} & $\displaystyle \lim_{z\rightarrow 0} \frac{3z}{1+\ln(1+z)}$ &
\subQ{e} & $\displaystyle \lim_{x\to 3} \frac{\cos(\pi/x)}{x^2-5}$ & &
\end{tabular}
\end{center}
\label{4Q9}
\end{question}

\begin{question}
Déterminez si les limites suivantes existent.  Évaluez la limite
quand elle existe.
\begin{center}
\begin{tabular}{*{2}{l@{\hspace{0.5em}}l@{\hspace{3em}}}l@{\hspace{0.5em}}l}
\subQ{a} & $\displaystyle \lim_{x\to 3} \frac{(x-3)}{2-\sqrt{x^2-5}}$ &
\subQ{b} & $\displaystyle \lim_{x\to 5} \frac{|x-5|}{x^2-25}$ &
\subQ{c} & $\displaystyle \lim_{x\rightarrow 0}
    \frac{3-\sqrt{9-3x}}{8x}$ \\[0.9em]
\subQ{d} & $\displaystyle \lim_{x\to 1} \frac{|x-2|-1}{x^2-1}$ &
\subQ{e} & $\displaystyle \lim_{x\to -3} \frac{|x+3|(5+x)}{x+3}$ &
\subQ{f} & $\displaystyle \lim_{x\to 1} \frac{5x^4 + 3x^2-8}{12x^2-11x-1}$
\end{tabular}
\end{center}
\label{4Q10}
\end{question}

\begin{question}[\life]
Le volume d'une culture au temps $t$ en secondes est donnée par la
formule $V(t)=V_0 e^{\alpha t}$ ml où $V_0 = V(0) =1$ ml est le
volume initial.  Sachant que le volume est de $2.71828$ ml après
$1000$ secondes (i.e. $V(1000) = 2.71828$ ml), trouvez $\alpha$.
Déterminez les valeurs de $t$ pour lesquelles le volume $V(t)$ après
$t$ secondes est $2.71828$ ml avec une marge d'erreur de $0.1$ ml.
C'est-à-dire, trouvez $t$ tel que
$2.71828 -0.1 < V(t) < 2.71828 +0.1$.
\label{4Q11}
\end{question}

\begin{question}
Donnez une formule mathématique pour définir la fonction continue $f$
telle que $f(x)= -1$ pour $x<-0.1$, $f(x)= 1$ pour $x>0.1$, et $f(x)$
est linéaire pour $-0.1 < x < 0.1$.
\label{4Q12}
\end{question}

\begin{question}[\life]
Un neurone a la réaction suivante lorsqu'il reçoit une impulsion
électrique.  Si le voltage $V$ de l'impulsion électrique est
supérieure à une valeur $V_0$, le neurone produit une impulsion
électrique de voltage $2V$. Au contraire, si le voltage $V$ de
l'impulsion électrique est inférieure à cette valeur $V_0$, le neurone
produit une impulsion électrique de voltage $V_1$.  Donnez une formule
mathématique pour la réponse du neurone à une impulsion électrique?
Si nous savons que la réponse du neurone à une impulsion électrique est
une fonction continue, quelle doit être la valeur de $V_1$?
\label{4Q13}
\end{question}

\begin{question}
Soit
\[
f(x) =
\begin{cases}
\displaystyle a \sin\left(\frac{\pi x}{2}\right) &
\quad \text{si} \quad x < 3 \\
\displaystyle \frac{x^2}{2} - 4 & \quad \text{si} \quad x \geq 3
\end{cases}
\]
Trouvez la valeur de $a$ pour que $f$ soit continue sur la droite
réelle.  Donnez une justification claire et complète.
\label{4Q14}
\end{question}

\begin{question}
Soit
\[
f(x) = \begin{cases}
a x^2 + 4x & \quad \text{si}\quad x < 3 \\
x^2 + a x &  \quad \text{si}\quad x \geq 3
\end{cases}
\]
Trouvez la valeur de $a$ pour que $f$ soit continue sur la droite
réelle.  Donnez une justification claire et complète.
\label{4Q15}
\end{question}

\begin{question}
Soit
\[
f(x) = \begin{cases}
\displaystyle \frac{x^2 - 2 x- 3}{x-3} & \quad \text{si} \quad x<3 \\
a x^2 + x + b & \quad \text{si}\quad  3 \leq x < 4 \\
\displaystyle b \sqrt{x} + \frac{5 a x}{2} &  \quad \text{si}\quad x \geq 4
\end{cases}
\]
Trouvez les valeurs de $a$ et $b$ pour que $f$ soit continue sur la
droite réelle.  Donnez une justification claire et complète.
\label{4Q16}
\end{question}

\begin{question}[\life \eng]
Utilisez le Théorème des valeurs intermédiaires pour démontrer que
l'équation $e^x + x^2 -2 = x$ a au moins une solution.
\label{4Q17}
\end{question}

\begin{question}
Montrez qu'il existe au moins une solution de l'équation suivante sur
l'intervalle donnée.  Ne pas oublier d'indiquer le théorème que vous
avez utilisé et de vérifier que ses hypothèses sont satisfaites.
\begin{center}
\begin{tabular}{*{1}{l@{\hspace{0.5em}}l@{\hspace{6em}}}l@{\hspace{0.5em}}l}
\subQ{a} & $\tan(x) + 4x = x^2+e^x$ sur $[0,\pi/4]$ &
\subQ{b} & $\displaystyle e^{\cos(x/2)}=2\sin(x/2)$ sur $[0,\pi]$
\end{tabular}
\end{center}
\label{4Q18}
\end{question}

\begin{question}[\eco]
Le prix de l'essence a augmenté de \$$2.10$ à \$$2.50$ par litre au cours de
la semaine passée, pouvons-nous conclure à l'aide du Théorème des valeurs
intermédiaires que le prix de l'essence a été de \$$2.25$ le litre à un
moment au cours de la semaine passée?  Justifiez votre réponse.
\label{4Q19}
\end{question}

\subsection{Limites à l'infini et limites infinies}

\begin{question}[\eco]
Un diapason est un petit instrument en acier qui a la forme d'une
fourche et qui produit (approximativement) la note {\bfseries La}
lorsqu'il vibre.  La fréquence $x$ (en hertz) d'un bon diapason devrait
être très proche de la fréquence exacte du {\bfseries La} qui est de
$440$ hertz.  Une marque de diapason coûte $5/|x-440|$ dollars où $x$
est la fréquence du diapason.  Combien coûtera un tel diapason si nous
demandons une précision de $0.1$\%? De $0.01$\%?  Pouvons-nous se permettre
un diapason parfait?
\label{4Q20}
\end{question}

\begin{question}
Nous savons que
$\displaystyle \lim_{x\rightarrow 0^+} \frac{1}{\sqrt{x}} = +\infty$.
Pour quelles valeurs de $x$ aurons-nous que
$\displaystyle \frac{1}{\sqrt{x}} >10$? Que 
$\displaystyle \frac{1}{\sqrt{x}} >100$?  Est-ce que
$\displaystyle \frac{1}{\sqrt{x}}$ croît rapidement ou lentement
lorsque $x>0$ approche $0$?
\label{4Q21}
\end{question}

\begin{question}
Utilisez des suites pour évaluer
$\displaystyle \lim_{t\rightarrow 1}\ (1-t)^{-4}$.
\label{4Q22}
\end{question}

\begin{question}
Évaluez numériquement la limite suivante.
\[
\lim_{y\rightarrow 1^+} y^2\ln(y-1) \ .
\]
\label{4Q23}
\end{question}

\begin{question}
Déterminez si les limites suivantes existent.  Évaluez celles qui
existent.  Pour celles qui n'existent pas, expliquez pourquoi.
\begin{center}
\begin{tabular}{*{2}{l@{\hspace{1em}}l@{\hspace{3.5em}}}l@{\hspace{1em}}l}
\subQ{a} &
$\displaystyle \lim_{x\rightarrow \infty} \frac{x^2+5x-4}{3x^2 +1}$ &
\subQ{b} &
$\displaystyle \lim_{x\rightarrow \infty} \frac{x-(8x^3+3)^{1/3}}{x}$ &
\subQ{c} & $\displaystyle \lim_{x\rightarrow 0}\; \frac{1}{x}$ \\[1em]
\subQ{d} &
$\displaystyle \lim_{x\rightarrow \infty} \left(\sqrt{x^2+3x}
- \sqrt{x^2 + 7x}\right)$ &
\subQ{e} &
$\displaystyle \lim_{x\rightarrow \infty} \frac{2e^{2x} -e^{-3x}}
{3e^{2x} - 4e^{-5x}}$ & &
\end{tabular}
\end{center}
\label{4Q24}
\end{question}

\begin{question}[\life \eng]
Évaluez si possible la limite
$\displaystyle \lim_{x\to \infty} \frac{\sin^2(x)}{1+x^2}$.
\label{4Q25}
\end{question}


%%% Local Variables: 
%%% mode: latex
%%% TeX-master: "notes"
%%% End: 

\chapter[Intégrale]{Intégrale}\label{chapter_integr}

\compileTHEO{

L'intégration de fonctions est le deuxième concept fondamental de ce
manuel.  Nous définissons ce qu'est la {\em primitive d'une fonction} dans
la première partie du chapitre.  Nous donnons aussi les principales
techniques pour calculer les primitives d'une fonction.   La deuxième
partie du chapitre présente {\em l'intégrale d'une fonction} d'une
variable sur un intervalle de longueur finie.

Le {\em Théorème fondamental du calcul} fera le lien 
entre le calcul des primitives et l'intégration de fonctions sur un
intervalle de longueur finie.  C'est sans aucun doute le plus
important théorème du calcul différentiel et intégral d'où son nom de
théorème fondamental.

Avant de terminer le chapitre, nous donnons la définition de l'intégrale
d'une fonction sur un intervalle de longueur infinie.  Nous terminons le
chapitre en présentant quelques techniques pour calculer numériquement
les intégrales quand il s'avère trop compliqué ou fastidieux de les
calculer analytiquement avec les techniques que nous présentons.

Les applications de l'intégrale d'une fonction seront présentées au
chapitre suivant.

\section{Primitives et intégrales indéfinies}

\begin{defn} \index{Primitive}
Une fonction $F:]a,b[\to \RR$ est une {\bfseries primitive} de la
fonction $f:]a,b[\to \RR$ si $F'(x) = f(x)$ pour tout $x \in ]a,b[$.
\end{defn}

\begin{egg}
La fonction $F(x) = \sin(x)$ est une primitive de $f(x) = \cos(x)$ car
$F'(x) = \cos(x) = f(x)$ pour tout $x$.  La fonction $F(x) = x^7/7$ est une
primitive de $f(x) = x^6$ car $F'(x) = x^6 = f(x)$ pour tout $x$.
La fonction $F(x) = \ln|x|$ est une primitive de $f(x) = 1/x$ car
$F'(x) = 1/x = f(x)$ pour tout $x \neq 0$. 
\end{egg}

Si $F$ est une primitive de $f$ alors, quelle que soit la constante
$C$, la fonction $G$ définie par $G(x)=F(x) + C$ pour tout
$x\in]a,b[$ est aussi une primitive de $f$ car
\[
\dfdx{G(x)}{x} = \dfdx{F(x)}{x} + \dfdx{C}{x} = f(x) + 0 = f(x)
\]
pour tout $x \in ]a,b[$.  Il y a donc une infinité de primitives de
$f$.  Nous avons une différente primitive de $f$ pour chaque valeur de $C$.

En fait, la différence entre deux primitives de $f$ est toujours une
constante.  Supposons que $F_1$ et $F_2$ soient deux primitives de $f$
et posons $G = F_1 - F_2$.  Puisque
\[
G'(x) = F_1'(x) - F_2'(x) = f(x) - f(x) = 0
\]
pour tout $x \in]a,b[$, la fonction $G$ est une fonction constante;
c'est-à-dire qu'il existe une constante $C$ telle que
$G(x) = F_1(x) - F_2(x) = C$ pour tout $x\in ]a,b[$.
Ainsi, $F_1(x) = F_2(x) + C$ pour tout $x \in ]a,b[$.

\begin{rmk}[\theory]
En fait, la justification précédente assume que si $G'(x)=0$ pour tout
$x\in]a,b[$ alors $G$ est une fonction constante.  Nous utilisons
l'interprétation de la dérivée comme la pente de la tangente à la
courbe pour tirer cette conclusion.  Une démonstration rigoureuse fait
appel au Théorème des accroissements finis, le théorème~\ref{MVT}.

En effet, soit $x \in ]a,b[$ et $c \in ]a,b[$.  Grâce au Théorème des
accroissements finis, nous savons qu'il existe $\xi$ entre $x$ et $c$ (où
$\xi$ dépend de $x$ et $c$) telle que
\[
G(x) - G(c) = G'(\xi)(x-c)  \quad .
\]
Puisque $G'(x) = 0$ pour tout $x \in]a,b[$, nous avons que
$G(x) - G(c) = 0$.  Comme ceci est vrai pour tout $x\in ]a,b[$, alors
$G(x) = G(c)$ pour tout $x\in ]a,b[$.
\end{rmk}

\begin{defn}\index{Intégrale indéfinie}\index{Intégrande}
\index{Variable d'intégration}
{\bfseries L'intégrale indéfinie} d'une fonction $f$ est la famille de
primitives pour cette fonction.  Nous dénotons l'intégrale indéfinie par
$\displaystyle \int f(x) \dx{x}$.  Si $F$ est une primitive de
$f$, alors
\[ 
\int f(x)  \dx{x} = F(x) + C
\]
où $C \in \RR$.  La fonction $f$ est appelée {\bfseries l'intégrande}
et $x$ est la {\bfseries variable d'intégration}.  Le symbole $\dx{x}$
indique que la variable d'intégration est $x$.
\end{defn}

Le symbole $\dx{x}$ {\em n'est pas une variable}, il indique
seulement que la variable d'intégration est $x$.  Aucune
manipulation algébrique avec $\dx{x}$ n'est permise.

\begin{egg}
Si nous utilisons les résultats de l'exemple précédent, nous obtenons les
intégrales indéfinies suivantes:
\[
\int \cos(x)  \dx{x} = \sin(x) + C \ , \quad
\int x^6  \dx{x} = \frac{x^7}{7} + C \quad \text{et} \quad
\int \frac{1}{x}  \dx{x} = \ln|x| + C \; .
\]
\end{egg}

À partir des formules de dérivation que nous avons présentées à la
section~\ref{basic_der}, nous pouvons construire la table des intégrales
que nous retrouvons dans le tableau~\ref{TABII}.

\begin{table}
\[
\begin{array}{c|c|cl}
f(x) & \displaystyle \int f(x)\dx{x} & \text{contraintes} \\
\hline
\rule{0em}{1.5em} x^\alpha & x^{\alpha+1}/(\alpha+1) + C &
\alpha \neq -1 \ \text{et}\ x^\alpha\ \text{est définie} \\ 
\rule{0em}{1.5em} 1/x & \ln|x| + C & x \neq 0 \\
\rule{0em}{1.5em} \cos(x) & \sin(x) + C & \\
\rule{0em}{1.5em} \sin(x) & -\cos(x) + C & \\
\rule{0em}{1.5em} \sec^2(x) & \tan(x) + C & \\
\rule{0em}{1.5em} e^x & e^x + C & \\
\rule{0em}{1.5em} 1/\sqrt{1-x^2} & \arcsin(x) + C & |x|<1 \\
\rule{0em}{1.5em} 1/(1+x^2) & \arctan(x) + C & \\
\hline
\end{array}
\]
\caption{Quelques intégrales indéfinies \label{TABII}}
\end{table}

Le prochain résultat est une conséquence de la linéarité de la
dérivée.

\begin{theorem}
Si $F$ et $G$ sont des primitives de $f$ et $g$ respectivement, alors
$a F + b G$ est une primitive de $a f + b g$.  Par conséquent,
\[
\int (a f(x) + b g(x)) \dx{x} = a \int f(x) \dx{x}
+ b \int g(x) \dx{x} \ .
\]
\end{theorem}

La démonstration de ce théorème est très simple.  Par linéarité de la
dérivée, nous avons
\[
\dfdx{\left(a F(x) + b G(x)\right)}{x}
= a \dfdx{F(x)}{x} + b \dfdx{G(x)}{x}
= a f(x) + b g(x)
\]
pour tout $x$.  Donc $aF+bG$ est une primitive de $af + bg$.  

Le théorème précédent peut être résumé en une seule phrase.
L'intégrale indéfinie d'une somme de fonctions est la somme des
intégrales indéfinies des fonctions de la somme, et l'intégrale
indéfinie du produit d'une fonction avec une constante est le produit
de l'intégrale indéfinie de la fonction avec cette constante. 

\begin{egg}
Calculons l'intégrale indéfinie de
$g(x) = 5 x^{-8} + 3 \cos(x) + 7/\sqrt{1-x^2}$.
\begin{align*}
\int g(x)\dx{x} &=
\int \left(5 x^{-8} + 3 \cos(x) + \frac{7}{\sqrt{1-x^2}} \right)\dx{x} \\
&= 5 \int x^{-8} \dx{x} + 3 \int \cos(x)\dx{x}
+ 7 \int \frac{1}{\sqrt{1-x^2}}\dx{x}  \\
&= 5 \left( \frac{x^{-7}}{-7}\right) + 3 \sin(x) + 7 \arcsin(x) + C\\
&= -\frac{5}{7x^7} + 3\cos(x) + 7 \arcsin(x) + C \; .
\end{align*}
\end{egg}

\begin{egg}
Si $f'(x) = x^4 + 5/(1+x^2)$ et $f(0)=2$, trouvons $f$.

La fonction $f$ est une primitive de $x^4 + 5/(1+x^2)$.  L'intégrale
indéfinie de $x^4 + 5/(1+x^2)$ est
\[
\int \left( x^4 + \frac{5}{1+x^2} \right) \dx{x}
= \int  x^4  \dx{x} + 5 \int\, \frac{1}{1+x^2}  \dx{x}
= \frac{x^5}{5} + 5 \arctan(x) + C \; .
\]
La fonction $f$ est donnée par $f(x) = x^5/5 + 5 \arctan(x) + C$ où
$C$ est choisi pour satisfaire $f(0)=2$.  Il faut donc avoir $C=2$ et
la fonction $f$ cherchée est $f(x) = x^5/5 + 5 \arctan(x) + 2$.
\end{egg}

\begin{egg}
L'exemple suivant se trouve dans (presque) tous les livres de calcul
différentiel qui ont été écrits.  Nous allons donc continuer la tradition
pour ne pas paraître trop radical.

Si $p(t)$ est la position d'un objet (se déplaçant en ligne droite),
sa vitesse au temps $t$ (i.e. le taux de changement instantané de la
position) est $v(t) = p'(t)$.  La position est une primitive de la
vitesse. L'accélération au temps $t$ (i.e. le taux de changement
instantané de la vitesse) est $a(t) = v'(t) = p''(t)$.  La vitesse est
une primitive de l'accélération.

Avec cette information, nous pouvons trouver le temps que prendra un objet
que nous laissons tomber d'une hauteur de $100$ m pour atteindre le sol.
Nous pouvons aussi trouver la vitesse à laquelle l'objet frappe le
sol.  Nous supposons que la friction de l'air n'a aucun effet sur l'objet.

Au départ, la position de l'objet est $p(0) = 100$ m.  Puisque nous
laissons tomber l'objet, sa vitesse initiale est $v(0)=0$ m/s.

L'accélération dû à l'attraction terrestre est bien connue et est
$-9.8$ m/s$^2$.  Nous utilisons le signe négatif pour l'accélération
pour indiquer que la direction positive du déplacement est vers le
haut.  Ainsi, $a(t) = -9.8$ pour tout $t$.

Puisque
\[
\int a(t)  \dx{t} = \int -9.8 \, \dx{t} = -9.8\, t + C
\]
pour une constante $C$, nous avons $v(t) = -9.8\,t +C$.  La constante $C$
est déterminée par la condition $v(0)=0$.  Ainsi, $C=0$ et
$v(t) = -9.8\, t$ m/s.

De même, puisque
\[
\int v(t)  \dx{t} = \int -9.8 \,t\, \dx{t} = -9.8\, \int t \dx{t}
= -9.8 \, \frac{t^2}{2} + C
\]
pour une constante $C$, nous avons $p(t) = -4.9\,t^2 + C$.  La constante $C$
est déterminée par la condition $p(0) = 100$.  Ainsi, $C=100$ et
$p(t) = -4.9\, t^2 + 100$ m.

L'objet va toucher le sol lorsque $p(t) = -4.9\, t^2 + 100 = 0$.  nous
trouvons $t=4.5175395\ldots$ secondes.

L'objet prend donc $t=4.5175395\ldots$ secondes pour atteindre le sol
qu'il frappe à une vitesse de
$v(4.5175395\ldots) = -9.8 \times 4.5175395\ldots = -44.271887\ldots$ m/s.
Notons que le signe négatif pour la vitesse indique seulement que
l'objet se dirige vers le sol.
\end{egg}

\section{Techniques d'intégration}

Nous présentons quelques règles qui nous permettront de transformer
une intégrale indéfinie complexe en une autre intégrale indéfinie qui
fait appel seulement à des intégrales indéfinies simples comme celles
que nous retrouvons dans le tableau~\ref{TABII}.

\subsection{Substitutions}

La première méthode d'intégration que nous allons voir nous permettra
d'évaluer l'intégrale indéfinie de fonctions composées comme
$\sin(4x)$, $\sqrt{x+7}$, etc.

\begin{egg}
Calculons les intégrales indéfinies suivantes.
\begin{center}
\begin{tabular}{*{2}{l@{\hspace{0.5em}}l@{\hspace{2.5em}}}l@{\hspace{0.5em}}l}
\subQ{a} & $\displaystyle \int e^{5x} \dx{x}$  &
\subQ{b} & $\displaystyle \int x\sin(x^2) \dx{x}$ &
\subQ{c} & $\displaystyle \int \frac{x^3}{\sqrt{x^2+5}} \dx{x}$
\end{tabular}
\end{center}

\subQ{a} Nous utilisons la règle de la dérivée de fonctions composées pour
obtenir
\begin{equation}\label{inte5x}
\int e^{5x}  \dx{x} = \frac{e^{5x}}{5} + C
\end{equation}
car
\[
\dfdx{\left(\frac{e^{5x}}{5}\right)}{x} = \frac{5 \,e^{5x}}{5} = e^{5x} \; .
\]

\subQ{b} Nous avons
\begin{equation}\label{intcx2}
\int x\,\sin(x^2)  \dx{x} = \frac{-\cos(x^2)}{2} + C
\end{equation}
car
\[
\dfdx{\left(\frac{-\cos(x^2)}{2}\right)}{x} = \frac{2x \sin(x^2)}{2} =
x \sin(x^2) \; .
\]

\subQ{c} Que devons-nous faire?
\end{egg}

Alors que les deux premières intégrales indéfinies étaient assez
simple à deviner, il en est autrement de l'intégrale indéfinie en
(c).

Nous présentons une méthode d'intégration basée sur la règle de la
dérivée de fonctions composées qui nous permettra de trouver
l'intégrale indéfinie de $x^3/\sqrt{x^2+5}$.

En multipliant par $5$ des deux côtés de l'égalité en (\ref{inte5x}),
nous obtenons
\[
\int 5\,e^{5x}  \dx{x} = e^{5x} + C
\]
où $C$ est une nouvelle constante.  Cette intégrale indéfinie est de
la forme
\begin{equation}\label{guess_subs}
\int f(g(x)) g'(x)  \dx{x} = F(g(x)) + C
\end{equation}
où $f(x)=e^x$, $g(x)=5x$ et $F(x)=e^x$ est la primitive de $f$.
De même, si nous multiplions par $2$ des deux côtés de l'égalité en
(\ref{intcx2}), nous obtenons
\[
\int 2x\,\sin(x^2)  \dx{x} = -\cos(x^2) + C
\]
où $C$ est une nouvelle constante.  C'est une intégrale indéfinie de
la forme (\ref{guess_subs}) où $f(x)=\sin(x)$, $g(x)=x^2$ et
$F(x) = -\cos(x)$ est la primitive de $f$.

Est-ce que la formule (\ref{guess_subs}) est toujours vraie?

Soit $F$ une primitive d'une fonction $f$ et $g$ une fonction
différentiable.  Si $f$ et $F$ sont définies sur l'image de $g$, alors
la fonction composée $F\circ g$ est différentiable et
\[
\dfdx{F(g(x))}{x} = \left(\dfdx{F(y)}{y}\bigg|_{y=g(x)}\right) \dfdx{g(x)}{x}
= \left( f(y) \bigg|_{y=g(x)}\right) \dfdx{g(x)}{x} = f(g(x)) g'(x) \; .
\]
Ainsi, $F(g(x))$ est une primitive de $f(g(x))g'(x)$.  Nous obtenons le
résultat suivant.

\begin{theorem} \index{Règle de substitution}
\index{Changement de variable}
Supposons que
\begin{enumerate}
\item $g$ soit une fonction différentiable,
\item $F$ soit la primitive d'une fonction $f$, et
\item $f$ et $F$ soient définies sur l'image de $g$.
\end{enumerate}
Alors $F(g(x))$ est une primitive de $f(g(x))g'(x)$.  Nous en déduisons la
{\bfseries règle de substitution}
(ou {\bfseries méthode de changement de variable}) suivante.
\[
\int f(g(x))g'(x)  \dx{x} = \int f(y) \dx{y} \bigg|_{y=g(x)} =
F(g(x)) + C \; .
\]
\end{theorem}

Pour appliquer la règle de substitution, nous procédons de la façon
suivante.  Si nous posons $y=g(x)$, alors $\dx{y} = g'(x) \dx{x}$ et
\[
\int f(\,\underbrace{\; g(x) \; }_{=y}\,)
\underbrace{g'(x)  \dx{x}}_{=\dx{y}} = \int f(y) \dx{y}
\]
où il ne faut pas oublier de remplacer $y$ par $g(x)$ après avoir
calculé l'intégrale indéfinie de $f$.  L'expression
$\dx{y} = g'(x) \dx{x}$ {\em n'est pas une égalité algébrique} mais
une expression symbolique pour exprimer la procédure pour remplacer la
variable d'intégration $x$ par la variable d'intégration $y$.

Retournons à l'exemple précédent avec la règle de substitution en
main.

\begin{egg}
Évaluons les intégrales indéfinies suivantes.
\begin{center}
\begin{tabular}{*{2}{l@{\hspace{0.5em}}l@{\hspace{2.5em}}}l@{\hspace{0.5em}}l}
\subQ{a} & $\displaystyle \int e^{5x} \dx{x}$ &
\subQ{b} & $\displaystyle \int x\sin(x^2) \dx{x}$ &
\subQ{c} & $\displaystyle \int \frac{x^3}{\sqrt{x^2+5}} \dx{x}$
\end{tabular}
\end{center}

\subQ{a} Si $f(y) = e^y$ et $y = g(x)= 5x$, alors
$\dx{y} = g'(x) \dx{x} = 5 \dx{x}$ et
\[
\int e^{5x}\dx{x} = \frac{1}{5}
\int \underbrace{\quad e^{5x}\quad }_{=f(g(x))} \times
\underbrace{\quad 5 \quad}_{=g'(x)} \dx{x}
= \frac{1}{5} \int \underbrace{\quad e^y\quad }_{=f(y)} \dx{y}\bigg|_{y=5x}
= \frac{1}{5}\, e^y \bigg|_{y=5x} +C = \frac{e^{5x}}{5} +C \; .
\]

\subQ{b} Si $f(y) = \sin(y)$ et $y = g(x) = x^2$, alors
$\dx{y} = g'(x) \dx{x} = 2x \dx{x}$ et
\begin{align*}
\int x\,\sin(x^2)\dx{x} &= \frac{1}{2}
\int \underbrace{\sin(x^2)}_{=f(g(x))}
\times \underbrace{\quad 2x\quad}_{=g'(x)} \dx{x}
= \frac{1}{2} \int \underbrace{\sin(y)}_{=f(y)} \dx{y}\bigg|_{y=x^2}\\
&= -\frac{1}{2}\, \cos(y) \bigg|_{y=x^2} + C = -\frac{\cos(x^2)}{2} +C \; .
\end{align*}

\subQ{c} Si $f(y) = (y-5)/\sqrt{y}$ et $y = g(x) = x^2+5$,
alors $x^2= y-5$ et $\dx{y} = g'(x) \dx{x} = 2x \dx{x}$.  Ainsi,
\begin{align*}
\int \frac{x^3}{\sqrt{x^2+5}} \dx{x} &=
\frac{1}{2} \int \underbrace{\frac{x^2}{\sqrt{x^2+5}}}_{=f(g(x))}
\times \underbrace{\quad 2x\quad}_{=g'(x)} \dx{x}
= \frac{1}{2} \int \underbrace{\frac{y-5}{\sqrt{y}}}_{=f(y)}
\dx{y} \bigg|_{y=x^2+5} \\
&= \frac{1}{2} \int \left( y^{1/2} -5 y^{-1/2} \right)\dx{y} \bigg|_{y=x^2+5}
= \frac{1}{2}\,\left( \frac{2}{3}\,y^{3/2} - 10 \, y^{1/2} \right)
\bigg|_{y=x^2+5} \\
&= \frac{1}{3} (x^2+5)^{3/2} - 5 (x^2+5)^{1/2} + C \; .
\end{align*}
\end{egg}

Par tradition et afin de simplifier la notation, nous n'incluons
généralement pas les références à $f(y)$ et $g(x)$ dans la règle de
substitution comme nous l'avons fait dans l'exemple précédent.  C'est
ce que nous allons faire à partir de maintenant.

\begin{egg}
Évaluons les intégrales indéfinies suivantes.
\begin{center}
\begin{tabular}{*{1}{l@{\hspace{0.5em}}l@{\hspace{2.5em}}}l@{\hspace{0.5em}}l}
\subQ{a} & $\displaystyle \int \frac{\cos^3(x)}{\sqrt{\sin(x)}} \dx{x}$ &
\subQ{b} & $\displaystyle \int \frac{1}{1+x^{1/3}} \dx{x}$ \\[1em]
\subQ{c} & $\displaystyle \int \frac{x}{\sqrt{16 - 9x^4}} \dx{x}$ &
\subQ{d} & $\displaystyle \int \frac{1}{x^2+2x+10} \dx{x}$
\end{tabular}
\end{center}

\subQ{a} Si $y=\sin(x)$, alors $\dx{y} = \cos(x) \dx{x}$ et
\begin{align*}
\int \frac{\cos^3(x)}{\sqrt{\sin(x)}} \dx{x}
&= \int \frac{\cos^2(x)}{\sqrt{\sin(x)}}\, \cos(x) \dx{x}
= \int \frac{1-\sin^2(x)}{\sqrt{\sin(x)}}\, \cos(x) \dx{x} \\
&= \int \frac{1-y^2}{\sqrt{y}} \dx{y}\bigg|_{y=\sin(x)}
= \int \left( y^{-1/2} - y^{3/2} \right) \dx{y}\bigg|_{y=\sin(x)} \\
&= \left( 2y^{1/2} - \frac{2}{5} y^{5/2} \right)\bigg|_{y=\sin(x)} +C
= 2 \left(\sin(x)\right)^{1/2} - \frac{2}{5} \left(\sin(x)\right)^{5/2}
+C \; .
\end{align*}

\subQ{b}  Posons $y = 1 + x^{1/3}$.  Puisque
\[
\dfdx{\left(1 + x^{1/3}\right)}{x} = \frac{1}{3} x^{-2/3}
= \frac{1}{3x^{2/3}} \; ,
\]
nous obtenons
\[
\dx{y} = \frac{1}{3x^{2/3}} \dx{x} \; .
\]
De plus,
\[
y = 1 + x^{1/3} \Rightarrow x^{1/3} = y-1 \Rightarrow x^{2/3} =(y-1)^2 \ .
\]
Ainsi,
\begin{align*}
\int \frac{1}{1+x^{1/3}} \dx{x}
&=\int \frac{3 x^{2/3}}{1+x^{1/3}} \left( \frac{1}{3x^{2/3}} \right) \dx{x}
= \int \frac{3 (y-1)^2}{y} \dx{y} \bigg|_{y=1+x^{1/3}} \\
&= 3 \int \left( y - 2 + \frac{1}{y} \right) \dx{y} \bigg|_{y=1+x^{1/3}}
= 3 \left( \frac{y^2}{2} - 2y + \ln|y|\right) \bigg|_{y=1+x^{1/3}} +C \\
&= \frac{3}{2}(1+x^{1/3})^2 - 6(1+x^{1/3}) + 3 \ln|1+x^{1/3}| + C \; .
\end{align*}

\subQ{c} Puisque
\[
\int \frac{x}{\sqrt{16 - 9x^4}} \dx{x} = 
\frac{1}{4} \int \frac{x}{\sqrt{1 - \left(\frac{3x^2}{4}\right)^2}}
\dx{x} \; ,
\]
nous pouvons espérer qu'une bonne substitution va transformer cette
intégrale en une intégrale de la forme
\[
\int \frac{1}{\sqrt{1-y^2}} \dx{y} = \arcsin(y)+C \; .
\]
Si $\displaystyle y = \frac{3x^2}{4}$, alors
$\displaystyle \dx{y} = \frac{3x}{2} \dx{x}$ et
\begin{align*}
\int \frac{x}{\sqrt{16 - 9x^4}} \dx{x} &= 
\frac{1}{4} \int \frac{x}{\sqrt{1 - \left(\frac{3x^2}{4}\right)^2}} \dx{x}
= \frac{1}{6} \int \frac{1}{\sqrt{1 - \left(\frac{3x^2}{4}\right)^2}}
\left( \frac{3x}{2}\right) \dx{x} \\
&= \frac{1}{6} \int \frac{1}{\sqrt{1-y^2}} \dx{y} \bigg|_{y=3x^2/4}
= \frac{1}{6} \arcsin(y)\bigg|_{y=3x^2/4} + C \\
&= \frac{1}{6} \arcsin\left(\frac{3x^2}{4} \right) + C \; .
\end{align*}

\subQ{d} Remarquons que $x^2+2x+10$ n'a pas de racines réelles
(lorsque le dénominateur sera un polynôme qui peut être factorisé, nous
ferons appel aux fractions partielles que nous verrons prochainement).  En
complétant le carré, nous obtenons
\[
x^2+2x+10 = (x+1)^2 + 9 = (x+1)^2 +3^3 \; .
\]
Il faut donc évaluer l'intégrale
\[
\int \frac{1}{x^2+2x+10} \dx{x} = \int \frac{1}{(x+1)^2+3^2} \dx{x}
= \frac{1}{3^2} \, \int \frac{1}{\left(\frac{x+1}{3}\right)^2+1} \dx{x} \; .
\]
Nous pouvons espérer qu'une bonne substitution va transformer cette
intégrale en une intégrale de la forme
\[
\int \frac{1}{1+y^2} \dx{y} = \arctan(y)+C \; .
\]
Pour ce faire, posons $\displaystyle y = \frac{x+1}{3}$.  Ainsi,
$\displaystyle \dx{y} = \frac{1}{3} \dx{x}$ et
\begin{align*}
\int \frac{1}{(x+1)^2+3^2} \dx{x} &= \frac{1}{3}
\int \frac{1}{\left(\frac{x+1}{3}\right)^2 + 1} \times \frac{1}{3} \dx{x}
= \frac{1}{3} \int \frac{1}{y^2+1} \dx{y} \bigg|_{y=(x+1)/3} \\
&= \frac{1}{3} \arctan(y) \bigg|_{y=(x+1)/3} + C
= \frac{1}{3} \arctan\left(\frac{x+1}{3}\right) + C \; .
\end{align*}
\label{EGintegration}
\end{egg}

À l'occasion, il est préférable d'utiliser la règle de substitution
dans le sens inverse.  Nous utilisons $x = g(y)$ pour obtenir
\[
\int f(x) \dx{x}  = \int f(g(y))g'(y) \dx{y}
\]
Cela peut sembler contradictoire car nous avons l'impression que
l'intégrande à droite, $f(g(y))g'(y)$, sera plus compliquée que
l'intégrande à gauche, $f(x)$.  L'exemple qui suit nous prouve le
contraire.

\begin{egg}
Évaluons l'intégrale indéfinie
$\displaystyle \int \frac{\sqrt{x}}{\sqrt{x} - \sqrt[3]{x}} \dx{x}$.

Si les exposants de $x$ était des entiers, l'intégrale serait
probablement plus simple.  Nous aurions seulement le quotient de deux
polynômes.  Si nous multiplions le numérateur et le dénominateur de
l'intégrande par $\sqrt{x}$, nous obtenons
\[
\int \frac{\sqrt{x}}{\sqrt{x} - \sqrt[3]{x}} \dx{x}
= \int \frac{x^{1/2}}{x^{1/2} - x^{1/3}} \dx{x}
= \int \frac{x}{x - x^{5/6}} \dx{x} \ .
\]
Posons $x= u^6$ dans le but d'éliminer les racines sixièmes.  Nous avons
$\displaystyle \dx{x} = 6u^5 \dx{u}$ et
\[
\int \frac{x}{x - x^{5/6}} \dx{x}
= \int \left(\frac{u^6}{u^6 - u^5}\right) 6u^5 \dx{u}
= \int \frac{6u^{6}}{u - 1}\dx{u} \ .
\]
Un longue division donne
\[
\frac{u^6}{u-1} = u^5 + u^4 + u^3 + u^2 + u + 1 + \frac{1}{u-1} \ .
\]
Donc
\begin{align*}
\int \frac{6u^{6}}{u - 1}\dx{u} &=
\int 6\left(u^5 + u^4 + u^3 + u^2 + u + 1 + \frac{1}{u-1}\right) \dx{u} \\
&= 6\left(\frac{u^6}{6} + \frac{u^5}{5} + \frac{u^4}{4} + \frac{u^3}{3} +
\frac{u^2}{2} + u + \ln|u-1|\right) + C \\
&= u^6 + \frac{6u^5}{5} + \frac{3u^4}{2} + 2u^3 +
3u^2 + 6u + 6\ln|u-1| + C \ .
\end{align*}
Finalement, puisque $u=x^{1/6}$, nous obtenons
\begin{align*}
\int \frac{\sqrt{x}}{\sqrt{x} - \sqrt[3]{x}} \dx{x}
&= \int \frac{6u^{6}}{u - 1}\dx{u} \bigg|_{u=x^{1/6}} \\
&= \left( u^6 + \frac{6u^5}{5} + \frac{3u^4}{2} + 2u^3 +
3u^2 + 6u + 6\ln|u-1| + C\right)\bigg|_{u=x^{1/6}} \\
&= x + \frac{6x^{5/6}}{5} + \frac{3u^{2/3}}{2} + 2x^{1/2} +
3x^{1/3} + 6x^{1/6} + 6\ln|x^{1/6}-1| + C \ .
\end{align*}
\end{egg}

Pour évaluer l'intégrale de fonctions trigonométriques, nous avons souvent
recours aux identités trigonométriques.  Les trois plus importantes
identités sont:
\[
\sin^2(\alpha) + \cos^2(\alpha) = 1 \ ,
\ \cos^2(\alpha) = \frac{1}{2}(1 + \cos(2\alpha)) \ \text{ et } 
\ \sin^2(\alpha) = \frac{1}{2}(1 - \cos(2\alpha))\; .
\]
Les deux dernières formules sont connues sous le nom de
{\bfseries formules de l'angle double}\index{Formules de l'angle double}.

Les formules d'addition suivantes sont fréquemment utilisées pour
calculer des intégrales en mécanique.
\begin{align*}
\sin(\alpha + \beta) &=
\sin(\alpha)\cos(\beta) + \cos(\alpha)\sin(\beta) \;,\\
\cos(\alpha + \beta) &=
\cos(\alpha)\cos(\beta) - \sin(\alpha)\sin(\beta) \;,\\
\sin(\alpha)\sin(\beta) &=
\frac{1}{2}(\cos(\alpha-\beta)-\cos(\alpha+\beta))\;, \\
\cos(\alpha)\cos(\beta) &=
\frac{1}{2}(\cos(\alpha-\beta)+\cos(\alpha+\beta))
\intertext{et}
\sin(\alpha)\cos(\beta) &=
\frac{1}{2}(\sin(\alpha-\beta)+\sin(\alpha+\beta)) \; .
\end{align*}
Les trois dernières formules sont déduites des deux formules qui les
précèdent.

Finalement, si l'intégrande contient les fonctions trigonométriques
$\tan$, $\cot$, $\sec$ et $\csc$, il est généralement préférable de
réécrire ces fonctions en termes de $\sin$ et $\cos$, et de simplifier
l'intégrande.  Il y a cependant quelques exceptions où les identités
suivantes peuvent être utiles.
\[
\tan^2(\alpha) + 1 = \sec^2(\alpha) \quad \text{et} \quad
\cot^2(\alpha) + 1 = \csc^2(\alpha) \;.
\]

\begin{egg}
Évaluons les intégrales indéfinies suivantes.
\begin{center}
\begin{tabular}{*{2}{l@{\hspace{0.5em}}l@{\hspace{2em}}}l@{\hspace{0.5em}}l}
\subQ{a} & $\displaystyle \int \tan(x) \dx{x}$ &
\subQ{b} & $\displaystyle \int \sin^2(x)\dx{x}$ &
\subQ{c} & $\displaystyle \int \sin^3(x)\dx{x}$ \\[1em]
\subQ{d} & $\displaystyle \int \sin^3(x)\cos^2(x) \dx{x}$ &
\subQ{e} & $\displaystyle \int \sec(x) \dx{x}$ &
\subQ{f} & $\displaystyle \int \csc(x) \dx{x}$
\end{tabular}
\end{center}

\subQ{a}  Par définition de la tangente, nous avons
\[
\int \tan(x) \dx{x} = \int \frac{\sin(x)}{\cos(x)} \dx{x} \; .
\]
Si $y=\cos(x)$, alors $\dx{y} = -\sin(x) \dx{x}$ et
\begin{align*}
\int \tan(x) \dx{x} &= -\int \frac{1}{\cos(x)} (-\sin(x)) \dx{x}
= -\int \frac{1}{y} \dx{y} \bigg|_{y=\cos(x)} \\
&= -\ln|y|\,\bigg|_{y=\cos(x)} + C = -\ln|\cos(x)| + C \; .
\end{align*}

\subQ{b} Utilisons la formule de l'angle double pour obtenir
\[
\int \sin^2(x) \dx{x} = \frac{1}{2} \int (1-\cos(2x)) \dx{x} \; .
\]
Si $y=2x$, alors $\dx{y} = 2\,\dx{x}$ et
\begin{align*}
\int \sin^2(x) \dx{x} &= \frac{1}{2} \int (1-\cos(2x))  \dx{x}
= \frac{1}{4} \int (1-\cos(2x)) \times 2  \dx{x} \\
&= \frac{1}{4} \int (1-\cos(y))  \dx{y} \bigg|_{y=2x}
= \frac{1}{4} \left( y - \sin(y) \right) \bigg|_{y=2x} +C \\
&= \frac{1}{4} \left( 2x - \sin(2x) \right) +C
= \frac{x}{2} - \frac{1}{4}\sin(2x) +C \; .
\end{align*}

\subQ{c} Utilisons la relation $\sin^2(\theta) + \cos^2(\theta)=1$
pour écrire
\[
\int \sin^3(x)\dx{x} = \int \left(1-\cos^2(x)\right) \sin(x)\dx{x} \; .
\]
Si $y = \cos(x)$, alors $\dx{y} = -\sin(x)\dx{x}$ et
\begin{align*}
\int \left(1-\cos^2(x)\right) \sin(x)\dx{x} \dx{x} &=
 - \int \left(1- \cos^2(x)\right) \; ( -\sin(x)) \dx{x}
= - \int (1-y^2) \dx{y}\bigg|_{y=\cos(x)} \\
&= - \left( y - \frac{y^3}{3}\right)\bigg|_{y=\cos(x)} + C
= - \left(\cos(x) - \frac{1}{3} \cos^3(x)\right) + C\\
&= -\cos(x) + \frac{1}{3} \cos^3(x) + C \; .
\end{align*}

\subQ{d} Utilisons la relation
$\sin^2(\theta) + \cos^2(\theta)=1$ pour obtenir
\[
\int \sin^3(x) \cos^2(x) \dx{x}
= \int (1-\cos^2(x)) \cos^2(x) \sin(x) \dx{x} \; .
\]
Si $y=\cos(x)$, alors $\dx{y} = - \sin(x)\dx{x}$ et
\begin{align*}
\int (1-\cos^2(x)) \cos^2(x) \sin(x) \dx{x}
&= -\int (1-\cos^2(x)) \cos^2(x) (-\sin(x)) \dx{x} \\
&= -\int (1-y^2)y^2  \dx{y}\bigg|_{y=\cos(x)}
= -\int (y^2-y^4) \dx{y}\bigg|_{y=\cos(x)} \\
&= -\frac{y^3}{3} + \frac{y^5}{5}\bigg|_{y=\cos(x)} + C
= -\frac{1}{3} \cos^3(x) + \frac{1}{5} \cos^5(x) + C \; .
\end{align*}

\subQ{e} Il faut utiliser un petit truc.  Nous avons
\[
\int \sec(x) \dx{x} =
= \int \sec(x) \left(\frac{\sec(x)+\tan(x)}{\sec(x)+\tan(x)}\right) \dx{x}
= \int \frac{\sec^2(x)+\sec(x)\tan(x)}{\sec(x)+\tan(x)} \dx{x} \; .
\]
Si $y=\sec(x)+\tan(x)$, alors
\[
\dx{y} = \left( \sec(x)\tan(x) + \sec^2(x) \right) \dx{x} \; .
\]
Ainsi,
\begin{align*}
\int \sec(x) \dx{x} &
= \int \frac{1}{\sec(x)+\tan(x)}\left(\sec^2(x)+\sec(x)\tan(x)\right)\dx{x}
= \int \frac{1}{y} \dx{y} \bigg|_{y=\sec(x)+\tan(x)} \\
&= \ln|y|\bigg|_{y=\sec(x)+\tan(x)} + C
= \ln|\sec(x)+\tan(x)| + C  \; .
\end{align*}

\subQ{f} Il faut utiliser un petit truc semblable à celui utilisé en
(e). Nous avons
\[
\int \csc(x) \dx{x}
= \int \csc(x) \left(\frac{\csc(x)+\cot(x)}{\csc(x)+\cot(x)}\right) \dx{x}
= \int \frac{\csc^2(x)+\csc(x)\cot(x)}{\csc(x)+\cot(x)} \dx{x} \; .
\]
Si $y=\csc(x)+\cot(x)$, alors
\[
\dx{y} = \left( -\csc(x)\cot(x) - \cot^2(x) \right) \dx{x} \; .
\]
Ainsi,
\begin{align*}
\int \csc(x) \dx{x} &
= \int \frac{\csc^2(x)+\csc(x)\cot(x)}{\csc(x)+\cot(x)} \dx{x} \\
&= -\int \frac{1}{\csc(x)+\cot(x)}
\left(-\csc^2(x)-\csc(x)\cot(x)\right) \dx{x}
= -\int \frac{1}{y} \dx{y} \bigg|_{y=\csc(x)+\cot(x)} \\
&= -\ln|y|\bigg|_{y=\csc(x)+\cot(x)}
= -\ln|\csc(x)+\cot(x)| + C \; .
\end{align*}
\label{EGsubA}
\end{egg}

\begin{egg}[\eng]
Évaluons les intégrales indéfinies suivantes.
\begin{center}
\begin{tabular}{*{1}{l@{\hspace{0.5em}}l@{\hspace{7em}}}l@{\hspace{0.5em}}l}
\subQ{a} & $\displaystyle \int \sin(4x)\cos(3x) \dx{x}$ &
\subQ{b} &
$\displaystyle \int \sin\left(x+\frac{\pi}{6}\right)\cos(x) \dx{x}$ 
\end{tabular}
\end{center}

\subQ{a} Utilisons l'identité
\[
\sin(\alpha)\cos(\beta) = \frac{1}{2}(\sin(\alpha-\beta)+\sin(\alpha+\beta))
\]
avec $\alpha = 4x$ et $\beta = 3x$ pour obtenir
\[
\int \sin(4x)\cos(3x) \dx{x} = \frac{1}{2} \int \sin(x) \dx{x} +
\frac{1}{2} \int \sin(7x) \dx{x} \; .
\]
La première intégrale est
\[
\int \sin(x) \dx{x} = -\cos(x) + C_1 \; .
\]
Pour la deuxième intégrale, nous utilisons la substitution $y=7x$.  Donc
$\dx{y} = 7 \dx{x}$ et
\begin{align*}
\int \sin(7x) \dx{x} &= \frac{1}{7} \int \sin(7x)\times 7 \dx{x}
= \frac{1}{7} \int \sin(y) \dx{y} \bigg|_{y=7x} \\
&= - \frac{1}{7} \cos(y) \bigg|_{y=7x} + C_2 = - \frac{1}{7} \cos(7x) + C_2 \; .
\end{align*}
Ainsi,
\[
\int \sin(4x)\cos(3x) \dx{x} = - \frac{1}{2} \cos(x) - \frac{1}{14} \cos(7x)
+ C
\]
où $C=(C_2-C_1)/2$.

\subQ{b} Utilisons une des formules d'addition pour obtenir
\[
\sin\left(x+\frac{\pi}{6}\right) = \sin(x)\cos\left(\frac{\pi}{6}\right)
+ \cos(x)\sin\left(\frac{\pi}{6}\right)
= \frac{\sqrt{3}}{2}\,\sin(x) + \frac{1}{2}\cos(x) \ .
\]
Si nous substituons cette expression dans l'intégrale
\[
\int \sin\left(x+\frac{\pi}{6}\right)\cos(x) \dx{x} \ ,
\]
nous obtenons
\[
\int \sin\left(x+\frac{\pi}{6}\right)\cos(x) \dx{x}
= \frac{\sqrt{3}}{2}\,\int \sin(x)\cos(x) \dx{x} +
\frac{1}{2}\, \int \cos^2(x) \dx{x} \; .
\]
Si nous substituons
\[
\sin(x)\cos(x) = \frac{1}{2}\left(\sin(x-x)+\sin(x+x)\right)
= \frac{1}{2} \sin(2x)
\]
et
\[
  \cos^2(x) = \frac{1}{2}(1 + \cos(2x))
\]
dans l'intégrale précédente, nous obtenons
\begin{align*}
\int \sin\left(x+\frac{\pi}{6}\right)\cos(x) \dx{x}
&= \frac{\sqrt{3}}{4}\, \int \sin(2x) \dx{x} +
 \frac{1}{4}\,\int \left(1+\cos(2x)\right) \dx{x} \\
&= \frac{\sqrt{3}}{4}\,\int \sin(2x) \dx{x} +
\frac{1}{4} \int \dx{x} + \frac{1}{4} \int \cos(2x) \dx{x} \; . 
\end{align*}
Pour évaluer la première et la troisième intégrale dans l'expression
précédente, utilisons la substitution $y=2x$.
Donc $\dx{y} = 2\dx{x}$ et
\begin{align*}
\int \sin\left(x+\frac{\pi}{6}\right)\cos(x) \dx{x}
&= \frac{\sqrt{3}}{8}\,\int \sin(2x)\times 2 \dx{x} +
\frac{1}{4} \int \dx{x} + \frac{1}{8} \int \cos(2x) \times 2\dx{x} \\
  &= \frac{\sqrt{3}}{8}\,\int \sin(y) \dx{y}\bigg|_{y=2x} +
\frac{1}{4} \int \dx{x} + \frac{1}{8} \int \cos(y) \dx{y}\bigg|_{y=2x} \\
&= -\frac{\sqrt{3}}{8} \cos(y)\bigg|_{y=2x} +
\frac{x}{4} + \frac{1}{8} \sin(y)\bigg|_{y=2x} +C \\
&=  -\frac{\sqrt{3}}{8} \cos(2x) + \frac{x}{4} + \frac{1}{8} \sin(2x) +C \; .
\end{align*}

Nous aurions pu utiliser la substitution $y=\sin(x)$ pour
évaluer l'intégrale\\ $\displaystyle \int \sin(x) \cos(x) \dx{x}$
ci-dessus.  En effet, avec cette substitution, nous obtenons
$\dx{y} = \cos(x) \dx{x}$ 
et
\[
  \int \sin(x) \cos(x) \dx{x} = \int y \dx{y} \bigg|_{y=\sin(x)}
  = \frac{y^2}{2} \bigg|_{y=\sin(x)} + C
  = \frac{\sin^2(x)}{2} + C \; .
\]
Cette solution est équivalente à celle que nous avons trouvée
précédemment.  L'équivalence des deux solutions se démontre à l'aide 
des identités trigonométriques.
\end{egg}

\subsection{Intégration par parties}

Alors que la méthode de substitution nous a permis d'évaluer
l'intégrale indéfinie de certaines fonctions composées, la méthode
d'intégration par parties nous permettra d'évaluer l'intégrale
indéfinie du produit de deux fonctions comme $x e^x$, $x\sin(x)$,
etc.

Si $f$ et $g$ sont deux fonctions différentiables, alors
\[
\dfdx{\left(f(x)g(x)\right)}{x} = f'(x)\,g(x) + f(x)\,g'(x) \; .
\]
Ainsi,
\[
f(x)\,g'(x) = \dfdx{\left(f(x)g(x)\right)}{x} - f'(x)\,g(x) \; .
\]
Si nous utilisons la linéarité de l'intégrale indéfinie, nous obtenons
\begin{align*}
\int f(x)\,g'(x) \dx{x} &= 
\int \dfdx{\left(f(x)g(x)\right)}{x} \dx{x} - \int f'(x)\,g(x) \dx{x} \\
& = f(x)g(x) - \int f'(x)\,g(x) \dx{x}
\end{align*}
où nous avons utilisé le fait que $f(x)g(x)$ est une primitive de
$\displaystyle \dfdx{\left(f(x)g(x)\right)}{x}$.

Nous obtenons donc une primitive de $f(x)g'(x)$ en soustrayant de
$f(x)g(x)$ une primitive de $f'(x)\,g(x)$.  En d'autre mots,

\begin{theorem} \index{Intégration par parties}
Si $f$ et $g$ sont deux fonctions différentiables, alors
\[
\int f(x)\,g'(x)\,\dx{x} = f(x)g(x) - \int g(x)\,f'(x) \dx{x} \; .
\]
Cette formule est connue sous le nom
{\bfseries d'intégration par parties}.
\end{theorem}

\begin{egg}
Évaluons l'intégrale indéfinie $\displaystyle \int x e^x\dx{x}$.

Nous avons $x e^x = f(x) g'(x)$ pour $f(x)=x$ et $g'(x) = e^x$.  Donc
$g(x) = e^x$, $f'(x) = 1$ et
\[
\int x e^x\dx{x} = \int f(x)\,g'(x)\,\dx{x}
= f(x)g(x) - \int g(x)\,f'(x) \dx{x} 
= x e^x - \int e^x\dx{x} = x e^x - e^x + C
\]
car $\int e^x \dx{x} = e^x + C$.
\label{xexpx}
\end{egg}

\begin{egg}
Évaluons les intégrales indéfinies suivantes.
\begin{center}
\begin{tabular}{*{1}{l@{\hspace{0.5em}}l@{\hspace{5em}}}l@{\hspace{0.5em}}l}
\subQ{a} & $\displaystyle \int x^2 \cos(2x)\dx{x}$ &
\subQ{b} & $\displaystyle \int \sqrt{t} \ln(t) \dx{t}$ \\[1em]
\subQ{c} & $\displaystyle \int x^3 \ln(2x)\dx{x}$ &
\subQ{d} & $\displaystyle \int \ln(x)\dx{x}$
\end{tabular}
\end{center}

\subQ{a} Nous avons $x^2 \cos(2x) = f(x) g'(x)$ pour $f(x)=x^2$ et
$g'(x)= \cos(2x)$.  Donc $f'(x) = 2x$, $g(x)=\sin(2x)/2$ et
\begin{align*}
\int x^2 \cos(2x)\dx{x} &= \int f(x)\,g'(x)\,\dx{x}
= f(x)g(x) - \int g(x)\,f'(x) \dx{x} \\
&= \frac{x^2}{2}\,\sin(2x) - \int x\sin(2x) \dx{x} \; .
\end{align*}
Faisons appel pour une seconde fois à la méthode d'intégration par
parties pour évaluer $\int x\sin(2x)  \dx{x}$.  Nous avons
$x\sin(2x) = f(x)g'(x)$ pour $f(x)=x$ et $g'(x) = \sin(2x)$.  Donc
$f'(x) = 1$, $g(x) = -\cos(2x)/2$ et
\begin{align*}
\int x\sin(2x) \dx{x} &= \int f(x)\,g'(x)\,\dx{x}
= f(x)g(x) - \int g(x)\,f'(x) \dx{x} \\
&= -\frac{x}{2}\,\cos(2x) + \frac{1}{2}\int \cos(2x) \dx{x}
= -\frac{x}{2}\,\cos(2x) + \frac{1}{4} \, \sin(2x) + C \; .
\end{align*}
Finalement,
\begin{align*}
\int x^2 \cos(2x)\dx{x} &= \frac{x^2}{2}\,\sin(2x)
-\left( -\frac{x}{2}\,\cos(2x) + \frac{1}{4} \, \sin(2x) + C \right) \\
&= \frac{x^2}{2}\,\sin(2x) + \frac{x}{2}\,\cos(2x)
- \frac{1}{4} \, \sin(2x) - C \; .
\end{align*}
Remarquons que la méthode d'intégration par substitution a été
utilisée à deux reprises pour éliminer le polynôme de degré deux $x^2$ de
l'intégrande.

\subQ{b} Nous avons $t^{1/2} \ln(t) = f(t)\,g'(t)$ pour $f(t)=\ln(t)$ et
$g'(t)= t^{1/2}$.  Donc $\displaystyle f'(t) = \frac{1}{t}$,
$\displaystyle g(t) = \frac{2}{3} t^{3/2}$ et
\begin{align*}
\int t^{1/2} \ln(t)  \dx{t} &= \int f(t)\,g'(t)\,\dx{t}
= f(t)g(t) - \int g(t)\,f'(t) \dx{t} \\
&= \frac{2}{3} t^{3/2} \ln(t) - \frac{2}{3} \int t^{1/2} \dx{t}
= \frac{2}{3} t^{3/2} \ln(t) - \frac{4}{9} t^{3/2} + C \; .
\end{align*}

\subQ{c} Nous avons $x^3 \ln(2x) = f(x)\,g'(x)$ pour $f(x)=\ln(2x)$ et
$g'(x)=x^3$.  Donc $f'(x) = 1/x$, $g(x) = x^4/4$ et
\begin{align*}
\int x^3 \ln(2x)  \dx{x} &= \int f(x)\,g'(x)\,\dx{x}
= f(x)g(x) - \int g(x)\,f'(x) \dx{x} \\
&= \frac{x^4}{4}\,\ln(2x) - \frac{1}{4}\int x^3 \dx{x}
= \frac{x^4}{4}\,\ln(2x) - \frac{1}{16} \, x^4 + C \; .
\end{align*}

\subQ{d} Nous avons $\ln(x) = f(x) g'(x)$ pour $f(x)=\ln(x)$ et $g'(x) = 1$.
Donc $g(x) = x$, $f'(x) = 1/x$ et
\begin{align*}
\int \ln(x)\dx{x} &= \int f(x)\,g'(x)\,\dx{x}
= f(x)g(x) - \int g(x)\,f'(x) \dx{x} \\
&= x\,\ln(x) - \int 1\dx{x} = x \, \ln(x) - x + C \; .
\end{align*}
\end{egg}

Nous pouvons déduire des exemples précédents quelques règles pour
l'intégration par parties.
\begin{enumerate}
\item Si l'intégrande est de la forme $p(x) e^{\alpha x}$ où $p(x)$
est un polynôme, il faut choisir $f(x) = p(x)$ et
$g'(x) = e^{\alpha x}$.
\item Si l'intégrande est de la forme $p(x) \sin(\alpha x)$ ou
$p(x) \cos(\alpha x)$ où $p(x)$ est un polynôme, il faut choisir
$f(x) = p(x)$ et $g'(x) = \sin(\alpha x)$ ou $\cos(\alpha x)$ selon le
cas.
\item Si l'intégrande est de la forme $p(x) \ln(\alpha x)$ où $p(x)$
est une somme d'expressions de la forme $x^r$ avec $r$ un nombre
rationnel, il faut choisir $f(x) = \ln(x)$ et $g'(x) = p(x)$.
\end{enumerate}

\begin{egg}[\eng]
Évaluons les intégrales indéfinies suivantes.
\begin{center}
\begin{tabular}{*{2}{l@{\hspace{0.5em}}l@{\hspace{2.5em}}}l@{\hspace{0.5em}}l}
\subQ{a} & $\displaystyle \int x\,5^x \dx{x}$ &
\subQ{b} & $\displaystyle \int \arcsin(x) \dx{x}$ &
\subQ{c} & $\displaystyle \int \arctan(x) \dx{x}$
\end{tabular}
\end{center}

\subQ{a}  Nous avons $x 5^x = f(x)\,g'(x)$ pour $f(x)=x$ et
$g'(x)= 5^x$.  Donc $f'(x) = 1$, $g(x) = 5^x/\ln(5)$ et
\begin{align*}
\int x 5^x \dx{x} &= \int f(x)\,g'(x)\,\dx{x}
= f(x)g(x) - \int g(x)\,f'(x) \dx{x} \\
&= \frac{x 5^x}{\ln(5)} - \frac{1}{\ln(5)} \int 5^x \dx{x}
= \frac{x 5^x}{\ln(5)} - \frac{1}{(\ln(5))^2} 5^x + C
\end{align*}
Il ne faut pas oublier que $\displaystyle \dfdx{5^x}{x} = 5^x \ln(5)$.

\subQ{b} Nous avons $\arcsin(x) = f(x) g'(x)$ pour $f(x)=\arcsin(x)$ et
$g'(x) = 1$. Donc $g(x) = x$,
$\displaystyle f'(x) = \frac{1}{\sqrt{1-x^2}}$ et
\begin{align*}
\int \arcsin(x)\dx{x} &= \int f(x)\,g'(x)\,\dx{x}
= f(x)g(x) - \int g(x)\,f'(x) \dx{x} \\
&= x\,\arcsin(x) - \int \frac{x}{\sqrt{1-x^2}} \dx{x} \; .
\end{align*}
Pour évaluer cette dernière intégrale, nous utilisons la substitution
$y= 1-x^2$.  Ainsi, $\dx{y} = -2 x \dx{x}$ et
\begin{align*}
\int \frac{x}{\sqrt{1-x^2}} \dx{x}
&= - \frac{1}{2} \int \left(1-x^2\right)^{-1/2} \times (-2 x) \dx{x}
= -\frac{1}{2} \int y^{-1/2} \dx{y} \bigg|_{y=1-x^2} \\
&= - y^{1/2} \bigg|_{y=1-x^2} + C
= -\sqrt{1-x^2} + C \; .
\end{align*}
Donc
\[
\int \arcsin(x)\dx{x} = x\,\arcsin(x) + \sqrt{1-x^2} - C \; .
\]

\subQ{c} Nous avons $\arctan(x) = f(x) g'(x)$ pour $f(x)=\arctan(x)$ et
$g'(x) = 1$. Donc $g(x) = x$, $\displaystyle f'(x) = \frac{1}{1+x^2}$
et
\begin{align*}
\int \arctan(x)\dx{x} &= \int f(x)\,g'(x)\,\dx{x}
= f(x)g(x) - \int g(x)\,f'(x) \dx{x} \\
&= x\,\arctan(x) - \int \frac{x}{1+x^2} \dx{x} \; .
\end{align*}
Pour évaluer cette dernière intégrale, nous utilisons la substitution
$y= 1+x^2$.  Ainsi, $\dx{y} = 2 x \dx{x}$ et
\begin{align*}
\int \frac{x}{1+x^2} \dx{x}
&= \frac{1}{2} \int \frac{1}{1+x^2}\, \times 2 x \dx{x}
= \frac{1}{2} \int \frac{1}{y} \dx{y} \bigg|_{y=1+x^2} \\
&= \frac{1}{2} \ln(y) \bigg|_{y=1+x^2} + C
= \frac{1}{2} \ln(1+x^2) + C \; .
\end{align*}
Donc
\[
\int \arctan(x)\dx{x} = x\,\arctan(x) - \frac{1}{2}\ln(1+x^2) - C \; .
\]
\end{egg}

\begin{egg}
Évaluons l'intégrale indéfinie $\displaystyle \int e^{\sqrt{x}} \dx{x}$.

C'est un autre exemple où nous devons utiliser la règle de
substitution dans le sens inverse comme nous l'avons expliqué à la
section précédente.

Posons $x = t^2$ dans le but d'éliminer $\sqrt{x}$.  Alors
$\dx{x} = 2t \dx{t}$ et
\[
\int e^{\sqrt{x}} \dx{x} = 2 \int t e^t \dx{t} \bigg|_{t=\sqrt{x}}
= 2 \left( t e^t - e^t \right)\bigg|_{t=\sqrt{x}} + C
= 2 \left( \sqrt{x}\, e^{\sqrt{x}} - e^{\sqrt{x}} \right) + C
\]
où nous avons utilisé le résultat de l'exemple~\ref{xexpx}.
\end{egg}

L'exemple suivant démontre une technique pour évaluer les intégrales
dont l'intégrande est le produit de $e^{\alpha x}$ et
$\cos(\beta x)$, ou $e^{\alpha x}$ et $\sin(\beta x)$.

\begin{egg}
Évaluons les intégrales indéfinies suivantes.
\begin{center}
\begin{tabular}{*{1}{l@{\hspace{1em}}l@{\hspace{5em}}}l@{\hspace{1em}}l}
\subQ{a} & $\displaystyle \int e^x\cos(x) \dx{x}$ &
\subQ{b} & $\displaystyle \int \sin(\ln(x)) \dx{x}$ \quad , \quad $x>0$
\end{tabular}
\end{center}

\subQ{a} Posons
\[
I = \int e^x\cos(x) \dx{x} \; .
\]
Nous avons $e^x\cos(x) = f(x)\,g'(x)$ pour $f(x)=e^x$ et
$g'(x)=\cos(x)$.  Donc $f'(x) = e^x$, $g(x) = \sin(x)$ et
\begin{align}
I &= \int e^x\cos(x) \dx{x} = \int f(x)\,g'(x)\,\dx{x}
= f(x)g(x) - \int g(x)\,f'(x) \dx{x} \nonumber \\
&= e^x\sin(x) - \int e^x \sin(x)  \dx{x} \; . \label{twoIPs}
\end{align}
Même si cette expression n'est pas plus simple que l'intégrale du
départ, nous sommes quand même sur la bonne voie.  Utilisons une seconde
fois la méthode d'intégration par parties pour évaluer l'intégrale
\[
\int e^x \sin(x) \dx{x} \; .
\]
Nous avons $e^x\sin(x) = f(x)\,g'(x)$ pour $f(x)=e^x$ et $g'(x)=\sin(x)$.
Donc $f'(x) = e^x$, $g(x) = -\cos(x)$ et
\begin{align*}
\int e^x \sin(x)  \dx{x} &= \int f(x)\,g'(x)\,\dx{x}
= f(x)g(x) - \int g(x)\,f'(x) \dx{x} \\
&= - e^x \cos(x) + \int e^x \cos(x) \dx{x} = -e^x \cos(x) + I \; .
\end{align*}
Si nous substituons cette expression dans (\ref{twoIPs}), nous obtenons
\[
I = e^x\sin(x) - \left( -e^x \cos(x) + I \right)
\]
et, après avoir isolé $I$, nous trouvons que
\[
I = \frac{1}{2}\left(e^x\sin(x) + e^x \cos(x)\right)
\]
est une primitive de $e^x \cos(x)$.  Donc
\[
\int e^x\cos(x) \dx{x}
= \frac{1}{2}\left(e^x\sin(x) + e^x \cos(x)\right) + C \; .
\]

Nous aurions pu utiliser $e^x\cos(x) = f(x)\,g'(x)$ pour $f(x)=\cos(x)$
et $g'(x)=e^x$ lors de la première intégration par parties mais alors
il aurait fallu utiliser $e^x\sin(x) = f(x)\,g'(x)$ pour
$f(x)=\sin(x)$ et $g'(x)=e^x$ lors de la deuxième intégration par
parties.

\subQ{b} Commençons par une substitution.  Posons $x = e^y$ pour
éliminer $\ln(x)$.  Nous avons $\dx{x} = e^y \dx{y}$ et
\[
\int \sin(\ln(x)) \dx{x} = \int \sin(y) \, e^y \dx{y}\bigg|_{y=\ln(x)} \; .
\]
C'est une intégrale du même type que celle que nous avons évalué en (a).

Posons
\[
I = \int e^y\sin(y) \dx{y} \; .
\]
Nous avons $e^y \sin(y) = f(y)\,g'(y)$ pour $g'(y)=e^y$ et $f(y)=\sin(y)$.
Donc $g(y) = e^y$, $f'(y) = \cos(y)$ et
\begin{align}
I &= \int e^y\sin(y) \dx{y} = \int f(x)\,g'(x)\,\dx{x}
= f(x)g(x) - \int g(x)\,f'(x) \dx{x} \nonumber \\
&= e^y\sin(y) - \int e^y \cos(y) \dx{y} \label{twoIPs2}
\end{align}
De plus, pour l'intégrale indéfinie
\[
\int e^y \cos(y) \dx{y} \; ,
\]
nous avons $e^y \cos(y) = f(y)\,g'(y)$ pour $g'(y)=e^y$ et
$f(y)=\cos(y)$.  Donc $g(y) = e^y$, $f'(y) = -\sin(y)$ et
\begin{align*}
\int e^y \cos(y) \dx{y} &= \int f(x)\,g'(x)\,\dx{x}
= f(x)g(x) - \int g(x)\,f'(x) \dx{x} \\
&= e^y \cos(y) + \int e^y \sin(y) \dx{y}
= e^y \cos(y) + I \; .
\end{align*}
Si nous substituons cette expression dans (\ref{twoIPs2}), nous obtenons
\[
I = e^y\sin(y) - \left( e^y \cos(y) + I \right)
\]
et, après avoir isolé $I$, nous trouvons que
\[
I = \frac{1}{2}\left(e^y\sin(y) - e^y \cos(y)\right)
\]
est une primitive de $e^y\sin(y)$.  Ainsi, 
\[
\int e^y\sin(y) \dx{y}
= \frac{1}{2}\left(e^y\sin(y) - e^y \cos(y)\right) + C \; .
\]

Donc
\begin{align*}
\int \sin(\ln(x)) \dx{x} &= \int \sin(y) \, e^y \dx{y}\bigg|_{y=\ln(x)}
= \frac{1}{2}\left(e^y\sin(y) - e^y \cos(y)\right)\bigg|_{y=\ln(x)} + C \\
&= \frac{1}{2}\left(e^{\ln(x)}\sin(\ln(x))-e^{\ln(x)}
  \cos(\ln(x))\right) + C\\[1ex]
&= \frac{x}{2}\left(\sin(\ln(x))- \cos(\ln(x))\right) + C \; .
\end{align*}
\end{egg}

\subsubsection{Méthode de Weiestrass \eng}

Il existe une autre façon d'évaluer des intégrales de la forme
$\displaystyle \int F(\cos(x), \sin(x)) \dx{x}$ où
$F(\cos(x), \sin(x))$ est 
une fonction (rationnelle) de $\sin(x)$ et $\cos(x)$.  Nous pouvons
utiliser la substitution
\begin{equation} \label{weies1}
t = \tan\left(\frac{x}{2}\right) \quad \text{pour} \quad -\pi < x < \pi
\end{equation}
qui est due à Weiestrass.  Cette substitution est équivalente à
$x = 2 \arctan(t)$ pour $t\in \RR$.  Nous avons
\begin{equation} \label{weies2}
\dx{x} = \frac{2}{t^2+1} \dx{t} \; .
\end{equation}
De plus, grâce à la
{\bfseries formule de l'angle double pour la tangente}\index{Formule
  de l'angle double pour la tangente}, nous avons
\[
\sec^2(x) = \tan^2(x) + 1 = \left(\frac{2\tan(x/2)}
{1-\tan^2(x/2)}\right)^2 + 1
= \left(\frac{2t}{1-t^2}\right)^2 + 1 = \left(\frac{1+t^2}{1-t^2}\right)^2
\]
si $t\neq \pm 1$.  Donc $\displaystyle \sec(x) = \frac{1+t^2}{1-t^2}$
donne
\begin{equation} \label{weies3}
\cos(x) = \frac{1-t^2}{1+t^2}  \; .
\end{equation}
Cette formule est aussi valide pour $t = \pm 1$.  Si $t=1$, nous avons
$x = \pi/2$ et ainsi $\cos(x) = 0$.  Les deux côtés de
(\ref{weies3}) sont donc nuls.  De même, si $t= -1$, nous avons $x = -\pi/2$ et
ainsi $\cos(x) = 0$.  Les deux côtés de (\ref{weies3}) sont donc
encore nuls.  La relation
\[
\sin^2(x) = 1 - \cos^2(x) = 1 - \left(\frac{1-t^2}{1+t^2}\right)^2
= \left(\frac{2t}{1+t^2}\right)^2
\]
donne
\begin{equation} \label{weies4}
\sin(x) = \frac{2t}{1+t^2} \; .
\end{equation}

\begin{egg}
Évaluons l'intégrale indéfinie
$\displaystyle \int \frac{1}{2+\sin(x)} \dx{x}$.

Nous utilisons la substitution (\ref{weies1}) avec (\ref{weies2}) et
(\ref{weies4}) pour obtenir
\begin{align*}
\int \frac{1}{2+\sin(x)} \dx{x}
&= \int \frac{1}{2 + \frac{2t}{1+t^2}} \left(\frac{2}{t^2+1}\right) \dx{t}
= \int \frac{1}{t^2+t+1} \dx{t} \\
&= \int \frac{1}{ \left(t+\frac{1}{2}\right)^2 +
\left(\frac{\sqrt{3}}{2}\right)^2} \dx{t}
= \frac{4}{3}\, \int
\frac{1}{\left(\frac{2}{\sqrt{3}}\left(t+\frac{1}{2}\right)\right)^2 + 1}
\dx{t} \; .
\end{align*}
Avec la substitution
$\displaystyle y= \frac{2}{\sqrt{3}}\left(t+\frac{1}{2}\right)$, nous
obtenons
$\displaystyle \dx{y}= \frac{2}{\sqrt{3}} \dx{t}$ et
\begin{align*}
\int \frac{1}{2+\sin(x)} \dx{x} &=
\frac{2}{\sqrt{3}}
\int \frac{1}{\left(\frac{2}{\sqrt{3}}\left(t+\frac{1}{2}\right)\right)^2 + 1}
\left( \frac{2}{\sqrt{3}} \right) \dx{t}
= \frac{2}{\sqrt{3}}
\int \frac{1}{y^2+1} \dx{y}\bigg|_{y=2\left(t+1/2\right)/\sqrt{3}} \\
&= \frac{2}{\sqrt{3}} \arctan(y) \bigg|_{y=2\left(t+1/2\right)/\sqrt{3}} + C
= \frac{2}{\sqrt{3}}
\arctan\left(\frac{2}{\sqrt{3}}\left(t+\frac{1}{2}\right)\right) +C
\; .
\end{align*}
Finalement, puisque $\displaystyle t = \tan\left(\frac{x}{2}\right)$, nous
trouvons
\[
\int \frac{1}{2+\sin(x)} \dx{x}
= \frac{2}{\sqrt{3}}
\arctan\left(\frac{2}{\sqrt{3}}\left(\tan\left(\frac{x}{2}\right)
+\frac{1}{2}\right)\right) +C \; .
\]
\end{egg}

\subsection{Fractions partielles}

La méthode des fractions partielles permet de calculer des
intégrales de la forme
\[
\int \frac{p(x)}{q(x)}  \dx{x}
\]
où $p(x)$ et $q(x)$ sont deux polynômes.  Il faut suivre les
étapes suivantes pour calculer ce genre d'intégrales.

\begin{enumerate}
\item Si le degré de $p(x)$ est plus grand ou égal au degré de $q(x)$,
nous divisons $p(x)$ par $q(x)$ pour obtenir
\[
\frac{p(x)}{q(x)} = s(x) + \frac{r(x)}{q(x)}
\]
où le degré de $r(x)$ est plus petit que le degré de $q(x)$, et $s(x)$
est un polynôme (donc facile à intégrer).
\item Nous exprimons $q(x)$ comme un produit de facteurs irréductibles.
\[
q(x) = A(x-\alpha_1)^{n_1}(x-\alpha_2)^{n_2}
\ldots(x^2 +\beta_1 x +\gamma_1)^{m_1}(x^2 +\beta_2 x +\gamma_2)^{m_2}
\ldots
\]
où $A$, $\alpha_1$, $\alpha_2$, \ldots , $\beta_1$, $\gamma_1$,
$\beta_2$, $\gamma_2$, \ldots\ sont des constantes réelles et $n_1$,
$n_2$, \ldots, $m_1$, $m_1$, \ldots\ sont des entiers positifs.  Nous
assumons que les $\alpha_k$ et les $(\beta_k ,\gamma_k)$ sont distincts.
\item Nous exprimons $r(x)/q(x)$ sous la forme
\begin{align*}
\frac{r(x)}{q(x)} &= \frac{A_1}{(x-\alpha_1)} + 
\frac{A_2}{(x-\alpha_1)^2} + \ldots +
\frac{A_{n_1}}{(x-\alpha_1)^{n_1}} + \ldots \\
&+ \frac{B_1x + C_1}{(x^2+\beta_1 x +\gamma_1)} +
\frac{B_2x + C_2}{(x^2+\beta_1 x +\gamma_1)^2} + \ldots +
\frac{B_{m_1}x + C_{m_1}}{(x^2+\beta_1 x +\gamma_1)^{m_1}} + \ldots
\end{align*}
où $A_1$, $A_2$, \ldots , $B_1$, $C_1$, $B_2$, $C_2$, \ldots\ sont des
constantes.
\item Finalement, nous évaluons
\begin{align*}
\int \frac{p(x)}{q(x)} \dx{x} &= \int s(x) \dx{x} +
\int \frac{r(x)}{q(x)} \dx{x} \\
&= \int s(x) \dx{x} + \int \frac{A_1}{(x-\alpha_1)} \dx{x} + 
\int \frac{A_2}{(x-\alpha_1)^2} \dx{x} + \ldots
+ \int \frac{A_{n_1}}{(x-\alpha_1)^{n_1}} \dx{x}\\
& \qquad + \ldots + \int \frac{B_1x + C_1}{(x^2+\beta_1 x +\gamma_1)} \dx{x} +
\int \frac{B_2x + C_2}{(x^2+\beta_1 x +\gamma_1)^2} \dx{x}\\
& \qquad + \ldots
+ \int \frac{B_{m_1}x + C_{m_1}}{(x^2+\beta_1 x +\gamma_1)^{m_1}} \dx{x} +
\ldots
\end{align*}
\end{enumerate}

\begin{egg}
Évaluons l'intégrale indéfinie
$\displaystyle \int \frac{x^3}{x^2-5x+6}  \dx{x}$.

Étape 1: Comme le numérateur est un polynôme de degré plus grand que
le dénominateur, nous divisons pour obtenir 
\[
\frac{x^3}{x^2-5x+6} = x + 5 + \frac{19x-30}{x^2-5x+6} \; .
\]

Étape 2: Nous avons la factorisation $x^2-5x+6 = (x-2)(x-3)$.

Étape 3: Nous cherchons $A$ et $B$ tels que
\[
\frac{19x-30}{x^2-5x+6} = \frac{A}{x-2} + \frac{B}{x-3} \; .
\]
Si nous exprimons les fractions sur un même dénominateur commun, l'égalité
est satisfaite lorsque nous avons le même numérateur de chaque côté de
l'égalité.  C'est-à-dire, lorsque
\[
19x -30 = A(x-3) + B(x-2) = (A+B)x - (3A+2B) \; .
\]
Si nous comparons les coefficients des puissances de $x$, nous trouvons
$A+B = 19$ et $-(3A+2B) = -30$.  Donc $A=-8$ et $B=27$.

Étape 4:
\begin{align*}
\int \frac{x^3}{x^2-5x+6}  \dx{x} &=
\int \left( x + 5 - \frac{8}{x-2} + \frac{27}{x-3}\right) \dx{x} \\
&= \frac{1}{2}x^2 + 5 x + -8 \ln|x-2| + 27 \ln|x-3| + C \\
&= \frac{1}{2}x^2 + 5x + \ln\left(\frac{|x-3|^{27}}{|x-2|^8}\right) + C \; .
\end{align*}
\end{egg}

\begin{rmk}
Lorsque que les racines du dénominateur sont simples, nous avons une
expression de la forme suivante à l'étape 3.
\[
\frac{r(x)}{q(x)} = \frac{A_1}{(x-\alpha_1)} + 
\frac{A_2}{(x-\alpha_2)} + \ldots + \frac{A_n}{(x-\alpha_n)}
\]
où  $\alpha_1$, $\alpha_2$, \ldots, $\alpha_n$ sont les racines
distinctes de $q(x)$.  Il y a alors une façon simple de déterminer les
valeurs de $A_1$, $A_2$, \ldots, $A_n$.  En mettant sur un dénominateur
commun, nous obtenons au numérateur une équation de la forme 
\begin{align*}
r(x) &= A_1(x-\alpha_2)(x-\alpha_3) \ldots (x-\alpha_n)
+ A_2(x-\alpha_1)(x-\alpha_3)\ldots(x-\alpha_n) + \ldots \\
&+ A_n(x-\alpha_1)(x-\alpha_2) \ldots (x-\alpha_{n-1})  .
\end{align*}
Il suffit d'évaluer chacun des côtés de l'égalité précédente à
$\alpha_1$, $\alpha_2$, \ldots , $\alpha_n$ pour obtenir
respectivement les valeurs de $A_1$, $A_2$, \ldots, $A_n$.

Si nous utilisons cette méthode à l'exemple précédent, nous obtenons
\[
19x -30 = A(x-3) + B(x-2) \ .
\]
Ainsi, $27=B$ pour $x=3$ et $8=-A$ pour $x=2$.
\end{rmk}

\begin{egg}
Évaluons l'intégrale indéfinie
$\displaystyle \int \frac{x^4+x^3-x^2-x+1}{x^3-x} \dx{x}$.

Étape 1: Comme le numérateur est un polynôme de degré plus grand que
le dénominateur, nous divisons pour obtenir
\[
\frac{x^4+x^3-x^2-x+1}{x^3-x} = x + 1 + \frac{1}{x^3-x} \; .
\]

Étape 2: Nous avons la factorisation $x^3-x = x(x-1)(x+1)$.

Étape 3: Nous cherchons $A$, $B$ et $C$ tels que
\[
\frac{1}{x^3-x} = \frac{A}{x} + \frac{B}{x-1} + \frac{C}{x+1} \; .
\]
Si nous exprimons les fractions sur un même dénominateur commun, nous
obtenons l'égalité suivante pour les numérateurs.
\[
1 = A(x-1)(x+1) + Bx(x+1) + Cx(x-1) \; .
\]
Pour $x=0$, nous obtenons $1=-A$ et donc $A=-1$.  Pour $x=1$, nous obtenons
$1 = 2B$ et donc $B = 1/2$.  Finalement, pour $x=-1$, nous obtenons
$1 = 2C$ et donc $C = 1/2$.

Étape 4:
\begin{align*}
\int \frac{x^4+x^3-x^2-x+1}{x^3-x}  \dx{x} &=
\int (x + 1) \dx{x} + \int \frac{1}{x^3-x} \dx{x} \\
&= \frac{1}{2}x^2 + x - \int \frac{1}{x} \dx{x} +
\frac{1}{2}\int \frac{1}{x-1} \dx{x}
+ \frac{1}{2}\int \frac{1}{x+1} \dx{x} \\
&= \frac{1}{2}x^2 + x - \ln|x| + \frac{1}{2}\ln|x-1|
+\frac{1}{2}\ln|x+1| + C \\
&= \frac{1}{2}x^2 + x - \ln|x| + \frac{1}{2}\ln|(x-1)(x+1)|+ C \; .
\end{align*}
\end{egg}

L'exemple suivant ne fait partie en soit de la méthode d'intégration
par fractions partielles.  Nous avons déjà vu une intégrale semblable à
l'exemple~\ref{EGintegration}.  Cependant, nous rencontrons souvent cette
situation quand nous essayons d'utiliser la méthode d'intégration par
fractions partielles.  Il est donc important d'être capable de
calculer de telles intégrales.

\begin{egg}
Évaluons l'intégrale indéfinie
$\displaystyle \int \frac{1}{x^2+6x+12}  \dx{x}$.

Étape 1: Nous n'avons pas à diviser les polynômes car le degré du
dénominateur est plus grand que celui du numérateur.

Étape 2: Le polynôme $x^2+6x+12$ est irréductible car il n'admet pas
de racines réelles.

La méthode des fractions partielles ne s'applique pas.  Complétons le
carré du dénominateur pour obtenir $x^2+6x+12 = (x+3)^2 + 3$.
Ainsi,
\[
\int \frac{1}{x^2+6x+12} \dx{x} = \int \frac{1}{(x+3)^2 + 3} \dx{x}
= \frac{1}{3} \int \frac{1}{ \left((x+3)/\sqrt{3}\right)^2 + 1} \dx{x} \; .
\]
Si $\displaystyle y=\frac{x+3}{\sqrt{3}}$, alors
$\displaystyle \dx{y} = \frac{1}{\sqrt{3}} \, \dx{x}$ et
\begin{align*}
\frac{1}{3} \int \frac{1}{ \left((x+3)/\sqrt{3}\right)^2 + 1} \dx{x}
&= \frac{\sqrt{3}}{3} \int \frac{1}{\left((x+3)/\sqrt{3}\right)^2 + 1}
\,\left(\frac{1}{\sqrt{3}}\right)\dx{x} \\
&= \frac{\sqrt{3}}{3}\int \frac{1}{y^2 + 1} \dx{y}\bigg|_{y=(x+3)/\sqrt{3}}
= \frac{\sqrt{3}}{3} \arctan(y)\bigg|_{y=(x+3)/\sqrt{3}} + C \\
&= \frac{\sqrt{3}}{3} \arctan\left(\frac{x+3}{\sqrt{3}}\right) + C \; .
\end{align*}
\label{no_part_frac}
\end{egg}

\begin{egg}[\eng]
Évaluons l'intégrale indéfinie 
$\displaystyle \int \frac{4x+1}{(x-3)(x^2+6x+12)}  \dx{x}$.

Étape 1: Nous n'avons pas a diviser les polynômes car le degré du
dénominateur est plus grand que celui du numérateur.

Étape 2: Le dénominateur $(x-3)(x^2+6x+12)$ est déjà exprimé comme un
produit de facteurs irréductibles car le polynôme $x^2+6x+12$ n'admet
pas de racines réelles.

Étape 3: Nous cherchons $A$, $B$ et $C$ tels que
\[
\frac{4x+1}{(x-3)(x^2+6x+12)} = \frac{A}{x-3} + \frac{Bx+C}{x^2+6x+12} \; .
\]
Si nous exprimons les fractions sur un même dénominateur commun, nous
obtenons l'égalité suivante pour les numérateurs.
\[
4x+1 = A(x^2+6x+12) + (Bx+C)(x-3) = (A+B)x^2 + (6A-3B+C)x +(12A-3C) \; .
\]
Si nous comparons les coefficients des puissances de $x$, nous trouvons
$A+B = 0$, $6A-3B+C = 4$ et $12A-3C=1$.  Donc $A= 1/3$, $B=-A$ et
$C=1$.

Étape 4:
\begin{align*}
&\int \frac{4x+1}{(x-3)(x^2+6x+12)}  \dx{x}
= \frac{1}{3} \int \frac{1}{x-3} \dx{x} +
\int \frac{-\frac{1}{3}x + 1}{x^2+6x+12} \dx{x} \\
&\quad = \frac{1}{3} \ln|x-3|
- \frac{1}{3} \int \frac{x-3}{x^2+6x+12} \dx{x}
= \frac{1}{3}\ln|x-3|
- \frac{1}{3} \int \frac{x-3}{(x+3)^2 + 3} \dx{x} \\
&\quad = \frac{1}{3}\ln|x-3|
- \frac{1}{3} \int \frac{x+3}{(x+3)^2 + 3} \dx{x}
- \frac{1}{3} \int \frac{-6}{(x+3)^2 + 3} \dx{x} \; .
\end{align*}

Si $y=(x+3)^2$, alors $\dx{y} = 2(x+3)\dx{x}$ et
\begin{align*}
\int \frac{x+3}{(x+3)^2 + 3} \dx{x} &=
\frac{1}{2} \int \frac{1}{(x+3)^2 + 3} \times 2(x+3)\dx{x} \\
&= \frac{1}{2} \int \frac{1}{y + 3} \dx{y}\bigg|_{y=(x+3)^2}
= \frac{1}{2} \ln|y+3|\bigg|_{y=(x+3)^2} + C_1 \\
&= \frac{1}{2} \ln((x+3)^2+3) + C_1 \; .
\end{align*}

De plus, si nous utilisons le résultat de l'exemple~\ref{no_part_frac},
\[
\int \frac{-6}{(x+3)^2 + 3} \dx{x} =
-6 \int \frac{1}{x^2+6x+12} \dx{x} =
-2\sqrt{3} \arctan\left(\frac{x+3}{\sqrt{3}}\right) + C_2 \; .
\]

Ainsi,
\begin{align*}
&\int \frac{4x+1}{(x-3)(x^2+6x+12)}  \dx{x} \\
&\qquad = \frac{1}{3}\ln|x-3| 
-\frac{1}{6} \ln((x+3)^2+3)
+\frac{2\sqrt{3}}{3} \arctan\left(\frac{x+3}{\sqrt{3}}\right) + C
\end{align*}
où $C = -(C_1+C_2)/3$.
\end{egg}

\begin{egg}[\eng]
Évaluons l'intégrale indéfinie
$\displaystyle \int \frac{1}{3\sin(x)+\cos(x)} \dx{x}$.

Nous utilisons la substitution (\ref{weies1}) avec (\ref{weies2}),
(\ref{weies3}) et (\ref{weies4}) pour obtenir
\begin{align*}
\int \frac{1}{3\sin(x)+\cos(x)} \dx{x}
&= \int \frac{1}{\frac{6t}{1+t^2} + \frac{1-t^2}{1+t^2}}
\left(\frac{2}{t^2+1}\right) \dx{t}
= \int \frac{2}{1+6t-t^2} \dx{t} \\
&= -\int \frac{2}{(t-(3+\sqrt{10}))(t-(3-\sqrt{10}))} \dx{t} \\
&= -\frac{1}{\sqrt{10}} \int \left( \frac{1}{t-(3+\sqrt{10})} -
\frac{1}{t-(3-\sqrt{10})}\right) \dx{t} \\
&= -\frac{1}{\sqrt{10}} \left( \ln|t-(3+\sqrt{10})|
- \ln|t-(3-\sqrt{10})| \right) + C \\
&= -\frac{1}{\sqrt{10}} \ln\left|
\frac{t-(3+\sqrt{10})}{t-(3-\sqrt{10})}\right| + C \\
&= -\frac{1}{\sqrt{10}} \ln\left|
\frac{\tan(x/2)-(3+\sqrt{10})}{\tan(x/2)-(3-\sqrt{10})}\right| + C \ .
\end{align*}
La méthode d'intégration par fractions partielles a été utilisée pour
évaluer cet intégrale.
\end{egg}

\subsection{Substitutions trigonométriques \eng}

Si l'intégrande contient un facteur de la forme $\sqrt{a^2-x^2}$,
$\sqrt{a^2+x^2}$ ou $\sqrt{x^2-a^2}$, une substitution trigonométrique
pourrait être nécessaire.

\begin{center}
\begin{tabular}{l|l}
l'intégrande contient & substitution \\
\hline
\rule[4pt]{0em}{1em} $\sqrt{a^2-x^2}$ & $x=a\sin(t)$ ou $x=a\cos(t)$ \\
\rule[4pt]{0em}{1em} $\sqrt{a^2+x^2}$ & $x=a\tan(t)$ \\
\rule[4pt]{0em}{1em} $\sqrt{x^2-a^2}$ & $x=a\sec(t)$
\end{tabular}
\end{center}

Pour ce genre de substitutions, nous utilisons la règle de substitution
dans le sens inverse
\[
\int f(x)  \dx{x} = \int f(g(t))g'(t) \dx{t}  \ .
\]
La substitution $x=g(t)$ donne le côté droit.

\begin{egg}
Évaluons l'intégrale indéfinie
$\displaystyle \int \frac{1}{x^2\sqrt{1-x^2}} \dx{x}$ pour
$-1 < x < 1$.

Posons $x=\sin(t)$ avec $-\pi/2 < t < \pi/2$.  Donc
$\dx{x} = \cos(t)\dx{t}$ et
\begin{align*}
\int \frac{1}{x^2\sqrt{1-x^2}} \dx{x} &=
\int \frac{1}{\sin^2(t) \sqrt{1- \sin^2(t)}} \ \cos(t) \dx{t} \\
&= \int \frac{\cos(t)}{\sin^2(t) \cos(t)} \dx{t}
= \int \frac{1}{\sin^2(t)} \dx{t} = \int \csc^2(t) \dx{t}
\end{align*}
où nous avons utilisé
\[
\sqrt{1- \sin^2(t)} = \sqrt{\cos^2(t)} = |\cos(t)| = \cos(t) > 0
\]
pour $-\pi/2<t<\pi/2$.  Ainsi,
\[
\int \frac{1}{x^2\sqrt{1-x^2}} \dx{x} = \int \csc^2(t) \dx{t}
= - \cot(t) + C \; .
\]
Or $x=\sin(t)$ pour $-\pi/2<t<\pi/2$ donne $\cos(t) = \sqrt{1-x^2} >0$.
Donc
\[
\cot(t) = \frac{\cos(t)}{\sin(t)} = \frac{\sqrt{1-x^2}}{x}
\]
et
\[
\int \frac{1}{x^2\sqrt{1-x^2}} \dx{x}
= -\cot(t) + C = -\frac{\sqrt{1-x^2}}{x} + C \; .
\]
\end{egg}

\begin{egg}
Évaluons l'intégrale indéfinie
$\displaystyle \int \sqrt{x^2+1} \dx{x}$ pour $x \in \RR$.

Posons $x=\tan(t)$ pour $-\pi/2 < t < \pi/2$.  Donc
$\dx{x} = \sec^2(t) \dx{t}$ et 
\[
\sqrt{x^2+1} = \sqrt{\tan^2(t) +1} = \sqrt{\sec^2(t)} = |\sec(t)| =
\sec(t) >0
\]
pour $-\pi/2 < t < \pi/2$.  Ainsi,
\[
I = \int \sqrt{x^2+1} \dx{x} = \int \sec(t) \, \sec^2(t) \dx{t} \ .
\]
Nous utilisons la méthode d'intégration par parties pour évaluer cette
dernière intégrale.  Nous avons $\sec(t) \, \sec^2(t) = f(t)g'(t)$ pour
$f(t)=\sec(t)$ et $g'(t) = \sec^2(t)$.  Donc $f'(t) = \sec(t)\tan(t)$,
$g(t) = \tan(t)$ et
\begin{align*}
I &= \int \sec(t) \, \sec^2(t) \dx{t} = \int f(t) g'(t) \dx{t}
= f(t)g(t) - \int f'(t) g(t) \dx{t} \\
&= \sec(t) \tan(t) - \int \sec(t) \tan^2(t) \dx{t}
= \sec(t) \tan(t) - \int \sec(t) \left(\sec^2(t) - 1\right) \dx{t}\\
&= \sec(t) \tan(t) + \int \sec(t) \dx{t} - \int \sec(t)\sec^2(t)\dx(t) \\
&= \sec(t) \tan(t) + \int \sec(t) \dx{t} - I  \; .
\end{align*}
Si nous résolvons pour $I$, nous trouvons
\[
I = \int \sqrt{x^2+1} \dx{x} = \int \sec^3(t) \dx{t} = \frac{1}{2} \left(
\sec(t) \tan(t) + \int \sec(t) \dx{t} \right) \; .
\]
À l'exemple~\ref{EGsubA}(e), nous avons montré que
\[
\int \sec(t) \dx{t} = \ln|\sec(t)+\tan(t)| + C_1 \; .
\]
Donc
\[
I = \int \sqrt{x^2+1} \dx{x} = \int \sec^3(t) \dx{t} =
\frac{1}{2} \sec(t) \tan(t) + \frac{1}{2} \ln|\sec(t)+\tan(t)| + C
\]
où $C=C_1/2$.  Finalement, puisque $x=\tan(t)$ pour $-\pi/2 < t < \pi/2$,
nous avons $\displaystyle \sec(t) = \sqrt{1+\tan^2(t)} = \sqrt{1+x^2}$.
Ainsi,
\[
I = \int \sqrt{x^2+1} \dx{x} = \frac{1}{2}\, x\sqrt{1+x^2}
+ \frac{1}{2} \ln\left|x + \sqrt{1+x^2}\right| + C\ .
\]
\label{intSRQTxdpu}
\end{egg}

\begin{egg}
Évaluons l'intégrale indéfinie
$\displaystyle \int \left(\frac{x-1}{x}\right)^{1/2} \dx{x}$
pour $x>1$.

Nous avons
\[
\int \left(\frac{x-1}{x}\right)^{1/2} \dx{x}
= \int \frac{\sqrt{x-1}}{\sqrt{x}} \dx{x} \ .
\]
Posons $x=y^2$ pour $y>1$.  Donc $\dx{x} = 2 y \dx{y}$, 
$\sqrt{x} = y$ et
\[
\int \frac{\sqrt{x-1}}{\sqrt{x}} \dx{x}
= \int \frac{\sqrt{y^2-1}}{y} \times 2y \dx{y}
= 2 \int \sqrt{y^2-1} \dx{y}
\]
Pour évaluer cette dernière intégrale, nous utilisons la substitution
$y=\sec(\theta)$ pour $0<\theta<\pi/2$.  Donc
$\dx{y} = \sec(\theta)\tan(\theta)\dx{\theta}$ et
$\displaystyle \sqrt{y^2-1} = \sqrt{\sec^2(\theta)-1} = \tan(\theta)$ 
car $\tan(\theta)>0$ pour $0<\theta<\pi/2$.  Ainsi,
\begin{align*}
\int \sqrt{y^2-1} \dx{y} &= \int \tan^2(\theta)\sec(\theta)\dx{\theta} \\
&= \int (\sec^2(\theta)-1) \sec(\theta)\dx{\theta} 
= \int \sec^3(\theta)\dx{\theta} - \int \sec(\theta)\dx{\theta} 
\end{align*}
À l'exemple~\ref{EGsubA}(e), nous avons montré que
\[
\int \sec(t) \dx{t} = \ln|\sec(t)+\tan(t)| + C_1
\]
et, à l'exemple précédent, nous avons montré que
\[
\int \sec^3(t) \dx{t} = \frac{1}{2}\sec(t) \tan(t) + \frac{1}{2}
\ln|\sec(t)+\tan(t)| + C_2 \ .
\]
Ainsi,
\[
\int \sqrt{y^2-1} \dx{y}
= \int \tan^2(\theta)\sec(\theta)\dx{\theta}
= \frac{1}{2}\sec(t) \tan(t) - \frac{1}{2}
\ln|\sec(t)+\tan(t)| + C_3
\]
où $C_3 = C_2 - C_1$.  Puisque $\sec(\theta) =y$ et
$\tan(\theta) = \sqrt{y^2-1}$ pour $0<\theta<\pi/2$, nous obtenons
\[
\int \sqrt{y^2-1} \dx{y} = \frac{1}{2}y\sqrt{y^2-1} -
\frac{1}{2} \ln|y+\sqrt{y^2-1}| + C_3 \ .
\]

Finalement, puisque $y=\sqrt{x}$ pour $x>1$, nous obtenons
\[
\int \left(\frac{x-1}{x}\right)^{1/2} \dx{x}
= 2 \int \sqrt{y^2-1} \dx{y}
= \sqrt{x(x-1)}- \ln\left|\sqrt{x} + \sqrt{x-1}\right| + C
\]
où $C= 2C_3$.
\end{egg}

\begin{egg}
Évaluons l'intégrale indéfinie 
$\displaystyle \int \frac{\sqrt{x^2-1}}{x^3}  \dx{x}$ pour
$x>1$.

Si $x = \sec(t)$ pour $0 < t < \pi/2$, alors
$\dx{x} = \tan(t)\sec(t) \dx{t}$ et
\[
\int \frac{\sqrt{x^2-1}}{x^3} \dx{x}
= \int \frac{\sqrt{\sec^2(t) -1}}{\sec^3(t)} \sec(t)\tan(t) \dx{t}
= \int \frac{\tan^2(t)}{\sec^2(t)} \dx{t}
= \int \sin^2(t) \dx{t} \; ,
\]
où nous avons utilisé l'identité
\[
\sqrt{\sec^2(t) -1} = \sqrt{\tan^2(t)} = |\tan{x}| = \tan{x} > 0
\]
pour $0< t < \pi/2$.  Puisque $\sin^2(t) = (1-\cos(2t))/2$, nous avons
\[
\int \sin^2(t)  \dx{t} = \frac{1}{2} \int \left( 1- \cos(2t)\right) \dx{t}
= \frac{1}{2}\left( t - \frac{1}{2}\sin(2t) \right) + C
= \frac{t}{2} - \frac{1}{4}\sin(2t) + C \; .
\]
Pour exprimer le résultat de notre intégrale en fonction de $x$
seulement, notons que $t=\arcsec(x)$ car $x=\sec(t)$.  De plus,
$x = \sec(t)$ donne $\cos(t) = 1/x$ et
\[
\sin(t) = \sqrt{1-\cos^2(t)} = \sqrt{1-\frac{1}{x^2}}
= \frac{\sqrt{x^2 - 1}}{x}
\]
car $x = \sec(t) > 1$ et $\sin(t) > 0$ pour $0< t < \pi/2$.

Ainsi,
\begin{align*}
\int \frac{\sqrt{x^2-1}}{x^3}  \dx{x}
&= \frac{t}{2} - \frac{1}{4}\sin(2t) + C
= \frac{t}{2} - \frac{1}{2}\sin(t)\cos(t) + C \\
&= \frac{\arcsec(x)}{2} - \frac{\sqrt{x^2-1}}{2x^2} + C
\end{align*}
où l'identité trigonométrique $\sin(2t) = \sin(t+t) = 2\sin(t)\cos(t)$
a été utilisé pour la deuxième égalité ci-dessus.
\end{egg}

\begin{egg}
Évaluons l'intégrale indéfinie
$\displaystyle \int \frac{1}{(4x^2-25)^{3/2}} \dx{x}$
pour $x>5/2$.

Nous pouvons récrire cette intégrale de la façon suivante.
\[
\int \frac{1}{(4x^2-25)^{3/2}} \dx{x}
= \frac{1}{8} \, \int \frac{1}
{\left(x^2 - \left(\frac{5}{2}\right)^2\right)^{3/2}} \dx{x}\; .
\]
Si $\displaystyle x = \frac{5}{2} \sec(t)$ pour
$0 < t < \pi/2$, alors
$\displaystyle \dx{x} = \frac{5}{2} \tan(t)\sec(t) \dx{t}$ et
\begin{align*}
\frac{1}{8} \, \int \frac{1}
{\left(x^2 - \left(\frac{5}{2}\right)^2\right)^{3/2}} \dx{x}
&= \frac{1}{8} \,\int \frac{1} {\left(\left(\frac{5}{2}\right)^2\sec^2(t) -
\left(\frac{5}{2}\right)^2\right)^{3/2}}
\left( \frac{5}{2} \tan(t)\sec(t)\right) \dx{t} \\
&= \frac{1}{50} \, \int
\frac{\tan(t)\sec(t)}{\left(\sec^2(t) - 1\right)^{3/2}} \dx{t}
= \frac{1}{50} \, \int
\frac{\tan(t)\sec(t)}{\left(\tan^2(t)\right)^{3/2}} \dx{t} \\
&= \frac{1}{50} \, \int \frac{\tan(t)\sec(t)}{\tan^3(t)} \dx{t}
= \frac{1}{50} \, \int \frac{\cos(t)}{\sin^2(t)} \dx{t} \\
& = \frac{1}{50} \, \int \cot(t) \csc(t) \dx{t}
= - \frac{1}{50} \csc(t) + C \ .
\end{align*}
où nous avons utilisé l'identité
\[
\sec^2(t) -1 = \tan^2(t) \quad \text{et} \quad
\sqrt{\tan^2{x}} = \tan{x} > 0
\]
pour $0< t < \pi/2$.  Puisque $\displaystyle \sec(t) = \frac{2}{5}\, x$ donne
$\displaystyle \cos(t) = \frac{5}{2x}$, nous obtenons
\[
\sin(t) = \sqrt{1-\cos^2(t)} = \sqrt{1 -\left(\frac{5}{2x}\right)^2}
= \frac{\sqrt{4x^2-25}}{2x}
\]
car $\displaystyle x = \frac{5}{2}\sec(t) > \frac{5}{2} $ et
$\sin(t) \geq 0$ pour $0\leq t < \pi/2$.

Ainsi,
\[
\int \frac{1}{(4x^2-25)^{3/2}} \dx{x}
= - \frac{1}{50\sin(t)} + C = - \frac{x}{25 \, \sqrt{4x^2-25}} + C \; .
\]
\end{egg}

\section{Intégrales définies}

Considérons la région $R$ bornée par la courbe $y=e^{-x}$ et les
droites $x=0$, $x=2$ et $y=0$ (figure~\ref{AREA}).  La façon 
intuitive d'estimer l'aire $A$ de cette région est de partitionner en
petits rectangles cette région et de faire la somme de l'aire de
chaque rectangle car il est simple de calculer l'aire d'un rectangle.

\PDFfig{7_integrales/area}{La région bornée par la courbe $y=e^{-x}$
et les droites $x=0$, $x=2$ et $y=0$}{La région bornée par la courbe
$y=e^{-x}$ et les droites $x=0$, $x=2$ et $y=0$}{AREA}

Les rectangles qui sont représentés à la figure~\ref{AREA2} sont
construits de la façon suivante.  L'intervalle $[0,2]$ est partagé en
$10$ sous-intervalles de même longueur.  La longueur de chaque 
sous-intervalle est donc $2/10 = 1/5$.  Nous obtenons $10$
sous-intervalles de la forme $[x_i,x_{i+1}]$ pour $i=0$, $1$, $2$,
\ldots, $9$ où $x_0 = 0$, $x_1 = 1/5$, $x_2 = 2/5$, \ldots ,
$x_{10} = 10/5 = 2$.   Les $10$ rectangles représentés sont les
rectangles dont la base est l'intervalle $[x_i, x_{i+1}]$ et la
hauteur est $e^{-x_i}$ pour $i=0$, $1$, $2$, \ldots, $9$.

\PDFfig{7_integrales/area2}{Un somme à gauche avec $10$ termes pour
l'intégrale de $e^{-x}$}{Partition en $10$ petits rectangles de base
$[x_i,x_{i+1}]$ et de hauteur $e^{-x_i}$}{AREA2}

L'aire d'un rectangle, dont la base est l'intervalle $[x_i, x_{i+1}]$
et la hauteur est $e^{-x_i}$, est $e^{-x_i} \left(\frac{1}{5}\right)$.
La somme de l'aire de chaque rectangle est donc
\[
G_{10} = e^{0} \left(\frac{1}{5}\right)
+ e^{-1/5} \left(\frac{1}{5}\right)
+ e^{-2/5} \left(\frac{2}{5}\right)
+ \ldots + e^{-9/5} \left(\frac{1}{5}\right) \approx
0.9540114845\ldots
\]
Nous pouvons dire que l'aire $A$ de la région $R$ est approximativement
$G_{10} = 0.9540114845\ldots$.  En fait, nous avons surestimé l'aire de la
région $R$ car nos rectangles recouvrent la région.

Le choix de $10$ intervalles est arbitraire.  Pour obtenir une
meilleure approximation de l'aire de la région $R$, nous pourrions
prendre plus d'intervalles.  Par exemple, si nous partageons l'intervalle
$[0,2]$ en $20$ sous-intervalles de même longueur.  La longueur de
chaque sous-intervalle est alors de $2/20 = 1/10$ et nous obtenons $20$
intervalles de la forme $[x_i,x_{i+1}]$ pour $i=0$, $1$, $2$, \ldots,
$19$ où $x_0 = 0$, $x_1 = 1/10$, $x_2 = 2/10 = 1/5$, \ldots,
$x_{20} = 20/10 = 2$. Comme dans le cas précédent avec $10$
intervalles, nous utilisons les rectangles dont la base est l'intervalle
$[x_i, x_{i+1}]$ et la hauteur est $e^{-x_i}$ pour $i=0$, $1$, $2$,
\ldots, $19$ (figure~\ref{AREA3}).

\PDFfig{7_integrales/area3}{Un somme à gauche avec $20$ termes pour
l'intégrale de $e^{-x}$}{Partition en $20$ petits rectangles de base
$[x_i,x_{i+1}]$ et de hauteur $e^{-x_i}$}{AREA3}

L'aire d'un rectangle, dont la base est l'intervalle $[x_i, x_{i+1}]$
et la hauteur est $e^{-x_i}$, est maintenant
$e^{-x_i} \left(\frac{1}{10}\right)$.
La somme de l'aire de chaque rectangle est donc
\begin{align*}
G_{20} &= e^{0} \left(\frac{1}{10}\right)
+ e^{-1/10} \left(\frac{1}{10}\right)
+ e^{-1/5} \left(\frac{1}{10}\right)
+ e^{-3/10} \left(\frac{1}{10}\right) + \ldots \\
&\quad + e^{-9/5} \left(\frac{1}{10}\right)
+ e^{-19/10} \left(\frac{1}{10}\right) \approx 0.9086183864\ldots
\end{align*}
et nous obtenons que l'aire $A$ de la région $R$ est approximativement
$G_{20} = 0.9086183864\ldots$.  Nous avons encore surestimé l'aire de la
région $R$.  Par contre, cette surestimation est moins grande que la
précédente.

Si $k$ est le nombre de rectangles utilisés, nous avons que la somme $G_k$
de l'aire des $k$ rectangles approche l'aire $A$ de la région $R$
lorsque $k$ augmente.  Les sommes $G_k$ sont appelées des
{\bfseries sommes à gauche} car nous définissons la hauteur d'un rectangle
de base $[x_i, x_{i+1}]$ comme étant la valeur de $e^{-x}$ à l'extrémité
gauche $x_i$ de l'intervalle.\index{Somme à Gauche}

\begin{rmk}
Lorsque la fonction $f$ est décroissante, comme c'est le cas pour
$f(x)=e^{-x}$, les sommes à gauche $G_k$ sont toutes des surestimations
de l'aire $A$ de la région $R$ bornée par la courbe $y=f(x)$ et les
droites $x=a$, $x=b$ et $y=0$.
\end{rmk}

Revenons à la partition de l'intervalle $[0,2]$ en $10$
sous-intervalles $[x_i,x_{i+1}]$ pour $i=0$, $1$, \ldots, $9$ où
$x_0 = 0$, $x_1 = 1/5$, $x_2=2/5$, \ldots , $x_{9} = 9/5$ et
$x_{10} = 2$.  Ces intervalles sont tous de longueur égale à $1/5$.
Considérons maintenant les rectangles de base $[x_i,x_{i+1}]$ et de
hauteur $e^{-x_{i+1}}$ (figure~\ref{AREA4}).

\PDFfig{7_integrales/area4}{Un somme à droite avec $10$ termes pour
l'intégrale de $e^{-x}$}{Partition en $10$ petits rectangles de base
$[x_i,x_{i+1}]$ et de hauteur $e^{-x_{i+1}}$}{AREA4}

L'aire d'un rectangle, dont la base est l'intervalle $[x_i, x_{i+1}]$
et la hauteur est $e^{-x_{i+1}}$, est
$e^{-x_{i+1}} \left(\frac{1}{5}\right)$.  La somme de l'aire de chaque
rectangle est donc
\[
D_{10} = e^{-1/5} \left(\frac{1}{5}\right)
+ e^{-2/5} \left(\frac{1}{5}\right)
+ e^{-3/5} \left(\frac{1}{5}\right) + \ldots
+ e^{-2} \left(\frac{1}{5}\right) \approx
0.7810785411\ldots
\]
Nous pouvons dire que l'aire $A$ de la région $R$ est approximativement
$D_{10} = 0.7810785411\ldots$.  En fait, nous avons sous-estimé l'aire $A$ de
la région $R$ car nos rectangles sont tous à l'intérieur de la région $R$.
 
Finalement, nous considérons les $20$ rectangles de base $[x_i,x_{i+1}]$
et de hauteur $e^{-x_{i+1}}$ où les sous-intervalles $[x_i,x_{i+1}]$
pour $i=0$, $1$, \ldots, $19$ sont bornées par les points $x_0 = 0$,
$x_1 = 1/10$, $x_2=1/5$, \ldots, $x_{19} = 19/10$ et $x_{20} = 2$,
Chaque intervalle est de longueur égale à $1/10$
(figure~\ref{AREA5}).

\PDFfig{7_integrales/area5}{Un somme à droite avec $20$ termes pour
l'intégrale de $e^{-x}$}{Partition en $20$ petits rectangles de base
$[x_i,x_{i+1}]$ et de hauteur $e^{-x_{i+1}}$}{AREA5}

L'aire d'un rectangle, dont la base est l'intervalle $[x_i, x_{i+1}]$
et la hauteur est $e^{-x_{i+1}}$, est maintenant
$e^{-x_{i+1}} \left(\frac{1}{10}\right)$.  La somme de l'aire de chaque
rectangle est donc
\begin{align*}
D_{20} &= e^{-1/10} \left(\frac{1}{10}\right)
+ e^{-1/5} \left(\frac{1}{10}\right)
+ e^{-3/10} \left(\frac{1}{10}\right) + \ldots \\
& \quad + e^{-19/10} \left(\frac{1}{10}\right)
+ e^{-2} \left(\frac{1}{10}\right)  \approx 0.8221519147\ldots
\end{align*}
Nous pouvons dire que l'aire $A$ de la région $R$ est approximativement
$D_{20} = 0.8221519147\ldots$.  Nous avons toujours une sous-estimation de
l'aire $A$ de la région $R$ car nos rectangles sont tous à l'intérieur
de la région $R$ mais la sous-estimation est plus grande que lorsque
nous utilisons seulement $10$ rectangles.

Les sommes $D_k$ sont appelées des {\bfseries sommes à droite} car nous
définissons la hauteur d'un rectangle de base $[x_i, x_{i+1}]$ comme étant
la valeur de $e^x$ à l'extrémité droite $x_{i+1}$ de
l'intervalle. Comme pour les sommes à gauche, si $k$ est le nombre de
rectangles utilisés, nous avons que la somme $D_k$ de l'aire des $k$
rectangles approche l'aire $A$ de la région $R$ lorsque $k$
augmente.\index{Somme à droite} 

\begin{rmk}
Lorsque la fonction $f$ est décroissante, comme c'est le cas pour
$f(x)=e^{-x}$, les sommes à droite $D_k$ sont toutes des sous-estimations
de l'aire $A$ de la région $R$ bornée par la courbe $y=f(x)$ et les
droites $x=a$, $x=b$ et $y=0$.
\end{rmk}

Pour $1000$ sous-intervalles de $[0,2]$, nous obtenons
\[
G_{1000} = 0.8655296697\ldots \qquad \text{et} \qquad
D_{1000} = 0.8638003402\ldots
\]
Si nous combinons les sommes à droite et à gauche que nous avons
calculées, nous obtenons
\[
G_{10} > G_{20} > G_{1000} > A > D_{1000} > D_{20} > D_{10} \; .
\]
La valeur exacte de l'aire $A$ de la région $R$ est
$A = 0.8646647167\ldots$.

Comme le symbole de sommation $\displaystyle \sum$ sera très utile, il
est probablement important de revoir sa définition pour le bénéfice du
lecteur.

\begin{defn}
Soit $\{ a_i : i \in \ZZ \}$, un ensemble de nombres indexés par les
entiers.  Si $n$ et $m$ sont deux entiers tels que $n\leq m$, alors la
somme des nombres $a_n$, $a_{n+1}$, \ldots, $a_m$ est dénotée
\[
\sum_{i=n}^m a_i \equiv a_n + a_{n+1} + a_{n+2} + \ldots + a_{m-1} + a_m \; .
\]
\end{defn}

Si nous partitionnons l'intervalle $[0,2]$ en $k$ sous-intervalles de
longueur $2/k$, nous obtenons les intervalles de base $[x_i,x_{i+1}]$ où
$x_0=0$, $x_1=2/k$, $x_2=4/k$, \ldots, $x_{k-1} = 2(k-1)/k$ et
$x_k= 2$. 

Si nous utilisons le symbole de sommation, nous avons
\[
G_k = \frac{2}{k} \sum_{i=0}^{k-1} e^{-x_i}
\quad \text{et} \quad
D_k = \frac{2}{k} \sum_{i=0}^{k-1} e^{-x_{i+1}}
= \frac{2}{k} \sum_{i=1}^{k} e^{x_i}
\]
où $2/k$ est la longueur de la base des rectangles.  De plus,
\[
G_1 > \ldots > G_{k} > G_{k+1} > A > D_{k+1} > D_{k} > \ldots > D_1
\]
pour tout $k\geq 1$ et l'aire $A$ de la région $R$ est donnée par
\[
A= \lim_{k\rightarrow \infty} G_k = \lim_{k\rightarrow \infty} D_k \; .
\]

\subsection{Définition}

La définition de l'intégrale définie que nous donnons généralise le
calcul de l'aire qui a été fait à la section précédente pour la région
$R$ bornée par la courbe $y=e^{-x}$ et les droites $x=0$, $x=2$ et
$y=0$.

Avant de définir l'intégrale dans le cas général, il faut décrire
les fonctions pour lesquelles l'intégrale existera.  La raison pour
ces restrictions sur les fonctions que nous allons intégrer deviendra
plus clair lors de l'étude des intégrales impropres que nous ferons
prochainement.

\begin{defn}
Une fonction $f$ définie sur un intervalle $[a,b]$ est
{\bfseries bornée}\index{Fonction!bornée} s'il existe une constante
$M$ telle que $|f(x)|<M$ pour tout $x$ dans l'intervalle $[a,b]$.

Une fonction $f$ définie sur un intervalle $[a,b]$ est
{\bfseries continue par morceaux}\index{Fonction continue par
  morceaux} s'il y a une nombre fini de points $c$ où la fonction $f$
n'est pas continue et, à ces points $c$, nous avons que
\[
\lim_{\substack{x\rightarrow c+\\ x\in[a,b]}} f(x)
\qquad \text{et} \qquad
\lim_{\substack{x\rightarrow c-\\ x\in[a,b]}} f(x)
\]
existent.
\end{defn}

\begin{egg}
La fonction $f(x) = \lfloor x \rfloor$, où $\lfloor x \rfloor$ est le
plus grand entier plus petit que ou égale à $x$, est une fonction
continue par morceaux sur tout intervalle fermé de $\RR$.  Le graphe
de cette fonction est donné à la figure~\ref{STEP_FCN}.
\end{egg}

\PDFfig{7_integrales/steps}{Graphe de la fonction
$f(x) = \lfloor x \rfloor$}{Graphe de la fonction
$f(x) = \lfloor x \rfloor$ pour $-3\leq x \leq 5$.}{STEP_FCN}

Remarquons qu'une fonction continue par morceaux sur un intervalle
fermé est une fonction bornée.

\begin{defn}[Somme de Riemann] \index{Somme de Riemann}
Soit $f:[a,b]\rightarrow \RR$ un fonction continue par morceaux.  Soit
$a=x_0 < x_1 < x_2 < \ldots < x_k = b$ une partition ${\cal P}$
quelconque de l'intervalle $[a,b]$.  La longueur d'un sous-intervalle
$[x_i,x_{i+1}]$ est $\Delta x_i = x_{i+1}-x_i$.
La {\bfseries mesure} de la partition $\cal P$ est définie par
\[
\| {\cal P} \| = \max \left\{ \Delta x_i : 0\leq i < k \right\} \; .
\]
Pour chaque intervalle $[x_i,x_{i+1}]$, nous choisissons $x_i^\ast$ dans cet
intervalle.  La somme
\[
S_{\cal P} = \sum_{i=0}^{k-1} f\left(x_i^\ast\right) \Delta x_i
\]
est une {\bfseries somme de Riemann} pour la partition ${\cal P}$.
\end{defn}

Une même partition supporte un nombre infini de sommes de Riemann
selon le choix des points $x_i^\ast \in [x_i, x_{i+1}]$ pour
$0\leq i <k$.

\begin{defn} \label{def_non_rig}
Soit $f:[a,b]\rightarrow \RR$ un fonction continue par morceaux.
L'{\bfseries intégrale définie}\index{Intégrale définie} de $f$ de $a$
à $b$ est le nombre $I$ qui satisfait la condition suivante.
\[
S_{\cal P} \rightarrow I \quad \text{lorsque}
\quad \|{\cal P}\|\rightarrow 0 \ .
\]
Nous écrivons $\displaystyle \displaystyle \int_a^b f(x) \dx{x} = I$.

Les nombres $a$ et $b$ sont appelés les
{\bfseries bornes d'intégration}.\index{Borne d'intégration}
L'intervalle $[a,b]$ est appelé
{\bfseries l'intervalle d'intégration}.\index{Intervalle d'intégration} 
La fonction $f$ est appelée {\bfseries l'intégrande}\index{Intégrande}
et $x$ est la
{\bfseries variable d'intégration}.\index{Variable d'intégration}  Le
symbole $\dx{x}$ indique que la variable d'intégration est $x$.
\end{defn}

\begin{rmk}[\theory]
\[
S_{\cal P} \rightarrow I \quad \text{lorsque}
\quad \|{\cal P}\|\rightarrow 0 \ .
\]
veut dire que pour tout $\epsilon > 0$, il existe $\delta >0$ tel que
\[
\left| I - S_{\cal P} \right| < \epsilon
\]
quelle que soit la partition ${\cal P}$ de $[a,b]$
avec $\|{\cal P}\|<\delta$ et quelle que soit le choix des points
$x_i^\ast \in [x_i, x_{i+1}]$ associé à la partition ${\cal P}$.
\end{rmk}

La section~\ref{Riemann_int} présente une définition plus générale de
l'intégrale définie de Riemann.

Nous insistons sur le fait que le symbole $\dx{x}$
{\em n'est pas une variable}, il indique seulement que la variable
d'intégration est $x$.  Aucune manipulation algébrique avec $\dx{x}$
n'est permise.

Il découle de la définition de l'intégrale définie que nous pouvons
estimer la valeur de l'intégrale $\displaystyle \int_a^b f(x) \dx{x}$
en choisissant un partition très fine (i.e. $\|{\cal P}\|$ très petit)
avec une somme de Riemann subordonnée à cette partition.

Normalement, nous choisissons des partitions ${\cal P}$ telles que
$\Delta x_i$ est constant.

Soit $f:[a,b]\rightarrow \RR$ une fonction continue par morceaux.
Pour $k \in \NN^+$, posons $\Delta x = (b-a)/k$ et
$x_i = a + i \, \Delta x$ pour $i=0$, $1$, $2$, \ldots, $k$.
Les points $a=x_0 < x_1 < x_2 < \ldots < x_k = b$ forment une
partition ${\cal P}$ de l'intervalle $[a,b]$ en sous-intervalles
$[x_i, x_{i+1}]$ de même longueur $\Delta x$.  Nous avons que
$\displaystyle \|{\cal P}\| = \Delta x = (b-a)/k$
tend vers $0$ lorsque $k$ tend vers plus l'infini.

Pour chaque intervalle $[x_i,x_{i+1}]$, choisissons $x_i^\ast$ dans cet
intervalle.  Donc $x_i \leq x_i^\ast \leq x_{i+1}$ pour $0 \leq i < k$.
Nous obtenons la somme de Riemann
\[
S_k = \sum_{i=0}^{k-1} f(x_i^\ast) \Delta x \ .
\]
Cette somme représente la région en gris illustrée à la figure~\ref{AREA6}.

l'intégrale définie de $f$ de $a$ à $b$ peut être calculée à l'aide de
la formule
\begin{equation}\label{def_int_cpm}
\int_a^b f(x)\dx{x} = \lim_{k\rightarrow \infty} S_k \; .
\end{equation}

\begin{rmkList}
\begin{enumerate}
\item La limite (\ref{def_int_cpm}) existe toujours car $f$ est
  continue par morceau.  Cela pourrait être démontré à partir de la
  définition rigoureuse de l'intégrale donnée à la
  section~\ref{Riemann_int} ci-dessous.
\item Quelques rectangles de base $[x_i,x_{i+1}]$ et de hauteur
$|f(x_i^\ast)|$ sont représentés à la figure~\ref{AREA6}.  Comme
$f(x_i^\ast)$ peut être négatif, l'expression
$f(x_i^\ast)\,\Delta x$ représente l'aire du rectangle à un signe
près.
\item Si $x_i^\ast = x_i$ pour $0\leq i <k$, c'est-à-dire que
$x_i^\ast$ est l'extrémité gauche de l'intervalle $[x_i,x_{i+1}]$
pour $0\leq i <k$, alors $S_k = G_k$, la somme à gauche pour $k$
rectangles.
\item Si $x_i^\ast = x_{i+1}$ pour $0\leq i<k$, c'est-à-dire que
$x_i^\ast$ est l'extrémité droite de l'intervalle $[x_i,x_{i+1}]$ pour
$0\leq i <k$, alors $S_k = D_k$, la somme à droite pour $k$
rectangles.
\item Si $x_i^\ast$ est le point milieu de l'intervalle $[x_i, x_{i+1}]$
pour $0\leq i< k$, c'est-à-dire que $x_i^\ast = (x_i+x_{i+1})/2$ pour
$0\leq i<k$, alors $S_k$ est
{\bfseries une somme de Riemann pour le point milieu} 
\index{Somme de Riemann pour le point milieu} que nous dénotons $M_k$.
\end{enumerate}
\label{IDadgm}
\end{rmkList}

\PDFfig{7_integrales/area6}{Forme générale des rectangles utilisés pour
définir l'intégrale définie}{Forme générale des rectangles utilisés
pour définir l'intégrale définie}{AREA6}

\begin{egg}
Estimons $\displaystyle \int_{-1}^2  (1 + x^2) \dx{x}$.  Pour ce faire, nous
allons utiliser les sommes à gauche, les sommes à droites et les sommes
pour le point milieu avec $n=3$ et $n=6$ sous-intervalles.

Nous utiliserons le graphe de $f$ pour déterminer laquelle des
estimations que nous avons trouvées donne la meilleure approximation de
l'intégrale; c'est-à-dire, de l'aire sous la courbe $y=f(x) = 1+x^2$
pour $-1 \leq x \leq 2$.

Pour $n=3$, nous avons $\Delta x = (2-(-1))/3=1$ et
$x_i = -1 + i \Delta x = -1+i$ pour $i=0$, $1$, $2$ et $3$.
Nous obtenons les trois intervalles $[-1,0]$, $[0, 1]$ et $[1, 2]$.

Pour $n=6$, nous avons $\Delta x = (2-(-1))/6=1/2$ et
$x_i = -1 + i \Delta x = -1+i/2$ pour $i=0$, $1$, $2$, \ldots, $6$.
Nous obtenons les six intervalles $[-1, -0.5]$, $[-0.5, 0]$, \ldots,
$[1.5, 2]$.

La somme à droite avec $n=3$ est
\[
D_3 = ( f(0) + f(1) + f(2) ) (1) = 8 \ .
\]
Donc $\displaystyle \int_{-1}^2 (1+x^2) \dx{x} \approx D_3 = 8$.

La somme à droite avec $n=6$ est
\[
D_6 = ( f(-0.5) + f(0) + f(0.5) + f(1) + f(1.5) + f(2) ) (0.5) = 6.875 \ .
\]
Donc $\displaystyle \int_{-1}^2 (1+x^2) \dx{x} \approx D_6 = 6.875$.

La somme à gauche avec $n=3$ est
\[
G_3 = ( f(-1) + f(0) + f(1) ) (1) = 5 \ .
\]
Donc $\displaystyle \int_{-1}^2 (1+x^2) \dx{x} \approx G_3 = 5$.

La somme à gauche avec $n=6$ est
\[
G_6 = ( f(-1) + f(-0.5) + f(0) + f(0.5) + f(1) + f(1.5) )(0.5) = 5.375 \ .
\]
Donc $\displaystyle \int_{-1}^2 (1+x^2) \dx{x} \approx G_6 = 5.375$.

La somme pour le point milieu avec $n=3$ est
\[
M_3 = ( f(-0.5) + f(0.5) + f(1.5) ) (1)= 5.75 \ .
\]
Donc $\displaystyle \int_{-1}^2 (1+x^2) \dx{x} \approx S_3 = 5.75$.

La somme pour le point milieu avec $n=6$ est
\[
M_6 = ( f(-0.75) + f(-0.25) + f(0.25) + f(0.75) + f(1.25) + f(1.75) )(0.5)
= 5.9375 \ .
\]
Donc $\displaystyle \int_{-1}^2 (1+x^2) \dx{x} \approx S_6 = 5.9375$.

La meilleure approximation est donnée par la somme $M_6$.

\PDFfig{7_integrales/numcomp1}{Figure associée à
l'exemple~\ref{EGNumComp1}}{Figure utilisée pour comparer l'efficacité
des sommes à droite, des sommes à gauche et des sommes pour le point
milieu à l'exemple~\ref{EGNumComp1}}{FigNumComp1}

Considérons le dessin à gauche qui est donné à la
figure~\ref{FigNumComp1}.  Il représente le graphe de $f$ entre $x_i$
et $x_{i+1}$.  L'aire de la région $E_D$ (en rose) représente l'erreur
entre la valeur de $\displaystyle \int_{x_i}^{x_{i+1}} f(x) \dx{x}$ et
l'aire du rectangle donné par la somme à droite.  L'aire de la région
$E_G$ (en bleu) représente l'erreur entre la valeur de
$\displaystyle \int_{x_i}^{x_{i+1}} f(x) \dx{x}$ et l'aire du
rectangle donné par la somme à gauche.  Comme $f$ est positive,
croissante et convexe, nous avons que l'aire de la région $E_D$ est plus
grande que l'aire de la région $E_G$.  La somme à gauche donne donc
une meilleure estimation de la valeur de
$\displaystyle \int_{x_i}^{x_{i+1}} f(x) \dx{x}$ que la somme à droite.

Considérons maintenant le dessin à droite qui est donné à la
figure~\ref{FigNumComp1}.  Il représente le graphe de $f$ entre $x_i$
et $x_{i+1}$.  L'aire de la région $E_M$ (en vert) représente l'erreur
entre la valeur de $\displaystyle \int_{x_i}^{x_{i+1}} f(x) \dx{x}$ et
l'aire du rectangle \mbox{\rectangle$x_iAGx_{i+1}$} donné par la somme
pour le point milieu.  Remarquons que l'aire du rectangle
\mbox{\rectangle$x_iAGx_{i+1}$} est égale à l'aire du trapèze
\mbox{\trapeze$x_iCFx_{i+1}$} car les triangles $\triangle ACE$ et
$\triangle GFE$ sont congruents puisque
$|\overline{AE}| = |\overline{EG}|$, $|\overline{CE}| =
|\overline{EF}|$ et $\angle AEC = \angle GEF$.

Comme l'aire du rectangle \mbox{\rectangle$x_iBHx_{i+1}$}
donné par la somme à gauche est plus petite que l'aire du rectangle
\mbox{\rectangle$x_iAGx_{i+1}$} donné par la somme pour le point
milieu qui elle est plus petite que l'aire sous la courbe
$y=f(x)$ pour $x_i \leq x \leq x_{i+1}$, nous en concluons que la somme
pour le point milieu donne une meilleure estimation de la valeur de
$\displaystyle \int_{x_i}^{x_{i+1}} f(x) \dx{x}$ que la somme à gauche.

En conclusion, quand $f$ est positive, croissante et convexe, la somme
pour le point milieu donne la meilleure approximation de l'intégrale
de $f$.

Quand $f$ est positive, décroissante et convexe, Un raisonnement
semblable montre que la somme pour le point milieu donne encore la
meilleure approximation de l'intégrale de $f$.  La seule différence
dans ce cas est que la somme à droite donne une meilleure estimation
de l'intégrale de $f$ que la somme à gauche.  Nous invitons le lecteur a
vérifier cette conclusion.

De plus, nous déduisons du graphe de $f$ à la figure~\ref{FigNumComp1}
que, dans le cas d'une fonction convexe, l'aire du rectangle donné par
la somme pour le point milieu est toujours inférieure à
$\displaystyle \int_{x_i}^{x_{i+1}} f(x) \dx{x}$.  La somme pour
le point milieu est donc une sous-estimation de la valeur de l'intégrale.

En conséquence, puisque $M_3 < M_6$, nous avons que $M_6$ est la meilleure
approximation de l'intégrale.

{\bfseries Il y a cependant une contrainte majeure à la justification
que nous venons de donner}.  Notre justification est basée sur la
propriété que la fonction ne change pas de signe, de croissante à
décroissante et vice-versa, ou de courbure dans un intervalle
$[x_i,x_{i+1}]$.
\label{EGNumComp1}
\end{egg}

\subsection{Propriétés de l'intégrale définie}

Nous déduisons du deuxième item de la remarque~\ref{IDadgm} ci-dessus que
si $f(x)\geq 0$ pour tout $x$ dans l'intervalle $[a,b]$, alors
\[
\displaystyle \int_a^b f(x)\dx{x} = A
\]
où $A$ est l'aire de la région $R$ en dessous de la courbe $y=f(x)$,
au-dessus de l'axe des $x$, et entre $x=a$ et $x=b$.  Par contre, si
$f(x)\leq 0$ pour tout $x$ dans l'intervalle $[a,b]$, alors 
\[
\displaystyle \int_a^b f(x)\dx{x} = -A
\]
où $A$ est l'aire de la région $R$ au-dessus de la courbe $y=f(x)$,
en dessous de l'axe des $x$, et entre $x=a$ et $x=b$.  Si $f$ change
de signe à un seul point $c$ de l'intervalle $[a,b]$ comme c'est le
cas pour la fonction $f$ dont le graphe est donné à la
figure~\ref{AREA8}, alors
\[
\displaystyle \int_a^b f(x)\dx{x} = A_1 - A_2 \; ,
\]
où $A_1$ est l'aire de la région $R_1$ et $A_2$ est l'aire de la
région $R_2$.  La région $R_1$ est la région en dessous de la courbe
$y=f(x)$, au-dessus de l'axe des $x$, et entre $x=a$ et $x=c$.
La région $R_2$ est la région au-dessus de la courbe $y=f(x)$,
en dessous de l'axe des $x$, et entre $x=c$ et $x=b$.

\PDFfig{7_integrales/area8}{L'intégrale d'une fonction $f$ qui change de
signe à un seul point de son domaine d'intégration représente la
différence de deux aires}{L'intégrale d'une fonction $f$ qui change de
signe à un seul point $c$ comme c'est le cas ci-dessus donne l'aire de la
région $R_1$ moins l'aire de la région $R_2$.}{AREA8}

Il est parfois facile de calculer l'intégrale d'une fonction si cette
fonction et l'intervalle d'intégration possèdent certaines symétries.
Par exemple, si $f$ est une fonction impaire (i.e. $f(-x) = -f(x)$
pour tout $x$), alors
\[
\int_{-a}^a f(x) \dx{x} = 0
\]
La justification de ce résultat est fournie par le graphe de la
fonction $f$ donné à la figure~\ref{IMPAIR_INT}.  Soit $A_1$ l'aire de la
région $R_1$ au-dessus de la courbe $y=f(x)$, en dessous de l'axe des
$x$, et entre les droites $x=-a$ et $x=0$.  Soit $A_2$ l'aire de la
région $R_2$ en dessous de la courbe $y=f(x)$, au-dessus de l'axe des
$x$, et entre les droites $x=0$ et $x=a$.  Si $f$ est une fonction
impaire $A_1 = A_2$ et
\[
\int_{-a}^a f(x)  \dx{x} = A_2 - A_1 = 0 \; .
\]

Si $f$ est une fonction paire (i.e. $f(-x) = f(x)$ pour tout $x$),
alors
\[
\int_{-a}^a f(x) \dx{x} = 2 \int_0^a f(x) \dx{x} \; .
\]
La justification de ce résultat est fournie par le graphe de la
fonction $f$ donné à la figure~\ref{PAIR_INT}.  Soit $A_1$ l'aire de
la région $R_1$ en dessous
de la courbe $y=f(x)$, au-dessus de l'axe des $x$, et entre les
droites $x=-a$ et $x=0$.  Soit $A_2$ l'aire de la région $R_2$
en dessous de la courbe $y=f(x)$, au-dessus de l'axe des $x$, et entre
les droites $x=0$ et $x=a$.  Si $f$ est une fonction paire alors
$A_1 = A_2$ et 
\[
\int_{-a}^a f(x)  \dx{x} = A_2 + A_1 = 2A_1 = 2 \int_0^a f(x)\dx{x} \; .
\]

Nous insistons sur le fait que, dans les deux cas précédents, la borne
supérieure d'intégration est la réflexion par rapport à l'origine de
la borne inférieure d'intégration.

\PDFfig{7_integrales/impair_int}{Graphe d'une fonction impaire sur un
domaine symétrique par rapport à l'origine}{Graphe d'une fonction
impaire sur un domaine $[-a,a]$ symétrique par rapport à
l'origine.  Les régions $R_1$ et $R_2$ ont la même aire.}{IMPAIR_INT} 

\PDFfig{7_integrales/pair_int}{Graphe d'une fonction paire sur un
domaine symétrique par rapport à l'origine}{Graphe d'une fonction
paire sur un domaine $[-a,a]$ symétrique par rapport à
l'origine.  Les régions $R_1$ et $R_2$ ont la même aire.}{PAIR_INT} 

\begin{defn}
Si l'intégrale de $f$ de $a$ à $b$ existe, nous définissons
l'intégrale de $f$ de $b$ à $a$ par
\[
\int_b^a f(x)  \dx{x} = - \int_a^b f(x)  \dx{x} \; .
\]
\end{defn}

Les propriétés suivantes découlent de la définition de l'intégrale:

\begin{theorem}
Soit $a$, $b$ et $c$ trois nombres réels.  Si $f$ et $g$ sont deux
fonctions intégrables sur un intervalle qui contient $a$, $b$ et $c$, et si
$k$ est un nombre réel, alors 
\begin{align*}
\int_a^b k\, f(x)\dx{x} &= k \int_a^b f(x) \dx{x} \; , \\
\int_a^b (f(x) + g(x))\dx{x} &= \int_a^b f(x)\dx{x}
+ \int_a^b g(x)\dx{x}
\intertext{et}
\int_a^b f(x)\dx{x} &= \int_a^c f(x)\dx{x} + \int_c^b f(x)\dx{x} \; .
\end{align*}
\end{theorem}

Le graphe de la fonction $f$ donné à La figure~\ref{AREA8} fournit une
interprétation graphique de la dernière règle du théorème précédent.

\subsection{Évaluations des intégrales définies \eng}

\begin{egg}
Utilisons la définition de l'intégrale définie pour calculer
l'aire de la région $R$ bornée par la courbe $y=f(x) = x^2$ et les
droites $x=1$, $x=3$ et $y=0$ (figure~\ref{AREA7}).

Soit $k$ une entier positif.  Nous avons $\Delta x = 2/k$ et
$x_i = 1 + 2i/k$ pour $i=0$, $1$, $2$, \ldots, $k$.  Les intervalles
sont de la forme
\[
[x_i,x_{i+1}] = \left[1+ \frac{2i}{k}, 1+ \frac{2(i+1)}{k}\right]
\]
pour $i=0$, $1$, $2$, \ldots, $k-1$.  Pour chaque intervalle
$[x_i, x_{i+1}]$, nous choisissons $x_i^\ast$ comme étant la limite à gauche
de l'intervalle; c'est-à-dire,
\[
x_i^\ast = x_i = 1+ \frac{2i}{k}
\]
pour $i=0$, $1$, $2$, \ldots, $k-1$.

La somme à gauche $G_k$ est
\begin{align*}
G_k &= \sum_{i=0}^{k-1} f(x_i^\ast) \, \Delta x 
= \sum_{i=0}^{k-1} \left(1+\frac{2i}{k}\right)^2 \frac{2}{k}
= \sum_{i=0}^{k-1} \left(1 + \frac{4i}{k} + \frac{4 i^2}{k^2} \right)
\frac{2}{k} \\
&= \frac{2}{k} \underbrace{\left( \sum_{i=0}^{k-1} 1 \right)}_{=k}
+ \frac{8}{k^2} \underbrace{\left( \sum_{i=0}^{k-1} i \right)}_{=k(k-1)/2}
+ \frac{8}{k^3}
\underbrace{\left( \sum_{i=0}^{k-1} i^2 \right)}_{=k(k-1)(2k-1)/6} \\
&= 2 + \frac{4(k-1)}{k} + \frac{4(k-1)(2k-1)}{3k^2} \\
&= \frac{26}{3} - \frac{16}{3k} + \frac{4}{3k^2} \; .
\end{align*}
Ainsi,
\[
\int_1^3 x^2\dx{x} = \lim_{k\rightarrow \infty} G_k
= \lim_{k\rightarrow \infty}
\left(\frac{26}{3} - \frac{16}{3k} + \frac{4}{3k^2}\right) =
\frac{26}{3} \; .
\]
\label{egg_riemann}
\end{egg}

\PDFfig{7_integrales/area7}{L'aire de la région bornée par $y=x^2$,
l'axe des $x$, et les droites $x=1$ et $x=3$}{L'aire de la région $R$
définie à l'exemple~\ref{egg_riemann} est donné par l'intégrale de
$f(x)=x^2$ pour $1 \leq x \leq 3$}{AREA7}

\begin{egg}
Utilisons la définition de l'intégrale définie pour calculer
$\displaystyle \int_{-2}^2 (4-x^2) \dx{x}$.

Soit $k$ une entier positif.  Nous avons $\Delta x = 4/k$ et
$x_i = -2 + 4i/k$ pour $i=0$, $1$, $2$, \ldots, $k$.  Les intervalles
sont de la forme
\[
[x_i,x_{i+1}] = [-2+ \frac{4i}{k}, -2+ \frac{4(i+1)}{k}]
\]
pour $i=0$, $1$, $2$, \ldots, $k-1$.  Pour chaque intervalle
$[x_i, x_{i+1}]$, nous choisissons $x_i^\ast$ comme étant la limite à gauche
de l'intervalle; c'est-à-dire,
\[
x_i^\ast = x_i = -2+ \frac{4i}{k}
\]
pour $i=0$, $1$, $2$, \ldots, $k-1$.

La somme à gauche $G_k$ est
\begin{align*}
G_k &= \sum_{i=0}^{k-1} f(x_i^\ast) \, \Delta x 
= \sum_{i=0}^{k-1} \left(4-\left(-2+\frac{4i}{k}\right)^2\right) \frac{4}{k}
= \sum_{i=0}^{k-1} \left(\frac{-16i^2}{k^2}  + \frac{16i}{k}\right)
\frac{4}{k} \\
&= \frac{-64}{k^3} \sum_{i=0}^{k-1} i^2 + \frac{64}{k^2} \sum_{i=0}^{k-1} i
= \frac{-64}{k^3} \left(\frac{k(k-1)(2k-1)}{6} \right)
+ \frac{64}{k^2} \left(\frac{k(k-1)}{2} \right) \\
&= -\frac{64}{3} + \frac{32}{k} - \frac{32}{3k^2}
+ 32 - \frac{32}{k}  = \frac{32}{3} - \frac{32}{3k^2}
\end{align*}
Ainsi,
\[
\int_{-2}^2 (4-x^2)\dx{x} = \lim_{k\rightarrow \infty} G_k
= \lim_{k\rightarrow \infty}
\left(\frac{32}{3} - \frac{32}{3k^2}\right) =
\frac{32}{3} \; .
\]
\end{egg}

\begin{egg}
Quelle est la valeur de la limite suivante?
\[
\lim_{n\rightarrow \infty} \sum_{i=1}^n \frac{i^3}{n^4} \; .
\]

Remarquons que la somme
\[
\sum_{i=1}^n \frac{i^3}{n^4} = \sum_{i=1}^n \left(\frac{i}{n}\right)^3
\, \frac{1}{n}
\]
est une somme de Riemann à droite pour l'intégrale
$\displaystyle \int_0^1 x^3 \dx{x}$.
En effet, soit $\displaystyle \Delta x = \frac{1}{n}$ et
$\displaystyle x_i = i \Delta x = \frac{i}{n}$ pour $i=0$, $1$, $2$,
\ldots, $n$.  Nous obtenons la somme à droite suivante pour
$\displaystyle \int_0^1 x^3 \dx{x}$.
\[
\sum_{i=1}^n x_i^3 \, \Delta x
= \sum_{i=1}^n \left(\frac{i}{n}\right)^3 \, \frac{1}{n} \; .
\]
Donc
\[
\lim_{n\rightarrow \infty} \sum_{i=1}^n \frac{i^3}{n^4}
= \int_0^1 x^3 \dx{x} \; .
\]
Nous allons voir plus loin comment calculer cette intégrale sans faire
appel aux sommes de Riemann.
\end{egg}

\begin{egg}
Quelle est la valeur de la limite suivante?
\[
\lim_{n\rightarrow \infty} \sum_{i=1}^n \left( \frac{2}{n^{1/2}}
+ \frac{2i}{n^{3/2}}\right)^2 \; .
\]

Remarquons que la somme
\[
\sum_{i=1}^n \left( \frac{2}{n^{1/2}} + \frac{2i}{n^{3/2}}\right)^2
= \sum_{i=1}^n \frac{1}{2} \left( 2 + \frac{2i}{n}\right)^2 \, \frac{2}{n} \;,
\]
est une somme de Riemann à droite pour l'intégrale
$\displaystyle \int_2^4 \frac{x^2}{2} \dx{x}$.
En effet, soit $\displaystyle \Delta x = \frac{2}{n}$ et
$\displaystyle x_i = 2 + i \Delta x = 2 + \frac{2i}{n}$ pour
$i=0$, $1$, $2$, \ldots, $n$.  Nous obtenons la somme à droite suivante
pour $\displaystyle \int_2^4 \frac{x^2}{2} \dx{x}$.
\[
\sum_{i=1}^n \frac{x_i^2}{2} \, \Delta x
= \sum_{i=1}^n \frac{1}{2} \left( 2 + \frac{2i}{n}\right)^2 \, \frac{2}{n}
 \; .
\]
Donc
\[
\lim_{n\rightarrow \infty} \sum_{i=1}^n \left( \frac{2}{n^{1/2}}
+ \frac{2i}{n^{3/2}}\right)^2 = \int_2^4 \frac{x^2}{2} \dx{x} \; .
\]
\end{egg}

\subsection{Déplacement}\label{deplace}

Le problème que nous considérons dans la présente section est
classique.  Nous montrons comment l'intégrale peut être utilisée pour
déterminer la distance parcourue par une voiture si nous connaissons sa
vitesse en fonction du temps.

Soit $v(t)$, la vitesse en km/h d'une voiture au temps $t$ en heures.
Si nous voulons estimer la distance parcourue par la voiture entre $t=a$ et
$t=b$ heures, nous divisons l'intervalle de temps $[a,b]$ en (petits)
sous-intervalles de même longueur et nous supposons que la vitesse de la
voiture sur chacun des sous-intervalles de temps est constante.

Soit $k>0$.  Posons $\Delta t = (b-a)/k$ et $t_0 = a$,
$t_1 = a + \Delta t$, $t_2 = a + 2\,\Delta t$, $t_3 = a + 3\,\Delta t$,
\ldots, $t_k = a + k\,\Delta t = b$.  Si $v(t_i^\ast)$ est la vitesse
de la voiture mesurée au temps $t_i^\ast$ dans l'intervalle de temps
$[t_i, t_{i+1}]$ et si nous supposons que la voiture a voyagé à une
vitesse constante durant cet intervalle de temps, alors la distance
parcourue durant cet intervalle de temps est $v(t_i^\ast)\, \Delta t$
km.  Notons que la longueur de l'intervalle de temps $\Delta t$ est
en heures et la vitesse $v(t)$ est en km/h, ainsi
$v(t_i^\ast)\, \Delta t$ est bien en kilomètres. 

Si nous faisons cela pour chaque sous-intervalle de la forme
$[t_i,t_{i+1}]$, nous trouvons que la distance parcourue par la voiture
entre $a$ et $b$ heures est approximativement
\[
\sum_{i=0}^{k-1} v(t_i^\ast)\,\Delta t \; .
\]
C'est une somme de Riemann pour l'intégrale de la fonction $v$ de $a$
à $b$.  Si nous laissons $k$ tendre vers l'infini, nous obtenons que la
distance parcourue entre $a$ et $b$ heures est
\begin{equation} \label{conject_FTC}
p(b)-p(a) = \int_a^b v(t)\dx{t}
\end{equation}
où $p(t)$ est la position de la voiture au temps $t$.

La vitesse moyenne $v_m$ de la voiture entre $a$ et $b$ heures est
\[
v_m = \frac{p(b)-p(a)}{b-a} = \frac{1}{b-a}\int_a^b v(t) \dx{t} \; .
\]

\begin{rmk}
Puisque $p$ est une primitive de $v$ car $p'(t)=v(t)$ pour tout $t$,
l'équation (\ref{conject_FTC}) suggère la méthode suivante pour
évaluer l'intégrale définie d'une fonction $f$.   Si $F$ est une
primitive de $f$, alors
\[
\int_a^b f(x) \dx{x} = F(b) - F(a) \ .
\]
Est-ce vrai pour toute fonction $f$?  Nous donnerons une réponse
affirmative à cette question pour les fonctions continues $f$ lorsque
nous présenterons le Théorème fondamental du calcul.
\label{preTFC}
\end{rmk}

\PDFfig{7_integrales/toine}{La vitesse en fonction du temps pour les
voitures d'Antoine et d'Antoinette}{La vitesse en fonction du temps
pour les voitures d'Antoine et d'Antoinette}{TOINE}

\begin{egg}
Antoine et Antoinette travaillent tous les deux sur la colline
parlementaire à Ottawa.  Comme ils travaillent pour des partis
politiques différents, ils ne se connaissent pas.  Ils ont quand même
un intérêt commun pour la peinture et décident d'aller à Montréal pour
visiter le Musée d'Arts Contemporain.  Ils partent tous les deux pour
Montréal après leur journée d'ouvrage du vendredi.  Antoine part à
17h00 et Antoinette à 18h00.  Ils arrivent au Musée d'Arts
Contemporain en même temps.

Nous retrouvons à la figure~\ref{TOINE} le graphe de la vitesse en
fonction du temps pour la voiture d'Antoinette ainsi celui pour la
voiture d'Antoine lors de leur voyage pour se rendre à Montréal.
$v_1(t)$ est la vitesse de la voiture d'Antoine au temps $t$ en heures
et $v_2(t)$ est la vitesse de la voiture d'Antoinette au temps $t$ en
heures.  Vérifiez qu'ils parcourent $210$ km chacun.

\subQ{a} Quelle distance a parcouru Antoine quand Antoinette part?

La distance parcourue par Antoine après la première heure est
l'intégrale de la vitesse $v_1$ de $t=17$ à $t=18$.  Cette intégrale
correspond à l'aire sous la courbe $y=v_1(t)$ pour $17\leq t\leq 18$.
Nous trouvons $55$ km.

\subQ{b} À quel moment la distance entre Antoine et Antoinette est la
plus grande?  Quelle est cette distance?

Le seul moment où la distance entre Antoine et Antoinette peut passer
de croissante à décroissante et vice-versa est lorsque la voiture
d'Antoine et la voiture d'Antoinette vont à la même vitesse.  Cela
arrive à $t=18.25$ (i.e. 18h15) et $t=20.25$ (i.e. 20h15).

Avant $t=18.25$, Antoine va plus vite que Antoinette.  À $t=18.25$, la
distance parcourue par Antoine est l'aire sous la courbe $y=v_1(t)$
pour $17\leq t \leq 18.25$, soit $78.75$ km, et la distance parcourue
par Antoinette est l'aire sous la courbe $y=v_2(t)$ pour $18\leq t
\leq 18.25$, soit $21.25$ km.  Antoine a donc $78.75-21.25=57.5$
km. d'avance sur Antoinette.

Entre $t=18.25$ et $t=20.25$, la voiture d'Antoinette va plus vite que
celle d'Antoine et Antoinette gagne du terrain sur Antoine.

À $t=20.25$, la distance parcourue par Antoine est l'aire sous
la courbe $y=v_1(t)$ pour $17\leq t \leq 20.25$, soit $196.25$ km, et
la distance parcourue par Antoinette est l'aire sous la courbe
$y=v_2(t)$ pour $18\leq t \leq 20.25$, soit $198.75$ km.
Antoinette a maintenant $198.75-196.25= 2.5$ km. d'avance sur
Antoine.

Entre $t=20.25$ et $t=20.50$, la voiture d'Antoine va plus vite que
celle d'Antoinette et Antoine gagne du terrain sur Antoinette pour
arriver au Musée d'Arts Contemporain en même temps que Antoinette.

La plus grande distance entre la voiture d'Antoine et d'Antoinette est
donc de $57.5$ km.

\subQ{c} À quel moment Antoinette rattrape-t-elle Antoine?

Si $T$ est le temps où Antoinette rattrape Antoine, il faut que l'aire
sous la courbe $y=v_1(t)$ pour $17\leq t \leq T$ soit égale à l'aire
sous la courbe $y=v_2(t)$ pour $18\leq t \leq T$.

À $T = 20$ (i.e. 20h00), Antoine et Antoinette ont parcouru $185$ km.
Après 20h00, Antoinette devance Antoine.

L'histoire ne dit pas si Antoine et Antoinette se sont rencontrés au
Musée d'Arts Contemporain.
\end{egg}

\subsection{L'intégrale de Riemann (Stieljes) \theory} \label{Riemann_int}

La définition~\ref{def_non_rig} de l'intégrale est valable pour les
fonctions continues par morceaux.  Le fait que l'intégrande soit une
fonction continue par morceaux permet de démontrer que la limite en
(\ref{def_int_cpm}) ne dépend pas du choix de partitions de
l'intervalle d'intégration et du choix des points $x_i^\ast$.

La définition de l'intégrale définie d'une fonction que nous donnons
dans cette section généralise celle donnée à la définition~\ref{def_non_rig}.

Nous créditons Riemann (et Stieljes) pour la définition de l'intégrale
définie que voici.

Soit $f$, une fonction bornée sur l'intervalle fermé $[a,b]$.

Soit ${\cal P} = \{x_0, x_1, x_2, \ldots, x_k\}$, une partition de
l'intervalle $[a,b]$.  C'est-à-dire que
$a=x_0 < x_1 < x_2 < x_3 < \ldots < x_k = b$.

Pour $0\leq i < k$, nous posons $\Delta x_i = x_{i+1}-x_i$,
$m_i = \inf_{x\in[x_i,x_{i+1}]} f(x)$ et\\
$M_i = \sup_{x\in[x_i,x_{i+1}]} f(x)$.

$\Delta x_i$ est la longueur de l'intervalle $[x_i, x_{i+1}]$.
$m_i$ est la plus grande valeur telle que $m_i\leq f(x)$ pour tout
$x \in [x_i, x_{i+1}]$ et $M_i$ est la plus petite valeur telle que 
$M_i \geq f(x)$ pour tout $x \in [x_i, x_{i+1}]$.

La {\bfseries somme inférieure} \index{Somme inférieure} pour $f$ sur
l'intervalle $[a,b]$ associée à $\cal P$ est
\[
L_{\cal P} = \sum_{j=0}^{k-1} m_i \, \Delta x_i
\]
(figure~\ref{SUMINF}) et la {\bfseries somme supérieure}
\index{Somme supérieure} pour $f$ sur l'intervalle $[a,b]$ associée à
$\cal P$ est 
\[
U_{\cal P} = \sum_{j=0}^{k-1} M_i \, \Delta x_i
\]
(figure~\ref{SUMSUP}).

Soit
\begin{align*}
L = \sup \{ L_{\cal P} : {\cal P} \text{ est une partition de l'intervalle }
[a,b]\}
\intertext{et}
U = \inf \{ U_{\cal P} : {\cal P} \text{ est une partition de l'intervalle }
[a,b]\} \; .
\end{align*}

Si $U=L$, nous disons que {\bfseries $\mathbf f$ est intégrable au sens de
Riemann sur l'intervalle} $[a,b]$ et nous écrivons
\[
\int_a^b f(x) \dx{x} = L = U \; .
\]
\index{Intégrable au sens de Riemann}

Les fonctions continues par morceaux sur l'intervalle $[a,b]$ sont
intégrables au sens de Riemann et la formule donnée en
(\ref{def_int_cpm}) est satisfaite.

Nous utilisons la formule (\ref{def_int_cpm}) pour calculer les
intégrales définies de fonctions continues par morceaux car elle est
plus simple que celle que nous venons de donner ci-dessus.

\PDFfig{7_integrales/suminf}{Une somme inférieure pour une fonction $f$
sur l'intervalle ${[a,b]}$}{Une somme inférieure pour une fonction $f$
sur l'intervalle ${[a,b]}$}{SUMINF}

\PDFfig{7_integrales/sumsup}{Une somme supérieure pour une fonction $f$
sur l'intervalle ${[a,b]}$}{Une somme supérieure pour une fonction $f$
sur l'intervalle ${[a,b]}$}{SUMSUP}

\section{Théorème fondamental du calcul}

Dans cette section, nous verrons comment le calcul des intégrales
indéfinies peut quelques fois être utilisé pour évaluer les intégrales
définies. Par contre, dans les applications réelles, les intégrales
définies sont plus fréquemment évaluées avec un ordinateur à partir de
la définition de l'intégrale définie ou d'un algorithme qui découle de
cette définition.  Nous verrons quelques algorithmes pour évaluer
numériquement les intégrales définies dans une prochaine section.

\subsection{Première version du théorème fondamental du calcul}

À la section~\ref{deplace}, nous avons remarqué que pour évaluer la
distance parcourue par une voiture entre $t=a$ et $t=b$ heures, il
suffisait de calculer l'intégrale définie de la vitesse $v(t)$ de la
voiture au temps $t$ de $t=a$ à $t=b$.  De plus, à la
remarque~\ref{preTFC}, nous avons aussi remarqué que nous pouvions
remplacer le calcul de l'intégrale avec les sommes de Riemann par la
valeur $p(b)-p(a)$ où $p(t)$ est la primitive de la vitesse $v(t)$;
c'est-à-dire, $p'(t) = v(t)$.  C'est essentiellement l'énoncé du
théorème suivant.

\begin{theorem}[Théorème fondamental du calcul]
\index{Théorème fondamental du calcul} 
Si $f$ est une fonction continue sur l'intervalle fermé $[a,b]$ et si
$F$ est une fonction différentiable sur $[a,b]$ telle que $F'=f$
sur $[a,b]$  (i.e. $F$ est une primitive de $f$), alors
\[
\int_a^b f(x)  \dx{x} = F(x)\bigg|_{x=a}^{b} \equiv F(b) - F(a) \; .
\]
\end{theorem}

\begin{rmk}
Nous pouvons motiver le théorème précédent de la façon suivante.

Soit $a=x_0 < x_1 < x_2 < \ldots < x_k = b$ une partition ${\cal P}$
de l'intervalle $[a,b]$ en sous-intervalles $[x_i, x_{i+1}]$ de
longueur $\Delta x = (b-a)/k$.  Grâce au théorème de la moyenne, il existe
$x_i^\ast$ entre $x_i$ et $x_{i+1}$ tel que
\[
\frac{F(x_{i+1})-F(x_i)}{x_{i+1}-x_i} = F'(x_i^\ast) = f(x_i^\ast)
\]
pour $i=0$, $1$, \ldots, $k-1$.  Ainsi,
\[
F(x_{i+1})-F(x_i) = f(x_i^\ast) (x_{i+1}-x_i) = f(x_i^\ast) \Delta x
\]
pour $i=0$, $1$, \ldots, $k-1$.  Nous obtenons la somme de Riemann
\begin{align*}
S_k &= \sum_{i=0}^{k-1} f(x_i^\ast) \Delta x
= \sum_{i=0}^{k-1} \left( F(x_{i+1})-F(x_i) \right) \\
&= \left( F(x_1)-F(x_0) \right) + \left( F(x_2)-F(x_1) \right)
+ \left( F(x_3)-F(x_2) \right) + \ldots + \left( F(x_k)-F(x_{k-1}) \right) \\
&= F(x_k) - F(x_0) = F(b) - F(a) \ .
\end{align*}
Pour $k$ suffisamment grand, nous avons
\[
\int_a^b f(x) \dx{x} \approx S_k = F(b) - F(a) \ .
\]
C'est essentiellement l'idée de la démonstration du Théorème
fondamental du calcul.
\end{rmk}

\begin{rmk}
Le théorème fondamental du calcul est indépendant de la primitive $F$
qui est utilisée dans la formule
\[
\int_a^b f(x)  \dx{x} = F(b) - F(a) \; .
\]
Si $G$ est une autre primitive de $f$, nous avons vu qu'il existait une
constante $C$ telle que $G(x) = F(x) + C$ pour tout $x$.  Ainsi,
\[
\int_a^b f(x)  \dx{x} = G(b) - G(a)
\]
car $G(b) - G(a) = (F(b)+C)- (F(a)+C) = F(b)-F(a)$.
\end{rmk}

\begin{egg}
Si $f$ est la {\bfseries distribution de densité} d'une population en
fonction de l'âge en années, alors
\[
\int_\alpha^\beta f(t)\dx{t}
\]
est le pourcentage de la population âgée entre $\alpha$ et $\beta$
ans.  Si $T$ est l'âge maximal des individus de cette population, alors
\[
\int_0^T f(t)\dx{t} = 1 \; .
\]
C'est-à-dire, $100$\% de la population est âgé de $0$ à $T$ ans.

Si nous supposons que l'âge maximum des individus d'une population animale
est $10$ ans et que la distribution de densité de cette population en
fonction de l'âge est donnée par $f(t) = 0.006\,t(10-t)$ pour
$0\leq t \leq 10$ années, quel est le pourcentage de la population
entre $2$ et $7$ ans?

Le pourcentage de la population entre $2$ et $7$ ans est donné par
\[
\int_2^7 0.006\,t(10-t)\dx{t} \; .
\]
Pour évaluer cette intégrale, il faut trouver une primitive de
$f(t) = 0.006\, t(10-t)$. 
\[
\int 0.006\,t(10-t)\dx{t}
= 0.06 \int t  \dx{t} - 0.006 \int t^2\dx{t}
= 0.03 t^2 - 0.002 t^3 + C \; .
\]
Puisque nous avons besoin d'une seule primitive, nous pouvons prendre $C=0$
Donc
\[
\int_2^7 0.006\,t(10-t)\dx{t} =
\left(0.03\, t^2 - 0.002\, t^3\right)\bigg|_{t=2}^7 
= 0.784 - 0.104 = 0.68
\]
et $68$\% de la population animale est entre $2$ et $7$ ans.
\end{egg}

\begin{rmk}
À l'exemple précédent, l'intégrale de la fonction de densité de $2$ à
$7$ nous donne le pourcentage de la population entre $2$ et $7$ ans.
Est-ce que cela inclus les individus âgés de précisément $2$ et $7$
ans?  En fait, que ceux âgés de précisément $2$ et $7$ ans soient
inclus ou non n'a pas d'importance car il y a probablement aucun
individu âgé de précisément $2$ et $7$ ans.  Quelle est la chance
qu'un individu de l'espèce animale soit âgé de précisément $2$ ou $7$
ans?  Mathématiquement, il n'y a aucune chance. Peut-être que
l'individu aura deux ans et une seconde, ou deux ans et un centième de
seconde, mais la chance qu'il est exactement $2$ ans est
nulle.
\end{rmk}

\begin{egg}[\eng]
Le graphe de la dérivée d'une fonction $f$ est donnée ci-dessous.
\PDFgraph{7_integrales/circfp}
Si $f(0)=3$, quelle est la valeur de $f(5)$?

Grâce au théorème fondamental du calcul, nous avons que
\[
f(5) = f(0) + \int_0^5 f'(x) \dx{x} = 3 + \int_0^5 f'(x) \dx{x} \ .
\]
Il faut donc calculer l'aire sous la courbe $y=f'(x)$ pour
$0\leq x \leq 5$.  Or
\[
\int_0^5 f'(x) \dx{x} = \int_0^4 f'(x) \dx{x} + \int_4^5 f'(x) \dx{x} \ .
\]
Puisque $\displaystyle \int_0^4 f'(x) \dx{x}$ représente la moitié de
l'aire du disque de rayon $2$, nous avons que
\[
\int_0^4 f'(x) \dx{x} = \frac{1}{2}\; (2^2 \pi) = 2\pi \ .
\]
Pour évaluer l'intégrale de $f'$ entre $4$ et $5$, nous utilisons la
figure suivante.
\PDFgraph{7_integrales/circfp2}
L'angle $\theta$ est donnée par $\cos(\theta) = 1/2$.  Donc
$\theta=\pi/3$.  Ainsi, l'aire de la région A est un sixième de l'aire
du disque de rayon $2$; c'est-à-dire, l'aire de $A$ est
$2^2 \pi/6 = 2\pi/3$.  L'aire du triangle B est $\sqrt{3}/2$.  Donc
\[
\int_4^5 f'(x) \dx{x} = \frac{2\pi}{3} - \frac{\sqrt{3}}{2} \ .
\]
Nous obtenons
\[
f(5) = 3 + 2\pi + \frac{2\pi}{3} - \frac{\sqrt{3}}{2}
= \frac{8\pi}{3} + \frac{6-\sqrt{3}}{2} \ .
\]
\end{egg}

\begin{egg}
Évaluons les intégrales définies suivantes.
\begin{center}
\begin{tabular}{*{1}{l@{\hspace{1em}}l@{\hspace{5em}}}l@{\hspace{1em}}l}
\subQ{a} &
$\displaystyle \int_4^9 \left(\sqrt{t} + \frac{1}{\sqrt{t}}\right)^2 \dx{t}$ &
\subQ{b} & $\displaystyle \int_0^{1/2} \frac{1}{\sqrt{1-x^2}} \dx{x}$
\end{tabular}
\end{center}

Grâce au théorème fondamental du calcul, nous n'avons plus à utiliser
la définition de l'intégrale définie (et donc les sommes de Riemann)
pour évaluer ces intégrales.

\subQ{a}
\begin{align*}
\int_4^9 \left(\sqrt{t} + \frac{1}{\sqrt{t}}\right)^2 \dx{t}
&= \int_4^9 \left(t + 2 + \frac{1}{t}\right) \dx{t}
= \left( \frac{t^2}{2} + 2t + \ln(t) \right)\bigg|_{t=4}^9 \\
&= \left( \frac{9^2}{2} + 2\times 9 + \ln(9) \right) -
\left( \frac{4^2}{2} + 2\times 4 + \ln(4) \right)
= \frac{85}{2} + 2\ln\left(\frac{3}{2}\right)
\end{align*}
car $t^2/2 + 2t + \ln(t)$ est une primitive de $t+2+1/t$.

\subQ{b}
\[
\int_0^{1/2} \frac{1}{\sqrt{1-x^2}}  \dx{x}  =
\arcsin(x)\bigg|_{x=0}^{1/2}
= \arcsin\left(\frac{1}{2}\right) - \arcsin(0)
= \frac{\pi}{6} - 0 = \frac{\pi}{6}
\]
car $\arcsin(x)$ est une primitive de $1/\sqrt{1-x^2}$.
\end{egg}

La règle de substitution prend une forme particulière dans le contexte
des intégrales définies.

\begin{theorem}
\index{Règle de substitution}\index{Changement de variable}
Si
\begin{enumerate}
\item $g$ est une fonction différentiable (croissante ou décroissante),
\item $F$ est la primitive d'une fonction $f$, et
\item $f$ et $F$ sont définies sur l'image de $[a,b]$ par $g$,
\end{enumerate}
alors
\[
\int_a^b f(g(x))g'(x)  \dx{x} = \int_{g(a)}^{g(b)} f(y)  \dx{y}
= F(g(b)) - F(g(a)) \; .
\]
\label{OneDSubstRule}
\end{theorem}

Pour appliquer la règle de substitution, nous procédons de la façon
suivante.  Si nous posons $y=g(x)$, alors $\dx{y} = g'(x) \dx{x}$ et
\[
\int_a^b f(\,\underbrace{\ g(x) \ }_{=y}\,)
\underbrace{g'(x)  \dx{x}}_{=\dx{y}} = \int_{g(a)}^{g(b)} f(y) \dx{y} \; .
\]
Rappelons que le symbole $\dx{y} = g'(x) \dx{x}$
{\em n'est pas une égalité algébrique}, il indique seulement la
procédure pour remplacer la variable d'intégration $x$ par la variable
d'intégration $y$.

\begin{egg}
Évaluons les intégrales définies suivantes.
\begin{center}
\begin{tabular}{*{2}{l@{\hspace{0.5em}}l@{\hspace{2.5em}}}l@{\hspace{0.5em}}l}
\subQ{a} & $\displaystyle \int_1^3 \frac{x^2}{\sqrt{x^3+5}} \dx{x}$  &
\subQ{b} & $\displaystyle \int_0^{\pi/2} e^{-\cos(\theta)} \sin(\theta)
\dx{\theta}$ &
\subQ{c} & $\displaystyle \int_0^{1/\sqrt{2}} \frac{x}{\sqrt{1-x^4}} \dx{x}$
\\[1em]
\subQ{d} & $\displaystyle \int_0^{\pi^2} \frac{\cos(\sqrt{x})}{\sqrt{x}}
\dx{x}$ &
\subQ{e} & $\displaystyle \int_0^4 \frac{t+7}{\sqrt{5-t}} \dx{t}$
\end{tabular}
\end{center}

\subQ{a} Si $y = x^3+5$, alors $\dx{y} = 3 x^2\dx{x}$, $y=6$ pour
$x=1$ et $y=32$ pour $x=3$.  Donc
\[
\int_1^3 \frac{x^2}{\sqrt{x^3+5}}  \dx{x}
= \frac{1}{3} \int_1^3 \frac{1}{\sqrt{x^3+5}}\,3 x^2 \dx{x}
= \frac{1}{3} \int_6^{32} \frac{1}{\sqrt{y}} \dx{y}
= \frac{2}{3} \sqrt{y} \bigg|_{y=6}^{32} = \frac{2}{3} ( 4\sqrt{2}-\sqrt{6} )
\; .
\]

\subQ{b} Si $y = -\cos(\theta)$, alors
$\dx{y} = \sin(\theta)\dx{\theta}$, $y=-1$ pour $\theta=0$ et
$y=0$ pour $\theta=\pi/2$.  Donc
\[
\int_0^{\pi/2} e^{-\cos(\theta)} \sin(\theta)\dx{\theta}
= \int_{-1}^0 e^y \dx{y}
= e^y \bigg|_{y=-1}^{0} = 1 - e^{-1} \; .
\]

\subQ{c} Si $y = x^2$, alors $\dx{y} = 2x\dx{x}$, 
$y=0$ pour $x=0$ et $y=1/2$ pour $x=1/\sqrt{2}$.  Donc
\begin{align*}
\int_0^{1/\sqrt{2}} \frac{x}{\sqrt{1-x^4}}  \dx{x}
&= \frac{1}{2} \int_0^{1/\sqrt{2}} \frac{1}{\sqrt{1-x^4}}\,2x  \dx{x}
= \frac{1}{2} \int_0^{1/2} \frac{1}{\sqrt{1-y^2}}\dx{y} \\
&= \frac{1}{2} \arcsin(y) \bigg|_{y=0}^{1/2}
= \frac{1}{2} \left(\frac{\pi}{6} - 0\right) = \frac{\pi}{12} \; .
\end{align*}

\subQ{d} Si $y = \sqrt{x}$, alors
$\displaystyle \dx{y} = \frac{1}{2\sqrt{x}}\,\dx{x}$,
$y=0$ pour $x=0$ et $y=\pi$ pour $x=\pi^2$.  Donc
\[
\int_0^{\pi^2} \frac{\cos(\sqrt{x})}{\sqrt{x}} \dx{x}
= 2\int_0^{\pi^2} \cos(\sqrt{x})\,\frac{1}{2\sqrt{x}} \dx{x}
= 2\int_0^{\pi} \cos(y)\dx{y}
=  2\sin(y)\bigg|_{y=0}^{\pi} = 0 \; .
\]

\subQ{e} Si $y = 5-t$, alors $\dx{y} = -\dx{t}$, $t+7 = 12 -y$,
$y=5$ pour $t=0$ et $y=1$ pour $t=4$.  Donc
\begin{align*}
\int_0^4 \frac{t+7}{\sqrt{5-t}}  \dx{t}
&= - \int_0^4 \frac{t+7}{\sqrt{5-t}} \ (-1)  \dx{t}
= -\int_5^1 \frac{12-y}{\sqrt{y}}  \dx{y} \\
&= \int_1^5 \left( 12 y^{-1/2} - y^{1/2} \right)  \dx{y}
= \left( 24 y^{1/2} - \frac{2}{3} y^{3/2} \right) \bigg|_{y=1}^5
= \frac{62}{3} \sqrt{5}  - \frac{70}{3} \; .
\end{align*}
\end{egg}

\begin{rmk}[\eng]
Comme dans le cas des intégrales indéfinies, il est préférable à
l'occasion d'utiliser la règle de substitution dans le sens inverse.
Si nous substituons $x=g(y)$, nous obtenons

\[
\int_a^b f(x) \dx{x} = \int_\alpha^\beta f(g(y))g'(y) \dx{y}
\]
où $a = g(\alpha)$ et $b = g(\beta)$.  Il faut trouver $\alpha$ et
$\beta$ qui donneront les bornes d'intégration 
$a = g(\alpha)$ et $b = g(\beta)$ de l'intégrale de gauche.
\end{rmk}

À l'instar de la règle de substitution, la règle d'intégration par
parties prend la forme suivante.

\begin{theorem}
Si $f:[a,b]\rightarrow \RR$ et $g:[a,b]\rightarrow \RR$ sont deux
fonctions différentiables, alors
\[
\int_a^b f(x)\,g'(x) \dx{x} = f(x)g(x)\bigg|_{x=a}^b
- \int_a^b g(x)\,f'(x) \dx{x} \; .
\]
\end{theorem}

\begin{egg}
Évaluons les intégrales définies suivantes.
\begin{center}
\begin{tabular}{*{1}{l@{\hspace{1em}}l@{\hspace{5em}}}l@{\hspace{1em}}l}
\subQ{a} & $\displaystyle \int_1^3 (5x+2)e^{3x} \dx{x}$ &
\subQ{b} & $\displaystyle \int_0^1 \arctan(t) \dx{t}$
\end{tabular}
\end{center}

\subQ{a} Nous avons $(5x+2) \,e^{3x} = f(x)g'(x)$ où $f(x)=5x+2$ et
$g'(x) = e^{3x}$.  Donc $f'(x) = 5$, $g(x) = e^{3x}/3$ et
\begin{align*}
\int_1^3 (5x+2)e^{3x} \dx{x}
&= \int_1^3 f(x)\,g'(x) \dx{x} = f(x)g(x)\bigg|_{x=1}^3
- \int_1^3 g(x)\,f'(x) \dx{x} \\\
&= \frac{1}{3}(5x+2)e^{3x}\bigg|_{x=1}^3 - \frac{5}{3}\int_1^3 e^{3x}  \dx{x}
= \frac{1}{3}(5x+2)e^{3x}\bigg|_{x=1}^3 - \frac{5}{9} e^{3x} \bigg|_{x=1}^3 \\
&= \frac{17}{3}\,e^9 - \frac{7}{3}\,e^3 - \frac{5}{9}\,e^9 + \frac{5}{9}\,e^3
= \frac{46}{9}\,e^9 - \frac{16}{9}\,e^3 \; .
\end{align*}

\subQ{b} Nous avons $\arctan(t) = f(t)g'(t)$ où $f(t)=\arctan(t)$ et
$g'(t) = 1$.  Donc $f'(t) = 1/(1+t^2)$, $g(t)=t$ et
\begin{align*}
\int_0^1 \arctan(t) \, \dx{t} &= \int_0^1 f(x)\,g'(x) \dx{x}
= f(x)g(x)\bigg|_{x=0}^1 - \int_0^1 g(x)\,f'(x) \dx{x} \\
&= t\arctan(t)\bigg|_{t=0}^{1}
- \int_0^1 \frac{t}{1+t^2} \dx{t}
= \arctan(1)- 0 - \int_0^1 \frac{t}{1+t^2} \dx{t} \\
&= \frac{\pi}{4} - \int_0^1 \frac{t}{1+t^2} \dx{t} \; .
\end{align*}
Si $y=1+t^2$, alors $\dx{y} = 2 t \dx{t}$.  De
plus, $y=1$ lorsque $t=0$ et $y=2$ lorsque $t=1$.  Donc
\[
\int_0^1 \frac{t}{1+t^2} \dx{t}
= \frac{1}{2} \int_0^1 \frac{1}{1+t^2} \, 2t\dx{t}
= \frac{1}{2} \int_1^2 \frac{1}{y}  \dx{y}
= \frac{1}{2} \ln(y)\bigg|_{y=1}^2
= \frac{1}{2} \ln(2) \; .
\]
Finalement,
\[
\int_0^1 \arctan(t) \, \dx{t} = \frac{\pi}{4} - \frac{1}{2} \ln(2) \; .
\]
\end{egg}

\subsection{Deuxième version du théorème fondamental du calcul}

\begin{theorem}[Deuxième version du théorème fondamental du calcul]
\index{Théorème fondamental du calcul, deuxième version}
Si $f$ est une fonction continue par morceaux sur l'intervalle $[a,b]$
alors
\[
F(x) = \int_a^x f(t) \dx{t}  \quad , \quad x\in [a,b] \; ,
\]
est une fonction continue sur l'intervalle $[a,b]$ et $F'(c) = f(c)$
aux points $c$ où $f$ est continue.  En particulier, si $f$ est
continue sur l'intervalle $[a,b]$ alors $F$ est une primitive de $f$;
c'est-à-dire que $F'(x) = f(x)$ pour tout $x$ dans l'intervalle
$[a,b]$.
\end{theorem}

\begin{egg}
Une des plus importantes fonctions en statistique est la fonction de
densité
\[
f(x) = \frac{e^{-(x-\mu)^2/(2\sigma^2)}}{\sigma \sqrt{2\pi}}
\]
où $\mu$ est la moyenne et $\sigma$ est l'écart type.   Tous les
étudiants connaissent la distribution normale des notes.  La fonction
$f$ est la fonction qui gouverne cette distribution.

Une variable $X$ est dite {\bfseries aléatoire} si elle représente le
résultat d'une expérience que nous ne pouvons pas prédire exactement.
Par exemple, nous pouvons supposer que la durée de vie d'un type d'ampoules
électriques est une variable aléatoire.\index{Variable aléatoire}  Si
la durée de vie moyenne d'une ampoule est $400$ heures,
cela veut dire que certaines ampoules dureront un peut plus de $400$
heures et d'autres un peu moins de $400$ heures.  Nous ne pouvons pas
déterminer exactement la durée de vie d'une ampoule spécifique.

Nous disons que la variable aléatoire $X$ possède une
{\bfseries distribution normale}\index{Distribution normale} de
{\bfseries moyenne} $\mu$ et de {\bfseries d'écart type} $\sigma$ si
la probabilité que le résultat $X$ de l'expérience soit entre $a$ et
$b$ est
\[
P(a\leq X \leq b) = \int_a^b f(x) \dx{x} \; .
\]
C'est-à-dire que l'aire de la région bornée par la courbe
$y=f(x)$, l'axe des $x$, et les droites $x=a$ et $x=b$ est la
probabilité que le résultat $X$ de l'expérience soit entre $a$ et
$b$. Nous retrouvons le graphe de la fonction de densité pour
$\mu=400$ et $\sigma =20$ à la figure~\ref{NORMAL_DISTR}.

Par exemple, nous pouvons supposer que la durée de vie d'un type d'ampoules
électriques est normalement distribuée avec une durée de vie moyenne
de $400$ heures et un écart type de $20$ heures.

Il n'existe pas de fonction connue qui soit la primitive de $f$.  Par
contre, la deuxième version du théorème fondamental du calcul affirme
l'existence d'une primitive pour $f$.  Une primitive de $f$ est
\[
F(x) = \int_0^x f(t) \dx{t} =
\int_0^x \frac{e^{-(t-\mu)^2/(2\sigma^2)}}{\sigma \sqrt{2\pi}} \dx{t} \; .
\]
Pour évaluer la fonction $F$, il faut calculer numériquement
(e.g. avec les sommes de Riemann) cette intégrale. 
\end{egg}

\MATHfig{7_integrales/normal2}{9cm}{La fonction de densité pour la distribution
normale}{La fonction de densité pour la distribution normale de
moyenne $\mu=400$ et d'écart type $\sigma=20$.  L'aire de la région en
gris représente la probabilité que la variable aléatoire $X$ soit
entre $390$ et $430$.}{NORMAL_DISTR}

\begin{egg}
Calculons la dérivée de
$\displaystyle F(x) = \int_1^{x^2} (1+s^2)^{3/2} \dx{s}$.

Posons,
\[
G(u) = \int_1^u (1+s^2)^{3/2} \dx{s} \qquad \text{et} \qquad
H(x) = x^2 \; .
\]
Alors $F(x) = G(H(x))$ et $F'(x) = G'(H(x))\,H'(x)$.  La deuxième
version du théorème fondamental du calcul nous donne
$G'(u) = (1+u^2)^{3/2}$.  Ainsi, puisque $H'(x) = 2x$, nous obtenons
\[
F'(x) = G'(H(x))\,H'(x) = \left(1+\left(x^2\right)^2\right)^{3/2}
\big( 2x\big) = 2x\left(1+x^4\right)^{3/2} \; .
\]
\end{egg}

\begin{rmk}
Si $f$ est continue au point $c$, la deuxième version du théorème
fondamental du calcul nous assure que
$\displaystyle F(x) = \int_a^x f(s) \dx{s}$
est différentiable au point $c$ et que $F'(c) = f(c)$.  Si $f$ n'est
pas continue au point $c$, nous ne pouvons rien affirmer.

Soit
\[
f(x) = \begin{cases}
0 & \quad \text{si} \quad x<1 \\
2 & \quad \text{si} \quad x\geq 1
\end{cases}
\]
Nous avons
\[
F(x) = \int_0^x f(s)\dx{s} = \begin{cases}
0 & \quad \text{si} \quad x < 1 \\
2(x-1) & \quad \text{si} \quad x \geq 1
\end{cases} 
\]
(figure~\ref{sndFTC}).  Donc $F$ est une fonction continue comme il
est prédit par la deuxième version du théorème fondamental du calcul
car $f$ est continue par morceaux.    Par contre, $F$ n'est pas
différentiable au point $x=1$ où $f$ n'est pas continue.
\end{rmk}

\PDFfig{7_integrales/second_FTC}{Graphes de $f$ et de $F$ où
$F(x) = \int_0^x f(s)\dx{s}$}{Graphes de $f$ et de $F$ où
$\displaystyle F(x) = \int_0^x f(s)\dx{s}$}{sndFTC}

\section{Intégrales impropres}

\subsection{Intégrales sur un intervalle d'intégration de longueur
infinie}

Jusqu'à présent nous avons évalué des intégrales de la forme
\[
\int_a^b f(x) \dx{x}
\]
où les bornes d'intégration $a$ et $b$ sont des nombres réels.

Pour tout entier positif $k$, le fait que l'intervalle $[a,b]$ soit de
longueur finie nous a permis de définir le nombre réel
$\Delta x = (b-a)/k$ représentant la longueur de chacun des
sous-intervalles d'une partition de l'intervalle $[a,b]$.  Les $k$
sous-intervalles ainsi obtenus étaient de la forme $[x_I, x_{i+1}]$ où
$x_i = a +i \Delta x$ pour $i=0$, $1$, $2$, \ldots, $k$.  Pour tout
entier positif $k$, nous avons un nombre fini de sous-intervalles et donc
une somme de Riemann avec un nombre fini de termes.

Il serait utile et naturel de définir l'intégrale sur des intervalles
de longueur infinie de la forme $[a,\infty[$ où $]-\infty, b]$.  Nous ne
pouvons pas utiliser littéralement notre définition de l'intégrale
définie pour définir une intégrale sur un intervalle de longueur
infinie comme $[a,\infty[$ et $]\-\infty, b]$ car, pour un intervalle
de longueur infinie, il est impossible de définir $\Delta x$ comme le
rapport de la longueur de l'intervalle sur le nombre $k$ de
sous-intervalles.  Il faut regarder le problème d'un angle
différent. Si nous fixons la valeur de $\Delta x$, alors il faut un nombre
infini d'intervalles de longueur $\Delta x$ pour décomposer
l'intervalle d'intégration de longueur infinie.  Les sommes de Riemann
auront alors un nombre infini de termes.

Nous pouvons tout de même utiliser l'existence des intégrales définies sur
les intervalles bornés pour définir l'intégrale sur un intervalle de
longueur infinie.

\begin{defn}\index{Intégrale impropre!sur un domaine non borné}
\index{Intégrale impropre!convergence}\index{Intégrale impropre!divergence}
Soit $f$, une fonction définie sur l'intervalle $[a,\infty[$. Nous
définissons {\bfseries l'intégrale impropre de $\mathbf f$ de $\mathbf a$
à plus l'infini} comme étant la limite suivante si cette limite
existe.
\[
\int_a^\infty f(x) \dx{x} = \lim_{q\rightarrow +\infty} \int_a^q f(x)\dx{x} \; .
\]
Si la limite existe, nous disons que l'intégrale impropre
{\bfseries converge} autrement nous disons qu'elle {\bfseries diverge}.

Soit $f$, une fonction définie sur l'intervalle $]-\infty, b]$. Nous
définissons {\bfseries l'intégrale impropre de $\mathbf f$ de moins
l'infini à $\mathbf b$} comme étant la limite suivante si cette limite
existe.
\[
\int_{-\infty}^b f(x) \dx{x} = \lim_{q\rightarrow -\infty} \int_q^b f(x)\dx{x} \; .
\]
Si la limite existe, nous disons que l'intégrale impropre
{\bfseries converge} autrement nous disons qu'elle {\bfseries diverge}.
\end{defn}

Si $f(x)\geq 0$ pour tout $x$, l'intégrale impropre
$\displaystyle \int_a^\infty f(x) \dx{x}$ représente l'aire de la
région bornée par la courbe $y=f(x)$, l'axe des $x$ et la droite $x=a$
(figure~\ref{IMPR1}).  Nous pouvons énoncer un
résultat semblable pour $\displaystyle \int_{-\infty}^b f(x) \dx{x}$.

\PDFfig{7_integrales/improper1}{L'intégrale impropre comme la limite
d'intégrales définies}{L'intégrale de $f$ de $a$ à plus l'infini est
la limite de l'aire de la région $R_q$ lorsque $q$ tend vers plus
l'infini. Cet intégrale représente l'aire de la région $R$ bornée par
la courbe $y=f(x)$, l'axe des $x$ et la droite $x=a$.}{IMPR1}

\begin{egg}
Évaluons l'intégrale
$\displaystyle \int_{-\infty}^{-1} \frac{1}{(x-1)^{4/3}} \dx{x}$.

Par définition,
\[
\int_{-\infty}^{-1} \frac{1}{(x-1)^{4/3}} \dx{x} =
\lim_{q\rightarrow -\infty} \int_q^{-1} \frac{1}{(x-1)^{4/3}} \dx{x}
\]
si cette limite existe.  Si $y=x-1$, alors $\dx{y} = \dx{x}$ et
\[
\int_q^{-1} \frac{1}{(x-1)^{4/3}} \dx{x} = \int_{q-1}^{-2} y^{-4/3} \dx{y}
= \left( -3 y^{-1/3} \right)\bigg|_{y=q-1}^{-2}
= \frac{3}{2^{1/3}} + \frac{3}{(q-1)^{1/3}} \; .
\]
Ainsi,
\[
\int_{-\infty}^{-1} \frac{1}{(x-1)^{4/3}} \dx{x}
=\lim_{q\rightarrow -\infty} \int_q^{-1} \frac{1}{(x-1)^{4/3}} \dx{x}
=\lim_{q\rightarrow -\infty} \left(\frac{3}{2^{1/3}}
+ \frac{3}{(q-1)^{1/3}} \right)
= \frac{3}{2^{1/3}} \; .
\]
\end{egg}

\begin{egg}
Évaluons l'intégrale
$\displaystyle \int_1^\infty \frac{\ln(x)}{x^3} \dx{x}$.

Par définition,
\[
\int_1^\infty \frac{\ln(x)}{x^3} \dx{x} =
\lim_{q\rightarrow \infty} \int_1^q \frac{\ln(x)}{x^3} \dx{x}
\]
si cette limite existe.  Or,
\[
\int_1^q \frac{\ln(x)}{x^3} \dx{x}
\]
peut être intégrée à l'aide de la règle d'intégration par parties.
Nous avons $x^{-3} \ln(x) = f(x)g'(x)$ où $f(x) = \ln(x)$ et $g'(x) = x^{-3}$.
Donc $f'(x) = x^{-1}$, $g(x) = -\frac{1}{2}\, x^{-2}$ et 
\begin{align*}
\int_1^q x^{-3} \ln(x) \dx{x}
&= -\left(\frac{1}{2}\, x^{-2} \ln(x)\right)\bigg|_{x=1}^q
+ \int_1^q \frac{1}{2}\, x^{-3} \dx{x} \\
&= -\left(\frac{1}{2}\, x^{-2} \ln(x)\right)\bigg|_{x=1}^q
- \left(\frac{1}{4}\, x^{-2} \right)\bigg|_{x=1}^q
= -\frac{\ln(q)}{2q^{2}}  - \frac{1}{4q^{2}} +\frac{1}{4} \; .
\end{align*}
Ainsi,
\[
\int_1^\infty \frac{\ln(x)}{x^3} \dx{x} =
\lim_{q\rightarrow \infty} \int_1^q \frac{\ln(x)}{x^3} \dx{x}
= \lim_{q\rightarrow \infty}
\left(-\frac{\ln(q)}{2q^{2}}  - \frac{1}{4q^{2}} + \frac{1}{4} \right)
= \frac{1}{4}
\]
car, par la règle de l'Hospital,
\[
\lim_{q\rightarrow \infty} \frac{\ln(q)}{q^2}
= \lim_{q\rightarrow \infty} \frac{q^{-1}}{2q}
= \lim_{q\rightarrow \infty} \frac{1}{2q^2} = 0 \; .
\]
% \[
% \int_1^q \frac{\ln(x)}{x^3} \dx{x}
% \]
% peut être intégrée à l'aide de la règle de substitution.  Posons
% $u = \ln(x)$.  Alors $\dx{u} = (1/x) \dx{x}$.  De plus,
% $u = \ln(q)$ lorsque $x=q$ et $u=0$ lorsque $x=1$.  Ainsi,
% \[
% \int_1^q x^{-3} \ln(x) \dx{x}
% = \int_0^{\ln(q)} u e^{-2u} \dx{u} \quad .
% \]
% Nous pouvons utiliser la règle d'intégration par parties pour évaluer
% l'intégrale à la droite.  Nous avons $u e^{-2u} = f(u)g'(u)$ où $f(u) = u$
% et $g'(u) = e^{-2u}$.  Ainsi, $f'(u) = 1$,
% $g(u) = -\frac{1}{2}\, e^{-2u}$ et 
% \begin{align*}
% \int_0^{\ln(q)} u e^{-2u} \dx{u} &= -\frac{1}{2} u
% e^{-2u}\bigg|_0^{\ln(q)} + \int_0^{\ln(q)} \frac{1}{2} e^{-2u} \dx{u} \\
% &= -\frac{1}{2} u
% e^{-2u}\bigg|_0^{\ln(q)} - \frac{1}{4} e^{-2u}\bigg|_0^{\ln(q)} \\
% &= \frac{\ln{q}}{-2q^2} - \frac{1}{4q^2} + \frac{1}{4}
% \end{align*}
\end{egg}

Déterminons pour quelles valeurs de $p$ l'intégrale impropre
$\displaystyle \int_1^\infty \frac{1}{x^p} \dx{x}$ converge.  Pour les
valeurs de $p$ où l'intégrale impropre converge, nous donnerons la
valeur de l'intégrale.

Pour $p\neq 1$,
\[
\int_1^q x^{-p} \dx{x}
= \left( \frac{1}{-p+1} \, x^{-p+1} \right)\bigg|_{x=1}^q
= \frac{1}{1-p} \, q^{1-p} - \frac{1}{1-p} \; .
\]

Pour $p=1$,
\[
\int_1^q x^{-p} \dx{x} = \int_1^q \frac{1}{x} \dx{x}
= \ln(x)\bigg|_{x=1}^q = \ln(q) \; .
\]

Ainsi, pour $p>1$,
\begin{align*}
\int_1^\infty \frac{1}{x^p} \dx{x}
&= \lim_{q\rightarrow +\infty} \int_1^q x^{-p} \dx{x}
= \lim_{q\rightarrow +\infty}
\left( \frac{1}{1-p} \, q^{1-p} - \frac{1}{1-p} \right) \\
&= \lim_{q\rightarrow +\infty}
\left( \frac{1}{(1-p)q^{p-1}} + \frac{1}{p-1} \right)
= \frac{1}{p-1}
\end{align*}
car $p-1>0$.

Pour $p<1$,
\[
\int_1^\infty \frac{1}{x^p} \dx{x}
= \lim_{q\rightarrow +\infty} \int_1^q x^{-p} \dx{x}
= \lim_{q\rightarrow +\infty}
\left( \frac{1}{1-p} \, q^{-p+1} - \frac{1}{1-p} \right)
= +\infty
\]
car $1-p>0$.

Finalement, pour $p=1$,
\[
\int_1^\infty \frac{1}{x^p} \dx{x}
= \lim_{q\rightarrow +\infty} \int_1^q \frac{1}{x} \dx{x}
= \lim_{q\rightarrow +\infty} \ln(q) = \infty \; .
\]
Nous obtenons le résultat suivant.

\begin{prop} \label{impr_comp1}
$\displaystyle \int_1^\infty \frac{1}{x^p} \dx{x}$ converge pour $p>1$
et diverge pour $p\leq 1$.  De plus,
\[
\int_1^\infty \frac{1}{x^p} \dx{x} = \frac{1}{p-1} \quad \text{pour}
\quad p>1 \ .
\]
\end{prop}

La conclusion de la proposition précédente au sujet de la convergence
ou divergence ne changera pas si nous remplaçons
$\displaystyle \int_1^\infty \frac{1}{x^p} \dx{x}$
par $\displaystyle \int_a^\infty \frac{1}{x^p} \dx{x}$
où $a$ est un nombre réel quelconque.

\begin{egg}
Évaluons si possible l'intégrale impropre
$\displaystyle \int_{-\infty}^\infty \frac{1}{x^2+2x+2} \dx{x}$.

Pour évaluer cette intégrale, il faut diviser le domaine d'intégration
en deux.\footnote{Nous aurions pu aussi définir cette intégrale de la
façon suivante.
$\displaystyle \int_{-\infty}^\infty \frac{1}{x^2+2x+2} \dx{x}
= \lim_{q\to \infty} \int_{-q}^q \frac{1}{x^2+2x+2} \dx{x}$.  Nous parlons
alors de l'intégrale de Cauchy de la fonction.  Ce n'est pas la
tradition pour les fonctions à valeurs réelles.  Nous m'adopterons pas
cette convention.}  Nous allons évaluer
\[
\int_{-\infty}^\infty \frac{1}{x^2+2x+2} \dx{x}
= \int_{-\infty}^0 \frac{1}{x^2+2x+2} \dx{x}
+ \int_0^\infty \frac{1}{x^2+2x+2} \dx{x} \; .
\]
Nous aurions pu choisir une autre valeur que $0$ pour partager le domaine
d'intégration.  Nous laissons au lecteur le soin de vérifier que cela ne
changerait rien à la réponse finale.

Si $u=x+1$, alors $\dx{u} = \dx{x}$ et
\begin{align*}
\int \frac{1}{x^2+2x+2} \dx{x} &= \int \frac{1}{(x+1)^2 + 1} \dx{x}
= \int \frac{1}{u^2+1} \dx{u} \\
&= \arctan(u) + C = \arctan(x+1) + C \ .
\end{align*}
Ainsi,
\begin{align*}
\int_{-\infty}^0 \frac{1}{x^2+2x+2} \dx{x}
& = \lim_{q\rightarrow -\infty} \int_q^0 \frac{1}{x^2+2x+2} \dx{x}
= \lim_{q\rightarrow -\infty} \arctan(x+1)\bigg|_{x=q}^0 \\
&= \lim_{q\rightarrow -\infty} \left(\arctan(1) - \arctan(q)\right)
= \frac{\pi}{4} + \frac{\pi}{2} = \frac{3\pi}{4}
\end{align*}
et
\begin{align*}
\int_0^{\infty} \frac{1}{x^2+2x+2} \dx{x}
&= \lim_{q\rightarrow \infty} \int_0^q \frac{1}{x^2+2x+2} \dx{x}
= \lim_{q\rightarrow \infty} \arctan(x+1)\bigg|_{x=0}^q \\
&= \lim_{q\rightarrow \infty} \left(\arctan(q) - \arctan(1)\right)
= \frac{\pi}{2} - \frac{\pi}{4} = \frac{\pi}{4} \; .
\end{align*}
Donc
\[
\int_{-\infty}^\infty \frac{1}{x^2+2x+2} \dx{x} = \frac{3\pi}{4} +
\frac{\pi}{4} = \pi \; .
\]
\end{egg}
 
\begin{egg}
Évaluons si possible l'intégrale impropre
$\displaystyle \int_0^{\infty} x e^{-x} \dx{x}$

Nous utilisons la règle d'intégration par parties pour évaluer
$\displaystyle \int x e^{-x} \dx{x}$.

Nous avons $x e^{-x} = f(x) g'(x)$ où $f(x) = x$ et $g'(x) = e^{-x}$.
Donc $f'(x) = 1$, $g(x) = - e^{-x}$ et 
\[
\int x e^{-x} \dx{x}
= -x e^{-x} + \int e^{-x} \dx{x} = -x e^{-x} - e^{-x} + C \; .
\]

Ainsi,
\begin{align*}
\int_0^{\infty} x e^{-x} \dx{x} &
= \lim_{q\rightarrow \infty} \int_0^q x e^{-x} \dx{x}
= \lim_{q\rightarrow \infty} \left( -x e^{-x} - e^{-x} \right)\bigg|_{x=0}^q
= \lim_{q\rightarrow \infty} \left( -q e^{-q} - e^{-q} + 1\right) = 1
\end{align*}
car
\[
\lim_{q\rightarrow \infty} e^{-q} = \lim_{q\rightarrow \infty} \frac{1}{e^q} = 0
\]
et
\[
\lim_{q\rightarrow \infty} q e^{-q}
= \lim_{q\rightarrow \infty} \frac{q}{e^q} 
= \lim_{q\rightarrow \infty} \frac{1}{e^q} = 0 \; .
\]
Pour le calcul de cette dernière limite, nous avons utilisé la règle
de l'Hospital pour obtenir la deuxième égalité car
$\displaystyle \lim_{q\rightarrow \infty} q = +\infty$ et
$\displaystyle \lim_{q\rightarrow \infty} e^q = +\infty$.
\end{egg}

\subsection{Intégrales avec un intégrande non bornée}

Une autre situation où il serait utile de définir l'intégrale est
lorsque l'intégrande n'est pas borné.  Par exemple, pouvons-nous définir
l'intégrale de façon à pouvoir l'utiliser pour calculer l'aire $A$ de
la région $R$ représentée à la figure~\ref{IMPR2} qui est bornée par la courbe
$y=1/\sqrt{x}$ et les droites $y=0$, $x=0$ et $x=1$.

\PDFfig{7_integrales/improper2}{La région bornée par la courbe
$y= 1/\sqrt{x}$, l'axe des $x$ et les droites $x=0$ et $x=1$.}{La
région $R$ est la région bornée par la courbe $y=f(x) = 1/\sqrt{x}$,
l'axe des $x$ et les droites $x=0$ et $x=1$.}{IMPR2}

Nous définissons respectivement ci-dessous ce que sera l'intégrale d'une
fonction $f:]a,b]\to \RR$ qui n'est pas bornée en $x=a$, et
l'intégrale d'une fonction $f:[a,b[\to \RR$ qui n'est pas bornée en
$x=b$.

\begin{defn} \index{Intégrale impropre!pour une fonction non bornée}
\index{Intégrale impropre!convergence}\index{Intégrale impropre!divergence}
Soit $f$, une fonction définie sur l'intervalle $]a, b]$. Nous définissons
{\bfseries l'intégrale impropre de $\mathbf f$ de $\mathbf a$ à
$\mathbf b$} comme étant la limite suivante si cette limite existe.
\[
\int_a^b f(x) \dx{x}
= \lim_{q\rightarrow a^+} \int_q^b f(x)\dx{x}
\]
Si la limite existe, nous disons que l'intégrale impropre
{\bfseries converge} autrement nous disons qu'elle {\bfseries diverge}.

Soit $f$, une fonction définie sur l'intervalle $[a, b[$.  Nous définissons
{\bfseries l'intégrale impropre de $\mathbf f$ de $\mathbf a$ à
$\mathbf b$} comme étant la limite suivante si cette limite existe.
\[
\int_a^b f(x) \dx{x}
= \lim_{q\rightarrow b^-} \int_a^q f(x)\dx{x} \; .
\]
Si la limite existe, nous disons que l'intégrale impropre
{\bfseries converge} autrement nous disons qu'elle {\bfseries diverge}.
\end{defn}

Si $f(x)\geq 0$ pour tout $x$, ces intégrales impropres représentent
l'aire de la région bornée par la courbe $y=f(x)$, l'axe des $x$ et
les droites $x=a$ et $x=b$ (figure~\ref{IMPR3}).

\PDFfig{7_integrales/improper3}{L'intégrale impropre d'une fonction
définie sur un intervalle ouvert de longueur finie}{L'intégrale de $f$
de $a$ à $b$ est la limite de l'aire de la région $R_q$ lorsque $q$
tend vers $a$.  L'intégrale représente l'aire de la région $R$ bornée par la
courbe $y=f(x)$, l'axe des $x$ et les droites $x=a$ et $x=b$.}{IMPR3}

\begin{egg}
Évaluons l'intégrale
$\displaystyle \int_0^1 \frac{1}{\sqrt{x}} \dx{x}$.

Puisque $x=0$ est une asymptote verticale pour $f(x)=1/\sqrt{x}$, nous
avons que
\[
\int_0^1 \frac{1}{\sqrt{x}} \dx{x} =
\lim_{q\rightarrow 0^+} \int_q^1 \frac{1}{\sqrt{x}} \dx{x}
\]
si cette limite existe.  Or,
\[
\int_q^1 x^{-1/2} \dx{x} = \left( 2x^{1/2} \right)\bigg|_{x=q}^1
= 2 - 2\sqrt{q} \; .
\]
Ainsi,
\[
\int_0^1 \frac{1}{\sqrt{x}} \dx{x} =
\lim_{q\rightarrow 0^+} \int_q^1 \frac{1}{\sqrt{x}} \dx{x}
=\lim_{q\rightarrow 0^+} \left( 2 - 2\sqrt{q} \right) = 2 \; .
\]
\end{egg}

\PDFfig{7_integrales/improper4}{Graphe de $\displaystyle y=\frac{1}{x^2+x-6}$}
{Graphe de $\displaystyle y=\frac{1}{x^2+x-6}$ pour $0\leq x < 2$.}{IMPR4} 

\begin{egg}
Déterminons si l'intégrale impropre
$\displaystyle \int_0^2 \frac{1}{x^2+x-6} \dx{x}$ converge ou diverge.
Si elle converge, nous allons donner la valeur de l'intégrale.

Notons que $x^2+x-6 = (x+3)(x-2)$.  Ainsi,
\[
f(x) = \frac{1}{x^2+x-6} = \frac{1}{(x+3)(x-2)}
\]
a une asymptote verticale à $x=2$.  Le graphe de $f$ est donné à la
figure~\ref{IMPR4}.  Ainsi, il faut déterminer si
\[
\int_0^2 \frac{1}{x^2+x-6} \dx{x}
= \lim_{q\rightarrow 2^-} \int_0^q \frac{1}{x^2+x-6} \dx{x}
\]
existe.

Pour évaluer cette intégrale, nous utilisons la méthode des fractions
partielles.  C'est-à-dire qu'il faut déterminer les variables $A$ et $B$
telles que
\[
\frac{1}{x^2+x-6} = \frac{1}{(x+3)(x-2)} = \frac{A}{x+3} + \frac{B}{x-2} \; .
\]
Si nous écrivons ces fractions sur un même commun dénominateur, nous obtenons
\[
\frac{1}{(x+3)(x-2)} = \frac{A(x-2)}{(x+3)(x-2)} +
\frac{B(x+3)}{(x+3)(x-2)} \; .
\]
Nous avons égalité quand les numérateurs des deux côtés de
l'égalité sont égaux; c'est-à-dire, quand $1=A(x-2)+B(x+3)$.  Pour
$x=-3$, nous obtenons $1=-5A$ et donc $A=-1/5$.  Pour $x=2$, nous
obtenons $1=5B$ et donc $B=1/5$.  Nous avons donc que
\[
\frac{1}{x^2+x-6} = \frac{1}{(x+3)(x-2)}
= - \frac{1}{5} \left(\frac{1}{x+3} \right)
+ \frac{1}{5} \left(\frac{1}{x-2}\right) \; .
\]
Ainsi,
\begin{align*}
\int_0^q \frac{1}{x^2+x-6} \dx{x} &= 
- \frac{1}{5} \int_0^q \,\frac{1}{x+3} \dx{x}
+ \frac{1}{5} \int_0^q \,\frac{1}{x-2} \dx{x} \\
&= -\frac{1}{5} \ln(x+3) \bigg|_{x=0}^q 
+ \frac{1}{5} \ln|x-2| \bigg|_{x=0}^q \\
&= -\frac{1}{5} \left( \ln(q+3) - \ln(3) - \ln|q-2| + \ln(2) \right) \\
&= -\frac{1}{5} \left( \ln(q+3) -\ln|q-2| + \ln(2/3) \right) \; .
\end{align*}
Puisque $\displaystyle \lim_{q\rightarrow 2^-} \ln|q-2| = -\infty$ et
$\displaystyle \lim_{q\rightarrow 2^-} \ln(q+3) = \ln(5)$, nous obtenons
\begin{align*}
\int_0^2 \frac{1}{x^2+x-6} \dx{x}
&= \lim_{q\rightarrow 2^-} \int_0^q \frac{1}{x^2+x-6} \dx{x} \\
&= - \frac{1}{5} \lim_{q\rightarrow 2^-}
\left( \ln(q+3) -\ln|q-2|
+ \ln\left(\frac{2}{3}\right) \right) = -\infty \; .
\end{align*}
L'intégrale diverge.
\end{egg}

Déterminons pour quelles valeurs de $p$ l'intégrale
$\displaystyle \int_0^1 \frac{1}{x^p} \dx{x}$ converge.  Pour les
valeurs de $p$ où l'intégrale converge, nous donnerons la valeur de
l'intégrale.

Pour $p\neq 1$,
\[
\int_q^1 x^{-p} \dx{x}
= \left( \frac{1}{-p+1} \, x^{-p+1} \right)\bigg|_{x=q}^1
= \frac{1}{1-p} - \frac{1}{1-p} \, q^{1-p} \; .
\]

Pour $p=1$,
\[
\int_q^1 x^{-p} \dx{x} = \int_q^1 \frac{1}{x} \dx{x}
= \ln(x)\bigg|_{x=q}^1 = - \ln(q) \; .
\]

Ainsi, pour $p<1$,
\[
\int_0^1 \frac{1}{x^p} \dx{x}
= \lim_{q\rightarrow 0^+} \int_q^1 x^{-p} \dx{x}
= \lim_{q\rightarrow 0^+}
\left( \frac{1}{1-p} - \frac{1}{1-p} \, q^{1-p} \right) \\
= \frac{1}{1-p}
\]
car $1-p>0$.

Pour $p>1$,
\begin{align*}
\int_0^1 \frac{1}{x^p} \dx{x}
&= \lim_{q\rightarrow 0^+} \int_q^1 x^{-p} \dx{x}
= \lim_{q\rightarrow 0^+}
\left( \frac{1}{1-p} - \frac{1}{1-p} \, q^{1-p} \right) \\
&= \lim_{q\rightarrow 0^+}
\left( \frac{1}{1-p} - \frac{1}{(1-p)q^{p-1}} \right)
= \infty
\end{align*}
car $p-1>0$.

Finalement, pour $p=1$,
\[
\int_0^1 \frac{1}{x^p} \dx{x} = \int_0^1 \frac{1}{x} \dx{x}
= \lim_{q\rightarrow 0^+} \int_q^1 \frac{1}{x} \dx{x}
= - \lim_{q\rightarrow 0^+} \ln(q) = \infty \; .
\]
Nous obtenons le résultat suivant.

\begin{prop}
$\displaystyle \int_0^1 \frac{1}{x^p} \dx{x}$ converge pour
$p<1$ et diverge pour $p\geq 1$.  De plus,
$\displaystyle \int_0^1 \frac{1}{x^p} \dx{x} = \frac{1}{1-p}$
pour $p<1$.
\label{impr_comp2}
\end{prop}

La conclusion de la proposition précédente au sujet de la convergence
ou divergence ne changera pas si nous remplaçons
$\displaystyle \int_0^1 \frac{1}{x^p} \dx{x}$ par
$\displaystyle \int_0^a \frac{1}{x^p} \dx{x}$
où $a>0$ est un nombre réel quelconque.

\begin{rmk}[\theory]
La proposition~\ref{impr_comp2} peut être déduite de la
proposition~\ref{impr_comp1} et vice-versa.  Si nous posons $t=1/x$,
nous avons $\displaystyle \dx{t} = \frac{-1}{x^2} \dx{x}$ et
\[
\int_1^q \frac{1}{x^p} \dx{x}
= \int_1^q \frac{-1}{x^{p-2}}\left(\frac{-1}{x^2}\right) \dx{x} \\
= -\int_1^{1/q} \frac{1}{t^{2-p}} \dx{t}
= \int_{1/q}^1 \frac{1}{t^{2-p}} \dx{t} \; .
\]
Les énoncés suivants sont équivalents.

{
\renewcommand{\labelitemi}{$\Leftrightarrow$}
\begin{itemize}
\item[] $\displaystyle \lim_{q\to \infty} \int_1^q \frac{1}{x^p} \dx{x}$
existe si et seulement si $p>1$ (énoncé de la proposition~\ref{impr_comp1}).
\item $\displaystyle \lim_{q\to \infty} \int_{1/q}^1 \frac{1}{t^{2-p}} \dx{t}$
existe si et seulement si $p>1$ (après la substitution $t = 1/x$).
\item $\displaystyle \lim_{q\to \infty} \int_{1/q}^1 \frac{1}{t^{2-p}} \dx{t}$
existe si et seulement si $2-p < 1$ (car $p>1 \Leftrightarrow 2-p <1$).
\item $\displaystyle \lim_{q\to 0} \int_q^1 \frac{1}{t^r} \dx{t}$
existe si et seulement si $r<1$ (énoncé de la
proposition~\ref{impr_comp2} si nous remplaçons $2-p$ par $r$).
\end{itemize}
}\end{rmk}

\section{Test de comparaison \eng}

Il est généralement difficile d'évaluer algébriquement les intégrales.
C'est encore plus évident dans le cas des intégrales impropres.  Nous avons
donc souvent recours aux méthodes numériques pour estimer la valeur
des intégrales impropres.  Avant d'estimer numériquement une intégrale
impropre, il est nécessaire de déterminer si elle converge.  Le fait
que les calculs numériques d'une intégrale impropre produisent des
valeurs supérieures à ce qu'un ordinateur peut traiter (nous parlons de
\lgm overflow\rgm) n'indique pas que l'intégrale impropre diverge.
Il se pourrait très bien que la valeur de l'intégrale soit en fait une
valeur plus grande que l'ordinateur peut traiter.  Il est aussi
possible que ce problème soit dû à des erreurs de troncature ou à
l'utilisation d'une méthode d'intégration numérique inappropriée.

Nous présentons un test pour déterminer si une intégrale impropre
converge ou diverge.  Les propositions~\ref{impr_comp1} et
\ref{impr_comp2} seront d'une très grande utilité.

\begin{theorem}[Test de comparaison]
Soit $f:]a,b[\rightarrow \RR$ et $g:]a,b[\rightarrow \RR$ deux
fonctions continues telles que $0\leq f(x)\leq g(x)$ pour tout $x$
dans l'intervalle $]a,b[$.  Les bornes d'intégration $a$ et $b$
peuvent être $+\infty$ ou $-\infty$, ou ils peuvent correspondent à
des asymptotes verticales pour les fonctions $f$ et $g$.
\begin{enumerate}
\item Si $\displaystyle \int_a^b g(x) \dx{x}$ converge alors
$\displaystyle \int_a^b f(x) \dx{x}$ converge.
\item Si $\displaystyle \int_a^b f(x) \dx{x}$ diverge alors
$\displaystyle \int_a^b g(x) \dx{x}$ diverge.
\end{enumerate}
\end{theorem}

Dans le cas où $f$ et $g$ sont continues sur l'intervalle
$[a,+\infty[$ et $b=\infty$, le théorème précédent est une conséquence
de la relation $0\leq f(x)\leq g(x)$ pour toux $x$ dans l'intervalle
$[a,+\infty[$.  En effet, il découle de cette dernière inégalité que
\begin{equation}\label{impr_FLG}
0 \leq \int_a^q f(x) \dx{x} \leq \int_a^q g(x) \dx{x}
\end{equation}
pour tout $q \geq a$.  Définissons les fonctions
\begin{align}
F(q) = \int_a^q f(x) \dx{x} \label{imprF}
\intertext{et}
G(q) = \int_a^q g(x) \dx{x} \label{imprG}
\end{align}
Puisque $f$ et $g$ sont deux fonctions positives sur l'intervalle
$[a,+\infty[$, les fonctions $F$ et $G$ sont positives et croissantes
(l'aire sous la courbe augmente lorsque $q$ augmente).  De plus,
(\ref{impr_FLG}) devient $0 \leq F(q) \leq G(q)$ pour tout $q$.

Si nous supposons que
\[
\int_a^{+\infty} g(x) \dx{x} = \lim_{q\rightarrow +\infty} G(q)
\]
converge, alors $F$ est une fonction croissante bornée supérieurement par
$\displaystyle \int_a^{+\infty} g(x) \dx{x}$.  Il s'en suit que
\[
\int_a^{+\infty} f(x) \dx{x} = \lim_{q\rightarrow +\infty} F(q)
\]
converge grâce au théorème~\ref{suiteBORN}.  Nous retrouvons à la
figure~\ref{IMPR5} deux graphes qui peuvent représenter
qualitativement les graphes de $F$ et $G$.

Par contre si $\displaystyle \int_a^{+\infty} f(x) \dx{x}$ diverge
alors $F$ est une fonction qui croit sans limite supérieure.  Il en
est donc de même pour $G$ car $F(q) \leq G(q)$ pour tout $q$.  Ainsi,
$\displaystyle \lim_{q\rightarrow +\infty} G(q) = +\infty$ et
$\displaystyle \int_a^{+\infty} g(x) \dx{x}$ diverge.

\PDFfig{7_integrales/improper5}{Représentation qualitative du graphe de
fonctions définies par des intégrales}{Représentation qualitative du
graphe des fonctions $F$ et $G$ définies en (\ref{imprF}) et
(\ref{imprG}) respectivement.}{IMPR5}

La justification du test de comparaison pour les autres types
d'intégrales impropres est semblable.

\begin{egg}
Déterminons si les intégrales suivantes convergent ou divergent.
\begin{center}
\begin{tabular}{*{1}{l@{\hspace{1em}}l@{\hspace{5em}}}l@{\hspace{1em}}l}
\subQ{a} & $\displaystyle \int_1^\infty \frac{|\sin(x)|}{x^2} \dx{x}$  &
\subQ{b} & $\displaystyle \int_1^\infty \frac{x^2}{x^3-x+1}\dx{x}$
\end{tabular}
\end{center}

\subQ{a} Puisque
\[
0 \leq \frac{|\sin(x)|}{x^2} \leq \frac{1}{x^2}
\]
pour tout $x\geq 1$ et
\[
\int_1^\infty \frac{1}{x^2} \dx{x}
\]
converge car c'est le cas $p>1$ de la proposition~\ref{impr_comp1},
nous avons que
\[
\displaystyle \int_1^\infty \frac{|\sin(x)|}{x^2} \dx{x}
\]
converge grâce au test de comparaison.

\subQ{b} Remarquons que
$\displaystyle \frac{x^2}{x^3-x+1} > \frac{1}{x}$ pour $x\geq 1$.  En effet,
pour $x\geq 1$, nous avons $0\geq 1-x$ et ainsi
\begin{equation}\label{impr_ineq}
x^3 \geq x^3 -x + 1 \; .
\end{equation}
La fonction $p(x) = x^3-x+1$ est croissante pour $x\geq 1$ car
$p'(x) = 3x^2 -1 > 0$ pour $x\geq 1$.  Ainsi,
$x^3-x+1 = p(x) \geq p(1) = 1 > 0$ pour $x\geq 1$.  Nous pouvons donc
diviser les deux cotés de l'inégalité en (\ref{impr_ineq}) par
$x(x^3-x+1)$ sans changer la direction de l'inégalité et sans risquer
de diviser par zéro. Nous obtenons
\[
\frac{x^2}{x^3-x+1} \geq \frac{1}{x}
\]
pour $x\geq 1$.

Puisque
\[
\int_1^\infty \frac{1}{x} \dx{x}
\]
diverge car c'est le cas $p \leq 1$ de la proposition~\ref{impr_comp1},
nous avons que
\[
\displaystyle \int_1^\infty \frac{x^2}{x^3-x+1}\dx{x}
\]
diverge grâce au test de comparaison.
\label{testCompSin}
\end{egg}

\begin{egg}
Déterminons si l'intégrale
$\displaystyle \int_1^5 \frac{1}{(x^2-1)^{1/3}} \dx{x}$
converge ou diverge.

C'est une intégrale impropre car $x=1$ est une asymptote verticale de
$(x^2-1)^{-1/3}$.  Nous avons que
\[
\int_1^5 \frac{1}{(x^2-1)^{1/3}} \dx{x} = \lim_{q\to 1^+}
\int_q^5 \frac{1}{(x^2-1)^{1/3}} \dx{x} \ .
\]
Puisque notre intention est de comparer cette intégrale avec
l'intégrale entre $0$ et $a$ de la proposition~\ref{impr_comp2}, nous
commençons par une simple substitution pour obtenir $0$ comme borne
inférieure pour l'intégrale.  Si $u=x-1$. alors $\dx{u} = \dx{x}$,
$u=q-1$ lorsque $x=q$ et $u=4$ lorsque $x=5$.  Donc
\[
\int_q^5 \frac{1}{(x^2-1)^{1/3}} \dx{x}
= \int_{q-1}^4 \frac{1}{((u+1)^2-1)^{1/3}} \dx{u}
= \int_{q-1}^{4} \frac{1}{(u^2+2u)^{1/3}} \dx{u} \ .
\]
Ainsi, lorsque $q$ tend vers $1$, nous obtenons que
$\displaystyle \int_1^5 \frac{1}{(x^2-1)^{1/3}} \dx{x}$ converge si et
seulement si $\displaystyle \int_0^{4} \frac{1}{(u^2+2u)^{1/3}} \dx{u}$
converge.  La substitution $u=x-1$ est simplement une translation de
l'intervalle d'intégration de $1$ vers la gauche.  Il suffit de
terminer si cette dernière intégrale impropre converge ou diverge pour
obtenir la même conclusion pour l'intégrale impropre du départ.

Puisque
\[
(u^2+2u)^{1/3} = u^{2/3} \left(1+\frac{2}{u}\right)^{1/3} \geq u^{2/3}
\]
pour $u>0$, nous avons
\[
0 \leq \frac{1}{(u^2+2u)^{1/3}} \leq \frac{1}{u^{2/3}}
\]
pour tout $u>0$.  Puisque
\[
\int_0^{4} \frac{1}{u^{2/3}} \dx{u}
\]
converge car c'est le cas $p<1$ de la proposition~\ref{impr_comp2}, nous avons
que
\[
\int_0^{4} \frac{1}{(u^2+2u)^{1/3}} \dx{u}
\]
converge grâce au test de comparaison.
\end{egg}

Le test de comparaison peut aussi être utilisé pour déterminer la
convergence d'intégrales impropres dont l'intégrande peut avoir des
valeurs négatives.

\begin{theorem} \label{compabsolu}
Soit $f:]a,b[\rightarrow \RR$ une fonction continue sur l'intervalle
$]a,b[$.   Alors $\displaystyle \int_a^b f(x) \dx{x}$
converge si $\displaystyle \int_a^b |f(x)| \dx{x}$ converge.
Les bornes d'intégration $a$ et $b$ peuvent être $+\infty$
ou $-\infty$, ou ils peuvent correspondent à des asymptotes verticales
pour la fonction $f$. 
\end{theorem}

Par exemple, $\displaystyle \int_a^{\infty} f(x) \dx{x}$
converge si $\displaystyle \int_a^{\infty} |f(x)| \dx{x}$ converge.

Il est possible de justifier intuitivement ce résultat en considérant
le graphe de $f$.  Supposons que le graphe de $f$ soit celui donné
ci-dessous.
\PDFgraph{7_integrales/absoluA}

L'aire de la région $A$ plus l'aire de la région $B$ dans la figure
suivante donnent l'aire sous la courbe $y=|f(x)|$ pour $a\leq x < \infty$;
c'est-à-dire, $\displaystyle \int_a^{\infty} |f(x)| \dx{x}$.
\PDFgraph{7_integrales/absoluB}

Si nous supposons que la valeur de l'intégrale de $|f(x)|$ est finie,
alors l'aire de la région $A$ et l'aire de la région $B$ sont
aussi des valeurs finies.  Il s'en suit que la valeur de
$\displaystyle \int_a^{\infty} f(x) \dx{x}$ est finie
car c'est l'aire de la région $A$ moins l'aire de la région $B$, deux
nombres réels, comme il est illustré dans la figure suivante.
\PDFgraph{7_integrales/absoluC}

\begin{rmk}[\theory]
Nous présentons ci-dessous une démonstration que
$\displaystyle \int_a^{\infty} f(x) \dx{x}$ 
converge si $\displaystyle \int_a^{\infty} |f(x)| \dx{x}$ converge.
Cette démonstration formalise la justification graphique que nous
venons de donner ci-dessus.

Si $\displaystyle \int_a^\infty |f(x)| \dx{x}$ converge alors
$\displaystyle \int_a^\infty 2|f(x)| \dx{x}$ converge
car
\[
\lim_{q\rightarrow \infty} \int_a^q 2|f(x)| \dx{x} =
2 \lim_{q\rightarrow \infty} \int_a^q |f(x)| \dx{x} \; .
\]
Puisque
\[
0 \leq |f(x)| - f(x) \leq 2 |f(x)|
\]
pour tout $x$ dans l'intervalle $]a,\infty[$, nous obtenons que
$\displaystyle \int_a^\infty \left( |f(x)| - f(x) \right) \dx{x}$
converge grâce au test de comparaison.

Finalement, $\displaystyle \int_a^\infty f(x) \dx{x}$ converge car
\begin{align*}
\lim_{q\rightarrow \infty} \int_a^q f(x) \dx{x}
&= \lim_{q\rightarrow \infty} \left( \int_a^q |f(x)| \dx{x}
- \int_a^q \left( |f(x)| - f(x) \right) \dx{x} \right) \\
&= \lim_{q\rightarrow \infty} \int_a^q |f(x)| \dx{x}
- \lim_{q\rightarrow \infty} \int_a^q \left( |f(x)| - f(x) \right)
\dx{x}
\end{align*}
où les deux dernières limites existent.

Nous pourrions tirer la même conclusion en considérant les autres
formes pour l'intégrale impropre.
\end{rmk}

\begin{egg}
Nous avons vu à l'exemple~\ref{testCompSin} que l'intégrale
$\displaystyle \int_1^\infty \frac{|\sin(x)|}{x^2} \dx{x}$ convergeait.
Il s'en suit que l'intégrale
$\displaystyle \int_1^\infty \frac{\sin(x)}{x^2} \dx{x}$ converge.
\end{egg}

\begin{rmk}[\theory]
Si $\displaystyle \int_a^b |f(x)| \dx{x}$ diverge, nous ne pouvons rien dire au
sujet de $\displaystyle \int_a^b f(x) \dx{x}$.

Par exemple, $\displaystyle \int_1^\infty \frac{\sin(x)}{x} \dx{x}$
converge mais
$\displaystyle \int_1^\infty \left| \frac{\sin(x)}{x} \right| \dx{x}$
diverge.

\subQ{i} Commençons par montrer que
$\displaystyle \int_1^\infty \frac{\sin(x)}{x} \dx{x}$ converge.

Puisque $\sin(x)/x = f(x)g'(x)$ où $f(x) = 1/x$ et $g'(x) = \sin(x)$, nous
obtenons $f'(x) = -1/x^2$, $g(x) = -\cos(x)$ et
\[
\int_1^q \frac{\sin(x)}{x} \dx{x} = -\frac{\cos(x)}{x} \bigg|_{x=1}^q
- \int_1^q \frac{\cos(x)}{x^2} \dx{x}
\]
par la règle d'intégration par parties.  Puisque
\[
0 \leq \left| \frac{\cos(x)}{x^2} \right| \leq \frac{1}{x^2}
\]
et $\displaystyle \int_1^\infty \frac{1}{x^2} \dx{x}$ converge grâce
à la proposition~\ref{impr_comp1}, nous avons que
\[
\int_1^\infty \left| \frac{\cos(x)}{x^2} \right| \dx{x}
\]
converge.  Donc
$\displaystyle \int_1^\infty \frac{\cos(x)}{x^2} \dx{x}$ converge.  De
plus
\[
\lim_{q\rightarrow \infty} \left( -\frac{\cos(x)}{x} \bigg|_{x=1}^q \right)
= \lim_{q\rightarrow \infty} \left(\cos(1) -\frac{\cos(q)}{q} \right) =
\cos(1) \; . 
\]
Ainsi,
\begin{align*}
\int_1^\infty \frac{\sin(x)}{x} \dx{x}
&= \lim_{q\rightarrow \infty} \int_1^q \frac{\sin(x)}{x} \dx{x}
= \lim_{q\rightarrow \infty} \left( -\frac{\cos(x)}{x} \bigg|_{x=1}^q
- \int_1^q \frac{\cos(x)}{x^2} \dx{x} \right) \\
&= \cos(1) - \int_1^\infty \frac{\cos(x)}{x^2} \dx{x}
\end{align*}
est une nombre réel.

\subQ{ii} Pour montrer que
$\displaystyle \int_1^\infty \left| \frac{\sin(x)}{x} \right| \dx{x}$
diverge, nous montrons que
\[
\lim_{n\rightarrow \infty} 
\int_1^{n\pi} \left| \frac{\sin(x)}{x} \right| \dx{x} = +\infty \; .
\]
Nous remarquons que
\[
\int_{k\pi}^{(k+1)\pi} \left| \frac{\sin(x)}{x} \right| \dx{x}
\geq \frac{1}{(k+1)\pi} \int_{k\pi}^{(k+1)\pi} \left| \sin(x) \right| \dx{x}
= \frac{2}{(k+1)\pi} \geq \frac{2}{\pi} \int_{k+1}^{k+2} \frac{1}{x} \dx{x}
\]
car
\[
\int_{k\pi}^{(k+1)\pi} \left| \sin(x) \right| \dx{x} =
\begin{cases}
\displaystyle \int_0^{\pi} \sin(x) \dx{x} = -\cos(x)\bigg|_0^\pi = 2 & \quad
\text{pour $k$ pair.} \\[1em]
\displaystyle \int_\pi^{2\pi} -\sin(x) \dx{x} = \cos(x)\bigg|_\pi^{2\pi}
 = 2 & \quad \text{pour $k$ impair.}
\end{cases}
\]
Ainsi,
\begin{align*}
\int_1^{n\pi} \left| \frac{\sin(x)}{x} \right| \dx{x}
&= \int_1^\pi \left| \frac{\sin(x)}{x} \right| \dx{x}
+ \sum_{k=1}^{n-1} \int_{k\pi}^{(k+1)\pi} \left| \frac{\sin(x)}{x}
\right| \dx{x} \\
&\geq \int_1^\pi \left| \frac{\sin(x)}{x} \right| \dx{x}
+ \frac{2}{\pi} \sum_{k=1}^{n-1} \int_{k+1}^{k+2} \frac{1}{x}\dx{x} \\
&= \int_1^\pi \left| \frac{\sin(x)}{x} \right| \dx{x}
+ \frac{2}{\pi} \int_2^{n+1} \frac{1}{x}\dx{x} \\
&= \int_1^\pi \left| \frac{\sin(x)}{x} \right| \dx{x}
+ \frac{2}{\pi} \left( \ln(n+1)-\ln(2)\right) \rightarrow +\infty
\end{align*}
lorsque $n$ tend vers plus l'infini.
\end{rmk}

\section{Méthodes numériques d'intégration \eng}

Soit $f:[a,b]\rightarrow \RR$ une fonction suffisamment
différentiable.  Notre but est de développer des méthodes numériques
qui nous permettrons d'évaluer l'intégrale
\[
\int_a^b\,f(x) \dx{x}
\]
lorsque nos techniques algébriques d'intégration s'avèrent impuissante
à évaluer cette intégrale ou lorsque quelles aboutissent à des calculs
algébriques qui sont longs et complexes.

L'idée principale qui supporte les méthodes numériques que nous
présentons est la suivante.  Nous cherchons une fonction
$p:[a,b] \rightarrow \RR$ telle que $|p(x)-f(x)|$ est très petit pour
tout $x$ et telle que $\displaystyle \int_a^b\,p(x) \dx{x}$ est simple
à évaluer.  Si $|p(x)-f(x)|$ est très petit pour tout $x$, nous pouvons
espérer que
\[
\int_a^b\,f(x) \dx{x} \approx \int_a^b\,p(x) \dx{x} \; .
\]

Il y a une infinité de choix possible pour $p$.  Les trois méthodes
que nous présentons sont les suivantes.
\begin{enumerate}
\item La méthode du point milieu où $p$ sera une fonction constante
par morceaux.
\item La méthode des trapèzes où $p$ sera une fonction linéaire par
morceaux.
\item La méthode de Simpson où $p$ sera une fonction quadratique par
morceaux.
\end{enumerate}

\subsection{Méthode du point milieu}

Soit $f:[a,b]\rightarrow \RR$ une fonction qui possède une dérivée
continue d'ordre deux.

Nous partageons l'intervalle $[a,b]$ en $2n$ sous-intervalles
où $n$ est un entier positif.  Soit $h = (b-a)/(2n)$ et $x_i = a + i h$
pour $i=0$, $1$, $2$, \ldots, $2n$.  L'intervalle $[a,b]$ est l'union
des $2n$ sous-intervalles de la forme $[x_i,x_{i+1}]$ pour $i=0$, $1$,
$2$, \ldots, $2n-1$.

Nous définissons la fonction $p:[a,b] \rightarrow \RR$ de la façon suivante.
\[
p(x) = f(x_{2i+1})  \quad \text{si} \quad x_{2i} \leq x < x_{2i+2} \; .
\]
C'est une fonction qui est constante par morceaux.  Sur un intervalle
de la forme $[x_{2i},x_{2i+2}]$, la fonction $p$ prend la valeur de
$f$ au point milieu $x_{2i+1}$ (figure~\ref{MIDPOINT}).  Nous avons que
\begin{align*}
\int_a^b p(x) \dx{x}
&= \sum_{i=0}^{n-1} \int_{x_{2i}}^{x_{2i+2}} p(x) \dx{x} 
= \sum_{i=0}^{n-1} \int_{x_{2i}}^{x_{2i+2}} f(x_{2i+1}) \dx{x} \\
&= \sum_{i=0}^{n-1} f(x_{2i+1}) (x_{2i+2} - x_{2i})
= 2h \sum_{i=0}^{n-1} f(x_{2i+1})
\end{align*}

\PDFfig{7_integrales/midpoint}{Méthode du point milieu pour évaluer
numériquement une intégrale}{Méthode du point milieu pour évaluer
numériquement une intégrale}{MIDPOINT}

La somme
$\displaystyle 2h \sum_{i=0}^{n-1} f(x_{2i+1})$ est une somme de
Riemann avec $\Delta x = 2h$ pour l'intégrale
$\displaystyle \int_a^b f(x) \dx{x}$.  Donc, en théorie,
$\displaystyle 2h \sum_{i=0}^{n-1} f(x_{2i+1})$ tend vers
$\displaystyle \int_a^b f(x) \dx{x}$ lorsque $h$ tend vers $0$.  Nous
pourrions démontrer le théorème suivant.

\begin{prop}
Soit $f:[a,b]\rightarrow \RR$ une fonction qui possède une dérivée
continue d'ordre deux.  Soit $h=(b-a)/(2n)$ et $x_j = a+j\,h$ for
$j=0$, $1$ \ldots, $2n$.  Alors
\[
\int_{a}^{b} f(x)\dx{x} = 2h \sum_{j=0}^{n-1}\,f(x_{2j+1}) +
\frac{f''(\xi)\,(b-a)}{6}\,h^2
\]
pour une valeur $\xi \in [a,b]$.
\end{prop}

Cette proposition nous donne notre première méthode d'intégration
numérique.

\begin{meth}[Méthode du point milieu]
\index{Méthode du point milieu}
Une approximation de la valeur de l'intégrale de $f$ sur l'intervalle
$[a,b]$ est donnée par la formule suivante.
\[
\int_{a}^{b} f(x)\,\text{d}x \approx M_n = 2h \sum_{j=0}^{n-1}\,f(x_{2j+1})
\]
où $h=(b-a)/(2n)$ et $x_j = a+j\,h$ for $j=0$, $1$ \ldots, $2n$.

{\bfseries l'erreur de troncature}\index{Erreur de troncature} pour la
méthode du point milieu est
$\displaystyle \frac{f''(\xi)\,(b-a)}{6}\,h^2$.
\label{CMR}
\end{meth}

Comme $\xi$ à la proposition précédente est inconnu, l'erreur de
troncature a une utilité limitée.

\begin{egg}
Utilisons la méthode du point milieu pour estimer la valeur de
l'intégrale $\displaystyle \int_0^1 e^{x^2}\dx{x}$.  Nous allons
choisir le nombre de sous-intervalles de $[0,1]$ (i.e. la longueur de
$h$) pour obtenir une erreur de troncature inférieur à $10^{-4}$.

Dans la formula pour la méthode du point milieu, nous avons $a=0$, $b=1$,
$f(x) = e^{-x^2}$, $h= (b-a)/(2n) = 1/(2n)$ et $x_i = 0 +i h = ih$.
Il faut choisir $n$ tel que
\[
\left| \frac{f''(\xi)\,(b-a)}{6}\,h^2 \right|
\]
est plus petit que $10^{-4}$.

Comme nous ne connaissons pas $\xi$, nous remplaçons $|f''(\xi)|$ par
la plus grande valeur que $|f''(x)|$ peut prendre sur l'intervalle $[0,1]$.
Puisque
\[
|f''(x)| = (2+4x^2)e^{x^2} < 6e
\]
pour tout $x\in [0,1]$, nous pouvons utiliser $6e$ pour le maximum cherché.

Ainsi, il faut choisir $n$ tel que
\[
\left| \frac{f''(\xi)\,(b-a)}{6}\,h^2 \right|
\leq \frac{6e}{6}\, \left(\frac{1}{2n}\right)^2 
= \frac{e}{4n^2} < 10^{-4} \, ;
\]
c'est-à-dire,
\[
n > \sqrt{ \frac{e}{4} 10^4} = \frac{10^2}{2} \sqrt{e} \approx 82.436 \; .
\]
Un bon choix est $n=83$.  Avec cette valeur de $n$,
$\displaystyle h = \frac{1}{2n} = \frac{1}{166}$ et
$\displaystyle x_{2i+1} = \frac{2i+1}{166}$ pour $i=0$, $1$, $2$, \ldots, $82$,
Nous obtenons
\begin{align*}
\int_0^1 e^{-x^2}\dx{x} & \approx \frac{2}{166} \sum_{j=0}^{82}\,
e^{((2i+1)/166)^2} \\
&\approx \frac{1}{83} \left( e^{(1/166)^2} + e^{(3/166)^2} +
e^{(5/166)^2} + \ldots + e^{(163/166)^2} + e^{(165/166)^2}\right) \\
&\approx 1.4626189 \; .
\end{align*}
\end{egg}

\begin{rmk}
Si $f$ est une fonction convexe, la méthode du point milieu donne une
sous-estimation de la valeur de l'intégrale.  Si $f$ est une fonction
concave, la méthode du point milieu donne une surestimation de la
valeur de l'intégrale.  Nous pouvons justifier cette observation de deux
façons.

Si $f$ est convexe, nous avons que $f''(x) >0$ pour tout $x$.  Ainsi,
l'erreur de troncature\\
$\displaystyle \frac{f''(\xi)\,(b-a)}{6}\,h^2$ est positif.
Par contre si, $f$ est concave, nous avons que $f''(x) <0$ pour tout
$x$.  Ainsi, l'erreur de troncature
$\displaystyle \frac{f''(\xi)\,(b-a)}{6}\,h^2$ est négatif.

Nous pouvons aussi justifier notre remarque à l'aide des graphes qui
sont donnés aux figures~\ref{MIDPOINTF} et \ref{MIDPOINTFF}.

À la figure~\ref{MIDPOINTF}, nous avons le graphe d'une fonction
concave.  L'aire du rectangle\\
\mbox{\rectangle$CDEG$} dont la base est
l'intervalle $[x_{2i},x_{2i+2}]$ et la hauteur est $f(x_{2i+1})$ est
égale à l'aire du trapèze \mbox{\trapeze$ABCD$} car les triangles
$\triangle AEF$ et $\triangle BGF$ sont similaires.  Comme la surface
sous la courbe $y=f(x)$ pour $x_{2i} \leq x \leq x_{2i+2}$ est
complètement recouverte par le trapèze \mbox{\trapeze$ABCD$}, l'aire
sous la courbe $y=f(x)$ est donc inférieure ou égale à l'aire du
rectangle dont la base est l'intervalle $[x_{2i},x_{2i+2}]$ et la
hauteur est $f(x_{2i+1})$.  Puisque cela est vrai pour tous les
sous-intervalles de $[a,b]$ utilisés par la méthode du point milieu et
puisque que l'aire sous la courbe $y=f(x)$ représente l'intégrale de
$f$, nous avons que
\[
\int_a^b f(x) \dx{x} \leq 2h \sum_{i=0}^{n-1} f(x_{2i+1}) \; .
\]

À la figure~\ref{MIDPOINTFF}, nous avons le graphe d'une fonction
convexe.  L'aire du rectangle\\
\mbox{\rectangle$CDEG$} dont la base est
l'intervalle $[x_{2i},x_{2i+2}]$ et la hauteur est $f(x_{2i+1})$ est
égale à l'aire du trapèze \mbox{\trapeze$ABCD$} car les triangles
$\triangle AEF$ et $\triangle BGF$ sont similaires.  Comme la surface
sous la courbe $y=f(x)$ pour $x_{2i} \leq x \leq x_{2i+2}$ recouvre
complètement le trapèze \mbox{\trapeze$ABCD$}, l'aire sous la courbe
$y=f(x)$ est
donc supérieure à l'aire du rectangle dont la base est l'intervalle
$[x_{2i},x_{2i+2}]$ et la hauteur est $f(x_{2i+1})$.  Puisque cela est
vrai pour tous les sous-intervalles de $[a,b]$ utilisé par la méthode
du point milieu et puisque que l'aire sous la courbe $y=f(x)$
représente l'intégrale de $f$, nous avons que
\[
\int_a^b f(x) \dx{x} \geq 2h \sum_{i=0}^{n-1} f(x_{2i+1}) \; .
\]
\end{rmk}

\PDFfig{7_integrales/midpointF}{La méthode du point milieu surestime la
valeur de l'intégrale lorsque la fonction est concave}{La méthode du
point milieu surestime la valeur de l'intégrale lorsque la fonction
est concave comme c'est le cas pour la fonction $f$ ci-dessus.}{MIDPOINTF}

\PDFfig{7_integrales/midpointFF}{La méthode du point milieu sous-estime
la valeur de l'intégrale lorsque la fonction est convexe}{La méthode
du point milieu sous-estime la valeur de l'intégrale lorsque la
fonction est convexe comme c'est le cas pour la fonction $f$
ci-dessus.}{MIDPOINTFF}

\subsection{Méthode des trapèzes}

Soit $f:[a,b]\rightarrow \RR$ une fonction qui possède une dérivée
continue d'ordre deux.

Nous partageons l'intervalle $[a,b]$ en $n$ sous-intervalles
où $n$ est un entier positif.  Soit $h = (b-a)/n$ et $x_i = a + i h$
pour $i=0$, $1$, $2$, \ldots, $n$.  L'intervalle $[a,b]$ est l'union
des $n$ sous-intervalles de la forme $[x_i,x_{i+1}]$ pour $i=0$, $1$,
$2$, \ldots, $n-1$.

Nous définissons la fonction $p:[a,b] \rightarrow \RR$ de la façon suivante.
\[
p(x) = \frac{f(x_{i+1})-f(x_i)}{x_{i+1}-x_i} \, (x-x_i) + f(x_i)
\quad \text{si} \quad x_{i} \leq x < x_{i+1} \; .
\]
C'est une fonction qui est linéaire par morceau.  Sur un intervalle de
la forme $[x_{i},x_{i+1}]$, le graphe de la fonction $p$ est la droite
qui lie les points $(x_i,f(x_i))$ et $(x_{i+1},f(x_{i+1}))$
(figure~\ref{TRAPEZE}). 

\PDFfig{7_integrales/trapeze}{Méthode des trapèzes pour évaluer
numériquement une intégrale}{Méthode des trapèzes pour évaluer
numériquement une intégrale}{TRAPEZE}

Nous avons que
\begin{align*}
\int_{x_i}^{x_{i+1}} p(x) \dx{x} &= 
\frac{f(x_{i+1})-f(x_i)}{x_{i+1}-x_i} \, \int_{x_i}^{x_{i+1}} (x-x_i) \dx{x}
+ \int_{x_i}^{x_{i+1}} f(x_i) \dx{x} \\
&= \frac{f(x_{i+1})-f(x_i)}{x_{i+1}-x_i} \,
\frac{(x-x_i)^2}{2}\bigg|_{x_i}^{x_{i+1}} +
f(x_i)\, \left( x_{i+1} - x_i\right) \\
&= \frac{f(x_{i+1})+f(x_i)}{2} \, \left( x_{i+1} - x_i\right) \; .
\end{align*}
Dans le cas où $f(x_i)>0$ et $f(x_{i+1})>0$, c'est la formule
pour calculer l'aire du trapèze dont la base est l'intervalle
$[x_i,x_{i+1}]$ de longueur $h$ et les côtés adjacents sont de
longueurs $f(x_i)$ et $f(x_{i+1})$.

Nous avons donc que
\begin{align*}
\int_a^b p(x) \dx{x} &= \sum_{i=0}^{n-1} \int_{x_i}^{x_{i+1}}p(x)  \dx{x}
= \sum_{i=0}^{n-1} h \,\frac{f(x_{i})+f(x_{i+1})}{2} \\
&= h\,\left( \frac{f(x_0)+f(x_1)}{2}
+\frac{f(x_1)+f(x_2)}{2} + \frac{f(x_2)+f(x_3)}{2} + \ldots \right. \\
&\qquad \left. +\frac{f(x_{n-2})+f(x_{n-1})}{2} +\frac{f(x_{n-1})+f(x_n)}{2}
 \right) \\
&=\frac{h}{2}\,\left( f(x_0) + 2 \sum_{i=1}^{n-1} f(x_i) + f(x_n) \right) \; .
\end{align*}

Nous pourrions démontrer le théorème suivant.

\begin{prop}
Soit $f:[a,b]\rightarrow \RR$ une fonction qui possède une dérivée
continue d'ordre deux.  Soit $h=(b-a)/n$ et $x_j = a+j\,h$ pour $j=0$,
$1$ \ldots, $n$.  Alors
\[
\int_{a}^{b} f(x)\dx{x} = \frac{h}{2} \left( f(x_0) + 2\sum_{j=1}^{n-1}\,f(x_j)
+ f(x_n) \right) - \frac{f''(\xi)\,(b-a)}{12}\,h^2
\]
pour une valeur $\xi \in [a,b]$.
\end{prop}

Cette proposition nous donne notre deuxième méthode d'intégration
numérique.

\begin{meth}[Méthode des trapèzes]
\index{Méthode des trapèzes}
Une approximation de la valeur de l'intégrale de $f$ sur l'intervalle
$[a,b]$ est donnée par la formule suivante.
\[
\int_{a}^{b} f(x)\,\text{d}x \approx T_n =
\frac{h}{2} \left( f(x_0) + 2 \sum_{j=1}^{n-1}\,f(x_j) + f(x_n) \right)
\]
où $h=(b-a)/n$ et $x_j = a+j\,h$ for $j=0$, $1$ \ldots, $n$.

{\bfseries L'erreur de troncature}\index{Erreur de troncature}
pour la méthode des trapèzes est
$\displaystyle - \frac{f''(\xi)\,(b-a)}{12}\,h^2$.
\label{CTR}
\end{meth}

\begin{rmk}
Nous pouvons exprimer la formule d'approximation pour la méthode du trapèze
à l'aide des sommes à droite et à gauche.  En effet, $T_n = ( G_n + D_n)/2$.
\end{rmk}

\begin{egg}
Utilisons la méthode des trapèzes pour estimer l'intégrale
$\displaystyle \int_0^1 e^{x^2}\dx{x}$.  Nous allons choisir le nombre de
sous-intervalles de $[0,1]$ pour que l'erreur de troncature soit
inférieur à $10^{-4}$.

Dans la formula pour la méthode des trapèzes, nous avons $a=0$, $b=1$,
$f(x) = e^{-x^2}$, $h= (b-a)/n = 1/n$ et $x_i = 0 +i h = ih$.  Il faut
trouver $n$ tel que
\[
\left| \frac{f''(\xi)\,(b-a)}{12}\,h^2 \right|
\]
est plus petit que $10^{-4}$.

Comme nous ne connaissons pas $\xi$, nous remplaçons $|f''(\xi)|$ par la plus
grande valeur que $|f''(x)|$ peut prendre sur l'intervalle $[0,1]$.
Puisque
\[
|f''(x)| = (2+4x^2)e^{x^2} < 6e
\]
pour tout $x\in [0,1]$, nous pouvons utiliser $6e$ pour le maximum
cherché.

Ainsi, il faut choisir $n$ tel que
\[
\left| \frac{f''(\xi)\,(b-a)}{12}\,h^2 \right|
\leq \frac{6e}{12}\, \left(\frac{1}{n}\right)^2 
= \frac{e}{2n^2} < 10^{-4} \, ;
\]
c'est-à-dire,
\[
n > \sqrt{ \frac{e}{2} 10^4} = 10^2 \sqrt{\frac{e}{2}} \approx
116.5821 \ .
\]
Un bon choix est $n=117$.  Avec cette valeur de $n$, 
$\displaystyle h = \frac{1}{n} = \frac{1}{117}$ et
$x_i = \frac{i}{117}$ pour $i=0$, $1$, $2$, \ldots, $117$.  Nous obtenons
\begin{align*}
\int_0^1 e^{-x^2}\dx{x} & \approx \frac{1}{234} \left(
e^0 + 2\,\sum_{j=1}^{116}\, e^{(i/117)^2} + e^1 \right) \\
&\approx \frac{1}{234} \left( 1 + 2\,\left(e^{(1/117)^2} +
e^{(2/1117)^2} + \ldots + e^{(116/117)^2}\right) + e \right) \\
&\approx 1.46268 \ .
\end{align*}
\end{egg}

\begin{rmk}
Comme nous avons fait pour la méthode du point milieu, il est possible
d'analyser l'effet de la courbure d'une fonction pour déterminer si
la méthode des trapèzes donnera une surestimation ou une
sous-estimation de la valeur de l'intégrale.

Si $f$ est une fonction convexe, la méthode des trapèzes donne
une surestimation de la valeur de l'intégrale.  Si $f$ est une
fonction concave, la méthode des trapèzes donne une sous-estimation de
la valeur de l'intégrale.  Nous pouvons justifier cette observation de deux
façons.

Si $f$ est convexe, nous avons que $f''(x) >0$ pour tout $x$.  Ainsi,
l'erreur de troncature\\
$\displaystyle -\frac{f''(\xi)\,(b-a)}{12}\,h^2$ est
négatif.  Par contre si, $f$ est concave, nous avons que $f''(x) <0$
pour tout $x$.  Ainsi, l'erreur de troncature
$\displaystyle -\frac{f''(\xi)\,(b-a)}{12}\,h^2$ est positif.

Nous pouvons aussi justifier notre remarque à l'aide des graphes que
nous retrouvons à la figure~\ref{TRAPEZEF}.

Le graphe de gauche qui est donné à la figure~\ref{TRAPEZEF} représente une
fonction concave.  Le trapèze \mbox{\trapeze$ABCD$} est le trapèze dont la
base est l'intervalle $[x_i,x_{i+1}]$ et les deux cotés adjacents à la
base sont de longueurs $f(x_i)$ et $f(x_{i+1})$.  Puisque le trapèze
\mbox{\trapeze$ABCD$} est complètement recouvert par la région sous la courbe
$y=f(x)$ pour $x_i \leq x \leq x_{i+1}$, nous avons que l'aire du trapèze
est plus petite ou égale à l'aire sous la courbe $y=f(x)$.  Puisque
cela est vrai pour tous les sous-intervalles de $[a,b]$ utilisé par la
méthode des trapèzes et puisque que l'aire sous la courbe $y=f(x)$
représente l'intégrale de $f$, nous avons que
\[
\int_a^b f(x) \dx{x} \geq
\frac{h}{2} \left( f(x_0) + \sum_{j=1}^{n-1}\,f(x_j) + f(x_n) \right) \; .
\]

Le graphe de droite qui est donné à la figure~\ref{TRAPEZEF} représente une
fonction convexe.  Comme dans le cas précédent, le trapèze
\mbox{\trapeze$ABCD$} est le trapèze dont la base est l'intervalle
$[x_i,x_{i+1}]$ et les deux cotés adjacents à la base sont de
longueurs $f(x_i)$ et $f(x_{i+1})$.  Puisque
le trapèze \mbox{\trapeze$ABCD$} recouvre complètement la région sous la
courbe $y=f(x)$ pour $x_i \leq x \leq x_{i+1}$, nous avons que l'aire du
trapèze est plus grande ou égale à l'aire sous la courbe $y=f(x)$.
Puisque cela est vrai pour tous les sous-intervalles de $[a,b]$
utilisé par la méthode des trapèzes et puisque que l'aire sous la
courbe $y=f(x)$ représente l'intégrale de $f$, nous avons que
\[
\int_a^b f(x) \dx{x} \leq 
\frac{h}{2} \left( f(x_0) + \sum_{j=1}^{n-1}\,f(x_j) + f(x_n) \right) \; .
\]
\end{rmk}

\PDFfig{7_integrales/trapezeF}{La concavité de l'intégrande détermine si
la méthode des trapèzes sous-estime ou surestime la valeur de
l'intégrale}{La courbure de la fonction $f$ détermine si la méthode
des trapèzes donne une surestimation ou une sous-estimation de la
valeur exacte de l'intégrale de $f$.  Dans le cas d'une fonction
concave, nous avons une sous-estimation de la valeur de l'intégrale car la
fonction.  Par contre, dans le cas d'une fonction convexe, nous avons une
surestimation de la valeur de l'intégrale.}{TRAPEZEF}

L'idée de majoré une fonction $f$ par une autre fonction $g$ pour
obtenir la relation
\[
\int_a^\infty f(x) \dx{x}  \leq \int_a^\infty g(x) \dx{x}
\]
peut être utilisée pour estimer la valeur d'intégrales impropres.

\begin{egg}
Estimons la valeur de l'intégrale
$\displaystyle \int_1^\infty e^{-x^2/2} \dx{x}$ avec une précision de
$10^{-4}$.

Il y a deux étapes pour trouver cette approximation.

\subQ{i} Puisque que
\[
\int_1^\infty e^{-x^2/2} \dx{x}
= \int_1^c e^{-x^2/2} \dx{x} + \int_c^\infty e^{-x^2/2} \dx{x}
\]
nous cherchons une valeur $c$ telle que
\[
0 \leq \int_c^\infty e^{-x^2/2} \dx{x} < \frac{1}{2} \, 10^{-4} \; .
\]

\subQ{ii} Lorsque nous aurons $c$, nous utiliserons une méthode numérique
pour trouver une approximation $I$ de
$\displaystyle \int_1^c e^{-x^2/2} \dx{x}$ telle que
\[
\bigg| \int_1^c e^{-x^2/2} \dx{x} - I \bigg| < \frac{1}{2} \, 10^{-4} \; .
\]
Nous aurons alors que
\begin{align*}
\bigg| \int_1^{\infty} e^{-x^2/2} \dx{x} - I \bigg| 
& = \bigg| \int_1^c e^{-x^2/2} \dx{x} +
\int_c^{\infty} e^{-x^2/2} \dx{x} - I \bigg| \\
& \leq \bigg| \int_1^c e^{-x^2/2} \dx{x} - I \bigg| +
\bigg| \int_c^{\infty} e^{-x^2/2} \dx{x} \bigg|
< \frac{1}{2} \, 10^{-4} + \frac{1}{2} \, 10^{-4} = 10^{-4} \; .
\end{align*}
La valeur $I$ est l'approximation cherchée.

\subQ{i} Comme aucune de nos techniques d'intégration peut être
utilisée pour évaluer l'intégrale
\[
\int_c^\infty e^{-x^2/2} \dx{x} \; ,
\]
il faut majorer cette intégrale par une intégrale que nous pouvons
évaluer.

Pour $x\geq 2$, nous avons $x^2 \geq 2x$.  Ainsi, $-x \geq -x^2/2$ pour
$x\geq 2$ et
\begin{align*}
0\leq \int_c^\infty e^{-x^2/2} \dx{x} & \leq \int_c^\infty e^{-x} \dx{x}
= \lim_{q\rightarrow \infty} \int_c^q e^{-x} \dx{x} \\
&= \lim_{q\rightarrow \infty} \left( -e^{-x} \right)\big|_c^q
= \lim_{q\rightarrow \infty} \left( -e^{-q} + e^{-c} \right)
= e^{-c}
\end{align*}
pour $c \geq 2$.  Nous prenons $c$ tel que
\[
e^{-c} < \frac{1}{2} \, 10^{-4} \, ;
\]
c'est-à-dire,
\[
c > - \ln\left(\frac{1}{2} \, 10^{-4}\right)
= -\ln\left(2^{-1}\right) - \ln\left(10^{-4}\right)
= \ln(2) + 4\ln(10) \approx 9.903487 \ .
\]
Donc $c=10$ est un bon choix.

\subQ{ii} Nous utilisons la méthode des trapèzes pour trouver une première
approximation de l'intégrale
$\displaystyle \int_1^{10} e^{-x^2/2} \dx{x}$.

Si nous prenons $n = 600$ dans la formule pour la méthode des trapèzes,
puisque $a=1$ et $b=10$, nous obtenons $h = 9/600 = 3/200$,
$x_i = 1+ 3i/200$ pour $i=0$, $1$, $2$, \ldots, $600$ et
\[
\int_1^{10} e^{-x^2/2} \dx{x} \approx I_1 = 
\frac{3}{400} \left( e^{-1/2} + 2 \, \sum_{j=1}^{599}\,e^{(1+3j/200)^2/2} +
e^{10^2/2} \right) \approx 0.397701117958515 \ .
\]
Comme $e^{-x^2/2}$ est convexe, $I_1$ est une surestimation de
la valeur exacte de l'intégrale.

Nous utilisons la méthode du point milieu pour trouver une deuxième
approximation de l'intégrale
$\displaystyle \int_1^{10} e^{-x^2/2} \dx{x}$.

Si nous prenons $n = 600$ dans la formule pour la méthode du point milieu,
puisque $a=1$ et $b=10$, nous obtenons $h = 9/1200 = 3/400$,
$x_i = 1+ 3i/400$ pour $i=0$, $1$, $2$, \ldots, $1200$ et
\[
\int_1^{10} e^{-x^2/2} \dx{x} \approx I_2 = 
\frac{3}{200}\, \sum_{j=0}^{599}\,e^{(1+3(2j+1)/400)^2/2}
\approx 0.397684059123784 \ .
\]
Comme $e^{-x^2/2}$ est convexe, $I_2$ est une sous-estimation de
la valeur exacte de l'intégrale.

Nous avons
\[
0.3976840 \leq \int_1^{10} e^{-x^2/2} \dx{x} \leq 0.3977012 \; .
\]
Puisque
\[
0.3977012 - 0.3976840 = 0.172 \times 10^{-4} < \frac{1}{2} \, 10^{-4} \; ,
\]
nous pouvons prendre $I= (0.3977012 + 0.3976840)/2 = 0.3976926$ comme
approximation de $\displaystyle \int_1^\infty e^{-x^2/2} \dx{x}$.

Nous avons assumé que les calculs numériques étaient exactes.
Ce n'est généralement pas le cas.  Il faudrait considéré les
\lgm round-off errors\rgm\ qui sont dû à l'ordinateur utilisé pour
effectuer les calculs.

Pour être plus rigoureux, nous aurions pu utiliser nos formules pour
calculer l'erreur de troncature afin de choisir une valeur de $n$ qui
soit possiblement plus petite que $600$.  Nous avons ici simplement choisit
une valeur de $n$ au hasard en espérant qu'elle soit assez grande pour
obtenir la précision requise.
\end{egg}

\subsection{Méthode de Simpson}

Soit $f:[a,b]\rightarrow \RR$ une fonction qui possède une dérivée
d'ordre quatre qui est continue.

Nous partageons l'intervalle $[a,b]$ en $2n$ sous-intervalles
où $n$ est un entier positif.  Soit $h = (b-a)/(2n)$ et $x_i = a + i h$
pour $i=0$, $1$, $2$, \ldots, $2n$.  L'intervalle $[a,b]$ est l'union
des $2n$ sous-intervalles de la forme $[x_i,x_{i+1}]$ pour $i=0$, $1$,
$2$, \ldots, $2n-1$.

Nous définissons la fonction $p:[a,b] \rightarrow \RR$ suivante.  Pour
$x_{2i} \leq x < x_{2i+2}$, nous définissons
\begin{equation}\label{interp_quadr}
m_1 = \frac{f(x_{2i+1})-f(x_{2i})}{x_{2i+1}-x_{2i}} \ , \ 
m_2 = \frac{f(x_{2i+2})-f(x_{2i+1})}{x_{2i+2}-x_{2i+1}} \ , \ 
m_3 = \frac{m2-m1}{x_{2i+2} - x_{2i}}
\end{equation}
et nous posons
\[
p(x) = m_3\left(x-x_{2i}\right)\left(x-x_{2i+1}\right)
+ m_1 \left(x-x_{2i}\right) + f(x_{2i}) \; .
\]
C'est une fonction quadratique par morceau.  Sur un intervalle de la
forme $[x_{2i},x_{2i+2}]$, le graphe de la fonction $p$ est le graphe
du polynôme de degré $2$ qui passe par les points
$\left(x_{2i},f(x_{2i})\right)$, 
$\left(x_{2i+1},f(x_{2i+1})\right)$ et $\left(x_{2i+2}, f(x_{2i+2})\right)$
(figure~\ref{SIMPSON}).

Il est intéressant de noter que grâce au théorème fondamental de
l'algèbre\footnote{Un polynôme de degré $n$ a exactement $n$ racines
complexes si nous incluons les racines multiples.}, il n'y a qu'un seul
polynôme de degré deux qui passe par les trois points
$\left(x_{2i}, f(x_{2i})\right)$,
$\left(x_{2i+1}, f(x_{2i+1})\right)$ et $\left(x_{2i+2}, f(x_{2i+2})\right)$.

\PDFfig{7_integrales/simpson}{Méthode de Simpson pour évaluer
numériquement une intégrale}{Méthode de Simpson pour évaluer
numériquement une intégrale.  Nous utilisons le polynôme de degré $2$ qui
passe par les points
$\left(x_{2i}, f(x_{2i})\right)$, $\left(x_{2i+1}, f(x_{2i+1})\right)$ et
$\left(x_{2i+2}, f(x_{2i+2})\right)$ pour estimer $f(x)$ lorsque
$x_{2i} \leq x < x_{2i+2}$}{SIMPSON}

L'intégrale du polynôme
\[
p(x) = m_3\left(x-x_{2i}\right)\left(x-x_{2i+1}\right)
+ m_1 \left(x-x_{2i}\right) + f(x_{2i})
\]
sur l'intervalle $[x_{2i}, x_{2i+2}]$ est
\begin{align*}
&\int_{x_{2i}}^{x_{2i+2}} p(x) \dx{x} \\
&= m_3 \int_{x_{2i}}^{x_{2i+2}}
\left(x-x_{2i}\right)\left(x-x_{2i+1}\right) \dx{x}
+ m_1 \int_{x_{2i}}^{x_{2i+2}} \left(x-x_{2i}\right) \dx{x} +
\int_{x_{2i}}^{x_{2i+2}} \! f(x_{2i}) \dx{x} \\
&= m_3 \,\left( \frac{x^3}{3} - \frac{x_{2i}+x_{2i+1}}{2}\, x^2 +
x_{2i}x_{2i+1} \, x\right)\bigg|_{x_{2i}}^{x_{2i+2}}
+m_1 \frac{\left(x-x_{2i}\right)^2}{2}\bigg|_{x_{2i}}^{x_{2i+2}} \\
&\qquad + f(x_{2i}) \left(x-x_{2i}\right)\bigg|_{x_{2i}}^{x_{2i+2}} \; .
\end{align*}
Si nous substituons les expressions pour $m_1$ et $m_3$ données en
(\ref{interp_quadr}) dans la formule ci-dessus et si nous simplifions le
résultat, nous obtenons, après un long calcul que nous laissons aux lecteurs
le soin d'effectuer, la formule suivante
\[
\int_{x_{2i}}^{x_{2i+2}} p(x) \dx{x} =
\frac{h}{3} \left(f(x_{2i}) + 4 f(x_{2j+1}) + f(x_{2j+2}) \right) \; .
\]

Nous avons donc que
\begin{align*}
\int_a^b p(x) \dx{x} &= \sum_{j=0}^{n-1} \int_{x_{2j}}^{x_{2j+2}}p(x)  \dx{x}
= \sum_{i=0}^{n-1} \frac{h}{3}
\bigg(f(x_{2i}) + 4 f(x_{2j+1}) + f(x_{2j+2}) \bigg) \\
&= \frac{h}{3} \left(f(x_0) + 2\sum_{j=1}^{n-1}\,f(x_{2j}) + 4
\sum_{j=0}^{n-1}\,f(x_{2j+1}) + f(x_{2n}) \right) \; .
\end{align*}

Nous pourrions démontrer le théorème suivant.

\begin{prop}
Soit $f:[a,b]\rightarrow \RR$ une fonction qui possède une dérivée
continue d'ordre quatre.  Soit $h=(b-a)/(2n)$ et $x_j = a+j\,h$ pour
$j=0$, $1$ \ldots, $2n$.  Alors
\begin{align*}
\int_{a}^{b} f(x)\dx{x} &= \frac{h}{3} \left(f(x_0)
+ 2\sum_{j=1}^{n-1}\,f(x_{2j}) + 4 \sum_{j=0}^{n-1}\,f(x_{2j+1})
+ f(x_{2n}) \right) \\
& \qquad \qquad - \frac{f^{(4)}(\xi)\,(b-a)}{180}\, h^4
\end{align*}
pour une valeur $\xi \in [a,b]$.
\end{prop}

Cette proposition nous donne notre troisième et dernière méthode
d'intégration numérique.

\begin{meth}[Méthode de Simpson] \index{Méthode de Simpson}
Une approximation de la valeur de l'intégrale de $f$ sur l'intervalle
$[a,b]$ est donnée par la formule suivante.
\[
\int_{a}^{b} f(x)\,\text{d}x \approx  S_n =
\frac{h}{3} \left(f(x_0) + 2\sum_{j=1}^{n-1}\,f(x_{2j}) + 4
\sum_{j=0}^{n-1}\,f(x_{2j+1}) + f(x_{2n}) \right)
\]
où $h=(b-a)/(2n)$ et $x_j = a+j\,h$ for $j=0$, $1$ \ldots, $2n$.

{\bfseries L'erreur de troncature}\index{Erreur de troncature}
pour la méthode de Simpson est
$\displaystyle  - \frac{f^{(4)}(\xi)\,(b-a)}{180}\, h^4 $.
\label{CSR}
\end{meth}

\begin{rmk}
Nous pouvons exprimer la formule d'approximation pour la méthode de Simpson
à l'aide des formules d'approximation $M_n$ pour la méthode du point
milieu et $T_n$ pour la méthode des trapèzes.  En effet,
$S_n = (2 M_{n/2} + T_n)/3$.  Pour la méthode des trapèzes, nous utilisons
seulement les intervalles de la forme $[x_{2j},x_{2j+2}]$.
\end{rmk}

\begin{egg}
Utilisons la méthode de Simpson pour estimer l'intégrale
$\displaystyle \int_0^1 e^{x^2}\dx{x}$.  Nous allons choisir le nombre de
sous-intervalles de $[0,1]$ pour que l'erreur de troncature soit
inférieur à $10^{-4}$.

Dans la formula pour la méthode de Simpson, nous posons $a=0$, $b=1$,
$f(x) = e^{-x^2}$, $h= (b-a)/(2n) = 1/(2n)$ et $x_i = 0 +i h = i h$.
Il faut choisir $n$ tel que
\[
\left| \frac{f^{(4)}(\xi)\,(b-a)}{180}\, h^4 \right|
\]
soit plus petit que $10^{-4}$.

Comme nous ne connaissons pas $\xi$, nous remplaçons $|f^{(4)}(\xi)|$ par la
plus grande valeur que $|f^{(4)}(x)|$ peut prendre sur l'intervalle
$[0,1]$.  Puisque
\[
|f^{(4)}(x)| = 4e^{x^2}\left( 3 + 12\,x^2 + 4\,x^4 \right) \leq 76 e
\]
pour tout $x\in [0,1]$, nous pouvons utilisé $76 e$ pour le maximum
cherché.

Ainsi, il faut choisir $n$ tel que
\[
\left| \frac{f^{(4)}(\xi)\,(b-a)}{180}\, h^4 \right|
\leq \frac{76e}{180}\, \left(\frac{1}{2n}\right)^4 
= \frac{19 e}{720 n^4} < 10^{-4} \, ;
\]
c'est-à-dire,
\[
n > \left( \frac{19 e}{720} 10^4\right)^{1/4}
= 5 \left(\frac{19 e}{45}\right)^{1/4} \approx 5.175220 \; .
\]
Un bon choix est $n=6$.  Avec cette valeur de $n$,
$\displaystyle h = \frac{1}{2n} = \frac{1}{12}$ et
$\displaystyle x_i = \frac{i}{12}$ pour $i=0$, $1$, $2$, \ldots, $12$.
Nous obtenons
\begin{align*}
\int_0^1 e^{-x^2}\dx{x} & \approx 
\frac{1}{36}\left(e^0 + 2\sum_{j=1}^{5}\,e^{(j/6)^2}
+ 4 \sum_{j=0}^{5}\,e^{((2j+1)/12)^2} + e^1\right) \\
&= \frac{1}{36}\left( 1 + 2 \left( e^{(2/12)^2} + e^{(4/12)^2} +\ldots
+ e^{(10/12)^2}\right) \right. \\
& \qquad \left. + 4 \left(e^{(1/12)^2} + e^{(3/12)^2} + \ldots
+ e^{(11/12)^2} \right) + e \right)
\approx 1.46267 \; .
\end{align*}
\label{S1}
\end{egg}

\begin{rmkList}
\begin{enumerate}
\item Comme nous avons vu, pour déterminer si l'estimation fournie par la
méthode du point milieu ou la méthode des trapèzes est une
surestimation ou une sous-estimation, il suffi de déterminer la
courbure (convexe ou concave) du graphe de l'intégrande; c'est à dire,
le signe de la dérivée seconde de l'intégrande.
Ce n'est plus le cas pour la méthode de Simpson.  Puisque l'erreur de
troncature est
\[
- \frac{f^{(4)}(\xi)\,(b-a)}{180}\, h^4  \; ,
\]
il faut considérer le signe de la quatrième dérivée de l'intégrande.
Nous ne ferons pas cette analyse.
\item Puisqu'une fonction peut avoir un graphe qui change de courbure
lorsque la variable indépendante varie, il n'est souvent pas possible
d'utiliser la courbure pour déterminer si la méthode du point milieu
ou la méthode des trapèzes vont donner une sous-estimation ou une
surestimation de la valeur de l'intégrale.  De même, Il est rarement
possible d'utiliser le signe de la quatrième dérivée de l'intégrande
pour déterminer si la méthode de Simpson va donner une sous-estimation
ou une surestimation de la valeur de l'intégrale car le signe de cette
dérivée peut changer lorsque la valeur de la variable indépendante
varie.
\item Pour une même précision, la méthode de Simpson demande de
subdiviser l'intervalle d'intégration en beaucoup moins de
sous-intervalles que les méthodes du point milieu et des
trapèzes. Pour justifier cette remarque, il suffit de considérer
l'erreur de troncature pour les trois méthodes que nous avons présentées.

Si nous supposons que $f''$ et $f^{(4)}$ varient lentement par rapport à
la variable indépendante, nous pouvons comparer l'ordre de grandeur de
l'erreur de troncature pour les trois méthodes qui nous intéresse.
Pour $h$ donné, l'erreur de troncature pour les méthodes du point
milieu et des trapèzes est de l'ordre de $h^2$, alors qu'elle est de
l'ordre de $h^4$ pour la méthode de Simpson.  Lorsque $h$ diminue
(i.e.\ $h \to 0$), l'erreur de troncature de la méthode de Simpson
diminue beaucoup plus rapidement que celle pour les méthodes du point
milieu ou des trapèzes.  C'est ce qui fait que la méthode de Simpson
est généralement supérieure au deux autres méthodes.

Des trois méthodes que nous avons étudiées, c'est la méthode de Simpson
qui demande le moins de sous-intervalles et donc le moins d'opération
arithmétiques pour estimer une intégrale avec une précision donnée. De
plus, si nous tenons compte des \lgm round-off errors\rgm\ lors des
calculs sur ordinateurs, il est préférable d'utiliser une méthode qui
demande moins d'opérations arithmétiques pour minimiser ce type d'erreurs.
\end{enumerate}
\end{rmkList}

}  % End of theory

\section{Exercices}

\subsection{Intégrales indéfinies}

\begin{question}
Évaluez les intégrales indéfinies suivantes.
\begin{center}
\begin{tabular}{*{2}{l@{\hspace{0.5em}}l@{\hspace{3em}}}l@{\hspace{0.5em}}l}
\subQ{a} & $\displaystyle \int -x^{-2} \dx{x}$ &
\subQ{b} & $\displaystyle \int \frac{10}{x^9} \dx{x}$ &
\subQ{c} & $\displaystyle \int \left(5z^{-1.2} -1.2\right) \dx{z}$ \\
\subQ{d} & $\displaystyle \int \left( \frac{1}{\sqrt[5]{t}}
+ 3t \right) \dx{t}$ &
\subQ{e} & $\displaystyle \int 2^x \dx{x}$ &
\subQ{f} & $\displaystyle \int \left( e^x + \frac{1}{x}\right) \dx{x}$ \\[1em]
\subQ{g} & $\displaystyle \int \frac{(3t+2)^2}{4t^2}\dx{t}$ &
\subQ{h} & $\displaystyle \int \frac{(x^{1/3} + 1)^2}{x^{2/3}} \dx{x}$ & &
\end{tabular}
\end{center}
\label{7Q1}
\end{question}

\begin{question}
Évaluez les intégrales indéfinies suivantes.
\begin{center}
\begin{tabular}{*{2}{l@{\hspace{0.5em}}l@{\hspace{3em}}}l@{\hspace{0.5em}}l}
\subQ{a} & $\displaystyle \int \frac{1}{1+4t} \dx{t}$ &
\subQ{b} & $\displaystyle \int \frac{1}{5-3x} \dx{x}$ &
\subQ{c} & $\displaystyle \int 3 e^{3x/7} \dx{x}$ \\[0.8em]
\subQ{d} & $\displaystyle \int \left( 1 + \frac{t}{3}\right)^7 \dx{t}$ &
\subQ{e} & $\displaystyle \int \frac{e^z}{1+ e^z} \dx{z}$ &
\subQ{f} & $\displaystyle \int 3y \sqrt{1+y^2} \dx{y}$ \\[0.8em]
\subQ{g} & $\displaystyle \int \frac{1}{x\ln(x)} \dx{x}$ &
\subQ{h} & $\displaystyle \int \frac{1+4x}{\sqrt{1+x+2x^2}} \dx{x}$ &
\subQ{i} & $\displaystyle \int x^2 \sqrt{x^3+1} \dx{x}$ \\[0.8em]
\subQ{j} & $\displaystyle \int e^t\left( 1 + e^t\right)^4 \dx{t}$ &
\subQ{k} & $\displaystyle \int \frac{1}{\sqrt{x}(1+\sqrt{x})^{11}}
\dx{x}$ &
\subQ{l} & $\displaystyle \int \frac{e^{1/t}}{t^2} \dx{t}$ \\[0.8em]
\subQ{m} & $\displaystyle \int \frac{\ln(x)}{x} \dx{x}$ &
\subQ{n} & $\displaystyle \int \frac{x+3}{x^2-9} \dx{x}$ &
\subQ{o} & $\displaystyle \int e^{t}\left( 2 + e^{2t}\right) \dx{t}$ \\[0.8em]
\subQ{p} & $\displaystyle \int \frac{t^{3/5}}{1+t^{2/5}} \dx{t}$ &
\subQ{q} & $\displaystyle \int \frac{x}{(1+x^2)^9}\dx{x}$ &
\subQ{r} & $\displaystyle \int \frac{x^3}{(1+x^2)^9}\dx{x}$
\end{tabular}
\end{center}
\label{7Q2}
\end{question}

\begin{question}[\eng \life]
Évaluez les intégrales indéfinies suivantes.
\begin{center}
\begin{tabular}{*{2}{l@{\hspace{0.5em}}l@{\hspace{3em}}}l@{\hspace{0.5em}}l}
\subQ{a} & $\displaystyle \int \cos(2\pi(x-2)) \dx{x}$ &
\subQ{b} & $\displaystyle \int \frac{\cos(\theta)}{1-\cos^2(\theta)}
\dx{\theta}$ &
\subQ{c} & $\displaystyle \int \frac{\sin(x)}{1-\sin^2{x}} \dx{x}$ \\[0.8em]
\subQ{d} & $\displaystyle \int \sqrt{x}\, \sin(x^{3/2}+1) \dx{x}$ &
\subQ{e} & $\displaystyle \int \frac{\cos(1/t)}{t^2} \dx{t}$ &
\subQ{f} & $\displaystyle \int \frac{1}{x^2+9} \dx{x}$ \\[0.8em]
\subQ{g} & $\displaystyle \int \frac{x^3+1}{x^2+3} \dx{x}$ &
\subQ{h} & $\displaystyle \int \frac{x}{\sqrt{4-x^4}} \dx{x}$ &
\subQ{i} & $\displaystyle \int \tan(x) \sec^4(x) \dx{x}$ \\[0.8em]
\subQ{j} & $\displaystyle \int \frac{1}{2 - \cos(x)} \dx{x}$ &
\subQ{k} & $\displaystyle \int x^2 \sin(x^3+1) \dx{x}$ &
\subQ{l} & $\displaystyle \int \frac{(2t+1)^2}{1+t^2}\dx{t}$ \\[0.8em]
\subQ{m} & $\displaystyle \int \sin^5(x) \cos^4(x) \dx{x}$ &
\subQ{n} & $\displaystyle \int \frac{\sin^5(x)}{\cos^4(x)} \dx{x}$ &
\subQ{o} & $\displaystyle \int \cos^3(x) \sin^2(x) \dx{x}$
\end{tabular}
\end{center}
\label{7Q3}
\end{question}

\begin{question}
Évaluez les intégrales indéfinies suivantes.
\begin{center}
\begin{tabular}{*{2}{l@{\hspace{0.5em}}l@{\hspace{3em}}}l@{\hspace{0.5em}}l}
\subQ{a} & $\displaystyle \int \ln\left(\sqrt{x}\right) \dx{x}$ &
\subQ{b} & $\displaystyle \int x^2 e^x \dx{x}$ &
\subQ{c} & $\displaystyle \int \frac{x}{e^{3x}} \dx{x}$ \\[0.8em]
\subQ{d} & $\displaystyle \int x^2 \, e^{-x} \dx{x}$ &
\subQ{e} & $\displaystyle \int x^3 \, e^{x^2} \dx{x}$ &
\subQ{f} & $\displaystyle \int (x^2+x^6)\ln(x) \dx{x}$  \\[0.8em]
\subQ{g} & $\displaystyle \int x^3\ln\left(x^2+2\right) \dx{x}$ &
& & &
\end{tabular}
\end{center}
\label{7Q4}
\end{question}

\begin{question}[\eng \life]
Évaluez les intégrales indéfinies suivantes.
\begin{center}
\begin{tabular}{*{2}{l@{\hspace{0.5em}}l@{\hspace{3em}}}l@{\hspace{0.5em}}l}
\subQ{a} & $\displaystyle \int \theta \cos(\pi\theta) \dx{\theta}$ &
\subQ{b} & $\displaystyle \int e^x \sin(x) \dx{x}$ &
\subQ{c} & $\displaystyle \int \theta^3 \sin(\theta^2) \dx{\theta}$
\end{tabular}
\end{center}
\label{7Q5}
\end{question}

\begin{question}
Évaluez les intégrales indéfinies suivantes.
\begin{center}
\begin{tabular}{*{2}{l@{\hspace{0.5em}}l@{\hspace{3em}}}l@{\hspace{0.5em}}l}
\subQ{a} & $\displaystyle \int \frac{x^4+3}{x^2-4x+3} \dx{x}$ &
\subQ{b} & $\displaystyle \int \frac{x-9}{(x+5)(x-2)} \dx{x}$ &
\subQ{c} & $\displaystyle \int \frac{1}{x^2+x+1} \dx{x}$ \\[0.8em]
\subQ{d} & $\displaystyle \int \frac{1}{t^2+6t+8} \dx{t}$ &
\subQ{e} & $\displaystyle \int \frac{1}{x^2-4x+13} \dx{x}$ &
\subQ{f} & $\displaystyle \int \frac{1}{x^2-9}\dx{x}$ \\[0.8em]
\subQ{g} & $\displaystyle \int \frac{1}{x^2 -x -2} \dx{x}$ &
\subQ{h} & $\displaystyle \int \frac{t^2+1}{t^2+3t+2} \dx{t}$ &
\subQ{i} & $\displaystyle \int \frac{x^3-x-2}{x^2-4} \dx{x}$ \\[0.8em]
\subQ{j} & $\displaystyle \int \frac{9x+29}{x^2+2x-15}\dx{x}$ &
& & &
\end{tabular}
\end{center}
\noindent Suggestion: Pour certaines des intégrales, vous pourriez avoir à
compléter le carré du dénominateur et à utiliser une substitution.
\label{7Q6}
\end{question}

\begin{question}[\eng]
Évaluez chacune des intégrales indéfinies suivantes.
\begin{center}
\begin{tabular}{*{1}{l@{\hspace{0.5em}}l@{\hspace{6em}}}l@{\hspace{0.5em}}l}
\subQ{a} & $\displaystyle \int \frac{x^2-x+6}{x^3+3x} \dx{x}$ &
\subQ{b} & $\displaystyle \int \frac{x^2 -3x -23}{(x-2)(x+3)^2} \dx{x}$
\end{tabular}
\end{center}
\label{7Q7}
\end{question}

\begin{question}
Trouvez toutes les fonctions $f$ telles que
$\displaystyle f''(x) = 1 + x^{4/5}$.
\label{7Q8}
\end{question}

\subsection{Définition de l'intégrale définie}

\begin{question}
Donnez la somme de Riemann à droite et la somme de Riemann à gauche
pour l'intégrale $\displaystyle \int_0^1 (1+t^3)\dx{t}$ où
l'intervalle $[0,1]$ est subdivisée en $5$ sous-intervalles égaux.
\label{7Q9}
\end{question}

\begin{question}[\life]
La vitesse d'un abeille en vol a été mesuré à toutes les cinq secondes
pendant $50$ secondes.  Les résultats sont donnés dans le tableau suivant.
\[
\begin{array}{c|ccccccccccc}
t \text{ (s)} & 0 & 5 & 10 & 15 & 20 & 25 & 30 & 35 & 40 & 45 & 50 \\
\hline
v \text{ (m/s)} & \rule{0em}{1em} 12.7 & 12.2 & 11.8 & 11.5 & 11.3 & 11.2 &
11.2 & 11.3 & 11.6 & 12.0 & 12.5
\end{array}
\]
Supposons que l'abeille vole en ligne droite.  Utilisez une somme de Riemann
à droite pour estimer la distance parcourue par l'abeille.  Utilisez une
somme de Riemann à gauche pour estimer la distance parcourue par l'abeille.
\label{7Q10}
\end{question}

\begin{question}
Donnez la somme de Riemann à droite et celle à gauche pour l'intégrale
$\displaystyle \int_0^2 t^2 \dx{t}$ lorsque la partition de l'intervalle
$[0,2]$ comprend cinq sous-intervalles égaux.
\label{7Q11}
\end{question}

\begin{question}
Si $f(x) = \sqrt{x}-2$, calculez la somme de Riemann
pour le point milieu de l'intégrale
$\displaystyle \int_1^6 f(x)\dx{x}$ lorsque l'intervalle $[1,6]$ est
subdivisé en $n=5$ sous-intervalles égaux.  Tracez le graphe de $f$ et
les rectangles de la somme de Riemann.
\label{7Q12}
\end{question}

\begin{question}
Utilisez les sommes de Riemann pour le point milieu avec $n=5$
sous-intervalles égaux pour obtenir une approximation de la valeur de
l'intégrale
\[
\int_{2.5}^{10} \sin(\sqrt{x}) \dx{x} \ .
\]
Donnez votre réponse avec une précision de quatre chiffres décimaux.
\label{7Q13}
\end{question}

\begin{question}
Le tableau ci-dessous contient quelques valeurs d'une fonction croissante.
Utilisez une somme à droite et une somme à gauche pour trouver une
borne supérieure et une borne inférieure de l'intégrale
\[
\int_0^{25} f(x) \dx{x} \ .
\]
Utilisez le plus grand nombre de points possible pour chaque somme.
\begin{center}
\begin{tabular}{|c|c|c|c|c|c|c|}
\hline
$x$ & 0 & 5 & 10 & 15 & 20 & 25 \\
\hline
$f(x)$ & -42 & -37 & -25 & -6 & 15 & 36 \\
\hline
\end{tabular}
\end{center}
\label{7Q14}
\end{question}

\begin{question}
Un objet se déplace en ligne droite durant 8 secondes.  Le tableau
ci-dessous donnent la vitesse $v$ de l'objet en mètres par seconde
toutes les 2 secondes.
\begin{center}
\begin{tabular}{|c|c|c|c|c|c|}
\hline
$t$  (sec) & 0 & 2 & 4 & 6 & 8 \\
\hline
$v$  (m/s) & 10.0 & 9.5 & 9.0 & 8.0 & 6.0 \\
\hline
\end{tabular}
\end{center}
Comme nous pouvons voir, la vitesse est décroissante.  Répondre aux
questions suivantes à l'aide des sommes de Riemann à droite et
à gauche.

\subQ{a} Donnez une borne supérieure et un borne inférieure de la
distance parcourue pendant les 8 secondes.

\subQ{b} À quelle fréquence devons-nous mesurer la vitesse de l'objet
pour obtenir des bornes supérieures et inférieures qui soient à $0.1$
mètre de la distance parcourue pendant les 8 secondes?
\label{7Q15}
\end{question}

\begin{question}
La somme $\displaystyle 0.4 ( 0.4^3 + 0.8^3 + 1.2^3 + 1.6^3 + 2^3)$
est une somme de Riemann à droite pour une intégrale.  Quelle est
cette intégrale?
\label{7Q16}
\end{question}

\begin{question}
Exprimez la limite suivante comme une intégrale définie.
\[
\lim_{n\rightarrow \infty} \frac{\pi}{n} \sum_{j=1}^n
\left(\frac{\pi j}{n}\right) \sin\left(\frac{\pi j}{n}\right) \ .
\]
\label{7Q17}
\end{question}

\begin{question}
Donnez la somme de Riemann à droite pour l'intégrale
\[
\int_2^5 \sqrt{2 + x^{1/3}} \dx{x}
\]
où nous assumons que l'intervalle $[2,5]$ est subdivisé en $N$
sous-intervalles égaux.
\label{7Q18}
\end{question}

\begin{question}
Utilisez le graphe de la fonction $f$ donné ci-dessous pour évaluer
les intégrales suivantes.
\begin{center}
\begin{tabular}{*{1}{l@{\hspace{0.5em}}l@{\hspace{6em}}}l@{\hspace{0.5em}}l}
\subQ{a} & $\displaystyle \int_{4}^{9} f(x) \dx{x}$ &
\subQ{b} & $\displaystyle \int_{-2}^{2} f(x) \dx{x}$
\end{tabular}
\end{center}
\PDFgraph{7_integrales/area10}
\label{7Q19}
\end{question}

\begin{question}
Utilisez le graphe de la fonction $f$ donné ci-dessous pour évaluer
les intégrales suivantes.
\begin{center}
\begin{tabular}{*{1}{l@{\hspace{0.5em}}l@{\hspace{6em}}}l@{\hspace{0.5em}}l}
\subQ{a} & $\displaystyle \int_{-2}^{5} f(x) \dx{x}$ &
\subQ{b} & $\displaystyle \int_{0}^{10} f(x) \dx{x}$
\end{tabular}
\end{center}
\PDFgraph{7_integrales/integral5}
\label{7Q20}
\end{question}

\begin{question}
Le graphe de $f'$ est donné ci-dessous.
\PDFgraph{7_integrales/integral6}
Si $f(-1)=2$, quelle est la valeur de $f(2)$?
\label{7Q21}
\end{question}

\begin{question}
Le graphe de $f'$ est donné ci-dessous.
\PDFgraph{7_integrales/integral1}
Si $f(0)= 1$, tracez un graphe possible pour $f$.  Soyez aussi précis que
possible.
\label{7Q22}
\end{question}

\begin{question}
Le graphe de $F$ est donné ci-dessous.
\PDFgraph{7_integrales/integral3}
Tracez le graphe de la primitive $f$ de $F$ qui satisfait $f(1) = 3$.
\label{7Q23}
\end{question}

\begin{question}
Le graphe de $G$ est donné ci-dessous.
\PDFgraph{7_integrales/integral4}
Tracez le graphe de la primitive $g$ de $G$ qui satisfait $g(1) = 10$.
\label{7Q24}
\end{question}

\begin{question}
Le graphe de $f'(x)$ est donné ci-dessous.  Nous supposons de plus que
$f(0)=2$.
\PDFgraph{7_integrales/area9a}

\subQ{a} Trouvez tous les points critiques et points d'inflexion de
$f(x)$ avec la valeur de $f(x)$ à ces points.

\subQ{b} Tracez le graphe de $f(x)$ en indiquant bien tous les points
critiques et les points d'inflexion.
\label{7Q25}
\end{question}

\begin{question}
Le graphe de la fonction $V'$ est donné ci-dessous.
\PDFgraph{7_integrales/ass8B}
Tracez le graphe de $V$, la primitive de la fonction $V'$, qui passe
par le point $(0,1000)$.
\label{7Q26}
\end{question}

\subsection{Intégrales définies}

\begin{question}
Évaluez les intégrales suivantes.
\[
\int_1^2 \left(\frac{2}{t} + \frac{t}{2}\right)\dx{t} \quad , \quad
\int_2^3 \left(\frac{2}{t} + \frac{t}{2}\right)\dx{t} \quad \text{et} \quad
\int_1^3 \left(\frac{2}{t} + \frac{t}{2}\right)\dx{t} \ .
\]
Si vous connaissez la valeur de deux des intégrales ci-dessus, montrez que
vous connaissez la valeur de la troisième intégrale.
\label{7Q27}
\end{question}

\begin{question}
Si $\displaystyle \int_2^8 f(x) \dx{x} = 5$ et
$\displaystyle \int_5^8 f(x) \dx{x} = 7$, calculez les intégrales
définies suivantes.
\begin{center}
\begin{tabular}{*{2}{l@{\hspace{0.5em}}l@{\hspace{3em}}}l@{\hspace{0.5em}}l}
\subQ{a} & $\displaystyle \int_2^5 f(x) \dx{x}$ &
\subQ{b} & $\displaystyle \int_5^8 (2f(x)+3x)\dx{x}$ &
\subQ{c} & $\displaystyle \int_{-2/3}^{2/3} f(2-3t) \dx{t}$
\end{tabular}
\end{center}
\label{7Q28}
\end{question}

\begin{question}
Évaluez les intégrales suivantes.
\begin{center}
\begin{tabular}{*{1}{l@{\hspace{0.5em}}l@{\hspace{6em}}}l@{\hspace{0.5em}}l}
\subQ{a} & $\displaystyle \int_1^5 \frac{5}{x^3} \dx{x}$ &
\subQ{b} &
$\displaystyle \int_1^8 \left( \frac{2}{\sqrt[3]{x}} +3\right)\dx{x}$ \\[0.8em]
\subQ{c} & $\displaystyle \int_1^4 \left(e^x + \frac{1}{x}\right)\dx{x}$ &
\subQ{d} & $\displaystyle \int_1^2 \frac{3}{t^4} \dx{t}$
\end{tabular}
\end{center}
\label{7Q29}
\end{question}

\begin{question}
Évaluez les intégrales définies suivantes.
\begin{center}
\begin{tabular}{*{2}{l@{\hspace{0.5em}}l@{\hspace{3em}}}l@{\hspace{0.5em}}l}
\subQ{a} & $\displaystyle \int_0^5 3 e^{x/5} \dx{x}$ &
\subQ{b} & $\displaystyle \int_0^4 \left(1+\frac{t}{2}\right)^4 \dx{t}$ &
\subQ{c} & $\displaystyle \int_{1}^{10} (1+2t)^{-4} \dx{t}$ \\[0.8em]
\subQ{d} & $\displaystyle \int_{0}^{2} \frac{1}{1+4t} \dx{t}$ &
\subQ{e} & $\displaystyle \int_0^1 x^2(1+2x^3)^5 \dx{x}$ &
\subQ{f} & $\displaystyle \int_1^4 \frac{e^{\sqrt{x}}}{\sqrt{x}} \dx{x}$
  \\[0.8em]
\subQ{g} & $\displaystyle \int_1^e \frac{1}{x(1+(\ln(x))^2)} \dx{x}$ &
\subQ{h} & $\displaystyle \int_1^e x^2 \ln(x) \dx{x}$  &
\subQ{i} & $\displaystyle \int_1^2 \frac{4x^2 -14x+10}{2x^2-7x+3} \dx{x}$
\end{tabular}
\end{center}
\label{7Q30}
\end{question}

\begin{question}[\theory]
Utilisez la méthode d'intégration par parties pour obtenir une
relation entre $F_n$ et $F_{n-1}$ où
$\displaystyle F_n = \int_0^1 \frac{x^n}{(1+x)^n} \dx{x}$.
\label{7Q31}
\end{question}

\begin{question}[\eng \life]
Évaluez les intégrales définies suivantes.
\begin{center}
\begin{tabular}{*{1}{l@{\hspace{0.5em}}l@{\hspace{6em}}}l@{\hspace{0.5em}}l}
\subQ{a} & $\displaystyle \int_0^\pi\left(2\sin(\theta)+3\cos(\theta)\right)$ &
\subQ{b} & $\displaystyle \int_{\pi/2}^{3\pi/4} \sin^5(x) \cos^3(x)
\dx{x}$ \\[0.9em]
\subQ{c} & $\displaystyle \int_0^{\pi/2} \sin^2(x)\cos^2(x)\dx{x}$ &
\subQ{d} & $\displaystyle \int_0^{\pi/4} \tan^2(x)\sec^4(x)\dx{x}$ \\[0.9em]
\subQ{e} & $\displaystyle \int_0^2 \frac{x^3+x+1}{x^2+4} \dx{x}$ &
\subQ{f} & $\displaystyle \int_0^1 \frac{1}{(1 + 2x -x^2)^{1/2}} \dx{x}$
\end{tabular}
\end{center}
\label{7Q32}
\end{question}

\begin{question}[\eng]
Évaluez les intégrales définies suivantes.
\begin{center}
\begin{tabular}{*{1}{l@{\hspace{0.5em}}l@{\hspace{6em}}}l@{\hspace{0.5em}}l}
\subQ{a} & $\displaystyle \int_{\sqrt{2}}^{2} \frac{1}{t^3\sqrt{t^2-1}} \dx{x}$ &
\subQ{b} & $\displaystyle \int_0^{2\sqrt{3}} \frac{t^3}{\sqrt{16-t^2}} \dx{t}$
\end{tabular}
\end{center}
\label{7Q33}
\end{question}

\begin{question}
Évaluez les intégrales définies suivantes en fessant le moins de
calculs possible.  En fait, dans certains cas, aucun calcul n'est
nécessaire.
\begin{center}
\begin{tabular}{*{1}{l@{\hspace{0.5em}}l@{\hspace{3em}}}l@{\hspace{0.5em}}l}
\subQ{a} & $\displaystyle \int_{-2}^{2}\left(y^4+5y^3\right) \dx{y}$ &
\subQ{b} & $\displaystyle \int_{-a}^{a} x \sqrt{x^2+a^2} \dx{x}$
\end{tabular}
\end{center}
\label{7Q34}
\end{question}

\begin{question}[\eng \life]
Évaluez les intégrales définies suivantes en fessant le moins de
calculs possible.  En fait, dans certains cas, aucun calcul n'est
nécessaire.
\begin{center}
\begin{tabular}{*{2}{l@{\hspace{0.5em}}l@{\hspace{3em}}}l@{\hspace{0.5em}}l}
\subQ{a} & $\displaystyle \int_{-\pi/2}^{\pi/2}\left(x^2 -20
\sin(x)\right) \dx{x}$ &
\subQ{b} & $\displaystyle \int_{2}^{5}\cos(2\pi(x-2)) \dx{x}$ &
\subQ{c} & $\displaystyle \int_{-\pi}^{\pi} x^{16} \sin(2x) \dx{x}$
\end{tabular}
\end{center}
\label{7Q35}
\end{question}

\begin{question}[\eng \life]
Vérifiez que $\displaystyle \dfdx{ \int_a^x f(s) \dx{s} }{x} = f(x)$ pour
$f(s) = (5s+1)^7$ et $a$ une constante.
\label{7Q36}
\end{question}

\begin{question}
Calculez les dérivées des fonctions suivantes.
\begin{center}
\begin{tabular}{*{1}{l@{\hspace{0.5em}}l@{\hspace{6em}}}l@{\hspace{0.5em}}l}
\subQ{a} & $\displaystyle g(y) = \int_2^y t^2\sin(t) \dx{t}$ &
\subQ{b} & $\displaystyle h(x) = \int_2^{1/x} \arctan(t) \dx{t}$
\end{tabular}
\end{center}
\noindent Suggestion: Utilisez le Théorème fondamental du calcul différentiel.
\label{7Q37}
\end{question}

\subsection{Intégrales impropres}

\begin{question}
Déterminez si les intégrales impropres suivantes convergent ou divergent.
Évaluez les intégrales impropres qui convergent.
\begin{center}
\begin{tabular}{*{2}{l@{\hspace{0.5em}}l@{\hspace{3em}}}l@{\hspace{0.5em}}l}
\subQ{a} & $\displaystyle \int_0^\infty e^{-3x} \dx{x}$ &
\subQ{b} & $\displaystyle \int_0^\infty \frac{1}{(2+5x)^4} \dx{x}$ &
\subQ{c} & $\displaystyle \int_0^\infty \frac{1}{\sqrt[3]{x}} \dx{x}$ \\[0.8em]
\subQ{d} & $\displaystyle \int_2^\infty \frac{1}{x\sqrt{x}}\,\dx{x}$ &
\subQ{e} & $\displaystyle \int_1^e \frac{2}{x\sqrt{\ln(x)}} \dx{x}$ &
\subQ{f} & $\displaystyle \int_{-\infty}^0 3x^2 e^{-x^3} \dx{x}$ \\[0.8em]
\subQ{g} & $\displaystyle \int_{-1}^1 \frac{x+1}{\sqrt[3]{x^4}} \dx{x}$ &
\subQ{h} & $\displaystyle \int_0^\infty x\,e^{-x} \dx{x}$ &
\subQ{i} & $\displaystyle\int_0^3\frac{2}{(x-2)^{4/3}}\dx{x}$ \\[0.8 em]
\subQ{j} & $\displaystyle \int_0^\infty \frac{x}{1+x^4} \dx{x}$ &
\subQ{k} & $\displaystyle \int_0^\infty \frac{1}{(1+3x)^{3/2}} \dx{x}$ &
\subQ{l} & $\displaystyle \int_0^4 \frac{1}{x^2+x-6} \dx{x}$ \\[0.8em]
\subQ{m} & $\displaystyle \int_0^2 \,\frac{1}{4-x^2} \, \dx{x}$ &
\subQ{n} & $\displaystyle \int_1^2\frac{3}{\sqrt[3]{x-1}} \dx{x}$ &
\subQ{o} & $\displaystyle \int_0^{\pi/2}
\frac{\sin(x)}{\sqrt{\cos(x)}} \dx{x}$ \\[0.8em]
\subQ{p} & $\displaystyle \int_1^2 \frac{x}{\sqrt{2-x}} \dx{x}$ &
\subQ{q} & $\displaystyle \int_0^8 \frac{x}{(8-x)^{2/3}} \dx{x}$ & &
\end{tabular}
\end{center}
\label{7Q38}
\end{question}

\begin{question}[\eng]
Utilisez le test de comparaison des intégrales pour déterminer si les
intégrales suivantes converge.
\begin{center}
\begin{tabular}{*{2}{l@{\hspace{0.5em}}l@{\hspace{3em}}}l@{\hspace{0.5em}}l}
\subQ{a} & $\displaystyle \int_0^1 \frac{1}{x^{1/3} + x^3} \dx{x}$ &
\subQ{b} & $\displaystyle \int_1^\infty \frac{1}{x^{1/3}+x^3} \dx{x}$ &
\subQ{c} & $\displaystyle \int_1^\infty \frac{x}{x^2+x^{4/3}} \dx{x}$
\end{tabular}
\end{center}
\label{7Q39}
\end{question}

\begin{question}[\eng]
Utilisez le test de comparaison pour déterminer si les intégrales impropres
suivantes conver\-gent ou divergent.  Donnez une borne supérieure pour la
valeur des intégrales impropres qui convergent.
\begin{center}
\begin{tabular}{*{2}{l@{\hspace{0.5em}}l@{\hspace{3em}}}l@{\hspace{0.5em}}l}
\subQ{a} & $\displaystyle \int_0^1 \frac{1}{x^2\, e^x} \dx{x}$ &
\subQ{b} & $\displaystyle \int_1^\infty \frac{1}{\sqrt{x}+ e^x} \dx{x}$ &
\subQ{c} & $\displaystyle \int_1^\infty \frac{\sin^2(x)}{x^3 + 2} \dx{x}$
\\[1em]
\subQ{d} & $\displaystyle \int_1^\infty\frac{1+\cos^2x}{x^2} \dx{x}$ &
\subQ{e} & $\displaystyle \int_0^1 \frac{2\sqrt{x}+1}{x^4+3x} \dx{x}$ &
\subQ{f} & $\displaystyle \int_1^\infty
\frac{x^2 \cos^2 x + 1}{x^4 + x^2 + 1} \dx{x}$ \\[1em]
\subQ{g} & $\displaystyle \int_1^\infty \frac{2\sqrt{x}+3}{x^2+2x} \dx{x}$ &
& & &
\end{tabular}
\end{center}
\label{7Q40}
\end{question}

%%% Local Variables: 
%%% mode: latex
%%% TeX-master: "notes"
%%% End: 

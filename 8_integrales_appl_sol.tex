\section{Applications de l'intégrale}

\subsection{Aire entre deux courbes}

\compileSOL{\SOLUb}{\ref{8Q1}}{
\subQ{a} Notons que $x^2 < 2x$ pour $0 < x < 2$.  Ainsi, l'aire de la
région pour $0\leq x \leq 2$ est
\[
\int_0^2 |2x-x^2| \dx{x}
= \int_0^2 (2x-x^2) \dx{x} = \left( x^2 - \frac{x^3}{3}\right)\bigg|_0^2
= 4 - \frac{8}{3} = \frac{4}{3} \ .
\]

\subQ{b} Notons que $x^2>2x$ pour $2< x \leq 4$.  Ainsi, l'aire de la
région pour $0\leq x \leq 4$ est
\begin{align*}
\int_0^4 |2x-x^2| \dx{x}
&= \int_0^2 (2x-x^2) \dx{x} + \int_2^4 (x^2-2x) \dx{x} \\
&= \left( x^2 - \frac{x^3}{3}\right)\bigg|_0^2 +
\left( \frac{x^3}{3} - x^2 \right)\bigg|_2^4
= \frac{4}{3} + \left(\frac{4^3}{3} - 4^2\right) + \frac{4}{3}
= 8 \ .
\end{align*}

\subQ{c} Notons que $e^{2x}/2 < e^x$ pour $x < \ln(2)$ et $\ln(2)>1$.
Ainsi, l'aire de la région pour $0\leq x \leq 1$ est
\[
\int_0^1 \left|e^x- \frac{e^{2x}}{2}\right| \dx{x} =
\int_0^1 \left(e^x- \frac{e^{2x}}{2}\right) \dx{x} =
\left( e^x - \frac{e^{2x}}{4}\right)\bigg|_0^1
= e - \frac{e^2}{4} - \frac{3}{4} \approx 0.12101780\ . 
\]

\subQ{d} Les valeurs de $x$ aux points où la parabole
$\displaystyle y=\frac{x^2}{2}$ et la droite
$\displaystyle y=\frac{3}{2}+x$ se
coupent sont les solutions de
\[
  \frac{x^2}{2} = \frac{3}{2}+x \Leftrightarrow 
  x^2-2x-3=0 = (x-3)(x+1)=0 \ .
\]
Donc la parabole et la droite se coupent lorsque $x=-1$ et $x=3$.  Puisque
$\displaystyle \frac{3}{2} + x > \frac{x^2}{2}$ pour $-1 < x < 3$,
l'aire de la région bornée par la droite et la parabole est
\[
A = \int_{-1}^3 \left( \frac{3}{2} + x - \frac{1}{2}x^2 \right)\dx{x}
= \left( \frac{3}{2}x + \frac{1}{2}x^2 - \frac{1}{6}x^3
\right)\big|_{-1}^3 = \frac{16}{3} \ .
\]

\subQ{e} Les valeurs de $x$ aux points où la parabole
$\displaystyle y=7x -x^2$ et la droite $\displaystyle y=x$ se
coupent sont les solutions de $7x-3x^2 = x$;
c'est-à-dire, de $0 = 6x-3x^2 = 3x(2-x)$.  Nous trouvons
$x=0$ et $x=2$.  Entre $0$ et $2$, nous avons que $7x - x^2 > x$.
Donc, l'aire de la région bornée par la droite et la parabole est
\[
A = \int_0^{2} \left( 7x - 3x^2 - x \right) \dx{x}
= \left( - x^3 + 3 x^2 \right)\bigg|_0^{2} = 4 \ .
\]

\subQ{f} Les valeurs de $y$ aux points où la parabole
$\displaystyle x=y^2$ et la droite $\displaystyle x-y=2$ se
coupent sont les solutions de $y^2 = x = y+2$;
c'est-à-dire, $y^2 -y - 2 = (y-2)(y+1) = 0$.  On obtient
$y = -1$ et $y = 2$.  Nous obtenons la figure suivante.
\PDFgraph{8_integrales_appl/aire4}

Ainsi, l'aire $A$ de la région bornée par la droite et la parabole est
\[
A = \int_0^1 \left( \sqrt{x} - (- \sqrt{x})\right) \dx{x} +
\int_1^4 \left( \sqrt{x} - (x-2)\right) \dx{x} \ .
\]
nous ne sommes pas obligé d'intégrer par rapport à $x$.  Le calcul de l'aire
devient beaucoup plus simple si nous intégrons par rapport à $y$.  Ainsi,
\[
A = \int_{-1}^2 \left( (y+2) - y^2 \right) \dx{y}
= \left( \frac{y^2}{2} + 2 y - \frac{y^3}{3} \right)\bigg|_{-1}^2
= \frac{9}{2} \ .
\]

\subQ{g} Les valeurs de $x$ aux points où la courbe
$\displaystyle y=\frac{x}{1+x}$ et la droite $\displaystyle y=\frac{x}{2}$ se
coupent sont les solutions de 
$\displaystyle \frac{x}{1+x} = \frac{x}{2}$.  Nous avons la solution $x=0$.
Si $x\neq 0$, nous pouvons diviser les deux côtés de l'équation par $x$ pour
obtenir $\displaystyle \frac{1}{1+x} = \frac{1}{2}$.  Une seconde
solution de l'équation est donc $x=1$.

Puisque $f(1/2) = 1/3 > 1/4 = g(1/2)$, nous avons que $f(x)>g(x)$ pour $0<x<1$.
Donc, l'aire de la région entre la courbe et la droite est
\begin{align*}
A &= \int_0^1 \left( \frac{x}{1+x} - \frac{x}{2} \right) \dx{x}
= \int_0^1 \left( 1 - \frac{1}{1+x} - \frac{x}{2} \right) \dx{x}
= \left( x - \ln(1+x) - \frac{x^2}{4} \right)\bigg|_0^1 \\
&= \frac{3}{4} - \ln(2) \approx 0.056852819  \ .
\end{align*}
}

\compileSOL{\SOLUb}{\ref{8Q2}}{
Notons que $\sin(2x)$ et $\cos(2x)$ sont de période $\pi$.  Nous
retrouvons le graphe de ces deux fonctions dans la figure ci-dessous.
\PDFgraph{8_integrales_appl/sin_cos}

Pour être en mesure de calculer l'aire entre les courbes $\sin(2x)$ et
$\cos(2x)$ pour $0\leq x \leq \pi$, il faut déterminer sur quels
intervalles $\cos(x)>\sin(x)$ et sur quels intervalles nous avons l'inégalité
inverse.  Il faut donc trouver $x$ entre $0$ et $\pi$ tel que
$\cos(2x) = \sin(2x)$.  C'est-à-dire,
\[
\tan(2x) = \frac{\cos(2x)}{\sin(2x)} = 1
\]
et $0\leq x \leq \pi$.  Puisque $\tan(\theta)=1$ pour $\theta = \pi/4$,
$5\pi/4$, $9\pi/4$, \ldots\ (nous pouvons ignorer les valeurs
négatives de $\theta$ car $x\geq 0$), nous obtenons
$2x = \pi/4$, $5\pi/4$, \ldots\  La restriction
$0\leq x \leq \pi$ donne seulement deux valeurs $x= \pi/8$ et $x = 5\pi/8$.

L'aire entre $y=\cos(2x)$ et $y=\sin(2x)$ est la somme de l'aire des
trois régions $A$, $B$ et $C$ représentées dans la figure précédente.
\begin{align*}
\text{aire de A } &= \int_0^{\pi/8} (\cos(2x)-\sin(2x)) \dx{x} \ .\\
\text{aire de B } &= \int_{\pi/8}^{5\pi/8} (\sin(2x)-\cos(2x)) \dx{x} \ .\\
\text{aire de C } &= \int_{5\pi/8}^{\pi} (\cos(2x)-\sin(2x)) \dx{x} \ .
\end{align*}
Nous avons
\begin{align*}
\text{aire de A } &= \left(\frac{1}{2}\sin(2x) +
  \frac{1}{2}\cos(2x)\right)\bigg|_0^{\pi/8}
= \frac{1}{2}\sin(\pi/4) + \frac{1}{2} \cos(\pi/4) - \frac{1}{2}
= \frac{\sqrt{2}-1}{2} \ , \\
\text{aire de B } &= \left(-\frac{1}{2}\cos(2x)
 -\frac{1}{2}\sin(2x)\right)\bigg|_{\pi/8}^{5\pi/8} \\
&= -\frac{1}{2}\sin(5\pi/4) - \frac{1}{2} \cos(5\pi/4)
+ \frac{1}{2}\sin(\pi/4) + \frac{1}{2} \cos(\pi/4)
= \sqrt{2}
\intertext{et}
\text{aire de C } &= \left(\frac{1}{2}\sin(2x) +
  \frac{1}{2}\cos(2x)\right)\bigg|_{5\pi/8}^\pi
= \frac{1}{2} - \frac{1}{2}\sin(5\pi/4) - \frac{1}{2} \cos(5\pi/4)
= \frac{1+\sqrt{2}}{2} \ .
\end{align*}
L'aire total est donc
\[
\frac{\sqrt{2}-1}{2} + \sqrt{2} + \frac{1+\sqrt{2}}{2}
= 2 \sqrt{2} \ .
\]
}

\compileSOL{\SOLUb}{\ref{8Q3}}{
\subQ{a}
Le nombre de bactéries est
\begin{align*}
n(T) &= 10^6 + \int_0^T r(t) \dx{t}
= 10^6 + \int_0^T \frac{1000}{(2+3t)^{3/4}} \dx{t} \\
&= 10^6 + \frac{4000}{3} (2+3t)^{1/4}\bigg|_0^T
= 10^6 + \frac{4000}{3} \left( (2+3T)^{1/4} - 2^{1/4}\right) \ .
\end{align*}

\subQ{b}
Puisque $\displaystyle \lim_{T\to \infty} (2+3T)^{1/4} = +\infty$, nous avons que
$\displaystyle \lim_{T\to \infty} n(T) = +\infty$.  Cette population de
bactéries ne peut pas supporter le taux de croissance.

\subQ{c}
Nous cherchons $T$ tel que $n(T) = 2\times 10^6$.  Donc,
\begin{align*}
& 10^6 + \frac{4000}{3} \left( (2+3T)^{1/4} - 2^{1/4}\right) = 2\times 10^6
 \Leftrightarrow (2+3T)^{1/4} - 2^{1/4} = \frac{3}{4000} \times 10^6
= 750\\
& \qquad \Leftrightarrow  T =
\frac{1}{3}\left(\left(750 + 2^{1/4}\right)^4-2\right)
\approx 1.06139271674 \times 10^{11} \ \text{min} \ .
\end{align*}
Concrètement, nous pourrions dire que la population supporte le taux de
croissance car le nombre de bactéries augmente très lentement; il faut
plus de $73,707,827$ jours pour que la population double.
}

\subsection{Valeur moyenne d'une fonction}

\compileSOL{\SOLUb}{\ref{8Q4}}{
La valeur moyenne est
\[
M = \frac{1}{2} \int_{-1}^1 (x - x^3) \dx{x} =
\frac{1}{2}\left(\frac{x^2}{2} - \frac{x^4}{4} \right)\bigg|_{-1}^1 = 0 \ .
\]
Cela ne devrait pas surprendre personne car la fonction $f$ est
impaire et le domaine d'intégration $[-1,1]$ est un intervalle symétrique
par rapport à l'origine, donc l'intégrale est nulle.

\MATHgraph{8_integrales_appl/moyenne1}{8cm}
}

\compileSOL{\SOLUa}{\ref{8Q5}}{
\subQ{a}
Le volume d'eau qui a été versé dans le bocal au cours des deux
premières minutes est
\[
V = \int_0^2 r(t) \dx{t} = 4 \int_0^2 (3t-t^2) \dx{t}
= 4 \left( \frac{3t^2}{2} - \frac{t^3}{3} \right)\bigg|_0^2
= 4 \left(6 - \frac{8}{3} \right)= \frac{40}{3} \quad \text{litres}.
\]
\subQ{b} Le taux moyen pour les deux premières minutes est
\[
M_2 = \frac{1}{2} \int_0^2 r(t) \dx{t} = \frac{20}{3} \quad
\text{litres/minute}.
\]
\subQ{c} le taux moyen pour la première minute est
\[
M_1 = \frac{1}{1} \int_0^1 r(t) \dx{t} 
= 4 \left( \frac{3t^2}{2} - \frac{t^3}{3} \right)\bigg|_0^1 
= 4 \left( \frac{3}{2} - \frac{1}{3} \right) =
\frac{14}{3}  \quad \text{litres/minute}.
\]
Il n'est pas surprenant que $M_1$ soit plus petit que $M_2$ car l'aire sous
le graphe de $r$ et au-dessus de l'axe des $t$ augmente plus rapidement
lorsque $t$ est près du milieu de l'intervalle $[0,3]$.

\subQ{d}
\MATHgraph{8_integrales_appl/moyenne2}{8cm}
}

\compileSOL{\SOLUb}{\ref{8Q6}}{
Notons que $r(0)=r(15)=0$ et $r(t)>0$ pour $0 < t < 15$.

Le volume d'eau qui a entré dans le réservoir au cours des $15$
premières heures est
\[
V = \int_0^{15} r(t) \dx{t} = \int_0^{15} (360t-39 t^2 + t^3) \dx{t}
= \left( 180 t^2 - 13 t^3 + \frac{t^4}{4} \right)\bigg|_0^{15}
= 9,281.25 \quad \text{litres}.
\]
Le taux moyen pour les $15$ premières heures est
\[
M_2 = \frac{1}{15} \int_0^{15} r(t) \dx{t} = \frac{9,281.25}{15}
= 618.75 \quad \text{litres/heure}. 
\]
}

\compileSOL{\SOLUa}{\ref{8Q7}}{
Puisque $E(t) = 360t -39t^2+t^3$ est un polynôme de degré trois dont le
coefficient de $t^3$ est positif, nous avons que
\[
\begin{array}{c|c|c|c|c|c|c|c}
t & t < 0 & 0 & 0 <t < 15 & 15 & 15 < t < 24 & 24 & 24 < 4 \\
\hline
E(t) & - & 0 & + & 0 & - & 0 & +
\end{array}
\]
Donc,
\[
| 360t -39t^2+t^3| =
\begin{cases}
360t -39t^2+t^3 & \quad \text{si}\quad  0\leq t \leq 15 \\
-360t +39t^2-t^3 & \quad \text{si} \quad 15\leq t \leq 24
\end{cases}
\]
L'énergie totale produite est
\begin{align*}
T &= \int_0^{15} (360t-39t^2+t^3) \dx{t}
+ \int_{15}^{24} (-360t+39t^2-t^3) \dx{t} \\
&= \left(180t^2-13t^3+\frac{t^4}{4}\right)\bigg|_0^{15}
+ \left(-180t^2+ 13t^3- \frac{t^4}{4}\right)\bigg|_{15}^{24}
= 11650.5 \ \text{J}.
\end{align*}
Le taux moyen de production d'énergie est
$\displaystyle M = \frac{T}{24} = \frac{11650.5}{24} = 485.4375$ J/hr.
}

\compileSOL{\SOLUb}{\ref{8Q8}}{
\subQ{a} Il faut trouver le maximum absolu et le minimum absolu de
$\rho$ sur l'intervalle $0\leq x \leq 200$ où la variable $x$ est
mesurée en cm.
\begin{align*}
\rho'(x) &= 4\times 10^{-8}\, x(240-x)- 2\times10^{-8}x^2 \\
&= 6\times 10^{-8} x ( 160 - x) \ .
\end{align*}
Les points critiques dans l'intervalle $[0,200]$ sont
$x=0$ et $x= 160$.

Puisque $\rho(0)=1$, $\rho(160)=1.04096$ et $\rho(200) = 1.032$,
le maximum absolu est $1.04096$ g/cm lorsque $x=1.60$ m et le minimum absolu
est $1$ g/cm lorsque $x=0$ cm.

\subQ{b} La masse $m$ du fils est
\begin{align*}
m &= \int_0^{200}\rho(x) \dx{x}
= \int_0^{200} (1+2\times10^{-8}x^2(240-x) )\dx{x} \\
&= \int_0^{200} (1+4.8\times 10^{-6}\,x^2-2\times10^{-8}\,x^3)\dx{x}
= \left(x+1.6\times10^{-6}x^3-5\times10^{-9}x^4\right)\bigg|_0^{200} \\
&= 200+12.8-8 = 204.8 \ \text{g.}
\end{align*}

\subQ{c} Remarquons que la densité moyen du fils est $204.8/200=1.024$
g/cm.  C'est bien une valeur entre la densité minimale $1$ g/cm et la densité
maximale $1.0496$ g/cm du fils.

\subQ{d}
\MATHgraph{8_integrales_appl/average}{8cm}
}

\subsection{Volume d'un objet}

\compileSOL{\SOLUb}{\ref{8Q9}}{
Les tranches perpendiculaires à l'axe du cône sont des disques.  Le
rayon de ces disques varie de $2$ à la base du cône à $0$ au sommet du
cône.  L'aire des disques est donc $A(h) = \pi (r(h))^2$ où le rayon $r$
est une fonction linéaire de la hauteur $h$ à laquelle se trouve la
tranche.  L'équation qui définit $r$ est donc l'équation de la droite
qui passe par les points $(h,r)=(0,2)$ et $(h,r) = (3,0)$.  La pente
de cette droite est $(0-2)/(3-0) = -2/3$.  L'équation de cette droite
dans la forme point-pente  est $(r-2) = (-2/3) (h-0) = -2h/3$.  Donc,
$r= 2 - 2h/3$ et ainsi $A(h) = \pi (2 - 2h/3)^2$.  Le volume $V$ du
cône est
\[
V = \int_0^3 A(h)\dx{h} = \pi \int_0^3 (2-2h/3)^2 \dx{h} \ .
\]
Si $u= 2 -2 h/3$, alors $\dx{u} = (-2/3) \dx{h}$,
$u=2$ lorsque $h=0$ et $u=0$ lorsque $h=3$.  Ainsi,
\[
V = -\frac{3\pi}{2} \int_2^0 u^2 \dx{u} = \frac{3\pi}{2} \int_0^2 u^2 \dx{u}
= \frac{\pi}{2} u^3 \bigg|_{u=0}^2 = 4 \pi \ ,
\]
}

\compileSOL{\SOLUa}{\ref{8Q11}}{
L'aire d'une section transverse est donnée par la formule
$\displaystyle A(x) = \pi \left(\frac{e^{2x}-e^x}{2}\right)^2$.
Le volume est donc
\begin{align*}
V &= \int_0^1 A(x) \dx{x}
= \frac{\pi}{4} \int_0^1 (e^{2x}-e^x)^2 \dx{x}
= \frac{\pi}{4} \int_0^1 (e^{4x} - 2 e^{3x} + e^{2x}) \dx{x} \\
&= \frac{\pi}{4} \left( \frac{1}{4}e^{4x} - \frac{2}{3}e^{3x}
 + \frac{1}{2}e^{2x}\right)\bigg|_0^1
= \frac{\pi}{4} \left(\frac{e^4}{4} - \frac{2e^3}{3}
+ \frac{e^2}{2} - \frac{1}{12} \right) \approx 3.03978 \ .
\end{align*}                                                                  
}

\compileSOL{\SOLUb}{\ref{8Q12}}{
Si nous assumons que l'origine est au centre du ballon et que l'axe des
$z$ est l'axe principal du ballon, le ballon peut alors être décrit
par l'équation
\[
  \frac{x^2}{9^2} + \frac{y^2}{9^2} + \frac{z^2}{15^2} = 1 \ .
\]
Nous représentons le ballon dans la figure ci-dessous.
\PDFgraph{8_integrales_appl/football}

Nous allons calculer le volume de la moitié du ballon; c'est-à-dire, pour
$z \geq 0$.  Le volume du ballon sera donc le double de la valeur que
nous trouverons.

Soit $k$, un entier positif.  Posons $\Delta z = 15/k$ et
$z_i = i \Delta z = 15 i/k$ pour $i=0$, $1$, $2$, \ldots, $k$.  Nous
obtenons une partition de l'intervalle $[0,15]$ en sous-intervalles de
la forme $[z_i,z_{i+1}]$ pour $i=0$, $1$, \ldots, $k-1$.

La $i^e$ tranche du ballon est donnée par l'intersection du ballon
avec la région définie par $\{ (x,y,z) : z_i \leq z \leq z_{i+1}\}$.
Pour estimer le volume $V_i$ de la $i^e$ tranche, nous choisissons $z_i^\ast$
dans l'intervalle $[z_i,z_{i+1}]$ et nous considérons le cylindre $S_i$,
qui sont représentés dans la figure ci-dessus.  La hauteur de ce
cylindre est $\Delta z$ et son rayon est
$\displaystyle 9 \sqrt{1 - \frac{(z_i^\ast)^2}{15^2}}$.  Le volume du
cylindre $S_i$ est donc
\[
  9^2 \pi \left(1 - \frac{(z_i^\ast)^2}{15^2}\right) \Delta z
= \frac{9\pi}{25} \left(15^2 - (z_i^\ast)^2\right) \Delta z \ .
\]
Pour $\Delta z$ très petit, le volume de la $i^e$ tranche du ballon
est approximativement le volume de $S_i$.  La somme du volume de
chaque cylindre $S_i$ donne donc un estimé du volume $V$ de l'a
moitié du ballon.
\[
  V \approx \sum_{i=0}^{k-1} \frac{9\pi}{25} \left( 15^2 - (z_i^\ast)^2\right)
  \Delta z \ .
\]
C'est une somme de Riemann.  Si $k \to \infty$, nous obtenons
\[
V = \int_0^{15} \frac{9\pi}{25} \left( 15^2 - z^2\right) \dx{z} \ .
\]
Ainsi, le volume de la moitié du ballon est
\[
V = \frac{9\pi}{25} \int_0^{15} \left( 15^2 - z^2\right) \dx{z}
= \frac{9\pi}{25} \left( 15^2z - \frac{z^3}{3}\right)\bigg|_0^{15}
= \frac{9\pi}{25} \left( \frac{2\times 15^3}{3}\right)
= 810 \pi \ \text{cm}^3 \ .
\]
}

\compileSOL{\SOLUb}{\ref{8Q13}}{
\subQ{b} Nous intégrons pour $0 \leq x \leq 2$.  La tranche à $x$ est un
demi-cercle de rayon $\sqrt{x}$.  Donc l'aire de la
tranche est
$\displaystyle A(x) = \frac{1}{2} \pi (\sqrt{x})^2 = \frac{1}{2} \pi x$.
Ainsi,
\[
V = \frac{1}{2} \int_0^2 \pi x \dx{x}
= \left( \frac{1}{4} \pi x^2 \right)\bigg|_0^2
= \pi \ .
\]

\subQ{c}
Nous avons la situation suivante.
\PDFgraph{8_integrales_appl/vol6}

L'aire d'un tranche est $\displaystyle A(x) = (3x+x^2)^2$.
Ainsi, le volume est donné par
\[
V = \int_0^2 A(x) \dx{x} = \int_0^2 (9x^2+6x^3+x^4)\dx{x}
=\left( 3x^3 + \frac{3x^4}{2} + \frac{x^5}{5}\right)\bigg|_0^2
= 54.4 \ .
\]
}

\compileSOL{\SOLUb}{\ref{8Q15}}{
\subQ{b} Le volume $V$ obtenu par la rotation de la région bornée par
la courbe $y=\cos(x/2)$, l'axe des $x$, et les droites $x=-\pi$ et
$x=\pi$ est donné par
\[
V = \pi\int_{-\pi}^{\pi}\cos^2\left(\frac{x}{2}\right)\dx{x} \ .
\]
Puisque
\[
\cos^2(\theta) = \frac{1}{2}\left( 1 + \cos(2\theta)\right) \ ,
\]
nous obtenons
\[
V = \frac{\pi}{2}\int_{-\pi}^{\pi} \left(1 + \cos(x)\right)\dx{x}
= \frac{\pi}{2} \left(x + \sin(x)\right)\bigg|_{-\pi}^{\pi}
= \frac{\pi}{2} \left( 2\pi \right) = \pi^2 \ .
\]
}

\compileSOL{\SOLUb}{\ref{8Q16}}{
\subQ{a} Nous devons premièrement trouver l'intersection de la droite avec
la parabole.  Les valeurs de $x$ aux points où la droite et la
parabole se coupent sont les solutions de $x^2 = x$; c'est-à-dire,
$x= 0$ ou $1$. 

Puisque $x > x^2$ pour $0<x<1$, le volume est donné par
\begin{align*}
V &= \pi \int_0^1 \left( \left( x - (-1) \right)^2 -
\left( x^2 - (-1) \right)^2 \right) \dx{x} \\
&= \pi \int_0^1 \left( -x^4 - x^2 + 2x \right) \dx{x}
= \pi \left( -\frac{x^5}{5} - \frac{x^3}{3} + x^2 \right)\bigg|_0^1
= \frac{7 \pi}{15} \ .
\end{align*}

\subQ{b} 
Nous devons premièrement trouver l'intersection de la droite avec la
parabole.  Les valeurs de $x$ aux points où les deux courbes se
coupent sont les solutions de $x^2 = 2x$; c'est-à-dire, $x = 0$ ou $2$.

Nous obtenons la figure ci-dessous.
\PDFgraph{8_integrales_appl/rot16}

Nous utilisons la méthode des cylindres,
\[
V = 2\pi \int_a^b |x-c||f(x)-g(x)|\dx{x} \ ,
\]
pour calculer le volume du solide de révolution.  Nous avons
$f(x) = 2x$, $g(x) = x^2$, $c=4$, $a=0$ et $b=2$.  Ainsi,
\[
V = 2\pi \int_0^2 (4-x)(2x -x^2) \dx{x}
=2\pi \int_0^2 (8x -6x^2 +x^3) \dx{x}
=2\pi \left( 4x^2 - 2x^3 + \frac{x^4}{4} \right)\bigg|_0^2
= 8\pi \ .
\]
}

\subsection{Masse d'un objet}

\subsection{Travail}

\compileSOL{\SOLUb}{\ref{8Q23}}{
\PDFgraph{8_integrales_appl/reservoir3_sol}

Partageons l'intervalle $[-5,-2]$ en $N$ sous-intervalles de longueur
$\Delta x = 3/N$ pour obtenir des sous-intervalles de la forme
$[x_i,x_{i+1}]$ où $x_i = -5 + i \Delta x$ pour $i=0$, $1$, \ldots,
$N$.  De plus, choisissons $x_i^\ast$ entre $x_i$ et $x_{i+1}$ pour $i=0$,
$1$, \ldots, $N-1$.

Le travail nécessaire pour retirer du réservoir la mince couche d'eau
entre $x_i$ et $x_{i+1}$ est approximativement
\[
9.8 \times 1000 \times \underbrace{0-x_i^\ast}_{\text{profondeur}} \times
\underbrace{\pi (9 - (-2 - x_i^\ast)^2)
\Delta x}_{\text{volume de la couche d'eau}} \ .
\]
Il ne faut pas oublier que $x_i^\ast<0$ par construction.

Donc, le travail pour vider le réservoir est approximativement
\[
W \approx \sum_{i=0}^{N-1} -9.8 \times 1000 \times x_i^\ast \times
\pi (9 - (-2 - x_i^\ast)^2) \Delta x
= -9,800 \pi \sum_{i=0}^{N-1} x_i^\ast (9 - (-2 -x_i^\ast)^2) \Delta x
\ .
\]
C'est une somme de Riemann pour
$\displaystyle -9,800 \pi \int_{-5}^{-2} x (9 - (-2-x)^2)\dx{x}$.
Donc, si $N \to \infty$, nous obtenons
que le travail pour vider le réservoir est
\begin{align*}
W &= -9,800 \pi \int_{-5}^{-2} x (9 - (-2-x)^2)\dx{x}\\
& = -9,800 \pi \int_{-5}^{-2} (5x - 4x^2 - x^3)\dx{x}
= -9,800 \pi \left( \frac{5x^2}{2} - \frac{4x^3}{3} +
  \frac{x^4}{4}\right) \bigg|_{-5}^{-2} \\
&= 9,800 \pi \times 56.25 \approx 1.731 \times 10^6 \ \text{J.}
\end{align*}
}

\compileSOL{\SOLUb}{\ref{8Q25}}{
Si l'axe des $x$ est vertical et l'origine se trouve à trois mètres
au dessus du citerne, il faut intégrer de $-9$ à $-3$.
De plus, une tranche horizontal est un disque de $2$ m de
rayon.  Son aire est donc $A(x) = \pi \times 2^2$.  Le travail pour vider le
réservoir est
\[
W = 9.8 \times 1000 \times \pi \times 2^2 \int_{-9}^{-3} (0-x) \dx{x}
= 39200  \pi \left( -\frac{x^2}{2} \right) \bigg|_{-9}^{-3}
= 1411200 \pi \ \text{J.}
\]
}

\subsection{Force}

\compileSOL{\SOLUb}{\ref{8Q27}}{
\PDFgraph{8_integrales_appl/dam1_sol}

Partageons le segment $[-17,-12]$ en $N$ sous intervalles
$[x_i,x_{i+1}]$ de longueur $\Delta x = 5/N$ où
$x_i = -17 + i \Delta x$ pour $i=0$, $1$, $2$, \ldots , $N$.
De plus, choisissons $x_i^\ast$ entre $x_i$ et $x_{i+1}$ pour $i=0$, $1$, $2$,
\ldots , $N-1$.

La pression sur la petite section (en blanc) d'une largeur de $2$ m
et d'une hauteur de $\Delta x$ à une profondeur de $x_i^\ast$ est
approximativement
\[
9.8 \times 1000 \times \underbrace{(0-x_i^\ast)}_{\text{profondeur}}
\times \underbrace{2 \Delta x}_{\text{aire de la section}} \ .
\]
Comme toujours, il ne faut pas oublier que $x_i^\ast <0$ par
construction.

Donc la pression totale sur la porte est approximativement
\[
\sum_{j=0}^{N-1} -9.8 \times 1000\times x_i^\ast \times 2 \Delta x
= -19,600 \sum_{i=0}^{N-1} x_i^\ast \Delta x \ .
\]
C'est une somme de Riemann pour l'intégrale
$\displaystyle -19,800 \int_{-17}^{-12} x \dx{x}$.
Donc, si $n \to \infty$, nous obtenons que la pression sur la porte
est
\[
P = -19,800 \int_{-17}^{-12} x \dx{x} =
-9,800 x^2 \bigg|_{-17}^{-12} = 1.421 \times 10^6 \ \text{N.}
\]
}

\compileSOL{\SOLUb}{\ref{8Q29}}{
Supposons que l'origine est au niveau de l'eau du lac et que la
direction positive est vers le haut.  Il faut calculer la force
exercée sur le côté du cylindre pour $-6 \leq x \leq -5.90$.

Partageons l'intervalle $[-6,-5.9]$ en $k$ sous intervalles
$[x_i,x_{i+1}]$ de longueur $\Delta x = 0.1/k$ où
$x_i = -6 + i \Delta x$ pour $i=0$, $1$, $2$, \ldots , $k$.
\PDFgraph{8_integrales_appl/cylindreQuestion}

la pression à une profondeur de $x_i^\ast$ mètres, où
$x_i \leq x_i^\ast \leq x_{i+1}$, est
$-9.8 \times 1000 \times x_i^\ast = -9800\, x_i^\ast$.  Comme
toujours, il ne fait pas oublier que $x_i^\ast <0$ par construction.

La force $F_i$ exercée sur la surface entre $x_i$ et
$x_{i+1}$ mètres est approximativement la force exercée sur le
rectangle (en bleu dans la figure ci-dessus) de hauteur $\Delta x$ et
de longueur $2\sqrt{0.0025 - (x_i^\ast + 5.95)^2}$.  Donc,
\begin{align*}
F_i &\approx \underbrace{\left(- 9800 \, x_i^\ast\right)}_{\text{pression}}
\underbrace{ \left( 2\sqrt{0.0025 - (x_i^\ast + 5.95)^2}
  \Delta x \right)}_{\text{aire}} \\
&= -19600 x_i^\ast \sqrt{0.0025 - (x_i^\ast + 5.95)^2} \Delta x \ .
\end{align*}
Ainsi, la force $F$ à une des extrémités est approximativement
\[
  F \approx \sum_{i=0}^{k-1} F_i
  = -19600 x_i^\ast \sqrt{0.0025 - (x_i^\ast + 5.95)^2} \Delta x \ .
\]
Cette somme approche la valeur de $F$ si $k\to \infty$.  Comme
c'est aussi une somme de Reimann pour
$\displaystyle -19600  \int_{-6}^{-5.9} x \sqrt{0.0025 - (x + 5.95)^2} \dx{x}$,
nous obtenons
\[
  F = -19600  \int_{-6}^{-5.9} x \sqrt{0.0025 - (x + 5.95)^2} \dx{x}
\]
lorsque $k \to \infty$.

Si $u = x+5.95$, alors $\dx{u} = \dx{x}$, $u=-0.05$ lorsque
$x=-6$ et $u=0.05$ lorsque $x = -5.95$.  Donc
\begin{align*}
  F &= -19600  \int_{-0.05}^{0.05} (u-5.95) \sqrt{0.0025 - u^2} \dx{u} \\
&= -19600  \int_{-0.05}^{0.05} u\sqrt{0.0025 - u^2} \dx{u}
+ 19600 \times 5.95 \int_{-0.05}^{0.05} \sqrt{0.0025 - u^2} \dx{u} \ .
\end{align*}
Puisque la première intégrale est l'intégrale d'une fonction
impaire sur un intervalle symétrique à l'origine, cette intégrale est
nul.  La deuxième intégrale est l'aire d'un demi-disque de rayon
$0.05$, donc sa valeur est $\pi \frac{0.0025}{2}$.

Donc $\displaystyle F = 19600 \times 5.95 \times \frac{0.0025}{2}
=145.775\pi$ N.
}

\subsection{Centre de masse}

\compileSOL{\SOLUb}{\ref{8Q31}}{
La masse de la région est
\[
  m = 3 \int_0^2 e^x \dx{x} = 3 e^x\bigg|_0^2 = 3 (e^2 - 1) \ .
\]
Le moment par rapport à l'axe des $y$ est
\[
M_y = 3 \int_0^2 x e^x \dx{x} \ .
\]
Utilisons la méthode d'intégration par parties pour évaluer cette
intégrale.  Si $f(x)=x$ et $g'(x)= e^x$, alors $f'(x) = 1$ et $g(x)=e^x$.
Donc
\[
  M_y = 3 \left( xe^x\bigg|_0^2 - \int_0^2 e^x \dx{x} \right)
  = 3\left( 2 e^2 - e^x\bigg|_0^2 \right) = 3\left( 2e^2 - e^2+1\right) =
  3(e^2 - 1) \ .
\]
Le moment par rapport à l'axe des $x$ est
\[
  M_x = \frac{3}{2} \int_0^2 \left(e^{x}\right)^2 \dx{x}
  = \frac{3}{4} e^{2x}\bigg|_0^2 = \frac{3(e^4-1)}{4} \ .
\]
Ainsi, le centre de masse est
\[
\left( \frac{M_y}{m} , \frac{M_x}{m}\right)
= \left( \frac{3(e^2 -1)}{3(e^2 -1)} , \frac{3(e^4-1)/4}{3(e^2-1)} \right)
= \left( 1 , \frac{e^2+1}{4} \right)
\approx ( 1 , 2.097265 ) \ .
\]
}

\subsection{Applications à l'économie}

\subsection{Test de l'intégrale}

\compileSOL{\SOLUb}{\ref{8Q36}}{
\subQ{a}
Nous avons la séries $\displaystyle \sum_{n=1}^\infty a_n$ où $a_n = g(n)$
pour $\displaystyle g(x) = x e^{-2x^2}$.

Puisque $g$ est une fonction positive et décroissante pour $x>0.5$ car
$g'(x) = (1-4x^2)e^{-2x^2} < 0$ pour $x>0.5$, nous pouvons utiliser le
test de l'intégrale.

Évaluons l'intégrale
\[
  \int_1^N x e^{-2x^2} \dx{x} \ .
\]
Soit $u = -2x^2$, alors $\dx{u} = -4x \dx{x}$, $u= -2N^2$ lorsque
$x=N$ et $u = -2$ lorsque $x=1$.  Donc
\[
\int_1^N x e^{-2x^2} \dx{x} = -\frac{1}{4} \int_1^{-2N^2} e^{u} \dx{u}
= \frac{1}{4} \int_{-2N^2}^{-2} e^{u} \dx{u}
= \frac{1}{4} e^{u} \bigg|_{-2N^2}^{-2} = 
\frac{1}{4} \left(e^{-2} - e^{-2N^2} \right)  \ .
\]
Ainsi,
\[
\int_1^\infty x e^{-2x^2} \dx{x}
= \lim_{N\rightarrow \infty} \int_1^N x e^{-x^2} \dx{x}
= \lim_{N\rightarrow \infty} \frac{1}{4} \left(e^{-2} -e^{-2N^2} \right) 
= \frac{e^{-2}}{4} \ .
\]
Donc, puisque l'intégrale $\displaystyle \int_1^\infty x e^{-2x^2} \dx{x}$
converge, la série converge.
}

\compileSOL{\SOLUb}{\ref{8Q38}}{
Nous avons la séries $\displaystyle \sum_{n=1}^\infty a_n$ où $a_n = f(n)$
pour $\displaystyle f(x) = \frac{1}{x^{8/3}}$.

Puisque $f(x)$ est une fonction positive décroissante qui converge vers
$0$ lorsque $x\to \infty$ et
$\displaystyle \int_1^\infty f(x) \dx{x} = \frac{3}{5}$, il découle
du test de l'intégrale que la séries
$\displaystyle \sum_{n=1}^\infty \frac{1}{n^{8/3}}$
converge car $\displaystyle \int_1^\infty f(x) \dx{x}$ converge.

De plus,
\[
0 \leq S - S_N \leq \int_N^\infty \frac{1}{x^{8/3}} \dx{x}
= \left( -\frac{3}{5x^{5/3}} \right)\bigg|_N^\infty
= \frac{3}{5N^{5/3}} \ .
\]
Il faut choisir $N$ assez grand (mais pas inutilement grand) pour que
$\displaystyle \frac{3}{5N^{5/3}} < 10^{-3}$.  Donc, il faut avoir
\[
600 < N^{5/3} \Rightarrow
\left(600\right)^{3/5} \approx 46.4398 < N \ .
\]
Nous pouvons prendre $N = 47$.
}

%%% Local Variables: 
%%% mode: latex
%%% TeX-master: "notes"
%%% End: 

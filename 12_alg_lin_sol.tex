\section{Algèbre linéaire}

\subsection{Matrices}

\compileSOL{\SOLUb}{\ref{12Q1}}{
\subQ{a}
\[
D+E = \begin{pmatrix} 1 & 5 & 2 \\ -1 & 0 & 1 \\ 3 & 2 & 4 \end{pmatrix}
+ \begin{pmatrix} 6 & 1 & 3 \\ -1 & 1 & 2 \\ 4 & 1 & 3 \end{pmatrix}
= \begin{pmatrix}
1+6 & 5+1 & 2+3 \\
-1-1 & 0+1 & 1+2 \\
3+4 & 2+1 & 4+3
\end{pmatrix}
= \begin{pmatrix} 7 & 6 & 5 \\ -2 & 1 & 3 \\ 7 & 3 & 7 \end{pmatrix} \ .
\]

\subQ{b}
\[
D-E = \begin{pmatrix} 1 & 5 & 2 \\ -1 & 0 & 1 \\ 3 & 2 & 4 \end{pmatrix}
- \begin{pmatrix} 6 & 1 & 3 \\ -1 & 1 & 2 \\ 4 & 1 & 3 \end{pmatrix}
= \begin{pmatrix}
1-6 & 5-1 & 2-3 \\
-1+1 & 0-1 & 1-2 \\
3-4 & 2-1 & 4-3
\end{pmatrix}
= \begin{pmatrix}
-5 & 4 & -1 \\ 0 & -1 & -1 \\ -1 & 1 & 1
\end{pmatrix} \ .
\]

\subQ{c}
\[
5C = 5\begin{pmatrix} 1 & 4 & 2 \\ 3 & 1 & 5 \end{pmatrix}
= \begin{pmatrix}
5\times 1 & 5\times 4 & 5\times 2 \\
5\times 3 & 5\times 1 & 5\times 5
\end{pmatrix}
=  \begin{pmatrix} 5 & 20 & 10 \\ 15 & 5 & 25 \end{pmatrix} \ .
\]

\subQ{d}
\[
-7C = -7\begin{pmatrix} 1 & 4 & 2 \\ 3 & 1 & 5 \end{pmatrix}
= \begin{pmatrix}
-7\times 1 & -7\times 4 & -7\times 2 \\
-7\times 3 & -7\times 1 & -7\times 5
\end{pmatrix}
= \begin{pmatrix} -7 & -28 & -14 \\ -21 & -7 & -35 \end{pmatrix} \ .
\]

\subQ{e} L'opération $2B - C$ n'est pas définie car les matrices $2B$
et $C$ n'ont pas les mêmes dimensions.

\subQ{f}
\[
4E-2D = 
4\begin{pmatrix} 6 & 1 & 3 \\ -1 & 1 & 2 \\ 4 & 1 & 3 \end{pmatrix}
-2 \begin{pmatrix} 1 & 5 & 2 \\ -1 & 0 & 1 \\ 3 & 2 & 4 \end{pmatrix}
=\begin{pmatrix} 22 & -6 & 8 \\ -2 & 4 & 6 \\ 10 & 0 & 4 \end{pmatrix} \ .
\]

\subQ{g}
\[
-3(D+2E) = -3D - 6E
= -3\begin{pmatrix} 1 & 5 & 2 \\ -1 & 0 & 1 \\ 3 & 2 & 4 \end{pmatrix}  
-6 \begin{pmatrix} 6 & 1 & 3 \\ -1 & 1 & 2 \\ 4 & 1 & 3 \end{pmatrix}
= \begin{pmatrix}
-39 & -21 & -24 \\
9 & -6 & -15 \\
-33 & -12 & -30
\end{pmatrix} \ .
\]

\subQ{h}
\[ A-A = 0 = \begin{pmatrix} 0 & 0 \\ 0 & 0 \\ 0 & 0 \end{pmatrix} \ . \]

\subQ{i}
\[
\tr(D) =
\tr \begin{pmatrix} 1 & 5 & 2 \\ -1 & 0 & 1 \\ 3 & 2 & 4 \end{pmatrix}
= 1 + 0 + 4 = 5 \ .
\]

\subQ{j}
\[
\tr(D-3E) = \tr
\begin{pmatrix} -17 & 2 & -7 \\ 2 & -3 & -5 \\ -9 & -1 & -5 \end{pmatrix}
= -17 -3 -5 = -25 \ .
\]

\subQ{k}
$\tr(A)$ n'est pas définie car $A$ n'est pas une matrice carrée.

\subQ{l}
\[
2A^\top+C
= 2 \begin{pmatrix} 3 & 0 \\ -1 & 2 \\ 1 & 1 \end{pmatrix}^\top
+\begin{pmatrix} 1 & 4 & 2 \\ 3 & 1 & 5 \end{pmatrix}
= 2 \begin{pmatrix} 3 & -1 & 1 \\ 0 & 2 & 1 \end{pmatrix}
+ \begin{pmatrix} 1 & 4 & 2 \\ 3 & 1 & 5 \end{pmatrix}
= \begin{pmatrix} 7 & 2 & 4 \\ 3 & 5 & 7 \end{pmatrix}
\ .
\]

\subQ{m}
\begin{align*}
D^\top - E^\top &=
\begin{pmatrix} 1 & 5 & 2 \\ -1 & 0 & 1 \\ 3 & 2 & 4 \end{pmatrix}^\top
- \begin{pmatrix} 6 & 1 & 3 \\ -1 & 1 & 2 \\ 4 & 1 & 3 \end{pmatrix}^\top \\
& = \begin{pmatrix} 1 & -1 & 3 \\ 5 & 0 & 2 \\ 2 & 1 & 4 \end{pmatrix}
- \begin{pmatrix} 6 & -1 & 4 \\ 1 & 1 & 1 \\ 3 & 2 & 3 \end{pmatrix}
= \begin{pmatrix} -5 & 0 & -1 \\ 4 & -1 & 1 \\ -1 & -1 & 1 \end{pmatrix}
= ( D-E)^\top \ .
\end{align*}
Voir (b) pour la valeur de $D-E$.

\subQ{n}
\[
(D-E)^\top = D^\top - E^\top =
\begin{pmatrix} -5 & 0 & -1 \\ 4 & -1 & 1 \\ -1 & -1 & 1 \end{pmatrix} \ .
\]

\subQ{o}
l'opération $B^\top+5C^\top$ n'est pas définie car les matrices
$B^\top$ et $5C^\top$ n'ont pas la même dimension.

\subQ{p}
\[
B - B^\top
= \begin{pmatrix} 6 & -1 \\ 0 & 2 \end{pmatrix}
- \begin{pmatrix} 6 & 0 \\ -1 & 2 \end{pmatrix}
= \begin{pmatrix} 0 & -1 \\ 1 & 0 \end{pmatrix} \ .
\]

\subQ{q}
\[
AB = \begin{pmatrix} 3 & 0 \\ -1 & 2 \\ 1 & 1 \end{pmatrix}
\begin{pmatrix} 6 & -1 \\ 0 & 2 \end{pmatrix}
= \begin{pmatrix}
3\times 6 + 0 & 3\times (-1) + 0 \\
-1\times 6 + 0 & (-1)\times (-1) + 2\times 2 \\
1\times 6 + 0 & 1 \times  (-1) + 1 \times 2
\end{pmatrix}
= \begin{pmatrix} 18 & -3 \\ -6 & 5 \\ 6 & 1 \end{pmatrix}
\ .
\]

\subQ{r}
L'opération $BA$ n'est pas définie car le nombre de colonnes de $B$ n'est pas
égale au nombre de ligne de $A$.

\subQ{s}
\[
(3E)D = 3\begin{pmatrix} 6 & 1 & 3 \\ -1 & 1 & 2 \\ 4 & 1 & 3 \end{pmatrix}
\begin{pmatrix} 1 & 5 & 2 \\ -1 & 0 & 1 \\ 3 & 2 & 4 \end{pmatrix}
= 3 \begin{pmatrix} 14 & 36 & 25 \\ 4 & -1 & 7 \\ 12 & 26 & 21 \end{pmatrix}
=\begin{pmatrix} 42 & 108 & 75 \\ 12 & -3 & 21 \\ 36 & 78 & 63 \end{pmatrix}
\ .
\]

\subQ{t}
\[
A(BC) = \begin{pmatrix} 3 & 0 \\ -1 & 2 \\ 1 & 1 \end{pmatrix}
\left( \begin{pmatrix} 6 & -1 \\ 0 & 2 \end{pmatrix}
  \begin{pmatrix} 1 & 4 & 2 \\ 3 & 1 & 5 \end{pmatrix} \right)
= \begin{pmatrix} 3 & 0 \\ -1 & 2 \\ 1 & 1 \end{pmatrix}
\begin{pmatrix} 3 & 23 & 7 \\ 6 & 2 & 10 \end{pmatrix}
= \begin{pmatrix}
9 & 69 & 21 \\
9 & -19 & 13 \\
9 & 25 & 17
\end{pmatrix} \ .
\]

\subQ{u}
\[
(AB)C = A(BC)
= \begin{pmatrix} 9 & 69 & 21 \\ 9 & -19 & 13 \\ 9 & 25 & 17 \end{pmatrix} \ .
\]

\subQ{v}
\[
CC^\top = \begin{pmatrix} 1 & 4 & 2 \\ 3 & 1 & 5 \end{pmatrix}
\begin{pmatrix} 1 & 3 \\ 4 & 1 \\ 2 & 5 \end{pmatrix}
= \begin{pmatrix} 21 & 17 \\ 17 & 35 \end{pmatrix} \ .
\]

\subQ{w} Puisque
\[
DA = \begin{pmatrix} 1 & 5 & 2 \\ -1 & 0 & 1 \\ 3 & 2 & 4 \end{pmatrix}
\begin{pmatrix} 3 & 0 \\ -1 & 2 \\ 1 & 1 \end{pmatrix}
= \begin{pmatrix} 0 & 12 \\ -2 & 1 \\ 11 & 8 \end{pmatrix} \ ,
\]
nous avons
\[
(DA)^\top = \begin{pmatrix} 0 & -2 & 11 \\ 12 & 1 & 8 \end{pmatrix} ,
\]

\subQ{x}
\[
E^\top D^\top =
\begin{pmatrix} 6 & -1 & 4 \\ 1 & 1 & 1 \\ 3 & 2 & 3 \end{pmatrix}
\begin{pmatrix} 1 & -1 & 3 \\ 5 & 0 & 2 \\ 2 & 1 & 4 \end{pmatrix}
= \begin{pmatrix} 9 & -2 & 32 \\ 8 & 0 & 9 \\ 19 & 0 & 25 \end{pmatrix} \ .
\]

\subQ{y}
\[
(DE)^\top = E^\top D^\top =
\begin{pmatrix} 9 & -2 & 32 \\ 8 & 0 & 9 \\ 19 & 0 & 25 \end{pmatrix} \ .
\]

\subQ{z}
\begin{align*}
(C^\top B)A^\top
&= \left(
\begin{pmatrix} 1 & 3 \\ 4 & 1 \\ 2 & 5 \end{pmatrix}
\begin{pmatrix} 6 & -1 \\ 0 & 2 \end{pmatrix}\right)
\begin{pmatrix} 3 & -1 & 1 \\ 0 & 2 & 1 \end{pmatrix} \\
&= \begin{pmatrix} 6 & 5 \\ 24 & -2 \\ 12 & 8 \end{pmatrix}
\begin{pmatrix} 3 & -1 & 1 \\ 0 & 2 & 1 \end{pmatrix}
= \begin{pmatrix} 18 & 4 & 11 \\ 72 & -28 & 22 \\ 36 & 4 & 20 \end{pmatrix} ,
\end{align*}
}

\compileSOL{\SOLUc}{\ref{12Q2}}{
\subQ{a}
\[
A^2 = \begin{pmatrix}1 & -1 \\0 & 1 \end{pmatrix}
\begin{pmatrix}1 & -1 \\0 & 1 \end{pmatrix}
= \begin{pmatrix}1 & -2 \\0 & 1 \end{pmatrix} \ .
\]

\subQ{b}
\[
A^3 = A^2\,A = \begin{pmatrix}1 & -2 \\0 & 1 \end{pmatrix}
\begin{pmatrix}1 & -1 \\0 & 1 \end{pmatrix}
= \begin{pmatrix}1 & -3 \\0 & 1 \end{pmatrix} \ .
\]

\subQ{c} Nous avons
\begin{equation}\label{solIter}
A^n = \begin{pmatrix}1 & -n \\0 & 1 \end{pmatrix}
\end{equation}
pour $n=0$, $1$, $2$, \ldots

La démonstration de (\ref{solIter}) qui suit est optionnelle.
Nous avons que (\ref{solIter}) est vrai pour $n=1$ (de plus, nous
avons montré en (a) et (b) que (\ref{solIter}) est vrai si $n=2$ et
$3$ respectivement).  Supposons que (\ref{solIter}) soit vrai pour
$n=k$; c'est-à-dire,
\[
A^k = \begin{pmatrix}1 & -k \\0 & 1 \end{pmatrix} \ .
\]
Alors,
\[
A^{k+1} = A^k \, A = \begin{pmatrix}1 & -k \\0 & 1 \end{pmatrix}
\begin{pmatrix}1 & -1 \\0 & 1 \end{pmatrix}
= \begin{pmatrix}1 & -1 -k \\ 0 & 1 \end{pmatrix}
= \begin{pmatrix}1 & -(k+1) \\ 0 & 1 \end{pmatrix} \ .
\]
C'est (\ref{solIter}) avec $n=k+1$.  Ce qui démontre que
(\ref{solIter}) est vrai pour tout entier $n$.  Ce type de
démonstration est appelée
{\bfseries preuve par induction}\index{Preuve par induction}.
}

\subsection{Représentations matricielles des systèmes d'équations
 li\-néaires}

\compileSOL{\SOLUb}{\ref{12Q3}}{
\subQ{a}
La matrice augmentée de ce système est
\[
\left(\begin{array}{rrr|r}
1 & 1 & 2 & 9 \\ 2 & 4 & -3 & 1 \\ 3 & 6 & -5 & 0
\end{array}\right) \; .
\]
$R_2 - 2R_1 \to R_2$ et $R_3 - 3R_1 \to R_3$ donnent
\[
\left(\begin{array}{rrr|r}
1 & 1 & 2 & 9 \\ 0 & 2 & -7 & -17 \\ 0 & 3 & -11 & -27
\end{array}\right) \; .
\]
$(1/2) R_2 \to R_2$ donne
\[
\left(\begin{array}{rrr|r}
1 & 1 & 2 & 9 \\ 0 & 1 & -7/2 & -17/2 \\ 0 & 3 & -11 & -27
\end{array}\right) \; .
\]
$R_3 - 3 R_2 \to R_3$ donne
\[
\left(\begin{array}{rrr|r}
1 & 1 & 2 & 9 \\ 0 & 1 & -7/2 & -17/2 \\ 0 & 0 & -1/2 & -3/2
\end{array}\right) \; .
\]
Finalement, $-2 R_3 \to R_3$ donne
\[
\left(\begin{array}{rrr|r}
1 & 1 & 2 & 9 \\ 0 & 1 & -7/2 & -17/2 \\ 0 & 0 & 1 & 3
\end{array}\right) \; .
\]
Il n'est pas nécessaire de complètement réduire le coté gauche de la
matrice augmentée pour obtenir la matrice identité.  Nous obtenons
$z= 3$ de la troisième ligne.  Ainsi, la deuxième ligne nous donne
$y = (-17/2) + (7/2) z = 2$.  Finalement,  nous obtenons
$x = 9 - y - 2z = 1$ de la première ligne.

\subQ{b}
La matrice augmentée de ce système est
\[
\left(\begin{array}{rrr|r}
5 & 2 & 6 & 0 \\ -2 & 1 & 3 & 1
\end{array}\right) \; .
\]
$R_1 + 2R_2 \to R_1$ donne
\[
\left(\begin{array}{rrr|r}
1 & 4 & 12 & 2 \\ -2 & 1 & 3 & 1
\end{array}\right) \; .
\]
$R_2 + 2 R_1 \to R_2$ donne
\[
\left(\begin{array}{rrr|r}
1 & 4 & 12 & 2 \\ 0 & 9 & 27 & 5
\end{array}\right) \; .
\]
$(1/9) R_2 \to R_2$ donne
\[
\left(\begin{array}{rrr|r}
1 & 4 & 12 & 2 \\ 0 & 1 & 3 & 5/9
\end{array}\right) \; .
\]
Finalement, $R_1 - 4R_2 \to R_1$ donne
\[
\left(\begin{array}{rrr|r}
1 & 0 & 0 & -2/9 \\ 0 & 1 & 3 & 5/9
\end{array}\right) \; .
\]
Nous trouvons donc $x= -2/9$, $y = 5/9 -3\alpha$ et $z=\alpha \in \RR$.  Il y a
une infinité de solutions.

\subQ{c}
La matrice augmentée de ce système est
\[
\left(\begin{array}{rrr|r}
2 & 0 & 2 & 1 \\ 3 & -1 & 4 & 7 \\ 6 & 1 & -1 & 0
\end{array}\right) \; .
\]
$R_3 - 2R_2 \to R_3$ suivie de $R_2 - R_1 \to R_2$ donnent
\[
\left(\begin{array}{rrr|r}
2 & 0 & 2 & 1 \\ 1 & -1 & 2 & 6 \\ 0 & 3 & -9 & -14
\end{array}\right) \; .
\]
$R_2 \leftrightarrow R_1$ donne
\[
\left(\begin{array}{rrr|r}
1 & -1 & 2 & 6 \\ 2 & 0 & 2 & 1 \\ 0 & 3 & -9 & -14
\end{array}\right) \; .
\]
$R_2 - 2 R_1 \to R_2$ donne
\[
\left(\begin{array}{rrr|r}
1 & -1 & 2 & 6 \\ 0 & 2 & -2 & -11 \\ 0 & 3 & -9 & -14
\end{array}\right) \; .
\]
$R_3 - R_2 \to R_3$ donne
\[
\left(\begin{array}{rrr|r}
1 & -1 & 2 & 6 \\ 0 & 2 & -2 & -11 \\ 0 & 1 & -7 & -3
\end{array}\right) \; .
\]
$R_2 - 2 R_3 \to R_2$ donne
\[
\left(\begin{array}{rrr|r}
1 & -1 & 2 & 6 \\ 0 & 0 & 12 & -5 \\ 0 & 1 & -7 & -3
\end{array}\right) \; .
\]
$(1/12) R_2 \to R_2$ suivie de $R_2 \leftrightarrow R_3$ donnent
\[
\left(\begin{array}{rrr|r}
1 & -1 & 2 & 6 \\ 0 & 1 & -7 & -3 \\ 0 & 0 & 1 & -5/12
\end{array}\right) \; .
\]
En procédant à partir de la troisième ligne jusqu'à la première ligne, nous
trouvons $x_3 = -5/12$, $x_2 = -3 + 7 x_3 = -71/12$ et
$x_1 = 6 + x_2 -2 x_3 = 11/12$.

\subQ{d}
La matrice augmentée de ce système est
\[
\left(\begin{array}{rrrr|r}
7 & 2 & 1 & -3 & 5 \\ 1 & 2 & 4 & 0 & 1
\end{array}\right) \; .
\]
$R_1 -7 R_2 \to R_1$ suivie de $R_1 \leftrightarrow R_2$ donne
\[
\left(\begin{array}{rrrr|r}
1 & 2 & 4 & 0 & 1 \\  0 & -12 & -27 & -3 & -2
\end{array}\right) \; .
\]
$-(1/12) R_2 \to R_2$ donne
\[
\left(\begin{array}{rrrr|r}
1 & 2 & 4 & 0 & 1 \\ 0 & 1 & 9/4 & 1/4 & 1/6
\end{array}\right) \; .
\]
$R_1 - 2R_2 \to R_2$ donne
\[
\left(\begin{array}{rrrr|r}
1 & 0 & -1/2 & -1/2 & 2/3 \\ 0 & 1 & 9/4 & 1/4 & 1/6
\end{array}\right) \; .
\]
Nous trouvons donc $x_1= 2/3 + \alpha_1/2 + \alpha_2/2$,
$x_2 = 1/6 - 9\alpha_1/4 - \alpha_2/4$, $x_3 = \alpha_1 \in \RR$ et
$x_4 = \alpha_2 \in \RR$.

\subQ{e}
La matrice augmentée de ce système est
\[
\left(\begin{array}{rrr|r}
3 & 2 & -1 & -15 \\ 3 & 1 & 3 & 11 \\ -6 & -4 & 2 & 30
\end{array}\right) \; .
\]
$R_2 - R_1 \to R_2$ et $R_3 + 2R_1 \to R_3$ donnent
\[
\left(\begin{array}{rrr|r}
3 & 2 & -1 & -15 \\ 0 & -1 & 4 & 26 \\ 0 & 0 & 0 & 0
\end{array}\right) \; .
\]
$R_1 + 2R_2 \to R_1$ suivie de $-R_2 \to R_2$ donnent
\[
\left(\begin{array}{rrr|r}
3 & 0 & 7 & 37 \\ 0 & 1 & -4 & -26 \\ 0 & 0 & 0 & 0
\end{array}\right) \; .
\]
$(1/3) R_1 \to R_1$ donne
\[
\left(\begin{array}{rrr|r}
1 & 0 & 7/3 & 37/3 \\ 0 & 1 & -4 & -26 \\ 0 & 0 & 0 & 0
\end{array}\right) \; .
\]
Ainsi, $x=37/3 - 7\alpha/3$, $y=-26 + 4 \alpha$ et $z = \alpha\in \RR$.
}

\compileSOL{\SOLUb}{\ref{12Q4}}{
Nous avons la matrice augmentée
\[
\left(\begin{array}{rrr|r}
1 & 1 & 2 & a \\
1 & 0 & 1 & b \\
2 & 1 & 3 & c
\end{array}\right) \ .
\]
$R_3 - R_1 - R_2 \to R_3$ donne
\[
\left(\begin{array}{rrr|r}
1 & 1 & 2 & a \\
1 & 0 & 1 & b \\
0 & 0 & 0 & c - a - b
\end{array}\right) \ .
\]
$R_2 - R_1 \to R_2$ donne
\[
\left(\begin{array}{rrr|r}
1 & 1 & 2 & a \\
0 & -1 & -1 & b - a \\
0 & 0 & 0 & c - a - b
\end{array}\right) \ .
\]
$R_1 + R_2 \to R_1$ suivie de $-R_2 \to R_2$ donnent
\[
\left(\begin{array}{rrr|r}
1 & 0 & 1 & b \\
0 & 1 & 1 & a - b \\
0 & 0 & 0 & c - a - b
\end{array}\right) \ .
\]
La seule contrainte pour que le système ait une solution est donnée
par la troisième ligne et c'est $0 = c - a - b$.  Aucune autre
contrainte n'est imposée au système.
}

\compileSOL{\SOLUb}{\ref{12Q5}}{
\subQ{a} Nous avons la matrice augmentée
\[
\left(\begin{array}{cc|c}
1 & a & 1 \\
2 & 3 & b
\end{array}\right) \ .
\]
$R_2 -2R_1 \to R_2$ donne
\begin{equation} \label{twoNSens}
\left(\begin{array}{cc|c}
1 & a & 1 \\
0 & 3- 2a & b - 2
\end{array}\right) \ .
\end{equation}

\subI{I} Si $3-2a \neq 0$, le système a une solution unique.  Nous pouvons
effectuer l'opération $(1/(3-2a)) R_2 \to R_2$ pour obtenir
\[
\left(\begin{array}{cc|c}
1 & a & 1 \\
0 & 1 & (b - 2)/(3 - 2a)
\end{array}\right)
\]
et $R_1 - a R_2 \to R_1$ pour obtenir
\[
\left(\begin{array}{cc|c}
1 & 0 & (3-ab)/(3-2a) \\
0 & 1 & (b - 2)/(3 - 2a)
\end{array}\right) \ .
\]  
Nous avons donc la solution $\displaystyle \begin{pmatrix} x \\ y \end{pmatrix}
= \begin{pmatrix} (3-ab)/(3-2a) \\ (b - 2)/(3 - 2a) \end{pmatrix}$.

\subI{II} Si $3-2a = 0$ et $b-2 =0$, alors (\ref{twoNSens})
est réduit à
\[
\left(\begin{array}{rr|r}
1 & a & 1 \\
0 & 0 & 0
\end{array}\right) \ .
\]
Il y a une infinité de solutions de la forme $x = 1 - a s$
et $y=s \in \RR$ si $a \neq 0$, ou $x=1$ et $y=s \in \RR$ si $a = 0$.

\subI{III} Si $3-2a = 0$ et $b-2 \neq 0$ alors il n'y a pas de solutions
car la dernière ligne de la matrice augmentée (\ref{twoNSens}) est
$0 = b-2 \neq 0$. 

\subQ{b} La matrice augmentée du système est
\[
\left(\begin{array}{cc|c}
1 & a & 1 \\ b & 5 & 2
\end{array}\right) \ .
\]
$R_2 - bR_1 \to R_2$ donne
\begin{equation}\label{twoMSens}
\left(\begin{array}{cc|c}
1 & a & 1 \\ 0 & 5-ab & 2-b
\end{array}\right) \ .
\end{equation}

\subI{I} Si $5-ab \neq 0$, nous pouvons effectuer l'opération
$(1/(5-ab))R_2 \to R_2$ pour obtenir
\[
\left(\begin{array}{cc|c}
1 & a & 1 \\ 0 & 1 & (2-b)/(5-ab)
\end{array}\right)
\]
et l'opération $R_1 - aR_2 \to R_1$ pour obtenir
\[
\left(\begin{array}{cc|c}
1 & 0 & (5-2a)/(5-ab) \\ 0 & 1 & (2-b)/(5-ab)
\end{array}\right) \ .
\]
Ainsi, il y a une solution unique donnée par $x=(5-2a)/(5-ab)$ et
$y=(2-b)/(5-ab)$.

\subI{II} Si $5-ab = 2-b = 0$, nous obtenons la matrice augmentée
\[
\left(\begin{array}{cc|c}
1 & a & 1 \\ 0 & 0 & 0
\end{array}\right) \ .
\]
Il y a une infinité de solution de la forme $x = 1-as$ et
$y = s \in\RR$ si $a\neq 0$, ou $x=1$ et $y = s \in \RR$ si $a=0$.

\subI{III} Si $5-ab =0$ et $ 2-b \neq 0$, il n'y a pas de solutions car
la dernière ligne de la matrice augmentée (\ref{twoMSens}) donne
$0 = 2-b \neq 0$.
}

\compileSOL{\SOLUa}{\ref{12Q6}}{
La matrice augmentée du système est
\[
\left(\begin{array}{ccc|c}
2 & -1 & 3 & -1 \\
1 & 1 & 4 & h \\
2 & -3 & h & 2
\end{array}\right) \ .
\]
$R_1 \leftrightarrow R_2$ donne
\[
\left(\begin{array}{ccc|c}
1 & 1 & 4 & h \\
2 & -1 & 3 & -1 \\
2 & -3 & h & 2
\end{array}\right) \ .
\]
$R_2-2R_1 \rightarrow R_2$ et $R_3-2R_1\rightarrow R_3$ donnent
\[
\left(\begin{array}{ccc|c}
1 & 1 & 4 & h \\
0 & -3 & -5 & -1-2h \\
0 & -5 & h-8 & 2-2h
\end{array}\right) \ .
\]
$(-1/3)R_2 \rightarrow R_2$ donne
\[
\left(\begin{array}{ccc|c}
1 & 1 & 4 & h \\
0 & 1 & 5/3 & (1+2h)/3 \\
0 & -5 & h-8 & 2-2h
\end{array}\right) \ .
\]
Finalement, $R_3+5R_2 \rightarrow R_3$ donne
\[
\left(\begin{array}{ccc|c}
1 & 1 & 4 & h \\
0 & 1 & 5/3 & (1+2h)/3 \\
0 & 0 & (1+3h)/3 & (11+4h)/3
\end{array}\right) \ .
\]
La dernière ligne est
\[
\left(\frac{1+3h}{3}\right)z = \frac{11+4h}{3} \ .
\]

\subQ{a} Si $h\neq -1/3$, alors il y a une seule solution avec
$z = (11+4h)/(1+3h)$, $y=(2h^2-5h-18)/(1+3h)$ et $x=(h^2-10h-26)/(1+3h)$.

\subQ{b} Si $h=-1/3$, nous obtenons $0 = 29/9$ qui n'a pas de sens.  Il n'y a
donc pas de solutions.

\subQ{c} Il n'y a pas de valeurs de $h$ pour lesquelles le système ci-dessus
a un nombre infini de solutions.  Nous ne pouvons pas avoir $1+3h = 0$ et
$11+4h=0$ en même temps.
}

\compileSOL{\SOLUb}{\ref{12Q7}}{
Si $x$ est le nombre d'individus de l'espèce X et $y$ est le nombre
d'individus de l'espèce Y, alors $x$ et $y$ doivent satisfaire
\begin{align*}
5x + 2y &= 900 \\
3x + 4y &= 960
\end{align*}

La matrice augmentée du système est
\[
\left(\begin{array}{cc|c}
5 & 2 & 900 \\
3 & 4 & 960
\end{array}\right) \ .
\]
$R_1 - 2 R_2 \leftrightarrow R_1$ donne
\[
\left(\begin{array}{cc|c}
-1 & -6 & -1020 \\
3 & 4 & 960
\end{array}\right) \ .
\]
$R_2 + 3R_1 \rightarrow R_2$ suivi de $-R_1\rightarrow R_3$ donnent
\[
\left(\begin{array}{cc|c}
1 & 6 & 1020 \\
0 & -14 & -2100
\end{array}\right) \ .
\]
$(-1/14)R_2 \rightarrow R_2$ donne
\[
\left(\begin{array}{cc|c}
1 & 6 & 1020 \\
0 & 1 & 150
\end{array}\right) \ .
\]
Finalement, $R_1- 6 R_2 \rightarrow R_1$ donne
\[
\left(\begin{array}{cc|c}
1 & 0 & 120 \\
0 & 1 & 150
\end{array}\right) \ .
\]
La solution de ce système est $x=120$ et $y=150$.
}

\subsection{Déterminant}

\compileSOL{\SOLUb}{\ref{12Q9}}{
\subQ{a}
La matrice est inversible car $\det(A) = 3 \neq 0$.  Nous avons  
\[ A^{-1} = \begin{pmatrix} 1/3 & 1/3 \\ -2/3 & 1/3 \end{pmatrix}\ . \]

\subQ{b}
La matrice est inversible car $\det(A) = -4 \neq 0$.  Nous avons  
\[ A^{-1} = \begin{pmatrix} 1/4 & 0 \\ 0 & -1 \end{pmatrix}\ . \]

\subQ{c}
La matrice est inversible car $\det(A) = 3 \neq 0$.  Nous avons  
\[ A^{-1} = \begin{pmatrix} 1 & -2/3 \\ 0 & 1/3 \end{pmatrix}\ .\]

\subQ{d}
$A$ n'est pas inversible car $\det(A) = 0$.

\subQ{e}
$A$ n'est pas inversible car $\det(A) = 0$.

\subQ{f}
La matrice est inversible car $\det(A) = -1 \neq 0$.  Nous avons  
\[
A^{-1} = \begin{pmatrix}
-40 & 16 & 9 \\
13 & -5 & -3 \\
5 & -2 & -1
\end{pmatrix} \ .
\]

\subQ{g} Si nous développons selon la deuxième ligne, nous avons
\[
\det(A) = -\det \begin{pmatrix} 1 & 2 \\ 1 & 4 \end{pmatrix}
+0 -\det \begin{pmatrix} 1 & 1 \\ 2 & 1 \end{pmatrix}
= -(4-2) -(1-2) = -1 \neq 0 \ .
\]
Puisque le déterminant de la matrice $A$ est différent de $0$, la
matrice $A$ est inversible.  Nous considérons la matrice augmentée
\[
\left(\begin{array}{ccc@{\quad | \quad}ccc}
1 & 1 & 2 & 1 & 0 & 0 \\1 & 0 & 1 & 0 & 1 & 0 \\2 & 1 & 4 & 0 & 0 & 1
\end{array}\right) \ .
\]
$R_2-R_1\rightarrow R_2$ et $R_3-2\,R_1\rightarrow R_3$ donnent
\[
\left(\begin{array}{ccc@{\quad | \quad}ccc}
1 & 1 & 2 & 1 & 0 & 0 \\ 0 & -1 & -1 & -1 & 1 & 0 \\ 0 & -1 & 0 & -2 & 0 & 1
\end{array}\right) \ . 
\]
$R_2-R_3\rightarrow R_2$ et $R_1+R_3\rightarrow R_1$ donnent
\[
\left(\begin{array}{ccc@{\quad | \quad}ccc}
1 & 0 & 2 & -1 & 0 & 1 \\ 0 & 0 & -1 & 1 & 1 & -1 \\ 0 & -1 & 0 & -2 & 0 & 1
\end{array}\right) \ .
\]
$R_1+2\,R_2\rightarrow R_1$ suivie de $-R_3\rightarrow R_3$ et
$-R_2\rightarrow R_2$ donnent
\[
\left(\begin{array}{ccc@{\quad | \quad}ccc}
1 & 0 & 0 & 1 & 2 & -1 \\ 0 & 0 & 1 & -1 & -1 & 1 \\ 0 & 1 & 0 & 2 & 0 & -1
\end{array}\right) \ .
\]
Finalement, $R_2 \leftrightarrow R_3$ donne
\[
\left(\begin{array}{ccc@{\quad | \quad}ccc}
1 & 0 & 0 & 1 & 2 & -1 \\ 0 & 1 & 0 & 2 & 0 & -1 \\ 0 & 0 & 1 & -1 & -1 & 1
\end{array}\right) \ .
\]
Nous avons donc
\[
A^{-1} =
\begin{pmatrix} 1 & 2 & -1 \\ 2 & 0 & -1 \\ -1 & -1 & 1 \end{pmatrix} \ .
\]
}

\compileSOL{\SOLUb}{\ref{12Q10}}{
\subQ{a} Nous avons
\begin{align*}
AB &= \begin{pmatrix}
1 & 3 & -1 \\
0 & 2 & 1 \\
4 & -2 & 5
\end{pmatrix}\,
\begin{pmatrix}
11 & -5 & 0 \\
-3 & 8 & -1 \\
6 & 2& 4
\end{pmatrix} \\
&= \begin{pmatrix}
11-9-6 & -5+24-2 & -3-4 \\
-6+6 & 16+2 & -2+4 \\
44+6+30 & -20-16+10 & 2+20
\end{pmatrix}
= \begin{pmatrix}
-4 & 17 & -7 \\
0 & 18 & 2 \\
80 & -26 & 22
\end{pmatrix}
\end{align*}
et
\begin{align*}
BA &= \begin{pmatrix}
11 & -5 & 0 \\
-3 & 8 & -1 \\
6 & 2& 4
\end{pmatrix} \,
\begin{pmatrix}
1 & 3 & -1 \\
0 & 2 & 1 \\
4 & -2 & 5
\end{pmatrix} \\
&= \begin{pmatrix}
11 & 33-10 & -11-5 \\
-3-4 & -9+16+2 & 3+8-5 \\
6+16 & 18+4-8 & -6+2+20
\end{pmatrix}
= \begin{pmatrix}
11 & 23 & -16 \\
-7 & 9 & 6 \\
22 & 14 & 16
\end{pmatrix} \ .
\end{align*}
Donc $AB \neq BA$.

\subQ{b} Puisque
\begin{align*}
2A-3B^T-C &= 2\begin{pmatrix}
1 & 3 & -1 \\
0 & 2 & 1 \\
4 & -2 & 5
\end{pmatrix}
-3\begin{pmatrix}
11 & -5 & 0 \\
-3 & 8 & -1 \\
6 & 2& 4
\end{pmatrix}^\top
-\begin{pmatrix}
6 & -1 & 0 \\
15 & 7 & 9 \\
-7 & -2 & -3
\end{pmatrix} \\
&= \begin{pmatrix}
2 & 6 & -2 \\
0 & 4 & 2 \\
8 & -4 & 10
\end{pmatrix}
-\begin{pmatrix}
33 & -9 & 18 \\
-15 & 24 & 6 \\
0 & -3 & 12
\end{pmatrix}
-\begin{pmatrix}
6 & -1 & 0 \\
15 & 7 & 9 \\
-7 & -2 & -3
\end{pmatrix} \\
&= \begin{pmatrix}
2-33-6 & 6+9+1 & -2-18 \\
15-15 & 4-24-7 & 2-6-9 \\
8+7 & -4+3+2 & 10-12+3
\end{pmatrix}
= \begin{pmatrix}
-37 & 16 & -20 \\
0 & -27 & -13 \\
15 & 1 & 1
\end{pmatrix} \ ,
\end{align*}
nous avons
\begin{align*}
\det(2A-3B^T-C) &=
-37 \det
\begin{pmatrix}
-27 & -13 \\ 1 & 1
\end{pmatrix}
-0+15\det
\begin{pmatrix}
16 & -20 \\ -27 & -13
\end{pmatrix} \\
&= -37(-27+13) + 15(-16\times 13 - 20 \times 27) = -10,702
\end{align*}
où le déterminant est développé selon la première colonne.

\subQ{c} Si nous calculons le déterminant de $C$ selon la première ligne,
nous trouvons
\begin{align*}
\det(C) &=
6 \det
\begin{pmatrix}
7 & 9 \\ -2 & -3
\end{pmatrix}
+\det
\begin{pmatrix}
15 & 9 \\ -7 & -3
\end{pmatrix}
+ 0 \\
&= 6(-21+18) + (-45 + 63) = 0 \ .
\end{align*}
Ainsi, la matrice $C$ n'est pas inversible.

\subQ{d} Nous considérons la matrice augmentée
\[
\left(\begin{array}{ccc|ccc}
1 & 3 & -1 & 1 & 0 & 0 \\
0 & 2 & 1 & 0 & 1 & 0 \\
4 & -2 & 5 & 0 & 0 & 1
\end{array}\right) \ .
\]
$R_3-4R_1 \rightarrow R_3$ donne
\[
\left(\begin{array}{ccc|ccc}
1 & 3 & -1 & 1 & 0 & 0 \\
0 & 2 & 1 & 0 & 1 & 0 \\
0 & -14 & 9 & -4 & 0 & 1
\end{array}\right) \ .
\]
$R_3+7R_2 \rightarrow R_3$ donne
\[
\left(\begin{array}{ccc|ccc}
1 & 3 & -1 & 1 & 0 & 0 \\
0 & 2 & 1 & 0 & 1 & 0 \\
0 & 0 & 16 & -4 & 7 & 1
\end{array}\right) \ .
\]
$(1/16)R_3 \rightarrow R_3$ et $(1/2)R_2 \rightarrow R_2$ donnent
\[
\left(\begin{array}{ccc|ccc}
1 & 3 & -1 & 1 & 0 & 0 \\
0 & 1 & 1/2 & 0 & 1/2 & 0 \\
0 & 0 & 1 & -1/4 & 7/16 & 1/16
\end{array}\right) \ .
\]
$R_1+R_3\rightarrow R_1$ et $R_2 - (1/2)R_3 \rightarrow R_2$ donnent
\[
\left(\begin{array}{ccc|ccc}
1 & 3 & 0 & 3/4 & 7/16 & 1/16 \\
0 & 1 & 0 & 1/8 & 9/32 & -1/32 \\
0 & 0 & 1 & -1/4 & 7/16 & 1/16
\end{array}\right) \ .
\]
Finalement, $R_1-3R_2 \rightarrow R_1$ donne
\[
\left(\begin{array}{ccc|ccc}
1 & 0 & 0 & 3/8 & -13/32 & 5/32 \\
0 & 1 & 0 & 1/8 & 9/32 & -1/32 \\
0 & 0 & 1 & -1/4 & 7/16 & 1/16
\end{array}\right) \ .
\]
Ainsi,
\[
A^{-1} =
\begin{pmatrix}
3/8 & -13/32 & 5/32 \\
1/8 & 9/32 & -1/32 \\
-1/4 & 7/16 & 1/16
\end{pmatrix} \ .
\]
}

\subsection{Valeurs propres et vecteurs propres}

\compileSOL{\SOLUb}{\ref{12Q11}}{
Pour chacune des réponses ci-dessous, nous donnons les valeurs propres
de la matrice, un ensemble de vecteurs propres linéairement
indépendant associé à chacune des valeurs propres, et la multiplicité
algébrique des valeurs propres.

\subQ{a}
\begin{center}
\begin{tabular}{c|c|c}
valeur propre & vecteur propre & multiplicité algébrique \\
\hline
\rule{0em}{2em} $-2$ &
$\displaystyle \begin{pmatrix} 1 \\ 0 \end{pmatrix}$ & 1 \\[1em]
$4$ & $\displaystyle \begin{pmatrix} 0 \\ 1 \end{pmatrix}$ & 1
\end{tabular}
\end{center}

\subQ{b}
\begin{center}
\begin{tabular}{c|c|c}
valeur propre & vecteur propre & multiplicité algébrique \\
\hline
\rule{0em}{2em} $4$ &
$\displaystyle \begin{pmatrix} 4 \\ 1 \end{pmatrix}$ & 1 \\[1em]
$-4$ & $\displaystyle \begin{pmatrix} 0 \\ 1 \end{pmatrix}$ & 1
\end{tabular}
\end{center}

\subQ{c}
\begin{center}
\begin{tabular}{c|c|c}
valeur propre & vecteur propre & multiplicité algébrique \\
\hline
\rule{0em}{2.5em}
$3$ & $\displaystyle \begin{pmatrix} 2 \\ -2 \\ 1 \end{pmatrix}$ & 1 \\[1.5em]
$6$ & $\displaystyle \begin{pmatrix} 1 \\ 2 \\ 2 \end{pmatrix}$ & 1 \\[1.5em]
$9$ & $\displaystyle \begin{pmatrix} 2 \\ 1 \\ -2 \end{pmatrix}$ & 1
\end{tabular}
\end{center}

\subQ{d}
\begin{center}
\begin{tabular}{c|c|c}
valeur propre & vecteur propre & multiplicité algébrique \\
\hline
\rule{0em}{2.5em} $-3$ &
$\displaystyle \begin{pmatrix} 1 \\ 2 \\ -1 \end{pmatrix}$ & 1 \\[1.5em]
$5$ & $\VEC{u} = \displaystyle \begin{pmatrix} 3 \\ 0 \\ 1 \end{pmatrix}$
\ et \ 
$\VEC{v} = \displaystyle \begin{pmatrix} -2 \\ 1 \\ 0 \end{pmatrix}$ & 2
\end{tabular}
\end{center}
Nous avons donné deux vecteurs propres, $\VEC{u}$ et $\VEC{v}$, associés à
la valeur propre $5$ en prenant bien soin de choisir deux vecteurs qui
ne sont pas un multiple l'un de l'autre.  Toute
{\em combinaison linéaire} $\alpha \VEC{u} + \beta \VEC{v}$ pour
$\alpha, \beta \in \RR$ est aussi un vecteur propre de $A$ associé à
la valeur propre $5$.  Le lecteur est invité à vérifier que cela est
vrai.

\subQ{e}
\begin{center}
\begin{tabular}{c|c|c}
valeur propre & vecteur propre & multiplicité algébrique \\
\hline
\rule{0em}{2em} $0.8+0.6 i$ &
$\displaystyle \begin{pmatrix} 1 \\ -i \end{pmatrix}$ & 1 \\[1em]
$0.8- 0.6 i$ & $\displaystyle \begin{pmatrix} 1 \\ i \end{pmatrix}$ & 1
\end{tabular}
\end{center}

\subQ{f}
\begin{center}
\begin{tabular}{c|c|c}
valeur propre & vecteur propre & multiplicité algébrique \\
\hline
\rule{0em}{2em} $3$ &
$\VEC{u} = \displaystyle \begin{pmatrix} -2 \\ 1 \end{pmatrix}$ & 2
\end{tabular}
\end{center}
Nous ne pouvons pas trouver d'autres vecteurs propres qui ne soient pas un
multiple de $\VEC{u}$.

\subQ{g} Les valeurs propres de $A$ sont les racines du polynôme
caractéristique
\begin{align*}
p(\lambda) &= \det(A-\lambda \Id) \\
&= \det
\begin{pmatrix}
-\lambda & 0 & 2 \\1 & 3-\lambda & 1 \\2 & 0 & -\lambda
\end{pmatrix} \ .
\end{align*}
Si nous développons selon la deuxième colonne, nous obtenons
\begin{align*}
p(\lambda) &= 0 + (3-\lambda)
\det\begin{pmatrix}
-\lambda & 2 \\ 2 & -\lambda
\end{pmatrix} + 0 \\
&= (3-\lambda)(\lambda^2 - 4) = (3-\lambda)(\lambda - 2)(\lambda + 2) \ .
\end{align*}
Les valeurs propres sont $\lambda_1 = 3$, $\lambda_2 = 2$ et
$\lambda_3 = -2$.

Pour $\lambda_1 = 3$, les vecteurs propres $\VEC{x}$ doivent
satisfaire $(A-\lambda_1\Id)\VEC{x}  = \VEC{0}$, c'est-à-dire,
\[
(A-3 \Id)\VEC{x}
= \begin{pmatrix} -3 & 0 & 2 \\1 & 0 & 1\\2 & 0 & -3 \end{pmatrix}
\begin{pmatrix} x_1 \\x_2 \\x_3 \end{pmatrix}
= \begin{pmatrix} 0 \\ 0 \\ 0 \end{pmatrix} \ .
\]
Nous considérons la matrice augmentée
\[
\left(\begin{array}{ccc@{\quad | \quad}c}
-3 & 0 & 2 & 0 \\1 & 0 & 1 & 0 \\2 & 0 & -3 & 0
\end{array}\right) \ .
\]
$R_1+R_2+R_3\rightarrow R_1$ donne
\[
\left(\begin{array}{ccc@{\quad | \quad}c}
0 & 0 & 0 & 0 \\1 & 0 & 1 & 0 \\2 & 0 & -3 & 0
\end{array}\right) \ .
\]
$R_3-2\,R_2 \rightarrow R_3$ donne
\[
\left(\begin{array}{ccc@{\quad | \quad}c}
0 & 0 & 0 & 0 \\1 & 0 & 1 & 0 \\0 & 0 & -5 & 0
\end{array}\right) \ .
\]
$-(1/5)\,R_3\rightarrow R_3$ suivit de $R_2-R_3 \rightarrow R_2$
donnent
\[
\left(\begin{array}{ccc@{\quad | \quad}c}
0 & 0 & 0 & 0 \\1 & 0 & 0 & 0 \\0 & 0 & 1 & 0
\end{array}\right) \ .
\]
Ainsi, $x_1 = x_3 =0$ et $x_2$ est libre.  Si nous prenons $x_2=1$,
nous obtenons le vecteur propre
\[
\VEC{u}_1 = \begin{pmatrix} 0 \\ 1 \\ 0 \end{pmatrix} \ .
\]

Pour $\lambda_2 = 2$, les vecteurs propres $\VEC{x}$ doivent
satisfaire $(A-\lambda_2\Id)\VEC{x} = \VEC{0}$; c'est-à-dire,
\[
(A-2 \Id)\VEC{x} =
\begin{pmatrix} -2 & 0 & 2 \\1 & 1 & 1 \\2 & 0 & -2 \end{pmatrix}
\begin{pmatrix} x_1 \\ x_2 \\ x_3 \end{pmatrix}
= \begin{pmatrix} 0 \\ 0 \\ 0 \end{pmatrix}  \ .
\]
Nous considérons la matrice augmentée
\[
\left(\begin{array}{ccc@{\quad | \quad}c}
-2 & 0 & 2 & 0 \\1 & 1 & 1 & 0 \\2 & 0 & -2 & 0
\end{array}\right) \ .
\]
$R_3+R_1\rightarrow R_3$ suivit de $-(1/2)R_1\rightarrow R_1$ donnent
\[
\left(\begin{array}{ccc@{\quad | \quad}c}
1 & 0 & -1 & 0 \\1 & 1 & 1 & 0 \\0 & 0 & 0 & 0
\end{array}\right) \ .
\]
$R_2-R_1 \rightarrow R_2$ donne
\[
\left(\begin{array}{ccc@{\quad | \quad}c}
1 & 0 & -1 & 0 \\0 & 1 & 2 & 0 \\0 & 0 & 0 & 0
\end{array}\right) \ .
\]
Ainsi, $x_1-x_3=0$ et $x_2+2x_3=0$.  Donc $x_1=x_3$ et $x_2=-2x_3$.  Si
nous prenons $x_3=1$, nous obtenons le vecteur propre
\[
\VEC{u}_2 = \begin{pmatrix} 1 \\ -2 \\ 1 \end{pmatrix} \ .
\]

Pour $\lambda_3 = -2$, les vecteurs propres $\VEC{x}$ doivent
satisfaire $(A-\lambda_3\Id)\VEC{x} = \VEC{0}$.  Nous avons
\[
(A+2 \Id)\VEC{x} =
\begin{pmatrix} 2 & 0 & 2 \\1 & 5 & 1 \\2 & 0 & 2 \end{pmatrix}
\begin{pmatrix} x_1 \\ x_2 \\ x_3 \end{pmatrix}
= \begin{pmatrix} 0 \\ 0 \\ 0 \end{pmatrix} \ .
\]
Nous considérons la matrice augmentée
\[
\left(\begin{array}{ccc@{\quad | \quad}c}
2 & 0 & 2 & 0 \\1 & 5 & 1 & 0 \\2 & 0 & 2 & 0
\end{array}\right) \ .
\]
$R_3-R_1\rightarrow R_3$ suivit de $(1/2)R_1\rightarrow R_1$ donnent
\[
\left(\begin{array}{ccc@{\quad | \quad}c}
1 & 0 & 1 & 0 \\1 & 5 & 1 & 0 \\0 & 0 & 0 & 0
\end{array}\right) \ .
\]
$R_2-R_1 \rightarrow R_2$ donne
\[
\left(\begin{array}{ccc@{\quad | \quad}c}
1 & 0 & 1 & 0 \\0 & 5 & 0 & 0 \\0 & 0 & 0 & 0
\end{array}\right) \ .
\]
$(1/5)R_2\rightarrow R_2$ donne
\[
\left(\begin{array}{ccc@{\quad | \quad}c}
1 & 0 & 1 & 0 \\0 & 1 & 0 & 0 \\0 & 0 & 0 & 0
\end{array}\right) \ .
\]
Ainsi, $x_1+x_3=0$ et $x_2=0$.  Donc $x_1=-x_3$ et $x_2=0$.  Si
nous prenons $x_3=1$, nous obtenons le vecteur propre
\[
\VEC{u}_3 = \begin{pmatrix} -1 \\ 0 \\ 1 \end{pmatrix} \ .
\]

En conclusion,
\begin{center}
\begin{tabular}{c|c|c}
valeur propre & vecteur propre & multiplicité algébrique \\
\hline
\rule{0em}{2.5em} $\lambda_1 = 3$ &
$\VEC{u}_1
=\displaystyle \begin{pmatrix} 0 \\ 1 \\ 0 \end{pmatrix}$ & 1 \\[1.5em]
$\lambda_2 = 2$ &
$\VEC{u}_2
= \displaystyle \begin{pmatrix} 1 \\ -2 \\ 1 \end{pmatrix}$ & 1 \\[1.5em]
$\lambda_3 -2$ &
$\VEC{u}_3
=\displaystyle \begin{pmatrix} -1 \\ 0 \\ 1 \end{pmatrix}$ & 1
\end{tabular}
\end{center}

\subQ{h}
\begin{center}
\begin{tabular}{c|c|c}
valeur propre & vecteur propre & multiplicité algébrique \\
\hline
\rule{0em}{2.5em} $-2$ &
$\displaystyle \VEC{u} = \begin{pmatrix} 1 \\ 1 \\ 0\end{pmatrix}$ et
$\displaystyle \VEC{v} = \begin{pmatrix} 1 \\ 0 \\ -1\end{pmatrix}$
& 2 \\[1.5em]
$4$ & $\displaystyle \begin{pmatrix} 1 \\ 1 \\ 2\end{pmatrix}$  & 1
\end{tabular}
\end{center}
Nous avons donné deux vecteurs propres, $\VEC{u}$ et $\VEC{v}$, associés à
la valeur propre $-2$ en prenant bien soin de choisir deux vecteurs qui
ne sont pas un multiple l'un de l'autre.  Toute
{\em combinaison linéaire} $\alpha \VEC{u} + \beta \VEC{v}$ pour
$\alpha, \beta \in \RR$ est aussi un vecteur propre de $A$ associé à
la valeur propre $-2$.

\subQ{i}
\begin{center}
\begin{tabular}{c|c|c}
valeur propre & vecteur propre & multiplicité algébrique \\
\hline
\rule{0em}{2.5em} $2$ &
$\displaystyle \begin{pmatrix} 1 \\ 0 \\ -1 \end{pmatrix}$ & 1 \\[1.5em]
$2+\sqrt{2}$ &
$\displaystyle \begin{pmatrix} 1 \\ -\sqrt{2} \\ 1 \end{pmatrix}$ & 1 \\[1.5em]
$2-\sqrt{2}$ &
$\displaystyle \begin{pmatrix} 1 \\ \sqrt{2} \\ 1 \end{pmatrix}$ & 1
\end{tabular}
\end{center}

\subQ{j}
\begin{center}
\begin{tabular}{c|c|c}
valeur propre & vecteur propre & multiplicité algébrique \\
\hline
\rule{0em}{2.5em} $-2$ &
$\displaystyle \VEC{u} = \begin{pmatrix} 1 \\ -2 \\ 0\end{pmatrix}$ et
$\displaystyle \VEC{v} = \begin{pmatrix} 0 \\ 3 \\ 1\end{pmatrix}$
& 2 \\[1.5em]
\rule{0em}{2em} $6$ &
$\displaystyle \begin{pmatrix} 2 \\ 1 \\ -1 \end{pmatrix}$ & 1
\end{tabular}
\end{center}
Nous avons donné deux vecteurs propres, $\VEC{u}$ et $\VEC{v}$, associés à
la valeur propre $-2$ en prenant bien soin de choisir deux vecteurs qui
ne sont pas un multiple l'un de l'autre.  Toute
{\em combinaison linéaire} $\alpha \VEC{u} + \beta \VEC{v}$ pour
$\alpha, \beta \in \RR$ est aussi un vecteur propre de $A$ associé à
la valeur propre $-2$.
}

\compileSOL{\SOLUb}{\ref{12Q12}}{
\subQ{a}\\
\subI{I} Si nous développons selon la première colonne, nous trouvons
\[
\det \begin{pmatrix} -5 & 2 & -3 \\ 0 & 1 & -2 \\ 0 & 1 & 4 \end{pmatrix} 
= -5 \det\begin{pmatrix} 1 & -2 \\ 1 & 4 \end{pmatrix} + 0 + 0
= -5 (4 +2) = -30 \neq 0 \ .
\]
Puisque $\det A \neq 0$, la matrice $A$ est inversible.

\subI{II} Considérons la matrice augmentée
\[
\left(\begin{array}{ccc|ccc}
-5 & 2 & -3 & 1 & 0 & 0 \\ 0 & 1 & -2 & 0 & 1 & 0 \\ 0 & 1 & 4 & 0 & 0
& 1 \end{array}\right) \ .
\]
$R_3-R_2 \rightarrow R_3$ donne
\[
\left(\begin{array}{ccc|ccc}
-5 & 2 & -3 & 1 & 0 & 0 \\ 0 & 1 & -2 & 0 & 1 & 0 \\ 0 & 0 & 6 & 0 & -1
& 1 \end{array}\right) \ .
\]
$(1/6)R_3 \rightarrow R_3$ donne
\[
\left(\begin{array}{ccc|ccc}
-5 & 2 & -3 & 1 & 0 & 0 \\ 0 & 1 & -2 & 0 & 1 & 0 \\ 0 & 0 & 1 & 0 & -1/6
& 1/6 \end{array}\right) \ .
\]
$R_2+2R_3\rightarrow R_2$ et $R_1 +3R_3 \rightarrow R_1$ donnent
\[
\left(\begin{array}{ccc|ccc}
-5 & 2 & 0 & 1 & -1/2 & 1/2 \\ 0 & 1 & 0 & 0 & 2/3 & 1/3
\\ 0 & 0 & 1 & 0 & -1/6 & 1/6 \end{array}\right) \ .
\]
$R_1-2R_2 \rightarrow R_1$ donne
\[
\left(\begin{array}{ccc|ccc}
-5 & 0 & 0 & 1 & -11/6 & -1/6 \\ 0 & 1 & 0 & 0 & 2/3 & 1/3
\\ 0 & 0 & 1 & 0 & -1/6 & 1/6 \end{array}\right) \ .
\]
Finalement, $(-1/5)R_1 \rightarrow R_1$ donne
\[
\left(\begin{array}{ccc|ccc}
1 & 0 & 0 & -1/5 & 11/30 & 1/30 \\ 0 & 1 & 0 & 0 & 2/3 & 1/3
\\ 0 & 0 & 1 & 0 & -1/6 & 1/6 \end{array}\right) \ .
\]
Ainsi,
\[
A^{-1} =
\begin{pmatrix}
-1/5 & 11/30 & 1/30 \\ 0 & 2/3 & 1/3 \\ 0 & -1/6 & 1/6
\end{pmatrix} \ .
\]

\subI{III} Puisque $A$ est inversible, la solution
de $A\VEC{x} = \VEC{b}$ est $\VEC{x} = A^{-1}\VEC{b}$.  Donc
\[
\VEC{x} = 
\begin{pmatrix}
-1/5 & 11/30 & 1/30 \\ 0 & 2/3 & 1/3 \\ 0 & -1/6 & 1/6
\end{pmatrix}
\begin{pmatrix} 3 \\ -5 \\ 15 \end{pmatrix}
= \begin{pmatrix} -29/15 \\ 5/3 \\ 10/3 \end{pmatrix} \ .
\]

\subI{IV} Les valeurs propres de $A$ sont les racines du polynôme
caractéristique
\begin{align*}
p(\lambda) &= \det(A -\lambda \Id)
=\det \begin{pmatrix}
-5-\lambda & 2 & -3 \\ 0 & 1-\lambda & -2 \\ 0 & 1 & 4-\lambda
\end{pmatrix} \\
&=(-5-\lambda) \det \begin{pmatrix} 1-\lambda & -2 \\  1 & 4-\lambda
\end{pmatrix}
=(-5-\lambda)\left( (1-\lambda)(4-\lambda)+2\right) \\
&= -(5+\lambda)(\lambda^2 -5\lambda +6)
= -(5+\lambda)(\lambda -2)(\lambda -3) \ .
\end{align*}
Les valeurs propres de $A$ sont donc $\lambda_1 = -5$, $\lambda_2 = 2$
et $\lambda_3 = 3$.

\subI{V} Un vecteur propre associé à la valeur propre
$\lambda_1 = -5$ est donné par la solution du système
\[
(A-\lambda_1\Id)\VEC{x} = 
\begin{pmatrix}
0 & 2 & -3 \\ 0 & 6 & -2 \\ 0 & 1 & 9
\end{pmatrix}
\begin{pmatrix} x_1 \\ x_2 \\ x_3 \end{pmatrix}
= \begin{pmatrix} 0 \\ 0 \\ 0 \end{pmatrix} \ .
\]
La matrice augmentée de ce système d'équations linéaires est
\[
\left(\begin{array}{ccc|c}
0 & 2 & -3 & 0 \\ 0 & 6 & -2 & 0 \\ 0 & 1 & 9 & 0
\end{array}\right) \ .
\]
$R_1-2R_3 \rightarrow R_1$ et $R_2-6R_3 \rightarrow R_2$ donnent
\[
\left(\begin{array}{ccc|c}
0 & 0 & -21 & 0 \\ 0 & 0 & -56 & 0 \\ 0 & 1 & 9 & 0
\end{array}\right) \ .
\]
$(-1/21)R_1 \rightarrow R_1$ donne
\[
\left(\begin{array}{ccc|c}
0 & 0 & 1 & 0 \\ 0 & 0 & -56 & 0 \\ 0 & 1 & 9 & 0
\end{array}\right) \ .
\]
$R_2+56 R_1\rightarrow R_2$ et $R_3-9R_1 \rightarrow R_3$ donnent
\[
\left(\begin{array}{ccc|c}
0 & 0 & 1 & 0 \\ 0 & 0 & 0 & 0 \\ 0 & 1 & 0 & 0
\end{array}\right) \ .
\]
Finalement, $R_3 \leftrightarrow R_1$ suivie de
$R_2\leftrightarrow R_3$ donnent
\[
\left(\begin{array}{ccc|c}
0 & 1 & 0 & 0 \\ 0 & 0 & 1 & 0 \\ 0 & 0 & 0 & 0
\end{array}\right) \ .
\]
La solution de ce système est $x_2=x_3=0$ et $x_1$ libre.  Avec $x_1=1$,
nous obtenons le vecteur propre
$\displaystyle \VEC{v}_1 = \begin{pmatrix} 1 \\ 0 \\ 0 \end{pmatrix}$.

Un vecteur propre associé à la valeur propre
$\lambda_2 = 2$ est donné par la solution du système
\[
(A-\lambda_2\Id)\VEC{x} = 
\begin{pmatrix}
-7 & 2 & -3 \\ 0 & -1 & -2 \\ 0 & 1 & 2
\end{pmatrix}
\begin{pmatrix} x_1 \\ x_2 \\ x_3 \end{pmatrix}
= \begin{pmatrix} 0 \\ 0 \\ 0 \end{pmatrix} \ .
\]
La matrice augmentée de ce système d'équations linéaires est
\[
\left(\begin{array}{ccc|c}
-7 & 2 & -3 & 0 \\ 0 & -1 & -2 & 0 \\ 0 & 1 & 2 & 0
\end{array}\right) \ .
\]
$R_3+R_2 \rightarrow R_3$ et $R_1+2R_2 \rightarrow R_1$ donnent
\[
\left(\begin{array}{ccc|c}
-7 & 0 & -7 & 0 \\ 0 & -1 & -2 & 0 \\ 0 & 0 & 0 & 0
\end{array}\right) \ .
\]
$(-1/7)R_1 \rightarrow R_1$ et $-R_2\rightarrow R_2$ donnent
\[
\left(\begin{array}{ccc|c}
1 & 0 & 1 & 0 \\ 0 & 1 & 2 & 0 \\ 0 & 0 & 0 & 0
\end{array}\right) \ .
\]
La solution de ce système est $x_1=-x_3$ et $x_2=-2x_3$.  Avec $x_3=-1$,
nous obtenons le vecteur propre
$\displaystyle \VEC{v}_2 = \begin{pmatrix} 1 \\ 2 \\ -1 \end{pmatrix}$.

Finalement, un vecteur propre
associé à la valeur propre $\lambda_3 = 3$ est donné par la solution
du système
\[
(A-\lambda_3\Id)\VEC{x} = 
\begin{pmatrix}
-8 & 2 & -3 \\ 0 & -2 & -2 \\ 0 & 1 & 1
\end{pmatrix}
\begin{pmatrix} x_1 \\ x_2 \\ x_3 \end{pmatrix}
= \begin{pmatrix} 0 \\ 0 \\ 0 \end{pmatrix} \ .
\]
La matrice augmentée de ce système d'équations linéaires est
\[
\left(\begin{array}{ccc|c}
-8 & 2 & -3 & 0 \\ 0 & -2 & -2 & 0 \\ 0 & 1 & 1 & 0
\end{array}\right) \ .
\]
$R_2+2R_3 \rightarrow R_2$ et $R_1-2R_3 \rightarrow R_1$ donnent
\[
\left(\begin{array}{ccc|c}
-8 & 0 & -5 & 0 \\ 0 & 0 & 0 & 0 \\ 0 & 1 & 1 & 0
\end{array}\right) \ .
\]
$(-1/8)R_1 \rightarrow R_1$ et $R_2\leftrightarrow R_3$ donnent
\[
\left(\begin{array}{ccc|c}
1 & 0 & 5/8 & 0 \\ 0 & 1 & 1 & 0 \\ 0 & 0 & 0 & 0
\end{array}\right) \ .
\]
La solution de ce système est $x_1=-5x_3/8$ et $x_2=-x_3$.  Avec $x_3=-8$,
nous obtenons le vecteur propre
$\displaystyle \VEC{v}_3 = \begin{pmatrix} 5 \\ 8 \\ -8 \end{pmatrix}$.

\subQ{b}\\
\subI{I} Si nous développons selon la première colonne, nous trouvons
\[
\det \begin{pmatrix} 3 & 2 & -6 \\ 0 & 1 & -2 \\ 0 & -4 & 3
\end{pmatrix} 
= 3 \det\begin{pmatrix} 1 & -2 \\ -4 & 3 \end{pmatrix} + 0 + 0
= 3 (3 -8 ) = -15 \neq 0 \ .
\]
Puisque $\det A \neq 0$, la matrice $A$ est inversible.

\subI{II} Considérons la matrice augmentée
\[
\left(\begin{array}{ccc|ccc}
3 & 2 & -6 & 1 & 0 & 0 \\ 0 & 1 & -2 & 0 & 1 & 0 \\ 0 & -4 & 3 & 0 & 0
& 1 \end{array}\right) \ .
\]
$R_3+4\,R_2 \rightarrow R_3$ donne
\[
\left(\begin{array}{ccc|ccc}
3 & 2 & -6 & 1 & 0 & 0 \\ 0 & 1 & -2 & 0 & 1 & 0 \\ 0 & 0 & -5 & 0 & 4
& 1 \end{array}\right) \ .
\]
$-(1/5)\,R_3 \rightarrow R_3$ donne
\[
\left(\begin{array}{ccc|ccc}
3 & 2 & -6 & 1 & 0 & 0 \\ 0 & 1 & -2 & 0 & 1 & 0 \\ 0 & 0 & 1 & 0 & -4/5
& -1/5 \end{array}\right) \ .
\]
$R_2+2\,R_3\rightarrow R_2$ et $R_1 +6\,R_3 \rightarrow R_1$ donnent
\[
\left(\begin{array}{ccc|ccc}
3 & 2 & 0 & 1 & -24/5 & -6/5 \\ 0 & 1 & 0 & 0 & -3/5 & -2/5
\\ 0 & 0 & 1 & 0 & -4/5 & -1/5 \end{array}\right) \ .
\]
$R_1-2R_2 \rightarrow R_1$ donne
\[
\left(\begin{array}{ccc|ccc}
3 & 0 & 0 & 1 & -18/5 & -2/5 \\ 0 & 1 & 0 & 0 & -3/5 & -2/5
\\ 0 & 0 & 1 & 0 & -4/5 & -1/5 \end{array}\right) \ .
\]
Finalement, $(1/3)R_1 \rightarrow R_1$ donne
\[
\left(\begin{array}{ccc|ccc}
1 & 0 & 0 & 1/3 & -6/5 & -2/15 \\ 0 & 1 & 0 & 0 & -3/5 & -2/5
\\ 0 & 0 & 1 & 0 & -4/5 & -1/5 \end{array}\right) \ .
\]
Ainsi,
\[
A^{-1} =
\begin{pmatrix}
1/3 & -6/5 & -2/15 \\ 0 & -3/5 & -2/5 \\ 0 & -4/5 & -1/5
\end{pmatrix} \ .
\]

\subI{III} Puisque $A$ est inversible, la solution de $A\VEC{x} =
\VEC{b}$ est $\VEC{x} = A^{-1}\VEC{b}$.  Donc
\[
\VEC{x} = 
\begin{pmatrix}
1/3 & -6/5 & -2/15 \\ 0 & -3/5 & -2/5 \\ 0 & -4/5 & -1/5
\end{pmatrix}
\begin{pmatrix} 3 \\ -5 \\ 15 \end{pmatrix}
= \begin{pmatrix} 5 \\ -3 \\ 1 \end{pmatrix} \ .
\]

\subI{IV} Les valeurs propres de $A$ sont les racines du polynôme
caractéristique
\begin{align*}
p(\lambda) &= \det(A -\lambda \Id)
=\det \begin{pmatrix}
3-\lambda & 2 & -6 \\ 0 & 1-\lambda & -2 \\ 0 & -4 & 3-\lambda
\end{pmatrix} \\
&=(3-\lambda) \det \begin{pmatrix} 1-\lambda & -2 \\  -4 & 3-\lambda
\end{pmatrix}
=(3-\lambda)\left( (1-\lambda)(3-\lambda)-8\right) \\
&= (3-\lambda)(\lambda^2 -4\lambda -5)
= (3-\lambda)(\lambda -5)(\lambda +1) \ .
\end{align*}
Les valeurs propres de $A$ sont donc $\lambda_1 = -1$, $\lambda_2 = 3$
et $\lambda_3 = 5$.

\subI{V} Un vecteur propre
associé à la valeur propre $\lambda_1 = -1$ est donné par la solution
du système
\[
(A-\lambda_1\Id)\VEC{x} = 
\begin{pmatrix}
4 & 2 & -6 \\ 0 & 2 & -2 \\ 0 & -4 & 4
\end{pmatrix}
\begin{pmatrix} x_1 \\ x_2 \\ x_3 \end{pmatrix}
= \begin{pmatrix} 0 \\ 0 \\ 0 \end{pmatrix} \ .
\]
La matrice augmentée de ce système d'équations linéaires est
\[
\left(\begin{array}{ccc|c}
4 & 2 & -6 & 0 \\ 0 & 2 & -2 & 0 \\ 0 & -4 & 4 & 0
\end{array}\right) \ .
\]
$R_3+2R_2 \rightarrow R_3$ et $R_1-2R_2 \rightarrow R_1$ donnent
\[
\left(\begin{array}{ccc|c}
4 & 0 & -4 & 0 \\ 0 & 2 & -2 & 0 \\ 0 & 0 & 0 & 0
\end{array}\right) \ .
\]
$(1/4)R_1 \rightarrow R_1$ et $(1/2)R_2 \rightarrow R_2$ donnent
\[
\left(\begin{array}{ccc|c}
1 & 0 & -1 & 0 \\ 0 & 1 & -1 & 0 \\ 0 & 0 & 0 & 0
\end{array}\right) \ .
\]
La solution de ce système est $x_2=x_3$ et $x_1=x_3$.  Avec $x_3=1$,
nous obtenons le vecteur propre $\displaystyle \VEC{v}_1 =
\begin{pmatrix} 1 \\ 1 \\ 1 \end{pmatrix}$.

Un vecteur propre
associé à la valeur propre $\lambda_2 = 3$ est donné par la solution
du système
\[
(A-\lambda_2\Id)\VEC{x} = 
\begin{pmatrix}
0 & 2 & -6 \\ 0 & -2 & -2 \\ 0 & -4 & 0
\end{pmatrix}
\begin{pmatrix} x_1 \\ x_2 \\ x_3 \end{pmatrix}
=
\begin{pmatrix} 0 \\ 0 \\ 0 \end{pmatrix} \ .
\]
La matrice augmentée de ce système d'équations linéaires est
\[
\left(\begin{array}{ccc|c}
0 & 2 & -6 & 0 \\ 0 & -2 & -2 & 0 \\ 0 & -4 & 0 & 0
\end{array}\right) \ .
\]
$-(1/4)R_3 \rightarrow R_3$ donne
\[
\left(\begin{array}{ccc|c}
0 & 2 & -6 & 0 \\ 0 & -2 & -2 & 0 \\ 0 & 1 & 0 & 0
\end{array}\right) \ .
\]
$R_1-2R_3 \rightarrow R_1$ et $R_2 +2R_3 \rightarrow R_2$ donnent
\[
\left(\begin{array}{ccc|c}
0 & 0 & -6 & 0 \\ 0 & 0 & -2 & 0 \\ 0 & 1 & 0 & 0
\end{array}\right) \ .
\]
$R_1-3R_2 \rightarrow R_1$ donne
\[
\left(\begin{array}{ccc|c}
0 & 0 & 0 & 0 \\ 0 & 0 & -2 & 0 \\ 0 & 1 & 0 & 0
\end{array}\right) \ .
\]
$(-1/2)R_2 \rightarrow R_2$ et $R_1 \rightarrow R_3$ donnent
\[
\left(\begin{array}{ccc|c}
0 & 1 & 0 & 0 \\ 0 & 0 & 1 & 0 \\ 0 & 0 & 0 & 0
\end{array}\right) \ .
\]
La solution de ce système est $x_2=x_3=0$ et $x_1$ est libre.  Avec
$x_1=1$, nous obtenons le vecteur propre $\displaystyle \VEC{v}_2 =
\begin{pmatrix} 1 \\ 0 \\ 0 \end{pmatrix}$.

Finalement, un vecteur propre
associé à la valeur propre $\lambda_3 = 5$ est donné par la solution
du système
\[
(A-\lambda_3\Id)\VEC{x} = 
\begin{pmatrix}
-2 & 2 & -6 \\ 0 & -4 & -2 \\ 0 & -4 & -2
\end{pmatrix}
\begin{pmatrix} x_1 \\ x_2 \\ x_3 \end{pmatrix}
= \begin{pmatrix} 0 \\ 0 \\ 0 \end{pmatrix} \ .
\]
La matrice augmentée de ce système d'équations linéaires est
\[
\left(\begin{array}{ccc|c}
-2 & 2 & -6 & 0 \\ 0 & -4 & -2 & 0 \\ 0 & -4 & -2 & 0
\end{array}\right) \ .
\]
$R_3-R_2 \rightarrow R_3$ et $-(1/2)R_1\rightarrow R_1$ donnent
\[
\left(\begin{array}{ccc|c}
1 & -1 & 3 & 0 \\ 0 & -4 & -2 & 0 \\ 0 & 0 & 0 & 0
\end{array}\right) \ .
\]
$(-1/4)R_2 \rightarrow R_2$ donne
\[
\left(\begin{array}{ccc|c}
1 & -1 & 3 & 0 \\ 0 & 1 & 1/2 & 0 \\ 0 & 0 & 0 & 0
\end{array}\right) \ .
\]
$R_1+R_2 \rightarrow R_1$ donne
\[
\left(\begin{array}{ccc|c}
1 & 0 & 7/2 & 0 \\ 0 & 1 & 1/2 & 0 \\ 0 & 0 & 0 & 0
\end{array}\right) \ .
\]
La solution de ce système est $x_1=-7x_3/2$ et $x_2=-x_3/2$.  Avec
$x_3=2$, nous obtenons le vecteur propre
$\displaystyle \VEC{v}_3 = \begin{pmatrix} -7 \\ -1 \\ 2 \end{pmatrix} $.

\subQ{c}\\
\subI{I} Si nous développons selon la première colonne, nous trouvons
\[
\det \begin{pmatrix} 4 & 2 & -5 \\ 0 & 1 & 2 \\ 0 & 6 & -3
\end{pmatrix}
= 4 \det\begin{pmatrix} 1 & 2 \\ 6 & -3 \end{pmatrix} + 0 + 0
= 4 (-3 -12) = -60 \neq 0 \ .
\]
Puisque $\det A \neq 0$, la matrice $A$ est inversible.

\subI{II} Considérons la matrice augmentée
\[
\left(\begin{array}{ccc|ccc}
4 & 2 & -5 & 1 & 0 & 0 \\ 0 & 1 & 2 & 0 & 1 & 0 \\ 0 & 6 & -3 & 0 & 0
& 1 \end{array}\right) \ .
\]
$R_1-2R_2 \rightarrow R_1$ et $R_3-6R_2\rightarrow R_3$ donnent
\[
\left(\begin{array}{ccc|ccc}
4 & 0 & -9 & 1 & -2 & 0 \\ 0 & 1 & 2 & 0 & 1 & 0 \\ 0 & 0 & -15 & 0 &
-6 & 1 \end{array}\right) \ .
\]
$-(1/15)R_3 \rightarrow R_3$ et $(1/4)\,R_1\rightarrow R_1$ donnent
\[
\left(\begin{array}{ccc|ccc}
1 & 0 & -9/4 & 1/4 & -1/2 & 0 \\ 0 & 1 & 2 & 0 & 1 & 0 \\ 0 & 0 & 1 & 0 & 2/5
& -1/15 \end{array}\right) \ .
\]
$R_2-2R_3\rightarrow R_2$ et $R_1 +(9/4)\,R_3 \rightarrow R_1$ donnent
\[
\left(\begin{array}{ccc|ccc}
1 & 0 & 0 & 1/4 & 2/5 & -3/20 \\ 0 & 1 & 0 & 0 & 1/5 & 2/15
\\ 0 & 0 & 1 & 0 & 2/5 & -1/15 \end{array}\right) \ .
\]
Ainsi,
\[
A^{-1} =
\begin{pmatrix}
1/4 & 2/5 & -3/20 \\ 0 & 1/5 & 2/15 \\ 0 & 2/5 & -1/15
\end{pmatrix} \ .
\]

\subI{III} Puisque $A$ est inversible, la solution de $A\VEC{x} = \VEC{b}$
est $\VEC{x} = A^{-1}\VEC{b}$.  Donc
\[
\VEC{x} = 
\begin{pmatrix}
1/4 & 2/5 & -3/20 \\ 0 & 1/5 & 2/15 \\ 0 & 2/5 & -1/15
\end{pmatrix}
\begin{pmatrix} 3 \\ -5 \\ 15 \end{pmatrix}
= \begin{pmatrix} -7/2 \\ 1 \\ -3 \end{pmatrix} \ .
\]

\subI{IV} Les valeurs propres de $A$ sont les racines du polynôme
caractéristique
\begin{align*}
p(\lambda) &= \det(A -\lambda \Id)
=\det \begin{pmatrix}
4-\lambda & 2 & -5 \\ 0 & 1-\lambda & 2 \\ 0 & 6 & -3-\lambda
\end{pmatrix} \\
&=(4-\lambda) \det \begin{pmatrix} 1-\lambda & 2 \\  6 & -3-\lambda
\end{pmatrix}
=(4-\lambda)\left( (1-\lambda)(-3-\lambda)-12\right) \\
&= (4-\lambda)(\lambda^2 +2\,\lambda -15 )
= (4-\lambda)(\lambda -3)(\lambda +5) \ .
\end{align*}
Les valeurs propres de $A$ sont donc $\lambda_1 = -5$, $\lambda_2 = 3$
et $\lambda_3 = 4$.

\subI{V} Un vecteur propre
associé à la valeur propre $\lambda_1 = -5$ est donné par la solution
du système
\[
(A-\lambda_1\Id)\VEC{x} = 
\begin{pmatrix}
9 & 2 & -5 \\ 0 & 6 & 2 \\ 0 & 6 & 2
\end{pmatrix}
\begin{pmatrix} x_1 \\ x_2 \\ x_3 \end{pmatrix}
=
\begin{pmatrix} 0 \\ 0 \\ 0 \end{pmatrix} \ .
\]
La matrice augmentée de ce système d'équations linéaires est
\[
\left(\begin{array}{ccc|c}
9 & 2 & -5 & 0 \\ 0 & 6 & 2 & 0 \\ 0 & 6 & 2 & 0
\end{array}\right) \ .
\]
$R_3-R_2 \rightarrow R_3$ donne
\[
\left(\begin{array}{ccc|c}
9 & 2 & -5 & 0 \\ 0 & 6 & 2 & 0 \\ 0 & 0 & 0 & 0
\end{array}\right) \ .
\]
$(1/6)R_2 \rightarrow R_2$ et $(1/9)R_1 \rightarrow R_1$ donnent
\[
\left(\begin{array}{ccc|c}
1 & 2/9 & -5/9 & 0 \\ 0 & 1 & 1/3 & 0 \\ 0 & 0 & 0 & 0
\end{array}\right) \ .
\]
$R_1 - (2/9)R2 \rightarrow R_1$ donne
\[
\left(\begin{array}{ccc|c}
1 & 0 & -17/27 & 0 \\ 0 & 1 & 1/3 & 0 \\ 0 & 0 & 0 & 0
\end{array}\right) \ .
\]
La solution de ce système est $x_1=17x_3/27$ et $x_2=-x_3/3$ est
libre.  Avec $x_3=27$, nous obtenons le vecteur propre
$\displaystyle \VEC{v}_1 = \begin{pmatrix} 17 \\ -9 \\ 27 \end{pmatrix} $.

Un vecteur propre
associé à la valeur propre $\lambda_2 = 3$ est donné par la solution
du système
\[
(A-\lambda_2\Id)\VEC{x} = 
\begin{pmatrix}
1 & 2 & -5 \\ 0 & -2 & 2 \\ 0 & 6 & -6
\end{pmatrix}
\begin{pmatrix} x_1 \\ x_2 \\ x_3 \end{pmatrix}
=
\begin{pmatrix} 0 \\ 0 \\ 0 \end{pmatrix} \ .
\]
La matrice augmentée de ce système d'équations linéaires est
\[
\left(\begin{array}{ccc|c}
1 & 2 & -5 & 0 \\ 0 & -2 & 2 & 0 \\ 0 & 6 & -6 & 0
\end{array}\right) \ .
\]
$R_3+3R_2 \rightarrow R_3$ donne
\[
\left(\begin{array}{ccc|c}
1 & 2 & -5 & 0 \\ 0 & -2 & 2 & 0 \\ 0 & 0 & 0 & 0
\end{array}\right) \ .
\]
$(-1/2)R_2 \rightarrow R_2$ donne
\[
\left(\begin{array}{ccc|c}
1 & 2 & -5 & 0 \\ 0 & 1 & -1 & 0 \\ 0 & 0 & 0 & 0
\end{array}\right) \ .
\]
$R_1-2\,R_2 \rightarrow R_1$ donne
\[
\left(\begin{array}{ccc|c}
1 & 0 & -3 & 0 \\ 0 & 1 & -1 & 0 \\ 0 & 0 & 0 & 0
\end{array}\right) \ .
\]
La solution de ce système est $x_1=3x_3$ et $x_2=x_3$.  Avec $x_3=1$,
nous obtenons le vecteur propre
$\displaystyle \VEC{v}_2 = \begin{pmatrix} 3 \\ 1 \\ 1 \end{pmatrix}$.

Finalement, un vecteur propre
associé à la valeur propre $\lambda_3 = 4$ est donné par la solution
du système
\[
(A-\lambda_3\Id)\VEC{x} = 
\begin{pmatrix}
0 & 2 & -5 \\ 0 & -3 & 2 \\ 0 & 6 & -7
\end{pmatrix}
\begin{pmatrix} x_1 \\ x_2 \\ x_3 \end{pmatrix}
=
\begin{pmatrix} 0 \\ 0 \\ 0 \end{pmatrix} \ .
\]
La matrice augmentée de ce système d'équations linéaires est
\[
\left(\begin{array}{ccc|c}
0 & 2 & -5 & 0 \\ 0 & -3 & 2 & 0 \\ 0 & 6 & -7 & 0
\end{array}\right) \ .
\]
$R_3+2R_2 \rightarrow R_3$ et $R_1+R_2 \rightarrow R_1$ donnent
\[
\left(\begin{array}{ccc|c}
0 & -1 & -3 & 0 \\ 0 & -3 & 2 & 0 \\ 0 & 0 & -3 & 0
\end{array}\right) \ .
\]
$-R_1 \rightarrow R_1$ et $(-1/3)R_3\leftrightarrow R_3$ donnent
\[
\left(\begin{array}{ccc|c}
0 & 1 & 3 & 0 \\ 0 & -3 & 2 & 0 \\ 0 & 0 & 1 & 0
\end{array}\right) \ .
\]
$R_1-3\,R_3 \rightarrow R_1$ et $R_2-2\,R_3 \leftrightarrow R_2$
donnent
\[
\left(\begin{array}{ccc|c}
0 & 1 & 0 & 0 \\ 0 & -3 & 0 & 0 \\ 0 & 0 & 1 & 0
\end{array}\right) \ .
\]
$R_2+3\,R_1 \rightarrow R_2$ suivie de $R_2\leftrightarrow R_3$
donnent
\[
\left(\begin{array}{ccc|c}
0 & 1 & 0 & 0 \\ 0 & 0 & 1 & 0 \\ 0 & 0 & 0 & 0
\end{array}\right) \ .
\]
La solution de ce système est $x_2=x_3=0$ et $x_1$ est libre.  Avec $x_1=1$,
nous obtenons le vecteur propre
$\displaystyle \VEC{v}_3 = \begin{pmatrix} 1 \\ 0 \\ 0 \end{pmatrix}$.

\subQ{d}\\
\subI{I} Si nous développons selon la deuxième ligne, nous obtenons
\begin{align*}
\det A  &= -a_{2,1} \det(A_{2,1}) + a_{2,2} \det(A_{2,2})
- a_{2,3}\det(A_{2,3}) \\
&= 0 - 4 \det\begin{pmatrix}2 & 4 \\ 4 & 2 \end{pmatrix} - 0
= -4( 4-16) = 48 \ .
\end{align*}
Puisque $\det A \neq 0$, la matrice $A$ est inversible.

\subI{II} Considérons la matrice augmentée
\[
(A\,|\, \Id) =
\left(
\begin{array}{ccc|ccc}
2 & 0 & 4 & 1 & 0 & 0 \\
0 & -4 & 0 & 0 & 1 & 0 \\
4 & 0 & 2 & 0 & 0 & 1
\end{array} \right) \ .
\]
$(-1/4) R_2 \to R_2$ et $(1/2)R_1 \to R_1$ donnent
\[
\left(
\begin{array}{ccc|ccc}
1 & 0 & 2 & 1/2 & 0 & 0 \\
0 & 1 & 0 & 0 & -1/4 & 0 \\
4 & 0 & 2 & 0 & 0 & 1 
\end{array} \right) \ .
\]
$R_3 - 4 R_1 \to R_3$ donne
\[
\left(
\begin{array}{ccc|ccc}
1 & 0 & 2 & 1/2 & 0 & 0 \\
0 & 1 & 0 & 0 & -1/4 & 0 \\
0 & 0 & -6 & -2 & 0 & 1 
\end{array} \right) \ .
\]
$(-1/6) R_3 \to R_3$ donne
\[
\left(
\begin{array}{ccc|ccc}
1 & 0 & 2 & 1/2 & 0 & 0 \\
0 & 1 & 0 & 0 & -1/4 & 0 \\
0 & 0 & 1 & 1/3 & 0 & -1/6 
\end{array} \right) \ .
\]
Finalement, $R_1 -2 R_3 \to R_1$ donne
\[
\left(
\begin{array}{ccc|ccc}
1 & 0 & 0 & -1/6 & 0 & 1/3 \\
0 & 1 & 0 & 0 & -1/4 & 0 \\
0 & 0 & 1 & 1/3 & 0 & -1/6 
\end{array} \right)  \ .
\]
L'inverse de la matrice $A$ est donc
\[
A^{-1} = \begin{pmatrix}
-1/6 & 0 & 1/3 \\
0 & -1/4 & 0 \\
1/3 & 0 & -1/6 
\end{pmatrix} \ .
\]

\subI{III} Puisque $A$ est inversible, la solution de $A\VEC{x} =
\VEC{b}$ est $\VEC{x} = A^{-1}\VEC{b}$.  Donc
\[
\begin{pmatrix} x \\ y \\ z \end{pmatrix}
= A^{-1}\begin{pmatrix} 12 \\ 12 \\ 12 \end{pmatrix}
= \begin{pmatrix}
-1/6 & 0 & 1/3 \\
0 & -1/4 & 0 \\
1/3 & 0 & -1/6 
\end{pmatrix}\begin{pmatrix} 12 \\ 12 \\ 12 \end{pmatrix}
= \begin{pmatrix}  2 \\ -3 \\ 2 \end{pmatrix} \ . 
\]

\subI{IV}
Les valeurs propres de la matrice $A$ sont les racines du polynôme
caractéristique
\begin{align*}
p(\lambda) &= \det(A-\lambda \Id)=
\begin{pmatrix}
2-\lambda & 0 & 4 \\
0 & -4-\lambda & 0 \\
4 & 0 & 2-\lambda
\end{pmatrix} \\
&= (-4-\lambda)\left( (2-\lambda)^2 - 16 \right)
= (-4-\lambda)\left( \lambda^2 - 4 \lambda -12 \right) \\
&= (-4-\lambda)(\lambda - 6)(\lambda +2) \ .
\end{align*}
Les valeurs propres de $A$ sont donc $\lambda_1 = -4$, $\lambda_2 = -2$
et $\lambda_3 = 6$.

\subI{V} Un vecteur propre
associé à la valeur propre $\lambda_1 = -4$ est donné par la solution
du système
\[
(A-\lambda_1\Id)\VEC{x} = 
\begin{pmatrix}
6 & 0 & 4 \\0 & 0 & 0 \\4 & 0 & 6
\end{pmatrix}
\begin{pmatrix} x_1 \\ x_2 \\ x_3 \end{pmatrix}
= \begin{pmatrix} 0 \\ 0 \\ 0 \end{pmatrix} \ .
\]
Nous pourrions considérer la matrice augmentée de ce système d'équations
linéaires pour trouver le vecteur $\VEC{v}_1$.  Cependant, il est
facile de voir que 
$\displaystyle \VEC{v}_1 = \begin{pmatrix} 0 \\ 1 \\ 0 \end{pmatrix}$
est un vecteur propre associé à $\lambda_1 = -4$.

Un vecteur propre
associé à la valeur propre $\lambda_2 = -2$ est donné par la solution
du système
\[
(A-\lambda_2\Id)\VEC{x} = 
\begin{pmatrix}
4 & 0 & 4 \\0 & -2 & 0 \\4 & 0 & 4
\end{pmatrix}
\begin{pmatrix} x \\ y \\ z \end{pmatrix}
= \begin{pmatrix} 0 \\ 0 \\ 0 \end{pmatrix} \ .
\]
La matrice augmentée de ce système d'équations linéaires est
\[
\left(\begin{array}{ccc|c}
4 & 0 & 4 & 0 \\0 & -2 & 0 & 0 \\4 & 0 & 4 & 0 
\end{array}\right) \ .
\]
$R_3-R_1 \rightarrow R_3$ et $(-1/2)R_2 \rightarrow R_2$ donnent
\[
\left(\begin{array}{ccc|c}
4 & 0 & 4 & 0 \\ 0 & 1 & 0 & 0 \\ 0 & 0 & 0 & 0
\end{array}\right) \ .
\]
$(1/4)R_1 \rightarrow R_1$ donne
\[
\left(\begin{array}{ccc|c}
1 & 0 & 1 & 0 \\ 0 & 1 & 0 & 0 \\ 0 & 0 & 0 & 0
\end{array}\right) \ .
\]
La solution de ce système est $x_1=-x_3$ et $x_2=0$.  Avec $x_1=1$,
nous obtenons le vecteur propre
$\displaystyle \VEC{v}_2 = \begin{pmatrix} 1 \\ 0 \\ -1 \end{pmatrix}$.

Un vecteur propre
associé à la valeur propre $\lambda_3 = 6$ est donné par la solution
du système
\[
(A-\lambda_3\Id)\VEC{x} = 
\begin{pmatrix}
-4 & 0 & 4 \\0 & -10 & 0 \\4 & 0 & -4
\end{pmatrix}
\begin{pmatrix} x_1 \\ x_2 \\ x_3 \end{pmatrix}
= \begin{pmatrix} 0 \\ 0 \\ 0 \end{pmatrix} \ .
\]
La matrice augmentée de ce système d'équations linéaires est
\[
\left(\begin{array}{ccc|c}
-4 & 0 & 4 & 0 \\0 & -10 & 0 & 0 \\4 & 0 & -4 & 0 
\end{array}\right) \ .
\]
$R_3+R_1 \rightarrow R_3$ et $(-1/10)R_2 \rightarrow R_2$ donnent
\[
\left(\begin{array}{ccc|c}
-4 & 0 & 4 & 0 \\ 0 & 1 & 0 & 0 \\ 0 & 0 & 0 & 0
\end{array}\right) \ .
\]
$(-1/4)R_1 \rightarrow R_1$ donne
\[
\left(\begin{array}{ccc|c}
1 & 0 & -1 & 0 \\ 0 & 1 & 0 & 0 \\ 0 & 0 & 0 & 0
\end{array}\right) \ .
\]
La solution de ce système est $x_1=x_3$ et $x_2=0$.  Avec $x_1=1$,
nous obtenons le vecteur propre
$\displaystyle \VEC{v}_3 = \begin{pmatrix} 1 \\ 0 \\ 1 \end{pmatrix}$.
}

\subsection{Chaînes de Markov}

\compileSOL{\SOLUb}{\ref{12Q13}}{
\subQ{a} Nous avons que $x_1 = 0.7\,x_0 + 0.2 \, y_0$ et
$y_1 = 0.3\,x_0 + 0.8\,y_0$, ainsi
\[
\begin{pmatrix} x_1 \\ y_1 \end{pmatrix} =
\begin{pmatrix} 0.7 & 0.2 \\ 0.3 & 0.8  \end{pmatrix}
\begin{pmatrix} x_0 \\ y_0 \end{pmatrix} \ .
\]
Donc
\[
A = \begin{pmatrix} 0.7 & 0.2 \\ 0.3 & 0.8  \end{pmatrix} \ .
\]

\subQ{b}
\[
A^2 = \begin{pmatrix} 0.7 & 0.2 \\ 0.3 & 0.8  \end{pmatrix}
\begin{pmatrix} 0.7 & 0.2 \\ 0.3 & 0.8  \end{pmatrix}
= \begin{pmatrix} 0.55 & 0.3 \\ 0.45 & 0.7  \end{pmatrix} \ .
\]

\subQ{c} Puisque la multiplication par la matrice $A$ donne le nombre
de voitures dans chaque succursale à la fin du mois (et donc au début
du mois suivant) en fonction du nombre de voitures dans chaque
succursale au début du mois, nous avons que la multiplication par la matrice
$A^2$ donne le nombre de voiture dans chaque succursale à la fin du
deuxième mois en fonction du nombre de voitures dans chaque succursale
au début de la période de deux mois.  Nous pouvons généraliser.  La matrice
$A^n$ donne le nombre de voitures dans chaque succursale à la fin du
n$^e$ mois en fonction du nombre de voitures dans chaque succursale
au début de la période de $n$ mois.

Nous avons au début de mars
\[
\begin{pmatrix} x_0 \\ y_0 \end{pmatrix} =
\begin{pmatrix} 60 \\ 40 \end{pmatrix} \ .
\]
Puisque
\[
\begin{pmatrix} x_1 \\ y_1 \end{pmatrix}
= A \begin{pmatrix} 60 \\ 40 \end{pmatrix}
= \begin{pmatrix} 0.7 & 0.2 \\ 0.3 & 0.8  \end{pmatrix}
\begin{pmatrix} 60 \\ 40 \end{pmatrix}
= \begin{pmatrix} 50 \\ 50 \end{pmatrix} \ ,
\]
nous aurons $50$ voitures dans la succursale S et aussi $50$ voitures dans la
succursale T à la fin du mois de mars (et donc au début d'avril).

Puisque
\[
\begin{pmatrix} x_2 \\ y_2 \end{pmatrix}
= A^2 \begin{pmatrix} 60 \\ 40 \end{pmatrix}
= \begin{pmatrix} 0.55 & 0.3 \\ 0.45 & 0.7  \end{pmatrix}
\begin{pmatrix} 60 \\ 40 \end{pmatrix}
=
\begin{pmatrix} 45 \\ 55 \end{pmatrix} \ ,
\]
nous aurons $45$ voitures dans la succursale S et $55$ voitures dans la
succursale T à la fin du mois d'avril (et donc au début du mois de mai).

Pour allez du début d'avril au début de mai, nous aurions pu procéder de la
façon suivant.  Si
\[
\begin{pmatrix} x_1 \\ y_1 \end{pmatrix} =
\begin{pmatrix} 50 \\ 50 \end{pmatrix}
\]
($50$ voitures dans chaque succursale au début d'avril), alors
\[
\begin{pmatrix} x_2 \\ y_2 \end{pmatrix} =
\begin{pmatrix} 0.7 & 0.2 \\ 0.3 & 0.8  \end{pmatrix}
\begin{pmatrix} 50 \\ 50 \end{pmatrix}
=
\begin{pmatrix} 45 \\ 55 \end{pmatrix} \ .
\]
Donc, au début de mai, il y aura $45$ voitures dans la succursale S et
$55$ voitures dans la succursale T.

\subQ{d} Il faut trouver un vecteur $\displaystyle \VEC{v}$ tel
que $A\VEC{v} = \VEC{v}$.  Nous cherchons donc un vecteur propre non nul
associé à la valeur propre $1$.

Comme $A$ est une matrice de Markov, $1$ est bien une valeur propre de
$A$.  Nous pouvons vérifier que $1$ est une valeur propre de $A$ en
trouvant les valeurs propre de $A$.  Les valeurs propres de $A$ sont
les racines du polynôme caractéristique
\begin{align*}
p(\lambda) &= \det(A-\lambda \Id) =
\det\begin{pmatrix} 0.7-\lambda & 0.2 \\ 0.3 & 0.8-\lambda
\end{pmatrix} \\
&=(0.7-\lambda)(0.8-\lambda) - 0.06 = \lambda^2 -1.5\lambda+0.5
= (\lambda-1)(\lambda-0.5)
\end{align*}
Donc $1$ est bien une valeur propre de $A$,  Un vecteur propre
$\VEC{v}$ de $A$ associé à la valeur propre $1$ est donné par une
solution de $(A-\Id)\VEC{v}=\VEC{0}$; c'est-à-dire,
\[
(A-\Id)\VEC{v}
= \begin{pmatrix} -0.3 & 0.2 \\ 0.3 & -0.2 \end{pmatrix}
\begin{pmatrix} v_1 \\ v_2 \end{pmatrix}
= \begin{pmatrix} 0 \\ 0 \end{pmatrix} \ .
\]
Nous considérons la matrice augmentée
\[
\left(\begin{array}{cc@{\quad |\quad}c}
-0.3 & 0.2 & 0 \\ 0.3 & -0.2 & 0 \end{array}\right) \ .
\]
$R_2+R_1 \rightarrow R_2$ suivit de $-(10/3)R_1 \rightarrow R_1$
donnent
\[
\left(\begin{array}{cc@{\quad |\quad}c}
1 & -2/3 & 0 \\ 0 & 0 & 0 \end{array}\right) \ .
\]
Nous avons que $v_1 = 2v_2/3$.  Comme le nombre de voitures est constant, nous
devons aussi avoir $v_1 + v_2 = 100$.  Si $v_1 = 2v_2/3$, alors
$v_1 + v_2 = 100$ donne $v_2 = 60$.  Donc $v_1 = 40$.
Il faut qu'il y est $40$ voitures à la succursale S et $60$ voitures à
la succursale T.
}

\compileSOL{\SOLUa}{\ref{12Q14}}{
\subQ{a} Si $x_i$ et $y_i$ dénotent le nombre de lots qui ont été
choisis et le nombre de lots qui n'ont pas été choisis à la $i^e$
année respectivement, alors la chaîne de Markov est décrite par
\[
\begin{pmatrix} x_{i+1} \\ y_{i+1} \end{pmatrix}
= A \begin{pmatrix} x_i \\ y_i \end{pmatrix}
\]
pour $i=0$, $1$, $2$, \ldots\ avec la matrice de transition
\[
A = \begin{pmatrix} 0.2 & 0.5 \\ 0.8 & 0.5 \end{pmatrix} \ .
\]

\subQ{b} Puisque
\[
\begin{pmatrix} x_1 \\ y_1 \end{pmatrix}
= \begin{pmatrix} 0.2 & 0.5 \\ 0.8 & 0.5 \end{pmatrix}
\begin{pmatrix} 300 \\ 700 \end{pmatrix}
= \begin{pmatrix} 410 \\ 590 \end{pmatrix} \ ,
\]
$410$ lots seront choisis après un an.

\subQ{c} Puisque
\[
\begin{pmatrix} x_2 \\ y_2 \end{pmatrix}
= \begin{pmatrix} 0.2 & 0.5 \\ 0.8 & 0.5 \end{pmatrix}
\begin{pmatrix} x_1 \\ x_1 \end{pmatrix}
= \begin{pmatrix} 377 \\ 623 \end{pmatrix} \ ,
\]
$377$ lots seront choisis après deux ans.

\subQ{d}
Il faut trouver un vecteur propre
$\displaystyle \VEC{v} = \begin{pmatrix} v_1 \\ v_2 \end{pmatrix}$
de la matrice $A$ associé à la valeur propre $1$ de $A$, et tel que
$v_1 + v_2 = 1000$ lots.

Pour $\lambda = 1$, les vecteurs propres $\VEC{v}$ sont les solutions de
$(A-\lambda \Id)\VEC{v} = (A- \Id)\VEC{v} = \VEC{0}$ où
\[
A - \Id = \begin{pmatrix} -0.8 & 0.5 \\ 0.8 & -0.5 \end{pmatrix} \ .
\]
La matrice augmentée est
\[
\left(\begin{array}{cc|c}
-0.8 & 0.5 & 0 \\ 0.8 & -0.5 & 0
\end{array}\right) \ .
\]
$R_2 + R_1 \leftrightarrow R_2$ donne
\[
\left(\begin{array}{cc|c}
-0.8 & 0.5 & 0 \\0 & 0 & 0
\end{array}\right) \ .
\]
$(-5/4)R_1 \rightarrow R_1$ donne
\[
\left(\begin{array}{cc|c}
1 & -5/8 & 0 \\0 & 0 & 0
\end{array}\right) \ .
\]
Nous obtenons $v_1 = 5v_2/8$.  Il découle de
$1000 = v_1+v_2 = (5v_2/8)+v_2 = 13v_2/8$ que
$v_2 = 8000/13 \approx 615.38$ et donc
$v_1 = 5v_2/8 = 5000/13\approx 384.62$.
Environ $385$ lots seront donc choisis.
}

\compileSOL{\SOLUb}{\ref{12Q15}}{
Soit $x_j$ le pourcentage de femelles du group I à la $j^e$
génération, $y_j$ le pourcentage de femelles du group II à la $j^e$
génération, et $z_j$ le pourcentage de femelles du group III à la
$j^e$ génération.  la chaîne de Markov est décrite par
\[
\begin{pmatrix} x_{j+1} \\ y_{j+1} \\ z_{i+1} \end{pmatrix} =
\begin{pmatrix} 0.5 & 0.25 & 0.3 \\ 0.45 & 0.5 & 0.3 \\
0.05 & 0.25 & 0.4 \end{pmatrix}
\begin{pmatrix} x_j \\ y_j \\ z_j\end{pmatrix}
\]
pour $j=0$, $1$, $2$,\ldots\ D'après la
proposition~\ref{alglinMarkoV}, quelle que soit la condition initiale
$\displaystyle \begin{pmatrix} x_0 \\ y_0 \\ z_0 \end{pmatrix}$, le
vecteur
$\displaystyle \begin{pmatrix} x_j \\ y_j \\ z_j \end{pmatrix}$ tend vers
un vecteur propre associée à la valeur propre $1$ lorsque $j$ tend vers
l'infini.  Trouvons ce vecteur propre.

Pour trouver les vecteurs propres de la matrice
\[
A = \begin{pmatrix}
0.5 & 0.25 & 0.3 \\
0.45 & 0.5 & 0.3 \\
0.05 & 0.25 & 0.4 \end{pmatrix}
\]
associé à la valeur propre $\lambda_1=1$, il faut résoudre le système
d'équations linéaires $(A-\lambda_1 \Id) \VEC{v} = \VEC{0}$;
c'est-à-dire,
\[
(A- \Id) \VEC{v}
= \begin{pmatrix}
-0.5 & 0.25 & 0.3 \\
0.45 & -0.5 & 0.3 \\
0.05 & 0.25 & -0.6
\end{pmatrix}
\begin{pmatrix} v_1 \\ v_2 \\ v_3  \end{pmatrix} =
\begin{pmatrix} 0 \\ 0 \\ 0 \end{pmatrix}\; .
\]
La matrice augmentée de ce système est
\[
\left(\begin{array}{rrr|r}
-0.5 & 0.25 & 0.3 & 0 \\
0.45 & -0.5 & 0.3 & 0 \\
0.05 & 0.25 & -0.6 & 0
\end{array}\right) \; .
\]
$R_2 + 0.9 R_2 \to R_1$ et $R_3 + 0.1 R_1 \to R_3$ donnent
\[
\left(\begin{array}{rrr|r}
-0.5 & 0.25 & 0.3 & 0 \\
0 & -0.275 & 0.57 & 0 \\
0 & 0.275 & -0.57 & 0
\end{array}\right) \; .
\]
$R_3 + R_2 \to R_3$ donne
\[
\left(\begin{array}{rrr|r}
-0.5 & 0.25 & 0.3 & 0 \\
0 & -0.275 & 0.57 & 0 \\
0 & 0 & 0 & 0
\end{array}\right) \; .
\]
$-2R_1 \to R_1$ et $(-40/11) R_2 \to R_2$ donnent
\[
\left(\begin{array}{rrr|r}
1 & -1/2 & -3/5 & 0 \\
0 & 1 & -114/55 & 0 \\
0 & 0 & 0 & 0
\end{array}\right) \; .
\]
$R_1 + (1/2)R_2 \to R_1$ donne
\[
\left(\begin{array}{rrr|r}
1 & 0 & -18/11 & 0 \\
0 & 1 & -114/55 & 0 \\
0 & 0 & 0 & 0
\end{array}\right) \; .
\]
Les solutions $\VEC{v}$ du système $(A-\Id) \VEC{v} = \VEC{0}$ satisfont
donc $v_1 - (18/11) v_3 = 0$ et $v_2 - (114/55) v_3 = 0$;
c'est-à-dire, $v_1=(18/11)v_3$ et $v_2 = (114/55) v_3$.

Puisque les $v_j$'s représentent des pourcentages de la population totale de
femelles, nous avons
\[
1 = v_1 + v_2 + v_3 = \frac{18}{11} v_3 + \frac{114}{55} v_3 + v_3 =
\frac{259}{55} v_3 \ .
\]
Donc $\displaystyle v_3 = \frac{55}{259} = 0.212355\ldots$,
$\displaystyle v_2 = \frac{114}{55}\; v_3 = \frac{114}{259} = 0.44015\ldots$
et
$\displaystyle v_1 = \frac{18}{11}\; v_3 = \frac{90}{259} = 0.34749\ldots$
La tendance à long terme est donc qu'approximativement $34.75$\% des
femelles appartiennent au groupe I à la naissance, $44.02$\%
appartiennent au groupe II et $21.24$\% appartiennent au groupe III.
}

\compileSOL{\SOLUa}{\ref{12Q16}}{
Soit $x_j$ la probabilité que les pommes Spartan soient en vente à la
$j^e$ journée, $y_j$ la probabilité que les pommes Cortland soient en
vente à la $j^e$ journée, et $z_j$ la probabilité que les pommes
McIntosh soient en vente à la $j^e$ journée.  La chaîne de
Markov est décrite par
\[
\begin{pmatrix} x_{j+1} \\ y_{j+1} \\ z_{i+1} \end{pmatrix} =
A \begin{pmatrix} x_j \\ y_j \\ z_j\end{pmatrix}
\]
pour $i=0$, $1$, $2$, \ldots\ où
\[
A = \begin{pmatrix}
1/2 & 1/6 & 1/2 \\
1/6 & 1/2 & 1/2 \\
1/3 & 1/3 & 0
\end{pmatrix} \ .
\]
$a_{1,3}$, $a_{2,3}$ et $a_{3,3}$ proviennent des énoncés (1) et
(2) de la question, $a_{1,1}$ et $a_{2,2}$ proviennent
de l'énoncé (3),  $a_{3,1}$ et $a_{3,2}$ proviennent de l'énoncé (4).
Les autres éléments de $A$ proviennent du fait que la somme sur chaque
colonne doit être $1$; il y a 100\% de chance qu'un item soit en vente
chaque jour.

La probabilité qu'un item soit en vente après $100$ jours est donnée par
\[
\begin{pmatrix} x_{100} \\ y_{100} \\ z_{100} \end{pmatrix} =
A^{100} \begin{pmatrix} 1/3 \\ 1/3 \\ 1/3 \end{pmatrix} \ .
\]

D'après la Proposition~\ref{alglinMarkoV}, quelle que soit la
condition initiale
$\displaystyle \begin{pmatrix} x_0 \\ y_0 \\ z_0 \end{pmatrix}$, nous avons que
$\displaystyle \begin{pmatrix} x_j \\ y_j \\ z_j \end{pmatrix}$ tend vers
un vecteur propre associé à la valeur propre $1$ lorsque $j$ tend vers
l'infini.  Trouvons ce vecteur propre.

Pour trouver les vecteurs propres de la matrice $A$ associé à la valeur
propre $\lambda_1=1$, il faut résoudre le système d'équations linéaires
$(A-\lambda_1 \Id) \VEC{v} = \VEC{0}$; c'est-à-dire,
\[
(A-\Id)\VEC{v}
= \begin{pmatrix}
-1/2 & 1/6 & 1/2 \\
1/6 & -1/2 & 1/2 \\
1/3 & 1/3 & -1
\end{pmatrix}
\begin{pmatrix} v_1 \\ v_2 \\ v_3  \end{pmatrix} =
\begin{pmatrix} 0 \\ 0 \\ 0 \end{pmatrix}\ .
\]
La matrice augmentée de ce système est
\[
\left(\begin{array}{rrr|r}
-1/2 & 1/6 & 1/2 & 0 \\
1/6 & -1/2 & 1/2 & 0 \\
1/3 & 1/3 & -1 & 0
\end{array}\right) \ .
\]
$-6R_1 \to R_1$, $6R_2 \to R_2$ et $3R_3 \to R_3$ donnent
\[
\left(\begin{array}{rrr|r}
3 & -1 & -3 & 0 \\
1 & -3 & 3 & 0 \\
1 & 1 & -3 & 0
\end{array}\right) \ .
\]
$R_1 - 3R_3 \to R_1$ et $R_2-R_3 \to R_2$ donnent
\[
\left(\begin{array}{rrr|r}
0 & -4 & 6 & 0 \\
0 & -4 & 6 & 0 \\
1 & 1 & -3 & 0
\end{array}\right) \ .
\]
$R_1 - R_2 \to R_1$ suivie de $(-1/4)R_2 \to R_2$ donnent
\[
\left(\begin{array}{rrr|r}
0 & 0 & 0 & 0 \\
0 & 1 & -3/2 & 0 \\
1 & 1 & -3 & 0
\end{array}\right) \ .
\]
$R_3 - R_2 \to R_3$ donne
\[
\left(\begin{array}{rrr|r}
0 & 0 & 0 & 0 \\
0 & 1 & -3/2 & 0 \\
1 & 0 & -3/2 & 0
\end{array}\right) \ .
\]
Les solutions $\VEC{v}$ du système $(A-\Id) \VEC{v} = \VEC{0}$ satisfont
donc $v_1 - (3/2) v_3 = 0$ et $v_2 - (3/2) v_3 = 0$;
c'est-à-dire, $v_1=(3/2)v_3$ et $v_2 = (3/2) v_3$.

Puisque les $v_j$'s représentent les probabilités d'être en vente pour
chaque item et qu'il est certain à $100$\% qu'un item est en vente à
chaque jour, nous avons
\[
1 = v_1 + v_2 + v_3 = \frac{3}{2} v_3 + \frac{3}{2} v_3 + v_3 = 4 v_3 \ .
\]
Donc $\displaystyle v_3 = \frac{1}{4} = 0.25$,
$\displaystyle v_2 = \frac{3}{2}\; v_3 = \frac{3}{8} = 0.375$ et
$\displaystyle v_1 = \frac{3}{2}\; v_3 = \frac{3}{8} = 0.375$.
Nous pouvons conclure que dans un future éloigné, il y aura $37.5$\%
des chances que les pommes Spartan soient en vente, $37.5$\% des
chances que les pommes Cortland soient en vente et $25$\% des chances
que les pommes McIntosh soient en vente.
}

%%% Local Variables: 
%%% mode: latex
%%% TeX-master: "notes"
%%% End: 

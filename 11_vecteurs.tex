\chapter[Vecteurs]{Vecteurs}\label{CHAPvecteurs}

\compileTHEO{

Dans ce chapitre, nous revoyons rapidement quelques notions de
géométrie analytique que le lecteur devrait avoir étudié à l'école
secondaire.  Une grande partie des concepts présentés dans ce chapitre
seront utilisés lors de l'étude des fonctions de plusieurs variables
à partir du chapitre~\ref{chapFonctPlusVar}.

\section{Définition}

\begin{defn} \index{Vecteur}
Un {\bfseries vecteur} du plan $\RR^2$ est une flèche qui part de
l'origine $\VEC{0} = (0,0)$ pour se terminer en un point
$\VEC{p} = (p_1,p_2)$.   Puisque chaque vecteur est associé à un
unique point $\VEC{p}$ et vice-versa, le vecteur est dénoté par
$\VEC{p}$

De même, un {\bfseries vecteur} de l'espace $\RR^3$ est une flèche qui
part de l'origine $\VEC{0} = (0,0,0)$ pour se terminer en un point
$\VEC{p} = (p_1,p_2,p_3)$.   Puisque chaque vecteur est associé à un
unique point $\VEC{p}$ et vice-versa, le vecteur est dénoté par
$\VEC{p}$
\end{defn}

La définition de vecteur pourrait être généralisée à l'espace $\RR^n$
pour $n>3$.  Cela n'est cependant pas nécessaire pour le chapitre
présent.  Nous aborderons les espace de dimensions plus grande que $3$
au chapitre~\ref{chapALLin} sur l'algèbre linéaire.

Puisqu'il y a une bijection entre les vecteurs et les points, le
contexte déterminera comment il faut interpréter $\VEC{p}$.

\begin{egg}
Nous retrouvons à la figure~\ref{VECT1} la représentation graphique du
vecteur $\VEC{p} = (2,5,3)$.
\label{vect1}
\end{egg}

\PDFfig{11_vecteurs/vect1}{Représentation graphique du vecteur
$\VEC{p} = (2,5,3)$ de l'espace}{Représentation graphique du vecteur
$\VEC{p} = (2,5,3)$ de l'espace}{VECT1}

\begin{defn} \index{Vecteur!longueur} \label{norm} 
La {\bfseries longueur (euclidienne)} du vecteur $\VEC{p} = (p_1,p_2)$ est
\[
\|\VEC{p}\| \equiv \sqrt{p_1^2+p_2^2} \ ,
\]
alors que celle du vecteur $\VEC{p} = (p_1,p_2,p_3)$ est
\[
\|\VEC{p}\| \equiv \sqrt{p_1^2+p_2^2+p_3^2} \;.
\]
\end{defn}

Cette définition découle du théorème de Pythagore qui affirme que le
carré de l'hypoténuse d'un triangle droit est égal à la somme des
carrés des deux côtés adjacents à l'angle droit. La justification de
la formule pour la longueur d'un vecteur dans $\RR^3$ est illustrée à la
figure~\ref{VECT2}.

\PDFfig{11_vecteurs/vect2}{Longueur du vecteur $\VEC{p}=(p_1,p_2,p_3)$}
{Longueur du vecteur $\VEC{p}=(p_1,p_2,p_3)$}{VECT2}

\section{Opérations sur les vecteurs}

Nous pouvons définir des opérations algébriques sur les vecteurs comme ils
en existent pour les nombres.  De plus, nous pouvons aussi définir des
opérations sur les vecteurs qui ont une importante signification
géométrique comme nous verrons dans les prochaines sections.

Nous commençons par définir deux opérations algébriques sur les vecteurs:
{\em le produit d'un vecteur par un nombre réel} et
{\em la somme de vecteurs}.

Par la suite, nous définirons deux opérations sur les vecteurs,
{\em le produit scalaire} et {\em le produit vectoriel}, qui jouent
des rôles majeurs dans l'analyse géométrique.

\subsection{Produit d'un vecteur par un nombre réel}

\begin{defn} \index{Vecteur!produit avec un nombre réel}
Soit $\VEC{p} = (p_1,p_2)$ et $\rho$ est un nombre réel.  Le
{\bfseries produit} de $\rho$ avec le vecteur $\VEC{p}$ est le
vecteur $\VEC{q}$ défini par la formule suivante.
\[
\VEC{q} = \rho\, \VEC{p} = \rho (p_1, p_2) = (\rho p_1 , \rho p_2) \; .
\]

De même, soit $\VEC{p} = (p_1,p_2,p_3)$ et $\rho$ est un nombre réel.
Le {\bfseries produit} de $\rho$ avec le vecteur $\VEC{p}$ est le vecteur
$\VEC{q}$ défini par la formule suivante.
\[
\VEC{q} = \rho\, \VEC{p} = \rho (p_1,p_2,p_3)
= (\rho p_1 , \rho p_2, \rho p_3) \; .
\]
\end{defn}

Le vecteur $\VEC{q}$ est de longueur $|\rho| \|\VEC{p}\|$ et pointe dans la
même direction que $\VEC{p}$ si $\rho > 0$ et dans la direction opposée à
$\VEC{p}$ si $\rho < 0$.

\begin{egg}
Le dessin du vecteur $\displaystyle \VEC{q} = -2 \VEC{p}$, où
$\VEC{p}$ est le vecteur donné à l'exemple~\ref{vect1}, est donné à la
figure~\ref{VECT3}.  Nous avons
\[
\VEC{q} = -2 \VEC{p} = -2 (2, 5, 3) = (-4,-10,-6) \; .
\]
\end{egg}

\PDFfig{11_vecteurs/vect3}{Représentation du vecteur $\VEC{q} = -2\VEC{p}$}
{Représentation du vecteur $\VEC{q} = -2\VEC{p}$ où le vecteur
$\VEC{p}=(2,5,3)$ est représenté à la figure~\ref{VECT1}}{VECT3}

\subsection{Somme de vecteurs}

\begin{defn} \index{Vecteur!somme de vecteurs}
Soit $\VEC{p} = (p_1,p_2)$ et $\VEC{q} = (q_1, q_2)$.  La
{\bfseries somme} des vecteurs $\VEC{p}$ et $\VEC{q}$ est le
vecteur $\VEC{r}$ défini par la formule suivante.
\[
\VEC{r} = \VEC{p} + \VEC{q} = ( p_1, p_2) + (q_1, q_2)
= (p_1+q_1, p_2+q_2) \; .
\]

De même, soit $\VEC{p} = (p_1,p_2,p_3)$ et $\VEC{q} = (q_1, q_2, q_3)$.
La {\bfseries somme} des deux vecteurs $\VEC{p}$ et
$\VEC{q}$ est le vecteur $\VEC{r}$ défini par la formule suivante.
\[
\VEC{r} = \VEC{p} + \VEC{q} = ( p_1, p_2, p_3) + (q_1, q_2, q_3)
= (p_1+q_1, p_2+q_2, p_3+q_3) \; .
\]
\end{defn}

\begin{egg}
Le résultat $\VEC{r} = (5,9,4)$ de la somme des vecteurs $\VEC{p} = (6,6,2)$
et $\VEC{q} = (-1, 3, 2)$ est illustré à la figure~\ref{VECT4}.

Remarquons que les trois vecteurs $\VEC{p}$, $\VEC{q}$ et $\VEC{r}$ sont
tous dans le même plan.  Nous verrons à la section~\ref{PROD_VECT} sur
le produit vectoriel comment nous pouvons facilement vérifier cette
affirmation.
\label{EX_SUM}
\end{egg}

\PDFfigD[t]{11_vecteurs/vect4}{11_vecteurs/vect5}{Représentation
graphique de la somme de deux vecteurs}{Ces dessins illustrent
l'aspect géométrique de la somme des deux vecteurs donnés à
l'exemple~\ref{EX_SUM}}{VECT4}

\subsection{Notation vectorielle classique}

Les vecteurs suivants sont fondamental en analyse vectorielle.  Pour
cette raison, ils ont leur propre notation.

\begin{defn} \index{Vecteur!canonique}
Dans le plan, les vecteurs $\ii = (1,0)$ et $\jj = (0.1)$ sont appelés
les {\bfseries vecteurs canoniques}.  Si
$\VEC{p}=(p_1,p_2)$, alors $\displaystyle \VEC{p} = p_1 \ii + p_2 \jj$.

De même, dans l'espace $\RR^3$, les {\bfseries vecteurs canoniques}
sont les vecteurs $\ii = (1,0,0)$, $\jj = (0.1,0)$ et $\kk = (0,0,1)$.  Si
$\VEC{p}=(p_1,p_2,p_3)$, alors
$\displaystyle \VEC{p} = p_1 \ii + p_2 \jj + p_3 \kk$.
\end{defn}

Dans les livres de mathématiques, la notation
$\VEC{e}_1$ pour $\ii$, $\VEC{e}_2$ pour $\jj$ et $\VEC{e}_3$ pour
$\kk$ est plus souvent utilisée.

\subsection{Produit scalaire \eng}

Alors que le produit d'un vecteur par un nombre réel et l'addition de
vecteurs ont un rôle principalement algébrique, le produit scalaire est l'une
des deux opérations sur les vecteurs qui a un usage principalement
géométrique; l'autre opération étant le produit vectoriel.

\begin{defn} \index{Vecteur!produit scalaire}
Le {\bfseries produit scalaire} de deux vecteurs,
$\VEC{p} = (p_1,p_2)$ et $\VEC{q} = (q_1,q_2)$, est défini par la
formule
\[
\ps{\VEC{p}}{\VEC{q}} = p_1 q_1 + p_2 q_2  \; .
\]

De même, Le {\bfseries produit scalaire} de deux vecteurs,
$\VEC{p} = (p_1,p_2,p_3)$ et $\VEC{q} = (q_1,q_2,q_3)$, est défini
par la formule
\[
\ps{\VEC{p}}{\VEC{q}} = p_1 q_1 + p_2 q_2 + p_3 q_3 \; .
\]
Le produit scalaire des vecteurs $\VEC{p}$ et $\VEC{q}$ est aussi
dénoté $\VEC{p} \cdot \VEC{q}$ quand cela ne cause pas de confusion.
\end{defn}

Pourquoi définir un produit de deux vecteurs comme nous venons de le faire?
N'aurions-nous pas pu définir le produit de deux vecteurs composante par
composante comme nous l'avons fait pour la somme de deux vecteurs?  En
fait, nous aurions pu le faire mais cela n'a pas d'intérêt car il n'y a pas
d'interprétations géométriques du produit de deux vecteurs composante par
composante.  Par contre, le produit scalaire défini ci-dessus a une
interprétation géométrique très importante.

\begin{defn} \label{dfncos}
Le cosinus de l'angle entre les vecteurs non nuls $\VEC{p}$ et
$\VEC{q}$ (et ainsi l'angle entre ces vecteurs) est déterminé par
\begin{equation}\label{CosVectors}
\cos(\alpha) = \frac{\ps{\VEC{p}}{\VEC{q}}}{\|\VEC{p}\|\,\|\VEC{q}\|} \; .
\end{equation}
\end{defn}

C'est cette définition qui fait le lien entre l'opération algébrique
qu'est le produit scalaire de deux vecteurs et l'interprétation
géométrique que nous avons des vecteurs.

\begin{rmk}
Remarquons que la longueur d'un vecteur est la racine carrée du produit
scalaire de ce vecteur par lui-même.  Si $\VEC{p} = (p_1,p_2,p_3)$, alors
\[
\ps{\VEC{p}}{\VEC{p}} = p_1^2 + p_2^2 + p_3^2 = \|\VEC{p}\|^2 \; .
\]

Si $\displaystyle \ps{\VEC{p}}{\VEC{q}} = 0$, alors le cosinus entre
les deux vecteurs $\VEC{p}$ et $\VEC{q}$ est $0$.  Nous avons donc que
l'angle entre $\VEC{p}$ et $\VEC{q}$ est $\pi/2$ ( $+n\pi$ pour
$n \in \ZZ$).  Ceci justifie la définition suivante.
\end{rmk}

\begin{defn}
Soit $\VEC{p}$ et $\VEC{q}$, deux vecteurs.  Le vecteur $\VEC{p}$ est
{\bfseries perpendiculaire}\index{Vecteur!perpendiculaire} ou
{\bfseries orthogonal}\index{Vecteur!orthogonal} au vecteur
$\VEC{q}$ (et vice-versa) si $\displaystyle \ps{\VEC{p}}{\VEC{q}} = 0$.
\end{defn}

\begin{rmk}
Vous avez probablement appris que le produit des pentes de deux droites
orthogonales est $-1$.  Il est facile à l'aide des vecteurs de
vérifier cette énoncé.  Supposons que $\VEC{p} = (p_1,p_2)$ soit un
vecteur parallèle à une droite $\ell_1$ et que $\VEC{q} = (q_1,q_2)$
soit un vecteur parallèle à une droite $\ell_2$.  Si les droites
$\ell_1$ et $\ell_2$ sont perpendiculaires, alors il en est de même
pour les vecteurs $\VEC{p}$ et $\VEC{q}$, et vice-versa.  Nous avons donc
\[
\ps{\VEC{p}}{\VEC{q}} = p_1 q_1 + p_2 q_2 = 0 \; .
\]
Ainsi,
\[
\frac{p_2}{p_1} = - \frac{q_1}{q_2} \; .
\]
Puisque $\displaystyle m_1 = \frac{p_2}{p_1}$ est la pente de la droite
$\ell_1$ et $\displaystyle m_2 = \frac{q_2}{q_1}$ est celle de la droite
$\ell_2$, nous obtenons bien que $\displaystyle m_1 = -\frac{1}{m_2}$.
\end{rmk}

\begin{rmk}
Nous justifierons ci-dessous la définition~\ref{dfncos} dans le plan.  La
justification est identique pour les vecteurs dans l'espace.

Commençons par un petit rappel.  Considérons le triangle $\triangle ABC$
représenté à la figure~\ref{VECT6}.  La {\bfseries loi des cosinus} dit que
\[
c^2 = a^2 + b^2 - 2 a b\, \cos(\gamma)  \; .
\]

\PDFfig{11_vecteurs/vect6}{La loi des cosinus}{La loi des cosinus dit que
$c^2 = a^2 + b^2 - 2 a b \cos(\gamma)$}{VECT6}

Soit $\VEC{p} = (p_1,p_2)$ et $\VEC{q} = (q_1,q_2)$.
De la loi des cosinus, nous déduisons que
\begin{equation} \label{ps_interpr}
\| \VEC{q} - \VEC{p} \|^2 = \|\VEC{p}\|^2 + \|\VEC{q}\|^2
- 2 \|\VEC{p}\| \|\VEC{q}\| \cos(\alpha)
\end{equation}
où $\alpha$ est l'angle entre $\VEC{p}$ et $\VEC{q}$ (figure~\ref{VECT7}).

\PDFfig{11_vecteurs/vect7}{Interprétation géométrique du produit scalaire}
{Interprétation géométrique du produit scalaire}{VECT7}

Puisque $\displaystyle \VEC{q} - \VEC{p} = (q_1-p_1,q_2-p_2)$,
nous avons que
\begin{align*}
\| \VEC{q} - \VEC{p} \|^2
&= (q_1 - p_1)^2 + (q_2-p_2)^2  \\
&= \left(p_1^2 + p_2^2\right) + \left(q_1^2 + q_2^2\right)
- 2 ( p_1 q_1 + p_2 q_2) \\
&= \| \VEC{p} \|^2 + \| \VEC{q} \|^2 - 2 \ps{\VEC{p}}{\VEC{q}} \; .
\end{align*}
Ainsi, (\ref{ps_interpr}) devient
\[
\| \VEC{p} \|^2 + \| \VEC{q} \|^2 - 2 \ps{\VEC{p}}{\VEC{q}}
= \|\VEC{p}\|^2 + \|\VEC{q}\|^2 - 2 \|\VEC{p}\| \|\VEC{q}\|
\cos(\alpha) \; .
\]
Après avoir simplifié les termes identiques de chaque côté de l'égalité,
nous obtenons
\[
\ps{\VEC{p}}{\VEC{q}} = \|\VEC{p}\|\, \|\VEC{q}\| \cos(\alpha) \; .
\]
Si nous isolons $\cos(\alpha)$, nous retrouvons la définition~\ref{dfncos}.
\end{rmk}

\begin{rmk}[\theory]
Il est facile de démontrer la loi des cosinus en utilisant les
coordonnées cartésiennes.

Pour démontrer la loi des cosinus pour l'angle au sommet $C$ d'un
triangle $\triangle ABC$. nous plaçons le sommet $C$ à l'origine et le
sommet $A$ (ou $B$) sur l'axe des $x_1$ comme cela est fait à la
figure~\ref{VECT16}.

\PDFfig{11_vecteurs/vect16}{Démonstration de la loi des cosinus}{Démonstration
de la loi des cosinus}{VECT16} 

Les coordonnées du point $B$ sont donc
$(a\cos(\gamma), a \sin(\gamma))$ et celle du point $A$ sont $(b,0)$.
Le triangle $\triangle DAB$ a un angle droite au sommet $D$.  Le
théorème de Pythagore nous donne donc
\[
c^2 = \left( a\sin(\gamma) \right)^2 + \left( -a\cos(\gamma) + b\right)^2 \; .
\]
Si nous développons le côté droit de cette équation, nous trouvons
\begin{align*}
c^2 &= a^2 \sin^2(\gamma) + a^2 \cos^2(\gamma) - 2 ab\, \cos(\gamma) +b^2 \\
&= a^2 \left(\cos^2(\gamma)+\sin^2(\gamma)\right) + b^2 - 2ab\,\cos(\gamma) \\
&= a^2 + b^2 - 2ab\,\cos(\gamma)
\end{align*}
puisque $\cos^2(\gamma) + \sin^2(\gamma) = 1$.
\end{rmk}

\subsection{Produit vectoriel \eng}\label{PROD_VECT}

Le produit vectoriel est une opération sur les vecteurs qui a un
usage principalement géométrique.

\begin{defn} \index{Vecteur!produit vectoriel}
Si $\VEC{p} = (p_1,p_2,p_3)$ et $\VEC{q} = (q_1, q_2, q_3)$ sont deux
vecteurs qui ne sont pas colinéaires (i.e.\ dont l'un n'est pas un
multiple de l'autre), nous définissons le
{\bfseries produit vectoriel} des vecteurs
$\VEC{p}$ et $\VEC{q}$ comme étant le vecteur
\begin{equation}\label{sp_form}
\VEC{m} = \VEC{p} \times \VEC{q} =
\left(p_2 q_3 - p_3 q_2\right)\ii + \left(p_3 q_1 - p_1 q_3\right) \jj
+ \left(p_1 q_2 - p_2 q_1\right) \kk \; .
\end{equation}
Donc $\VEC{m} = \left(p_2 q_3 - p_3 q_2, p_3 q_1 - p_1 q_3,
p_1 q_2 - p_2 q_1\right)$.
\label{VecTProdDef}
\end{defn}

Nous fournirons à la remarque~\ref{DetermVectProd} de la
section~\ref{DetermSect} une formule symbolique simple pour calculer
le produit vectoriel de deux vecteurs.

Le vecteur qui résulte du produit vectoriel de deux vecteurs a une relation
géométrique très particulière avec les deux vecteurs du produit.

\begin{prop}
Le vecteur $\VEC{m} = \VEC{p} \times \VEC{q}$ est
perpendiculaire aux vecteurs $\VEC{p}$ et $\VEC{q}$.
\end{prop}

Pour vérifier que $\VEC{p} \times \VEC{q}$ est perpendiculaire à
$\VEC{p}$, il suffit de vérifier que
$\ps{(\VEC{p} \times \VEC{q})}{\VEC{p}} = 0$.  En effet,
\begin{align*}
\ps{(\VEC{p} \times \VEC{q})}{\VEC{p}} &=
\ps{ \left(p_2 q_3 - p_3 q_2, p_3 q_1 - p_1 q_3,
p_1 q_2 - p_2 q_1\right)}{(p_1,p_2,p_3)} \\
&= p_1\left(p_2 q_3 - p_3 q_2\right) + p_2\left(p_3 q_1 - p_1 q_3\right)
+ p_3\left(p_1 q_2 - p_2 q_1\right) \\
&= p_1 p_2 q_3 - p_1 p_3 q_2 + p_2 p_3 q_1 - p_1 p_2 q_3
+ p_1 p_3 q_2 - p_2 p_3 q_1 \\
&= 0 \; .
\end{align*}
De même, nous pouvons vérifier que
$\ps{(\VEC{p} \times \VEC{q})}{\VEC{q}} = 0$.

\begin{defn} \index{Règle de la main droite}
La direction du vecteur $\VEC{p} \times \VEC{q}$ est déterminée par la
{\bfseries règle de la main droite}.  C'est-à-dire, si la paume de
votre main droite entoure la droite qui contient le vecteur
$\VEC{p} \times \VEC{q}$ de telle sorte que vos doigts indiquent la
plus petite direction angulaire de $\VEC{p}$ à $\VEC{q}$, alors votre
pouce pointe dans la direction de $\VEC{p} \times \VEC{q}$
(figure~\ref{VECT10}).
\end{defn}

\PDFfig{11_vecteurs/vect10}{Le vecteur obtenu du produit vectoriel de deux
vecteurs est orthogonal au plan généré par ces deux vecteurs}{Le vecteur
obtenu du produit vectoriel de deux vecteurs est orthogonal au plan généré
par ces deux vecteurs.}{VECT10}

\begin{egg}
Nous pouvons facilement vérifier avec la règle de la main droite que
$\kk = \ii \times \jj$ et $\jj \times \ii = -\kk$.
\end{egg}

Quelques manipulations algébriques vont nous permettre de trouver une
formule pour la longueur du vecteur $\VEC{p} \times \VEC{q}$.
Soit $\theta$, le plus petit angle entre $\VEC{p}$ et $\VEC{q}$
(donc $0 < \theta < \pi$).  Nous avons que
\[
\ps{\VEC{p}}{\VEC{q}} = \|\VEC{p}\| \|\VEC{q}\| \cos(\theta)  \; .
\]
Ainsi,
\begin{align*}
\| \VEC{p} \times \VEC{q} \|^2 &= 
\left(p_2 q_3 - p_3 q_2\right)^2 + \left(p_3 q_1 - p_1 q_3\right)^2
+ \left(p_1 q_2 - p_2 q_1\right)^2 \\
&= \left(p_1^2+p_2^2+p_3^2\right)\left(q_1^2+q_2^2+q_3^2\right)
-\left(p_1q_1+p_2q_2+p_3q_3\right)^2 \\
&= \|\VEC{p}\|^2 \|\VEC{q}\|^2
- \left( \ps{\VEC{p}}{\VEC{q}} \right)^2
= \|\VEC{p}\|^2 \|\VEC{q}\|^2
- \|\VEC{p}\|^2 \|\VEC{q}\|^2\cos^2(\theta) \\
&= \|\VEC{p}\|^2 \|\VEC{q}\|^2 \left( 1 - \cos^2(\theta)\right)
= \|\VEC{p}\|^2 \|\VEC{q}\|^2 \sin^2(\theta) \; .
\end{align*}

Nous obtenons le résultat suivant.

\begin{prop}
\[
\| \VEC{p} \times \VEC{q} \| = 
= \|\VEC{p}\| \|\VEC{q}\| \sin(\theta)
\]
où $0 \leq \theta < \pi$ est l'angle entre $\VEC{p}$ et $\VEC{q}$.
\end{prop}

\begin{egg}
Nous avons que le plus petit angle entre les vecteurs $\ii$ et $\jj$ est
$\displaystyle \frac{\pi}{2}$.  Ainsi,
\[
\|\kk \| = \| \ii \times \jj \| = \|\ii \| \| \jj \|
\sin \left(\frac{\pi}{2}\right) = 1
\]
comme il se doit.
\end{egg}

Il y a une interprétation géométrique très intéressante de la longueur du
vecteur obtenu d'un produit vectoriel.  Comme nous pouvons constater à
partir du dessin à la figure~\ref{VECT11},
$\|\VEC{p}\| \|\VEC{q}\| \sin(\theta)$ est
l'aire du parallélogramme définie par $\VEC{p}$ et $\VEC{q}$.  La
longueur du vecteur $\VEC{p}\times \VEC{q}$ est donc l'aire de ce
parallélogramme.

\PDFfig{11_vecteurs/vect11}{L'aire du parallélogramme définie par deux vecteurs}
{L'aire du parallélogramme définie par les deux vecteurs $\VEC{p}$ et
$\VEC{q}$ est
$\|\VEC{p}\times \VEC{q}\| = \|\VEC{p}\| \|\VEC{q}\| \sin(\theta)$.}{VECT11}

\section{Équation d'une droite \eng}

Nous pouvons utiliser les vecteurs pour décrire les droites.

La droite qui passe par le point $\VEC{p}$ et qui est parallèle au
vecteur $\VEC{q}$ (figure~\ref{VECT8}) est formé des points
$\VEC{x}$ où 
\begin{equation}\label{PVline}
\VEC{x} = \VEC{p} + \alpha \VEC{q}
\end{equation}
et $\alpha$ prend toutes les valeurs réelles.

Rappelons que nous utilisons la même notation pour désigner un point ou
un vecteur.  Nous pouvons donc faire référence au point $\VEC{x}$ et à ses
coordonnées à un certain moment et traiter $\VEC{x}$ comme un vecteur
à d'autres moments.

\begin{defn} \index{Représentation vectorielle d'une droite}
La formule (\ref{PVline}) est une {\bfseries représentation
vectorielle} de la droite qui passe par $\VEC{p}$ et est parallèle au
vecteur $\VEC{q}$.
\end{defn}

\PDFfig{11_vecteurs/vect8}{Représentation vectorielle d'une droite}
{Représentation vectorielle d'une droite}{VECT8}

Si $\VEC{p}=(p_1,p_2,p_3)$, $\VEC{q} = (q_1, q_2, q_3)$ et
$\VEC{x}=(x_1,x_2,x_3)$,  les composantes de la représentation vectorielle
$\VEC{x} = \VEC{p} + \alpha \VEC{q}$ sont
\begin{equation}\label{RPline}
x_1 = p_1 + \alpha\; q_1 \ , \ x_2 = p_2 + \alpha\; q_2 \quad \text{et}
\quad x_3 = p_3 + \alpha\; q_3 \; .
\end{equation}

\begin{defn} \index{Représentation paramétrique d'une droi\-te}
la formule (\ref{RPline}) est une {\bfseries représentation
paramétrique} de la droite qui passe par $\VEC{p}$ et est parallèle au
vecteur $\VEC{q}$.  Le {\bfseries paramètre} est $\alpha$.
\end{defn}

Si nous supposons que $q_1$, $q_2$ et $q_3$ sont non nuls et que nous
résolvons chacune des trois équations de la représentation paramétrique
pour $\alpha$, nous obtenons 
\begin{equation}\label{RSline1}
\frac{x_1-p_1}{q_1} = \frac{x_2-p_2}{q_2} = \frac{x_3-p_3}{q_3} \; .
\end{equation}
Si $q_2$ et $q_3$ sont non nuls mais $q_1 = 0$, nous obtenons
\begin{equation}\label{RSline2}
\frac{x_2-p_2}{q_2} = \frac{x_3-p_3}{q_3} \quad \text{et} \quad x_1 = p_1 \; .
\end{equation}
C'est une droite dans le plan $x=p_1$.  Le lecteur est invité à
analyser les autres cas possibles.

\begin{defn} \index{Représentation standard d'une droite}
Les formules (\ref{RSline1}) et (\ref{RSline2}) sont des
{\bfseries représentations standard}
d'une droite qui passe par $\VEC{p}$ et est parallèle au vecteur
$\VEC{q}$.
\end{defn}

\begin{rmk}
Nous pouvons déduire à partir du dessin à la figure~\ref{VECT15} que,
dans le plan, les points $\VEC{x}$ de la droite qui passe par le point
$\VEC{p}$ et qui est perpendiculaire au vecteur $\VEC{q}$ sont donnés par
\[
\ps{\VEC{x} - \VEC{p}}{\VEC{q}} = 0 \; .
\]
\end{rmk}

\PDFfig{11_vecteurs/vect15}{Équation d'une droite du plan qui est
perpendiculaire à un vecteur }{Les points $\VEC{x}$ d'une droite
perpendiculaire à un vecteur $\VEC{q}$ et passant par un point
$\VEC{p}$ satisfont $\ps{\VEC{x} - \VEC{p}}{\VEC{q}} = 0$.}{VECT15}

\begin{egg}
Si $\VEC{p}=(1,2,3)$ et $\VEC{q}=(-1,2,2)$, alors l'ensemble des points
$\VEC{x}=(x_1,x_2,x_3)$ de la droite passant par $\VEC{p}$ et
parallèle à $\VEC{q}$ est donnée par la relation
\[
(x_1,x_2,x_3) = \VEC{x} = \VEC{p} + \alpha \VEC{q} = (1,2,3) + \alpha (-1,2,2)
= (1-\alpha, 2 +2 \alpha, 3 + 2\alpha)
\]
pour $\alpha \in \RR$.  Nous avons donc la représentation paramétrique
\[
x_1 = 1-\alpha \ , \ x_2 = 2+2\alpha \quad \text{et} \quad x_3 = 3+2\alpha
\]
et la représentation standard
\[
- (x_1-1) = \frac{x_2-2}{2} = \frac{x_3-3}{2} \; .
\]
\end{egg}

\begin{egg}
La représentation classique d'une droite dans le plan est $x_2=m x_1+b$
où $m$ est la pente de la droite et $b$ est l'abscisse à l'origine.

L'exemple suivant illustre la relation entre les représentations
d'une droite que nous venons d'introduire et la représentation
classique de cette droite.

Soit $\VEC{p}=(2,3)$ et $\VEC{q} = (4,3)$.  Pour trouver l'équation
classique de la droite qui passe par le point $\VEC{p}$ et qui est
parallèle au vecteur $\VEC{q}$, il faut premièrement trouver la pente
de la droite qui contient le vecteur $\VEC{q}$.  La pente de
cette droite est $3/4$.  La forme {\bfseries point-pente} de la droite
cherchée est donc
\[
x_2-3 = \frac{3}{4}\, (x_1 - 2) \; .
\]
Nous obtenons la représentation classique
\[
x_2 = \frac{3}{4} x_1 + \frac{3}{2}
\]
de la droite.

L'ensemble des points $\VEC{x}=(x_1,x_2)$ de la droite qui passe par le
point $\VEC{p}$ et est parallèle au vecteur $\VEC{q}$ est donnée par
\[
(x_1,x_2) = \VEC{x} = \VEC{p} + \alpha \VEC{q}
= (2,3) + \alpha (4,3) = (2+ 4\alpha, 3 +3\alpha) \; .
\]
Une représentation paramétrique de la droite qui passe par le point
$\VEC{p}$ et est parallèle au vecteur $\VEC{q}$ est donc
$x_1 = 2 + 4 \alpha$ et $x_2 = 3 + 3 \alpha$.  De la première
équation, nous obtenons que $\alpha = (x_1-2)/4$.  Si
nous substituons cette expression pour $\alpha$ dans l'équation
$x_2 = 3 + 3 \alpha$, nous obtenons
\[
x_2 = 3 + 3 \left(\frac{x_1-2}{4}\right) = 3 + \frac{3}{4} (x_1-2)
= \frac{3}{4} x_1 + \frac{3}{2} \ ,
\]
la représentation classique de le droite.
\end{egg}

\subsection{Droites tangentes}

\begin{egg}
Trouvons l'équation de la droite tangent au point
$\VEC{p}= (1,\sqrt{3})$ du cercle de rayon $2$ centré à l'origine
\PDFgraph{11_vecteurs/vect14}

Nous présentons trois méthodes pour résoudre ce problème.

\subQ{a} Nous remarquons que le vecteur $\VEC{p}$ est perpendiculaire à la
droite tangente au cercle car $\VEC{p}$ représente un rayon du cercle.
Les points $\VEC{x}=(x_1,x_2)$ de la droite tangente cherchée
satisfont donc
\[
\ps{\VEC{x} - \VEC{p}}{\VEC{p}} = \ps{ (x_1-1, x_2-\sqrt{3})}{(1,\sqrt{3})}
= (x_1-1) + \sqrt{3} (x_2-\sqrt{3}) = 0 \; .
\]
Si nous résolvons pour $x_2$, nous obtenons
\[
x_2 = \frac{-1}{\sqrt{3}}\ x_1 + \frac{4}{\sqrt{3}} \ ;
\]
une représentation classique pour la droite tangente.

\subQ{b} Une autre méthode pour résoudre notre problème fait appel à
la représentation paramétrique d'une droite.  Premièrement, nous trouvons un
vecteur parallèle à la tangent au cercle au point $\VEC{p}$.
C'est-à-dire que nous cherchons un vecteur perpendiculaire au vecteur
$\VEC{p} = (1,\sqrt{3})$.  Il est facile de voir que
$\VEC{q} = (\sqrt{3}, -1)$ est perpendiculaire au vecteur
$\VEC{p}$ car $\ps{\VEC{q}}{\VEC{p}} = 0$.  Ainsi, les points
$\VEC{x}=(x_1,x_2)$ de la droite tangent au cercle au point $\VEC{p}$
satisfont
\[
\VEC{x} = \VEC{p} + \alpha \VEC{q} \; .
\]
Une représentation paramétrique de la droite tangent au cercle au point
$(1,\sqrt{3})$ est alors
\[
x_1 = 1 + \alpha \, \sqrt{3} \quad \text{et} \quad x_2 = \sqrt{3} - \alpha
\]
pour tout nombre réel $\alpha$.

Si nous résolvons pour $\alpha$ la première équation de la représentation
paramétrique, nous trouvons $\alpha = (x_1-1)/\sqrt{3}$.  Si nous
substituons cette expression pour $\alpha$ dans la deuxième équation
de la représentation paramétrique, nous obtenons une représentation
classique pour l'équation d'une droite tangente.
\[
x_2 = \sqrt{3} - \frac{x_1-1}{\sqrt{3}} = \frac{-1}{\sqrt{3}}\ x_1 
+ \frac{4}{\sqrt{3}} \; .
\]

\subQ{c} Finalement, la façon traditionnelle de trouver l'équation de
la tangente au point $\VEC{p}$ est de commencer par trouver la pente de la
tangente.  Pour cela, nous notons que la pente de la droite qui contient
le vecteur $\VEC{p}$ est $\displaystyle m_1 = \sqrt{3}$.  Comme la
droite tangente est perpendiculaire à la droite qui contient le
vecteur $\VEC{p}$, sa pente est donc
\[
m_2 = \frac{-1}{m_1} = \frac{-1}{\sqrt{3}} \; .
\]
La forme point-pente de l'équation de la tangente au cercle au point
$\VEC{p}$ est donc
\[
x_2 - \sqrt{3} = \frac{-1}{\sqrt{3}} ( x_1 - 1) \; .
\]
Donc
\[
x_2 = \frac{-1}{\sqrt{3}} \ x_1 + \frac{4}{\sqrt{3}}  \; .
\]
\label{TANG_CIRCL}
\end{egg}

\subsection{Intersection de deux droites}

Dans le plan, deux droites non parallèles se coupent en un point.
Dans l'espace, deux droites non parallèles ne se coupent généralement
pas en un point.  L'exemple suivant illustre une méthode pour trouver
l'intersection de deux droites si cette intersection existe.

\begin{egg}
Trouvons l'intersection (s'il y en a une) des droites $\ell$ et $\ell'$
données par les représentations paramétriques suivantes.  La droite
$\ell$ possède la représentation paramétrique
\[
x_1 = 2s -3 \ , \ x_2 = s -1 \quad \text{et} \quad  x_3 = -s +5
\]
pour $s\in \RR$, et la droite $\ell'$ possède la représentation
paramétrique
\[
x_1 = t -3 \ , \ x_2 = -2 t + 1 \quad \text{et} \quad  x_2 = t - 1 \; ,
\]
pour $t \in \RR$.

Ces deux droites ne sont pas parallèles car $\ell$ est parallèle au
vecteur $(2,1,-1)$ et $\ell'$ au vecteur $(1,-2,1)$, et ces deux
vecteurs ne sont pas parallèles.  Pour que ces deux droites se coupent
en un point $(x_1,x_2,x_3)$ commun, il faut qu'ils existent $s$ et $t$
tels que
\[
t -3 = 2s -3 \quad , \quad -2 t + 1 = s -1 \quad \text{et} \quad
t - 1 = -s +5 \; .
\]
La première équation donne $t = 2s$.  Si nous substituons cette expression
pour $t$ dans la deuxième équation, alors $-4s +1 = s-1$ et donc
$s=2/5$.  Nous obtenons de $t = 2s$ que $t= 4/5$.  Malheureusement, si
nous substituons $s=2/5$ dans la troisième équation, nous obtenons
$t = 28/5$.  Il n'y a donc pas de valeurs de $s$ et $t$ qui satisfont
les trois équations simultanément.  Les deux droites ne se
coupent pas.
\end{egg}

% \begin{egg}
% Trouvons l'intersection (s'il y en a une) de la droite $\ell$ donnée par la
% représentation paramétrique
% \[
% x = 2s -2 \ , \ y = -s +2 \ \text{et} \  z = 2s
% \]
% pour $s\in \RR$, et de la droite $\ell'$ donnée par la représentation
% paramétrique
% \[
% x = t -3 \ , \ y = -2 t + 1 \ \text{et} \  z = t - 1
% \]
% pour $t \in \RR$.
% \end{egg}

\section{Équation d'un plan \eng} \label{EQU_PLAN1}

\begin{prop} \label{RSplan}
Le plan qui passe par un point $\VEC{p}$ et qui est perpendiculaire au
vecteur $\VEC{m}$ (figure~\ref{VECT9}) est l'ensemble des
points $\VEC{x}$ tels que le vecteur $\displaystyle \VEC{x}-\VEC{p}$ est
perpendiculaire à $\VEC{m}$.  Si nous utilisons le produit scalaire, nous
pouvons définir ce plan comme l'ensemble des points $\VEC{x}$ tels que
\[
\ps{\VEC{x}-\VEC{p}}{\VEC{m}} = 0 \; .
\]
\end{prop}

\PDFfig{11_vecteurs/vect9}{Représentation vectorielle d'un plan}
{Représentation vectorielle d'un plan}{VECT9}

Si $\displaystyle \VEC{x}=(x_1,x_2,x_3)$, $\displaystyle \VEC{p}=(p_1,p_2,p_3)$
et $\displaystyle \VEC{m} = (m_1,m_2,m_3)$, la formule donnée à la
proposition~(\ref{RSplan}) devient
\begin{equation} \label{EQU_PLAN_S}
m_1 (x_1-p_1) + m_2 (x_2-p_2) + m_3 (x_3-p_3) = 0
\end{equation}
ou simplement
\begin{equation} \label{EQU_PLAN_S2}
m_1 x_1 + m_2 x_2 + m_3 x_3 - d = 0
\end{equation}
où $d = m_1 p_1 + m_2 p_2 + m_3 p_3 = \ps{\VEC{m}}{\VEC{p}}$.  Nous
pourrions penser que la valeur de $d$ va changer si nous choisissons
différent points $\VEC{p}$ du plan.  En fait, ce n'est pas le cas.
Soit $\VEC{t}$ un autre point du plan.  Puisque
$\VEC{t} - \VEC{p}$ est un vecteur parallèle au plan et donc
perpendiculaire à $\VEC{m}$, nous avons
\[
\ps{\VEC{m}}{\VEC{t}} - \ps{\VEC{m}}{\VEC{p}} 
= \ps{\VEC{m}}{\VEC{t} - \VEC{p}} = 0
\]
Ainsi, $\ps{\VEC{m}}{\VEC{t}} = \ps{\VEC{m}}{\VEC{p}}$.

\begin{defn}
la formule (\ref{EQU_PLAN_S}) (ou (\ref{EQU_PLAN_S2})\;) est une
{\bfseries représentation standard}\index{Représentation standard d'un
  plan} d'un plan qui contient le point $\VEC{p}$ et qui est
perpendiculaire au vecteur $\VEC{m}$.  Nous disons que le vecteur $\VEC{m}$
est {\bfseries orthogonal} ou {\bfseries perpendiculaire}
ou {\bfseries normal} au plan.\index{Vecteur!Perpendiculaire à un plan}
\index{Vecteur!Orthogonal à un plan}\index{Vecteur!Normal à un plan}
\end{defn}

Si $m_3 \neq 0$, nous pouvons déduire de (\ref{EQU_PLAN_S}) que
\[
x_3 = p_3 - \frac{m_1}{m_3} (x_1-p_1) - \frac{m_2}{m_3} (x_2- p_2) \;.
\]
Si $m_3 = 0$ et $m_2 \neq 0$, l'équation du plan est simplement
\[
x_2 = p_2 - \frac{m_1}{m_2} (x_1- p_1)
\]
et $x_3$ est libre (i.e. $x_3$ va de $-\infty$ à $\infty$).
C'est un plan qui est parallèle à l'axe des $x_3$.  Nous laissons aux lecteurs
l'étude des autres cas possibles.

\begin{egg}
Si $\VEC{p}=(1,2,3)$ et $\VEC{m} = (1,-2,4)$, l'ensemble des points
$\VEC{x}=(x_1,x_2,x_3)$ du plan qui est perpendiculaire à $\VEC{m}$ et
contient le point $\VEC{p}$ est donnée par la relation
\[
\ps{\VEC{x}-\VEC{p}}{\VEC{m}} =
\ps{(x_1,x_2,x_3) - (1,2,3)}{(1,-2,4)} = (x_1-1) -2 (x_2-2) + 4(x_3-3) = 0 \; .
\]
Si nous résolvons pour $x_3$, nous trouvons
\[
x_3 = 3 - \frac{1}{4}(x_1-1)  + \frac{1}{2} (x_2-2) \ .
\]
\end{egg}

\begin{egg}
Trouvons l'équation du plan tangent au point $\VEC{p}= (1,2,2)$ de la
sphère de rayon $3$ centré à l'origine.

Un vecteur perpendiculaire au plan tangent de la sphère au point
$\VEC{p}$ est donné par le vecteur $\VEC{p}$ lui-même.  Ainsi, les
points $\VEC{x}=(x_1,x_2,x_3)$ du plan tangent cherché satisfont l'équation
\[
\ps{\VEC{x} - \VEC{p}}{\VEC{p}} = 
\ps{(x_1-1,x_2-2,x_3-2)}{(1,2,2)} = (x_1-1) + 2(x_2-2) + 2(x_3-2) = 0 \ .
\]
Si nous résolvons pour $x_3$, nous trouvons
\[
x_3 = 2 - \frac{1}{2}(x_1-1) - (x_2-2) \ .
\]
\label{TANG_PLANE}
\end{egg}

Nous savons qu'un plan est déterminé par trois points non alignés.
Comment pouvons-nous trouver l'équation du plan qui passe par trois points
non alignés?  Nous reformulons la question de la façon suivante. Étant
donné trois points non alignés, comment pouvons-nous trouver un vecteur
normal au plan qui contient ces trois points?  La réponse à cette
question nous est fournie par le produit vectoriel.

\begin{prop}
Si $\VEC{p}$, $\VEC{u}$ et $\VEC{v}$ sont trois points non alignés
d'un plan, alors $\VEC{s} = \VEC{u} - \VEC{p}$ et
$\VEC{t} = \VEC{v} - \VEC{p}$ sont deux vecteurs parallèles au plan
qui ne sont pas colinéaires, et $\VEC{m} = \VEC{s} \times \VEC{t}$ est un
vecteur perpendiculaire au plan.  Si $\VEC{x} = (x_1,x_2,x_3)$ est un
point du plan, la représentation standard du plan est alors donnée par
\[
\ps{\VEC{x} - \VEC{p}}{\VEC{m}} = 0 \; .
\]
\end{prop}

\begin{egg}
Trouvons l'équation du plan qui contient les trois points $(2,1,3)$,
$(1,-2,3)$ et $(1,5,4)$.

Choisissons $\VEC{p}=(2,1,3)$, $\VEC{u}=(1,-2,3)$ et
$\VEC{v}=(1,5,4)$.  Tout autre choix pour $\VEC{p}$, $\VEC{u}$ et
$\VEC{v}$ est valable.  Ainsi,
\begin{align*}
\VEC{s} &= \VEC{u} - \VEC{p} = (1,-2,3)-(2,1,3) = (-1,-3,0)
\intertext{et}
\VEC{t} &= \VEC{v} - \VEC{p} = (1,5,4)-(2,1,3) = (-1,4,1)
\end{align*}
sont deux vecteurs parallèles au plan.  Le vecteur
\[
\VEC{m} = \VEC{s} \times \VEC{t} = (-3, 1, -7)
\]
qui est perpendiculaire au plan.  Les points $\VEC{x}=(x_1,x_2,x_3)$
du plan contenant $\VEC{p}$, $\VEC{u}$ et $\VEC{v}$ satisfont donc
l'équation
\[
\ps{\VEC{x} - \VEC{p}}{\VEC{m}} =
\ps{(x_1,x_2,x_3)-(2,1,3)}{(-3,1,-7)} = -3(x_1-2) + (x_2-1) -7(x_3-3) = 0 \; .
\]
Nous pouvons résoudre pour $z$ pour obtenir
\[
x_3 = 3 -\frac{3}{7}(x_1-2) + \frac{1}{7}(x_2-1) \; .
\]
Le dessin du plan représemté par cette équation est donné ci-dessous.
\MATHgraph{11_vecteurs/vect13}{8cm}

Remarquons que $\VEC{m} = (-3,1,-7)$ satisfait bien la règle de
la main droite avec les vecteurs $\VEC{s} = (-1,3,0)$ et
$\VEC{t} = (-1,4,1)$.
\end{egg}

\subsection{Représentations vectorielles et paramétriques du plan}

Comme pour la droite dans le plan, il existe des représentations
vectorielles et paramétriques pour le plan.

Le plan qui contient le point $\VEC{p}$ et qui est parallèle à deux
vecteurs $\VEC{s}$ et $\VEC{t}$ qui ne sont pas colinéaires
(figure~\ref{VECT12}) est formé des points $\VEC{x} = (x_1,x_2,x_3)$
de la forme
\begin{equation}\label{pr_pln}
\VEC{x} = \VEC{p} + \alpha \VEC{s} + \beta \VEC{t}
\end{equation}
où $\alpha$ et $\beta$ sont des nombres réels.

\begin{defn} \index{Représentation vectorielle du plan}
La formule (\ref{pr_pln}) est une {\bfseries représentation vectorielle}
du plan qui contient le point $\VEC{p}$ et est parallèle à deux
vecteurs $\VEC{s}$ et $\VEC{t}$ qui ne sont pas colinéaires.
\end{defn}

\PDFfig{11_vecteurs/vect12}{Représentation d'un plan défini par un
point et deux vecteurs qui ne sont pas colinéaires}{Représentation
d'un plan défini par un point et deux vecteurs qui ne sont pas
colinéaires}{VECT12}

Si $\VEC{p}=(p_1,p_2,p_3)$, $\VEC{s} = (s_1,s_2,s_3)$,
$\VEC{t}=(t_1,t_2,t_3)$ et $\VEC{x}=(x_1,x_2,x_3)$, les composantes de la
représentation vectorielle
$\VEC{x} = \VEC{p} + \alpha \VEC{s} + \beta \VEC{s}$ sont
\begin{subequations} \label{paramRep}
\begin{align}
x_1 &= p_1 + \alpha s_1 + \beta t_1 \label{paramRepa} \\
x_2 &= p_2 + \alpha s_2 + \beta t_2 \label{paramRepb} \\
x_3 &= p_3 + \alpha s_3 + \beta t_3 \label{paramRepc}
\end{align}
\end{subequations}
pour $\alpha$ et $\beta$ des nombres réels.

\begin{defn} \index{Représentation paramétrique du plan}
L'ensemble des formules en (\ref{paramRep}) est une {\bfseries
représentation paramétrique} du plan contenant $\VEC{p}$ et parallèle
à deux vecteurs $\VEC{s}$ et $\VEC{t}$ qui ne sont pas colinéaires.
Les {\bfseries paramètres} sont $\alpha$ et $\beta$.
\end{defn}

\begin{egg}
Les points $\VEC{x}=(x_1,x_2,x_3)$ du plan contenant $\VEC{p}=(1,2,3)$ et
parallèle aux vecteurs $\VEC{s}=(-1,2,2)$ et $\VEC{t}=(1,-1,2)$ sont
donnés par
\begin{align*}
(x_1,x_2,x_3) &= \VEC{r} = \VEC{p} + \alpha \VEC{s} + \beta \VEC{t}
= (1,2,3) + \alpha (-1,2,2) + \beta (1,-1,2) \\
&= (1-\alpha +\beta, 2 +2 \alpha - \beta, 3 + 2\alpha +2\beta) \; .
\end{align*}

Nous avons donc la représentation paramétrique
\[
x_1 = 1-\alpha+\beta \ , \ x_2 = 2+2\alpha-\beta \quad \text{et} \quad 
x_3 = 3+2\alpha + 2\beta \; .
\]
\end{egg}

\begin{rmk}[\theory]
Il existe une relation entre la représentation paramétrique et standard du
plan.

Si nous soustrayons (\ref{paramRepb}) multipliée par $t_1$ de
(\ref{paramRepa}) multipliée par $t_2$, nous obtenons
\[
x_1 t_2 - x_2 t_1 = (p_1 t_2 - p_2 t_1) + \alpha (s_1 t_2 - s_2 t_1) \; .
\]
Ainsi,
\[
\alpha = \frac{1}{s_1 t_2 - s_2 t_1}
\left(x_1 t_2 - x_2 t_1 - p_1 t_2 + p_2 t_1\right) = 
\frac{t_2}{s_1 t_2 - s_2 t_1} \left(x_1 - p_1\right)
-\frac{t_1}{s_1 t_2 - s_2 t_1} \left(x_2 - p_2\right)
\]
si $s_1 t_2 - s_2 t_1 \neq 0$.

Si nous soustrayons (\ref{paramRepb}) multipliée par $s_1$ de (\ref{paramRepa})
multipliée par $s_2$, nous obtenons
\[
x_1 s_2 - x_2 s_1 = (p_1 s_2 - p_2 s_1) + \beta (t_1 s_2 - t_2 s_1) \; .
\]
Ainsi,
\[
\beta = \frac{1}{t_1 s_2 - t_2 s_1}
\left(x_1 s_2 - x_2 s_1 - p_1 s_2 + p_2 s_1\right) = 
\frac{s_2}{t_1 s_2 - t_2 s_1} \left(x_1 - p_1\right)
-\frac{s_1}{t_1 s_2 - t_2 s_1} \left(x_2 - p_2\right)
\]
si $s_1 t_2 - s_2 t_1 \neq 0$.

Si nous substituons $\alpha$ et $\beta$ dans (\ref{paramRepc}), nous
trouvons
\begin{align*}
x_3 &= p_3 +
\frac{t_2s_3}{s_1 t_2 - s_2 t_1} \left(x_1 - p_1\right)
-\frac{t_1 s_3}{s_1 t_2 - s_2 t_1} \left(x_2 - p_2\right)
+ \frac{s_2t_3}{t_1 s_2 - t_2 s_1} \left(x_1 - p_1\right) \\
& \qquad -\frac{s_1t_3}{t_1 s_2 - t_2 s_1} \left(x_2 - p_2\right) \\
&= p_3 - \frac{s_2 t_3 - t_2s_3}{s_1 t_2 - s_2 t_1} \left(x_1 -p_1\right)
- \frac{t_1 s_3 - s_1 t_3}{s_1 t_2 - s_2 t_1} \left(x_2 - p_2\right) \; .
\end{align*}
Nous pouvons récrire cette expression sous la forme suivante.
\[
\left(s_2 t_3 - t_2s_3\right) \left(x_1 -p_1\right) 
+ \left(t_1 s_3 - s_1 t_3\right) \left(x_2 - p_2\right)
+ \left(s_1 t_2 - s_2 t_1\right) \left(x_3 - p_3\right) = 0 \; .
\]
C'est la représentation standard donnée en (\ref{EQU_PLAN_S}) où
$\VEC{m} = \VEC{s} \times \VEC{t}$.
\end{rmk}

\subsection{Intersection d'une droite et d'un plan}

Si une droite $\ell$ n'est pas parallèle au plan $\cal M$, alors la
droite coupe le plan en un point.  La droite $\ell$ est parallèle au
plan $\cal M$ si le produit scalaire d'un vecteur $\VEC{q}$ parallèle
à la droite $\ell$ avec un vecteur $\VEC{m}$ perpendiculaire au plan
est $0$ car le plan $\cal M$ est une translation de l'ensemble des
vecteurs $\VEC{u}$ tels que $\ps{\VEC{u}}{\VEC{m}} = 0$ (figure~\ref{PROJ3}).

\PDFfig{11_vecteurs/proj3}{Droite parallèle à un plan}{La droite $\ell$ est
parallèle au plan $\cal M$ si $\ps{\VEC{q}}{\VEC{m}} = 0$ où le vecteur
$\VEC{q}$ est parallèle à la droite $\ell$ et $\VEC{m}$ est un vecteur
perpendiculaire au plan $\cal M$.}{PROJ3}

\begin{egg}
Considérons la droite $\ell$ définie par la représentation standard
\[
\frac{x_1-5}{3} = \frac{x_2-1}{2} = x_3+2
\]
et le plan $\cal M$ donné par $x_1+2x_2+x_3 = 4$.  Trouvons l'intersection
(s'il y en a une) de la droite $\ell$ et du plan $\cal M$.
  
Une représentation paramétrique de la droite $\ell$ est
\[
x_1 = 3\alpha +5 \ , \ x_2 = 2\alpha + 1 \ \text{et} \ x_3= \alpha - 2 \; .
\]
Le point $\VEC{x} = (x_1,x_2,x_3)$ de la droite $\ell$ qui appartient
aussi au plan $\cal M$ doit satisfaire
\[
(3\alpha +5) + 2(2\alpha + 1) + (\alpha - 2) = 4 \; .
\]
Nous obtenons $\alpha = -1/8$.  Les coordonnées du point d'intersection
sont donc
\[
x_1 = 3\left(\frac{-1}{8}\right)+5 = \frac{37}{8} \ , \ 
x_2 = 2\left(\frac{-1}{8}\right) + 1 = \frac{3}{4} \quad \text{et} \quad
x_3 = \left(\frac{-1}{8}\right) - 2 = -\frac{17}{8} \; .
\]
\label{planline}
\end{egg}

\begin{rmk}
Il est facile de trouver un vecteur tangent à la droite $\ell$ de
l'exemple~\ref{planline} et un vecteur perpendiculaire au plan
$\cal M$. 

À partir de la représentation standard ou paramétrique de la droite
$\ell$, nous déduisons la représentation vectorielle
$\VEC{x} = \VEC{p} + \alpha \VEC{q}$ où $\VEC{x} = (x_1,x_2,x_3)$,
$\VEC{p} = (5,1,-2)$ et $\VEC{q} = (3,2,1)$.  Le vecteur
$\VEC{q} = (3,2,1)$ est parallèle à la droite $\ell$.

Un vecteur $\VEC{m} = (m_1.m_2.m_3)$ perpendiculaire au plan $\cal M$
est donné par les coefficients de $x_1$, $x_2$ et $x_3$ dans l'équation
$x_1+2x_2+x_3-4 = 0$.  En effet, $x_1+2x_2+x_3-4 = 0$ est la représentation
standard (\ref{EQU_PLAN_S2}) du plan $\cal M$.  Ainsi,
$\VEC{m} = (1,2,1)$ et $\ps{\VEC{x}}{\VEC{m}} = 4$ 
pour les points $\VEC{x}$ du plan $\cal M$.

Puisque $\ps{\VEC{m}}{\VEC{q}} = 8 \neq 0$, nous avons que $\VEC{q}$ n'est
pas perpendiculaire à $\VEC{m}$ et donc que la droite $\ell$
n'est pas parallèle au plan $\cal M$.  Ainsi, $\ell$ coupe le plan
$\cal M$.
\end{rmk}

\subsection{Intersection de deux plans}

Deux plans distincts non parallèles vont se couper en une droite.  Si
$\VEC{m}$ est un vecteur perpendiculaire au plan $\cal M$ et $\VEC{n}$
est un vecteur perpendiculaire au plan $\cal N$, alors les plans
$\cal M$ et $\cal N$ sont parallèles si les vecteurs $\VEC{m}$ et $\VEC{n}$
sont colinéaires; c'est-à-dire, un multiple l'un de l'autre.

Si le plan $\cal M$ est représenté par l'équation
$m_1 x_1 + m_2 x_2 + m_3 x_3 = k_{\VEC{m}}$ et le plan $\cal N$ par 
$n_1 x_1 + n_2 x_2 + n_3 x_3 = k_{\VEC{n}}$, l'intersection des plans
$\cal M$ et $\cal N$ est donc l'ensemble des points
$\VEC{x} = (x_1,x_2,x_3)$ qui satisfont ces deux équations 
simultanément.  En d'autres mots, l'intersection des deux plans est
l'ensemble des solutions du {\em système d'équations linéaires}
\begin{subequations}\label{ls_plan}
\begin{align}
m_1 x_1 + m_2 x_2 + m_3 x_3 &= k_{\VEC{m}} \label{ls_plana} \\
n_1 x_1 + n_2 x_2 + n_3 x_3 &= k_{\VEC{n}} \label{ls_planb}
\end{align}
\end{subequations}

\subsubsection{Les deux plans sont parallèles}

Si les deux plans sont parallèles, alors $\VEC{m} = \lambda \VEC{n}$
pour un nombre réel $\lambda$.
\begin{enumerate}
\item Si $k_{\VEC{m}} \neq \lambda k_{\VEC{n}}$, il ne peut pas y
avoir de points 
$(p_1,p_2,p_3)$ qui satisfassent (\ref{ls_plana}) et (\ref{ls_planb})
simultanément.  Si c'était le cas, alors nous aurions
\begin{align*}
m_1 p_1 + m_2 p_2 + m_3 p_3 &= k_{\VEC{m}} \\
n_1 p_1 + n_2 p_2 + n_3 p_3 &= k_{\VEC{n}}
\end{align*}
avec $m_i = \lambda n_i$.  Si nous soustrayons $\lambda$ fois la
deuxième équation de la première équation, nous obtenons
$0 = k_{\VEC{m}} - \lambda k_{\VEC{n}}$.  Ce qui contredit notre
hypothèse.  Nous avons donc deux plan distincts et parallèles.
\item Si $k_{\VEC{m}} = \lambda k_{\VEC{n}}$ alors l'équation
(\ref{ls_plana}) est un 
multiple par $\lambda$ de l'équation (\ref{ls_planb}) et tout point
$\VEC{x} = (x_1,x_2,x_3)$ qui satisfait (\ref{ls_planb}) satisfait aussi
(\ref{ls_plana}) et vice-versa.  Les plans $\cal M$ et $\cal N$ sont
en fait un seul et même plan.
\end{enumerate}

\subsubsection{Les deux plans ne sont pas parallèles}

Si les deux plan ne sont pas parallèles, $\VEC{m}$ n'est pas un multiple de
$\VEC{n}$.  Nous verrons lors de l'étude de l'algèbre linéaire que les
solutions du système d'équations linéaires donné par (\ref{ls_plana}) et
(\ref{ls_planb}) représente alors une droite $\ell$.  Puisque cette droite
est contenue dans les plans $\cal M$ et $\cal N$, elle doit être
perpendiculaire aux vecteurs $\VEC{m}$ et $\VEC{n}$.  Donc $\ell$ est
parallèle au vecteur $\VEC{r} = \VEC{m} \times \VEC{n}$.

Si $\VEC{p}$ est un point de $\ell$, donc une solution du système
d'équations linéaires donné par (\ref{ls_plana}) et (\ref{ls_planb}),
alors l'équation de la droite d'intersection $\ell$ est donnée par
\[
  \ps{\VEC{x} - \VEC{p}}{\VEC{r}} =  r_1(x_1 - p_1) + r_2(x_2-p_2) +
  r_3(x_3 - p_3) = 0 \ .
\]

Le problème est donc de trouver au moins une solution du système
d'équations linéaires donné par (\ref{ls_plana}) et (\ref{ls_planb}).
En fait, trouver une solution n'est pas plus difficile que de trouver
toutes les solutions et donc l'équation de la droite d'intersection.
C'est ce que nous décrivons ci-dessous.

Si (\ref{ls_plana}) définit un plan dans l'espace, ces coefficients
$m_i$ ne peuvent pas tous être nuls.  Supposons que $m_1 \neq 0$; 
le raisonnement est semblable si $m_2$ ou $m_3$ est non nul.  Si nous
soustrayons le produit de (\ref{ls_plana}) par $n_1/m_1$ de
(\ref{ls_planb}), nous obtenons
\begin{equation}\label{ls_planc}
  \frac{n_2 m_1 - n_1 m_2}{m_1} x_2 + \frac{m_1 n_3 - n_1 m_3}{m_1} x_3 =
  \frac{k_{\VEC{n}} m_1 - n_1 k_{\VEC{m}}}{m_1} \ .
\end{equation}
Puisque nous supposons que $\VEC{m}$ et $\VEC{n}$ ne sont pas
colinéaires, les coefficients de $x_2$ et $x_3$ ne peuvent pas être tous
les deux nuls.  Supposons que le coefficient de $x_2$ soit non nul.  Le
raisonnement est semblable si le coefficient de $x_3$ est non nul.  Nous
pouvons alors résoudre pour $x_2$. 
\begin{equation}\label{ls_planP1}
  x_2 = C_1 + C_2 x_3 \ ,
\end{equation}
où
$\displaystyle  C_1 = \frac{k_{\VEC{n}} m_1 - n_1 k_{\VEC{m}}}{n_2 m_1 - n_1 m_2}$ et
$\displaystyle  C_2 = - \frac{m_1 n_3 - n_1 m_3}{n_2 m_1 - n_1 m_2}$.
Si nous substituons l'expression pour $x_2$ obtenue en (\ref{ls_planP1})
dans (\ref{ls_plana}), et résolvons pour $x_1$, nous trouvons
\begin{equation}\label{ls_planP2}
  x_1 = C_3 + C_4 x_3 \ ,
\end{equation}
où
$\displaystyle  C_3 =
\frac{k_{\VEC{m}} n_2 - m_2 k_{\VEC{n}}}{n_2 m_1 - n_1 m_2}$ et
$\displaystyle  C_4 = - \frac{m_2 n_3 - n_2 m_3}{n_2 m_1 - n_1 m_2}$.
Notez que $C_2$ et $C_4$ ne peuvent pas être tous les deux nuls pour
la même raison que les coefficients de $x_2$ et $x_3$ dans
(\ref{ls_planc}) ne peuvent pas être tous les deux nuls\footnote{Nous
verrons dans le chapitre sur l'algèbre linéaire que, si la matrice de
dimension \nm{2}{3} dont les lignes sont données par les vecteurs
$\VEC{m}$ et $\VEC{n}$ est de rang 2, alors au moins deux des
expressions $m_1 n_2 - n_1 m_2$, $m_2 n_3 - n_2 m_3$ et
$m_1 n_3 - n_1 m_3$ sont non nuls.}

Nous obtenons donc la représentation paramétrique
\[
  x_1 = C_3 + C_4 \alpha \quad , \quad x_2 = C_1 + C_2 \alpha \quad \text{et}
\quad x_3 = \alpha
\]
pour la droite d'intersection des plan $\cal M$ et $\cal N$.

Il ne faut surtout pas essayer de mémoriser les formules ci-haut pour trouver
la représentation paramétrique de la droite d'intersection de deux plans.
L'exemple suivant illustre comment trouver l'intersection de deux plans.

\begin{egg}
Trouvons, si elle existe, l'intersection des plans $\cal M$ et $\cal N$
donnés respectivement par les équations
\begin{subequations}\label{ls_planB}
\begin{align}
x_1 + 2x_2 + x_3 &= 5 \label{ls_planBa}
\intertext{et}
2x_1 + x_2 + x_3 &= 3 \ . \label{ls_planBb}
\end{align}
\end{subequations}

Le vecteur $\VEC{m} = (1,2,1)$ est perpendiculaire au plan $\cal M$ et le
vecteur $\VEC{n} = (2,1,1)$ est perpendiculaire au plan $\cal N$.  Comme ils
ne sont pas colinéaires (nous ne pouvons pas écrire $\VEC{m}$ comme un
multiple de $\VEC{n}$), les deux plans se coupent en une droite $\ell$.

\subI{1$^{er}$ méthode}  Le vecteur
$\VEC{r} = \VEC{m} \times \VEC{n} = (1, 1, -3)$ est
parallèle à la droite $\ell$.  Il est facile de vérifier que
$(0, 2, 1)$ est un point de la droite $\ell$ puisque c'est une
point qui appartient aux deux plans.  La représentation standard de
$\ell$ est donc
\begin{equation} \label{ls_plan8}
  x_1 = x_2 - 2 = \frac{x_3-1}{-3}
\end{equation}

\subI{2$^{e}$ méthode} Si nous soustrayons deux fois l'équation
(\ref{ls_planBa}) de l'équation (\ref{ls_planBb}), nous obtenons
$-3 x_2 -x_3 = -7$.  Ainsi,
\[
x_2 = \frac{x_3-7}{-3} = -\frac{1}{3}x_3 +\frac{7}{3} \; .
\]
Si nous substituons cette expression pour $y$ dans (\ref{ls_planBa}), nous
obtenons
\[
x_1 + 2\left(\frac{x_3-7}{-3}\right) + x_3 = 5 \; .
\]
Ceci donne
\[
x_1 = -2\left(\frac{x_3-7}{-3}\right) - x_3 + 5
= -\frac{1}{3} x_3 + \frac{1}{3} \; .
\]
Nous obtenons donc la représentation paramétrique suivante pour la droite
$\ell$ produite par l'intersection des plans $\cal M$ et $\cal N$.
\begin{equation} \label{ls_plan6}
x_1 = -\frac{1}{3} \alpha + \frac{1}{3} \quad , \quad 
x_2 = -\frac{1}{3}\alpha +\frac{7}{3} \quad \text{et} \quad x_3 = \alpha \; .
\end{equation}

\subI{3$^{e}$ méthode} Nous pouvons aussi résoudre pour
$x_3$ l'équation $-3 x_2 -x_3 = -7$ obtenue au début de la
$2^e$ méthode.  Nous trouvons alors $x_3= -3x_2 + 7$.  Si nous substituons
cette expression dans (\ref{ls_planBa}), nous obtenons $x_1+2x_2+(-3x_2+7)=5$.
Ainsi, $x_1 = x_2 -2$.  Nous obtenons une nouvelle représentation
paramétrique pour la droite $\ell$ produite par l'intersection des
plans $\cal M$ et $\cal N$.
\begin{equation} \label{ls_plan7}
  x_1 = \beta - 2 \quad , \quad x_2 = \beta \quad \text{et} \quad
  x_3 = -3 \beta + 7 \; .
\end{equation}

(\ref{ls_plan8}), (\ref{ls_plan6}) et (\ref{ls_plan7}) sont trois 
représentations équivalentes de la droite d'intersection $\ell$.

Nous avons bien obtenu des représentations équivalentes de la droite
$\ell$.  En effet, notons que les trois représentations décrivent des
droites qui ont la même direction: $(1,1,-3)$, $(-1/3,-1/3,1)$ et
$(1,1,-3)$ respectivement.  De plus, ils contiennent tous le point
$(0,2,1)$.  Nous obtenons $(0,2,1)$ avec $\alpha=1$ dans (\ref{ls_plan6})
et $\beta =2$ dans (\ref{ls_plan7}).  Puisque les trois représentation
décrivent des droites parallèles qui passent par le même point, nous avons
donc la même droite.

Nous remarquons de plus, que si nous substituons $\alpha = -3\beta + 7$ dans
(\ref{ls_plan6}), nous obtenons (\ref{ls_plan7}).  De même, si nous
substituons $\beta = -(\alpha-7)/3$ dans (\ref{ls_plan7}), nous obtenons
(\ref{ls_plan6}).  Ainsi, l'ensemble des points produits par la
représentation (\ref{ls_plan6}) est le même que l'ensemble des points
produits par la représentation (\ref{ls_plan7}).   Par exemple, si
$\alpha = 1$ dans la représentation (\ref{ls_plan6}), nous obtenons
$(x_1,x_2,x_3) = ( 0, 2, 1)$.  La représentation (\ref{ls_plan7}) donne ce
même point avec $\beta = -(1-7)/3 = 2$.  Si $\beta = 1$ dans la
représentation (\ref{ls_plan7}), nous obtenons $(x_1,x_2,x_3) = ( -1, 1, 4)$.
La représentation (\ref{ls_plan6}) donne ce même point avec
$\alpha = -3 (1) + 7 = 4$.
\end{egg}

\subsection{Intersection de trois plans}

Les exemples suivants montre comment nous pouvons résoudre un systèmes de trois
équations avec trois inconnues en utilisant la technique de substitution.
Malheureusement, cette méthode ne nous permet pas de classifier les cas
possibles d'intersections; c'est-à-dire, une intersection vide, une droite ou
un plan.

L'algèbre linéaire que nous verrons dans un prochain chapitre simplifiera
grandement l'étude des cas possibles d'intersections de trois plans.

\begin{egg}
Quelle est l'intersection, si elle existe, des trois plans suivants?
\begin{subequations}\label{PlanA}
\begin{align}
x_1 + x_2 + x_3 &= 3 \label{planAa} \\
3x_1 + 2x_2 + x_3 &= 5 \label{planAb} \\
x_1 + x_2 - 2x_3 &= 1 \label{planAc}
\end{align}
\end{subequations}

De (\ref{planAa}), nous obtenons $x_3=3-x_1-x_2$ que nous substituons dans
(\ref{planAb}) et (\ref{planAc}) pour obtenir
\begin{subequations}\label{PlanE}
\begin{align}
2x_1 + x_2 &= 2 \label{planEa}\\
3x_1 + 3x_2 &= 7 \label{planEb}
\end{align}
\end{subequations}

De (\ref{planEa}), nous obtenons $x_2=2-2x_1$ que nous substituons dans
(\ref{planEb}) pour obtenir $-3x_1 = 1$.  Ainsi, $x_1=-1/3$, $x_2=2-2x_1=8/3$
et $x_3=3-x_1-x_2 = 2/3$.  Le point $(-1/3, 8/3 , 2/3)$ est le seul point
commun (i.e. qui appartient) au trois plans.
\end{egg}

\begin{egg}
Quelle est l'intersection, si elle existe, des trois plans suivants?
\begin{subequations}\label{PlanC}
\begin{align}
x_1 + 2x_2 +x_3 &= 5 \label{planCa}\\ 
2x_1 -x_2 + x_3 &= 2 \label{planCb}\\
3x_1 + x_2 + 2x_3 &= 7 \label{planCc}
\end{align}
\end{subequations}

De (\ref{planCa}), nous obtenons $x_3=5-x_1-2x_2$ que nous substituons dans
(\ref{planCb}) et (\ref{planCc}) pour obtenir $x_1 -3 x_2 = -3$ dans les deux
cas.   Ainsi, $x_2= x_1/3  + 1$ et $x_3=5-x_1-2x_2 = - 5x_1/3 + 3$.  Nous
trouvons la représentation paramétrique
\[
x_1=\alpha \ , \ x_2 = \frac{\alpha}{3} + 1 \ \text{et}
\ x_3 = -\frac{5x}{3} + 3
\]
d'une droite $\ell$.  La droite $\ell$ est l'ensemble des points
commun au trois plans.
\end{egg}

\begin{egg}
Quelle est l'intersection, si elle existe, des trois plans suivants?
\begin{subequations}\label{PlanD}
\begin{align}
x_1 + 2x_2 +x_3 &= 5 \label{planDa}\\ 
2x_1 -x_2 + x_3 &= 2 \label{planDb}\\
-x_1 + 3x_2 &= 7 \label{planDc}
\end{align}
\end{subequations}

De (\ref{planDa}), nous obtenons $x_3=5-x_1-2x_2$ que nous substituons dans
(\ref{planDb}) pour obtenir $x_1 -3 x_2 = -3$.  Ainsi, $x_2 = x_1/3 +1$.
Malheureusement, si nous substituons cette expression pour $x_2$ dans
(\ref{planDc}), nous trouvons $3 = 7$.  Ce qui est absurde.  Il n'existe donc
aucun point $(x_1,x_2,x_3)$ qui satisfasse les trois équations simultanément;
c'est-à-dire, qui est commun au trois plans.
\end{egg}

\section{Projections \theory}

\subsection{Plus courte distance entre un point et une droite}

Quelle est la plus courte distance entre un point $\VEC{a}$ et une
droite $\ell$ qui passe par l'origine?  Pour répondre à cette
question, nous avons premièrement besoin de trouver le point $\VEC{b}$
de la droite $\ell$ qui est le plus près de $\VEC{a}$ comme il est
illustré à la figure~\ref{PROJ1}.

\PDFfig{11_vecteurs/proj1}{Projection d'un vecteur sur une droite}
{Le point $\VEC{b}$ est le point de la droite $\ell$ qui est le plus
près de $\VEC{a}$.  Le vecteur $\VEC{b}$ est appelé la projection du
vecteur $\VEC{a}$ sur la droite $\ell$ passant pas l'origine.}{PROJ1}

Soit $\VEC{q}$ un vecteur contenu dans la droite $\ell$.  Pour trouver
les coordonnées de $\VEC{b}$, nous remarquons que $\VEC{b} = \alpha \VEC{q}$
pour un certain nombre réel $\alpha$.  De plus $\VEC{a}-\VEC{b}$ est
perpendiculaire à $\VEC{q}$.  Nous avons donc
\[
0 = \ps{\VEC{a}-\VEC{b}}{\VEC{q}}
= \ps{\VEC{a}- \alpha\VEC{q}}{\VEC{q}}
= \ps{\VEC{a}}{\VEC{q}} - \alpha \ps{\VEC{q}}{\VEC{q}} \; .
\]
Ainsi,
\begin{equation}\label{expr_x}
\alpha = \frac{\ps{\VEC{a}}{\VEC{q}}}{\ps{\VEC{q}}{\VEC{q}}} \; .
\end{equation}
Nous avons donc trouvé que
\[
\VEC{b} = \frac{\ps{\VEC{a}}{\VEC{q}}}{\ps{\VEC{q}}{\VEC{q}}} \VEC{q} \; .
\]

Il est maintenant facile de calculer la plus courte distance entre le
point $\VEC{a}$ et la droite $\ell$; c'est-à-dire, la distance entre
les points $\VEC{a}$ et $\VEC{b}$.  Il suffit de calculer la longueur
du vecteur $\VEC{a}-\VEC{b}$.

% Pour ceux intéressés à une formule générale, nous avons
% \[
% \|\VEC{a}-\VEC{b}\|^2 = \ps{\VEC{a} - \alpha\VEC{q}}{\VEC{a} - \alpha\VEC{q}}
% = \ps{\VEC{a}}{\VEC{a}} - \alpha\ps{\VEC{q}}{\VEC{a}} -
% x\ps{\VEC{a}}{\VEC{q}} + \alpha^2 \ps{\VEC{q}}{\VEC{q}}  \; .
% \]
% Si nous substituons l'expression pour $\alpha$ en (\ref{expr_x})
% dans cette équation, nous trouvons
% \begin{align*}
% \|\VEC{a}-\VEC{b}\|^2 &= \ps{\VEC{a}}{\VEC{a}} -
% \frac{\ps{\VEC{a}}{\VEC{q}}}{\ps{\VEC{q}}{\VEC{q}}} \ps{\VEC{q}}{\VEC{a}} -
% \frac{\ps{\VEC{a}}{\VEC{q}}}{\ps{\VEC{q}}{\VEC{q}}} \ps{\VEC{a}}{\VEC{q}} +
% \left( \frac{\ps{\VEC{a}}{\VEC{q}}}{\ps{\VEC{q}}{\VEC{q}}} \right)^2
% \ps{\VEC{q}}{\VEC{q}} \\
% &= \frac{\ps{\VEC{a}}{\VEC{a}}\ps{\VEC{q}}{\VEC{q}} -
% \left( \ps{\VEC{a}}{\VEC{q}} \right)^2}{\ps{\VEC{q}}{\VEC{q}}} \; .
% \end{align*}

Pour résumer, nous avons le résultat suivant.

\begin{defn} \index{Projection orthogonale}
Soit $\ell$ une droite qui contient le vecteur $\VEC{q}$.  Si
$\VEC{a}$ un point quelconque, la {\bfseries projection (orthogonale)}
du vecteur $\VEC{a}$ sur la droite $\ell$ est le vecteur
\[
\VEC{b} =  \frac{\ps{\VEC{a}}{\VEC{q}}}{\ps{\VEC{q}}{\VEC{q}}} \VEC{q} \; .
\]
Le point $\VEC{b}$ est le point de la droite $\ell$ qui est le plus
près de $\VEC{a}$.  Le vecteur $\VEC{a}-\VEC{b}$ est perpendiculaire à
la droite $\ell$.  La plus courte distance entre $\VEC{a}$ et la droite
$\ell$ est la longueur du vecteur $\VEC{a}-\VEC{b}$.
\end{defn}

\begin{egg}
Quelle est la plus courte distance entre le point $\VEC{a}=(1,1,1)$ et
la droite $\ell$ donnée par $x_1 = 4x_2 = 2x_3$?

Il faut premièrement trouver la projection $\VEC{b}$ du vecteur
$\VEC{a} = (1,1,1)$ sur la droite $\ell$.  Pour cela, il faut choisir un
vecteur $\VEC{q}$ sur la droite $\ell$.  Comme nous considérons seulement
des droites qui passent par l'origine, il suffit de prendre le vecteur
qui donne la direction de la droite dans l'une des représentations de
la droite.  Par exemple, la représentation standard de la droite
$\ell$ est
\[
  \frac{x_1}{1} = \frac{x_2}{1/4} = \frac{x_3}{1/2} \ .
\]
Donc $\VEC{q} = \left(1, 1/4, 1/2\right)$ est un bon choix.  En fait,
tout multiple non nul de $\VEC{q}$ est acceptable.  Ainsi,
\[
\VEC{b} = \frac{\ps{\VEC{a}}{\VEC{q}}}{\ps{\VEC{q}}{\VEC{q}}} \VEC{q}
= \frac{4}{3} \left(1, \frac{1}{4}, \frac{1}{2}\right)
= \left(\frac{4}{3}, \frac{1}{3},\frac{2}{3} \right) \; .
\]
La plus courte distance entre le point $\VEC{a}$ et la droite $\ell$
est donc
\[
\|\VEC{a}-\VEC{b}\| =
\| \left(-\frac{1}{3}, \frac{2}{3},\frac{1}{3} \right) \|
= \sqrt{ \left(-\frac{1}{3}\right)^2 + \left(\frac{2}{3}\right)^2
+ \left(\frac{1}{3}\right)^2}
= \frac{\sqrt{6}}{3} \; .
\]
\end{egg}

\begin{egg}
Quelle est la plus courte distance entre le point $\VEC{a}=(2,7,6)$ et
la droite $\ell$ donnée par la représentation standard
\[
x_1 - 1 = x_2 - 6 = \frac{x_3 - 2}{-3} \; ?
\]

Comme la droite $\ell$ ne passe pas par l'origine, nous
ne pouvons pas utiliser directement les formules précédentes.  Pour
remédier à ce problème, il suffit de faire une translation du point $\VEC{a}$
et de la droite $\ell$ par un vecteur $\VEC{c}$ de telle sorte que la
nouvelle droite $\ell'$ passe par l'origine.  La plus courte distance
entre le point $\VEC{a}'$ et la droite $\ell'$ obtenus de la
translation par le vecteur $\VEC{c}$ sera égale à la plus courte
distance entre le point $\VEC{a}$ et la droite $\ell$ car les 
translations préservent la distance entre les objets (figure~\ref{PROJ4}).

\PDFfig{11_vecteurs/proj4}{Translation dans l'espace}
{Le point $\VEC{a}'$ et la droite $\ell'$ résultent d'une translation
par le vecteur $\VEC{c}$ du point $\VEC{a}$ et de la droite $\ell$
respectivement.}{PROJ4}

La droite $\ell$ a la représentation vectorielle
$\VEC{x} = \VEC{p} + \alpha \VEC{q}$ où $\VEC{x} = (x_1,x_2,x_3)$,
$\VEC{p} = (1,6,2)$ et $\VEC{q} = (1,1,-3)$.  Si nous faisons une
translation par $\VEC{c} = - \VEC{p}$, nous obtenons la droite $\ell'$
donnée par la formule $\VEC{x} = \alpha \VEC{q}$ et le point
\[
\VEC{a}' = \VEC{a} + \VEC{c} = \VEC{a} - \VEC{p} = (1,1,4) \ .
\]
La droite $\ell'$ passe par l'origine car, avec cette translation, le
point $\VEC{p}$ de la droite $\ell$ est envoyé à l'origine.
Notez que $\ell$ et $\ell'$ sont parallèles comme il se doit pour deux
droites dont l'une est la translation de l'autre.

Le vecteur $\VEC{q} = (1,1,-3)$ est naturellement sur la droite
$\ell'$.  La projection $\VEC{b}'$ du vecteur $\VEC{a}$' sur la droite
$\ell'$ est
\[
\VEC{b}' = \frac{\ps{\VEC{a}'}{\VEC{q}}}{\ps{\VEC{q}}{\VEC{q}}} \VEC{q}
= -\frac{10}{11} (1,1,-3)
= \left(-\frac{10}{11}, -\frac{10}{11}, \frac{30}{11} \right) \; .
\]

La plus courte distance entre le point $\VEC{a}'$ et la droite $\ell'$
(et donc entre le point $\VEC{a}$ et la droite $\ell$) est
\[
\|\VEC{a}'-\VEC{b}'\| =
\| \left(\frac{21}{11}, \frac{21}{11}, \frac{14}{11} \right) \|
= \sqrt{ \left(\frac{21}{11}\right)^2 + \left(\frac{21}{11}\right)^2
+ \left(\frac{14}{11}\right)^2 } 
= \frac{7\sqrt{22}}{11} \; .
\]
\end{egg}

\subsection{Plus courte distance entre un point et un plan de
l'espace}

Quelle est la plus courte distance entre un point $\VEC{a}$ et un plan
$\cal M$ qui contient l'origine?  La méthode utilisée pour répondre à
cette question est très semblable à la méthode utilisée pour trouver
la plus courte distance entre un point et une droite.

Nous avons premièrement besoin de trouver le point $\VEC{b}$ du plan
$\cal M$ qui est le plus près de $\VEC{a}$ comme il est illustré à la
figure~\ref{PROJ2}.

\PDFfig{11_vecteurs/proj2}{Projection d'un vecteur sur un plan passant par
l'origine}{Le point $\VEC{b}$ est le point du plan $\cal M$ qui est le
plus près de $\VEC{a}$.  Le vecteur $\VEC{b}$ est appelé la projection
du vecteur $\VEC{a}$ sur le plan $\cal M$ contenant l'origine.}{PROJ2}

Soit $\VEC{p}$ et $\VEC{q}$ deux vecteurs contenus dans le
plan $\cal M$.  Nous assumons que ces deux vecteurs sont perpendiculaires
pour simplifier les calculs qui vont suives.  Il est en générale plus
simple de travailler avec un système de coordonnées orthogonales.

Pour trouver les coordonnées de $\VEC{b}$, nous remarquons que
$\VEC{b} = \alpha \VEC{p} + \beta \VEC{q}$ pour deux nombres réels
$\alpha$ et $\beta$.  De plus $\VEC{a}-\VEC{b}$ est perpendiculaire au
deux vecteurs $\VEC{p}$ et $\VEC{q}$.  Nous avons donc
\begin{align*}
0 &= \ps{\VEC{a}-\VEC{b}}{\VEC{p}}
= \ps{\VEC{a}-\alpha \VEC{p}-\beta \VEC{q}}{\VEC{p}}
= \ps{\VEC{a}}{\VEC{p}} - \alpha \ps{\VEC{p}}{\VEC{p}}
- \beta \ps{\VEC{q}}{\VEC{p}} = \ps{\VEC{a}}{\VEC{p}}
- \alpha \ps{\VEC{p}}{\VEC{p}}
\intertext{et}
0 &= \ps{\VEC{a}-\VEC{b}}{\VEC{q}}
= \ps{\VEC{a}-\alpha \VEC{p}-\beta \VEC{q}}{\VEC{q}}
= \ps{\VEC{a}}{\VEC{q}} - \alpha \ps{\VEC{p}}{\VEC{q}}
- \beta \ps{\VEC{q}}{\VEC{q}}
= \ps{\VEC{a}}{\VEC{q}} - \beta \ps{\VEC{q}}{\VEC{q}}
\end{align*}
car $\ps{\VEC{p}}{\VEC{q}} = 0$.  Ainsi,
\begin{equation}\label{expr_x_y} 
\alpha = \frac{\ps{\VEC{a}}{\VEC{p}}}{\ps{\VEC{p}}{\VEC{p}}}
\quad \text{et} \quad
\beta = \frac{\ps{\VEC{a}}{\VEC{q}}}{\ps{\VEC{q}}{\VEC{q}}} \; .
\end{equation}
Donc
\[
\VEC{b} = \frac{\ps{\VEC{a}}{\VEC{p}}}{\ps{\VEC{p}}{\VEC{p}}} \VEC{p}
+ \frac{\ps{\VEC{a}}{\VEC{q}}}{\ps{\VEC{q}}{\VEC{q}}} \VEC{q} \; .
\]

Il est maintenant facile de calculer la plus courte distance entre le point
$\VEC{a}$ et le plan $\cal M$; c'est-à-dire, la distance entre les
points $\VEC{a}$ et $\VEC{b}$.  Il suffit de calculer la longueur du
vecteur $\VEC{a}-\VEC{b}$.

% Pour ceux intéressés à une formule générale, nous notons que
% \begin{align*}
% \|\VEC{a}-\VEC{b}\|^2
% &= \ps{\VEC{a} - \alpha \VEC{p} - \beta \VEC{q}}{\VEC{a} - \alpha
% \VEC{p} - \beta \VEC{q}} \\
% &= \ps{\VEC{a}}{\VEC{a}} - 2 \alpha \ps{\VEC{a}}{\VEC{p}}
% -2 \beta \ps{\VEC{a}}{\VEC{q}} + \alpha^2 \ps{\VEC{p}}{\VEC{p}}
% + \beta^2 \ps{\VEC{q}}{\VEC{q}} \; .
% \end{align*}
% Si nous substituons les expressions pour $\alpha$ et $\beta$ que nous
% retrouvons en (\ref{expr_x_y}) dans cette équation, nous obtenons
% \begin{align*}
% \|\VEC{a}-\VEC{b}\|^2
% &= \ps{\VEC{a}}{\VEC{a}} - 2
% \left(\frac{\ps{\VEC{a}}{\VEC{p}}}{\ps{\VEC{p}}{\VEC{p}}}\right) 
% \ps{\VEC{a}}{\VEC{p}}
% -2 \left(\frac{\ps{\VEC{a}}{\VEC{q}}}{\ps{\VEC{q}}{\VEC{q}}}\right)
% \ps{\VEC{a}}{\VEC{q}} \\
% &\qquad + \left(\frac{\ps{\VEC{a}}{\VEC{p}}}{\ps{\VEC{p}}{\VEC{p}}}\right)^2
%  \ps{\VEC{p}}{\VEC{p}}
% + \left(\frac{\ps{\VEC{a}}{\VEC{q}}}{\ps{\VEC{q}}{\VEC{q}}}\right)^2
% \ps{\VEC{q}}{\VEC{q}} \\
% &= \frac{
% \ps{\VEC{a}}{\VEC{a}}\ps{\VEC{p}}{\VEC{p}}\ps{\VEC{q}}{\VEC{q}}
% -\left(\ps{\VEC{a}}{\VEC{p}}\right)^2\ps{\VEC{q}}{\VEC{q}}
% -\left(\ps{\VEC{a}}{\VEC{q}}\right)^2\ps{\VEC{p}}{\VEC{p}} }
% {\ps{\VEC{p}}{\VEC{p}}\ps{\VEC{q}}{\VEC{q}}} \; .
% \end{align*}

Pour résumer, nous avons le résultat suivant.

\begin{defn} \index{Projection orthogonale}
Soit $\cal M$ un plan qui contient deux vecteurs perpendiculaires
$\VEC{p}$ et $\VEC{q}$.  Si $\VEC{a}$ un point quelconque, la
{\bfseries projection (orthogonale)} du vecteur $\VEC{a}$ dans le plan
$\cal M$ est le vecteur
\[
\VEC{b} =  \frac{\ps{\VEC{a}}{\VEC{p}}}{\ps{\VEC{p}}{\VEC{p}}} \VEC{p}
+ \frac{\ps{\VEC{a}}{\VEC{q}}}{\ps{\VEC{q}}{\VEC{q}}} \VEC{q} \; .
\]
Le point $\VEC{b}$ est le point du plan $\cal M$ qui est le plus près
de $\VEC{a}$.  Le vecteur $\VEC{a}-\VEC{b}$ est perpendiculaire au
plan $\cal M$.  La plus courte distance entre $\VEC{a}$ et le plan
$\cal M$ est la longueur du vecteur $\VEC{a}-\VEC{b}$.
\end{defn}

}  % End of theory

\section{Exercices}

\subsection{Équation d'une droite}

\begin{question}
Donnez une représentation de la droite qui passe par les points
$(2,-5,5)$ et $(-4,3,4)$.
\label{11Q1}\end{question}

\subsection{Équation d'un plan}

\begin{question}
Trouvez l'équation du plan parallèle au plan $2x_1 + 3x_2 - x_3 = 1$ et qui
passe par le point $(2,3,2)$.
\label{11Q2}
\end{question}

\begin{question}
Trouvez l'équation du plan $\cal M$ qui contient la droite $\ell$
définie par
\[
  \frac{x_1-2}{2} = \frac{x_2+3}{3} = \frac{x_3-3}{-2}
\]
et le point $(1,2,1)$.
\label{11Q3}
\end{question}

\begin{question}
Déterminez si les deux plans suivants se coupent et, si c'est le cas,
trouvez l'intersection des deux plans.
\[
 2x_1 + 5x_3  + 3 = 0 \qquad \text{et}  \qquad x_1 - 3x_2 + x_3 + 2 = 0 \ .
\]
\label{11Q4}
\end{question}

%%% Local Variables: 
%%% mode: latex
%%% TeX-master: "notes"
%%% End:

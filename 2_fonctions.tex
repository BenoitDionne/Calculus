\chapter[Fonctions]{Fonctions}\label{ChapFonct}

\compileTHEO{

Ce chapitre présente les objets fondamentaux sur lesquels nous allons
travailler dans les prochains chapitres.  Ces objets sont les
{\em fonctions}.  Les principales propriétés des fonctions sont aussi
définies dans ce chapitre.  Nous terminons le chapitre avec une
introduction à quelques unes des fonctions de base: les fonctions
trigonométriques, exponentielles et logarithmiques.  Les fonctions
exponentielles sont définies de façon intuitive dans ce chapitre.
Elles sont revues de façon plus rigoureuse au chapitre suivant.

\section{Qu'est-ce qu'une fonction?}

\begin{focus}{\dfn} \index{Fonction}
Une {\bfseries fonction} $f$ est une opération d'un
ensemble $X$ à un ensemble $Y$ qui, à chaque élément de $X$, associe
un seul élément de $Y$.  Nous écrivons $f:X\rightarrow Y$ pour désigner une
fonction $f$ de $X$ dans $Y$.
\end{focus}

\begin{egg}
À la figure~\ref{FUNCT1}, nous définissons à l'aide d'un diagramme une
fonction $f$ d'un ensemble $X$ de pays à un ensemble $Y$ de villes.
La fonction $f$ donne la capitale du pays.  À chacun des pays de $X$
est associé une seule ville dans $Y$ qui est sa capital.

Il n'est pas nécessaire que toutes les villes de $Y$ soient des
capitales de pays.  Par exemple, Toronto n'est pas la capitale d'un
pays mais d'une province.

Nous écrivons
\[
f(\text{Canada}) = \text{Ottawa} \quad , \quad f(\text{Angleterre}) =
\text{Londres} \quad , \quad \text{etc}.
\]
\label{CAPITALS}
\end{egg}

\PDFfig{2_fonctions/funct1}{Définition d'une fonction}{Définition d'une
  fonction $f$ à l'aide d'un diagramme}{FUNCT1}

\begin{egg}
Le tableau suivant définit une fonction $g$ qui, à chaque nombre
entier plus grand que $1$ (ligne du haut), associe les diviseurs
premiers de ce nombre entier (ligne du bas).
\[
\begin{array}{c|c|c|c|c|c|c|c|c|c}
x & 2 & 3 & \ldots & 6 & \ldots & 27 & \ldots & 1372455084 &
\ldots \\
\hline
g(x) & \{2\} & \{3\} & \ldots & \{2,3\} & \ldots & \{3\} &
\ldots & \{2,3,7,11,13,17,47\} & \ldots
\end{array}
\]

La fonction $g$ est donc une fonction qui va de l'ensemble $X$ des
nombres entiers plus grand que $1$ à l'ensemble $Y$ des ensembles de
nombres premiers.

Nous écrivons
\[
g(2) = \{2\} \quad , \quad g(1372455084) = \{2,3,7,11,13,17,47\}
\quad , \ldots
\]
\label{TABLE}
\end{egg}

\begin{egg}
La fonction $h$ qui suit est définie à l'aide d'une expression
algébrique.
\[
h(x) = x^3-2x+1 \; .
\]
La fonction $h$ est définie pour tout $x \in X = \RR$ (l'ensemble des
nombres réels) et donne une seule valeur $h(x) \in Y = \RR$.  Nous
disons alors que $h$ est une fonction de $\RR$ dans $\RR$ que nous
dénotons $h:\RR \rightarrow \RR$. 

Nous écrivons
\[
h(1) = 0 \quad , \quad h(2) = 5 \quad , \quad h(\pi)=
12.26706787812110\ldots \quad , \quad \ldots
\]
Pour la valeur $h(\pi)$, les points de suspension après le
dernier $0$ indiquent qu'il y a une infinité de chiffres qui
suivent.

Dans l'expression
\[
y=h(x)=x^3 -2x +1 \ ,
\]
la variable $x$ est appelée la {\bfseries variable indépendante}
\index{Fonction!variable indépendante} et la variable $y$ est appelée
la {\bfseries variable dépendante} \index{Fonction!variable dépendante}
car elle dépend de $x$.
\label{GRAPH_SET}
\end{egg}

\begin{focus}{\dfn} \index{Fonction!graphe}
Si $f$ est une fonction qui va d'un ensemble $X$ à un ensemble $Y$,
nous définissons le {\bfseries graphe de la fonction}
$f$ comme étant l'ensemble
\[
\text{graphe de }f = \left\{ (x,f(x)) : x \in X \right\} \; .
\]
\end{focus}

Pour une fonction qui va de $\RR$ dans $\RR$, il est plus fréquent de
représenter le graphe de cette fonction par la courbe du plan
cartésien qui est tracée par l'ensemble des points du graphe.  Le
graphe est donc un sous-ensemble du plan cartésien.

\begin{egg}
Nous retrouvons à la figure~\ref{FUNCT2} la représentation dans le plan
cartésien du graphe de la fonction $h$ définie à
l'exemple~\ref{GRAPH_SET}.
\end{egg}

\MATHfig{2_fonctions/funct2}{8cm}{Graphe de $h(x) = x^3-2x+1$}
{Graphe de $y= h(x) = x^3-2x+1$ pour $-2 \leq x \leq 2$.}{FUNCT2}

\section{Image et domaine d'une fonction}

\begin{focus}{\dfn} \index{Fonction!image}
L'{\bfseries image} d'une fonction $f$ qui va
d'un ensemble $X$ à un ensemble $Y$ est l'ensemble des éléments $y \in Y$
pour lesquelles il existe au moins un élément $x \in X$ tel que
$f(x) = y$.  Nous écrivons
\[
\IM f \equiv \left\{ y \in Y : f(x) = y \quad
\text{pour au moins un élément} \quad x \in X \right\} \; .
\]
\end{focus}

\begin{egg}
Pour l'exemple~\ref{CAPITALS}, l'image de $f$ est l'ensemble
\[
\IM f = \left\{ \text{Berlin, Londres, Paris, Ottawa} \right\} \; .
\]
\end{egg}

\begin{rmk}
Pour l'exemple~\ref{TABLE}, nous ne connaissons pas tous les éléments de
l'image de $g$ et il n'existe pas de formule pour les générer.
\end{rmk}

\begin{egg}
Pour l'exemple~\ref{GRAPH_SET}, l'image de $h$ est l'ensemble des
nombres réels comme nous pouvons le constater à partir du graphe de $h$ à
la figure~\ref{FUNCT2}.  Nous écrivons
\[
\IM h = \RR\; .
\]
\end{egg}

\begin{focus}{\dfn} \index{Fonction!domaine}
Le {\bfseries domaine} d'une fonction $f$ qui
va d'un ensemble $X$ à un ensemble $Y$ est l'ensemble $X$ sur lequel
la fonction $f$ est définie.  Nous écrivons
\[
\DO f = X \; .
\]
\end{focus}

\begin{egg}
Pour l'exemple~\ref{CAPITALS}, la fonction $f$ est définie pour tous
les pays de l'ensemble $X$.  Le domaine de $f$ est donc l'ensemble
$X$ au complet.
\[
\DO f = X = \left\{ \text{Allemagne, Angleterre, Canada, France}
\right\} \; .
\]
\end{egg}

\begin{egg}
La fonction $h$ de l'exemple~\ref{GRAPH_SET} est définie pour tous les
nombres réels.  Le domaine de $h$ est donc $\RR$.
\end{egg}

\begin{egg}
Le graphe de la fonction $h(x) = \sqrt{x}$ est donné à la
figure~\ref{SQRT}.  Nous avons que
\[
\DO h = \{ x \in \RR :  x\geq 0 \}
\qquad \text{ et } \qquad
\IM h = \{ y \in \RR : y \geq 0 \}\; .
\]
\end{egg}

\MATHfig{2_fonctions/squareroot}{8cm}{Graphe de $y= \sqrt{x}$}
{Graphe de $y= \sqrt{x}$ pour $0 \leq x \leq 3$}{SQRT}

\section{Composition de fonctions}

\begin{focus}{\dfn} \index{Fonction!composition}
Si $f$ est une fonction qui va de l'ensemble $X$ à l'ensemble $Y$ et
$g$ est une fonction qui va de l'ensemble $Y$ à l'ensemble $Z$, nous
pouvons définir une nouvelle fonction $g\circ f$ qui va aller de
l'ensemble $X$ à l'ensemble $Z$ en posant
\[
(g\circ f)(x) \equiv g(f(x))
\]
pour tous les éléments $x$ dans l'ensemble $X$.  La nouvelle
fonction $g\circ f$ est appelée la {\bfseries composition}
des fonctions $f$ et $g$.
\end{focus}

\begin{egg}
Deux fonctions sont définies par le diagramme à la figure~\ref{COMPO}.
La fonction $f$ donne la ville où demeure chacune des personnes de
l'ensemble $X$ et la fonction $g$ donne la province où se situe
chacune des villes de l'ensemble $Y$.  La composition $g\circ f$ est
donc la fonction qui donne la province où demeure chacune des
personnes de l'ensemble $X$.  Ainsi, nous obtenons
\begin{align*}
(g\circ f)(\text{Françoise}) &= g(f(\text{Françoise})) =
g(\text{Montréal}) = \text{Québec}
\intertext{et}                               
(g\circ f)(\text{Patrice}) &= g(f(\text{Patrice})) = g(\text{Winnipeg})
= \text{Manitoba} \; .
\end{align*}
\label{F_COMPO}
\end{egg}

\PDFfig{2_fonctions/composition}{Composition de fonctions}
{Définition des fonctions $f$ et $g$ pour l'exemple~\ref{F_COMPO}}{COMPO}

\begin{egg}
Soit $F$ la fonction qui donne pour chaque personne sa mère biologique
et soit $M$ la fonction qui donne pour chaque personne son père
biologique.

La composition des fonctions $F$ et $M$, dénotée $F \circ M$, est la
fonction qui donne pour chaque personne la mère biologique du père
biologique de cette personne (une des grand-mères de la personne).
Rappelons que dans la composition $F \circ M$,
{\em la fonction $M$ est exécutée en premier et la fonction $F$ en
second}.  La fonction $M$ donne le père biologique de la personne
donnée initialement.  Puis, la fonction $F$ avec comme argument le
père biologique de la personne donnée initialement donne la mère
biologique de ce père.

Quel est le résultat de la composition $F \circ M \circ F$?
\label{PEOPLES}
\end{egg}

\begin{egg}
Regardons un exemple de fonctions dont le domaine et l'image font
parties des nombres réels.  Plus précisément, considérons les
fonctions $f(x) = x^2+3$ et $g(x) = 3 x - 2$.  Ainsi,
\begin{align*}
(f\circ g)(x) &\equiv f(g(x)) = f(3x - 2) = (3x-2)^2 + 3
= 9 x^2 - 12 x + 7
\intertext{et}
(g\circ f)(x) &\equiv g(f(x)) = g(x^2+3) = 3(x^2+3) - 2
= 3 x^2 + 7 \; .
\end{align*}
Ces compositions donnent bien des fonctions qui sont différentes
du produit $f g$ qui est
\[
(f g)(x) \equiv f(x) g(x) = (x^2+3)(3 x-2) = 3 x^3  -2 x^2 + 9x -6 \; .
\]
\end{egg}

\MATHfig{2_fonctions/compog}{8cm}{Graphe de $g(x) = x^2 -1$}
{Graphe de $y= g(x) = x^2 -1$ pour $-2 \leq x \leq 2$}{COMPOG}

\MATHfig{2_fonctions/compohg}{8cm}{Graphe de $y= h(g(x))$}
{Graphe de $y= h(g(x)) = \sqrt{x^2 -1}$ pour $1 \leq |x| \leq 3$}{COMPOHG}

\begin{egg}
Nous pouvons composer les fonctions $g(x) = x^2 - 1$
(figure~\ref{COMPOG}) et $h(x) = \sqrt{x}$ pour obtenir de nouvelles
fonctions.  Le domaine de $g$ est $\RR$ et
l'image de $g$ est $\{ y \in \RR : y \geq -1 \}$.  De plus, le domaine
de $h$ est $\{ x \in \RR : x \geq 0 \}$ et l'image de $h$ est
$\{ y \in \RR : y \geq 0 \}$.

Pour définir la composition $h \circ g$, il faut avoir
\begin{equation} \label{funct_domimg1}
\IM g \subset \DO h \; .
\end{equation}
Ce qui n'est pas le cas présentement.  Il faut restreindre $g$
à l'ensemble des $x$ tels que $x^2-1 \geq 0$ pour satisfaire
(\ref{funct_domimg1}).  Avec la restriction $|x|\geq 1$, nous avons
maintenant que
\begin{equation} \label{funct_domimg2}
\IM g = \{ x : x\geq 0\} = \DO h
\end{equation}
et la composition $h\circ g$ est définie par
\[
(h\circ g)(x) = h(g(x)) = h(x^2-1) = \sqrt{x^2-1} \quad \text{pour}
\quad |x|\geq 1 \; .
\]
Il découle de (\ref{funct_domimg2}) que l'image de $h\circ g$ est
l'image de $h$ car l'image de $g$ est tout le domaine de $h$.  Le
domaine de $h \circ g$ est $\{ x \in \RR : |x| \geq 1 \}$ et l'image
de $h\circ g$ est $\{ y \in \RR : y \geq 0 \}$.  Le
graphe de $h\circ g$ est donné à la figure~\ref{COMPOHG}.

Puisque
\[
\IM h = \{ y | y \geq 0 \} \subset \RR = \DO g \; ,
\]
La fonction $g\circ h$ est définie par
\[
(g\circ h)(x) = g(h(x)) = g(\sqrt{x}) = \left(\sqrt{x}\right)^2-1 = x
- 1 \quad \text{pour} \quad x\geq 0 \; .
\]
Le domaine de $g\circ h$ est déterminé par le domaine de la racine
carrée.  Puisque $h$ peut atteindre toutes les valeurs réelles plus
grandes ou égales à zéro, nous avons que l'image de $g\circ h$ est
l'image de $g$.  Ainsi,
\[
\DO g\circ h = \{ x \in \RR : x \geq 0 \} \quad
\text{et} \quad \IM g\circ h = \{ y \in \RR : y \geq -1 \} \;.
\]
Le graphe de $g\circ h$ est donné à la figure~\ref{COMPOGH}.
\end{egg}

\MATHfig{2_fonctions/compogh}{8cm}{Graphe de $y= g(h(x))$}
{Graphe de $y= g(h(x)) = x -1$ pour $0 \leq x \leq 2$}{COMPOGH}

\begin{rmk}
Il serait tentant de dire que
$\displaystyle (g\circ h)(x) = x - 1$ pour tout $x$ mais cela est vrai
seulement si $x\geq 0$ car la fonction $h$ n'est pas définie pour les
nombres négatifs et nous ne pouvons donc pas définir $g(h(x))$ pour $x<0$;
l'expression $\left(\sqrt{x}\right)^2-1$ n'est pas définie pour $x<0$.

La fonction $f(x) = x -1$ n'est pas la fonction $g\circ h$.  Il est
vrai que $(g\circ h)(x) = f(x)$ pour $x \geq 0$ mais $f$ est définie
pour toutes les valeurs réelles de $x$ alors que $g\circ h$ ne l'est
pas.  Nous disons que $f$ est une {\bfseries extension}
\index{Fonction!extension} de la fonction $g\circ h$.  Il ne faut donc
pas utiliser l'extension d'une fonction pour déterminer le domaine et
l'image de celle ci car cela peut conduire à des erreurs.  Dans le cas
présent,
\[
\DO f = \RR \qquad \text{et} \qquad \IM f = \RR \; ,
\]
ce qui est différent du domaine et de l'image de $g\circ h$.
\end{rmk}

\section{Fonctions inverses (réciproques)}

Commençons par revoir la définition de l'inverse pour l'addition et
de l'inverse pour la multiplication, cela nous sera utile pour bien
comprendre ce qu'est l'inverse d'une fonction.

\subsection{Inverse additif d'un nombre réel}

Le nombre $0$ est {\bfseries l'élément neutre} \index{Élément neutre}
pour l'addition; c'est le nombre $y$ tel que $y + x = x + y = x$ pour
tout nombre réel $x$.

\begin{egg} Nous avons $5+0=0+5=5$, $27/4 + 0 = 0 + 27/4 = 27/4$ et
$\pi+0 = 0 + \pi = \pi$.
\end{egg}

{\bfseries L'inverse additif} \index{Inverse additif} 
d'un nombre réel $x$ est le nombre réel $z$ tel que $x+z = z+x =0$.

\begin{egg} L'inverse additif de $5$ est $-5$ car $5 + (-5) = 0$.  De même,
l'inverse additif de $\pi$ est $-\pi$ car $\pi + (-\pi) = 0$.
\end{egg}

\subsection{Inverse multiplicatif d'un nombre réel}

Le nombre $1$ est {\bfseries l'élément neutre} \index{Élément neutre}
pour la multiplication; c'est le nombre $y$ tel que $y \times x = x
\times y = x$ pour tout nombre réel $x$.

\begin{egg}
Nous avons $5\times 1= 1\times 5 = 5$,
$(27/4) \times 1 = 1 \times (27/4) = 27/4$
et $\pi \times 1 = 1 \times \pi = \pi$.
\end{egg}

{\bfseries L'inverse multiplicatif}\index{Inverse multiplicatif}
d'un nombre réel $x \neq 0$ est le nombre réel $z$ tel que
$x \times z = z \times x = 1$.

\begin{egg}
L'inverse multiplicatif de $5$ est $0.2$ car
$5 \times 0.2 = 0.2 \times 5 = 1$.  De même, l'inverse multiplicatif
de $27/4$ est $4/27$ car $(27/4) \times (4/27) = (4/27) \times (27/4) = 1$.
\end{egg}

\begin{rmk}
Le nombre $0$ n'a pas d'inverse multiplicatif car il n'existe pas de
nombre réel $y$ tel que $0 \times y = 1$.
\end{rmk}

\subsection{Inverse (réciproque) d'une fonction}

Avant de définir l'inverse d'une fonction, il faut bien comprendre que
l'opération pour laquelle nous voulons définir un inverse n'est pas
l'addition ou la multiplication de fonctions mais la composition de
fonctions.

L'addition et la multiplication de fonctions sont en fait des
opérations sur les éléments de l'image des fonctions (à valeurs
réelles).  La composition de fonctions est indépendante des opérations
d'addition et de multiplication des fonctions.

Comme nous l'avons fait pour l'inverse additif et l'inverse multiplicatif,
il faut définir l'élément neutre de la composition de fonctions.

\begin{focus}{\dfn}
La fonction qui joue le rôle {\bfseries d'élément neutre}
\index{Élément neutre} pour la composition de fonctions est la
{\bfseries fonction identité} \index{Fonction identité}, dénotée
$\Id$, qui est définie par $\Id(z) = z$ pour tout élément $z$ du
domaine.  Pour chaque élément $z$ du domaine, la fonction identité
redonne cet élément $z$.
\end{focus}

Si $f$ est une fonction d'un ensemble $X$ dans un ensemble $Y$, nous
pouvons facilement voir que
\[
f \circ \Id = f \qquad \text{et} \qquad \Id \circ f = f \; .
\]
C'est à dire, $(f\circ \Id)(x) = f(\Id(x)) = f(x)$ et
$(\Id\circ f)(x) = \Id(f(x))= f(x)$ pour tout $x$.

Remarquons la similarité avec la propriété de l'élément neutre pour
l'addition (i.e. $0 + x = x + 0 = x$ pour tout $x$) et de l'élément
neutre pour la multiplication (i.e. $1 \times x = x \times 1 = x$ pour
tout $x$).

\begin{egg}
La fonction $\Id$, qui à une personne redonne cette même personne, est
l'élément neutre pour la composition de fonctions qui agissent sur les
personnes comme les fonctions $F$ et $M$ de l'exemple~\ref{PEOPLES}.

Par exemple, nous avons $\Id \circ F = F$ car $F$ donne pour chaque
personne sa mère biologique et par la suite $\Id$ redonne cette mère.
\end{egg}

\begin{egg}
Pour la composition de fonctions dont le domaine et l'image sont des
sous-ensembles des nombres réels, l'élément neutre $\Id$ est la
fonction qui à chaque nombre réel $x$ redonne le nombre réel $x$.

Par exemple, si $p(x) = x^2 + 5x$, alors $\Id \circ p = p \circ \Id = p$
car $(\Id \circ p)(x) = \Id(p(x)) = p(x)$ et
$(p \circ \Id)(x) = p(\Id(x)) = p(x)$ pour tout nombre réel $x$.
\end{egg}

En s'inspirant de la définition de l'inverse pour l'addition et de
l'inverse pour la multiplication, nous définissons l'inverse pour la
composition de fonctions.

\begin{focus}{\dfn} \index{Fonction!inverse}
{\bfseries L'inverse d'une fonction} $f$ qui
va d'un ensemble $X$ à un ensemble $Y$ est la fonction $g$ qui va de
l'ensemble $Y$ à l'ensemble $X$ et qui satisfait
\[
f\circ g = \Id \qquad \text{et}  \qquad  g \circ f = \Id \; ,
\]
C'est-à-dire, $g(f(x)) = x$ pour tout $x \in X$ et $f(g(y)) = y$
pour tout $y\in Y$.  Nous dénotons par $f^{-1}$ la fonction $g$ qui est
l'inverse de $f$.
\end{focus}

Cette définition est équivalente à l'énoncé suivant qui est
souvent utilisé comme défini\-tion de l'inverse d'une fonction.

\begin{focus}{\dfn}
L'inverse de la fonction $f$ qui va d'un ensemble $X$ à un ensemble
$Y$ est la fonction $f^{-1}$ qui va de l'ensemble $Y$ à l'ensemble $X$
et qui satisfait
\[
x = f^{-1}(y) \qquad \text{si et seulement si} \qquad y = f(x) \; .
\]
\end{focus}

En effet, si $y=f(x)$ alors $f^{-1}(y) = f^{-1}(f(x)) = x$ et si
$x = f^{-1}(y)$ alors $f(x) = f(f^{-1}(y)) = y$ par définition de l'inverse
d'une fonction.

\begin{egg}
Nous retrouvons à la figure~\ref{INVERSE1} un diagramme qui définit
d'une fonction $f$ qui à chaque pays dans $X$ assigne sa capitale dans
$Y$.  La fonction inverse $f^{-1}$ est donc la fonction qui, à chaque
capitale dans $Y$, assigne le pays dont elle est la capitale dans $X$.

La fonction $f^{-1}$ est définie à la figure~\ref{INVERSE1} par les
flèches formées de tirets.
\label{FMOINS1}
\end{egg}

\PDFfig{2_fonctions/inverse1}{Définition des fonctions $f$ et $f^{-1}$}
{Définition des fonctions $f$ et $f^{-1}$ de l'exemple~\ref{FMOINS1}}
{INVERSE1}

\begin{egg}
L'inverse de la fonction $f(x) = x/3 + 1$ est la fonction $g(x) = 3x - 3$ car
\[
(g \circ f)(x) = g(f(x)) = g\left( \frac{1}{3}x + 1 \right)
= 3\left(\frac{1}{3}x+1\right) - 3 = x = \Id(x)
\]
pour tout $x$.  De même, nous pouvons vérifier que
$(f \circ g)(x) = f(g(x)) = \Id(x)$ pour tout $x$.

Ainsi, pour $f(x) = x/3 + 1$, nous obtenons $f^{-1}(x) = 3x + 3$.
\end{egg}

Nous avons vu que $0$ n'avait pas d'inverse multiplicatif mais que tous les
autres nombres réels avaient un inverse multiplicatif.  La situation
est encore plus complexe pour les fonctions car elles n'ont pas toutes
un inverse.

\begin{egg}
Par exemple, la fonction $F$ de l'exemple~\ref{PEOPLES} qui donne pour
chaque personne sa mère biologique n'a pas d'inverse.  Si nous donnons une
mère, alors nous ne pouvons pas déterminer uniquement la personne pour qui
elle est la mère sauf si cette mère a eu un seul enfant.
\end{egg}

\begin{rmk}
En Ontario, nous utilisons très fréquemment l'expression
{\bfseries la réciproque d'une fonction} pour désigner l'inverse
d'une fonction.  Ce n'est pas le cas dans tous les pays de la
francophonie.  Les anglophones utilisent l'inverse d'une fonction. 
\end{rmk}

\subsection{Comment déterminer si une fonction a un inverse}

Pour déterminer si une fonction $f$ (en tant que fonction de son
domaine à son image) a un inverse il faut vérifier que pour chaque
élément $y$ de son image il existe un seul et unique élément $x$ du
domaine de $f$ tel que la fonction $f$ évaluée à $x$ donne $y$
(i.e. $f(x)=y$).  Nous donnons un nom spéciale aux fonctions qui
possèdent cette propriété.

\begin{focus}{\dfn} \index{Fonction!injective}
Une fonction $f$ qui va d'un ensemble $X$ à un ensemble $Y$ est
{\bfseries injective} si $f(x_1) = f(x_2)$
implique que $x_1 = x_2$.
\end{focus}

\PDFfig{2_fonctions/diagram}{Fonction qui n'a pas d'inverse}
{Définition d'une fonction $f$ qui n'a pas d'inverse.  Cette fonction
est discutée à l'exemple~\ref{EX_DIAGR}}{DIAGR}

\begin{egg}
Une fonction $f$ est définie par le diagramme à la
figure~\ref{DIAGR}.  Cette fonction n'a pas d'inverse car les nombres
$3$ et $4$ donnent tous les deux la couleur noir.  Nous ne pouvons
donc pas définir une fonction inverse de $Y$ dans $X$ pour $f$; quelle
serait alors la valeur assignée à la couleur noir?
\label{EX_DIAGR}
\end{egg}

Une autre notion important dans le contexte générale de l'étude des
fonctions est la notion suivante.

\begin{focus}{\dfn} \index{Fonction!surjective}
Une fonction $f$ qui va d'un ensemble $X$ à un ensemble $Y$ est
{\bfseries surjective} si, pour tout
$y \in Y$, il existe au moins un $x \in X$ tel que $f(x) = y$.
\end{focus}

En d'autres mot, une fonction $f: X \to Y$ est surjective si
$\IM f = Y$; tous les éléments de $Y$ sont dans l'image de la fonction.
Ce concept jouera un rôle secondaire dans notre recherche de
fonctions inverses car nous considérerons seulement les fonctions
$f : X \to Y$ avec $Y = \IM f$.

Il y a plusieurs façons de vérifier si une fonction est injective;
c'est-à-dire que pour chaque valeur $y$ dans l'image d'une fonction
$f$ il existe une seule valeur $x$ dans le domaine de cette fonction
$f$ qui satisfasse $f(x) = y$.

\subsubsection{Méthode algébrique}

Illustrons cette méthode à l'aide d'un exemple.

\begin{egg}
Considérons la fonction $f(x) = x/3 +1$ que nous avons vu précédemment.
Supposons que $x_1$ et $x_2$ soient deux nombres tels que
$f(x_1) = f(x_2) = y$.  Le but est de montrer que $x_1 = x_2$ et que
nous avons en fait évalué la fonction $f$ au même point pour obtenir
$y$.

Nous obtenons de $f(x_1) = f(x_2)$ que $x_1/3  + 1 = x_2/3 +1$.
Après avoir soustrait $1$ de chaque côté de l'égalité, nous trouvons
$x_1/3 = x_2/3$.  Une multiplication par $3$ des deux cotés de cette
nouvelle égalité nous donne $x_1 = x_2$.  Ce que nous voulions
démontrer.
\end{egg}

\begin{egg}
Est-ce que la fonction $\displaystyle f(x) = \frac{1+3x}{5-2x}$ a un
inverse sur $\RR \setminus \{5/2\}$, son domaine?

Cela revient à montrer que $f$ est injective.  Supposons que
$f(x_1) = f(x_2)$.  Alors
\begin{align*}
f(x_1) = f(x_2) & \Leftrightarrow \frac{1+3x_1}{5-2x_1}
= \frac{1+3x_2}{5-2x_2} \\
&\Leftrightarrow (1+3x_1)(5-2x_2) = (1+3x_2)(5-2x_1) \\
&\Leftrightarrow 5+15x_1-2x_2 -6 x_1x_2 = 5+ 15 x_2 -2x_1 - 6x_1 x_2 \\
&\Leftrightarrow 15x_1-2x_2= 15 x_2 -2x_1 \\
&\Leftrightarrow 17x_1 = 17 x_2 \\
&\Leftrightarrow x_1 = x_2  \ .
\end{align*}
\label{invEGG}
\end{egg}

\subsubsection{Test de la droite horizontale}

Pour ce test, nous utiliserons le graphe de la fonction $f$ et nous
vérifierons que chaque droite horizontale (i.e.\ $y$ est constant) coupe
le graphe de la fonction $f$ en un seul point.  L'abscisse $x$ de ce
point est la seule valeur du domaine de $f$ telle que $f(x) = y$.

\begin{egg}
À la figure~\ref{GRAPH}, nous avons tracé le graphe de $f(x) = x/3 +1$ et
quelques droites horizontales pour nous convaincre que chaque droite
horizontale coupe le graphe de la fonction en un seul point.  Ce qui
support le fait que la fonction $f$ est injective.
\end{egg}

\PDFfig{2_fonctions/one2one}{Graphe de $f(x) = x/3 +1$}
{Le graphe de $\displaystyle f(x) = x/3 +1$ et de quelques droites
horizontales}{GRAPH}

\subsection{Comment trouver l'inverse d'une fonction}

Après avoir vérifié qu'une fonction est injective et donc qu'un
inverse existe, nous pouvons chercher son inverse.

\subsubsection{Méthode algébrique}

Si $y=f(x)$, il faut résoudre pour $x$ en fonction de $y$.  L'exemple
suivant va illustrer cette méthode.

\begin{egg}
Dans le cas simple de la fonction $f(x) = x/3 +1$, nous posons
$y = f(x) = x/3 +1$ et nous résolvons pour $x$.  Nous soustrayons $1$
de chaque côté de l'égalité précédente pour obtenir $y - 1 =  x/3$.
Puis nous multiplions les deux côtés de cette nouvelle égalité par $3$
pour obtenir $x = 3y - 3$.  Donc $x=f^{-1}(y) = 3y-3$.

Par tradition, nous utilisons $y$ pour la variable dépendante et
$x$ pour la variable indépendante.  Nous échangeons donc $x$ et $y$ pour
obtenir $y = f^{-1}(x) = 3 x - 3$ pour $x\in \RR$.
\end{egg}

\begin{egg}
Même si $f$ est donnée par un simple polynôme, il n'est pas toujours
possible de trouver une expression algébrique pour l'inverse de $f$.

Par exemple, soit $f(x) = x^6 + x^2 + 3$.  Il est facile de tracer le
graphe de $f$ et de remarquer que $f$ satisfait le test de la droite
horizontale.  Nous sommes donc certain que l'inverse de $f$ existe.

Cependant, nous ne pouvons résoudre l'équation $y=x^6+x^2+3$ pour $x$
en fonction de $y$ afin d'obtenir $x = f^{-1}(y)$.  En fait, il a été
démontré par Evariste Galois, un mathématicien français du $18^e$
siècle, qu'il n'existe pas de formule générale pour trouver les
racines d'un polynôme de degré plus grand que quatre comme c'est le
cas pour les polynômes de degré deux.
\end{egg}

\begin{egg}
À l'exemple~\ref{invEGG}, nous avons montré que la fonction
$\displaystyle f(x) = \frac{1+3x}{5-2x}$ était injective.  Il existe
donc un inverse de $f$ en tant que fonction définie sur
$\RR\setminus \{5/2\}$.  Trouvons cet inverse.

Puisque
\begin{align*}
y = f(x) = \frac{1+3x}{5-2x} &
\Leftrightarrow y(5-2x) = 1+3x \Leftrightarrow 5y -2xy = 1 + 3x \\
&\Leftrightarrow 5y -1 = 2xy + 3x = (2y+3)x
\Leftrightarrow x = \frac{5y -1}{2y+3} \ ,
\end{align*}
nous avons $\displaystyle x=f^{-1}(y) = \frac{5y -1}{2y+3}$.  Comme par
tradition nous utilisions $y$ pour la variable dépendante et $x$
pour la variable indépendante, nous échangeons $x$ et $y$ pour obtenir
$\displaystyle y=f^{-1}(x) = \frac{5x -1}{2x+3}$.  La fonction
$f^{-1}$ a comme domaine l'image de la fonction $f$ qui est
$\RR \setminus \{-3/2\}$.
\end{egg}

\subsubsection{Méthode graphique}

\begin{focus}{\mth}
Si $f$ est une fonction qui possède une fonction inverse $f^{-1}$,
alors nous obtenons le graphe de $f^{-1}$ en faisant la réflexion du
graphe de $f$ par rapport à la droite $y=x$.
\end{focus}

\PDFfig{2_fonctions/inverse}{Graphe de $f^{-1}$ obtenu par réflexion du
graphe de $f$}{Graphe de $f^{-1}$ obtenu par réflexion du
graphe de $f$ par rapport à la droite $y=x$}{INVERSE} 

\begin{rmk}[\theory]
Pour justifier cet énoncé, nous utiliserons le dessin à la
figure~\ref{INVERSE}.
Nous fixons $z$ et cherchons les coordonnées du point $B$ qui est
le point symétrique au point $A = (z,f(z))$ par rapport à la droite
$y=x$.  Nous voulons montrer que les coordonnées du point $B$ sont
$(f(z),z)$, un point du graphe de $f^{-1}$ car $z = f^{-1}(f(z))$.

Si $B$ est le point qui est symétrique au point $A$ par rapport à la
droite $y=x$, alors les angles $\angle AEC$ et $\angle BEC$ sont des
angles droits et les segments $\sgm{AE}$ et $\sgm{BE}$ sont de même
longueur.

Nous traçons la droite horizontale $y = f(z)$ qui coupe la droite $y=x$ en
$C$.  Puisque les angles $\angle AEC$ et $\angle BEC$ sont égaux et les
côtés $\sgm{AE}$ et $\sgm{CE}$ adjacents à l'angle $\angle AEC$ sont
respectivement de même longueur que les côtés $\sgm{BE}$ et $\sgm{CE}$
adjacents à l'angle $\angle BEC$, il en découle que les triangles
$\triangle ACE$ et $\triangle BCE$ sont congruents.  Ainsi les
segments $\sgm{AC}$ et $\sgm{CB}$ sont de même longueur et les angles
$\angle ACE$ et $\angle BCE$ sont égaux.

Puisque la droite $y=x$ fait un angle de $\pi/4$ avec l'axe des $x$,
nous avons que $\angle ACE = \pi/4$.  Il s'en suit que $\angle ACB = \pi/2$.
La droite qui passe par $B$ et $C$ est donc verticale.  Comme les
coordonnées du point $C$ sont $(f(z),f(z))$, nous avons donc que
l'abscisse du point $B$ est $f(z)$.

Le point d'intersection de la droite qui passe par $B$ et $C$ avec
l'axe des $x$ est $F=(f(z),0)$.  Pour trouver l'ordonnée du point $B$,
nous utilisons le fait que la longueur du segment $\sgm{BC}$, qui est
aussi la longueur du segment $\sgm{AC}$, est $z-f(z)$.
L'ordonnée du point $B$ est donc la somme des longueurs des segments
$\sgm{FC}$ et $\sgm{CB}$.  C'est-à-dire que l'ordonnée du point $B$
est $f(z) + (z - f(z)) = z$.  Ce qui confirme que les coordonnées de
$B$ sont $(f(z),z)$.
\end{rmk}

\subsection[Influence du domaine et de l'image]{Influence du domaine
  et de l'image d'une fonction sur la définition de son inverse}

Une question que nous n'avons pas abordée précédemment est l'influence du
domaine et de l'image d'une fonction sur la définition de son
inverse.

\begin{egg}
Considérons la fonction $f(x) = x^2 -1$ pour toutes les valeurs
réelles de $x$.  Cette fonction n'a pas d'inverse car elle n'est pas
injective.  Par exemple, $f(-1) = f(1) = 0$.  Il est aussi facile de
voir à partir du graphe de $f$ que cette fonction ne satisfait pas le
test de la droite horizontale.

Par contre, si nous considèrons $f(x) = x^2-1$ pour $x\geq 0$ seulement,
alors $f$ a un inverse qui est $f^{-1}(x) = \sqrt{x+1}$ pour
$x\geq -1$.  Nous retrouvons à la figure~\ref{INVERSE2} les graphes de
$f$ et $f^{-1}$.

Utilisons la méthode algébrique pour trouver l'inverse de
$f(x) = x^2-1$ pour $x\geq 0$.  Nous additionnons $1$ de chaque
côté de $y = f(x) = x^2 -1$ pour obtenir $y + 1 =  x^2$.
Puis nous prenons la racine carrée des deux côtés de cette dernière
égalité pour obtenir $x = \sqrt{y + 1}$.
Donc $x = f^{-1}(y) = \sqrt{y+1}$ pour $y \geq -1$.

Par tradition, nous échangeons $x$ et $y$ pour obtenir
\[
y = f^{-1}(x) = \sqrt{x + 1} \quad \text{pour} \quad x\geq -1 \; .
\]

Sans l'hypothèse que $x\geq 0$, nous aurions obtenu
$\displaystyle x = \pm \sqrt{y + 1}$.  Or, cette formule ne peut pas
définir une fonction car nous avons deux valeurs pour chaque valeur de $y$.
\end{egg}

\PDFfig{2_fonctions/inverse2}{Graphes de $f(x) = x^2 -1$ et $f^{-1}$}
{Graphes de $f(x) = x^2 -1$ pour $x\geq 0$ et de
$f^{-1}(x) = \sqrt{x+1}$ pour $x\geq -1$.}{INVERSE2}

\section{Fonctions trigonométriques \life \eng}

Vous avez probablement vu la définition du cosinus et sinus d'un angle
à partir d'un triangle droit.  Si $\triangle ABC$ est un triangle avec
un angle droit au sommet $C$ et $\theta$ est l'angle au sommet $A$
(figure~\ref{TRIG1}), alors le cosinus et le sinus de $\theta$
sont définis par
\[
\cos(\theta) = \frac{|\sgm{AC}|}{|\sgm{AB}|}
\qquad \text {et} \qquad
\sin(\theta) = \frac{|\sgm{BC}|}{|\sgm{AB}|}
\]
où $|\sgm{AC}|$, $|\sgm{BC}|$ et $|\sgm{AB}|$ sont les longueurs des
segments $\sgm{AC}$, $\sgm{BC}$ et $\sgm{AB}$ respectivement.  Cette
définition est excellente pour les angles aiguës (moins de
$90^\circ$) mais comment définir le cosinus ou sinus d'un angle obtus
(plus de $90^\circ$).

\PDFfig{2_fonctions/trig0}{Définition du sinus et cosinus d'un angle à
partir du cercle unité}{Définition du sinus et cosinus de l'angle
$\theta$ à partir du cercle unité.}{TRIG0}

Une autre façon de définir le
{\bfseries cosinus et sinus d'un angle en radians} est avec le cercle
unité.

\begin{focus}{\dfn}
Soit $D$, l'intersection du cercle de rayon $1$ centré à l'origine
avec la droite émanant de l'origine et qui forme un angle $\theta$
avec l'axe des $x$ lorsque nous nous déplaçons dans le sens contraire aux
aiguilles d'une montre (figure~\ref{TRIG0}).  L'abscisse du
point $D$ est $\cos(\theta)$\index{Cosinus} et l'ordonnée du point $D$
est $\sin(\theta)$\index{Sinus}.
\end{focus}

\PDFfig{2_fonctions/trig1}{Définition du sinus et cosinus d'un angle à
partir d'un triangle}{Définition du sinus et cosinus de l'angle
$\theta$ à partir d'un triangle.}{TRIG1}

À partir de maintenant, au lieu de calculer les angles en degrés, nous
calculons les angles en radians.  N'oubliez pas que le nombre $\pi$
est le rapport de la circonférence d'un cercle sur son diamètre.
Ainsi,  $360^\circ$ correspond à $2\pi$ radians, la circonférence d'un
cercle de rayon $1$.

Les deux définitions du cosinus et sinus que nous venons de donner
sont équivalentes.  Considérons le dessin à la figure~\ref{TRIG1}.
Si le sommet $A$ du triangle $\triangle ABC$ est à l'origine,
$\sgm{AC}$ repose sur l'axe des $x$, $D$ est le point d'intersection
de la droite contenant le segment $\sgm{AB}$ avec le cercle de rayon
$1$ centré à l'origine, et $E$ est le point d'intersection de la
droite perpendiculaire à $\sgm{AC}$ passant par $D$ avec l'axe des
$x$, alors les triangles $\triangle ABC$ et $\triangle ADE$ sont
semblables.  Ainsi,
\[
\cos(\theta) = \frac{|\sgm{AC}|}{|\sgm{AB}|} = \frac{|\sgm{AE}|}{|\sgm{AD}|} =
|\sgm{AE}|
\qquad \text{ et } \qquad
\sin(\theta) = \frac{|\sgm{BC}|}{|\sgm{AB}|} = \frac{|\sgm{DE}|}{|\sgm{AD}|} =
|\sgm{DE}|
\]
car $|\sgm{AD}|=1$.

Il faut bien comprendre que l'angle positif $\theta$ représenté dans les
dessins des figures~\ref{TRIG0} et \ref{TRIG1} est l'angle mesuré lorsque nous
nous déplaçons dans le sens contraire aux aiguilles d'une montre à
partir de l'axe des $x$.

Le Tableau~\ref{BAS_VAL} donne quelques valeurs du sinus et cosinus
qu'il faut mémoriser.  Nous verrons plus tard que nous pouvons
utiliser les identités trigonométriques pour trouver d'autres valeurs
du sinus et cosinus.

\begin{table}
\begin{center}
\begin{tabular}{c|c|c|c|c|c}
$\theta$ & $0$ & $\pi/6$ & $\pi/4$ & $\pi/3$ & $\pi/2$ \\
\hline
$\cos(\theta)$ & $1$ & $\sqrt{3}/2$ &
$\sqrt{2}/2$ & $1/2$ & $0$ \\
\hline
$\sin(\theta)$ & $0$ & $1/2$ &
$\sqrt{2}/2$ & $\sqrt{3}/2$ & $1$
\end{tabular}
\end{center}
\caption{Valeurs de $\cos(\theta)$ et $\sin(\theta)$ pour quelques
valeurs de $\theta$. \label{BAS_VAL}}
\end{table} 

Soit $\triangle ABC$ le triangle avec un angle droit au point $C$ et
un angle de $\theta$ radians au sommet $A$ (figure~\ref{TRIG1}).
Pour $\theta \neq n\pi + \pi/2$ où $n \in \ZZ$, la tangente de l'angle
$\theta$ est définie par
\[
\tan(\theta) = \frac{|\sgm{BC}|}{|\sgm{AC}|} \; .
\]

Nous pouvons aussi définir la tangente de l'angle $\theta$ à l'aide du
cercle unité.  Si nous utilisons le triangle $\triangle ABC$
représenté à la figure~\ref{TRIG1}, alors la
{\bfseries tangente de l'angle $\theta$}\index{Tangent} est  
\[
\tan(\theta) = \frac{|\sgm{BC}|}{|\sgm{AC}|}
= \frac{|\sgm{DE}|}{|\sgm{AE}|}
= \frac{\sin(\theta)}{\cos(\theta)}
\quad , \quad \theta \neq \frac{\pi}{2} + n\pi \ \text{pour} \ n\in \ZZ \; .
\]

Il ne faut pas oublier que l'angle positif $\theta$ est l'angle mesuré
lorsque nous nous déplaçons dans le sens contraire aux aiguilles d'une
montre à partir de l'axe des $x$.

Il y a trois autres fonctions trigonométriques qui peuvent être utiles
de temps à autre.  Nous donnons leurs définitions à partir du triangle
$\triangle ABC$ représenté à la figure~\ref{TRIG1}.

La {\bfseries cotangente d'un angle $\theta$}\index{Cotangente} est
définie par la relation
\[
\cot(\theta) = \frac{|\overline{AC}|}{|\overline{BC}|} 
= \frac{|\overline{AE}|}{|\overline{DE}|} 
= \frac{\cos(\theta)}{\sin(\theta)} \ , \ \theta \neq
n \pi \ \text{pour} \ n\in \ZZ \; .
\]
La {\bfseries sécante d'un angle $\theta$}\index{Sécante} est définie
par la relation
\[
\sec(\theta) = \frac{|\overline{AB}|}{|\overline{AC}|} 
= \frac{|\overline{AD}|}{|\overline{AE}|} 
= \frac{1}{\cos(\theta)} \ , \ \theta \neq
\frac{\pi}{2} + n \pi \ \text{pour} \ n\in \ZZ \; .
\]
Finalement, la {\bfseries cosécante d'un angle $\theta$}\index{Cosécante}
est définie par la relation
\[
\csc(\theta) = \frac{|\overline{AB}|}{|\overline{BC}|} 
= \frac{|\overline{AD}|}{|\overline{DE}|} 
= \frac{1}{\sin(\theta)} \ , \ \theta \neq
n \pi \ \text{pour} \ n\in \ZZ \, .
\]

\begin{rmk}
Notons que  $\cos(\theta) \neq \cos(\theta)$.

La fonction $\cos(x)$ définie dans le cours de
{\bfseries Fonctions} de $11^e$ année en Ontario assume
que $x$ est mesuré en degrés.  Alors que la fonction $\cos(x)$
définie dans le cours de {\bfseries Fonctions Avancées} de $12^e$
assume que $x$ est mesuré en radians.

La même notation a été utilisée pour dénoter deux différentes
fonctions.  Par exemple, $\cos(10) \neq \cos(10)$ si l'angle du
premier cosinus est $10$ degrés et celui du deuxième cosinus est $10$
radians.  Il est aussi mélangeant d'écrire $\displaystyle \cos(10) =
\cos\left( 10 (\pi/180)\right)$; il faut comprendre que
l'angle du premier cosinus est $10$ degrés alors que celui du deuxième
cosinus est $\displaystyle 10 (\pi/180)$ radians.

Si nous dénotons par $\cos_{d}$ le cosinus où l'angle est mesuré en
degrés et par $\cos$ le cosinus où l'angle est mesuré en radians,
alors il est maintenant clair que $\cos_{d}(10) \neq \cos(10)$ et
$\displaystyle \cos_{d}(10) = \cos\left( 10 (\pi/180) \right)$.

Il aurait été préférable de donner des noms différents au cosinus pour
les angles en degrés et au cosinus pour les angles en radians.
Malheureusement, la tradition veut que nous utilisions le même nom
(i.e. $\cos$) dans les deux cas.

À moins d'avis contraire, l'argument des fonctions trigonométriques
sera toujours mesuré en radians.  Il sera très clairement spécifié si
jamais nous devons utiliser les fonctions trigonométriques où les angles
sont mesurés en degrés.

Cette nuance entre le cosinus où les angles sont mesurés en degrés et
celui où les angles sont mesurés en radians aura des conséquences très
importante lors de l'étude du calcul différentiel et intégral pour les
fonctions trigonométriques.

La remarque précédente au sujet du cosinus est aussi valable pour les
autres fonctions trigonométriques.
\end{rmk}

\begin{rmk}
Remarquons que $\sin \theta \neq \theta \sin$.  En fait, cette
dernière expression n'a aucun sens.  Il faut noter qu'un grand nombre
d'auteurs utilisent la notation  $\sin \theta$ pour désigner
$\sin(\theta)$.  C'est donc le sinus qui est évalué à $\theta$ et non
pas le produit du sinus par le monôme $\theta$.

L'utilisation de $\sin \theta$, $\cos \theta$ et $\tan \theta$ pour
désigner $\sin(\theta)$, $\cos(\theta)$ et $\tan(\theta)$
respectivement est une tradition que nous éviterons dans le présent
document.  De cette façon, nous ne risquerons pas de retrouver une
expression du genre $\sin \theta + 1$. Est-ce $\sin(\theta+1)$ ou
$\sin(\theta) + 1$?
\end{rmk}

\subsection{Identités trigonométriques}

Puisque $\cos(\theta)$ et $\sin(\theta)$ sont les coordonnés d'un
point du cercle de rayon $1$ centré à l'origine, nous obtenons le
résultat suivant.

\begin{focus}{\prp}
\[
\cos^2(\theta) + \sin^2(\theta) = 1 \quad \text{pour} \quad \theta \in
\RR \ .
\]
\end{focus}

Puisque $\theta$ et $\theta+2\pi$ représentent le même point du cercle
de rayon $1$ centré à l'origine, nous obtenons le résultat suivant.

\begin{focus}{\prp}
\[
\cos(\theta + 2\pi) = \cos(\theta) \quad \text{et} \quad
\sin(\theta + 2\pi) = \sin(\theta) \quad \text{pour} \quad \theta \in \RR \; .
\]
\end{focus}

Nous disons que le cosinus et le sinus sont des fonctions
{\bfseries périodiques} de {\bfseries période} $2\pi$\index{Période}.
Il s'en suit que nous avons les mêmes égalités si $2\pi$ est remplacé
par $-2\pi$, $4\pi$, $-4\pi$, etc.

Il est facile de déduire plusieurs identités trigonométriques à partir
du cercle unité.

Puisque les triangles $\triangle 0BE$, $\triangle OCG$,
$\triangle ODG$ et $\triangle OAE$ représentés à la figure~\ref{TRIG4} sont
congruents, nous obtenons les identités suivantes.

\begin{focus}{\prp}
\[
\begin{array}{lcl}
\cos(\theta) = \cos(-\theta) ,&& \sin(\theta) = -\sin(-\theta), \\
\cos(\theta) = -\cos(\pi - \theta) ,&& \sin(\theta) = \sin(\pi - \theta),\\
\cos(\theta) = -\cos(\pi + \theta) & \quad\text{et}\quad &
\sin(\theta) = -\sin(\pi+\theta)
\end{array}
\]
pour $\theta \in \RR$. \label{fonct_ident_trigsA}
\end{focus}

Puisque les triangles $\triangle 0AB$ et $\triangle ODC$ représentés à la
figure~\ref{TRIG5} sont congruents, nous obtenons les identités
suivantes.

\begin{focus}{\prp}
\[
\cos(\theta) = \sin\left(\frac{\pi}{2} - \theta\right)
\quad \text{et} \quad
\sin(\theta) = \cos\left(\frac{\pi}{2} - \theta\right)
\quad \text{pour} \quad \theta \in \RR \ .
\]
\label{fonct_ident_trigsB}
\end{focus}

Ce sont seulement quelques unes des identités trigonométriques que
nous pouvons déduire à partir du cercle unité.

\PDFfig{2_fonctions/trig4}{Identités trigonométriques provenant de
  réflexions par rapport aux axes}{Identités trigonométriques
  provenant de réflexions par rapport aux axes}{TRIG4} 

\PDFfig{2_fonctions/trig5}{Identités trigonométriques provenant d'une
  réflexion par rapport à la droite $y=x$}{Identités trigonométriques
  provenant d'une réflexion par rapport à la droite $y=x$}{TRIG5}

Finalement, les {\bfseries formules d'addition}
\index{Formules d'addition en trigonométrie}
suivantes seront d'une très grande utilité lors de la résolution
d'équations impliquant les cosinus et sinus.  Elles sont aussi
utilisées pour simplifier les {\em intégrales}
(chapitre~\ref{chapter_integr}) que nous retrouvons souvent dans les
applications.

\begin{focus}{\prp}
\[
\cos(\theta_1 + \theta_2) = \cos(\theta_1)\cos(\theta_2)
- \sin(\theta_1)\sin(\theta_2)
\]
et
\[
\sin(\theta_1 + \theta_2) = \cos(\theta_1)\sin(\theta_2) +
\sin(\theta_1)\cos(\theta_2)
\]
pour $\theta_1, \theta_2 \in \RR$.
\end{focus}

\begin{rmk}[\theory]
Nous entendons souvent dire qu'il n'y a qu'une et une seule façon de
résoudre un problème de mathématique.  Pour contredire cette
affirmation, nous donnons plusieurs démonstrations différentes des
formules d'addition pour le sinus et le cosinus. Toutes ces
démonstrations sont bonnes.  Il est donc possible d'avoir plus d'une
bonne façon de résoudre un problème mathématique.

\subI{Première démonstration}
Nous utilisons le dessin à la figure~\ref{fonct_trig_SUM}.  Nous
assumons donc que $\theta_1$ et $\theta_2$ sont entre $0$ et $\pi/2$.
Pour les autres valeurs de $\theta_1$ et $\theta_2$, la démonstration
peut être réduite au présent cas à l'aide des identités
trigonométriques données
au propositions~\ref{fonct_ident_trigsA} et \ref{fonct_ident_trigsB}.

\PDFfig{2_fonctions/trig_sum}{Règle d'addition pour les sinus et
  cosinus}{Cette figure sert à la première démonstration de la règle
  d'addition pour les sinus et cosinus qui est donnée à la
  remarque~\ref{proof_add}.}{fonct_trig_SUM}

Le triangle $\triangle OBC$ est semblable au triangle $\triangle EDC$.
En particulier, $\angle COB = \angle CED$.  Si nous considérons le
triangle $\triangle OCE$, nous trouvons que la longueur de l'hypoténuse
$\overline{EC}$ du triangle $\triangle EDC$ est $\sin(\theta_2)$.
Ainsi, $|\overline{ED}|=\sin(\theta_2)\cos(\theta_1)$ et
$|\overline{DC}|=\sin(\theta_2)\sin(\theta_1)$ par définition du
cosinus et du sinus à partir d'un triangle droit.  De même, si nous
considérons le triangle $\triangle OCE$, nous trouvons que la longueur
de l'hypoténuse $\overline{OC}$ du triangle $\triangle OBC$ est
$\cos(\theta_2)$.  Ainsi,
$|\overline{OB}|=\cos(\theta_2)\cos(\theta_1)$ et
$|\overline{BC}|=\cos(\theta_2)\sin(\theta_1)$ par définition du
cosinus et du sinus à partir d'un triangle droit.

Nous avons donc
\[
\sin(\theta_1+\theta_2) = |\overline{AE}| = |\overline{BC}| + |\overline{DE}|
= \cos(\theta_2)\sin(\theta_1) + \sin(\theta_2)\cos(\theta_1) 
\]
et
\[
\cos(\theta_1+\theta_2) = |\overline{OA}| = |\overline{OB}| - |\overline{DC}|
= \cos(\theta_2)\cos(\theta_1) - \sin(\theta_2)\sin(\theta_1) \; .
\]

\subI{Deuxième démonstration} Cette démonstration fait appel aux
nombres complexes que le lecteur n'a probablement pas vu.  Nous
donnons quand même cette démonstration puisqu'elle est très courte et
élégante, en espérant que cela pourra inciter certains lecteurs à
approfondir leurs connaissances en mathématiques au delà du cours de
calcul différentiel et intégral.

Puisque $e^{\theta i} = \cos(\theta) + i \sin(\theta)$ où $i$ est le
nombre complexe tel que $i^2=-1$ et $\theta \in \RR$, nous avons
\begin{equation} \label{complex_equ1}
e^{(\theta_1+\theta_2)i} = \cos(\theta_1+\theta_2) + i \sin(\theta_1+\theta_2)
\end{equation}
et
\begin{align}
&e^{(\theta_1+\theta_2)i} = e^{\theta_1 i} e^{\theta_2 i} =
\big( \cos(\theta_1) + i \sin(\theta_1) \big)
\big( \cos(\theta_2) + i \sin(\theta_2) \big) \nonumber \\
&\quad = \big( \cos(\theta_1)\cos(\theta_2)- \sin(\theta_1)\sin(\theta_2) \big)
+\big( \sin(\theta_1)\cos(\theta_2) + \cos(\theta_1)\sin(\theta_2)\big)i \ .
\label{complex_equ2}
\end{align}
Si nous comparons la partie réelle et imaginaire de (\ref{complex_equ1}) et 
(\ref{complex_equ2}), nous obtenons
\begin{align*}
\cos(\theta_1+\theta_2) &= \cos(\theta_1)\cos(\theta_2)-
\sin(\theta_1)\sin(\theta_2)
\intertext{et}
\sin(\theta_1+ \theta_2) &= \sin(\theta_1)\cos(\theta_2) +
\cos(\theta_1)\sin(\theta_2) \; .
\end{align*}
Évidemment, cette démonstration est très courte mais fait appel à
l'identité $e^{\theta i} = \cos(\theta) + i \sin(\theta)$ qui n'est
pas triviale à démontrer.

% \PDFfig{2_fonctions/trigadd}{Règle d'addition pour les sinus et
%   cosinus}{Cette figure sert à la troisième démonstration de la règle
%   d'addition pour les sinus et cosinus qui est donnée à la
%   remarque~\ref{proof_add}}{proof_pict}

% \subI{Troisième démonstration} Cette démonstration de la règle
% d'addition pour le sinus et cosinus demande beaucoup plus de calculs
% algébriques que la première démonstration.

% La démonstration est basée sur le dessin de la
% figure~\ref{proof_pict}.

% Puisque les triangles $\triangle ABD$ et $\triangle CED$ ont un angle
% commun au sommet $D$ et possèdent tous les deux un angle droit
% (i.e. $\angle CED = \angle ABD = \pi/2$), ils sont donc semblables.
% Il s'en suit que $\angle DCE = \angle DAB = \alpha + \beta$.

% À l'aide de la définition des fonctions trigonométriques pour le
% triangle $\triangle CED$, nous trouvons que
% \begin{align*}
% \cos(\alpha+\beta) &= \frac{\sin(\beta)}{a}
% \intertext{et}
% \sin(\alpha+\beta) &= \frac{b}{a} \; .
% \end{align*}

% Donc
% \begin{align*}
% a &= \frac{\sin(\beta)}{\cos(\alpha+\beta)}
% \intertext{et}
% b &= a \sin(\alpha+\beta) = \frac{\sin(\beta)\sin(\alpha+\beta)}
% {\cos(\alpha+\beta)} \; .
% \end{align*}

% De même. à l'aide de la définition des fonctions trigonométriques pour
% le triangle $\triangle ABD$, nous trouvons que
% \begin{align*}
% \cos(\alpha+\beta) &= \frac{\cos(\alpha)}{b+\cos(\beta)}
% \intertext{et}
% \sin(\alpha+\beta) &= \frac{a+\sin(\alpha)}{b+\cos(\beta)} \; .
% \end{align*}

% Ainsi,
% \begin{align}
% \cos(\alpha+\beta) &= \frac{\cos(\alpha)}
% {\displaystyle \frac{\sin(\beta)\sin(\alpha+\beta)}{\cos(\alpha+\beta)}
% + \cos(\beta)} \nonumber \\
% &= \frac{\cos(\alpha)\cos(\alpha+\beta)}
% {\sin(\beta)\sin(\alpha+\beta)+ \cos(\beta)\cos(\alpha+\beta)}
% \label{add_1}
% \intertext{et}
% \sin(\alpha+\beta) &=
% \frac{\displaystyle \frac{\sin(\beta)}{\cos(\alpha+\beta)}+\sin(\alpha)}
% {\displaystyle \sin(\beta)\frac{\sin(\alpha+\beta)}{\cos(\alpha+\beta)}
% + \cos(\beta)} \nonumber \\
% &= \frac{\sin(\beta)  +\sin(\alpha)\cos(\alpha+\beta)}
% {\sin(\beta)\sin(\alpha+\beta) + \cos(\beta)\cos(\alpha+\beta)} \; .
% \label{add_2}
% \end{align}

% Si nous résolvons l'équation (\ref{add_1}) ci-dessus en fonction de
% $\sin(\alpha+\beta)$, nous trouvons
% \begin{equation} \label{add_3}
% \sin(\alpha+\beta) = \frac{\cos(\alpha)-\cos(\beta)\cos(\alpha+\beta)}
% {\sin(\beta)} \; .
% \end{equation}
% Si maintenant nous substituons cette expression pour $\sin(\alpha+\beta)$
% dans (\ref{add_2}) et nous résolvons pour $\cos(\alpha+\beta)$, nous
% trouvons
% \begin{equation} \label{add_4}
% \cos(\alpha+\beta) = \frac{\cos^2(\alpha)-\sin^2(\beta)}
% {\cos(\alpha)\cos(\beta)+\sin(\alpha)\sin(\beta)} \; .
% \end{equation}

% Finalement, si nous substituons l'expression pour $\cos(\alpha+\beta)$ que
% nous venons de donner en (\ref{add_4}) dans (\ref{add_3}), nous trouvons
% l'expression suivante pour $\sin(\alpha+\beta)$.
% \begin{equation}\label{add_5}
% \sin(\alpha+\beta) = \frac{\cos(\alpha)\sin(\alpha)+\cos(\beta)\sin(\beta)}
% {\cos(\alpha)\cos(\beta)+\sin(\alpha)\sin(\beta)} \; .
% \end{equation}

% Or, grâce à l'identité $\cos^2(\theta)+\sin^2(\theta)=1$, nous avons que
% \[
% \cos^2(\alpha)-\sin^2(\beta) =
% \left(\cos(\alpha)\cos(\beta) + \sin(\alpha)\sin(\beta)\right)
% \left(\cos(\alpha)\cos(\beta) - \sin(\alpha)\sin(\beta)\right)
% \]
% et
% \begin{multline*}
% \cos(\alpha)\sin(\alpha)+\cos(\beta)\sin(\beta) = \\
% \left(\cos(\alpha)\cos(\beta) + \sin(\alpha)\sin(\beta)\right)
% \left(\cos(\alpha)\sin(\beta) + \sin(\alpha)\cos(\beta)\right) \ .
% \end{multline*}

% Si nous substituons ces deux expressions dans (\ref{add_4}) et
% (\ref{add_5}) respectivement, nous trouvons finalement
% \begin{align*}
% \cos(\alpha+\beta) &= \cos(\alpha)\cos(\beta) - \sin(\alpha)\sin(\beta)
% \intertext{et}
% \sin(\alpha+\beta) &= \cos(\alpha)\sin(\beta) + \sin(\alpha)\cos(\beta) \; .
% \end{align*}

% Le lecteur a probablement remarqué que nous avons supposé que
% $\alpha+\beta \neq \pi/2$ car, dans ce cas, $\cos(\alpha+\beta) = 0$
% et nous aurions une division par $0$.  Le cas $\alpha+\beta=\pi/2$ est en
% fait très simple à démontrer.  Pour cela, nous considérons la
% figure~\ref{Sproof_pict}.   Nous trouvons que
% \begin{align*}
% \cos(\alpha)\cos(\beta) - \sin(\alpha)\sin(\beta) &=
% \cos(\alpha)\sin(\alpha) - \sin(\alpha)\cos(\alpha) \\
% &= 0 = \cos(\pi/2) = \cos(\alpha+\beta)
% \intertext{et}
% \cos(\alpha)\sin(\beta) + \sin(\alpha)\cos(\beta) &=
% \cos^2(\alpha) +\sin^2(\alpha) = 1 = \sin(\pi/2) = \sin(\alpha+\beta) \; .
% \end{align*}

% \PDFfig{2_fonctions/trigadd2}{Règle d'addition pour les sinus et
%   cosinus}{Cette figure sert à la troisième démonstration de la règle
%   d'addition pour les sinus et cosinus qui est donnée à la
%   remarque~\ref{proof_add} lorsque la somme des angles est
%   $\pi/2$.}{Sproof_pict}

% Les autres cas où le raisonnement précédent ne fonctionne pas
% (e.g. $\cos(\alpha) = \sin(\beta)=0$) peuvent tous être facilement
% traités comme nous venons de le faire pour le cas $\alpha+\beta=\pi/2$.
\label{proof_add}
\end{rmk}

\begin{egg}
Quelle est la valeur de $\displaystyle \sin(2\pi/3)$?

Puisque $\sin(\theta) = \sin(\pi - \theta)$, nous avons que
\[
\sin\left(\frac{2\pi}{3}\right) = \sin\left(\pi - \frac{\pi}{3}\right)
= \sin\left(\frac{\pi}{3}\right) = \frac{\sqrt{3}}{2}
\]
où la dernière égalité provient du Tableau~\ref{BAS_VAL}.
\end{egg}

\begin{egg}
Quelle est la valeur de $\sin(7\pi/12)$?

Puisque $7\pi/12 = \pi/3 + \pi/4$, nous obtenons de la formule d'addition
pour le sinus que
\begin{align*}
\sin\left(\frac{7\pi}{12}\right) &=
\sin\left(\frac{\pi}{3} + \frac{\pi}{4}\right)
= \cos\left(\frac{\pi}{3}\right) \sin\left(\frac{\pi}{4}\right) +
\sin\left(\frac{\pi}{3}\right) \cos\left(\frac{\pi}{4}\right) \\
&= \frac{1}{2} \times \frac{\sqrt{2}}{2} + \frac{\sqrt{3}}{2}
\times \frac{\sqrt{2}}{2}
= \frac{\sqrt{2}}{4} \left( 1 + \sqrt{3}\right) \; .
\end{align*}
\end{egg}

\begin{egg}
Montrons que
\begin{equation}\label{db_angle}
\cos^2(\theta) = \frac{1}{2} \left( 1 + \cos(2\theta) \right)
\end{equation}
quel que soit $\theta$.

Si nous substituons $\theta_1$ et $\theta_2$ par $\theta$ dans la
formule d'addition pour le cosinus, nous obtenons
\[
\cos(2 \theta) = \cos^2(\theta) - \sin^2(\theta) \; .
\]
Puisque $\sin^2(\theta) = 1 - \cos^2(\theta)$, il découle de
l'équation ci-dessus que
\begin{equation} \label{doubleAngle}
\cos(2 \theta) = 2 \cos^2(\theta) - 1 \; .
\end{equation}
D'où
\[
\cos^2(\theta) = \frac{1}{2} \left( 1 + \cos(2\theta) \right) \; .
\]
\end{egg}

Il existe une formule pour $\sin^2(\theta)$ qui est
semblable à celle donnée en (\ref{db_angle}).  La proposition suivante
inclus ces deux formules. 

\begin{focus}{\prp}\index{Formules de l'angle double}
\[
\cos^2(\theta) = \frac{1}{2} \left( 1 + \cos(2\theta) \right)
\quad \text{et} \quad 
\sin^2(\theta) = \frac{1}{2} \left( 1 - \cos(2\theta) \right) \; .
\]
La première formule est la {\bfseries formule de l'angle double pour
le cosinus} alors que la deuxième formule est la {\bfseries formule de
l'angle double pour le sinus}. \label{db_angle_prp}
\end{focus}

Nous laissons au lecteur la tâche de vérifier la formule de l'angle
double pour le sinus comme nous l'avons fait pour la formule de l'angle
double pour le cosinus.

\begin{rmk}
Les deux formules données à la proposition~(\ref{db_angle_prp}) vont
s'avérer très utiles pour évaluer certaines {\em intégrales} au
chapitre~\ref{chapter_integr}.
\end{rmk}

Pour terminer, mentionnons deux identités trigonométriques qui
font appel à la tangente, la cotangente, la sécante et la cosécante.
Si nous divisons l'identité $\cos^2(\theta) + \sin^2(\theta) = 1$ tour
à tour par $\cos^2(\theta)$ et par $\sin^2(\theta)$, nous obtenons les
deux identités suivantes.

\begin{focus}{\prp}
\[
\begin{array}{ll}
1 + \tan^2(\theta) = \sec^2(\theta) &\quad \text{pour} \quad \theta \neq
\displaystyle \frac{\pi}{2} + n \pi \ \text{pour} \ n \in \ZZ \; . \\[0.5em]
\cot^2(\theta) + 1 = \csc^2(\theta) &\quad \text{pour} \quad \theta \neq
n \pi \ \text{pour} \ n \in \ZZ \; .
\end{array}
\]
\end{focus}

\subsection{Graphes des fonctions trigonométriques}

Le cosinus et le sinus sont deux fonctions définies pour tous les
nombres réels.  Les graphes de ces fonctions sont donnés à la
figure~\ref{TRIG2-3}.

Nous déduisons à partir de leurs définitions que
\[
-1 \leq \cos(\theta) \leq 1 \qquad \text{et} \qquad
-1 \leq \sin(\theta) \leq 1
\]
quel que soit l'angle $\theta$.   Ainsi,
\[
\DO \cos = \DO \sin = \RR \qquad et \qquad
\IM \cos = \IM \sin = \{ x : -1 \leq x \leq 1 \} \; .
\]

\PDFfigD[h]{2_fonctions/trig3}{2_fonctions/trig2}{Graphes de $x=\cos(\theta)$
et $y=\sin(\theta)$}{Graphes de $x=\cos(\theta)$ et $y=\sin(\theta)$}
{TRIG2-3}

Puisque $\cos(\theta) = \sin(\theta+ \pi/2)$ pour tout $\theta$, le
graphe du cosinus est une translation par $\pi/2$ vers le gauche du
graphe du sinus.

La tangente est une fonction qui n'est pas définie pour tous les
angles.  Le graphe de la tangente est donné à la figure~\ref{TRIG6}.
Nous avons que
\[
\DO \tan = \left\{ \theta : \theta \neq n \pi + \frac{\pi}{2} \; ,
\quad  n\in \ZZ \right\}
\qquad \text{et} \qquad \IM \tan = \RR \; .
\]
De plus, $\tan(\theta) = \tan(\theta+\pi)$ quel que soit
$\theta \neq n \pi + \pi/2$ pour $n\in\ZZ$.  La tangente est donc une
fonction {\bfseries périodique} de {\bfseries période}
\index{Période d'une fonction} $\pi$.

\MATHfig{2_fonctions/trig6}{8cm}{Graphe de $y=\tan(\theta)$}
{Graphe de $y=\tan(\theta)$}{TRIG6}

Le graphe de la cotangente, le graphe de la sécante et
le graphe de la cosécante sont donnés à la figure~\ref{TRIG7}.  Nous
avons que
\begin{align*}
\DO\ \cot &= \left\{ \theta : \theta \neq n \pi \; ,
\ n\in \ZZ \right\} \qquad \text{et} \qquad
\IM\ \cot = \RR \; ,\\
\DO\ \sec &= \left\{ \theta : \theta \neq n \pi + \frac{\pi}{2}
\ , \ n\in \ZZ \right\} \qquad \text{et} \qquad
\IM\ \sec = \{ y : |y| \geq 1 \}  \; ,
\intertext{et}
\DO\ \csc &= \left\{ \theta : \theta \neq n \pi
\ , \ n\in \ZZ \right\} \qquad \text{et} \qquad
\IM\ \cot = \{ y : |y| \geq 1 \} \; .
\end{align*}

Comme la tangente, la cotangente est une fonction
{\bfseries périodique} de {\bfseries période} $\pi$.  La
sécante et la cosécante sont des fonctions
{\bfseries périodiques} de {\bfseries période} $2\pi$.

\MATHfig{2_fonctions/trig7}{10cm}{Graphe de $y=\cot(\theta)$,
  $y=\sec(\theta)$, et $y=\csc(\theta)$}
{Graphe de $y=\cot(\theta)$, $y=\sec(\theta)$, et $y=\csc(\theta)$}
{TRIG7}

\begin{focus}{\dfn} \index{Fonctions sinusoïdales}
Une {\bfseries fonctions sinusoïdales} est une fonctions de la forme
\begin{equation} \label{sinuso}
y = M + A\;\cos\left(\frac{2\pi}{P}\left(x-X\right)\right)
\end{equation}
où $M$, $A$, $P$ et $X$ sont des constantes.  La signification
graphique de ces constantes est donnée à la figure~\ref{sinuso_fig}.  La
constante $M$ est la
{\bfseries moyenne}\index{Fonctions sinusoïdales!moyenne} de la fonction
sinusoïdale, la constante $A$ est
l'{\bfseries amplitude}\index{Fonctions sinusoïdales!amplitude} de
l'onde décrite par la fonction sinusoïdale, la constante $P$ est la
{\bfseries période}\index{Fonctions sinusoïdales!période} et la
constante $X$ est la
{\bfseries phase}\index{Fonctions sinusoïdales!phase} de l'onde
décrite par la fonction sinusoïdale.
\end{focus}

Dans certains livres, le sinus est utilisé pour définir les fonctions
sinusoïdales.  C'est-à-dire que
\[
y = M + A\;\sin\left(\frac{2\pi}{P}\left(x-X\right)\right) \ .
\]
En changeant seulement la phase, nous pouvons produire avec cette
formule n'importe laquelle des ondes données par la formule
(\ref{sinuso}).

\PDFfig{2_fonctions/sinusoid}{Graphe d'une fonction sinusoïdale}
{Graphe d'une fonction sinusoïdale}{sinuso_fig}

\begin{egg}
Traçons le graphe de la fonction $y=f(x) = 3+5 \cos((2\pi/7)(x+4))$.

La moyenne de cette fonction sinusoïdale est $M=3$, son amplitude est
$A=5$, sa période est $P=7$ et sa phase est $X=-4$.  Le graphe d'une
période de cette fonction est donné à la figure~\ref{sinuso_egg}.
\end{egg}

\PDFfig{2_fonctions/sinusoid2}{Graphe de $y=f(x) = 3+5 \cos((2\pi/7)(x+4))$}
{Graphe de $y=f(x) = 3+5 \cos((2\pi/7)(x+4))$}{sinuso_egg}

\begin{egg}
La température moyenne du corps pour une période de $24$
heures est de $36.8^\circ$C avec un minimum de $36.5^\circ$C à 2h00 et
un maximum de $37.1^\circ$C à 14h00.  Supposons que la
température $T$ du corps en fonction du temps $t$ soit décrite par
une fonction sinusoïdale.   Quelle est cette fonction?

Dans la formule (\ref{sinuso}), la moyenne est $M=36.8$, l'amplitude
est $A=(37.1-36.5)/2 = 0.3$, la phase est $X=14$ heures et la période
est naturellement $P=24$ heures.  Nous trouvons donc
\[
T = 36.8 + 0.3 \cos\left(\frac{2\pi}{24}(t-14)\right)
\]
où $T$ est la température en degrés centigrades et $t$ est le temps en
heures.
\end{egg}

\subsection{Fonctions trigonométriques inverses}

Le graphe de la fonction sinus est donné à la figure~\ref{SIN}.  Il
est clair par le test des droites horizontales que cette fonction
n'est pas injective et donc n'a pas d'inverse.

\MATHfig{2_fonctions/sin1}{8cm}{Graphe de $y=\sin(x)$}
{Graphe de $y=\sin(x)$ pour $-6 \leq x \leq 6$}{SIN}

Mais! Il y a bien une fonction $\sin^{-1}$ sur les calculatrices.
Quelle est cette fonction?

Nous définissons la fonction inverse pour le sinus en restreignant le
domaine du sinus à l'intervalle $-\pi/2 \leq x \leq \pi/2$.  C'est la
courbe tracée en vert que nous retrouvons sur le graphe
à la figure~\ref{SIN}.

\begin{focus}{\dfn} \index{Arcsinus}
Sur l'intervalle $-\pi/2 \leq x \leq \pi/2$, la fonction $\sin$ est
bien injective et nous pouvons définir son inverse $\sin^{-1}$ par la
formule standard
\[
x = \sin^{-1}(y) \qquad \text{si et seulement si} \qquad y = \sin(x)
\]
pour $-\pi/2 \leq x \leq \pi/2$ et $-1 \leq y \leq 1$.  La fonction
$\sin^{-1}$ est appelée l'{\bfseries arcsinus} et est
aussi dénotée $\arcsin$; c'est-à-dire, $\sin^{-1}(x) \equiv \arcsin(x)$.
\end{focus}

En d'autres mots, $x = \sin^{-1}(y)$ est l'angle en radians entre
$-\pi/2$ et $\pi/2$ qu'il faut pour que $\sin(x) = y$.

La définition de $\sin^{-1}$ explique pourquoi cette fonction sur
votre calculatrice n'acceptera jamais d'arguments plus petits que $-1$
et plus grands que $1$.

Nous retrouvons à la figure~\ref{ASIN} le graphe de la fonction
$\sin^{-1}$.  Le lecteur peut vérifier que le graphe de $\sin^{-1}$
est bien obtenu d'une réflexion par rapport à la droite $y=x$ du
graphe de $\sin$.

\MATHfig{2_fonctions/arcsin1}{8cm}{Graphe de $y=\sin^{-1}(x)$}
{Graphe de $y=\sin^{-1}(x)$ pour $-1 \leq x \leq 1$}{ASIN}

De même, nous pouvons définir l'inverse pour le cosinus.

\begin{focus}{\dfn} \index{Arccosinus}
Sur l'intervalle $0 \leq x \leq \pi$, la fonction $\cos$ est bien
injective et nous pouvons définir son inverse $\cos^{-1}$ par la formule
standard
\[
x = \cos^{-1}(y) \qquad \text{si et seulement si} \qquad y = \cos(x)
\]
pour $0 \leq x \leq \pi$ et $-1 \leq y \leq 1$.  La fonction
$\cos^{-1}$ est appelée {\bfseries arccosinus} et
est aussi dénotée $\arccos$; c'est-à-dire,
$\cos^{-1}(x) \equiv \arccos(x)$.
\end{focus}

Finalement, nous pouvons définir l'inverse pour la tangente.

\begin{focus}{\dfn} \index{Arctangente}
Sur l'intervalle $-\pi/2 < x < \pi/2$,  la fonction $\tan$ est bien
injective et nous pouvons définir son inverse $\tan^{-1}$ par la formule
standard
\[
x = \tan^{-1}(y) \qquad \text{si et seulement si} \qquad y = \tan(x)
\]
pour $-\pi/2 < x < \pi/2$ et $y$ réels.  La fonction $tan^{-1}$ est
appelée l'{\bfseries arctangente} et est aussi
dénotée $\arctan$; c'est-à-dire, $\tan^{-1}(x) \equiv \arctan(x)$.
\end{focus}

Le graphe de $\cos^{-1}$ est donné à la figure~\ref{ACOS} et celui de
$\tan^{-1}$ est donné à la figure~\ref{ATAN}.

\MATHfig{2_fonctions/arccos1}{8cm}{Graphe de $y=\cos^{-1}(x)$}
{Graphe de $y=\cos^{-1}(x)$ pour $-1 \leq x \leq 1$}{ACOS}

\MATHfig{2_fonctions/arctan1}{8cm}{Graphe de $y=\tan^{-1}(x)$}
{Graphe de $y=\tan^{-1}(x)$ pour $x$ réels}{ATAN}

\begin{rmk}
Remarquons que $\displaystyle \sin^{-1} \neq \sin^{-1}$

Il faut bien comprendre que $\sin^{-1}$ sur la calculatrice n'est pas
égale à $\left(\sin(x)\right)^{-1} = 1/\sin(x)$.  Par exemple,
\[
\sin^{-1}\left(\frac{1}{2}\right) = \frac{\pi}{6} = 0.52359877559830\ldots
\]
alors que
\[
\left(\sin\left(\frac{1}{2}\right)\right)^{-1}
= \frac{1}{\sin\left(1/2\right)} =
2.08582964293349\ldots
\]
\end{rmk}

\begin{egg}
Montrons que $\displaystyle \sin(\arccos(x)) = \sqrt{1-x^2}$ pour
$-1\leq x \leq 1$. 

Soit $\theta = \arccos(x)$ où $0 \leq \theta \leq \pi$.  Nous
déduisons des deux dessins à la figure~\ref{arcXX3} que
$\sin(\theta) = \sqrt{1-x^2}$.
\end{egg}

\PDFfig{2_fonctions/arcXX3}{Représentation graphique de
$\cos(\theta)$ pour $0\leq \theta \leq \pi$}{Représentation
graphique de $\cos(\theta)=x$ pour $-1 \leq x \leq 1$ et
$0\leq \theta \leq \pi$.}{arcXX3}

% \begin{egg}
% Montrons que
% $\displaystyle \sin(\arcsec(x)) = \sqrt{1-\frac{1}{x^2}}$
% pour $|x|>1$.

% Soit $\theta = \arcsec(x)$ où $\pi/2 < \theta \leq \pi$ pour $x\leq -1$
% et $0 \leq \theta < \pi/2$ pour $x \geq 1$.
% \end{egg}

\section{Fonctions exponentielles et logarithmiques}

\subsection{Fonctions exponentielles}\label{cerclevicieux}

Le but de cette section est de donner un sens à l'expression $b^p$
pour le plus grand ensemble possible de {\bfseries base} $b$ et
d'{\bfseries exposant} $p$.  Nous résumons ci-dessous ce qui est connu
au sujet de l'exponentiel d'un nombre.

{\renewcommand{\labelitemi}{\textbullet}
\begin{itemize}
\item L'expression $b^p$ n'est pas définie pour $b=p=0$.
\item $b^p =1$ pour $b>0$ et $p=0$.  Par exemple, $\pi^0 = 1$.
\item Si $b \in \RR$ et $p$ est un entier positif,
\[
b^{p} = \underbrace{b \times b \times \ldots \times b}_{\text{p fois}}
\; .
\]
Par exemple, $\pi^3 = \pi \times \pi \times \pi$.
\item Si $b \in \RR$ et $p = 1/n$ où $n$ est un nombre impair, $y=b^p$
  est le nombre réel tel que $y^n = b$.  Par exemple,
  $(-8)^{1/3} = -2$ car $(-2)^3 = -8$.
\item Si $b \geq 0$ et $p = 1/n$ où $n$ est un nombre pair, $y=b^p$
  est le {\bfseries nombre réel positif} tel que $y^n = b$.  Par exemple,
  $16^{1/4} = 2$ car $2^4 = 16$.
\item Si $p = m/n$, nous calculons $\displaystyle \left(b^{1/n}\right)^m$
  s'il est possible de calculer $\displaystyle b^{1/n}$ en premier.
  Il est toujours avantageux de simplifier la fraction $m/n$ avant de
  faire les calculs.  Par exemple, $(-8)^{2/3} = 4$ car $(-8)^{1/3} = -2$ et
  $\displaystyle \left((-8)^{1/3}\right)^2 = (-2)^2 = 4$.  De même,
  $16^{3/4} = 8$ car $16^{1/4} =2$ et
  $\displaystyle \left(16^{1/4}\right)^3 = 2^3 = 8$.
\item Si $p<0$, nous définissons $b^p$ comme étant le nombre
  $1/b^{-p}$ si $b^{-p}$ est définie.  Par exemple,
  $\pi^{-3} = 1/\pi^3 = 1/(\pi \times \pi \times \pi)$.
\end{itemize}
}

Aucune définition n'est donnée pour $b^p$ lorsque $b$ est un nombre
réel positif et $p$ est un exposant réel quelconque.  Pour calculer
$4^2$, il suffit de multiplier $4$ par lui-même.  Pour calculer
$4^{1/2}$, il suffit de trouver le nombre positif $y$ tel que $y^2=4$;
c'est-à-dire, $y=2$ dans le cas présent.  Mais que devons-nous faire
pour obtenir la valeur de $4^\pi$?  Nous pouvons utiliser
la définition suivante.

\begin{focus}{\dfn} \label{pr_def_of_exp}
Pour $b >1$ et $p \in \RR$, nous définissons le {\bfseries nombre exponentiel}
$b^p$ comme étant le plus petit nombre réel\footnotemark\ $M$ tel que
$M \geq  b^r$ pour tout nombre rationnel $r\leq p$.

Pour $0<b<1$ et $p \in \RR$, nous définissons le {\bfseries nombre exponentiel}
$b^p$ comme étant le plus grand nombre réel $M$ tel que $M \leq  b^r$
pour tout nombre rationnel $r\leq p$.

Naturellement, si $b=1$, nous obtenons $b^p = 1$ pour tout $p\in \RR$.
\end{focus}

\footnotetext{{\bfseries L'Axiome de complétude} des nombres réels dit
  que tout ensemble non vide et borné supé\-rieurement possède une
  plus petite borne supérieure.}

Ainsi, $4^\pi$ est le plus petit nombre réel $M$ tel que $4^r \leq M$
pour tout nombre rationnel $r\leq \pi$.  De même, $4^2 = 16$ car
$M=16$ est le plus petit nombre tel que $4^r \leq M$ pour tout nombre
rationnel $r\leq 2$; le nombre $2$ est une valeur possible pour $r$. 

La définition que nous venons de donner pour le nombre exponentiel
$b^p$ avec $b\in]0,\infty[$ et $p\in \RR$ n'est pas facile à utiliser.
Il faut généralement beaucoup de travail pour évaluer un nombre
exponentiel avec cette définition.  Pour l'instant, cette définition
sera suffisante pour nos besoins.

En fait, cette définition d'un nombre exponentiel manque de rigueur.
Il faudrait démontrer que cette définition est consistante.  Par
exemple, est-ce que cette définition de $b^p$ pour $b>1$ est
équivalente à \lgm $b^p$ est le plus grand nombre réel $M$ tel que $M
\leq  b^r$ pour tout nombre rationnel $r\geq p$\rgm ?  Comme nous verrons 
prochainement, cela revient à exiger que les limites à droite et à
gauche de $b^r$ soient égales à $r=p$.  Il y a bien d'autres questions
de ce genre que nous pourrions poser au sujet de cette définition.  
De plus, il est difficile de démontrer les propriétés des exposants
avec cette définition.

Nous donnerons à la section~\ref{DEFofE} une définition équivalent du
nombre exponentiel $b^p$, $b\in]0,\infty[$ et $p\in \RR$, qui sera
plus facile à manipuler que celle ci-dessus.  Avec la définition
présentée à la section~\ref{DEFofE}, il est facile de déduire les
propriétés des exposants.  De plus, les calculatrices et ordinateurs
utilisent cette définition (avec quelques subtilités additionnelles)
pour évaluer les nombres exponentiels.

Pour l'instant, la valeur d'un nombre exponentiel comme $4^\pi$ sera estimée
par $4^r$ pour des valeurs rationnelles de $r$ très près de $\pi$ et
inférieures à $\pi$.  Quelle valeur de $r$, proche de $\pi$, doit-on
utiliser pour obtenir une bonne approximation de $4^\pi$?  Cette
question ne sera pas abordée.

\begin{focus}{\dfn} \index{Fonction exponentielle}
Si $b$ est un nombre réel positif, la fonction
$f:\RR\rightarrow ]0,\infty[$ définie par $f(x) = b^x$ pour tout $x$
réel est appelée une {\bfseries fonction exponentielle}.
\end{focus}

Les fonctions exponentielles satisfont les propriétés suivantes.

\begin{focus}{\prp}
Si $a$ et $b$ sont des nombres réels positifs et $x$ et $y$ sont des
nombres réels, alors
\[
a^{-x} = \frac{1}{a^x} \; , \quad a^x a^y = a^{x+y} \; ,
\quad a^x b^x = (ab)^x \quad \text{et} \quad
\left(a^x\right)^y = \left(a^y\right)^x = a^{xy} \; .
\]
\end{focus}

Nous retrouvons les graphes de quelques fonctions exponentielles à la
figure~\ref{EXP1}.
\MATHfig{2_fonctions/exp1}{8cm}{Graphes de $y=3^x$, $y=2^x$
et $y=(1/2)^x$}{Graphes de $y=3^x$, $y=2^x$ et $y=(1/2)^x$}{EXP1}

Si $f$ est la fonction définie par $f(x) = b^x$ pour tout $x$, nous
déduisons des graphes à la figure~\ref{EXP1} que le domaine de $f$ est
$\RR$ et que son image est $\{ x \in \RR : x > 0 \}$.  De plus, $f$
est une fonction croissante pour $b>1$ et décroissante pour $0<b<1$.

\subsection{Fonctions logarithmiques}

La fonction logarithmique n'est rien d'autre que la fonction inverse
de $f(x) = b^x$ où $b$, la {\bfseries base}, est un nombre réel
positif.  Si nous appliquons la définition de l'inverse d'une fonction à
la fonction $b^y$, nous obtenons:

\begin{focus}{\dfn} \index{Fonction logarithmique}
La {\bfseries fonction logarithmique} $\log_b$ est définie par
\[
y = \log_b(x)  \quad \text{si et seulement si} \quad  b^y = x
\]
pour $x>0$ et $y\in\RR$.
\end{focus}

Nous donnons à la figure~\ref{LOG1} le graphe des fonctions $f(x) = b^x$
et $f^{-1}(x) = \log_b(x)$ pour $b=3$.  Puisque le domaine de
$f(x) = b^x$ est $\RR$ et son image est $\{ x \in \RR : x > 0 \}$, nous
obtenons bien que le domaine de $\log_b$ est $\{ x \in \RR : x > 0 \}$
et son image est $\RR$.

\MATHfig{2_fonctions/log1}{8cm}{Graphes de $y=3^x$ et de $y=\log_3(x)$}
{Graphes de $y=3^x$ (tirets) et de $y=\log_3(x)$ (ligne pleine)}{LOG1}

Puisque $y=\log_b(x)$ est l'exposant qu'il faut donner à $b$ pour
obtenir $x$ (i.e. pour que $b^y = x$), il est alors simple de vérifier
à partir de cet énoncé que les fonctions logarithmiques satisfont les
propriétés suivantes.

\begin{focus}{\prp}
Si $a$, $b$, $x$ et $y$ sont des nombres positifs réels et $p$ est
un nombre réel, alors
\begin{align}
\log_b(x^p) &= p \log_b(x) \; , \label{log_id1} \\
\log_b(x y ) &= \log_b(x) + \log_b(y) \label{log_id2}
\intertext{et}
\frac{\log_a(x)}{\log_a(y)} &= \log_y(x) \label{log_id3} \; .
\end{align}
\end{focus}

\begin{egg}
Si $b$, $x$ et $y$ sont des nombres positifs réels, alors
\[
\log_b(\frac{x}{y}) = \log_b(x) - \log_b(y)
\]
car
\[
\log_b(\frac{x}{y}) = \log_b(x\, y^{-1}) = \log_b(x) + \log_b(y^{-1})
= \log_b(x) - \log_b(y) \; ,
\]
où la deuxième égalité provient de (\ref{log_id2}) et la troisième
égalité provient de (\ref{log_id1}).
\end{egg}

\begin{egg}
Démontrons à partir de la définition du logarithme que
$\log_b(x^p) = p \log_b(x)$.

Si $y = \log_b(x^p)$ alors $b^y = x^p$ par définition du logarithme.
Donc $b^{y/p} = x$.  Une deuxième utilisation de la définition du
logarithme donne $\log_b(x) = y/p$.  Ainsi, si nous résolvons cette
dernière équation pour $y$, nous trouvons $p\log_b(x) = y = \log_b(x^p)$.
\end{egg}

\begin{egg}
Démontrons que $\displaystyle \frac{\log_a(x)}{\log_a(y)} = \log_y(x)$
à partir de la définition du logarithme.

Si $v = \log_a(x)$ et $w = \log_a(y)$, alors $a^v = x$ et $a^w = y$
par définition du logarithme.  Nous obtenons de $a^w = y$ que
$a = y^{1/w}$.  Si nous substituons cette expression pour $a$ dans
$a^v = x$, nous trouvons $x = \left(y^{1/w}\right)^v = y^{v/w}$.  Une
deuxième utilisation de la définition du logarithme donne
$\displaystyle \log_y(x) = \frac{v}{w} = \frac{\log_a(x)}{\log_a(y)}$ par
définition de $v$ et $w$.
\end{egg}

La base la plus importante pour les fonctions exponentielles et
logarithmiques est donnée par le {\bfseries nombre d'Euler}
$e = 2.718281828459\ldots$ que nous définirons correctement à la
section~\ref{nbrE}.  Comme pour $\pi$, ce nombre se retrouve sur
toutes les calculatrices scientifiques car il joue un rôle fondamental
en mathématique.  En raison de son importance, nous donnons un nom spécial
à la fonction logarithmique en base $e$.

\begin{focus}{\dfn} \index{Logarithme naturel}\index{Logarithme népérien}
Le {\bfseries logarithme naturel} ou {\bfseries logarithme népérien}
est défini par $\ln(x) = \log_e(x)$ pour $x>0$.
\end{focus}

La définition précédente est équivalente à
\[
y=\ln(x) \quad \Leftrightarrow \quad e^y = x \; .
\]
La fonction $\ln:]0,\infty[\rightarrow \RR$ est l'inverse de la fonction
exponentielle $f(x) = e^x$.

\begin{egg}
Démontrons à partir de la définition du logarithme que
$\log_b(x^p) = p \log_b(x)$.

Si $y = \log_b(x^p)$ alors $b^y = x^p$ par définition du logarithme.
Donc $b^{y/p} = x$.  Une deuxième utilisation de la définition du
logarithme donne $\log_b(x) = y/p$.  Ainsi, si nous résolvons cette
dernière équation pour $y$, nous trouvons $p\log_b(x) = y = \log_b(x^p)$.
\end{egg}

\begin{egg}
Un modèle fréquemment utilisé pour décrire une population est de la
forme $P(t) = P_0 e^{\alpha t}$ où $P(t)$ est le nombre d'individus au
temps $t$, $P_0$ est le nombre initial d'individus, et $\alpha$ est le
\lgm taux de croissance relatif\rgm.  Le taux de croissance relatif
est défini comme étant le taux de croissance (instantané) au temps $t$
divisé par le nombre total d'individus au temps $t$.  Dans ce modèle,
nous supposons que ce rapport est constant et égale à $\alpha$.  Nous
justifierons ce modèle au chapitre~\ref{chap_equ_diff} sur les équations
différentielles.

Déterminons le temps nécessaire pour que la population soit
$2.5$ plus grande si $\alpha =1.5$.  Supposons que le temps soit
mesuré en heures.

Il faut donc trouver $t$ tel que $\displaystyle P(t) = 2.5 P_0$.
C'est-à-dire, il faut trouver $t$ tel que $P_0 e^{1.5 t}  = 2.5 P_0$.  Ainsi,
\[
P_0 e^{1.5 t}  = 2.5 P_0 \Rightarrow e^{1.5 t}  = 2.5
\Rightarrow 1.5 t = \ln\left( e^{1.5 t} \right) = \ln(2.5)
\Rightarrow t = \frac{\ln(2.5)}{1.5} \approx 0.61086 \ .
\]
Il faut donc approximativement $0.61086$ heure pour que la population
soit $2.5$ fois plus grande.

Si la population prend $3$ jours pour être $2.5$ plus grande, quel est
le taux de croissance relatif?

Puisque $P(3) = 2.5 P_0$, nous avons
\[
P_0 e^{3 \alpha} = 2.5 P_0 \Rightarrow e^{3 \alpha} = 2.5
\Rightarrow 3\alpha = \ln\left( e^{3 \alpha} \right)= \ln(2.5)
\Rightarrow \alpha = \frac{\ln(2.5)}{3} \approx 0.3054 \ .
\]

Quelle est le nombre initial d'individus si nous avons $\alpha = 0.5$ et
$10^6$ individus après $10$ jours?

Puisque $\displaystyle 10^6 = P(10) = P_0 e^{0.5 \times 10}$, nous
obtenons $\displaystyle P_0 = \frac{10^6}{e^{5}} \approx 6738$ individus.
\end{egg}

}  % End of theory

\section{Exercices}

\subsection{Algèbre}

\begin{question}
Simplifiez si possible les expressions suivantes.
\begin{center}
\begin{tabular}{*{2}{l@{\hspace{0.5em}}l@{\hspace{3em}}}l@{\hspace{0.5em}}l}
\subQ{a} & $\displaystyle (3^4)^{0.5}$ &
\subQ{b} & $\displaystyle 2^{2^3} \times 2^{2^2}$ &
\subQ{c} & $\displaystyle \log_3(1)$  \\[0.7em]
\subQ{d} & $\displaystyle \log_{10}(5) + \log_{10}(20)$ &
\subQ{e} & $\displaystyle \log_{10}(500) - \log_{10}(50)$ &
\subQ{f} & $\displaystyle \log_{42.3}(42.3^7)$
\end{tabular}
\end{center}
\label{2Q1}
\end{question}

\begin{question}
Simplifiez si possible les expressions suivantes.
\begin{center}
\begin{tabular}{*{1}{l@{\hspace{0.5em}}l@{\hspace{6em}}}l@{\hspace{0.5em}}l}
\subQ{a} & $\displaystyle \frac{(x^4 y^{1/4})^{1/2}}{y^{1/2}}$ &
\subQ{b} & $\displaystyle \frac{x+1}{x-2} + \frac{x+1}{x-3} +
 \frac{-9x+21}{x^2-5x+6}$
\end{tabular}
\end{center}
\label{2Q2}
\end{question}

\begin{question}
Utilisez la division de polynômes pour écrire
$\displaystyle \frac{x^3 + 2x^2 +x+3}{x+2}$ sous la forme
$p(x)+r(x)/q(x)$ où le degré de $r(x)$ est plus petite que le degré de
$q(x)$.
\label{2Q3}
\end{question}

\begin{question}
Factorisez les polynômes suivants.
\begin{center}
\begin{tabular}{*{2}{l@{\hspace{0.5em}}l@{\hspace{3em}}}l@{\hspace{0.5em}}l}
\subQ{a} & $x^2+x-6$ &
\subQ{b} & $3x^2 - 5 x -2$ &
\subQ{c} & $\displaystyle x^{3/2} + x^{1/2}-12 x^{-1/2}$
\end{tabular}
\end{center}
\label{2Q4}
\end{question}

\begin{question}
Cette question fait appel à ce que vous avez appris au secondaire sur les
polynôme de degré deux.

\subQ{a} Quelles sont les racines du polynôme $\displaystyle x^2 + 2x -2$?\\
\subQ{b} Tracez la parabole $\displaystyle y=x^2+2x-2$.  N'oubliez
pas d'indiquer sur votre graphe l'intersection avec les axes.  S'il y
a une valeur maximale ou minimale, indiquez la sur le graphe.
\label{2Q5}
\end{question}

\begin{question}
Quelle valeur devez-vous donner à $c$ dans le polynôme $4x^2 -12x + c$
pour que les racines soient égales?
\label{2Q6}
\end{question}

\begin{question}
Résolvez les équations suivantes.
\begin{center}
\begin{tabular}{*{2}{l@{\hspace{0.5em}}l@{\hspace{3em}}}l@{\hspace{0.5em}}l}
\subQ{a} & $7\times 5^{3x} = 21$ &
\subQ{b} & $4\times 3^{-2x+1} = 7\time 3^{3x}$ &
\subQ{c} & $\displaystyle \log_{10}(100^x) = 3$
\end{tabular}
\end{center}
\label{2Q7}
\end{question}

\begin{question}
Quelles sont les valeurs de $x$ pour lesquelles
$\displaystyle e^x + 2e^{-x} = 3$?
\label{2Q8}
\end{question}

\begin{question}
Résolvez les équations suivantes.
\begin{center}
\begin{tabular}{*{1}{l@{\hspace{0.5em}}l@{\hspace{6em}}}l@{\hspace{0.5em}}l}
\subQ{a} & $\displaystyle \frac{2x}{x+1} = \frac{2x-1}{x}$ &
\subQ{b} & $\displaystyle \frac{1}{x+1} + \frac{3}{x-1} = 4$ \\[1em]
\subQ{c} & $\ln(x-3)+\ln(x-5) = \ln(2x-6)$ &
\subQ{d} & $|x-2| = 5$ \\[1em]
\subQ{e} & $|x-2| = |2x-5|$ &
\subQ{f} & $|x^2 - 4| = 4$ \\[1em]
\subQ{g} & $\ln(2) - \ln(x) = \ln(3-x)$ & &
\end{tabular}
\end{center}
\label{2Q9}
\end{question}

\begin{question}
Résolvez les inégalités suivantes.
\begin{center}
\begin{tabular}{*{2}{l@{\hspace{0.5em}}l@{\hspace{3em}}}l@{\hspace{0.5em}}l}
\subQ{a} & $\displaystyle 5 -\frac{6}{x} < 4$ &
\subQ{b} & $x^2 -4x + 3 > 0$ &
\subQ{c} & $\displaystyle \frac{2}{x+1} >\frac{1}{x-3}$ \\[1em]
\subQ{d} & $\displaystyle \frac{x^2}{x+2}  < 1$ &
\subQ{e} & $\displaystyle \frac{x}{x-3}  < \frac{-6}{x+1}$  &
\subQ{f} & $|x-2| < 5$ \\[1em]
\subQ{g} & $|x^2-2x - 5| < |x-1|$ & & & &
\end{tabular}
\end{center}
\label{2Q10}
\end{question}

\subsection{Functions}

\begin{question}
Quel est l'image de la fonction $f(x) = 3 + 2x-x^2$?
\label{2Q11}
\end{question}

\begin{question}
Une fonction $f$ est définie par le graphe ci-dessous.  Déterminez
si $f$ est injective.
\PDFgraph{2_fonctions/probl167}
\label{2Q12}
\end{question}

\begin{question}
Déterminez si les fonctions suivantes ont un inverse.  Si une fonction
a un inverse, trouvez cet inverse.  Bien indiquer le domaine et
l'image de la fonction et de son inverse.
\begin{center}
\begin{tabular}{*{2}{l@{\hspace{0.5em}}l@{\hspace{3em}}}l@{\hspace{0.5em}}l}
\subQ{a} & $\displaystyle g(x) =5^{x^3}$ &
\subQ{b} & $\displaystyle f(x) = \sqrt{ \frac{x+1}{x-1} }$ &
\subQ{c} & $\displaystyle h(x) = \left(\frac{x}{x+1}\right)^{10}$ \\
\subQ{d} & $\displaystyle y=f(x) = \left(\frac{2x+7}{3}\right)^{-1/2}$ &
& & &
\end{tabular}
\end{center}
\label{2Q13}
\end{question}

\begin{question}
Quelle est la forme la plus simple pour $(f\circ g)(x)$ où
$\displaystyle f(x) = \sqrt{x^2-1}$ et $g(x) = \sqrt{x^2+1}$?
\label{2Q14}
\end{question}

\begin{question}
Écrivez les fonctions suivantes comme la composition de fonctions
simples.  Pour nous, les fonctions simples sont $f(x)=x^a$,
$a^x$, $\log_b(x)$, ..., et le produit et la somme de telles
fonctions.
\begin{center}
\begin{tabular}{*{1}{l@{\hspace{0.5em}}l@{\hspace{6em}}}l@{\hspace{0.5em}}l}
\subQ{a} & $\displaystyle f(x) = \frac{5}{1+5^x}$ &
\subQ{b} & $\displaystyle h(t) = (1-t^2)^{-4}$
\end{tabular}
\end{center}
\label{2Q15}
\end{question}

\begin{question}[\eng \life]
Écrivez la fonction $\displaystyle g(x) = \cos(\sqrt{1+x^2})$ comme la
composition de fonctions simples.  Pour nous, les fonctions simples
sont $f(x)=x^a$, $a^x$, $\log_b(x)$, $\cos(x)$, ..., et le produit et
la somme de telles fonctions.
\label{2Q16}
\end{question}

\subsection{Trigonométrie}

\begin{question}[\eng \life]
Vérifiez les identités trigonométriques suivantes pour
$\theta = 0$, $\pi/4$, $\pi/2$ et $\pi$.

\subQ{a} $\displaystyle \cos(\theta-\pi) = - \cos(\theta)$.

\subQ{b} $\cos(2\theta) = \cos^2(\theta) - \sin^2(\theta)$.
\label{2Q17}
\end{question}

\begin{question}[\eng \life]
Si $\cos(\theta)=1/2$, quelles sont les valeurs possibles pour
$\sin(\theta)$?
\label{2Q18}
\end{question}

\begin{question}[\eng \life]
Pour chacun des graphes ci-dessous, donnez la moyenne, l'amplitude, la
période et la phase de la fonction sinusoïdale qui possède ce graphe.
Puis, écrivez la fonction sinusoïdale dans chacun des cas.
\begin{center}
\begin{tabular}{*{1}{l@{\hspace{0em}}l@{\hspace{2em}}}l@{\hspace{0em}}l}
\subQ{a} & \MATHgraphRow{2_fonctions/sinusoidale1_a}{7cm}{-5cm} &
\subQ{b} & \MATHgraphRow{2_fonctions/sinusoidale1_b}{7cm}{-5cm} \\
\subQ{c} & \MATHgraphRow{2_fonctions/sinusoidale1_c}{7cm}{-5cm} &
\subQ{d} & \MATHgraphRow{2_fonctions/sinusoidale1_d}{7cm}{-5cm}
\end{tabular}
\end{center}
\label{2Q19}
\end{question}

\begin{question}[\eng \life]
Écrivez les fonctions sinusoïdales suivantes sous la forme standard
\[
  f(t) = M + A \cos\left( \frac{2\pi}{P}\left( t- T\right)\right)
\]
avec $A>0$.
\begin{center}
\begin{tabular}{*{1}{l@{\hspace{0.5em}}l@{\hspace{6em}}}l@{\hspace{0.5em}}l}
\subQ{a} & $f(t) = 2 - \cos(t)$ & \subQ{b} & $f(t) = 2 + \sin(t)$ \\
\subQ{c} & $\displaystyle f(x) = -3 + 5\sin\left(\frac{\pi}{2}(2-x)\right)$ &
\end{tabular}
\end{center}
\label{2Q20}
\end{question}

\begin{question}[\eng \life]
Pour chacune des fonctions sinusoïdale suivantes, déterminez la
moyenne, l'amplitude, la période et la phase.  De plus, Tracez le
graphe de la fonction.  Bien indiquer sur votre dessin la moyenne,
l'amplitude, la période et la phase.
\begin{center}
\begin{tabular}{*{1}{l@{\hspace{0.5em}}l@{\hspace{3em}}}l@{\hspace{0.5em}}l}
\subQ{a} & $\displaystyle h(z) = 1 + 5 \cos\left(\frac{\pi z}{2} -
  \frac{3\pi}{2}\right)$ &
\subQ{b} & $f(x)= 6 \sin(3x-6) -4$\\[1em]
\subQ{c} & $W(y) = -2.0 + 3.0\cos\left(4\pi(y+0.1)\right)$ & &
\end{tabular}
\end{center}
\label{2Q21}
\end{question}

\begin{question}
Quelle est la fonction sinusoïdale $f$ qui oscille entre $36.6$ et
$37.2$, et dont la période est $24$ et la phase est $13$?
Tracez le graphe de cette fonction en indiquant la période,
l'amplitude, la moyenne et la phase sur votre graphe.
\label{2Q22}
\end{question}

\begin{question}[\eng \life]
Quel est le cosinus de l'angle $\theta$ dans le dessin suivant?
\PDFgraph{2_fonctions/triangle1}
\label{2Q23}
\end{question}

\begin{question}[\eng \life]
Si $\displaystyle \sin(\theta)=\frac{1}{x}$, quelles sont les valeurs
possible pour $\tan(\theta)$?
\label{2Q24}
\end{question}

\begin{question}[\eng \life]
Montrez que
$\displaystyle \tan(\arcsin(x)) = \frac{x}{\sqrt{1-x^2}}$ pour $-1<x<1$.
\label{2Q25}
\end{question}

\subsection{Fonctions exponentielles}

\begin{question}
Quelques valeurs de deux fonctions sont données dans le tableau
ci-dessous.  Une des fonctions est linéaire (de la forme $mx+b$) et
l'autre est exponentielle (de la forme $p\, b^x$).  Trouvez une formule
pour chacune des fonctions.
\[
\begin{array}{c|cccc}
$x$ & 2& 4& 6& 8 \\
\hline
$f(x)$ & 0.75& 0.1875 & 0.046875 & 0.011719 \\
\hline
$g(x)$ & 5.7& 10.3& 14.9& 19.5
\end{array}
\]
\label{2Q26}
\end{question}

\begin{question}
Le poids $S$ en kilogrammes d'un individu au temps $t$ en jours est
donné par la formule $\displaystyle S(t) = S_0 10^{\alpha t}$ où le
poids initial est $S_0 = S(0) = 2.34$ kg et le taux de croissance
par rapport au poids de l'individu est $\alpha = 0.693$.  Combien de temps
faut-il pour que le poids de l'individu double?  Pour que le poids
soit $10$ fois plus grand?
\label{2Q27}
\end{question}

\begin{question}
La population d'un territoire double à tous les $24$ ans.  Si la
population initiale est de $500$ individus (par km$^2$), quel sera le
nombre d'individus (par km$^2$) après $12$ années? Il faut se rappeler
que le nombre d'individus dans la population au temps $t$ en années
est donné par une formule de la forme
$\displaystyle p(t) = p_0 e^{\alpha t}$ où $p_0$ est le nombre initial
d'individus et $\alpha$ est une constante représentant le taux de
croissance par rapport à la taille de la population.

Suggestion: vous pouvez déterminer $\alpha$ à l'aide du temps
nécessaire pour que la population double.
\label{2Q28}
\end{question}

\begin{question}
Le volume (en cm$^3$) d'un organisme au temps $t$ (en seconde) est
donné par l'expression $V(t) = V_0 e^{\alpha t}$.  Quelles sont les
unités de $\alpha$?  Si $V_0 = 2$ et $\alpha = 0.1$, combien faut-il
de temps pour que le volume double?  Pour qu'il quadruple?
\label{2Q29}
\end{question}

\begin{question}
La quantité de Carbone-14 (en atomes) que nous retrouvons dans les os
d'un animal, $t$ années après son décès, est donnée par
$\displaystyle Q(t) = Q_0 e^{-0.000122\; t}$ où $Q_0$ est la quantité de
carbone-14 au moment du décès.  Quelle est la demie-vie du carbone-14?
\label{2Q30}
\end{question}

\begin{question}
Si une population a une demie-vie de $43$ ans et le nombre d'individus
au départ est de $1600$ (par m$^2$), combien de temps faut-il pour que
la population soit de $200$ individus (par m$^2$)?  Trouvez une
formule pour le nombre $P(t)$ d'individus (par m$^2$) au temps $t$
dans cette population.  Combien faut-il de temps pour que la
population soit de $437$ individus (par m$^2$)?
\label{2Q31}
\end{question}

\begin{question}[\eng \life]
Tracez le graphe de la fonction $g(t) = e^t \cos(2\pi t)$ pour $0<t<3$.
Il n'est pas nécessaire que le graphe soit très précis mais il doit
bien décrire le comportement qualitatif de la fonction.
\label{2Q32}
\end{question}

\begin{question}[\eng \life]
Tracez le graphe de la fonction $W(t) = e^{-t}\cos(2\pi t)$ pour $t\geq 0$.
Il n'est pas nécessaire que le graphe soit très précis mais il doit
bien décrire le comportement qualitatif de la fonction.
\label{2Q33}
\end{question}


%%% Local Variables: 
%%% mode: latex
%%% TeX-master: "notes"
%%% End: 

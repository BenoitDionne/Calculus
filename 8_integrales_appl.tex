\chapter[Applications de l'intégrale]{Applications de l'intégrale}

\compileTHEO{

Nous présentons plusieurs des applications de l'intégrale d'une fonction
dans ce chapitre.  Au risque de ce répéter, les applications
spécifiquement pour les sciences de la vie sont identifiées par le
symbole  \life  alors que ceux spécifiquement pour les domaines de la
physique et du génie sont identifiées par le symbole \eng.  Comme
c'était le cas pour les applications de la dérivée d'une fonction au
chapitre~\ref{chapApplDer}, la classification des applications est
très académique et arbitraire.  La majorité des applications dans ce
chapitre proviennent de la physique et du génie.  Il n'en reste pas
moins que certaines de ces applications ont aussi une utilité dans les
sciences de la vie.

\section{Aire entre deux courbes}

Pour estimer l'aire $A$ de la région $R$ entre les courbes $y=f(x)$ et
$y=g(x)$, et les droites $x=a$ et $x=b$, nous
pouvons procéder comme nous l'avons fait pour estimer l'aire sous la
courbe $y=e^{-x}$ au chapitre précédent.  Supposons que
$f(x) \geq g(x)$ pour tout $x$.

Soit $k$, un entier positif.  Posons $\Delta x = (b-a)/k$ et
$x_i = a + i \Delta x$ pour $i=0$, $1$, $2$, \ldots, $k$.  Nous
obtenons une partition de l'intervalle $[a,b]$ en sous-intervalles de la forme
$[x_i,x_{i+1}]$ pour $i=0$, $1$, \ldots, $k-1$.   Pour chaque valeur
de $i$, nous choisissons $x_i^\ast$ dans l'intervalle $[x_i,x_{i+1}]$
et définissons le rectangle
\[
R_i = \{ (x,y) : x_i\leq x\leq x_{i+1} \text{ et }
g(x_i^\ast) \leq  y \leq f(x_i^\ast) \} \; .
\]
L'aire d'un tel rectangle est $(f(x_i^\ast)-g(x_i^\ast))\Delta x$
(figure~\ref{AIRE1}).

\PDFfig{8_integrales_appl/aire1}{Estimation de l'aire entre deux courbes à
l'aide des sommes de Riemann}{Estimation de l'aire entre
deux courbes à l'aide des sommes de Riemann si les courbes ne se
croissent pas.}{AIRE1}

La somme de l'aire de chaque rectangle $R_i$ nous donne une approximation de
l'aire $A$ de la région $R$ entre les courbes $y=f(x)$ et $y=g(x)$, et les
droites $x=a$ et $x=b$.
\[
A \approx \sum_{i=0}^{k-1} \,
\left(f(x_i^\ast)-g(x_i^\ast)\right)\Delta x \; .
\]
De plus, nous remarquons que cette somme est une somme de Riemann pour
l'intégrale
\[
\int_a^b (f(x)-g(x))\dx{x} \; .
\]

Ainsi, si $k$ tend vers plus l'infini, nous obtenons que
\[
A = \int_a^b (f(x)-g(x))\dx{x} \; .
\]

Si $f(x) \geq g(x)$ pour tout $x$, alors
$f(x_i^\ast)-g(x_i^\ast) \geq 0$ est bien la hauteur du rectangle
$R_i$.  Par contre, dans le cas où $f(x) < g(x)$ pour certaines
valeurs de $x$ (figure~\ref{AIRE2}), nous pourrions avoir que
$f(x_i^\ast) < g(x_i^\ast)$ pour certaines valeurs de $i$.
La hauteur du rectangle $R_i$ est alors donnée par
$|f(x_i^\ast)-g(x_i^\ast)|$.  Nous obtenons le résultat suivant.

\PDFfig{8_integrales_appl/aire2}{Estimation de l'aire entre deux courbes à
l'aide des sommes de Riemann si les courbes se croissent}{Estimation
de l'aire entre deux courbes à l'aide des sommes de Riemann si les
courbes se croissent}{AIRE2}

\begin{focus}{\mth}
L'aire $A$ de la région $R$ entre les courbes $y=f(x)$ et $y=g(x)$, et
entre les droites $x=a$ et $x=b$ est donnée par
\[
A = \int_a^b |f(x)-g(x)|\dx{x} \; .
\]
\end{focus}

La procédure que nous avons utilisée pour obtenir la formule ci-dessus peut
évidemment être modifiée pour donner la formule suivante.

\begin{focus}{\mth}
L'aire $A$ de la région $R$ entre les courbes $x=f(y)$ et $x=g(y)$, et
entre les droites $y=a$ et $y=b$ est donnée par
\[
A = \int_a^b |f(y)-g(y)|\dx{y} \; .
\]
\end{focus}

\begin{egg}
Trouvons l'aire $A$ de la région $R$ bornée par les courbes $y=f(x)=x^2$ et
$y=g(x)=2-x^2$, et les droites $x=0$ et $x=2$.

Nous retrouvons le graphe de ces deux courbes à la figure~\ref{AIRE3}.
Remarquons que $f(x)<g(x)$ pour $0\leq x < 1$ et $f(x)>g(x)$ pour $1<x\leq 2$.
Le point $(1,1)$ est le point d'intersection des deux courbes que nous
pouvons trouver en résolvant $f(x)=g(x)$.  Ainsi,
\begin{align*}
A &= \int_0^2 |f(x)-g(x)|\dx{x} = \int_0^1 (g(x)-f(x))\dx{x}
+ \int_1^2 (f(x)-g(x))\dx{x} \\
&= \int_0^1 (2-x^2-x^2)\dx{x} + \int_1^2(x^2-(2-x^2))\dx{x}
= \int_0^1 (2-2x^2)\dx{x} + \int_1^2(2x^2-2)\dx{x} \\
&= \left( 2x-\frac{2}{3} \, x^3\right)\bigg|_{x=0}^1 +
\left( \frac{2}{3}\,x^3 - 2x \right)\bigg|_{x=1}^2 = 4 \; .
\end{align*}
\end{egg}

\PDFfig{8_integrales_appl/aire3}{Région bornée par les courbes $y=x^2$ et
$y=2-x^2$, et les droites $x=0$ et $x=2$}{Région bornée par les courbes
$y=x^2$ et $y=2-x^2$, et les droites $x=0$ et $x=2$}{AIRE3}

\begin{egg}
Trouvons l'aire $A$ de la région $R$ bornée par les courbes $y=f(x)=x$ et
$y=g(x)=\sin(x)$, et les droites $x=-\pi/4$ et $x=\pi/2$.

Le dessin de la région $R$ est donné à la figure~\ref{AIRE5}.  Remarquons
que $f(x)>g(x)$ pour $0< x \leq \pi/2$ et $f(x)<g(x)$ pour $-\pi/4<x <0$. 
Ainsi,
\begin{align*}
A &= \int_{-\pi/4}^{\pi/2} |f(x)-g(x)|\dx{x} = \int_{-\pi/4}^0 (g(x)-f(x))\dx{x}
+ \int_0^{\pi/2} (f(x)-g(x))\dx{x} \\
&= \int_{-\pi/4}^0 (\sin(x)-x) \dx{x} + \int_0^{\pi/2}(x-\sin(x))\dx{x} \\
&= \left( -\cos(x)- \frac{x^2}{2}\right)\bigg|_{x=-\pi/4}^0 +
\left( \frac{x^2}{2} + \cos(x) \right)\bigg|_{x=0}^{\pi/2}
= \frac{5\pi^2}{32} + \frac{\sqrt{2}}{2} - 2 \; .
\end{align*}
\end{egg}

\PDFfig{8_integrales_appl/aire5}{Région bornée par les courbes $y=x$ et
$y=\sin(x)$ pour $-\pi/4\leq x\leq \pi/2$}{Région bornée par les courbes
$y=x$ et $y=\sin(x)$ pour $-\pi/4\leq x\leq \pi/2$.}{AIRE5}

\begin{egg}
Trouvons l'aire $A$ de la région $R$ bornée par la parabole $4x+y^2=0$ et
la droite $y=2x+4$.

Pour trouver les valeurs de $x$ aux points d'intersection de la parabole
avec la droite, nous substituons $y=2x+4$ dans $4x+y^2=0$ pour obtenir
$4x +(2x+4)^2 = 4(x+4)(x+1) = 0$.  Donc $x=-4$ ou $x=-1$.  La parabole
et la droite se coupent aux points $(-1,2)$ et $(-4,-4)$.  Le dessin
de la région $R$ est donné à la figure~\ref{AIRE6}.

Remarquons que pour calculer l'aire
$\displaystyle A = \int_{-4}^0 |f(x)-g(x)|\dx{x}$, il faudrait utiliser
$f(x) = 2x+4$ et $g(x) = -2\sqrt{-x}$ entre $-4$ et $-1$, et
$f(x) = 2\sqrt{-x}$ et $g(x) = -2\sqrt{-x}$ entre $-1$ et $0$.  Il est donc
plus simple d'utiliser une intégrale par rapport à $y$ pour calculer l'aire
de la région $R$.

Si nous résolvons $4x+y^2=0$ et $y=2x+4$ pour $x$, nous obtenons les fonctions
$x=f(y) = -y^2/4$ et $x=g(y) = (y-4)/2$.  Nous avons $f(y)>g(y)$ pour
$-4< y < 2$.
Ainsi,
\[
A = \int_{-4}^2 |f(y)-g(y)|\dx{y} =
\int_{-4}^2 \left( -\frac{y^2}{4} - \frac{y-4}{2} \right)\dx{x}
= \left( -\frac{y^3}{12} - \frac{(y-4)^2}{4} \right)\bigg|_{x=-4}^2
= 9 \ .
\]
\end{egg}

\PDFfig{8_integrales_appl/aire6}{Région bornée par la parabole $4x-y^2=0$ et
la droite $y=2x+4$}{Région bornée par la parabole $4x-y^2=0$ et la
droite $y=2x+4$.}{AIRE6}

\section{Valeur moyenne d'une fonction }

\begin{focus}{\dfn} \index{Valeur moyenne}
La {\bfseries valeur moyenne} d'une fonction $f$ définie sur un intervalle
$[a,b]$ est
\[
M_f = \frac{1}{b-a} \int_a^b f(x)\dx{x} \; .
\]
\end{focus}

Pour justifier cette définition, rappelons que la formule pour calculer la
moyenne de $k$ nombres $m_1$, $m_2$, \ldots, $m_k$ est
\[
\frac{1}{k} \sum_{i=1}^k m_i \; .
\]

Pour estimer la valeur moyenne d'une fonction sur un intervalle, il suffit de
choisir un très grand nombre de points dans l'intervalle et de faire la
moyenne des valeurs de la fonction évaluée à tous ces points.  Plus nous
utilisons de points, plus nous sommes près de la valeur moyenne de la fonction.

Soit $k$, un entier positif.  Posons $\Delta x = (b-a)/k$ et
$x_i = a + i \Delta x$ pour $i=0$, $1$, $2$, \ldots, $k$.  Nous obtenons une
partition de l'intervalle $[a,b]$ en sous-intervalles de la forme
$[x_i,x_{i+1}]$ pour $i=0$, $1$, \ldots, $k-1$.  Nous choisissons $x_i^\ast$
dans l'intervalle $[x_i,x_{i+1}]$ pour $i=0$, $1$, $2$, \ldots, $k-1$.  La
valeur moyenne $M_f$ de $f$ sur l'intervalle $[a,b]$ est approximativement la
moyenne de $f(x_0^\ast)$, $f(x_1^\ast)$, \ldots, $f(x_{k-1}^\ast)$.
Nous avons donc
\[
M_f \approx \frac{1}{k} \sum_{i=0}^{k-1} f(x_i^\ast) \; .
\]
Or $k = (b-a)/\Delta x$.  Si nous substituons dans l'expression
précédente, nous trouvons
\[
M_f \approx \frac{1}{b-a} \sum_{i-0}^{k-1} f(x_i^\ast) \Delta x \; .
\]
La somme que nous retrouvons dans l'expression précédente est une somme de
Riemann pour $\displaystyle \int_a^b f(x) \dx{x}$.  Ainsi, si $k$ tend vers
plus l'infini, nous obtenons
\[
M_f = \frac{1}{b-a} \int_a^b f(x) \dx{x} \; .
\]

\begin{egg}
Quelle est la valeur moyenne de $f(x)=x^2$ sur l'intervalle $[-1,1]$?
\[
M_f = \frac{1}{2} \int_{-1}^1 x^2 \dx{x} =
\frac{x^3}{6}\bigg|_{x=-1}^1 = \frac{1}{3} \; .
\]
\end{egg}

Comme pour la dérivée d'une fonction, il y a un théorème de la moyenne pour
l'intégrale d'une fonction

\begin{focus}[][Théorème de la moyenne pour l'intégrale]{\thm}
\index{Théorème de la moyenne pour l'intégrale}
Si $f$ est une fonction continue sur l'intervalle $[a,b]$, il existe un point
$c \in [a,b]$ tel que
\[
f(c) = \frac{1}{b-a} \int_a^b f(x) \dx{x} \; .
\]
C'est-à-dire que $f(c)$ est la valeur moyenne de $f$ sur l'intervalle
$[a,b]$.
\end{focus}

La conclusion du théorème précédent est illustrée à la figure~\ref{MOY_INT}.
Cette figure est une représentation typique de la valeur moyenne d'une
fonction sur un intervalle.  L'aire sous la courbe $y=f(x)$ et au-dessus de
la droite $y=M_f$ (i.e. la région en bleu) est égale à l'aire
au-dessus de la courbe $y=f(x)$ et sous la droite $y=M_f$ (i.e. la
région en rouge).  En effet, puisque
\[
\frac{1}{b-a} \int_a^b M_f \dx{x} = M_f = \frac{1}{b-a}\int_a^b f(x) \dx{x}
\]
nous obtenons
\[
0 = \frac{1}{b-a}\int_a^b (f(x)-M_f) \dx{x} \; .
\]
Pour que cette intégrale soit nulle, il faut que l'aire sous la courbe
$y=f(x)-M_f$ et au-dessus de l'axe des $x$ soit égale à l'aire au-dessus de
la courbe $y=f(x)-M_f$ et sous l'axe des $x$.  Avec une translation de $M_f$
vers le haut, la courbe $y = f(x)-M_f$ devient le graphe de $f$ et
l'axe des $x$ devient la droite $y=M_f$.  L'énoncé sur l'égalité
entre l'aire des régions en bleu et l'aire des régions en rouge est
une conséquence de la propriété qu'ont les translations de préserver
l'aire.

\PDFfig{8_integrales_appl/int_mvt}{Représentation graphique du théorème de la
moyenne pour l'intégrale}{Il existe un point $c\in [a,b]$ tel que $f(c)$ est
la valeur moyenne de $f$ sur l'intervalle $[a,b]$}{MOY_INT}

\begin{egg}
Soit $f(x) = -x^2+4$, trouvons un point $c$ de l'intervalle $[0,2]$ tel que
$f(c)$ est la valeur moyenne de $f$ sur l'intervalle $[0,2]$.

\[
M_f = \frac{1}{2} \int_0^2 (-x^2+4) \dx{x}
= \frac{1}{2} \left(-\frac{x^3}{3} + 4x\right)\bigg|_{x=0}^2
= \frac{8}{3} \; .
\]
Le point $c$ est donné par $f(c) = -c^2+4 = 8/3$.  Si nous résolvons
pour $c$ dans l'intervalle $[0,2]$, nous trouvons $c= 2/\sqrt{3}$.

Le graphe de $f$ ainsi que la représentation de la valeur moyenne sont donnés
à la figure~\ref{MOY_INT2}.
\end{egg}

\PDFfig{8_integrales_appl/int_mvt2}{La valeur moyenne de $f(x) = -x^4 + 4$ sur
l'intervalle ${[0,2]}$ est $8/3$}{Graphe de $f(x) = -x^4 + 4$. La valeur
moyenne de $f$ sur l'intervalle $[0,2]$ est $8/3$.  L'aire sous la courbe
$y=f(x)$ et au-dessus de la droite $y=8/3$ (en bleu) est égale à
l'aire au dessus de la courbe $y=f(x)$ et sous la droite $y = 8/3$ (en
rouge) pour $0\leq x \leq 2$.}{MOY_INT2} 

\section{Volume d'un objet \eng}

Nous pouvons utiliser l'intégration pour calculer le volume $V$ d'un
solide $S$ comme celui représenté à la figure~\ref{VOL2}.  Supposons que le
solide $S$ soit contenu entre les plans $x=a$ et $x=b$.

Pour estimer le volume $V$ du solide $S$, nous partageons le solide
$S$ en fines tranches parallèles au plan $y$,$z$ et nous estimons le
volume de chacune de ces tranches.  La somme de ces approximations
nous donne une approximation du volume de $S$.

Soit $k$, un entier positif.  Posons $\Delta x = (b-a)/k$ et
$x_i = a + i \Delta x$ pour $i=0$, $1$, $2$, \ldots, $k$.  Nous
obtenons une partition de l'intervalle $[a,b]$ en sous-intervalles de
la forme $[x_i,x_{i+1}]$ pour $i=0$, $1$, \ldots, $k-1$.

La $i^e$ tranche du solide $S$ qui est représenté à la figure~\ref{VOL2} est
donnée par l'intersection du solide $S$ avec la région définie par
$\{ (x,y,z) : x_i \leq x \leq x_{i+1}\}$.

\PDFfig{8_integrales_appl/vol2}{Une tranche d'un solide $S$ quelconque}{La
tranche du solide $S$ pour $x_i \leq x \leq x_{i+1}$. Le volume de cette
tranche est $V_i$.}{VOL2}

Pour estimer le volume $V_i$ de la $i^e$ tranche, nous choisissons
$x_i^\ast$ dans l'intervalle $[x_i,x_{i+1}]$.  Nous considérons le
{\em cylindre} $S_i$ qui est représenté à la figure~\ref{VOL1}, dont la
hauteur est $\Delta x$ et la base est la section $B_i$ obtenue de
l'intersection du plan $x=x_i^\ast$ avec le solide $S$.

\PDFfig{8_integrales_appl/vol1}{L'approximation d'une tranche d'un solide}
{$S_i$ est approximativement une tranche du solide $S$.  L'épaisseur de $S_i$
est $\Delta x$.}{VOL1}

Nous retrouvons à la figure~\ref{VOL3} un exemple possible pour
la section $B_i$ du cylindre $S_i$; notons que $B_i$ n'est pas toujours un
disque comme nous avons dans la figure.  Les côtés du cylindre
$S_i$ sont parallèles à l'axe des $x$.

\PDFfig{8_integrales_appl/vol3}{La section d'un cylindre engendrée par
l'intersection avec un plan}{La section $B_i$ est engendrée par
l'intersection du plan $x=x_i^\ast$ avec le solide $S$.  L'aire de la section 
est $A(x_i^\ast)$.}{VOL3}

Le volume du cylindre $S_i$ est $A(x_i^\ast) \Delta x$ où $A(x_i^\ast)$ est
l'aire de la section $B_i$.  Pour $\Delta x$ petit, le volume $V_i$ de la
$i^e$ tranche de $S$ est approximativement le volume de $S_i$.  Ainsi,
\[
V_i \approx A(x_i^\ast) \Delta x
\]
pour $\Delta x$ petit.  La somme des volumes des cylindres $S_i$ donne un
estimé du volume $V$ de $S$.
\[
V = \sum_{i=0}^{k-1} V_i \approx \sum_{i=0}^{k-1} A(x_i^\ast) \Delta x
\]
Plus $k$ sera grand (et donc $\Delta x$ petit), plus la somme ci-dessus sera
près du volume $V$ du solide $S$.  À la limite, nous obtenons le volume $V$ de
$S$.

Mais, ces sommes représentent aussi des sommes de Riemann pour l'intégrale
\[
\int_a^b A(x)\dx{x}
\]
où $A(s)$ est l'aire de la section engendrée par l'intersection du plan $x=s$
avec le solide $S$.  Ainsi, si $k$ tend vers plus l'infini, nous
obtenons le résultat suivant.

\begin{focus}[][des tranches]{\mth}
Le volume $V$ d'un solide $S$ comme celui représenté à la
figure~\ref{VOL3} est donné par
\begin{equation}\label{comp_vol}
V = \int_a^b A(x)\dx{x}
\end{equation}
où $A(s)$ est l'aire de la section engendrée par l'intersection du
plan $x=s$ avec le solide $S$.
\end{focus}

\begin{egg}
Trouvons le volume $V$ du tronc de pyramide $S$ représenté à la
figure~\ref{TRONC} dont la base est un carré qui possède des côtés de
longueur $a$ et le sommet est aussi un carré mais qui possède des côtés de
longueur $b$.  La hauteur du tronc de pyramide est $h$.

Selon la façon dont nous tranchons $S$, nous obtenons des sections de
niveaux de complexité différents.  La façon de trancher $S$ qui donne
les sections les plus simples est de trancher horizontalement.  Les
sections sont alors des carrés.

Pour utiliser la formule (\ref{comp_vol}), il faut trouver une formule pour
l'aire $A(x)$ des sections obtenues par l'intersection de $S$ avec le plan
$x$ constant.  Comme nous pouvons le voir a partir du dessin d'une
coupe verticale de $S$ que nous retrouvons à la figure~\ref{TRONC}, la
longueur $y$ d'un côté d'une section varie linéairement en fonction de
$x$,  Nous avons $y=a$ pour $x=0$
et $y=b$ pour $x=h$.  La droite qui passe par ces deux points est de pente 
$m = (b-a)/(h-0) = (b-a)/h$.  Ainsi, son équation est
\[
y = \left(\frac{b-a}{h}\right) x + a
\]
car $y=a$ lorsque $x=0$.  Nous obtenons
\[
A(x) = \left( \left(\frac{b-a}{h}\right) x + a \right)^2 \; .
\]

Le volume $V$ de $S$ est donc
\begin{align*}
V &= \int_0^h \left( \left(\frac{b-a}{h}\right) x + a \right)^2 \dx{x}
= \frac{h}{3(b-a)}
\left( \left(\frac{b-a}{h}\right) x + a \right)^3\bigg|_{x=0}^h \\
&= \frac{h}{3(b-a)} \left( b^3 - a^3 \right)
= \frac{h}{3} \left( b^2+ab+a^2 \right)
\end{align*}
où nous avons utilisé la substitution $y = (b-a)x/h + a$ pour évaluer
l'intégrale.
\end{egg}

\PDFfig{8_integrales_appl/tronc}{Le tronc de pyramide et une section
perpendiculaire de ce tronc}{Le tronc de pyramide et une
section perpendiculaire de ce tronc.}{TRONC}

\begin{egg}
Trouvons le volume $V$ du solide $S$ dont la base est la région du plan
$x$,$y$ bornée par les courbes $y=x^2$ et $y=8-x^2$, et les sections
perpendiculaires à l'axe des $x$ sont des carrés.  Nous retrouvons une
représentation de ces sections à la figure~\ref{VOL4}.

Puisque les deux courbes se coupent à $x=-2$ et $x=2$ (les solutions
de $x^2=8-x^2$), la base du solide est la région entre les courbes $y=x^2$ et
$y=8-x^2$ pour $-2\leq x \leq 2$.

La section pour $x$ constant est un carré dont la longueur des côtés est
$(8-x^2)-x^2 = 8 - 2 x^2$.  L'aire de cette section est donc
$A(x)= (8-2x^2)^2$.  Le volume du solide est
\begin{align*}
V &= \int_{-2}^2 A(x)\dx{x} = \int_{-2}^2 (8-2x^2)^2\dx{x}
= \int_{-2}^2 \left(64 -32 x^2 + 4 x^4\right)\dx{x} \\
&= \left(64x - \frac{32}{3}\,x^3 + \frac{4}{5}\,x^5\right)\bigg|_{x=-2}^2
= \frac{2^{11}}{15} \; .
\end{align*}
\end{egg}

\PDFfig{8_integrales_appl/vol4}{Solide de base donnée dont les sections
perpendiculaires à l'axe des $x$ sont des carrés}{Solide dont la base est
bornée par les courbes $y=x^2$ et $y=8-x^2$, et les sections perpendiculaires
à l'axe des $x$ sont des carrés.}{VOL4}

% \begin{egg}
% Trouvons le volume $V$ du solide $S$ dont la base est la région du plan
% $x$,$y$ bornée par la courbe $y=x^{1/3}$, l'axe des $x$ et les droites $x=0$
% et $x=1$.  Les sections de $S$ perpendiculaires à l'axe des $x$ sont des
% carrés.
% \end{egg}

\subsection{Solides de révolution}

Nous pouvons utiliser (\ref{comp_vol}) pour trouver le volume d'un solide obtenu
par la rotation d'une région plane autour d'un axe.

\PDFfig{8_integrales_appl/rot1}{Un solide de révolution dont les sections
transversales sont les anneaux}{La rotation de la région $R$ autour de l'axe
$y=c$ (et $z=0$) donne une solide dont les sections perpendiculaires à
l'axe des $x$ sont les anneaux comme celui représenté à la
figure~\ref{ROT3}}{ROT1}

Supposons que la région $R$ soit la région bornée par les courbes $y=f(x)$ et
$y=g(x)$, et les droites $x=a$ et $x=b$ (figure~\ref{ROT1}).

\PDFfig{8_integrales_appl/rot2}{Un solide produit par la rotation
d'une région autour de l'axe $y=c$}{Le solide $S$ qui résulte de la
rotation de la région $R$ représenté à la figure~\ref{ROT1} autour de
l'axe $y=c$ (et $z=0$)}{ROT2}

Si nous faisons la rotation de la région $R$ autour de l'axe $y=c$ (et
$z=0$) qui ne coupe pas la région $R$, nous obtenons un solide $S$
semblable à celui illustré à la figure~\ref{ROT2}.

\PDFfig{8_integrales_appl/rot3}{Une section obtenue par l'intersection d'un
solide avec un plan où $x$ est constant}{Une section du solide $S$
représenté à la figure~\ref{ROT2} obtenue par l'intersection du solide
$S$ avec un plan où $x$ est constant.  Les sections sont donc
perpendiculaires à l'axe des $x$.}{ROT3}

Pour calculer le volume $V$ du solide $S$, il faut trouver une formule pour
l'aire $A(x)$ des sections perpendiculaires à l'axe des $x$.  Si nous
faisons la projection dans le plan $y$,$z$ d'une section, nous trouvons un
anneau centré au point $(c,0)$ dont le petit rayon est $|f(x)-c|$ et
le grand rayon par $|g(x)-c|$ (figure~\ref{ROT3}).

Puisque l'aire d'un disque de rayon $r$ est donnée par $\pi r^2$,
l'aire $A(x)$ de notre anneau est l'aire du disque de rayon $|g(x)-c|$
moins l'aire du disque de rayon $|f(x)-c|$.  Ainsi,
\[
A(x) = \pi\left(g(x)-c\right)^2 - \pi\left(f(x)-c\right)^2 \; .
\]
Nous obtenons donc la formule suivante pour calculer le volume du solide $S$.

\begin{focus}[][des anneaux]{\mth}
Soit $R$ une région bornée par les courbes $y=f(x)$ et $y=g(x)$, et les
droites $x=a$ et $x=b$.  Soit $S$ le solide généré par la rotation de la
région $R$ autour de l'axe $y=c$ qui ne coupe pas la région $R$.  Le volume
$V$ du solide $S$ est
\begin{equation}\label{vol_washer}
V = \pi \int_a^b \left|\left(g(x)-c\right)^2 -
  \left(f(x)-c\right)^2\right| \dx{x} \; .
\end{equation}
\end{focus}

\begin{egg}
Quelle est le volume $V$ du solide $S$ obtenu par la rotation autour de l'axe
$y=-1$ de la région bornée par les courbes $y=x^2+1$ et $y=3-x^2$?

Nous retrouvons à la figure~\ref{ROT1_ex} le graphe des deux courbes
$y=x^2+1$ et $y=3-x^2$, et la région $R$ qui est bornée par ces deux
courbes.  Pour
déterminer les points d'intersections de ces deux courbes, il faut résoudre
l'équation $3-x^2 = 1+x^2$ pour trouver les valeurs de $x$ où la valeur de
$y$ est la même sur les deux courbes. Ces valeurs de $x$ sont les
solutions de l'équation quadratique $2x^2-2 = 2(x^2-1)=0$.  Donc
$x=\pm 1$.  Nous obtenons le point d'intersection $(-1,2)$ pour $x=-1$ et
le point d'intersection $(1,2)$ pour $x=1$.

Calculons le volume $V$ du solide $S$ à l'aide de la formule
(\ref{vol_washer}).  Nous avons
\begin{align*}
V &= \pi \int_{-1}^1 \, \left( (3-x^2) - (-1)\right)^2
 - \left( (x^2+1) - (-1)\right)^2 \dx{x} \\
&= \pi \int_{-1}^1 \, \left( 4-x^2\right)^2-\left(x^2+2\right)^2 \dx{x}
= \pi \int_{-1}^1 \, (-12 x^2 + 12) \dx{x} \\
&= \pi \left( -4 x^3 + 12x \right)\bigg|_{x=-1}^1
= 16 \pi \; .
\end{align*}
\label{ROT_ex1}
\end{egg}

\PDFfig{8_integrales_appl/rot1_ex}{La rotation autour de l'axe $x=-1$ de la
région bornée par les deux courbes $y=x^2+1$ et $y=3-x^2$}{La rotation autour
de l'axe $x=-1$ de la région bornée par les deux courbes $y=x^2+1$ et
$y=3-x^2$ va produire un solide $S$ dans l'espace qui ressemblera à un
bagel.}{ROT1_ex}

\begin{egg}
Trouvons le volume $V$ de chacun des solides $S$ suivant.

Il est toujours préférable de dessiner la région qui fera une révolution
autour de l'axe donné.

\subQ{a} $S$ est le solide obtenu de la rotation autour de l'axe des $x$ de
la région bornée par la courbe $y=x^2-3x+2$ et l'axe des $x$.

Les valeurs de $x$ aux points d'intersection de la parabole
$y=x^2-3x+2$ avec l'axe des $x$ sont en fait les racines du
polynôme $x^2-3x+2=(x-2)(x-1)$.  Les racines sont $1$ et $2$.
Le dessin suivant représente la région $R$ bornée par
la parabole $y=x^2-3x+2$ et l'axe des $x$.
\PDFgraph{8_integrales_appl/rot8}

Calculons le volume $V$ du solide de révolution $S$ à l'aide de
la formule (\ref{vol_washer}).  Nous avons
\begin{align*}
V &= \pi \int_1^2 \left(x^2-3x+2\right)^2 \dx{x} =
\pi \int_1^2 \left( x^4-6x^3+13x^2 -12x +4\right) \dx{x} \\
&= \pi \left( \frac{x^5}{5} -\frac{3}{2} x^4 +\frac{13}{3}x^3 - 6x^2 +
  4x \right)\bigg|_{x=1}^2
= \frac{\pi}{30} \; .
\end{align*}

\subQ{b} $S$ est le solide obtenu de la rotation autour de l'axe des $x$ de
la région bornée par la courbe $y=\sin(x)$ pour $0\leq x \leq \pi$ et l'axe
des $x$.

Le dessin suivant représente la région $R$ bornée par la
courbe $y=\sin(x)$ pour $0\leq x \leq \pi$ et l'axe des $x$.
\PDFgraph{8_integrales_appl/rot10}

Le volume $V$ du solide de révolution est donc
\[
V = \pi \int_0^{\pi} \sin^2(x) \dx{x}
= \frac{\pi}{2}\int_0^\pi (1-\cos(2x)) \dx{x}
= \frac{\pi}{2} \left( x - \frac{1}{2}\; \sin(2x) \right)\bigg|_0^\pi
= \frac{\pi^2}{2} \ .
\]

\subQ{c} $S$ est le solide obtenu de la rotation autour de l'axe des $y$ de
la région bornée par la courbe $x=2y-y^2$ et l'axe des $y$.

Le dessin suivant représente la région $R$ bornée par
la parabole $x=y^2-2y$ et l'axe des $y$.
\PDFgraph{8_integrales_appl/rot11}

Nous pouvons évidemment écrire une formule identique à
(\ref{vol_washer}) où $x$ est remplacé par $y$.  Ainsi, le volume $V$
du solide de révolution est
\[
V = \pi \int_0^2 (2y-y^2)^2 \dx{y}
= \pi \int_0^2 (y^4-4y^3+4y^2)\dx{y}
= \pi \left( \frac{y^5}{5} - y^4 + \frac{4y^3}{3}\right)\bigg|_0^2
= \frac{16\pi}{15} \ .
\]

\subQ{d} $S$ est le solide obtenu de la rotation autour de l'axe $y=7$ de la
région bornée par la parabole $y=(x-4)^2+1$ et la droite $y=x-1$.

Pour obtenir les valeurs de $x$ aux points d'intersection de la parabole
$y=(x-4)^2+1$ et de la droite $y=x-1$, il faut résoudre l'équation
$x-1= (x-4)^2+1$; c'est-à-dire, l'équation
$x^2-9x+18 = (x-3)(x-6) = 0$.  Donc $x = 3$ ou $x=6$.  Nous obtenons 
les deux points d'intersection $(3,2)$ et $(6,5)$.  Le dessin suivant
représente la région $R$ bornée par la parabole $y=(x-4)^2+1$ et la
droite $y=x-1$.
\PDFgraph{8_integrales_appl/rot12}
Ainsi, le volume $V$ du solide de révolution est
\begin{align*}
V &= \pi \int_3^6 \left( ((x-4)^2+1-7)^2 - (x-1-7)^2\right) \dx{x} \\
&= \pi \int_3^6 \left( x^4 -16 x^3 + 83 x^2 -144 x + 36 \right) \dx{x} \\
&= \pi \left( \frac{x^5}{5} - 4 x^4 + \frac{83x^3}{3} - 72 x^2 +
36x\right)\bigg|_3^6 = 39.6 \pi \ .
\end{align*}

\subQ{e} $S$ est le solide obtenu de la rotation autour de l'axe $y=-1$ de la
région bornée par la courbe $y=\cos(x)$ pour $0\leq x\leq \pi/2$, et les
droites $x=0$ et $y=0$.

Le dessin suivant représente la région $R$ bornée par la courbe
$y=\cos(x)$ pour $0\leq x \leq \pi/2$, l'axe des $x$ et l'axe des $y$.
\PDFgraph{8_integrales_appl/rot13}
Ainsi, le volume $V$ du solide de révolution est
\begin{align*}
V &= \pi \int_0^{\pi/2} \left( (\cos(x) + 1)^2 - (0 +1)^2 \right) \dx{x}
= \pi \int_0^{\pi/2} \left( \cos^2(x) + 2\cos(x) \right) \dx{x} \\
& = \pi \int_0^{\pi/2} \left( \frac{1}{2} (1+\cos(2x)) + 2\cos(x)\right)
\dx{x} \\
&= \frac{\pi}{2} \left( x + \frac{1}{2}\ \sin(2x) + 4\sin(x)
\right)\bigg|_0^{\pi/2} 
= \frac{\pi}{2}\left(\frac{\pi}{2} + 4\right) \ .
\end{align*}
\label{eggSR_washer}
\end{egg}

Quelle sera le volume $V$ du solide $S$ représenté à la
figure~\ref{ROT4}, obtenu par la rotation de la région $R$ représentée à
la figure~\ref{ROT1} autour d'un axe $x=c$ qui ne coupe pas $R$?  Il
est généralement inconcevable d'utiliser les sections provenant de
l'intersection du solide $S$ avec les plans perpendiculaires à l'axe
des $y$ (i.e.\ les plans où $y$ est constant) car il est fort possible
que ces sections soient formées de plusieurs anneaux.

\PDFfig{8_integrales_appl/rot4}{La rotation d'une région autour d'un axe
$x=c$ donne un solide}{La rotation de la région $R$ autour de l'axe
$x=c$ (et $z=0$) donne un solide $S$.}{ROT4}

Il faut donc retourner à la définition de l'intégrale pour développer une
formule qui nous permettra de calculer le volume du solide $S$ dans le cas
présent.

Pour estimer le volume $V$ du solide $S$, nous partageons le solide $S$ en mince
tubes dont l'axe est l'axe $x=c$ autour de laquelle la région $R$ fait une
rotation.  Nous estimons le volume de chacun des tubes.  La somme de ces
approximations nous donne une approximation du volume de $S$.

Soit $k$, un entier positif.  Posons $\Delta x = (b-a)/k$ et
$x_i = a + i \Delta x$ pour $i=0$, $1$, $2$, \ldots, $k$.  Nous obtenons une
partition de l'intervalle $[a,b]$ en sous-intervalles de la forme
$[x_i,x_{i+1}]$ pour $i=0$, $1$, \ldots, $k-1$.

Soit $R_i$, la région donnée par l'intersection de la région $R$ avec
l'ensemble $\{(x,y) : x_i \leq x \leq x_{i+1} \}$.  Le $i^e$ tube qui
compose le solide $S$ (figure~\ref{ROT5}) est obtenue par la
rotation de la région $R_i$ autour de l'axe $x=c$.

\PDFfig{8_integrales_appl/rot5}{Une région dont la rotation autour d'un axe
produit une portion tubulaire d'une solide}{la région $R_i$ donnée par
l'intersection de la région $R$ avec l'ensemble
$\{(x,y) : x_i \leq x \leq x_{i+1} \}$.}{ROT5}

Pour estimer le volume du $i^e$ tube, nous aurons besoin de la formule pour
calculer l'aire d'un tube $T$ comme celui représenté à la
figure~\ref{ROT7}.  Le volume $V_T$ de ce tube est obtenu en
soustrayant le volume du cylindre de rayon $r_1$ du volume du cylindre
de rayon $r_2$.  Ainsi,
\begin{equation}\label{vol_tube}
V_T = \pi r_2^2 h - \pi r_1^2 h = 2\pi \left(\frac{r_1+r_2}{2} \right)
(r_2-r_1) \ .
\end{equation}

\PDFfig{8_integrales_appl/rot7}{Calcul de l'aire d'un tube}{Le volume de ce
tube est obtenu en soustrayant le volume du cylindre de rayon $r_1$ du volume
du cylindre de rayon $r_2$.}{ROT7}

Pour estimer le volume $V_i$ du $i^e$ tube, nous appliquons la formule
(\ref{vol_tube}) au tube $S_i$ qui est représenté à la figure~\ref{ROT6}.  La
valeur du rayon $r_1$ et celle du rayon $r_2$ dans la formule
(\ref{vol_tube}) sont $r_1=c-x_{i+1}$ et $r_2=c-x_i$.  Nous avons alors que
\begin{align*}
r_2-r_1 &= x_{i+1} - x_i = \Delta x
\intertext{et}
\frac{r_1+r_2}{2} &= \frac{(c-x_{i-1})+(c-x_i)}{2}
= c - \frac{x_{i-1}+x_i}{2}
\end{align*}
où $(x_{i-1}+x_i)/2$ est le point milieu de l'intervalle $[x_i,x_{i+1}]$.  Si
nous posons
\[
x_i^\ast = \frac{x_{i-1}+x_i}{2}
\qquad \text{et} \quad h = |f(x_i^\ast)-g(x_i^\ast)| \; ,
\]
alors le volume de $S_i$ est
\[
2\pi \,\frac{r_1+r_2}{2} \, (r_2-r_1) h
= 2 \pi \, (c-x_i^\ast) |f(x_i^\ast)-g(x_i^\ast)| \Delta x
\]

\PDFfig{8_integrales_appl/rot6}{Approximation d'une portion tubulaire d'une
solide}{Un tube $S_i$ d'épaisseur $\Delta x$ et de hauteur
$|f(x_i^\ast)-g(x_i^\ast)|$.}{ROT6}

Pour $\Delta x$ petit (i.e. pour des grandes valeurs de $k$), le volume $V_i$
du $i^e$ tube produit par la rotation de la région $R_i$ autour de l'axe
$x=c$ est approximativement le volume de $S_i$. 
Ainsi,
\[
V_i \approx
2 \pi \, (c-x_i^\ast) |f(x_i^\ast)-g(x_i^\ast)| \Delta x
\]
pour $i=0$, $1$, $2$, \ldots, $k-1$ et $\Delta x$ petit.

La somme du volume de chaque tube $S_i$ donne un estimé du volume $V$ de
$S$.
\[
V = \sum_{i=0}^{k-1} V_i \approx \sum_{i=0}^{k-1}
2 \pi \, (c-x_i^\ast) |f(x_i^\ast)-g(x_i^\ast)| \Delta x \; .
\]
Plus $k$ sera grand (et donc $\Delta x$ petit), plus la somme ci-dessus sera
près du volume $V$ du solide $S$.  À la limite, nous obtenons le volume $V$ de
$S$.

Mais, ces sommes représentent aussi des sommes de Riemann pour l'intégrale
\begin{equation}\label{vol_tube1}
2 \pi \int_a^b (c-x) |f(x)-g(x)| \dx{x} \; .
\end{equation}
Un raisonnement identique à celui qui a été utilisé pour développer
(\ref{vol_tube1}) donne
\[
2 \pi \int_a^b (x-c) |f(x)-g(x)| \dx{x}
\]
lorsque l'axe de rotation $x=c$ est à la gauche de la région $R$.

Ainsi, si $k$ tend vers plus l'infini, nous obtenons le résultat suivant.

\begin{focus}[][des cylindres]{\mth}
Soit $R$ une région bornée par les courbes $y=f(x)$ et $y=g(x)$, et les
droites $x=a$ et $x=b$.  Soit $S$ le solide généré par la rotation de la
région $R$ autour de l'axe $x=c$ qui ne coupe pas la région $R$. Le volume
$V$ du solide $S$ est
\begin{equation}\label{comp_vol_tube}
V = 2 \pi \int_a^b |c-x| |f(x)-g(x)| \dx{x} \; .
\end{equation}
\end{focus}

\begin{egg}
Quelle est le volume $V$ du solide $S$ obtenu de la rotation autour de l'axe
$x=2$ de la région $R$ donnée à l'exemple~\ref{ROT_ex1} et illustrée à la
figure~\ref{ROT1_ex}.

Utilisons la formule (\ref{comp_vol_tube}) pour calculer
le volume $V$ de $S$.  Nous avons
\begin{align*}
V &= 2 \pi \int_{-1}^1 (2-x) \left( (3-x^2) - (x^2+1) \right)\dx{x} \\
&= 2 \pi \int_{-1}^1 (2-x) (2 - 2x^2) \dx{x}
= 2 \pi \int_{-1}^1 (2x^3-4x^2-2x+4) \dx{x} \\
&= 2\pi \left( \frac{x^4}{2} - \frac{4x^3}{3} - x^2 +
4x \right)\bigg|_{x=-1}^1 = \frac{32\pi}{3} \; .
\end{align*}
\end{egg}

\begin{egg}
Trouvons le volume $V$ de chacun des solides $S$ suivant.

Au risque de ce répéter, il est préférable de dessiner la région qui fera
une révolution autour de l'axe donné.

\subQ{a} $S$ est le solide obtenu de la rotation autour de l'axe des $y$ de
la région bornée par la courbe $y=x^2-3x+2$ et l'axe des $x$.

Nous avons vu à l'exemple~\ref{eggSR_washer} que la région
$R$ bornée par la parabole $y=x^2-3x+2$ et l'axe des $x$ est la région
représentée dans la dessin ci-dessous.
\PDFgraph{8_integrales_appl/rot8}

Nous pouvons calculer le volume $V$ du solide $S$ à l'aide de la formule
(\ref{comp_vol_tube}).  Nous avons
\begin{align*}
V &= 2 \pi \int_1^2 x \, \left( 0 - \left(x^2-3x+2\right)\right) \dx{x} =
2\pi \int_1^2 \left( -x^3 + 3x^2 - 2x \right) \dx{x} \\
&= 2\pi \left( -\frac{x^4}{4} + x^3 - x^2 \right)\bigg|_{x=1}^2 =
\frac{\pi}{2} \; .
\end{align*}

\subQ{b} $S$ est le solide obtenu de la rotation autour de l'axe $x=1$ de la
région bornée par la courbe $y=x^2$ et les droites $y=0$, $x=1$ et $x=2$.

Le dessin suivant représente la région $R$ qui est bornée par la
parabole $y=x^2$, les droites $x=1$ et $x=2$, et l'axe des $x$.
\PDFgraph{8_integrales_appl/rot14}
Ainsi,
\[
V = 2\pi \int_1^2 (x-1) (x^2-0) \dx{x} =
= 2 \pi \int_1^2 \left( x^3 - x^2 \right)\dx{x}
= 2\pi \left(\frac{x^4}{4} - \frac{x^3}{3} \right)\bigg|_1^2
= \frac{17 \pi}{6} \ . 
\]

\subQ{c} $S$ est le solide obtenu de la rotation autour de l'axe $x=1$ de la
région bornée par la courbe $y=e^x$ et les droites $y=e$ et $x=0$.

Le dessin suivant représente la région $R$ qui est bornée
par la courbe $y=e^x$, la droite $y=e$ et l'axe des $y$.
\PDFgraph{8_integrales_appl/rot15}
La formule (\ref{comp_vol_tube}) pour le volume d'un solide de révolution
donne
\begin{align*}
V &= 2\pi \int_0^1 (1-x) (e-e^x) \dx{x}
= 2\pi \int_0^1 (e -e x - e^x + xe^x )\dx{x} \\
&= 2\pi\left(ex- \frac{e}{2}x^2 - e^x + xe^x - e^x\right)\bigg|_0^1
= 2\pi\left(2-\frac{e}{2}\right) \ .
\end{align*}

\subQ{d} $S$ est le solide obtenu de la rotation autour de l'axe $x=-2$ de la
région bornée par les paraboles $y=4x-x^2$ et $y=8x-2x^2$.

Le dessin suivant contient les graphes des paraboles $y=4x-x^2$
et $y=8x-2x^2$, et illustre la région $R$ qui est bornée par ces
deux paraboles.
\PDFgraph{8_integrales_appl/rot9}
Pour déterminer les valeurs de $x$ aux points d'intersections de ces
deux paraboles, il faut résoudre l'équation $4x-x^2 = 8x-2x^2$.
Ces valeurs de $x$ sont les solutions de l'équation quadratique
$x^2 - 4x = x(x-4)=0$; c'est-à-dire, $x=0$ ou $x=4$.  Nous avons le point
d'intersection $(0,0)$ pour $x=0$ et le point d'intersection
$(4,0)$ pour $x=4$.  Les deux graphes se coupent sur l'axe des $x$.

Nous pouvons calculer le volume $V$ du solide $S$ à l'aide de la formule
(\ref{comp_vol_tube}).  Nous avons
\begin{align*}
V &= 2 \pi \int_0^4 (x+2) \, \left( \left(8x-2x^2\right) -
\left(4x-x^2\right)\right) \dx{x} =
2\pi \int_0^4 (x+2) \left( 4x-x^2 \right) \dx{x} \\
&= 2\pi \int_0^4 \left( -x^3 +2x^2 +8x \right) \dx{x}
= 2\pi \left(-\frac{x^4}{4} + \frac{2}{3} x^3 + 4x^2 \right)\bigg|_{x=0}^4 =
\frac{4^4\pi}{3} \; .
\end{align*}
\end{egg}

\section{Masse d'un objet \eng}

Comme à la section précédente pour le calcul du volume d'un solide,
nous pouvons utiliser l'intégration pour calculer le masse $M$ d'un
solide $S$ comme celui représenté à la figure~\ref{VOL2}.

Nous supposons que la densité du solide varie de façon continue en fonction de
$x$ seulement.  C'est très proche de la réalité lorsque le solide $S$ est un
tube de petit diamètre.  Soit $\rho(s)$ la densité du solide pour la section
où $x=s$.

Pour estimer la masse $M$ du solide $S$, nous partageons le solide $S$
en fines tranches parallèles au plan $y$,$z$ et nous estimons la masse
de chacune de ces tranches. La somme de ces approximations nous donne
une approximation de la masse de $S$.

Nous utilisons les tranches (figure~\ref{VOL2}), les cylindres
(figure~\ref{VOL1}) et les sections (figure~\ref{VOL3}) qui
ont été définies à la section précédente. Nous pouvons assumer que la
densité de la $i^e$ tranche est approximativement constant à
$\rho(x_i^\ast)$; c'est-à-dire que $\rho(x) = \rho(x_i^\ast)$ pour
$x_i \leq x \leq x_{i+1}$.  Si $\Delta x$ est très petit, il est
raisonnable de supposer que la densité $\rho(x)$ ne changera presque
pas pour $x_i \leq x \leq x_{i+1}$.

Comme la masse d'un objet de densité constante est le produit de sa densité
par son volume, la masse $M_i$ de la $i^e$ tranche est approximativement la
masse du solide $S_i$ pour lequel nous assumons que la densité est
$\rho(x_i^\ast)$; c'est-à-dire,
\[
M_i \approx \rho(x_i^\ast) A(x_i^\ast) \Delta x
\]
où $A(s)$ est l'aire de la section engendrée par l'intersection du plan $x=s$
avec le solide $S$.  La somme de la masse de chaque solide $S_i$ donne une
estimation de la masse $M$ de $S$.
\[
M = \sum_{i=0}^{k-1} M_i \approx \sum_{i=0}^{k-1}
\rho(x_i^\ast) A(x_i^\ast) \Delta x \; .
\]
Plus $k$ sera grand (et donc $\Delta x$ petit), plus la somme ci-dessus sera
près de la masse $M$ du solide $S$.  À la limite, nous obtenons la masse $M$ de
$S$.

Mais, ces sommes représentent aussi des sommes de Riemann pour l'intégrale
\[
\int_a^b \rho(x) A(x)\dx{x} \; .
\]
Ainsi, si $k$ tend vers plus l'infini, nous obtenons le résultat suivant.

\begin{focus}{\mth} \index{Masse}
Soit $S$ est solide dont la densité varie seulement dans une
direction.  Nous pouvons supposer que cet direction est donnée par l'axe
des $x$ et que $\rho(x)$ est la \lgm densité\rgm\ de la section $x=s$
engendrée par l'intersection du plan $x=s$ avec le solide $S$.  Si
$A(s)$ est l'aire de cette section, alors le masse du solide $S$ est
\begin{equation}\label{comp_mass}
M = \int_a^b \rho(x)A(x)\dx{x} \; .
\end{equation}
\end{focus}

\section{Travail \eng}

La force exercée sur un objet qui se déplace en ligne droite est donnée par
le produit de sa masse par son accélération.  Ainsi, si $m$ est la masse en
kilogrammes de l'objet et $x(t)$ est sa position en mètres au temps $t$ en
secondes, l'accélération de l'objet en m/s$^2$ est
\[
a(t) = \dydxn{x}{t}{2}(t)
\]
et la force en Newtons exercée sur l'objet au temps $t$ est
\[
F(t) = m\ a(t) = m \ \dydxn{x}{t}{2}(t) \; .
\]
L'abréviation pour un Newton est la lettre N.  Ainsi, N = kg\ m/s$^2$.

Le travail en joules fait pour déplacer un objet en ligne droite sur une
distance $d$, lorsque la force $F$ exercée est constante durant tout le
déplacement, est donnée par
\begin{equation} \label{force}
W = F\ d \; .
\end{equation}
L'abréviation pour un joule est la lettre J.  Ainsi, J = N\ m.

Avec ces deux résultats de physique, nous sommes en mesure de répondre aux
questions suivantes.

\subsection{Travail pour déplacer un objet}

Un objet se déplace en ligne droite sous l'effet d'une force qui varie
avec la distance parcourue.  Si cette force est de $F(x)$ N lorsque
l'objet est à $x$ m de son point de départ, qu'elle est le travail
nécessaire pour déplacer l'objet sur une distance de $L$ m à partir de
l'origine, son point de départ? 

Si la force était constante, la formule (\ref{force}) résoudrait le
problème.  Malheureusement, ce n'est pas le cas présentement.

Divisons l'intervalle $[0,L]$ en $k$ sous-intervalles de longueur
$\Delta x = L/k$.  Chaque intervalle est donc de la forme $[x_i,x_{i+1}]$ où
$x_i= i\,\Delta x$ pour $i=0$, $1$, $2$, \ldots, $k$.

Si le sous-intervalle $[x_i,x_{i+1}]$ est très court, nous pouvons supposer
que la force $F$ est presque constante sur cet intervalle et est égale
à $F(x_i^\ast)$ où $x_i^\ast \in [x_i,x_{i+1}]$.  Le travail pour
déplacer l'objet de $x_i$ à $x_{i+1}$ est donc $F(x_i^\ast)\,\Delta x$
grâce à la formule (\ref{force}).

Ainsi, le travail $W$ pour déplacer l'objet de $L$ m à partir de
l'origine est donc approximativement
\[
W \approx \sum_{i=0}^{k-1} F(x_i^\ast) \, \Delta x  \; ;
\]
la somme du travail fait pour déplacer l'objet sur chacun des
sous-intervalles $[x_i,x_{i+1}]$.  Cette somme est une somme de Riemann pour
l'intégrale
\[
\int_0^L F(x) \dx{x} \; .
\]
Donc, si $k$ tend vers l'infini, nous trouvons que le travail pour déplacer
l'objet de $L$ m à partir de l'origine est donnée par
\[
W = \int_0^L F(x) \dx{x} \; .
\]

\begin{focus}{\mth} \index{Travail}
Le travail nécessaire pour déplacer un objet d'un point $x=a$ à un
point $x=b$ (en ligne droite) est
\[
W = \int_a^b F(x) \dx{x} \; ,
\]
où $F(x)$ est la force exercée sur l'objet lorsqu'il occupe la
position $x$.
\end{focus}

\begin{egg}
Un objet se déplace en ligne droite sous l'effet d'une force.  Si cette force
est de $F(x) = 5x^2 +3x+1$ N lorsque l'objet est à $x$ mètres de son point de
départ, qu'elle est le travail nécessaire pour déplacer l'objet sur une
distance de $6$ mètres à partir de l'origine?

Le travail est donnée par
\[
W = \int_0^6 \left(5x^2 + 3x + 1\right) \dx{x}
= \left(\frac{5}{3}\,x^3
+ \frac{3}{2}\, x^2 + x\right)\bigg|_{x=0}^6 = 420\ \text{J} \; .
\]
\end{egg}

\PDFfig{8_integrales_appl/icecube}{Travail pour lever un bloc de
glace de $50$ mètres}{Travail pour lever le bloc de glace jusqu`à une
hauteur de $50$ mètres.  Le bloc de glace est levé à la vitesse de
$1.5$ m/min et il font à une vitesse de $4$ kg/min.  Il faut aussi
tenir compte dans ce problème de la masse de la corde qui est de $0.5$
kg/m.}{ICE}

\begin{egg}
Nous voulons lever un bloc de glace de $100$ kg à partir du sol jusqu'à une
hauteur de $50$ mètres.  Le bloc de glace font à une vitesse de $4$ kg/min et
la vitesse à laquelle nous levons le bloc est de $1.5$ m/min.  Si la
masse de la corde est de $0.5$ kg/m, quel est le travail nécessaire
pour lever le bloc à la hauteur désirée?  Le montage est illustré à la
figure~\ref{ICE}.

Commençons par éliminer le temps dans l'énoncé du problème.  Le bloc de glace
font à une vitesse de $4$ kg/min et en une minute le bloc monte de $1.5$
mètres.  La masse du bloc diminue donc de $4$ kg par $1.5$ mètres;
c'est-à-dire de $8/3$ kg/m. 

Divisons l'intervalle $[0,50]$ en $k$ sous-intervalles de longueur
$\Delta y = 50/k$.  Chaque intervalle est donc de la forme $[y_i,y_{i+1}]$ où
$y_i= i\,\Delta y$ pour $i=0$, $1$, $2$, \ldots, $k$.  Nous choisissons
$y_i^\ast \in [y_i,y_{i+1}]$ pour tout $i$.

Si le sous-intervalle $[y_i,y_{i+1}]$ est très court, nous pouvons
assumer que le bloc de glace n'a pas le temps de fondre entre $y_i$ et
$y_{i+1}$ mètres et donc que sa masse est constante et égale à
$\left(100- (8/3) y_i^\ast\right)$ kg car le bloc font à $8/3$
kg/m.  Le travail pour lever le bloc de $y_i$ à
$y_{i+1}$ mètres, une distance de $\Delta y$, est approximativement 
\[
9.8 \left(100- \frac{8}{3} y_i^\ast\right) \Delta y \; .
\]

De même, si le sous-intervalle $[y_i,y_{i+1}]$ est très court, nous pouvons
assumer que la corde que nous devons lever entre $y_i$ et $y_{i+1}$ est de
longueur constante égale à $(50-y_i^\ast)$ mètres.  La masse de cette
portion de la corde est de $0.5\,(50-y_i^\ast)$ kg.  Le travail pour
lever cette portion de la corde de $y_i$ à $y_{i+1}$ mètres, une
distance de $\Delta y$, est approximativement
\[
9.8  \big( 0.5\,(50-y_i^\ast)\big) \Delta y \; .
\]
Le travail pour lever l'ensemble formé du bloc de glace et de la corde
de $y_i$ à $y_{i+1}$ mètres est donc approximativement
\[
9.8 \left(100- \frac{8}{3} y_i^\ast\right) \Delta y +
9.8 \big( 0.5\,(50-y_i^\ast) \big)\Delta y
= 9.8 \left(125-\frac{19}{6} y_i^\ast \right) \Delta y \; .
\]

Ainsi, le travail pour lever l'ensemble formé du bloc de glace et de la corde
du sol jusqu'à une hauteur de $50$ mètres est approximativement
\[
W \approx \sum_{i=0}^{k-1} 9.8 \left(125-\frac{19}{6} y_i^\ast\right)
\Delta y \; .
\]
Cette somme est une somme de Riemann pour l'intégrale
\[
\int_0^{50} 9.8 \left(125-\frac{19}{6} y\right) \dx{y} \; .
\]
Donc, si $k$ tend vers l'infini, nous trouvons que le travail pour lever
l'ensemble formé du bloc de glace et de la corde du sol jusqu'à une hauteur
de $50$ mètres est donnée par
\[
W = \int_0^{50} 9.8 \left(125-\frac{19}{6} y\right) \dx{y} \; .
\]
Ce qui donne
\[
W = 9.8 \, \left(125y - \frac{19}{12}\,y^2\right)\bigg|_{y=0}^{50} =
22,458.\overline{3} \ \text{J} \; .
\]
\end{egg}

Une autre situation où nous pouvons avoir à calculer le travail pour déplacer
une objet est lorsque cet objet est attaché à un ressort.

\PDFfig{8_integrales_appl/hooke}{Lois de Hooke pour les ressorts}
{Lois de Hooke pour les ressorts}{ressorT}

Nous considérons un mécanisme comme celui représenté à la
figure~\ref{ressorT} où nous ignorons toute friction et où la gravité
peut être négligée.  La {\bfseries lois de Hooke}\index{Lois de Hooke}
pour les ressorts dit que la force $F(x)$ nécessaire pour maintenir un
ressort à une longueur de $x$ m est $F(x)= k (x-x_0)$, où $x_0$ est la
longueur du ressort au repos et $k$ est une constante positive.  Si
$x>x_0$, la force est dans la direction positive alors que, si
$x<x_0$, la force est dans la direction négative.

\begin{egg}
Une force de $20$ N est nécessaire pour maintenir un ressort à une longueur
de $10$ m.  Si la longueur au repos du ressort est $5$ m, calculons le
travail nécessaire pour étirer le ressort de $15$ à $20$ m.

Si $F(x) = k(x-5)$ est la force nécessaire pour maintenir un ressort à une
longueur de $x$ m, nous obtenons de $F(10) = 20 = 5k$.  Donc $k=4$.
Le travail nécessaire pour étirer le ressort de $15$ à $20$ m est
\[
W = \int_{15}^{20} 4(x-5) \dx{x} = 2(x-5)^2\bigg|_{15}^{20}
= 2(15^2 - 10^2) = 250 \ \text{J} \ .
\]
\end{egg}

\subsection{Travail pour vider un réservoir}

Un réservoir rempli d'un liquide non visqueux est enfoui $h_1$ m sous le sol.
Ce réservoir est illustré à la figure~\ref{RESERVOIR}. Quelle est le
travail nécessaire pour vider ce réservoir de son contenu si le niveau
initial du liquide est à $h_2$ m sous le sol?  Pour les besoins de la
discussion, la distance positive est vers le haut et $0$ est le niveau
du sol.  Tout autre choix serait acceptable et conduirait à des
formules équivalents.

\PDFfig{8_integrales_appl/reservoir}{Un réservoir de forme quelconque enfoui
sous le sol}{Un réservoir de forme quelconque est enfoui à $h_1$ mètres sous
le sol.}{RESERVOIR}

Pour obtenir une formule qui nous permettra de trouver le travail fait pour
vider le réservoir, nous divisons l'intervalle $h_3 \leq x \leq h_2$ en $k$
sous-intervalles de longueur $\Delta x = (h_2-h_3)/k > 0$.  Chaque
sous-intervalle est de la forme $[x_i,x_{i+1}]$ où $x_i = h_3 + i \Delta x$
pour $i=0$, $1$, $2$, \ldots, $k-1$.

Utilisons (\ref{force}) pour estimer le travail fait pour retirer du
réservoir la mince couche de liquide entre $x_i$ et $x_{i+1}$
(figure~\ref{RESERVOIR2}).

\PDFfig{8_integrales_appl/reservoir2}{Une mince couche de liquide dans un
réservoir}{Une mince couche de liquide pour $x$ entre $x_i$ et
$x_{i+1}$}{RESERVOIR2}

Soit $x_i^\ast$ un point de l'intervalle $[x_i,x_{i+1}]$.  Si
$\Delta x$ est très petit, nous pouvons estimer le volume de la mince
couche de liquide entre $x_i$ et $x_{i+1}$ (figure~\ref{RESERVOIR2})
par le volume du solide $V_i$ qui est représenté à la
figure~\ref{RESERVOIR1}.  $V_i$ est un solide borné par
les plans $x=x_i$ et $x=x_{i+1}$, et dont les sections horizontales sont
tous identiques à la section produite par l'intersection du plan
$x=x_i^\ast$ avec le réservoir.  Nous pouvons assumer que la paroi de
$V_i$ pour $x_{i+1} \leq x \leq x_i$ est verticale.

Le volume de la mince couche de liquide entre $x_i$ et $x_{i+1}$ est
donc approximativement le volume du solide $V_i$; c'est-à-dire,
\begin{equation}\label{vol_couche}
A(x_i^\ast)\Delta x
\end{equation}
où $A(x_i^\ast)$ est l'air de la section produite par l'intersection
du plan $x=x_i^\ast$ avec le réservoir.

\PDFfig{8_integrales_appl/reservoir1}{Le volume d'une mince couche de liquide
dans un réservoir}{Le volume de la mince couche de liquide représenté à la
figure~\ref{RESERVOIR2} peut être estimé par le volume du solide $V_i$ bornée
par les plans $x=x_i$ et $x=x_{i+1}$, et ayant une paroi
verticale.}{RESERVOIR1}

En supposant que la densité $\rho(x)$ du liquide à $x$ m sous le sol ne varie
presque pas pour $x$ entre $x_i$ et $x_{i+1}$, nous pouvons assumer
que la densité du liquide dans la mince couche est constante et égale
à $\rho(x_i^\ast)$.  Ainsi, la masse de la mince couche entre $x_i$ et
$x_{i+1}$ est approximativement $\rho(x_i^\ast) A(x_i^\ast) \Delta x$.

La force exercée sur la mince couche de liquide est approximativement
\[
9.8 \rho(x_i^\ast) A(x_i^\ast) \Delta x
\]
où $9.8$ m/s$^2$ est l'accélération terrestre.

Grâce au résultat (\ref{force}), le travail fait pour déplacer la mince
couche de liquide jusqu'au niveau du sol (donc pour retirer cette couche de
liquide du réservoir) est approximativement
\[
9.8 \underbrace{(0-x_i^\ast)}_{\text{profondeur}}
\rho(x_i^\ast) A(x_i^\ast) \Delta x
= - 9.8 x_i^\ast) \rho(x_i^\ast) A(x_i^\ast) \Delta x \ .
\]
Il ne faut pas oublier que $x_i^\ast <0$ pour tout $i$ par construction.

Si nous faisons cela pour chaque intervalle $[x_i,x_{i+1}]$, nous
trouvons que le travail $W$ fait pour vider le réservoir est approximativement
\[
W \approx -9.8 \sum_{i=0}^{k-1} x_i^\ast \rho(x_i^\ast) A(x_i^\ast)
\Delta x \ .
\]
C'est une somme de Riemann pour l'intégrale
\[
-9.8 \int_{h_3}^{h_2} x \rho(x) A(x) \dx{x} \ .
\]
Donc, si $k$ tend vers plus l'infini, nous trouvons que le travail fait pour
vider le réservoir est
\begin{equation}\label{travail}
W = -9.8 \int_{h_3}^{h_2} x \rho(x) A(x) \dx{x} \ .
\end{equation}

\begin{rmk}
Lors de la discussion précédente, nous avons assumé que la densité du liquide
variait seulement en fonction de la profondeur.  Cette hypothèse est
réaliste.  Le lecteur a certainement déjà remarqué la séparation qui se
produit dans les vinaigrettes à salade où l'eau, le vinaigre et l'huile se
partage en couches biens distinctes.

De plus, nous supposons qu'il y a une prisse d'aire, qui n'est pas incluse
dans le dessin du réservoir à la figure~\ref{RESERVOIR}, pour permettre à
l'aire d'occuper l'espace libérée par le liquide qui s'échappe du
réservoir par le tuyau.  Autrement, le réservoir se déformerait suite
au retrait du liquide.

Finalement, le liquide est aspiré près du font du réservoir.  Nous
pourrions croire que le calcul du travail fait pour retirer la mince
couche de liquide
à la profondeur $x_i^\ast$ devrait être fait sur une distance plus grande que
$x_i^\ast$.  Ce n'est pas le cas.  La pression dû à l'attraction terrestre
maintient au même niveau le liquide dans le réservoir et dans le tuyau.
Le niveau du liquide dans le tuyau est la profondeur de la mince
couche de liquide qui est retirée du réservoir à ce moment.  Aucun
travail n'est fait par la pompe pour amener le liquide dans le tuyau
au niveau du liquide dans le réservoir.
\end{rmk}

\PDFfig{8_integrales_appl/trav_sphere}{Un réservoir sphérique}{Un réservoir
sphérique de $2$ mètres de rayon enfoui à $3$ mètres sous le
sol.}{TRAV_SPHERE}

\begin{egg}
Un réservoir d'eau de forme sphérique est enfoui à $3$ m sous le sol.
Le rayon du réservoir est de $2$ m (figure~\ref{TRAV_SPHERE}).  Quelle est le 
travail nécessaire pour vider ce réservoir de son contenu si le niveau
initial de l'eau est au trois quarts du réservoir?  La densité de
l'eau est constante et égale à $1000$ kg/m$^3$.

Dans la formule (\ref{travail}), nous avons que $h_3=-7$, $h_2=-4$ et
$\rho(x) \equiv 1000$.  Il ne manque que $A(x)$ qui, dans le cas présent,
représente l'aire d'une section où $x$ est constant.  Comme nous
pouvons le déduire à partir du dessin à droite dans la
figure~\ref{TRAV_SPHERE}, cette section est un cercle de rayon
$r(x) = \sqrt{2^2 - (-5-x)^2}$.  Ainsi,
\[
A(x) = \pi r^2(x) = \pi (2^2 - (-5-x)^2) = \pi(-21-10x-x^2) \; .
\]
Le travail fait pour vider le réservoir est
\begin{align*}
W &= -9,800\pi \int_{-7}^{-4} x(-21-10x-x^2) \dx{x}
= -9,800\pi \left( -\frac{21}{2}x^2 -\frac{10}{3}x^3
  -\frac{1}{4}x^4\right)\bigg|_{x=-7}^{-4}\\
&= \pi( 4.6305 \times 10^5) \ \text{J} \; .
\end{align*}
\end{egg}

\begin{egg}
Un réservoir en forme de cône inversé est enfoui à $2$ m sous le
sol. La base du cône à un rayon de $3$ m et le cône à une hauteur de
$5$ m (figure~\ref{CONE}).  Si le cône est rempli d'eau dont la densité est
de $1000$ kg/m$^3$, quel est le travail nécessaire pour vider ce réservoir?

Dans la formule (\ref{travail}), nous avons que
$h_2=-2$, $h_3=-7$ et $\rho(x) = 1000$ pour tout $x$.  L'attraction
terrestre est toujours $9.8$ m/s$^2$.  Les sections horizontales sont
des cercles de rayon $\displaystyle r= \frac{3}{5} (7+x)$.  Donc
$\displaystyle A(x) = \frac{9\pi}{25} (7+x)^2$.

Le travail nécessaire pour vider le réservoir est
\begin{align*}
W & = - \int_{-7}^{-2} 9,800 \left(\frac{9\pi}{25}\right)x(7+x)^2 \dx{x}
= - 9,800\left(\frac{9\pi}{25}\right) \int_{-7}^{-2}
\left( x^3 + 14x^2 + 49x \right) \dx{x} \\
&= - 9,800\left(\frac{9\pi}{25}\right) \left(
\frac{x^4}{4} + \frac{14 x^3}{3} + \frac{49 x^2}{2} \right)\bigg|_{-7}^{-2}
= \pi (4.7775 \times 10^5)\ \text{J} \; .
\end{align*}
\end{egg}

\PDFfig{8_integrales_appl/cone}{Un réservoir conique}{Un réservoir conique dont
la base à $3$ mètres de rayon et la hauteur est de $5$ mètres est enfoui
à $2$ mètres sous le sol.}{CONE}

\begin{egg}
Un réservoir dont les sections verticales dans une direction sont des
triangles renversés est enfoui à $2$ m sous le
sol (figure~\ref{TRIANGLE}).  Si le niveau d'eau initial dans le
réservoir est au $2/3$ d'eau, quelle est le travail nécessaire pour
vider ce réservoir?  La densité de l'eau est de $1000$ kg/m$^3$.

Dans la formule (\ref{travail}), nous avons $h_2=-3$ car le réservoir
est rempli au $2/3$, $h_3=-5$ et $\rho(x) = 1000$ pour tout $x$.  L'attraction
terrestre est toujours $9.8$ m/s$^2$.  Les sections horizontales
sont des rectangles dont l'un des côtés est de $8$ m de long et le deuxième
est de $2r = 5+x$ m de long.  Donc $\displaystyle A(x) = 8(5+x)$.

Le travail nécessaire pour vider le réservoir est
\begin{align*}
W & = - \int_{-5}^{-3} 9,800\big( 8\, x(5+x) \big)\dx{x}
= - 9,800 \times 8 \, \int_{-5}^{-3} \left( x^2 + 5x \right) \dx{x} \\
&= - 9,800 \times 8 \, \left( \frac{x^3}{3} + \frac{5x^2}{2}
 \right)\bigg|_{-5}^{-3} 
= 5.74933\overline{3}\times 10^5\ \text{J} \; .
\end{align*}
\end{egg}

\PDFfig{8_integrales_appl/triangle}{Un réservoir dont les sections verticales
dans une direction sont des triangles renversés}{Un réservoir dont les
sections verticales dans une direction sont des triangles est enfoui à $2$ m
sous le sol.  Les triangles ont une hauteur de $3$ m et une basse de $3$ m.}
{TRIANGLE}

\begin{egg}
Un réservoir cylindrique est enfoui sur le côté à $5$ mètres de
profondeur (figure~\ref{CYLINDRE}).  Le cylindre à un rayon de $2$
mètres et une longueur de $9$ mètres.  Si nous supposons que le cylindre
est rempli d'eau, quel est le travail nécessaire pour vider le
réservoir?  La densité de l'eau est de $1000$ kg/m$^3$.

Dans la formule (\ref{travail}), nous avons que
$h_2=-5$, $h_3=-9$ et $\rho(x) = 1000$ pour tout $x$.  L'attraction
terrestre est toujours $9.8$ m/s$^2$.  Les sections horizontales sont
des rectangles dont l'un des côtés
est de $9$ m de long et le deuxième est de
$\displaystyle 2r = 2\sqrt{2^2-(x+7)^2}$ m de long.  Donc
$\displaystyle A(x) = 18\sqrt{2^2-(x+7)^2}$.

Le travail nécessaire pour vider le réservoir est
\begin{align*}
W & = - \int_{-9}^{-5} 9,800\big( 18 x \sqrt{2^2-(x+7)^2} \big) \dx{x}
= - 9,800\times 18 \, \int_{-2}^{2} (u-7) \sqrt{2^2-u^2} \dx{u}
\end{align*}
où nous avons utilisé la substitution $u=x+7$ qui donne $u=-2$ lorsque $x=-9$,
$u=2$ lorsque $x=-7$ et $\dx{u}=\dx{x}$.  Or,
\[
\int_{-2}^{2} u \sqrt{2^2-u^2} \dx{u} = 0
\]
car $u \sqrt{2^2-u^2}$ est une fonction impaire que nous intégrons sur un
intervalle symétrique à l'origine.  Pour évaluer
$\displaystyle -7 \int_{-2}^{2} \sqrt{2^2-u^2} \dx{u}$, nous posons
$u = 2\sin(t)$ pour $-\pi/2 \leq t \leq \pi/2$.  Donc $u = -2$ pour
$t = -\pi/2$, $u=2$ pour $t = \pi/2$ et $\dx{u} = 2\cos(t) \dx{t}$.
Ainsi,
\begin{align*}
-7 \int_{-2}^{2} \sqrt{2^2-u^2} \dx{u} &=
-7 \int_{-\pi/2}^{\pi/2} \sqrt{4-4\sin^2(t)} \; 2\cos(t)\dx{t}
= -7 \int_{-\pi/2}^{\pi/2} 4\cos^2(t)\dx{t} \\
&= -7 \int_{-\pi/2}^{\pi/2} 2\left( 1+\cos(2t)\right) \dx{t}
= -14 \left( t+ \frac{1}{2}\sin(2t)\right)\bigg|_{-\pi/2}^{\pi/2} 
= -14 \pi \; .
\end{align*}

Donc
\[
W  = - 9,800\times 18 \times (-14 \pi) = 2469600 \pi \ \text{j} \; .
\]
\end{egg}

\PDFfig{8_integrales_appl/cylindre}{Un réservoir cylindrique}{Un réservoir
cylindrique est enfoui sur le côté à $5$ mètres de profondeur.  Le cylindre à
un rayon de $2$ mètres et une longueur de $9$ mètres.}{CYLINDRE}

\section{Force et pression hydrostatique \eng}

Pour un liquide donné (e.g. l'eau dans l'océan), plus nous sommes
profond plus la pression exercée par le liquide augmente.  Si
$\rho(x)$ est la densité d'un
liquide en kg/m$^3$ à une profondeur de $x$ mètres et $9.8$ m/s$^2$
est l'attraction terrestre en m/s$^2$, alors la pression $p(x)$ en
N/m$^2$ à une profondeur de $x$ mètres est donnée par la formule
\[
p(x) =  9.8 \; x \; \rho(x) \; .
\]

La force en Newtons exercée sur une surface de {\it très petite hauteur} à
la profondeur de $x$ mètres est approximativement
\[
F(x) = p(x) A(x)
\]
où $A(x)$ est l'aire en m$^2$ de la surface.  Nous avons supposé dans le
raisonnement précédent que la {\it hauteur} de la surface est très
petite.  Si la {\it hauteur} de la surface est grande comme celle
représenté à la figure~\ref{PRESS1},  il faut décomposer la surface en
minces tranches horizontales de {\it très petite hauteur} et calculer
la force sur chacune de ces tranches.

\PDFfig{8_integrales_appl/pression}{Vue transversale d'une surface verticale
soumisse à la force exercée par la pression}{Vue transversale d'une surface
verticale soumisse à la force exercée par la pression.}{PRESS1}

Supposons que l'origine soit à la surface du liquide et la direction
positive soit vers le haut comme nous l'avons fait pour le calcul du travail
à la section précédente.  Nous désirons calculer la force exercée sur une
surface entre $h_3$ et $h_2$ mètres de profondeur.

Partageons le segment $[h_3,h_2]$ en $k$ sous intervalles
$[x_i,x_{i+1}]$ de longueur $\Delta x = (h_2-h_3)/k > 0$ où
$x_i = h_3 + i \Delta x$ pour $i=0$, $1$, $2$, \ldots , $k$.

la pression à une profondeur de $x_i^\ast$ mètres, où
$x_i \leq x_i^\ast \leq x_{i+1}$, est
\[
p(x_i^\ast) =  9.8 \underbrace{(0-x_i^\ast)}_{\text{profondeur}}
\rho(x_i^\ast) = - 9.8 x_i^\ast \rho(x_i^\ast) \ .
\]
Il ne faut pas oublier que $x_i^\ast <0$ par construction. Nous obtenons
que la force exercée sur la surface entre $x_i$ et $x_{i+1}$ mètres
est approximativement
\[
F(x_i^\ast) = p(x_i^\ast) A(x_i^\ast) = - 9.8 \; x_i^\ast \;
\rho(x_i^\ast) A(x_i^\ast) \ .
\]
où $A(x_i^\ast)$ est approximativement l'aire de la surface entre $x_i$ et
$x_{i+1}$ mètres.  Puisque l'aire de la surface entre $x_i$ et $x_{i+1}$
mètres tend vers $0$ lorsque la longueur $\Delta x = x_{i+1}-x_i$
tend vers $0$, nous pouvons exprimer $A(x_i^\ast)$ comme le produit
$S(x_i^\ast) \Delta x$ où $S$ est une fonction qu'il faut
déterminer pour chaque type de surfaces.  Dans les cas simple,
$S(x_i^\ast)$ sera une longueur.  Cependant, ce ne sera pas toujours le cas
comme nous verrons dans les exemples qui suivent.

Si nous faisons la somme de la force exercée sur la surface entre
$x_i$ et $x_{i+1}$ mètres pour $i=0$, $1$, \ldots , $k-1$, nous trouvons
que la force $F$ exercée sur la surface entre $h_3$ et $h_2$ mètres de
profondeur est approximativement
\[
F \approx \sum_{j=0}^{k-1}
- 9.8 \; x_i^\ast  \; \rho(x_i^\ast) \; S(x_i^\ast) \Delta x \; .
\]
La somme dans l'expression précédente est une somme de Riemann pour
l'intégrale
\[
- \int_{h_3}^{h_2} 9.8\, x\, \rho(x)\, S(x) \dx{x} \; .
\]
Ainsi, si nous considérons la limite lorsque $k$ tend vers plus
l'infini, nous trouvons que la force $F$ exercée sur la surface
verticale entre $h_3$ et $h_2$ mètres de profondeur est
\begin{equation}\label{pression}
F = - \int_{h_3}^{h_2} 9.8 \, x \, \rho(x)\, S(x)\dx{x} \; .
\end{equation}
 
\begin{egg}
Un réservoir de forme cylindrique repose sur le côté au font d'un lac de $30$
mètres de profondeur (figure~\ref{PRESS2}). Le réservoir a $2$ mètres
de diamètre et $5$ mètres de longueur.  Calculons la force sur les
deux extrémités du réservoir et sur le long du cylindre si le
réservoir est vide. L'attraction terrestre est de $9.8$ m/s$^2$ et la
densité de l'eau est de $1000$ kg/m$^3$.

\PDFfig{8_integrales_appl/pression2}{La force exercée sur une des
extrémités d'un réservoir circulaire qui repose au font d'un lac}{La
force exercée sur une des extrémités d'un réservoir circulaire qui
repose au font d'un lac de $30$ mètres de profondeur.}{PRESS2}

\subQ{i}
Si nous utilisons l'information fournie dans le dessin à la
figure~\ref{PRESS2}, nous avons que l'aire $A(x_i^\ast)$ de la surface
de l'extrémité du réservoir entre $x_i$ et $x_{i+1}$ mètres est
approximativement l'aire du rectangle $R_i$ de base
$2\sqrt{1-(x_I^\ast + 29)^2}$ et de hauteur $\Delta x$.  Donc
$A(x_i^\ast) \approx S(x_i^\ast) \Delta x$ où
\[
S(x) = 2\sqrt{1-(x+29)^2} \; .
\]

La formule (\ref{pression}) avec $h_3=-30$ et $h_2 =-28$ nous donne la
force $F$ exercée sur une des extrémités du réservoir.
\[
F = -\int_{h_3}^{h_2} 9.8 \, x\, \rho(x)\, S(x)\dx{x}
= - 2\times 9,800 \int_{-30}^{-28} x\sqrt{1-(x+29)^2} \dx{x} \; .
\]
Si $y = x+29$, alors $\dx{y} = \dx{x}$, $y=-1$
pour $x=-30$ et $y=1$ pour $x=-28$.  Donc
\begin{align*}
F &= -19,600 \int_{-1}^{1} (y-29)\sqrt{1-y^2} \dx{y} \\
&= -19,600 \int_{-1}^{1} y\sqrt{1-y^2} \dx{y}
+ 19,600\times 29 \int_{-1}^{1} \sqrt{1-y^2} \dx{y} \; .
\end{align*}

Or $y\sqrt{1-y^2}$ est une fonction impaire que nous intégrons sur un
intervalle symétrique à l'origine.  Donc
\[
\int_{-1}^{1} y\sqrt{1-y^2} \dx{y} = 0 \; .
\]

Pour calculer l'intégrale
\[
\int_{-1}^{1} \sqrt{1-y^2} \dx{y} \ ,
\]
il suffit de noter que c'est la formule pour calculer l'aire d'une
demi-disque de rayon $1$.  Sa valeur est donc $\pi/2$.
Nous pouvons aussi vérifier cela en utilisant la substitution
trigonométrique $y=\sin(\theta)$ pour $-\pi/2 \leq \theta \leq \pi/2$.
Ainsi, $\dx{y} = \cos(\theta) \dx{\theta}$,
$\sin(\pi/2) = -1$ et $\sin(\pi/2) = 1$.  Donc 
\begin{align*}
\int_{-1}^1 \sqrt{1-y^2} \dx{y}
&= \int_{-\pi/2}^{\pi/2} \sqrt{1-\sin^2(\theta)} \; \cos(\theta) \dx{\theta}
= \int_{-\pi/2}^{\pi/2} \cos^2(\theta) \dx{\theta} \\
&= \frac{1}{2} \int_{-\pi/2}^{\pi/2}
\left(1 + \cos(2\theta)\right) \dx{\theta}
= \frac{1}{2} \left(\theta + \frac{1}{2}\sin(2\theta)\right)
\bigg|_{\theta=-\pi/2}^{\pi/2} \\
&= \frac{1}{2} \left(\frac{\pi}{2} + \frac{1}{2}\sin(\pi)\right)
- \frac{1}{2} \left(-\frac{\pi}{2} + \frac{1}{2}\sin(\pi)\right)
= \frac{\pi}{2} \; .
\end{align*}

Nous trouvons que
$\displaystyle F = 19,600\times 29 \times \frac{\pi}{2} = 284,200\pi$
Newtons.

\PDFfig{8_integrales_appl/pression3}{La force exercée sur la
surface courbée externe d'un réservoir circulaire qui repose au font
d'un lac}{La force exercée sur la surface courbée externe
d'un réservoir circulaire qui repose au font d'un lac de $30$ mètres
de profondeur.}{PRESS3}

\subQ{ii}
Pour calculer la force exercée sur le côté courbé du cylindre illustré à la
figure~\ref{PRESS3}, il faut calculer l'aire $A(x_i^\ast)$ de la surface
courbée $P_i$ du réservoir entre $x_i$ et $x_{i+1}$ mètres.  Cette surface
représente un rectangle dont deux des côtés ont $5$ mètres de longueur et la
longueur des deux autres côtés est la longueur de l'arc de cercle
$y=\sqrt{1-(x+29)^2}$ pour $x$ entre $x_i$ et $x_{i+1}$.

Nous montrerons à la section sur le calcul de la longueur d'une courbe que, si
$\Delta x$ est assez petit, la longueur de l'arc de cercle
$y=\sqrt{1-(x+29)^2}$ pour $x$ entre $x_i$ et $x_{i+1}$ est approximativement
$\sqrt{1+(y'(x_i^\ast))^2} \, \Delta x$ pour un nombre $x_i^\ast$ entre $x_i$
et $x_{i+1}$.  Donc
\[
A(x_i^\ast) \approx 5 \sqrt{1+(y'(x_i^\ast))^2} \, \Delta x
= \frac{5}{\sqrt{1-(x_i^\ast+29)^2}} \, \Delta x
\]
car $y'(x) = -(x+29)/\sqrt{1-(x+29)^2}$.
Ainsi, $A(x_i^\ast) \approx S(x_i^\ast) \, \Delta x$ où
\[
S(x) = \frac{5}{\sqrt{1-(x+29)^2}} \; .
\]

Il découle de (\ref{pression}) avec $h_3 = -30$ et $h_2 = -28$ que la
force exercée sur l'un des deux côtés courbés est
\[
F = -\int_{h_3}^{h_2} 9.8 \, x\, \rho(x)\, S(x)\dx{x}
= - 5\times 9,800 \int_{-30}^{-28} \frac{x}{\sqrt{1-(x+29)^2}} \dx{x} \; .
\]
Si $y = x+29$, alors $\dx{y} = \dx{x}$, $y=-1$ pour $x=-30$
et $y=1$ pour $x=-28$.  Donc
\[
F = -49,000 \int_{-1}^{1} \frac{y-29}{\sqrt{1-y^2}} \dx{y}
= - 49,000 \int_{-1}^{1} \frac{y}{\sqrt{1-y^2}} \dx{y}
+ 49,000\times 29 \int_{-1}^{1} \frac{1}{\sqrt{1-y^2}} \dx{y} \; .
\]

Puisque $y/\sqrt{1-y^2}$ est une fonction impaire que nous intégrons sur un
intervalle symétrique à l'origine, nous avons
\[
\int_{-1}^{1} \frac{y}{\sqrt{1-y^2}} \dx{y} = 0 \; .
\]
De plus,
\[
\int_{-1}^{1} \frac{1}{\sqrt{1-y^2}} \dx{y} =
\arctan(y)\bigg|_{y=-1}^1 = \pi \; .
\]

Nous trouvons que $F = 49,000\times 29 \; \pi = 1,421,000\pi$ Newtons.
\end{egg}

\section{Centre de masse \eng}

Pour cette section, nous assumons que le lecteur sait ce qu'est un vecteur
et connaît les opérations de bases sur les vecteurs.  Le matériel que
nous retrouvons dans les deux premières sections du
chapitre~\ref{CHAPvecteurs} suivit amplement.

Le {\bfseries centre de masse}\index{Centre de masse} d'une mince
surface plane $S$ est le point de la surface pour lequel la surface
sera en été d'équilibre si nous la déposons sur un cône de telle sorte que
la pointe du cône soit en contact avec ce point.  

\PDFfig{8_integrales_appl/mass2}{Centre de masse (cas discret)}{Centre de
masse d'une surface plane dont la masse est concentrée en un nombre fini de
points}{CDMassA}

Supposons que la masse de la surface plane $S$ soit concentrée en $n$ points
de $S$ (figure~\ref{CDMassA}).  Si $m_j$ est la masse au point
$\VEC{v}_j$ pour $1\leq j \leq n$, nous cherchons le point $\VEC{x}$
de $S$ tel que
\begin{equation} \label{massequA}
\sum_{j=1}^n m_j(\VEC{v}_j-\VEC{x}) = \VEC{0} \ .
\end{equation}
La somme des forces exercées sur le point $\VEC{x}$ est nulle.  Si
$\displaystyle \VEC{v}_j = \begin{pmatrix} p_j \\ q_j \end{pmatrix}$ et
$\VEC{x} = \begin{pmatrix} p \\ q \end{pmatrix}$, nous obtenons des coordonnées
de (\ref{massequA}) que
\[
\sum_{j=1}^n m_j (p_j - p) = 0 \quad \text{et} \quad
\sum_{j=1}^n m_j (q_j - q) = 0 \ .
\]
Donc
\[
p = \frac{\sum_{j=1}^n m_j p_j}{\sum_{j=1}^n m_j} \quad \text{et} \quad
q = \frac{\sum_{j=1}^n m_j q_j}{\sum_{j=1}^n m_j} \ .
\]
$\displaystyle M_x = \sum_{j=1}^n m_j q_j$ est le
{\bfseries moment par rapport à l'axe des $\mathbf x$}\index{Moment par
  rapport à l'axe des $x$}
et $\displaystyle M_y = \sum_{j=1}^n m_j p_j$ est le
{\bfseries moment par rapport à l'axe des $\mathbf y$}\index{Moment par
  rapport à l'axe des $y$}.  La masse totale est
$\displaystyle m = \sum_{j=1}^n m_j$.  Le
{\bfseries centre de masse}\index{Centre de masse} est
$\displaystyle \left( \frac{M_y}{m} , \frac{M_x}{m}\right)$.

Supposons maintenant que la surface plane $S$ soit de densité
constante $\rho$.\footnote{Les unités de la densité sont de la forme
masse/aire car nous assumons que la surface n'a pas d'épaisseur ou, du
moins, que nous pouvons ignorer l'épaisseur de la surface.}
De plus, supposons que la surface $S$ soit de la forme
$\{ (x,y) : g(x) \leq y \leq f(x) , a \leq x \leq b\}$ pour deux
fonctions continues $f$ et $g$ (figure~\ref{CDMassB}).

\PDFfig{8_integrales_appl/mass1}{Centre de masse (densité constante)}{Centre de
masse d'une surface plane de densité constante}{CDMassB}

Pour estimer le moment par rapport à l'axe des $y$ de $S$, partageons
le segment $[a,b]$ en $k$ sous intervalles $[x_j,x_{j+1}]$ de longueur
$\Delta x = (b-a)/k$ où $x_i = a + j \Delta x$ pour $j=0$, $1$, $2$,
\ldots , $k$.  Choisissons $x_j^\ast \in [x_j,x_{j+1}]$ pour $j=0$,
$1$, $2$, \ldots , $k-1$.  Nous avons donc
\[
  M_y \approx \sum_{j=0}^{k-1} x_j^\ast \rho\;
  \left( f(x_j^\ast) - g(x_j^\ast)\right) \Delta x
\]
car $x_j^\ast$ est la distance
approximative entre l'axe des $y$ et les points de la portion
$S_j = \{ (x,y) : g(x) \leq y \leq f(x) , x_j \leq x \leq x_{j+1}\}$
de $S$,
et $\rho \;\left(f(x_j^\ast) - g(x_j^\ast)\right) \Delta x$ est la masse
approximative de la portion $S_j$ de $S$.

De même, nous pouvons estimer le moment par rapport à l'axe des $x$ de $S$ à
l'aide de la formule
\[
M_x \approx \sum_{j=0}^{k-1} \frac{(f(x_j^\ast)+g(x_j^\ast))}{2}
\rho\; \left(f(x_j^\ast) - g(x_j^\ast)\right) \Delta x =
\sum_{j=0}^{k-1}\rho\; \frac{1}{2}
\left( f^2(x_j^\ast)-g^2(x_j^\ast)\right) \Delta x
\]
où $(f\left(x_j^\ast)+g(x_j^\ast)\right)/2$ est la distance moyenne
entre l'axe des $x$ et les points de la portion $S_j$ de $S$ et, comme
précédemment, $\rho\; \left(f(x_j^\ast) - g(x_j^\ast)\right) \Delta x$
est la masse approximative de la portion $S_j$ de $S$.

Lorsque $k\to \infty$, nous obtenons les formules
\[
M_y = \rho \int_a^b x(f(x)-g(x)) \dx{x} \qquad \text{et} \qquad
M_x = \frac{\rho}{2} \int_a^b \left(f^2(x)-g^2(x) \right) \dx{x}
\]
pour le
{\bfseries moment par rapport à l'axe des $\mathbf x$}\index{Moment
par rapport à l'axe des $y$} et le 
{\bfseries moment par rapport à l'axe des $\mathbf x$}\index{Moment par
rapport à l'axe des $x$} respectivement.

La masse de $S$ peut être estimer à l'aide de la formule
\[
  m \approx \sum_{j=0}^{k-1} \rho\;
  \left(f(x_j^\ast) - g(x_j^\ast)\right) \Delta x \ .
\]

Lorsque $k\to \infty$, nous obtenons, nous obtenons que La masse de
$S$ est donnée par
\[
m = \rho \int_a^b (f(x)-g(x)) \dx{x} \ .
\]

Le {\bfseries centre de masse}\index{Centre de masse} est
$\displaystyle \left( \frac{M_y}{m} , \frac{M_x}{m}\right)$.

\begin{egg}
Trouvons le centre de masse de la surface bornée par les courbes $y=1-x^2$ et
$y=0$.  La densité de la surface est $\rho = 10$.

La courbe $y=1-x^2$ coupe l'axe des $x$ aux points $x=-1$ et $x=1$.
La masse de la surface est donc
\[
m = \int_{-1}^1 10(1-x^2) \dx{x} = 10
\left( x - \frac{x^3}{3}\right)\bigg|_{-1}^1 = \frac{40}{3} \ .
\]
Le moment par rapport à l'axe des $x$ est
\[
M_x = \frac{10}{2}  \int_{-1}^1 (1-x^2)^2 \dx{x}
= 5 \int_{-1}^1 (x^4 - 2x^2 + 1) \dx{x}
= 5\left( \frac{x^5}{5} - \frac{2x^3}{3} + x \right)\bigg|_{-1}^1
= \frac{16}{3} \ .
\]
et le moment par rapport à l'axe des $y$ est
\[
M_y = 10 \int_{-1}^1 x(1-x^2) \dx{x} = 0
\]
car la fonction $h(x) = x(1-x^2)$ est impaire et le domaine d'intégration
est symétrique par rapport à l'axe des $y$.  Il n'est pas surprenant que
$M_y$ soit nul car la surface est symétrique par rapport à l'axe des
$y$ et elle est de densité constante.

Le centre de masse est
$\displaystyle \left(\frac{M_y}{m},\frac{M_x}{m}\right) = (0, 2/5)$.
\end{egg}

\section{Débit sanguin \life}

Le modèle de Poiseuille pour la distribution de vélocité d'un liquide visqueux
et incompressible qui s'écoule dans un tube de rayon constant $R$ cm et de
longueur $L$ cm (figure~\ref{SANG}) est
\[
v(r) = \frac{P}{4\eta L}\big( R^2-r^2\big) \quad , \quad 0 \leq r \leq R \ ,
\]
où $v(r)$ cm/s est la vélocité du liquide à un distance $r$ cm de l'axe
central du tube, $P$ dyn/cm$^2$ est la différence de pression entre les
deux extrémités du tube et $\eta$ est le coefficient de viscosité du liquide.
Dans ce modèle, nous supposons que la vélocité varie seulement en fonction de la
distance par rapport à l'axe du tube.  Nous supposons de plus que la
vélocité est nulle sur la paroi du tube.  Ce sont de bonnes
suppositions si le tube est un vaisseau sanguin et le liquide est du
sang.  La formule pour la vélocité donnée ci-dessus est une
approximation de la solution de l'équation de Navier et Stokes en
mécanique des fluides (\cite{BE}).

\PDFfig{8_integrales_appl/sang}{Écoulement laminaire}{La distribution de
vélocité d'après le modèle de Poiseuille}{SANG}

Calculons le débit dans le tube; c'est-à-dire, le volume de liquide
qui passe par une section du tube par unité de temps.  Pour ce faire, nous
utilisons la technique des sommes de Riemann.

Soit $k$, un entier positif.  Posons $\Delta r = R/k$ et
$r_j = j \Delta r$ pour $j=0$, $1$, $2$, \ldots, $k$.  Nous obtenons une
partition de l'intervalle $[0,R]$ en sous-intervalles de la forme
$[r_j,r_{j+1}]$ pour $j=0$, $1$, \ldots, $k-1$.

Nous considérons l'anneau définie par $r_j \leq r \leq r_{j+1}$.  Pour estimer
$D_j$, le volume de liquide qui passe au travers de cet anneau par unité de
temps, nous supposons que la vélocité du liquide est presque constante
dans cet anneau.  L'aire de l'anneau définie par
$r_j \leq r \leq r_{j+1}$ est
\[
\pi r_{j+1}^2 - \pi r_j^2 = 2\pi \left(\frac{r_{j+1}+r_j}{2} \right)
(r_{j+1}-r_j) = 2\pi r_j^\ast \Delta r \ ,
\]
où $r_j^\ast = (r_{j+1}+r_j)/2$.  Ainsi,
\[
D_j \approx 2\pi r_j^\ast v(r_j^\ast) \Delta r \ .
\]
Nous avons donc que $D$, le volume de liquide qui passe par une
section du tube par unité de temps, est donné approximativement par
\[
D = \sum_{j=0}^{k-1} D_j \approx 2\pi \sum_{j=0}^{k-1} 
r_j^\ast v(r_j^\ast) \Delta r \ .
\]
C'est une somme de Riemann pour l'intégrale
\[
2\pi \int_0^R r v(r) \dx{r} \ .
\]
Donc, si $k$ tend vers plus l'infini, nous obtenons
\[
D = 2\pi \int_0^R r v(r) \dx{r} \ .
\]
Pour le modèle de Poiseuille, le débit est donc donné par la formule
\[
D = 2\pi \int_0^R r v(r) \dx{r} 
= \frac{\pi P}{2\eta L} \int_0^R r \big( R^2-r^2\big) \dx{r}
= \frac{\pi P}{2\eta L} \left(\frac{r^2R^2}{2}-\frac{r^4}{4}\right)\bigg|_0^R
= \frac{\pi P R^4}{8\eta L} \ .
\]
Les unités de $D$ sont les cm$^3$/s.

\begin{egg}
Nous voulons mesurer le diamètre de l'aorte ascendante d'une
patiente afin de déterminer le risque de celle-ci d'avoir un accident
cardio-vasculaire.  Nous ne voulons pas avoir recours à une chirurgie.

Si nous déterminons le débit cardiaque $D$ (i.e.\ le débit sanguin
dans l'aorte ascendante), alors la formule
$\displaystyle D = \frac{\pi P  R^4}{8\eta L}$ 
peut être utilisée pour déterminer le rayon $R$ de l'aorte car la pression
sanguine $P$ peut être mesurée, $\eta$ est une constante connue pour le sang,
et $L$ est la longueur de l'aorte ascendante.  Pour un adulte de grandeur
moyenne, $L \approx 6$ cm, $P \approx 106,658$ dyn/cm$^2$ et
$\eta \approx 0.027$ \mbox{s\ dyn/cm$^2$}.

Pour mesurer expérimentalement le débit cardiaque, nous injectons $A$ mg d'un
{\em produit de contraste} dans le coeur (naturellement, pas directement dans
le coeur) et nous mesurons le débit sanguin dans l'aorte ascendante.  Nous
estimons qu'il faut $T$ secondes pour que le coeur de la patiente se
vide du produit de contraste.  Nous allons utiliser les sommes de Riemann
pour calculer la quantité total du produit de contraste qui passe par
l'aorte ascendante pendant la période de temps $T$.

Soit $k$, un entier positif.  Posons $\Delta t = T/k$ et
$t_j = j \Delta t$ pour $j=0$, $1$, $2$, \ldots, $k$.  Nous obtenons une
partition de l'intervalle $[0,T]$ en sous-intervalles de la forme
$[t_j,t_{j+1}]$ pour $j=0$, $1$, \ldots, $k-1$.

Soit $c(t)$ la concentration du produit de contraste en mg/cm$^3$ au
temps $t$ dans l'aorte ascendante.  Si nous supposons que la
concentration du produit de contraste est presque constante entre
$t_j$ et $t_{j+1}$, la quantité $Q_i$ du produit de contraste qui passe par
l'aorte ascendante entre $t_i$ et $t_{i+1}$ est approximativement
\[
Q_i \approx c(t_i) \left( \frac{\pi R^4 P}{8\eta L}\right) \Delta t
\]
car $\displaystyle \left( \frac{\pi R^4 P}{8\eta L}\right) \Delta t$
représente le volume de sang qui passe par l'aorte ascendante durant la
période de temps $\Delta t$.  La quantité total $Q$ du produit de contraste
qui passe par l'aorte ascendante durant les $T$ secondes est donc donnée
approximativement par
\[
Q = \sum_{i=0}^{k-1} Q_i \approx
\sum_{i=0}^{k-1} c(t_i) \left( \frac{\pi R^4 P}{8\eta L}\right) \Delta t \ .
\]
C'est une somme de Riemann pour l'intégrale
\[
\left( \frac{\pi R^4 P}{8\eta L}\right) \int_0^T c(t) \dx{t} \ .
\]
Donc, si $k$ tend vers plus l'infini, nous obtenons
\[
Q = \frac{\pi R^4 P}{8\eta L} \int_0^T c(t) \dx{t} \ .
\]
Or, nous aurons $Q=A$ après $T$ secondes. Le débit est donc
\[
D = \frac{\pi R^4 P}{8\eta L} = A\bigg/\left(\int_0^T c(t) \dx{t}\right) \ .
\]

Supposons que $6$ mg du produit de contraste soit injecté dans le coeur de la
patiente, et que la concentration du produit de contraste mesurée à
intervalle régulier donne les résultats suivants.
\[
\begin{array}{cr|c|c|c|c|c|c|c|c|c|c|c}
t &\text{(s)} & 0 & 0.5 & 1 & 1.5 & 2 & 2.5 & 3 & 3.5 & 4 & 4.5 & 5 \\
\hline
c(t) & \text{(mg/cm$^3$)} & 0 & 0.021 & 0.045 & 0.073 & 0.058 & 0.036 &
0.028 & 0.014 & 0.006 & 0.002 & 0
\end{array}
\]
Évaluons le débit sanguin avec les sommes à droite.  De plus, si
$P \approx 106,658$ dyn/cm$^2$, $L=6$ cm et $\eta \approx 0.027$,
déterminons le diamètre de l'aorte ascendante de cette patiente.

Posons $\displaystyle t_i = \frac{i}{2}$ pour $i=0$, $1$, \ldots,
$10$.  Nous avons
\[
\int_0^{5} c(t) \dx{t} \approx \sum_{i=1}^{10} c(t_i) \Delta t
= \frac{1}{2} \left( 0.021 + 0.045 + \ldots + 0\right) = 0.1415
\ \text{ s\;mg/cm${^3}$} \ .
\]
Le débit est donc
\[
D = \frac{6}{0.1415} \approx 42.4028 \quad \text{cm$^3$/s} \ .
\]
Nous déduisons de la formule
$\displaystyle D = \frac{\pi P R^4}{8\eta L}$ que
\[
R = \left(\frac{8\eta L D}{\pi P}\right)^{1/4} \approx 0.11317 \
\text{cm} \ .
\]
Le diamètre de l'aorte ascendante est donc approximativement de $0.2263$ cm.
Pour un adulte de grandeur normale, le diamètre de l'aorte ascendante est
beaucoup plus grand (au moins $10$ fois plus grand).  La patiente est
a risque.
\end{egg}

\section{Applications à l'économie \eco}

Avant de donner un exemple de l'utilité de l'intégrale en économie, il faut
introduire quelques concepts de base.

Si nous déposons dans un compte en banque la somme de $M_0$ dollars et que le
taux d'intérêt est de $I$\% par année, après un an nous aurons
\[
M_1 = M_0 + \frac{I}{100}\,M_0 = M_0\left(1+\frac{I}{100}\right) \;
\text{dollars}\; .
\]
C'est le montant en banque au début de l'année plus les intérêts pour
l'année.  Après deux ans nous aurons
\[
M_2 = M_1 + \frac{I}{100}\,M_1 = M_1\left(1+\frac{I}{100}\right) \quad
\text{dollars}\; .
\]
C'est le montant en banque au début de la deuxième année plus les intérêts
pour l'année.  Donc
\[
M_2 = M_0\left(1+\frac{I}{100}\right)^2 \quad \text{dollars} \; .
\]
Par induction, après $n$ années, nous obtenons le résultat suivant.

\begin{focus}{\mth} \index{Intérêt!Composé annuellement}
Si $M_0$ est le dépôt initial (en dollars) et si l'intérêt est
{\bfseries composé annuellement}
au taux de $I$\%, alors le montant $M_n$ dans le compte après $n$
années est donnée par la formule
\[
M_n = M_0\left(1+\frac{I}{100}\right)^n \quad \text{dollars} \; .
\]
\end{focus}

Si nous déposons dans un compte en banque la somme de $M_0$ dollars et que le
taux d'intérêt est de $I$\% par année calculé $k$ fois par année, à la fin
de la première période de l'année (i.e. au premier versement des intérêts),
nous aurons
\[
M_{0,1} = M_0+\frac{I}{100k}\,M_0 = M_0\left(1+\frac{I}{100k}\right)
\quad \text{dollars}\; .
\]
C'est le montant en banque au début de la première période plus les intérêts
pour la période qui sont $1/k$ des intérêts de l'année.  À la fin de la
deuxième période de l'année (i.e. au deuxième versement des intérêts), nous
aurons
\[
M_{0,2} = M_{0,1}+\frac{I}{100k}\,M_{0,1}
= M_{0,1}\left(1+\frac{I}{100k}\right)
\quad \text{dollars}\; .
\]
C'est le montant en banque au début de la deuxième période plus les intérêts
pour la période qui sont encore $1/k$ des intérêts de l'année.  Il est
naturellement assumé que l'année est divisée en $k$ périodes de même durées.
Nous avons donc
\[
M_{0,2} = M_0\left(1+\frac{I}{100k}\right)^2 \quad  \text{dollars}\; .
\]
Par induction, à la fin de l'année (c'est à dire à la fin de la $k^e$
période ou, si vous préférez, au $k^e$ versement), nous aurons
\[
M_1 = M_{0,k} = M_0\left(1+\frac{I}{100k}\right)^k \quad
\text{dollars}\; .
\]
De même, à la fin de la deuxième année, nous aurons
\[
M_2 = M_1\left(1+\frac{I}{100k}\right)^k \quad
\text{dollars}\; .
\]
Donc
\[
M_2 = M_0\left(1+\frac{I}{100k}\right)^{2k} \quad
\text{dollars}\; .
\]
Par induction, nous obtenons le résultat suivant.

\begin{focus}{\mth} \index{Intérêt!Composé $k$ fois par année}
Si $M_0$ est le dépôt initial (en dollars) et si l'intérêt est
{\bfseries composé $k$ fois par année} au taux de $I$\%, alors le
montant $M_n$ dans le compte après $n$ années est donnée par la formule
\[
M_n = M_0\left(1+\frac{I}{100k}\right)^{nk} \quad \text{dollars} \; .
\]
\end{focus}

Dans le cas précédent où l'intérêt est composé $k$ fois par année, si le
nombre $k$ de périodes devient de plus en plus grand (tend vers plus
l'infini), nous obtenons la formule suivante.

\begin{focus}{\mth} \index{Intérêt!Composé de façon continue} 
Si $M_0$ est le dépôt initial (en dollars) et si l'intérêt est
{\bfseries composé de façon continue}
au taux $I$\% par année, alors le montant $M_n$ dans le compte après $n$
année est donnée par la formule
\begin{equation} \label{eco_cont}
M_n = M_0 \, e^{(nI)/100} \; .
\end{equation}
\end{focus}

Dans le cas continue, $n$ peut être un nombre réel.  Nous ne sommes plus
limité aux entiers comme dans les deux premiers cas.

Pour démontrer la formule (\ref{eco_cont}), il suffit de noter que
\[
\lim_{k\rightarrow \infty} M_0\left(1+\frac{I}{100k}\right)^{nk}
= \lim_{k\rightarrow \infty} M_0\left(
\underbrace{\left(1+\frac{1}{100k/I}\right)^{100k/I}}_{\rightarrow e
\text{ lorsque } k \rightarrow \infty} \right)^{nI/100}
= M_0 \, e^{nI/100} \; .
\]

\subsection{Valeurs présentes et futures}

\begin{focus}{\dfn} \index{Valeur future}
La {\bfseries valeur future} d'un montant $M_0$
est le montant obtenu après $t$ années si le montant $M_0$ est placé
aujourd'hui dans un compte en banque dont le taux d'intérêt est de
$I$\% par année.
\end{focus}

Par exemple, si l'intérêt est composé de façon continue, la valeur
future après $n$ année est $M_n = M_0 e^{nI/100}$.

\begin{focus}{\dfn} \index{Valeur présente}
La {\bfseries valeur présente} d'un montant
$M_t$ est le montant $M_0$ qu'il faut placer aujourd'hui dans un
compte en banque dont le taux d'intérêt est de $I$\% pour obtenir
$M_t$ dans $t$ années.
\end{focus}

Par exemple, si l'intérêt est composé de façon continue, la valeur
présente de $M_t$ dollars dans $t$ années est
$M_0 = M_t e^{-I\,n/100}$.  Si l'intérêt est composé $k$ fois par
année, la valeur présente de $M_n$ dollars dans $n$ 
années est
$\displaystyle M_0 = M_n \left(1 + \frac{I}{100k}\right)^{-k\,n}$.

\begin{egg}
Vous venez de gagner \$$1,000,000$ à la loterie.  Le montant vous est donné
en quatre versements égaux; le premier versement aujourd'hui, le deuxième
dans un an à partir d'aujourd'hui, le troisième dans deux ans à partir
d'aujourd'hui, etc.  En Assumant que le taux d'intérêt pour les
prochaines années est de $4$\% composé de façon continue,
avez-vous réellement gagné \$$1,000,000$?  

Il faut déterminer la valeur présente de l'argent qui vous est donné.  La
valeur présente du montant de \$$250,000$ qui vous sera donné dans un an est
$\displaystyle 250,000\,e^{-4/100}$, la valeur présente du montant de
\$$250,000$ qui vous sera donné dans deux ans est
$\displaystyle 250,000\,\left(e^{-4/100}\right)^2$, etc.

La valeur présente de montant que vous avez gagné à la loterie est donc
\begin{align*}
& 250,000 + 250,000\,e^{-4/100} + 250,000\,\left(e^{-4/100}\right)^2
+ 250,000\,\left(e^{-4/100}\right)^3 \\
&= 250, 000 \left( \frac{1- \left(e^{-4/100}\right)^4}{1-e^{-4/100}} \right)
= 942,706.56 \quad \text{dollars}\; .
\end{align*}
\end{egg}

En raison de leur très grand volume de vente, nous pouvons supposer que le
taux de revenu d'une grande entreprisse (e.g. Une chaîne de restaurants
ou de magasins à rayons) est donné par une fonction continue $R(t)$ où
$t$ est le temps.  Puisque $R(t)$ est un taux de revenu, ses unités
sont des dollars/jour, des dollars/année, des euros/mois, etc.  Le
temps $t$ utilise les mêmes unités de temps que ceux utilisé pour le
taux de revenu.  Par exemple, si le taux de revenu est en
dollars/jour, le temps $t$ sera en jours.

Si l'entreprisse investi ses revenus dans un compte dont le taux
d'intérêt est de $I$\% composé de façon continue, quelle sera la valeur
future $V_f$ des placements de cette entreprisse au temps $T$?

Soit $k$, un entier positif.  Posons
$\Delta t = T/k$ et $t_i = i \Delta t$
pour $i=0$, $1$, $2$, \ldots, $k$.  Nous obtenons une partition de l'intervalle
$[0,T]$ en sous-intervalles de la forme $[t_i,t_{i+1}]$ pour $i=0$, $1$,
\ldots, $k-1$.

Pour $i=0$, $1$, $2$, \ldots, $k-1$, nous choisissons $t_i^\ast$ dans
l'intervalle $[t_i,t_{i+1}]$.  Sur l'intervalle $[t_i, t_{i+1}]$, les
revenus sont approximativement de $R(t_i^\ast)\, \Delta t$.  La valeur
future de ces revenus est alors approximativement
$\displaystyle e^{I\,(T-t_i^\ast)/100} \,R(t_i^\ast)\, \Delta t$ car ces
revenus sont investis pour une période d'environ $T-t_i^\ast$.

La somme des valeurs futures estimées pour chaque intervalle
$[t_i,t_{i+1}]$ donne l'approximation suivant de la valeur future $V_f$
des placements de l'entreprisse pour une période $T$.
\[
V_f \approx \sum_{i=0}^{k-1} \,e^{I\,(T-t_i^\ast)/100} \,R(t_i^\ast) \Delta t \; .
\]
Remarquons que cette somme est une somme de Riemann pour l'intégrale
\[
\int_0^T e^{I\,(T-t)/100} \,R(t) \dx{t} \; .
\]
Ainsi, si $k$ tend vers plus l'infini, nous obtenons le résultat suivant.

\begin{focus}{\mth} \index{Valeur future}
Si $R(t)$ représente est le taux investi dans un compte au temps $t$
et si le taux d'intérêt est de $I$\% composé de façon continue, alors la
valeur future $V_f$ des investissement après une période $T$ est
donnée par la formule
\[
V_f = \int_0^T e^{I\,(T-t)/100} \,R(t) \dx{t} \; .
\]
\end{focus}

De façon semblable, nous pouvons obtenir la valeur présente $V_p$ des gains
d'une grande entreprisse pour une période de temps $T$.  Nous assumons que
le taux d'intérêt sur les placements est de $I$\% composé de façon
continue.

Soit $k$, un entier positif.  Posons
$\Delta t = T/k$ et $t_i = i \Delta t$
pour $i=0$, $1$, $2$, \ldots, $k$.  Nous obtenons une partition de l'intervalle
$[0,T]$ en sous-intervalles de la forme $[t_i,t_{i+1}]$ pour $i=0$, $1$,
\ldots, $k-1$.

Pour $i=0$, $1$, $2$, \ldots, $k-1$, nous choisissons $t_i^\ast$ dans
l'intervalle $[t_i,t_{i+1}]$.  Sur l'intervalle $[t_i, t_{i+1}]$, les
revenus sont
approximativement de $R(t_i^\ast)\, \Delta t$.  La valeur présente de ces
revenus est alors approximativement
$\displaystyle e^{-I\,t_i^\ast/100} \,R(t_i^\ast)\, \Delta t$.

La somme des valeurs présentes estimées pour chaque intervalle
$[t_i,t_{i+1}]$ donne l'approximation suivant de la valeur présente
$V_p$ des revenus de l'entreprisse pour une période $T$.
\[
V_p \approx \sum_{i=0}^{k-1} \,e^{-I\;t_i^\ast/100} \,R(t_i^\ast)
\Delta t \; .
\]
Remarquons que cette somme est une somme de Riemann pour l'intégrale
\[
\int_0^T e^{-I\,t/100} \,R(t) \dx{t} \; .
\]
Ainsi, si $k$ tend vers plus l'infini, nous obtenons le résultat suivant.

\begin{focus}{\mth} \index{Valeur présente}
Si $R(t)$ représente est le taux investi dans un compte au temps $t$
et si le taux d'intérêt est de $I$\% composé de façon continue, alors la
valeur présente $V_p$ des investissement pour une période $T$ est
donnée par la formule
\[
V_p = \int_0^T e^{-I\;t/100} \,R(t) \dx{t} \; .
\]
\end{focus}

\begin{egg}
Si le taux de revenus d'une chaîne de locations de voitures est
$R(t) = 10,000$ dollars/jour, quelle sera la valeur future des revenus
de cette chaîne de locations dans $10$ ans si elle investi ses revenus
à un taux d'intérêt de $5$\% composé de façon continue.

Nous avons que la valeur future $V_f$ est
\[
V_f = \int_0^T e^{I\,(T-t)/100} \,R(t) \dx{t} \; ,
\]
où $T=10$, $I=5$ et $R(t)=10,000$ pour tout $t$.  Donc
\begin{align*}
V_f &= \int_0^{10} 10,000\, e^{5\,(10-t)/100} \dx{t}
= -2\times 10^5\, e^{5\,(10-t)/100} \bigg|_{t=0}^{10} \\
&= -2\times 10^5 \left(1-e^{0.5}\right) \approx 129,744.25 \text{ dollars}.
\end{align*}
\end{egg}

\subsection{Surplus du consommateur et du producteur} 

Soit $D(q)$ le prix par unité que le consommateur est prêt à payer s'il y a
$q$ unités d'un certain produit disponibles sur le marché et soit
$P(q)$ le prix par unité que le producteur (manufacturier, fermier,
\ldots) est prêt a demander s'il y a $q$ unités de ce produit
disponibles sur le marché.

Il est raisonnable de penser que $D$ est une fonction décroissante.
Quand le nombre d'unités disponibles sur le marché est grand, les
consommateurs espèrent que les vendeurs vont baisser le prix par unité
pour pouvoir écouler leur stock.  Par contre, la fonction $P$ sera
probablement croissante.  Si les marchants commandent un grande nombre
d'unités, les producteurs vont augmenter le prix par unité pour
bénéficier de l'enjouement pour le produit et pour pouvoir augmenter
leur production.

Nous retrouvons à la figure~\ref{ECO1} le graphe de $D$ et celui de $P$.  Dans
cette figure, $q_0$ est le nombre maximal d'unité que le marché peut
supporter (tous ceux susceptibles d'acheter le produit en ont déjà fait
l'achat), $p_0$ est le prix minimal par unité que le producteur est
prêt à demander, et $p_1$ est le prix maximal par unité que le
consommateur est prêt à payer.  Le point $q^\ast$ est le
{\bfseries point d'équilibre} pour le nombre d'unités sur le marché.  À ce
point, les consommateurs et producteurs tirent avantage
du prix de $p^\ast$ par unité.

\PDFfig{8_integrales_appl/eco1}{Représentation graphique du surplus du
consommateur et du producteur}{La région en bleu représente le surplus
du consommateur alors que la région en vert représente le surplus du
producteur.}{ECO1}

Le {\bfseries surplus du consommateur}\index{Surplus du consommateur}
$S_c$ est le montant économisé par les consommateurs s'ils payent
$p^\ast$ au lieu du prix par unité qu'ils auraient été prêt à payer
normalement pour $q < q^\ast$.
\[
S_c = \int_0^{q^\ast} D(q) \dx{q} - p^\ast \, q^\ast \; .
\]
C'est l'aire de la région en bleu représenté à la figure~\ref{ECO1}. 

Le {\bfseries surplus du producteur}\index{Surplus du producteur}
$S_p$ est le revenu additionnel fait par les producteurs ou
manufacturiers s'ils demandent $p^\ast$ au lieu du prix par unité
qu'ils auraient été prêt à demander normalement pour $q < q^\ast$.
\[
S_c = p^\ast \, q^\ast - \int_0^{q^\ast} P(q) \dx{q} \; .
\]
C'est l'aire de la région en vert représenté à la figure~\ref{ECO1}. 

\begin{egg}
En mai $2003$, nous retrouvions sur le marché $50$ unités d'un
modèle de voiture de lux que le consommateur (pour ce genre de voitures)
était prêt à acheter pour \$$370,000$ l'unité.  Nous avons estimé que le point
d'équilibre pour le nombre d'unités était de $350$ voitures au prix de
\$$250,000$ par unités.  Quelle a été le surplus du consommateur si le
prix par unité que le consommateur est prêt à payer est une fonction
affine du nombre d'unités disponibles?

Le graphe de $D$ est une droite qui passe par les points
$(50, 370,000)$ et \\
$(q^\ast, p^\ast) = (350, 250,000)$.
L'équation de cette droite est
\[
P(q) = \frac{370,000-250,000}{50-350} \, (p-350) + 250,000
= -400 (p-350) + 250,000 \; .
\]
Le surplus du consommateur est
\begin{align*}
S_c &= \int_0^{350} \left(-400 (p-350) + 250,000\right)\dx{p} -
350\times 250,000 \\
&= \left(-200 (p-350)^2 + 250,000 \, p \right)\bigg|_{p=0}^{350}
- 350\times 250,000
= 24,500,000 \text{ dollars}.
\end{align*}
\end{egg}

\section{Test de l'intégrale \eng}\label{integralTest}

Il est possible de déterminer si une série de termes positifs
$\displaystyle \sum_{n=1}^\infty a_n$ converge à l'aide d'une intégrale 
impropre.

\begin{focus}[][Le test de l'intégrale]{\thm} \index{Test de l'intégrale}
Soit $f:[1,\infty[\rightarrow [0,\infty[$ une fonction continue et
décroissante.  Si $\displaystyle \sum_{n=1}^\infty a_n$ est une série dont
les termes sont donnés par $a_n = f(n)$ pour tout $n$, alors la série
$\displaystyle \sum_{n=1}^\infty a_n$ converge si et seulement si l'intégrale
impropre $\displaystyle \int_1^\infty f(x)\dx{x}$ converge.
\end{focus}

La démonstration de ce théorème est simple et repose sur les deux
graphes donnés à la figure~\ref{int_TEST}.

\PDFfig{8_integrales_appl/int_test}{La justification du test de l'intégrale
pour déterminer si une série converge}{La justification du test de
l'intégrale pour déterminer si une série converge.}{int_TEST}

Posons $\displaystyle T_n = \int_1^n f(x)\dx{x}$ et
$\displaystyle S_n = \sum_{k=1}^n a_k$.  Les sommes $S_n$ sont les
sommes partielles de la séries.

La région en gris dans le dessin à gauche dans la
figure~\ref{int_TEST} représente la valeur de la somme
$\displaystyle \sum_{k=2}^6 a_k = S_6 - a_1$ alors que la région en gris
dans le dessin à droite dans la figure~\ref{int_TEST} représente la
valeur de la somme $\displaystyle \sum_{k=1}^6 a_k = S_6$.

\subQ{A} Supposons que $\displaystyle \int_1^\infty f(x)\dx{x}$ diverge. Cela
implique que $\displaystyle \lim_{n\rightarrow \infty} T_n = +\infty$.
Or, à partir du dessin à droite dans la figure~\ref{int_TEST}, nous
obtenons que $0 \leq T_n \leq S_{n-1}$ pour tout $n\geq 2$.  Ainsi
\[
\lim_{n\rightarrow \infty} S_n =
\lim_{n\rightarrow \infty} S_{n-1} = +\infty
\]
et nous concluons que la séries $\displaystyle \sum_{n=1}^\infty a_n$ diverge.
Cela implique que si la série $\displaystyle \sum_{n=1}^\infty a_n$
converge alors l'intégrale impropre
$\displaystyle \int_1^\infty f(x)\dx{x}$ converge.

\subQ{B} Supposons que $\displaystyle \int_1^\infty f(x)\dx{x}$ converge.
Cela implique que la limite $\displaystyle \lim_{n\rightarrow \infty} T_n$
existe et est réel.  Or, à partir du dessin à gauche dans la
figure~\ref{int_TEST}, nous obtenons que
$0 \leq S_n - a_1 \leq T_n$ pour tout $n\geq 2$.  Donc
$0 \leq S_n \leq T_n + a_1$ pour tout $n\geq 2$.  Ainsi
$\displaystyle \{S_n\}_{n=1}^\infty$ est une suite croissante (car
$a_n \geq 0$ pour tout $n$) et bornée par
\[
a_1 + \lim_{n\rightarrow \infty} T_n = a_1 + \int_1^\infty f(x)\dx{x}
\]
Il découle du théorème~\ref{suiteBORN} que la suite
$\displaystyle \{S_n\}_{n=1}^\infty$ converge.  C'est à dire que
$\displaystyle \sum_{n=1}^\infty a_n$ converge.

\begin{rmk}
Dans la cas où la série $\displaystyle \sum_{n=1}^\infty a_n$
converge, et donc l'intégrale impropre
$\displaystyle \int_1^\infty f(x)\dx{x}$ converge, nous pouvons utiliser
une intégrale pour estimer $S - S_n$ où $S$ est la somme de la
série $\displaystyle \sum_{n=1}^\infty a_n$.
Notons en passant que la suite des sommes partielles
$\displaystyle \{S_n\}_{n=0}^\infty$ est une suite croissante car les
termes de la série $\displaystyle \sum_{n=1}^\infty a_n$ sont
positifs,  Donc
$0 \leq S_1 \leq S_2 \leq \ldots \leq S_n \leq \ldots \leq S$.

Nous pouvons conclure à partir du dessin à la figure~\ref{ITborne} que
\[
0 \leq S - S_n = \sum_{k=n+1}^\infty a_n \leq \int_n^\infty f(x) \dx{x} \ .
\]
La région en gris représente la somme
$\displaystyle \sum_{k=n+1}^\infty a_n$.
\end{rmk}

\PDFfig{8_integrales_appl/ITborne}{Approximation de la somme d'une
série de termes positifs convergente}{Le test de l'intégrale donne une
approximation de la somme d'une série de terme positifs convergente.   
La région en gris représente la différence entre la somme $S$ de la
série et sa $n^{e}$ somme partielle $S_n$.}
{ITborne}

\begin{egg}
Utilisons le test de l'intégrale pour fournir une autre
démonstration de le proposition~\ref{Pseries}

Si $p \leq 0$, alors la suite $\displaystyle \{n^{-p}\}_{n=1}^\infty$
ne converge pas vers zéro.  Donc, par le théorème~\ref{NCforCofS},
la séries $\displaystyle \sum_1^\infty \frac{1}{x^p} \dx{x}$
diverge.

Nous pouvons donc supposer que $p>0$.  La proposition~\ref{Pseries} est alors
une conséquence immédiate du résultat que nous avons obtenu à
la proposition~\ref{impr_comp1}; c'est-à-dire que l'intégrale impropre
$\displaystyle \int_1^\infty \frac{1}{x^p} \dx{x}$ converge si seulement et
seulement si $p>1$.

Considérons $f(x) = 1/x^p$.  Nous avons que $f(x)>0$ pour $x>0$ et
$f(x)$ est un fonction continue et décroissant pour $x>0$ car
$f'(x) = -p x^{-p+1} <0$ pour $x>0$.  Nous pouvons donc utiliser le test de
l'intégrale pour conclure que
\[
\sum_{n=1}^\infty f(n)= \sum_{n=1}^\infty \frac{1}{n^p}
\text{ converge} \Leftrightarrow 
\int_1^\infty f(x) \dx{x} = \int_1^\infty \frac{1}{x^p} \dx{x}
\text{ converge} \Leftrightarrow
p>1 \ .
\]
\end{egg}

\begin{egg}
Déterminons si la série $\displaystyle \sum_{n=1}^\infty n 5^{-n}$
converge ou diverge.

Cette série est de la forme $\displaystyle \sum_{n=1}^\infty a_n$ où
$a_n = f(n)$ pour $f(x) = x 5^{-x}$.

Nous avons que $f(x) > 0$ pour tout $x\geq 1$ et $f$ est une fonction
continue et décroissante; en fait,
$f'(x) = (1 - x\ln(5))5^{-x} < 0$ pour $x\geq 1$.

Considérons l'intégrale $\displaystyle \int_1^q x 5^{-x}\dx{x}$.
Cette intégrale se calcul à l'aide de la méthode
d'intégration par parties.  Nous avons que
$x 5^{-x} = u(x) \; v(x)$ pour $u(x) = x$ et $v'(x) = 5^{-x}$.  Donc
$u'(x)=1$, $v(x) = -5^{-x}/\ln(5)$ et
\begin{align*}
\int_1^q x 5^{-x} \dx{x}
&= \int_1^q u(x)v'(x)\dx{x}
= u(x)v(x)\bigg|_{x=1}^q - \int_1^q u'(x) v(x)\dx{x} \\
&= -\frac{x 5^{-x}}{\ln(5)}\bigg|_{x=1}^q
+ \int_1^q \frac{5^{-x}}{\ln(5)}\dx{x}
= -\frac{x 5^{-x}}{\ln(5)}\bigg|_{x=1}^q
- \frac{5^{-x}}{(\ln(5))^2}\bigg|_{x=1}^q \\
&= -\frac{q 5^{-q}}{\ln(5)} + \frac{1}{5\ln(5)}
- \frac{5^{-q}}{(\ln(5))^2} + \frac{1}{5(\ln(5))^2} \; .
\end{align*}
Grâce à la Règle de l'Hospital,
$\displaystyle \lim_{q\rightarrow \infty}q 5^{-q} = 
\lim_{q\rightarrow \infty} \frac{q}{5^q} = 0$.  De plus,
$\displaystyle \lim_{q\rightarrow \infty} 5^{-q} =
\lim_{q\rightarrow \infty} \frac{1}{5^q} = 0$.  Donc
\[
\lim_{q\rightarrow \infty} \int_1^q x 5^{-x} \dx{x}
= \frac{1}{5\ln(5)} + \frac{1}{5(\ln(5))^2} \; .
\]
Puisque $\displaystyle \int_1^q x 5^{-n}\dx{x}$ converge, la série
$\displaystyle \sum_{n=1}^\infty n 5^{-n}$ converge.
\end{egg}

\begin{egg}
Déterminons si la série
$\displaystyle \sum_{n=2}^\infty \frac{1}{n\ln{n}}$ converge ou diverge.

Cette série est de la forme $\displaystyle \sum_{n=2}^\infty a_n$ où
$\displaystyle a_n = f(n)$ pour
$\displaystyle f(x) = \frac{1}{x\ln(x)}$.

Nous avons que $f(x) > 0$ pour tout $x\geq 2$ et $f$ est une fonction
continue et décroissante car $x\ln(x)$ est croissante pour $x\geq 2$.

Considérons l'intégrale $\displaystyle \int_2^q \frac{1}{x\ln(x)} \dx{x}$.
Cette dernière intégrale se calcul à l'aide de la méthode de
substitution.  Si $u = \ln(x)$, alors
$\displaystyle \dx{u} = \frac{1}{x} \dx{x}$, $u = \ln(q)$ lorsque
$x=q$ et $u = \ln(2)$ lorsque $x=2$.  Ainsi,
\[
\int_2^q \frac{1}{x\ln(x)} \dx{x}
= \int_{\ln(2)}^{\ln(q)} \frac{1}{u} \dx{u} \ .
\]
Or, cet intégrale diverge d'après la proposition~\ref{impr_comp1};
c'est le cas $p=1$.  Donc, grâce au test de l'intégrale, la série
$\displaystyle \sum_{n=2}^\infty \frac{1}{n\ln{n}}$ diverge.
\end{egg}

}  % End of theory

\section{Exercices}

\subsection{Aire entre deux courbes}

\begin{question}
Calculez l'aire de chacune des régions décrient ci-dessous.

\subQ{a} La région entre la droite $y=2x$ et la courbe
$y=x^2$ pour $0\leq x \leq 2$.\\
\subQ{b} La région entre la droite $y=2x$ et la courbe
$y=x^2$ pour $0\leq x \leq 4$.\\
\subQ{c} La région entre les courbes $y=e^x$ et
$\displaystyle y=\frac{e^{2x}}{2}$ pour $0\leq x \leq 1$.\\
\subQ{d} La région bornée par la droite $\displaystyle y = \frac{3}{2} + x$
et la courbe $\displaystyle y = \frac{x^2}{2}$.\\
\subQ{e} La région bornée par la droite $y=x$ et la courbe
$y=7x-3x^2$.\\
\subQ{f} La région bornée par la courbe $y^2 = x$ et la droite
$x -y = 2$.\\
\subQ{g} La région bornée par la courbe
$\displaystyle y = f(x)=\frac{x}{1+x}$ et la droite 
$\displaystyle y=g(x)=\frac{x}{2}$.\\
\subQ{h} La région bornée par les courbes
$\displaystyle y = (x-2)^2$ et $\displaystyle y= 10 - x^2$.
\label{8Q1}
\end{question}

\begin{question}[\eng \life]
Trouvez l'aire entre les courbes $y=f(x) = \sin(2x)$ et
$y=g(x)=\cos(2x)$ pour $0 \leq x \leq \pi$.
\label{8Q2}
\end{question}

\begin{question}[\life]
Le taux de croissance (instantané) d'une population de bactéries au temps
$t$ en minutes est donné par la formule
$\displaystyle r(t) = \frac{1000}{(2+3t)^{3/4}}$ bactéries/heure.

\subQ{a} Si la population initiale est de $10^6$ bactéries, quelle est le
nombre de bactéries après $T$ minutes?

\subQ{b} Est-ce que cette population de bactéries peut supporter ce taux de
croissance?  En d'autre mots, est-ce que le nombre de bactéries tend
vers une valeur finie lorsque $T$ tend vers plus l'infini?

\subQ{c} Si la population initiale est toujours de $10^6$ bactéries
comme en (a), combien de temps s'écoule-t-il avant que nous
atteignons $2\times 10^6$ bactéries?  Est-ce que ce résultat pourrait
justifier une réponse différente en (b)?
\label{8Q3}
\end{question}

\subsection{Valeur moyenne d'une fonction}

\begin{question}
Calculez la valeur moyenne de la fonction $f(x) = x-x^3$ pour
$-1 \leq x \leq 1$.  Tracez le graphe de $f$ sur l'intervalle $[-1,1]$ et la
droite horizontale qui représente la valeur moyenne.
\label{8Q4}
\end{question}

\begin{question}
Le taux (instantané) auquel l'eau est versé dans un bocal après $t$
minutes est donné par la formule $r(t) = 4t(3-t)$ litres/minute.

\subQ{a} Quelle est le volume d'eau qui a été versé dans le bocal au
cours des deux premières minutes (i.e. $0\leq t \leq 2$)?

\subQ{b} Quelle est le taux moyen auquel l'eau est versé dans le bocal
au cours des deux première minute?

\subQ{c} Comparez le taux moyen calculé en (b) avec le taux au
cours de la première minute $0 \leq t \leq 1$ minute.  Lequel est le plus
grand?

\subQ{d} Tracez le graphe du taux instantané en fonction du temps
(i.e. de $r$) et la droite horizontal qui correspond au taux moyen
durant les deux premières minutes.
\label{8Q5}
\end{question}

\begin{question}
Le taux (instantané) auquel l'eau entre un réservoir varie dans le temps et
est donné par la formule $r(t) = 360t -39t^2+t^3$ où $t$ est en heures et $r$
est en litres/heure.  Quelle est le volume d'eau qui a entré dans le
réservoir au cours des $15$ premières heures (i.e. pour $0\leq t \leq 15$)?
Quelle est le taux moyen au cours des $15$ premières heures?
\label{8Q6}
\end{question}

\begin{question}[\eng]
Si le taux d'énergie produite par une réaction (chimique) au temps
$t$ en heures est donnée par $E(t) = | 360t -39t^2+t^3|$ joules/heure,
Calculer l'énergie totale produite entre $t=0$ heure et $t= 24$
heures. Calculez le taux moyen de production d'énergie durant cette
période.\\
Suggestion: Les racines du polynôme $360t -39t^2+t^3$ sont
$0$, $15$ et $24$.
\label{8Q7}
\end{question}

\begin{question}[\life]
Un très mince fils de $2$ mètres de longueur est formé d'une substance
organique dont la densité linéaire à une distance de $x$ cm d'une de
ses extrémités est de $\rho(x)$ g/cm où
\[
\rho(x) = 1 + 2 \times 10^{-8}\, x^2(240-x) \ .
\]
\subQ{a} Trouvez la densité maximale et minimale le long du fils.  À quel
endroit avons-nous la plus forte densité?

\subQ{b} Quelle est la masse totale du fils?

\subQ{c} Quelle est la densité linéaire moyenne du fils?  Comparez avec la
valeur maximale et minimale que vous avez trouvée en (a).

\subQ{d} Dans une même figure, tracez le graphe de la densité et la droite
horizontale qui correspond à la densité moyenne.
\label{8Q8}
\end{question}

\subsection{Volume d'un objet}

\begin{question}[\eng]
Utilisez la méthode des tranches pour trouver le volume du cône droit
de hauteur $3$ dont la base est un disque de rayon $2$.
\label{8Q9}
\end{question}

\begin{question}[\eng]
Utilisez la méthode des tranches pour trouver {\em un somme de Reimann}
qui approche le volume d'un cône de hauteur $h$ dont la base est un
disque de rayon $r$.  Déduire de votre somme une formule exacte pour
calculer le volume d'un cône.
\label{8Q10}
\end{question}

\begin{question}[\eng]
Utilisez la méthode des tranches pour trouver le volume de la corne
représentée ci-dessous pour $0\leq x \leq 1$.  Les
sections transversales sont des cercles.
\PDFgraph{8_integrales_appl/corne}
\label{8Q11}
\end{question}

\begin{question}[\eng]
Utilisez la méthode des tranches pour trouver le volume d'un ballon de
football (ayant la forme d'un ellipsoïde) de $30$ cm de longueur et de
$18$ cm de diamètre dans sa partie la plus large.  Vous devez trouver
une somme de Riemann qui donne une approximation du volume du ballon
de football pour en déduire une intégrale définie pour le volume du
ballon de football.
\label{8Q12}
\end{question}

\begin{question}[\eng]
\subQ{a} Quel est le volume du solide dont la base est la région
$\displaystyle \{(x,y) | \, x^2 \leq y \leq 1\}$ et dont les sections
perpendiculaires à l'axe des $y$ sont des demi-cercles?

\subQ{b} Quel est le volume du solide dont la base est la région
$\displaystyle \{(x,y) | \, y^2 \leq x \leq 2\}$ et dont les sections
perpendiculaires à l'axe des $x$ sont des demi-cercles?

\subQ{c} Quel est le volume du solide dont la base est la région
bornée la courbe $y=-x^2$, et les droites $y=3x$ et $x=2$, et 
dont les sections perpendiculaires à l'axe des $x$ sont des carrés?
\label{8Q13}
\end{question}

\begin{question}[\eng]
Nous considérons la région $R$ bornée par l'axe des $x$, la courbe
$y = x^{1/3}$ et la droite $x = 1$.

\subQ{a} Trouvez le volume du solide produit en faisant la rotation de
la région $R$ autour de l'axe des $x$. \\
\subQ{b} Trouvez le volume du solide produit en faisant la rotation de la
région $R$ autour de la droite $y = -3$. \\
\subQ{c} Trouvez le volume du solide produit en faisant la rotation de la
région $R$ autour de la droite $y = 7$. \\
\subQ{d} Trouvez le volume du solide dont la base est la région $R$ et
les sections perpendiculaires à l'axe des $x$ sont des carrés. \\
\subQ{e} Trouvez le volume du solide dont la base est la région $R$ et
les sections perpendiculaires à l'axe des $x$ sont des demi-cercles.
\label{8Q14}
\end{question}

\begin{question}[\eng]
Pour chacun des problèmes suivants, trouvez le volume du solide
produit par la rotation de la région donnée autour de l'axe donné.

\subQ{a} La région est bornée par la courbe $y=\cos(x)$, l'axe des
$x$, et les droites $x=0$ et $x=\pi$.  L'axe de rotation est l'axe des
$x$.\\
\subQ{b} La région est bornée par la courbe $y=\cos(x/2)$, l'axe des
$x$, et les droites $x=-\pi$ et $x=\pi$.  L'axe de rotation est l'axe
des $x$.\\
\subQ{c} La région est bornée par la courbe $y=x^{3/2}$, l'axe des $x$
et la droite $x=1$.  L'axe de rotation est l'axe des $x$.\\
\subQ{d} La même région qu'en (c) mais l'axe de rotation est
l'axe des $y$.\\
\subQ{e} La région est bornée par la courbe $y=1/x$, l'axe des $x$, et
les droites $x=1$ et $x=2$.  L'axe de rotation est l'axe des $x$.
\label{8Q15}
\end{question}

\begin{question}[\eng]
Trouvez le volume du solide que nous obtenons dans chacun des cas
suivants.

\subQ{a} Si nous faisons la rotation de la région bornée par la courbe
$y=x^2$ et la droite $y=x$ autour de l'axe $y=-1$.\\
\subQ{b} Si nous faisons la rotation de la région bornée par la courbe
$y=x^2$ et la droite $y=2x$ autour de l'axe $x=4$.\\
\subQ{c} Si nous faisons la rotation de la région bornée par la courbe
$y=\sqrt{x}$ et la droite $y=x$ autour de l'axe des $x$.\\
\subQ{d} Si nous faisons la rotation de la région bornée par les courbes
$y=x^2$ et $x=y^2$ autour de l'axe des $y$?\\
\subQ{e} Si nous faisons la rotation de la région bornée par la courbe
$y= x^2 -6x+9$ et la droite $y=4$ autour de l'axe $x=-1$.\\
\subQ{f} Si nous faisons la rotation de la région bornée par la courbe
$y= x^2 -8x+16$ et la droite $y=4$ autour de l'axe $x=-2$.\\
\subQ{g} Si nous faisons la rotation de la région bornée par la courbe
$y= -x^2 +10x-25$ et la droite $y=-4$ autour de l'axe $x=-1$.\\
\subQ{h} Si nous faisons la rotation de la région bornée par les courbes
$x = y^2-4y+2$ et $x = -y^2 +4y-4$ autour de la droite $x = 1$.\\
\subQ{i} Si nous faisons la rotation de la région bornée par la courbe
$y=2/x$ et la droite $y=3-x$ autour de l'axe des $x$.
\label{8Q16}
\end{question}

\begin{question}[\eng]
Si nous faisons la rotation du cercle $x^2+y^2=1$ autour de l'axe $y=3$, nous
obtenons un tore (une objet de la forme d'un bagel).  Quelle est le
volume de ce tore?
\label{8Q17}
\end{question}

\subsection{Masse d'un objet}

\begin{question}[\eng]
La densité de l'huile à une distance de $r$ mètres du centre d'une
nappe d'huile circulaire sur la surface de l'océan est donnée par
$\rho(r) = 50/(1+r)$ kg/m$^2$.\\
\subQ{a} Si la nappe d'huile a un rayon de 1,000 m, donnez une somme
de Reimann qui approche la masse total de l'huile dans la nappe. \\
\subQ{b} En transformant la somme donnée en (a) en une intégrale,
trouvez la valeur exacte de la masse totale d'huile dans la nappe.\\
\subQ{c} Quelle est la valeur de $r$ pour laquelle nous avons la moitié de
la masse totale d'huile.
\label{8Q18}
\end{question}

\subsection{Travail}

\begin{question}[\eng]
Un travailleur est en haut d'un échafaudage de $30$ mètres de
hauteur.  Il doit lever à l'aide d'une corde une chaudière de
ciment à partir du sol jusqu'à une hauteur de 10 mètres.  La
chaudière a une masse de $50$ kg et la corde a une masse de $0.5$
kg/m.  Trouvez le travail total nécessaire pour lever cette
chaudière.
\label{8Q19}
\end{question}

\begin{question}
Le réservoir représenté ci-dessous est rempli d'eau.
\PDFgraph{8_integrales_appl/reservoir4}
Calculez le travail nécessaire pour vider ce réservoir s'il est enfoui
$2$ m sous le niveau du sol.  La densité de l'eau est de $1000$
kg/m$^3$ et l'accélération dû à la gravité est $g=9.8$ m/s$^2$.
\label{8Q20}
\end{question}

\begin{question}[\eng]
Une piscine représenté ci-dessous est remplie d'eau.  Calculez
le travail nécessaire pour vider cette piscine.  La densité de l'eau
est de $1000$ kg/m$^3$ et l'accélération dû à la gravité est $g=9.8$
m/s$^2$.

Vous devez trouver une somme de Riemann qui donne une approximation du
travail fait pour vider le réservoir et en déduire une intégrale
définie pour calculer le travail fait pour vider le
réservoir. Finalement vous devez calculer cette intégrale. 
Vous devez bien définir toutes vos variables.
\PDFgraph{8_integrales_appl/piscineTravail}
\label{8Q21}
\end{question}

\begin{question}[\eng]
Une citerne de forme rectangulaire possède les dimensions suivantes:
$3$ m de long, $1$ m de large et $2$ m de haut.  La citerne est plein
d'eau.  Calculez le travail nécessaire pour pomper la moitié de l'eau
à une hauteur de $1$ m au-dessus du citerne.  La densité de l'eau est
de $1000$ kg/m$^3$ et l'accélération dû à la gravité est $g=9.8$
m/s$^2$.
\label{8Q22}
\end{question}

\begin{question}[\eng]
Un réservoir à la forme d'une demi-sphère renversée dont le rayon est
de $3$ m.
\PDFgraph{8_integrales_appl/reservoir3}
Si ce réservoir est à $2$ m sous le sol et il est plein d'eau,
calculez le travail nécessaire pour vider ce réservoir.  La densité de
l'eau est de $1000$ kg/m$^3$ et l'accélération dû à la gravité est
$g=9.8$ m/s$^2$.

Vous devez trouver une somme de Riemann qui donne une approximation du
travail fait pour vider le réservoir et en déduire une intégrale
définie pour calculer le travail fait pour vider le
réservoir. Finalement vous devez calculer cette intégrale. 
Vous devez bien définir toutes vos variables.
\label{8Q23}
\end{question}

\begin{question}[\eng]
Les réservoirs ci-dessous sont plein d'huile dont la densité est de
570 kg/m$^3$.  Quelle est le travail nécessaire pour vider chacun des
réservoirs? 

Vous devez trouver une somme de Riemann qui donne une approximation du
travail fait pour vider le réservoir et en déduire une intégrale
définie pour calculer le travail fait pour vider le
réservoir. Finalement vous devez calculer cette intégrale. 

\subQ{a} le réservoir est enfoui 2 mètres sous le sol.
\PDFgraph{8_integrales_appl/huile1}

\subQ{a} le réservoir est enfoui 4 mètres sous le sol.
\PDFgraph{8_integrales_appl/huile2}
\label{8Q24}
\end{question}

\begin{question}[\eng]
Un réservoir cylindrique de deux mètres de rayon et de six mètres de
hauteur (donc l'axe du cylindre est vertical) est à moitié rempli
d'eau.  Trouvez le travail nécessaire pour vider le réservoir si nous
pompons l'eau a partir d'un point trois mètres au dessus réservoir.  La
densité de l'eau est de $1000$ kg/m$^3$ et l'accélération dû à la
gravité est $g=9.8$ m/s$^2$.
\label{8Q25}
\end{question}

\subsection{Force}

\begin{question}[\eng]
Un réservoir rectangulaire de 20 mètres de longueur, 10
mètres de largeur et 15 mètres de profondeur est rempli d'eau.  Quelle
est la force totale exercée sur le font et sur chaque coté du
réservoir?  La densité de l'eau est de $1000$ kg/m$^3$ et
l'accélération dû à la gravité est $g=9.8$ m/s$^2$.
\label{8Q26}
\end{question}

\begin{question}[\eng]
Un barrage à une porte rectangulaire à sa base.  La porte a une largeur
de $2$ m et une hauteur de $5$ m.  Quelle est la pression
hydrostatique (i.e.\ la force) exercée sur la porte si la hauteur de
l'eau derrière le barrage est de $17$ m.  La densité de l'eau est de
$1000$ kg/m$^3$ et l'accélération dû à la gravité est $g=9.8$
m/s$^2$.
\PDFgraph{8_integrales_appl/dam1}
Vous devez trouver une somme de Riemann qui donne une approximation de
la pression sur la porte et en déduire une intégrale définie pour
calculer la pression sur la porte. Finalement vous devez calculer
cette intégrale. Vous devez bien définir toutes vos variables.
\label{8Q27}
\end{question}

\begin{question}[\eng]
Une piscine représenté ci-dessous est remplie d'eau.  Calculez
la force exercée par la pression sur les deux côtés suivants de la
piscine.

Dans chacun des cas, vous devez trouver une somme de Riemann qui donne
une approximation de la force exercée sur le cotė et en déduire une
intégrale définie pour la force exercée sur le coté.  Finalement, vous
devez évaluer cette intégrale.  Prenez soin de bien définir toutes vos
variables.

\subQ{a} Pour le coté A de la piscine

\subQ{b} Pour le coté B de la piscine.
\PDFgraph{8_integrales_appl/piscinePression}
\label{8Q28}
\end{question}

\begin{question}[\eng]
Un cylindre fermé aux extrémités se retrouve sur le côté au font d'un
lac.  Le diamètre du cylindre est de $10$ cm et sa longueur de $20$
cm.  Si la profondeur du lac est de $6$ mètres, quelle est la force
exercée sur les extrémités du cylindre?  Il faut se rappeler que la
densité de l'eau est de $1000$ kg/m$^3$.

Vous devez trouver une somme de Riemann qui donne une approximation de
la force exercées sur les extrémités et en déduire une intégrale définie pour
calculer la force exercées sur les extrémités. Finalement vous devez calculer
cette intégrale. Vous devez bien définir toutes vos variables.
\label{8Q29}
\end{question}

\subsection{Centre de masse}

\begin{question}[\eng]
Trouvez la position du centre de masse de la région plane bornée par la
courbe $y=x^2$ et les droites $y=0$, $x=0$ et $x=2$.  La densité est
constante. 
\label{8Q30}
\end{question}

\begin{question}[\eng]
Trouvez la position du centre de masse de la région bornée par les
courbes $y=e^x$, $y=0$, $x=0$ et $x=2$.   La densité est constante et
égale à $3$ kg/m$^2$.
\label{8Q31}
\end{question}

\subsection{Applications à l'économie}

\begin{question}[\eco]
Si \$$5000$ est investit à un taux de $7$\% d'intérêt, quelle est la valeur
de l'investissement après $2$ ans si l'intérêt est calculé mensuellement?  De
façon continue?
\label{8Q32}
\end{question}

\begin{question}[\eco]
Un placement garantie un revenu de $100+10t$ dollars par année pour
une période de dix ans où $t$ est le nombre d'années depuis la
date du placement que nous assumons comme étant aujourd'hui.  Trouvez
la valeur présente de ce placement à la fin des dix ans si le taux
d'intérêt est de $5$\% par année, composé de façon continue.
\label{8Q33}
\end{question}

\begin{question}[\eco]
Nous savons qu'un bon vin prend de la valeur avec l'âge.  Supposer que
vous êtes un marchant de vin et que vous savez que le prix d'une
bouteille de votre vin sera de $P(1+20\sqrt{t})$ dollars dans $t$
années où $P$ dollars est le prix présent d'une bouteille de
votre vin.  Si le taux d'intérêt est de $5$\% par année, composé de
façon continue, qu'elle est le meilleur temps pour vendre votre vin?
\label{8Q34}
\end{question}

\begin{question}[\eco]
On vous doit une certaine somme d'argent.  On vous offre deux options
pour vous rembourser.  Pour la première option, on vous offre de payer
cette dette en quatre versements de \$5,000 chacun sur une période de
trois ans en commençant maintenant avec le premier versement de
\$5,000.  Pour la deuxième option, on vous offre de payer la dette en
un seul versement de \$23,000 dans exactement trois ans à partir de
maintenant.  Si on suppose que le taux d'intérêt est de 6\% par
année, composé de façon continue, quelle option de remboursement est le
plus profitable?
\label{8Q35}
\end{question}

\subsection{Test de l'intégrale}

\begin{question}[\eng]
Utilisez le test de l'intégrale pour déterminer si les séries
suivantes converge ou diverge.  Bien justifier que vous pouvez
utiliser ce test.
\begin{center}
\begin{tabular}{*{1}{l@{\hspace{0.5em}}l@{\hspace{6em}}}l@{\hspace{0.5em}}l}
\subQ{a} & $\displaystyle \sum_{n=1}^\infty n e^{-2n^2}$ &
\subQ{b} & $\displaystyle \sum_{n=3}^\infty \frac{1}{n(\ln(n))^2}$
\end{tabular}
\end{center}
\label{8Q36}
\end{question}

\begin{question}[\eng]
Utilisez le test de l'intégrale pour déterminer les valeurs de $k$
pour lesquelles la série
$\displaystyle \sum_{n=3}^{\infty} \frac{\ln n}{n^k}$ converge.
\label{8Q37}
\end{question}

\begin{question}
Démontrez que la séries $\displaystyle \sum_{n=1}^\infty \frac{1}{n^{8/3}}$
converge et trouvez une petite valeur $N$ pour que la somme partielle
$\displaystyle S_N = \sum_{n=1}^N \frac{1}{n^{8/3}}$ soit une
approximation de la valeur de la série
$\displaystyle S = \sum_{n=1}^\infty \frac{1}{n^{8/3}}$
avec une erreur inférieure à $10^{-3}$.
\label{8Q38}
\end{question}

%%% Local Variables: 
%%% mode: latex
%%% TeX-master: "notes"
%%% End:

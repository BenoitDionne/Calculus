\chapter[Dérivée]{Dérivée}

\compileTHEO{

Nous définissons dans ce chapitre la {\em dérivée d'une fonction}.  Nous
présentons quelques unes des propriétés importantes de la dérivée d'une
fonction ainsi que les principales techniques pour calculer
{\em facilement} la dérivée d'une fonction.  Quelques applications de
la dérivée seront fournies au prochain chapitre.

\section{Étude du graphe et comportement d'une fonction}

Avant de définir ce qu'est la {\em dérivée} d'une fonction, un des
piliers du {\em calcul différentiel et intégral}, et de plonger dans
l'étude de la dérivée d'une fonction, nous énonçons quelques
propriétés des fonctions que la dérivée nous permettra de déterminer.

La dérivée est un outil pour l'étude des fonctions.  Lorsque nous
parlons de l'étude des fonctions, nous parlons de l'étude du graphe et du
comportement des fonctions sur leur domaine.  Nous cherchons les
caractéristiques marquantes des fonctions.  Quelques unes des
caractéristiques qu'une fonction peut avoir sont énoncées dans les
trois prochaines définitions.

\begin{defn}
Soit $f:]a,b[ \rightarrow \RR$ une fonction.
\begin{enumerate}
\item La fonction $f$ est {\bfseries strictement croissante}
\index{Fonction!Strictement croissante}
si $f(x) < f(y)$ pour tout $x$ et $y$ dans l'intervalle $]a,b[$
tels que $x<y$.
\item La fonction $f$ est {\bfseries croissante}\index{Fonction!Croissante}
si $f(x) \leq f(y)$ pour tout $x$ et $y$ dans l'intervalle $]a,b[$
tels que $x<y$.
\item La fonction $f$ est {\bfseries strictement décroissante}
\index{Fonction!Décroissante} si $f(x) > f(y)$ pour tout $x$ et $y$
 dans l'intervalle $]a,b[$ tels que $x<y$.
\item La fonction $f$ est {\bfseries décroissante}
\index{Fonction!Décroissante} si $f(x) \geq f(y)$ pour tout $x$ et $y$
 dans l'intervalle $]a,b[$ tels que $x<y$.
\end{enumerate}
\end{defn}

\begin{defn}
Soit $f:\RR \rightarrow \RR$ une fonction.
\begin{enumerate}
\item La fonction $f$ a un {\bfseries maximum local}
\index{Fonction!Maximum local} au point $c$ si $f(x) < f(c)$ pour tout
$x \neq c$ suffisamment près de $c$.
\item La fonction $f$ a un {\bfseries minimum local}
\index{Fonction!Minimum local} au point $c$ si $f(x) > f(c)$ pour tout
$x \neq c$ suffisamment près de $c$.
\end{enumerate}
\end{defn}

\begin{defn}
Soit $f:[a,b] \rightarrow \RR$ une fonction.
\begin{enumerate}
\item La fonction $f$ a un {\bfseries maximum global ou absolu}
\index{Fonction!Maximum global} \index{Fonction!Maximum absolu}
s'il existe un point $c \in [a,b]$ tel que $f(c) > f(x)$
pour tout $x\neq c$ dans $[a,b]$.  La valeur $f(c)$ est le maximum
global de $f$ sur $[a,b]$.
\item La fonction $f$ a un {\bfseries minimum global ou absolu}
\index{Fonction!Minimum global} \index{Fonction!Minimum absolu}
s'il existe un point $c \in [a,b]$ tel que $f(c) < f(x)$
pour tout $x\neq c$ dans $[a,b]$.  La valeur $f(c)$ est le minimum
global de $f$ sur $[a,b]$.
\end{enumerate}
\label{maxminabs}
\end{defn}

\begin{egg}
Considérons la fonction continue dont le graphe est donné à la
figure~\ref{DER1}.  Quelques unes des caractéristiques de cette
fonction sont:
\begin{enumerate}
\item La fonction $f$ a un maximum absolu au point $c$ car $f(x)<f(c)$
pour tout $x\neq c$.  La valeur $f(c)$ est le maximum absolu de $f$.
Cela implique aussi que $f$ a un maximum local au point $c$.
\item La fonction $f$ a un minimum local au point $e$ car $f(x)>f(e)$
pour tout $x$ près de $e$, $x\neq e$.  Ce n'est pas un minimum absolu
car il y a des valeurs de $x$ pour lesquelles $f(x) < f(e)$; par
exemple, $f(a) < f(e)$.
\item La fonction $f$ est strictement croissante pour $x<c$ car
$f(x_1) < f(x_2)$ pour tous $x_1 < x_2 < c$.  De même, la fonction $f$
est strictement croissante pour $x>e$.  Par contre, la fonction $f$
est strictement décroissante pour $x$ entre $c$ et $e$ car
$f(x_1) > f(x_2)$ pour tout $a < x_1 < x_2 < c$.
\item $f(x)$ approche la valeur $Y$ lorsque $x$ devient de
plus en plus grand.  La droite $y=Y$ est une asymptote horizontale
pour $f$ lorsque $x$ tend vers plus l'infini.
\end{enumerate}
\label{caractF}
\end{egg}

\PDFfig{5_derivees/der1}{Caractéristiques du graphe d'une fonction}{À
l'exemple~\ref{caractF}, nous déterminons certaines caractéristiques de la
fonction qui possède ce graphe.}{DER1}

Nous verrons plus tard d'autres caractéristiques que les fonctions peuvent
avoir.

Si nous observons minutieusement l'exemple précédent, nous remarquons que la
fonction $f$ est strictement croissante au point $x$ si la pente de la
{\em droite tangente} au graphe de la fonction $f$ au point $(x,f(x))$
est positive, la fonction $f$ est strictement décroissante au point
$x$ si la pente de la {\em droite tangente} au graphe de la fonction
$f$ au point $(x,f(x))$ est négative, la pente de la
{\em droite tangente} au graphe de la fonction $f$ au point $(x,f(x)$
est zéro lorsque la fonction $f$ a un maximum local au point $x$, et
ainsi de suite.  La pente des {\em droites tangentes} au graphe
d'une fonction semble donc déterminer les caractéristiques importantes
d'une fonction.  Il serait donc important de pouvoir facilement
calculer la pente des tangentes au graphe d'une fonction. La dérivée
est l'outil qui nous permettra de facilement calculer la pente des
tangentes au graphe d'une fonction. Ainsi, elle nous permettra de
trouver les minimums et maximums locaux d'une fonction, les
intervalles de croissance et décroissance d'une fonction, et encore
plus.

Naturellement, il faut définir ce qu'est une {\em droite tangente} au
graphe d'une fonction en un point du graphe.  En fait, cela n'est pas
aussi évident que nous pourrions l'imaginer comme nous allons voir
prochainement.  Nous allons voir que définir ce qu'est la droite tangente au
graphe d'une fonction en un point du graphe revient à définir la
dérivée de la fonction en ce point.

\section{Taux de variation d'une fonction}\label{sect_TVI}

\begin{defn} \index{Taux de variation moyen}
Si $f$ est une fonction dont le domaine inclut l'intervalle $[a,b]$, le
{\bfseries taux de variation moyen} de $f$ entre $a$ et $b$ est
$\displaystyle \frac{f(b)-f(a)}{b-a}$
\end{defn}

\begin{egg}
Suite à une expérience de laboratoire sur une culture de bactéries
dans un milieu qui ne leur est pas favorable, nous observons que la
fonction $\displaystyle g(t) = 50 \; e^{-2t}$ représente le
nombre de bactéries par cm$^3$ au temps $t$ en heures.

Cette formule demande quelques clarifications.  Remarquons que $g$
peut retourner des valeurs réelles qui ne sont pas des entiers.  Cela
ne semble pas \lgm rationnel\rgm\ car nous ne pouvons pas avoir une
fraction de bactérie (e.g. nous ne pouvons pas avoir un dixième de
bactérie).  Il faut comprendre que $g$ retourne un nombre moyen de
bactéries par cm$^3$ dans un contenant qui pourrait avoir $10$ cm$^3$
par exemple.  De plus, dans nos exemples, nous utiliserons des petites
valeurs pour le nombre de bactéries.  Ce qui ne correspond pas à la
réalité.  Il faudrait généralement multiplier le nombre de bactéries
par un facteur de $10^9$ pour être plus prêt de la réalité.
Cependant, dans le but de ne pas alourdir le texte, nous avons choisi
d'utiliser de petites valeurs.  Malgré cela, le graphe de $g$
(figure~\ref{BACT1}) est une très bonne représentation du nombre
de bactéries par cm$^3$ en fonction du temps $t$ en heures.

Le taux de variation moyen du nombre de bactéries entre $t=0$ et
$t=10$ est
\[
\frac{g(10)-g(0)}{10 -0}
= \frac{50 \;e^{-20} - 50}{10 -0}
\approx -5.0000 
\]
bactéries par cm$^3$ par heure.  C'est-à-dire que le nombre de
bactéries décroît en moyenne de $5$ bactéries par cm$^3$ par heure
pendant les dix premières heures.  Nous disons que la population de
bactéries a un {\em taux de croissance moyen} de $-5$ bactéries par
cm$^3$ par heure.  Nous aurions pu parler de décroissance mais la
tradition veut que nous parlions de croissance négative.

Si nous regardons le graphe de $g$, nous voyons que la population de
bactéries décroît plus rapidement dans les premières heures.  Par
exemple, le taux de croissance moyen de la population de bactéries
entre $t=0$ et $t=3$ est
\[
\frac{g(3)-g(0)}{3 -0}
= \frac{50 \;e^{-6} - 50}{3 -0}
\approx -16.625
\]
bactéries par cm$^3$ par heure.  Le nombre de bactéries décroît
donc en moyenne de $16.625$ bactéries par cm$^3$ par heure pendant les
trois premières heures.

La population de bactéries a théoriquement disparu après $10$ heures
car il reste
\[
g(10) = 50 \;e^{-20} \approx  1.0306 \times 10^{-7}
\]
bactéries par cm$^3$ après $10$ heures.
\label{bacter0}
\end{egg}

\MATHfig{5_derivees/bact1}{8cm}{Graphe de $x= 50 \; e^{-2t}$}
{Graphe de $x= 50 \; e^{-2t}$ pour $ 0 \leq t \leq 5$ heures.}
{BACT1}

Qu'arrivera-t-il si nous calculons les taux de variation moyens sur des
intervalles $[a,b]$ de plus en plus petit; c'est-à-dire, pour
lesquelles $b$ tend vers $a$?

\begin{egg}
Revenons à l'exemple précédent au sujet d'une population de bactéries.
Si nous calculons le taux de croissance moyen (i.e. le taux de variation
moyen) sur des intervalles $[a,b]$ où $a=1$ et $b$ est de plus en plus
près de $a$, nous obtenons les données suivantes.
\begin{center}
\begin{tabular}{l|l|l}
\hline
$a$ heures & $b$ heures & \rule{0em}{1.3em}
$\frac{g(b)-g(a)}{b-a}$ bactéries par cm$^3$ par heure \\[1ex]
\hline
$1$ & $2$ & $-5.8509822\ldots $ \\
$1$ & $1.5$ & $-8.5548214\ldots$ \\
$1$ & $1.1$ & $-12.2660624\ldots$ \\
$1$ & $1.01$ & $-13.3990907\ldots$ \\
$1$ & $1.001$ & $-13.5200038\ldots$ \\
$1$ & $1.0001$ & $-13.5321750\ldots$ \\
$1$ & $1.00001$ & $-13.5333929\ldots$ \\
$1$ & $1.000001$ & $-13.5335147\ldots$ \\
$1$ & $1.0000001$ & $-13.5335269\ldots$ \\
\hline
\end{tabular}
\end{center}
Comme le taux de croissance moyen n'a pas vraiment le temps de changer
entre $t=1$ et $t=1.0000001$ heure, nous pouvons dire que le taux de
croissance à $t=1$ heure est approximativement $-13.5335269$
bactéries par cm$^3$ par heure.  Nous pourrions prendre des intervalles
$[a,b]$ où $a=1$ et $b$ est encore plus près de $1$ que $1.0000001$.
Si nous faisons cela, nous trouvons que le taux de croissance moyen
(i.e. le taux de variation moyen) approche $-13.53352832\ldots$
bactéries par cm$^3$ par heure.  Cette dernière valeur est le
{\bfseries taux de variation instantané} du nombre de bactéries à
$t=1$ heure.  Nous définissons le {\bfseries taux de croissance} de la
population de bactéries à $t=1$ heure comme étant ce taux de variation
instantané.

Nous aurions pu calculer des taux de variation moyens pour des
intervalles de la forme $[b,a]$, où $b<a$ tend vers $a$.  Nous aurions
trouvé le même taux de variation instantané.
\label{bacter1}
\end{egg}

\begin{defn} \index{Taux de variation instantané}
Si $f$ est une fonction définie sur un intervalle ouvert contenant
$a \in \RR$, le {\bfseries taux de variation instantané} de $f$ au
point $a$ est la valeur $M$ unique (si une telle valeur existe) telle
que le taux de variation moyen $\displaystyle \frac{f(b)-f(a)}{b-a}$
tend vers $M$ lorsque $b$ tend vers $a$ avec $b$ plus petit ou
plus grand que $a$.
\end{defn}

Remarquons que le taux de variation moyen
\[
\frac{f(b)-f(a)}{b-a}
\]
est aussi la pente de la droite qui passe par les points $(a,f(a))$ et
$(b,f(b))$.  Une droite qui passe par au moins deux points d'une
courbe est une {\bfseries sécante}. \index{Droite sécante}

\begin{egg}
Pour la population de bactéries étudiée précédemment, nous avons tracé à la
figure~\ref{BACT2} les sécantes qui passent par les points suivants.
\begin{enumerate}
\item $(1,50 \; e^{-2})$ et $(4,50 \; e^{-8})$
donnent une droite de pente
\[
\frac{g(4)-g(1)}{4-1} = \frac{50 \; e^{-8}- 50 \; e^{-2}}
{4-1} = -2.24999701\ldots
\]
\item $(1,50 \; e^{-2})$ et $(3,50 \; e^{-6})$
donnent une droite de pente
\[
\frac{g(3)-g(1)}{3-1} = \frac{50 \; e^{-6}- 50 \; e^{-2}}
{3-1} = -3.321413276\ldots
\]
\item $(1,50 \; e^{-2})$ et $(2,50 \; e^{-4})$
donnent une droite de pente
\[
\frac{g(2)-g(1)}{2-1} = \frac{50 \; e^{-4}- 50 \; e^{-2}}
{2-1} = -5.850982217\ldots
\]
\end{enumerate}
\end{egg}

\MATHfig{5_derivees/bact2}{8cm}{Quelques sécantes du graphe de
$x= 50 \; e^{-2t}$}{Quelques sécantes du graphe de $x= 50 \; e^{-2t}$.}{BACT2}

Que devient la sécante qui passe par $(a,f(a))$ et $(b,f(b))$ si $b$
est de plus en plus près de $a$?  L'exemple suivant va fournir une
réponse à cette question.

\begin{egg}
Toujours pour la population de bactéries de l'exemple~\ref{bacter0},
nous avons tracé à la figure~\ref{BACT3} les sécantes qui passent par les
points suivants.
\begin{enumerate}
\item $(1,50 \; e^{-2})$ et $(1.1,50 \; e^{-2.2})$
donnent une droite de pente
\[
\frac{g(1.1)-g(1)}{1.1-1} =
\frac{50 \; e^{-2.2}- 50 \; e^{-2}}{1.1-1}
= -12.2660624\ldots
\]
\item $(1,50\; e^{-2})$ et $(1.001,51 \; e^{-2.002})$
donnent une droite de pente
\[
\frac{g(1.001)-g(1)}{1.001-1} =
\frac{50 \; e^{-2.002}- 50 \; e^{-2}}{1.001-1}
= -13.5200038\ldots
\]
\end{enumerate}
Remarquons que, pour des valeurs de $b$ très près de $a=1$, il
devient de plus en plus difficile de différencier le graphe de la
fonction $g$ dans le voisinage de $x=1$ de la sécante qui passe par
les points $(a,g(a))=(1,g(1))$ et $(b,g(b))$.  À la limite, ces
sécantes approchent une droite que nous ne pouvons dissocier de la
courbe $x=g(t)$ pour $t$ assez près de $t=1$.  Cette droite est
appelée la {\em droite tangente} à la courbe au point $(1,g(1))$.  La
pente de cette droite est la valeur limite des pentes des sécantes qui
passent par $(a,g(a))$ et $(b,g(b))$ lorsque que $b$ tend vers $a=1$.
Dans le cas présent, cette pente est $-13.53352832\ldots$ que nous
avons trouvée à l'exemple~\ref{bacter1}.
\end{egg}

\MATHfig{5_derivees/bact3}{8cm}{Limite de sécantes pour le graphe de
$x= 50 \; e^{-2t}$}{La sécante qui passe par $(1, 50 e^{-2})$ et 
$(1.001, 50 e^{-2.002})$ se confond à la courbe $x= 50 \; e^{-2t}$
pour $t$ très près de $t=1$ heure.}{BACT3}

\begin{defn} \index{Droite tangent}
Si $f$ est une fonction définie sur un intervalle ouvert qui contient
$a \in \RR$.  La droite
\[
y = h(x) = f(a) + M(x-a) \; ,
\]
où $M$ est le taux de variation instantané de $f$ à $x=a$, est la
{\bfseries droite tangente} à la courbe
$y=f(x)$ au point $(a,f(a))$.

Nous avons que la distance entre $f(x)$ et $h(x)$ devient de plus en plus
petite lorsque $x$ approche $a$.
\end{defn}

Au moment de donner la définition de la dérivée d'une fonction en un
point, nous donnerons un sens mathématique à \lgm la distance entre $f(x)$
et $h(x)$ devient de plus en plus petite lorsque $x$ approche $a$\rgm.

En résumé, lorsque $b$ approche $a$, la sécante qui passe par
$(a,f(a))$ et $(b,f(b))$ approche une droite qui passe par $(a,f(a))$
et qui est indissociable (presque superposée) à la courbe $y=f(x)$
pour $x$ assez près de $x=a$.  Cette droite limite est la droite
tangente que nous venons de définir.

\section{Dérivée d'une fonction en un point}\label{def_der}

La {\bfseries dérivée} d'une fonction $f$ en un point n'est
rien d'autre que le taux de variation instantané de $f$ à ce point.

Avec la définition de limite d'une fonction en un point, nous pouvons
donner un sens mathéma\-tique au taux de variation instantané de $f$
au point $a$.  À la section~\ref{sect_TVI}, nous avons défini le taux de
variation instantané de $f$ au point $a$ comme étant le nombre $M$ tel
que le taux de variation moyen $\displaystyle \frac{f(b)-f(a)}{b-a}$
approche $M$ lorsque $b$ approche $a$ où $b$ peut être plus petit ou
plus grand que $a$.  En termes mathématiques, l'énoncé précédent est
simplement
\[
M = \lim_{b\rightarrow a} \frac{f(b)-f(a)}{b-a} \; .
\]
Nous obtenons la définition suivante.

\begin{defn} \index{Dérivée d'une fonction à un point}
Soit $f$ une fonction définie dans un voisinage de $a$ incluant le
point $a$.  La {\bfseries dérivée de $f$ au point $a$}, dénotée
$f'(a)$ ou $\displaystyle \dydx{f}{x}(a)$, est
\[
f'(a) \equiv \lim_{x\rightarrow a} \frac{f(x) - f(a)}{x-a}
\]
si cette limite existe.

La limite précédente est équivalente à
\[
f'(a) \equiv \lim_{h\rightarrow 0} \frac{f(a+h) - f(a)}{h}
\]
si cette limite existe.  Il suffit de substituer $x=a+h$ dans la
première limite pour obtenir la deuxième et de substituer $h=x-a$ dans
la deuxième limite pour obtenir la première.  Nous avons que $x$
approche $a$ si et seulement si $h=x-a$ approche $0$.  Notez que $h$
peut être plus petit ou plus grand que $0$.
\end{defn}

\begin{egg}
À l'exemple~\ref{bacter1}, nous avons montré numériquement que
\[
g'(1) = -13.53352832\ldots \quad \text{bactéries par cm$^3$/heure}
\]
où $g(t) = 50 e^{-2t}$.
\end{egg}

Pour l'exemple qui suit, nous aurons besoin de la définition suivante.

\begin{defn} \index{Taux de croissance relatif}
Le {\bfseries taux de croissance relatif} d'une population au temps
$t$ est
\[
\frac{\text{taux de croissance instantané (au temps $t$)}}
{\text{nombre d'individus dans la population (au temps $t$)}}
\ .
\]
En d'autres mots, si $p(t)$ est le nombre d'individus au temps $t$, le
taux de croissance relatif au temps $t$ est
$\displaystyle \frac{p'(t)}{p(t)}$.
\end{defn}

\begin{egg}
Estimons le taux de croissance relatif à $t=1$ et $t=2$ d'une
population dont le nombre d'individus au temps $t$ (en heures) est
donnée par $p(t) = 2^t$.

Commençons par estimer $p'(1)$, le taux de croissance instantané à
$t=1$.  Nous avons
\begin{align*}
\frac{p(1.01)-p(1)}{0.01} &= \frac{2^{1.01} - 2}{0.01} \approx 1.3911100,
\quad \frac{p(1.001)-p(1)}{0.001} = \frac{2^{1.001} - 2}{0.001} \approx
1.386774925, \\
\frac{p(1.0001)-p(1)}{0.0001} &= \frac{2^{1.0001} - 2}{0.0001} \approx
1.386342407529995, \quad \ldots 
\to 1.3862943611\ldots
\end{align*}
Donc $p'(1)=1.3862943611\ldots$

De même, nous pouvons estimer $p'(2)$, le taux de croissance instantané à
$t=2$.  Nous avons
\begin{align*}
\frac{p(2.01)-p(2)}{0.01} &= \frac{2^{2.01} - 4}{0.01} \approx 2.7822200,
\quad \frac{p(2.001)-p(2)}{0.001} = \frac{2^{2.001} - 4}{0.001} \approx
2.773549850, \\
\frac{p(2.0001)-p(2)}{0.0001} &= \frac{2^{2.0001} - 4}{0.0001} \approx
2.7726848151 , \quad \ldots 
\to 2.772588722\ldots
\end{align*}
Donc $p'(2) = 2.772588722\ldots$.

Le taux de croissance relatif à $t=1$ est
\[
\frac{p'(1)}{p(1)} = \frac{1.3862943611\ldots}{2}
= 0.693147180559945\ldots
\]
et le taux de croissance relatif à $t=2$ est
\[
\frac{p'(2)}{p(2)} = \frac{2.772588722\ldots}{4}
= 0.693147180559945\ldots
\]
Est-ce que $\displaystyle \frac{p'(t)}{p(t)} = 0.693147180559945\ldots$
pour tout $t$?  Nous verrons prochainement que la réponse est affirmative.
\end{egg}

Si
\[
f'(a) = \lim_{x\rightarrow a} \frac{f(x) - f(a)}{x-a} \ ,
\]
alors
\[
f'(a) \approx \frac{f(x) - f(a)}{x-a}
\]
pour $x$ près de $a$.  Si nous résolvons cette relation pour $f(x)$,
nous trouvons
\begin{equation} \label{faedt}
f(x) \approx f(a) + f'(a) (x-a)
\end{equation}
Le côté droit de cette relation n'est nul autre que l'équation de la
droite tangente à la courbe $y=f(x)$ au point $(a,f(a))$ définie à la
section~\ref{sect_TVI} car $f'(a)$ est le taux de variation
instantané.  Donc $f'(a)$ est la pente de la droite tangente à la
courbe $y=f(x)$ au point $(a,f(a))$.

Nous retrouvons à la figure~\ref{DER6} un résumé de l'interprétation
graphique de la dérivée.

\PDFfig{5_derivees/der6}{Interprétation graphique de la dérivée d'une
fonction en un point}{Interprétation graphique de la dérivée d'une
fonction $f$ au point $x=a$}{DER6}

\begin{prop} \label{der_approx_ft}
Si $f$ est une fonction différentiable au point $a$, alors
\[
f(x) \approx f(a) + f'(a) (x-a)
\]
pour $x$ très près de $a$.
\end{prop}

L'équation de la droite tangente est une formule simple pour estimer
la fonction $f$ au voisinage du point $x=a$ (figure~\ref{DER19}).
L'approximation de fonctions compliquées à l'aide
de droites tangentes ou de polynômes sera abordé dans une prochaine
section.

\begin{rmk}[\theory]
En Analyse Mathématique, nous définissons la dérivée d'une fonction $f$ au
point $a$ de la façon suivante.

\begin{defn}
Soit $f$ une fonction continue de $\RR$ dans $\RR$.  La dérivée de la
fonction $f$ au point $a$ est le nombre $f'(a)$ qui satisfait
\begin{equation} \label{RigDfn}
\lim_{x\rightarrow a} \frac{|f(x) - f(a) - f'(a)(x-a)|}{|x-a|} = 0 \; .
\end{equation}
\end{defn}

Cette définition de la dérivée est équivalent à la définition de la
dérivée que nous avons donnée.  En effet, puisque
\[
\lim_{x\rightarrow a} \left| \frac{f(x) - f(a) - f'(a)(x-a)}{x-a} \right|
= \lim_{x\rightarrow a} \left| \frac{f(x)-f(a)}{x-a} - f'(a) \right|
\]
et
\[
\lim_{x\rightarrow a} g(x) = 0 \qquad \text{si et seulement si} \qquad
\lim_{x\rightarrow a} |g(x)| = 0 \; ,
\]
nous déduisons que
\[
\lim_{x\rightarrow a} \frac{f(x)-f(a)}{x-a} = f'(a)
\]
si et seulement si
\[
\lim_{x\rightarrow a} \frac{|f(x) - f(a) - f'(a)(x-a)|}{|x-a|} = 0 \; .
\]

Il découle de cette définition que, lorsque $x$ approche $a$, la
distance entre $f(x)$ et $f(a)+f'(a)(x-a)$ approche $0$ plus vite que
la distance entre $x$ et $a$ (figure~\ref{DER19}).  Cette
propriété explique les énoncées \lgm la courbe $y=f(x)$ et la droite
$y=h(x)=f(a)+f'(a) (x-a)$ sont presque superposées lorsque nous
considérons des valeurs de $x$ assez près de $x=a$ \rgm\ et \lgm la
distance entre $f(x)$ et $h(x)$ devient de plus en plus petite lorsque
$x$ approche $a$ \rgm.
\end{rmk}

\PDFfig{5_derivees/der19}{Approximation linéaire d'une fonction}
{Distance entre la courbe $y=f(x)$ et la droite tangente
$y=f(a)+f'(a)(x-a)$ à $x=b$.  La tangente à la courbe $y=f(x)$ au
point $(a,f(a))$ a une pente de $f'(a)$, le taux de variation
instantané de $f$ à $x=a$.}{DER19} 

\begin{rmk}
Une fonction peut ne pas avoir de dérivée en un point.  Prenons la
fonction $g(x) = |x|$.  La dérivée de $g$ au point $x=0$ n'existe pas.
Les tableaux suivants donnent deux suites
$\displaystyle \left\{x_n\right\}_{n=1}^\infty$ qui tendent vers $0$.
Dans le premier tableau, les valeurs
$\displaystyle \frac{g(x_n)-g(0)}{x_n-0}$
approchent $1$ alors que dans le second tableau, les valeurs
$\displaystyle \frac{g(x_n)-g(0)}{x_n-0}$ approchent $-1$.
\[
\begin{array}{l|l|l}
\hline
\rule[-1em]{0ex}{2.7em} n & \displaystyle x_n = \frac{1}{n} &
\displaystyle \frac{g(x_n)-g(0)}{x_n-0} \\
\hline
1 & 1 & 1 \\
2 & 1/2 & 1 \\
3 & 1/3 & 1 \\
\vdots & \vdots & \vdots \\
10000 & 1/10000 & 1 \\
\vdots & \vdots & \vdots \\
\hline
\end{array}
\qquad
\begin{array}{l|l|l}
\hline
\rule[-1em]{0ex}{2.7em} n & \displaystyle x_n = -\frac{1}{n} &
\displaystyle \frac{g(x_n)-g(0)}{x_n-0} \\
\hline
1 & -1 & -1 \\
2 & -1/2 & -1 \\
3 & -1/3 & -1 \\
\vdots & \vdots & \vdots \\
10000 & -1/10000 & -1 \\
\vdots & \vdots & \vdots \\
\hline
\end{array}
\]
Nous ne pouvons pas trouver une valeur unique $M$ telle que
$\displaystyle \frac{g(x_n)-g(0)}{x_n-0}$ approche $M$ pour toute
suite $x_n$ qui approche $0$ lorsque $n \to \infty$.  Donc
$\displaystyle \lim_{x\rightarrow 0} \frac{g(x)-g(0)}{x-0}$ n'existe 
pas, et $g'(0)$ n'existe pas.

De plus, à partir du graphe de $g$ à la figure~\ref{DER7}, nous voyons
qu'il n'y a pas de droite qui passe par $(0,g(0))=(0,0)$ et qui soit
indissociable de la courbe $y=g(x)=|x|$ pour $x$ assez près de $0$ où
$x$ peut être inférieur ou supérieur à $0$.  Nous pouvons satisfaire la
condition précédente pour $x<0$ ou $x>0$ séparément mais elle ne peut
pas être satisfaite pour $x<0$ et $x>0$ en même temps.  La droite
$y=x$ est superposée à la courbe $y=g(x)$ pour $x>0$ mais ne l'est pas
pour $x<0$.  De même, la droite $y=-x$ est superposée à la courbe
$y=g(x)$ pour $x<0$ mais ne l'est pas pour $x>0$.  Les autres droites
qui passent par $(0,0)$ sont clairement distinctes de la courbe
$y=g(x)$.
\end{rmk}

\PDFfig{5_derivees/der7}{Une fonction continue qui n'est pas
différentiable en un point}{Le graphe de $y = g(x) = |x|$.  La
fonction $g$ n'a pas de dérivée à $x=0$ car il n'y a pas de droite
qui soit indissociable du graphe de $g$ près de l'origine.}{DER7}

\begin{rmk}
Il n'est pas nécessaire que le graphe de la fonction ait un \lgm
coin\rgm\ comme à la remarque précédente pour que la fonction ne soit
pas différentiable. La fonction $f$ dont le graphe se trouve à la
figure~\ref{DER24} n'est pas différentiable à $x=a$ car la
tangente à la courbe $y=f(x)$ au point $(a,f(a))$ est verticale; elle
a donc une pente infinie.
\end{rmk}

\PDFfig{5_derivees/der24}{Fonction avec une tangente verticale en un
point}{La fonction $f$ représentée par le graphe ci-dessus
n'est pas différentiable à $x=a$ car la tangente à la courbe $y=f(x)$
au point $(a,f(a))$ est verticale.}{DER24}

\section{Dérivée d'une fonction} \label{deff_der}

À la section précédente, nous avons vu la définition de la dérivée $f'(a)$
d'une fonction $f$ en un point $a$; c'est-à-dire,
\[
f'(a) = \lim_{h\rightarrow 0} \frac{f(a+h)-f(a)}{h}
\]
si la limite existe.  Si nous calculons $f'(a)$ pour toutes les valeurs de
$a$, nous obtenons la fonction qui, pour tout nombre réel $a$, donne la
valeur $f'(a)$ de la dérivée de $f$ au point $a$.  Cette fonction
est appelée la {\bfseries dérivée de $f$}.  Comme la tradition veut
que nous utilisions $x$ comme variable indépendante d'une fonction, nous
obtenons la définition suivante.

\begin{defn} \index{Dérivée d'une fonction} 
La {\bfseries dérivée} d'une fonction $f$ est la fonction $f'$ définie
par
\[
f'(x) \equiv \lim_{h\rightarrow 0} \frac{f(x+h)-f(x)}{h}
\]
pour toutes les valeurs de $x$ où la limite existe.  Nous utilisons aussi
la notation $\displaystyle \dydx{f}{x}$ pour désigner $f'$.
\end{defn}

Puisque $f'(x)$ est la pente de la tangente au graphe de $f$ au point
$(x,f(x))$, nous en déduisons les propriétés suivantes.

\begin{prop}\label{CDDth}
Soit $f:]a,b[\rightarrow \RR$ une fonction continue sur l'intervalle
$]a,b[$.
\begin{enumerate}
\item Si $f$ est différentiable sur l'intervalle $]a,b[$ et la dérivée
est positive en tout point de cet intervalle, alors la fonction $f$
est strictement croissante sur l'intervalle $]a,b[$.  La
pente $f'(x)$ de la droite tangente à la courbe $y=f(x)$ au point
$(x,f(x)$ est positive pour tout $x$ entre $a$ et $b$
(figure~\ref{DER9}).
\item Si $f$ est différentiable sur l'intervalle $]a,b[$ et la dérivée
est négative en tout point de cet intervalle, alors la fonction $f$
est strictement décroissante sur l'intervalle $]a,b[$.  La pente
$f'(x)$ de la droite tangente à la courbe $y=f(x)$ au point $(x,f(x)$
est négative pour tout $x$ entre $a$ et $b$ (figure~\ref{DER10}).
\item Si $f$ est différentiable sur l'intervalle $]a,b[$ sauf
possiblement au point $c$, et si $f'(x)>0$ pour $x<c$ près de $c$ et
$f'(x)<0$ pour $x>c$ près de $c$, alors la fonction $f$ a un maximum
local à $x = c\in]a,b[$.  La fonction $f$ est strictement croissante
pour $x<c$ près de $c$ et strictement décroissante pour $x>c$ près de
$c$ (figure~\ref{DER11}).
\item Si $f$ est différentiable sur l'intervalle $]a,b[$ sauf
possiblement au point $c$, et si $f'(x)<0$ pour $x<c$ près de $c$ et
$f'(x)>0$ pour $x>c$ près de $c$, alors la fonction $f$ a un minimum
local à $x = c\in]a,b[$.  La fonction $f$ est strictement
décroissante pour $x<c$ près de $c$ et strictement croissante pour
$x>c$ près de $c$ (figure~\ref{DER12}).
\end{enumerate}
\end{prop}

\PDFfig{5_derivees/der9}{Fonction croissante}
{$f'(x)>0$ pour $a < x < b$ entraîne que la fonction $f$ est
strictement croissante pour $a<x<b$.}{DER9}

\PDFfig{5_derivees/der10}{Fonction décroissante}
{$f'(x)<0$ pour $a < x < b$ entraîne que la fonction $f$ est
strictement décroissante pour $a<x<b$.}{DER10}

\PDFfig{5_derivees/der11}{Maximum local d'une fonction}
{La fonction $f$ a un maximum local à $x=c$ car $f'(x)<0$ pour $x>c$ et
$f'(x)>0$ pour $x<c$.}{DER11}

\PDFfig{5_derivees/der12}{Minimum local d'une fonction}
{La fonction $f$ a un minimum local à $x=c$ car $f'(x)>0$ pour $x>c$ et
$f'(x)<0$ pour $x<c$.}{DER12}

\PDFfig{5_derivees/der8}{Une dérivée nulle n'implique pas un extremum}
{$f'(c)=0$ mais il n'y a pas de minimum ou maximum local à $x=c$.}{DER8}

\begin{egg}
La fonction $f$ dont le graphe est donné à la figure~\ref{DER1} est
strictement croissante pour $x<c$ et $x>e$; ce sont les intervalles où
$f'(x)>0$.  De même, $f$ est strictement décroissante pour $c<x<e$;
c'est l'intervalle où $f'(x)<0$.  Nous avons un maximum local à $x=c$
qui est prédit par $f'(x)<0$ pour $x>c$ près de $c$ et $f'(x)>0$ pour
$x<c$ près de $c$.  Finalement, nous avons un minimum local à $x=e$
qui est prédit par $f'(x)<0$ pour $x<e$ près de $e$ et $f'(x)>0$ pour
$x>e$ près de $e$.
\end{egg}

\begin{rmk}
Si $f'$ est une fonction continue, puisque $f'$ change de signe à un
maximum ou un minimum local $c$, nous devons avoir $f'(c)=0$.  Par contre,
$f'(c)=0$ n'implique pas que nous ayons un maximum ou minimum local
comme c'est le cas pour la fonction $f$ dont le graphe est donné à la
figure~\ref{DER8}.

De plus, $f$ peut ne pas avoir de dérivée à un maximum ou minimum
local comme c'est le cas pour la fonction $f$ dont le graphe est
donné à la figure~\ref{DER20}.

Nous reviendrons plus en détails sur ce sujet au chapitre sur les
applications de la dérivée.
\label{rmk_crpt}
\end{rmk}

\PDFfig{5_derivees/der20}{Un maximum local sans que la dérivée soit
nulle}{Il y a un maximum local au point $x=c$ mais la fonction $f$ n'a
pas de dérivée à ce point.}{DER20}

\begin{egg}
Le graphe de $f'$ est donné ci-dessous.
\PDFgraph{5_derivees/der21}

Si nous savons que $f(1) = 3$, nous pouvons dessiner un graphe approximatif de
$f$ à partir de l'information fournie par le graphe de $f'$.

Entre $1$ et $1.5$, la fonction est strictement croissante car $f'$
est positive.  Plus précisément, comme $f'$ décroît lorsque $x$
augmente de $1$ à $1.5$, la pente de $f$ est de moins en moins abrupte
lorsque $x$ augmente de $1$ à $1.5$ et la fonction $f$ croît donc de
moins en moins rapidement.

Entre $1.5$ et $5$, la fonction $f$ est strictement décroissante car
$f'$ est négative.  Plus précisément, comme $f'$ décroît lorsque $x$
augmente de $1.5$ à $3.5$, la pente de $f$ devient de plus en plus
abrupte et $f$ décroît donc de plus en plus rapidement lorsque $x$
augmente de $1.5$ à $3.5$.  Par contre, $f'$ augmente lorsque $x$
augmente de $3.5$ à $5$.  Ainsi, la pente de $f$ devient de moins en
moins abrupte et donc $f$ décroît de moins en moins rapidement lorsque
$x$ augmente de $3.5$ à $5$.

En raisonnant comme nous l'avons fait aux paragraphes précédents, nous
obtenons que $f$ croît de plus en plus rapidement lorsque $x$ augmente de $5$ à
$6$, et de moins en moins rapidement lorsque $x$ augmente de $6$ à
$7$.  Finalement, la fonction $f$ décroît de plus en plus rapidement
lorsque $x$ augmente de $7$ à $7.5$.

Nous retrouvons ci-dessous un graphe possible pour $f$.
\PDFgraph{5_derivees/der22}
\label{exDER21}
\end{egg}

Le théorème qui suit a généralement peu d'utilité du point de vue
numérique.  Par contre, il est très utile du point de vue conceptuelle
comme les exemples qui suivront son énoncé vont démontrer. 

\begin{theorem}[Théorème de la moyenne / Théorème des accroisse\-ments
 finis] \label{MVT}
\index{Théorème de la moyenne}\index{Théorème des accroissement finis}
Si $f$ est une fonction continue sur l'intervalle $[a,b]$ et
différentiable sur l'intervalle $]a,b[$, alors il existe une valeur
$\xi = \xi(a,b)$ entre $a$ et $b$ telle que
\[
f'(\xi) = \frac{f(b)-f(a)}{b-a} \; .
\]
Nous insistons sur le fait que $\xi$ dépend de $a$ et $b$ (d'où la
notation $\xi = \xi(a,b)$\ ) et donc varie si $a$ et $b$
changent (figure~\ref{APPROX1}). 
\end{theorem}

\PDFfig{5_derivees/approx1}{Une représentation graphique du théorème de
la moyenne}{Une représentation graphique du théorème de la
moyenne.  Les deux droites sont parallèles car elles ont la même
pente.}{APPROX1}

\begin{egg}
Si un train prend deux heures pour parcourir la distance de $200$ km
entre Ottawa à Montréal, sa vitesse moyenne est de $100$ km/h.  Alors
il faut qu'à un moment durant le voyage la vitesse (instantanée) du
train soit de $100$ km/h.  Si ce n'est pas le cas, cela implique que
la plus grande vitesse du train durant le voyage a été inférieure à
$100$ km/h (si nous supposons que le train parte au repos) et donc la
vitesse moyenne est inférieure à $100$ km/h.

De même, si le taux moyen de croissance d'une population de bactéries
pour une période de $24$ heures est $1,000$ bactéries par heure,
alors, à un moment durant la période de $24$ heures, le taux
de croissance instantané de la population a été de $1,000$ bactéries par
heure.

Ces deux observations sont deux réalisations concrètes du théorème de
la moyenne.
\end{egg}

\begin{rmk}[\theory]
Le théorème de la moyenne peut être utilisé pour donner une
démonstration analytique de la proposition~\ref{CDDth}; une
démonstration qui ne fait pas appel à l'interprétation graphique de la
dérivée.

Montrons que si $f:]a,b[\rightarrow \RR$ est une fonction telle que
$f'(x)>0$ pour $x\in]a,b[$ alors $f$ est strictement croissante sur
l'intervalle $]a,b[$.  Soit $x_1 < x_2$ deux points de l'intervalle
$]a,b[$.  Le théorème de la moyenne donne $\xi$ entre $x_1$ et $x_2$
tel que
\[
f(x_2)-f(x_1) = f'(\xi)(x_2-x_1) \; .
\]
Puisque $\xi \in ]a,b[$, nous avons que $f'(\xi)>0$ et ainsi
\[
f(x_2)-f(x_1) = f'(\xi)(x_2-x_1) >0 \ .
\]
Donc $f(x_2) > f(x_1)$.

De la même façon, nous pourrions montrer que si $f:]a,b[\rightarrow \RR$
est une fonction telle que $f'(x)<0$ pour $x\in]a,b[$ alors $f$ est
strictement décroissante sur l'intervalle $]a,b[$.
\end{rmk}

Nous verrons très prochainement que la dérivée d'une fonction constante
est nulle en tout point.  Est-ce que les fonctions constantes sont les
seules fonctions dont la dérivée est nulle en tout point?  Le théorème
de la valeur moyenne nous permet de répondre par l'affirmative à cette
question.  En effet, supposons que $f$ soit une fonction telle que
$f'(x) = 0$ pour tout $x$.  Si $x_1$ et $x_2$ sont deux points
distincts quelconques, il existe grâce au théorème de la moyenne un
point $\xi$ entre $x_1$ et $x_2$ telle que
\[
\frac{f(x_2)-f(x_1)}{x_2-x_1} = f'(\xi) \ .
\]
Or $f'(\xi)=0$.  Donc $f(x_2)=f(x_1)$.

\subsection{Différentiable implique continue}

Remarquons que la fonction $f$ dont le graphe donnée à la
figure~\ref{DER23} n'est pas différentiable au point $a$ où elle n'est
pas continue.  La pente
des sécantes $\ell_1$, $\ell_2$, $\ell_3$, \ldots approche moins
l'infini car ces sécantes approchent une droite verticale.  Nous avons donc
\[
\lim_{b\rightarrow a+} \frac{f(b)-f(a)}{b-a} = -\infty \; .
\]
Il semble donc que les fonctions doivent être continues aux points où
elles sont différentiables.  Nous allons vérifier que cela est vrai à
partir de la définition de la dérivée.

\PDFfig{5_derivees/der23}{Une fonction n'est pas différentiable aux
points où elle n'est pas continue}{Cette fonction n'est pas continue
à $x=a$ et donc sa dérivée n'existe pas à ce point,}{DER23}

\begin{theorem}
Si $f:]a,b[\rightarrow \RR$ est une fonction différentiable au point
$c\in ]a,b[$, alors $f$ est continue au point $c$.
\end{theorem}

Nous démontrons ce théorème de la façon suivante.  Si la fonction
$f$ est différentiable au point $c$, nous avons donc que la limite
\[
\lim_{z \rightarrow c} \frac{f(z)-f(c)}{z-c}
\]
existe et est égale à $f'(c) \in \RR$.  En particulier, $f$ doit être
définie au point $c$.  Mais alors,
\begin{align*}
\lim_{z \rightarrow c} \left( f(z) - f(c) \right)
&= \lim_{z\rightarrow c} \left(\frac{f(z)-f(c)}{z-c}\right) (z-c) \\
&= \lim_{z\rightarrow c} \left(\frac{f(z)-f(c)}{z-c}\right)
\times \lim_{z\rightarrow c} (z-c)
= f'(c) \times 0 = 0 \; .
\end{align*}
Ainsi,
\[
\lim_{z\rightarrow c} f(z) = f(c) \; .
\]
Ce qui prouve que $f$ est continue au point $c$.

\subsection{Une première application de la dérivée; la vitesse d'un
  objet}

Soit $x(t)$ la position en kilomètres (km) au temps $t$ en heures (h)
d'un véhicule se déplaçant en ligne droite.  La vitesse moyenne d'un
véhicule est donnée par la distance parcourue divisée par le temps
nécessaire pour couvrir cette distance.  Si la distance est en
kilomètres et le temps est en heures, la vitesse moyenne est en
kilomètres par heure (km/h).  La formule pour calculer la vitesse
moyenne entre $t=a$ heures et $t=b>a$ heures est donnée par
\[
\frac{x(b)-x(a)}{b-a} \quad km/h \ .
\]
Si $b$ est très près de $a$, alors nous pouvons assumer que la vitesse du
véhicule entre $t=a$ et $t=b$ heures est (presque) constante et est
égale à la vitesse moyenne entre $t=a$ et $t=b$.  Si nous laissons $b$
tendre vers $a$, nous obtenons la vitesse instantanée du véhicule au
temps $t=a$ heures.  C'est la vitesse du véhicule à précisément $t=a$
heures.  Donc
\[
x'(a) = \lim_{b\rightarrow a} \frac{x(b)-x(a)}{b-a}
\]
est la vitesse instantanée en km/h du véhicule à $t=a$ heures.  

\section{Dérivées de quelques fonctions élémentaires}
\label{basic_der}

Nous avons vu précédemment comment calculer numériquement la dérivée d'une
fonction en un point.  Malheureusement, pour analyser le comportement
d'une fonction sur un intervalle donné à l'aide de la dérivée, nous ne
pouvons pas calculer numériquement la dérivée à tous les points de
l'intervalle pour déterminer le signe de la dérivée en ces points.  Il
faut donc chercher des formules qui nous permettront de calculer la
dérivée d'une fonction rapidement et efficacement.

Nous allons voir dans les prochaines sections qu'il n'est pas
nécessaire de calculer les dérivées de fonctions polynomiales,
rationnelles, exponentielles et trigonométriques à partir de la
définition de la dérivée.  Il existe des formules générales pour
calculer ces dérivées.

Nous pourrions réduire ce chapitre à une courte section énonçant
les règles de dérivation sans ou avec le minimum de justifications.
Le lecteurs n'aurait eu qu'à mémoriser ces règles.  Cependant, cet
approche n'aurait pas permis au lecteur de développer ses capacités
de raisonnement logique et mathématique.

Nous avons donc choisi de présenter dans cette section et celles qui
suivent les règles de dérivation de façon logique et avec le plus de
rigueur possible sans aller aux extrêmes.  Chaque règle est introduite
quand et seulement quand les concepts nécessaires pour justifier cette
règle ont été présentés.  C'est pour cette raison, par exemple, que la
règle de dérivation pour $f(x) = x^\alpha$ avec $\alpha$ un nombre
réel quelconque est présentée seulement à la presque toute fin du
chapitre.   C'est portant une des premières règles de dérivation qui
est donnée dans le cours de mathématiques de 12$^e$ année au niveau
secondaire.

\subsection{Dérivée de $\mathbf{f(x) = x^n}$ où $\mathbf n$ est
un entier positif ou nul}

Quelle est la dérivée de la fonction constante $f$; c'est-à-dire, de
la fonction définie par $f(x) = c$ pour tout $x$ où $c$ est une
constante?

Si nous utilisons le fait que la dérivée d'une fonction en un point est la
pente de la droite tangente au graphe de cette fonction en ce point,
nous trouvons que $f'(x) = 0$ pour tout $x$ car le graphe de $f$ est une
droite horizontal et les droites horizontales ont une pente nul.  Nous
utilisons le fait que la tangente à un droite est la droite elle-même.

Nous pouvons aussi utiliser la définition de la dérivée pour démontrer que
$f'(x) = 0$ pour tout $x$.  Puisque
\[
\frac{f(x+h)-f(x)}{h} = \frac{c-c}{h} = 0
\]
quel que soit $h$, nous avons que
\[
f'(x) = \lim_{h\rightarrow 0} \frac{f(x+h)-f(x)}{h}
= \lim_{h\rightarrow 0} 0 = 0
\]
pour tout $x$.  La fonction constante à $0$ approche évidemment $0$
quand nous approchons l'origine.  Nous obtenons donc la règle suivante.

\begin{prop}
Soit $c$ une constante.  Si $f(x) = c$ pour tout $x$, alors
$f'(x) = 0$ pour tout $x$.
\end{prop}

Considérons maintenant la fonction $f$ définie par $f(x) = x$ pour
tout $x$.  Si nous utilisons
encore le fait que la dérivée d'une fonction en un point est la pente
de la droite tangente au graphe de cette fonction en ce point, nous
trouvons que $f'(x) = 1$ pour tout $x$ car $y = f(x)= x$ est une 
droite de pente $1$ qui est sa propre tangente en tout point.

Nous pouvons démontrer à partir de la définition de la dérivée que la
dérivée de $f(x) = x$ est bien $f'(x) = 1$ pour tout $x$.  Pour faire 
cela, il suffit de remarquer que
\[
\frac{f(x+h)-f(x)}{h} = \frac{(x+h)-x}{h} = 1
\]
pour tout $h \neq 0$.  Donc
\[
\lim_{h\rightarrow 0} \frac{f(x+h)-f(x)}{h} = \lim_{h\rightarrow 0} 1
= 1 \ .
\]
La fonction constante à $1$ sur la droite réelle approche
évidemment $1$ quand nous approchons l'origine.

Considérons maintenant la fonction $f(x)=x^2$ pour tout $x$.  Dans le
tableau suivant, nous avons estimé $f'(x)$ pour quelques valeurs de $x$ à
l'aide du rapport $\displaystyle \frac{f(x+h)-f(x)}{h}$ où
$h = 0.00001$.
\[
\begin{array}{l|l|l}
\hline
x & \rule{0em}{1.7em}
\displaystyle f'(x) \approx \frac{f(x+0.00001)-f(x)}{0.00001}
& |2x - f'(x)| \\[0.8em]
\hline
-1 & -2 & 0 \\
1 & 2.00001000001\ldots & 0.1000001\ldots \times 10^{-4} \\
2 & 4.00001000002\ldots & 0.1000002\ldots \times 10^{-4} \\
\pi & 6.2831953071\ldots & 0.99999\ldots \times 10^{-5} \\
3.5 & 7.00001000009\ldots & 0.1000009\ldots \times 10^{-4} \\
\hline
\end{array}
\]
Nous remarquons que $f'(x) \approx 2x$ pour les valeurs de $x$ considérées
dans le tableau.  Nous pouvons conjecturer que $f'(x) = 2x$ pour tout $x$.

Montrons à partir de la définition de la dérivée que la dérivée de la
fonction $f(x)=x^2$ est bien $f'(x) = 2x$ pour tout $x$.  Nous avons
\[
\frac{f(x+h)-f(x)}{h} = \frac{(x+h)^2 - x^2}{h} = 2x + h
\]
qui approche $2x$ lorsque $h$ tend vers $0$.  Donc
\[
f'(x) = \lim_{h\rightarrow 0} \frac{f(x+h)-f(x)}{h}
= \lim_{h\rightarrow 0}\, (2x+h) = 2 x
\]
quel que soit $x$.

Maintenant que nous savons que la dérivée de $f(x)=x^2$ est $f'(x) = 2x$
pour tout $x$, il est très facile de trouver la pente de la tangente à
la courbe $y=f(x)$ en un point de cette courbe.  Il n'est plus
nécessaire de calculer numériquement les pentes des tangentes à l'aide
de limites de pentes de sécantes.

\begin{egg}
Trouvons l'équation de la droite tangente à la courbe $y= f(x) = x^2$
au point $(x,y)=(3,9)$ sur cette courbe.

Nous avons vu que la pente de la tangente à la courbe $y=f(x)$ au point
$(a,f(a))$ est donnée par $f'(a)$.  Ainsi, la pente de la tangente à la
courbe $y= f(x) = x^2$ au point $(3,9)$ est $f'(3) = 2\times 3 = 6$.

L'équation de la tangente dans la forme point-pente est donc
$y-9 = 6 (x-3)$, ce qui donne $y = 6x -9$ dans la forme standard.
Nous avons tracé le graphe de $y=x^2$ et de sa tangente au point
$(3,9)$ à la figure~\ref{DER13}.
\end{egg}

\MATHfig{5_derivees/der13}{8cm}{Tangente au graphe de $y=x^2$ en un point}
{Graphe de $y=x^2$ et de sa tangente au point $(3,9)$.}{DER13}

En procédant comme nous l'avons fait avec la fonction $f(x) = x^2$,
nous pouvons montrer que la dérivée de $f(x) = x^3$ est $f'(x) = 3x^2$,
celle de $f(x)=x^4$ est $f'(x) = 4 x^{3}$, et ainsi de suite.  Nous
pouvons obtenir un résultat dans le cas général où $f(x) = x^n$ avec
$n$ un entier positif à l'aide de la formule du binôme.

Puisque
\[
(x+h)^n = x^n + n h x^{n-1} + \frac{n(n-1)}{2} h^2 x^{n-2} + \ldots +
nh^{n-1}x + h^n \quad ,
\]
nous avons
\begin{align*}
f'(x) &= \lim_{h\to 0}\frac{f(x+h)-f(x)}{h}
= \lim_{h\to 0}\frac{(x+h)^n-x^n}{h} \\
&= \lim_{h\to 0}\frac{1}{h}
\left( \left(x^n + n h x^{n-1} + \frac{n(n-1)}{2} h^2 x^{n-2} + \ldots +
nh^{n-1}x + h^n\right)-x^n\right) \\
&= \lim_{h\to 0}
\left(n x^{n-1} + 
\underbrace{\frac{n(n-1)}{2} h x^{n-2}}_{\to 0 \text{ lorsque } h \to 0}
+ \ldots + \underbrace{\;nh^{n-2}x\;}_{\to 0 \text{ lorsque }h \to 0}
+ \underbrace{\quad h^{n-1}\quad }_{\to 0 \text{ lorsque } h\to 0}\right) \\
&= n x^{n-1}
\end{align*}
Nous obtenons la règle suivante.

\begin{prop}
Soit $f(x) = x^n$ pour un entier $n\neq 0$, alors $f'(x) = n x^{n-1}$
pour les valeurs de $x$ où $x^{n-1}$ est définie.
\end{prop}

\subsection{Dérivée du sinus et du cosinus \life \eng}

Si $f(x) = \cos(x)$, quel sera $f'(x)$?

Commençons par quelques observations qui nous guideront dans la
recherche de la dérivée de $f$.  Puisque $f$ est périodique de période
$2\pi$, alors $f'$ sera aussi périodique de période au plus $2\pi$
comme nous pouvons le constater à partir du graphe à la
figure~\ref{DER14}.  Par exemple, la tangente à la courbe $y = f(x)$ au point
$(\pi/6,\cos(\pi/6)) = (\pi/6,\sqrt{3}/2)$ est parallèle à la tangente
à cette même courbe au point
$(13\pi/6,\cos(13\pi/6)) = (13\pi/6,\sqrt{3}/2)$.  Les deux 
tangentes ont donc la même pente.  Ainsi, $f'(\pi/6) = f'(13\pi/6)$.

\MATHfig{5_derivees/der14}{8cm}{Périodicité de la tangente au graphe de
$\cos(x)$}{Le graphe de $y=f(x)=\cos(x)$ et de deux de ses tangentes.
Ces deux tangentes sont parallèles.}{DER14}

De plus, puisque $f(x) = f(2\pi-x)$, le graphe de $f$ est symétrique
par rapport à l'axe vertical $x=\pi$.  Ainsi, la tangente à la courbe
$y=f(x)$ au point $(2\pi-a,f(2\pi-a))$ est obtenue de la tangente à la
courbe $y=f(x)$ au point $(a,f(a))$ par une symétrie par rapport à
l'axe vertical $x=\pi$ (figure~\ref{DER25}).  En particulier,
$f'(2\pi-a)=-f'(a)$.

Puisque $f$ a des maximums locaux à $x=2n\pi$ et des minimums locaux à
$x = (2n-1)\pi$ quel que soit l'entier $n$, nous avons que $f'(n\pi) = 0$ 
pour tout entier $n$.

\MATHfig{5_derivees/der25}{8cm}{Symétrie de la tangente au graphe de
$\cos(x)$}{Le graphe de $y=f(x)=\cos(x)$ et de deux de ses tangentes.
Ces deux tangentes sont symétriques par rapport à l'axe
$x=\pi$.}{DER25}

À partir du graphe de $f$, nous pouvons supposer que la plus petite pente
pour les tangentes à la courbe $y=f(x)$ est lorsque $x=(4n+1)\pi/2$ et
la plus grande pente pour les tangentes de cette même courbe est
lorsque $x=(4n+3)\pi/2$ quel que soit l'entier $n$.  Cela sera
démontré plus loin.  Donc $f'$ aura des minimums locaux (tous de même
valeur) aux points $x=(4n+1)\pi/2$ et des maximums locaux (tous de
même valeur) aux points $x=(4n+3)\pi/2$ quel que soit l'entier $n$.

Si nous utilisons le fait que $f'(x)$ est la pente de la tangente à la
courbe $y=f(x)$ au point $(x,f(x))$ sur la courbe, nous pouvons déterminer
le signe de $f'(x)$ et même estimer la valeur de $f'(x)$ pour prédire
la forme du graphe de $f'$.  Nous avons tracé un graphe possible pour $f'$ à
la figure~\ref{DER25X}.

\PDFfig{5_derivees/der25X}{Un graphe possible pour la dérivée du
cosinus}{Un graphe possible pour la dérivée du cosinus basé sur le
signe de la pente de la tangente à la courbe $y=f(x) = \cos(x)$.}
{DER25X}

Dans le tableau~\ref{TAB_COS_P}, nous avons estimé $f'(x)$ pour quelques
valeurs de $x$ à l'aide du rapport $\displaystyle \frac{f(x+h)-f(x)}{h}$ où
$h = 0.00001$.

\begin{table}
{\scriptsize
\begin{center}
\begin{tabular}{l|c|c}
\cline{1-3}
\rule[-0.7em]{0ex}{2.5em} $x$
& $\displaystyle f'(x) \approx \frac{f(x+0.00001)-f(x)}{0.00001}$
& $|-\sin(x)-f'(x)|$ \\[0.8em]
\cline{1-3}
$0.000000\ldots$ & $-0.000005\ldots$ & $0.000005\ldots$ \\ 
$\pi/12 = 0.261799\ldots$ & $-0.258824\ldots$ & $0.000005\ldots$ \\ 
$\pi/6 = 0.523599\ldots$ & $-0.500004\ldots$ & $0.000004\ldots$ \\ 
$\pi/4 = 0.785398\ldots$ & $-0.707110\ldots$ & $0.000004\ldots$ \\ 
$\pi/3 = 1.047198\ldots$ & $-0.866028\ldots$ & $0.000002\ldots$ \\ 
$5\pi/12 = 1.308997\ldots$ & $-0.965927\ldots$ & $0.000001\ldots$ \\ 
$\pi/2 = 1.570796\ldots$ & $-1.000000\ldots$ & $0.000000\ldots$ \\ 
$7\pi/12 = 1.832596\ldots$ & $-0.965925\ldots$ & $0.000001\ldots$ \\ 
$2\pi/3 = 2.094395\ldots$ & $-0.866023\ldots$ & $0.000003\ldots$ \\ 
$3\pi/4 = 2.356194\ldots$ & $-0.707103\ldots$ & $0.000004\ldots$ \\ 
$5\pi/6 = 2.617994\ldots$ & $-0.499996\ldots$ & $0.000004\ldots$ \\ 
$11\pi/12 = 2.879793\ldots$ & $-0.258814\ldots$ & $0.000005\ldots$ \\ 
$\pi = 3.141593\ldots$ & $0.000005\ldots$ & $0.000005\ldots$ \\ 
$13\pi/12 = 3.403392\ldots$ & $0.258824\ldots$ & $0.000005\ldots$ \\ 
$7\pi/6 = 3.665191\ldots$ & $0.500004\ldots$ & $0.000004\ldots$ \\ 
$5\pi/4 = 3.926991\ldots$ & $0.707110\ldots$ & $0.000004\ldots$ \\ 
$4\pi/3 = 4.188790\ldots$ & $0.866028\ldots$ & $0.000002\ldots$ \\ 
$17\pi/12 = 4.450590\ldots$ & $0.965927\ldots$ & $0.000001\ldots$ \\ 
$3\pi/2 = 4.712389\ldots$ & $1.000000\ldots$ & $0.000000\ldots$ \\ 
$19\pi/12 = 4.974188\ldots$ & $0.965925\ldots$ & $0.000001\ldots$ \\ 
$5\pi/3 = 5.235988\ldots$ & $0.866023\ldots$ & $0.000003\ldots$ \\ 
$7\pi/4 = 5.497787\ldots$ & $0.707103\ldots$ & $0.000004\ldots$ \\ 
$11\pi/6 = 5.759587\ldots$ & $0.499996\ldots$ & $0.000004\ldots$ \\ 
$23\pi/12 = 6.021386\ldots$ & $0.258814\ldots$ & $0.000005\ldots$ \\ 
$2\pi = 6.283185\ldots$ & $-0.000005\ldots$ & $0.000005\ldots$ \\ 
\cline{1-3}
\end{tabular}
\end{center}
}
\caption[Approximations de la dérivée de $f(x) = \cos(x)$ à partir de
la définition de la dérivées.]{Approximations de la dérivée de $f(x) =
\cos(x)$ \`a l'aide du rapport $\displaystyle \frac{f(x+h)-f(x)}{h}$
où $h = 0.00001$ \label{TAB_COS_P}}
\end{table}

Remarquons que $f'(x) \approx -\sin(x)$ pour les quelques valeurs de $x$
dans le tableau~\ref{TAB_COS_P}.

Le graphe associé aux données du tableau~\ref{TAB_COS_P} ainsi que le
graphe de $y=-\sin(x)$ sont donnés à la figure~\ref{DER15}.  Les deux
graphes sont très semblables.  Pour obtenir un meilleur graphe de $f'$,
nous suggérons au lecteur d'ajouter plus de points au
tableau~\ref{TAB_COS_P} et de choisir $h$ plus petit dans
l'approximation $\displaystyle \frac{f(x+h)-f(x)}{h}$ de $f'(x)$.

\MATHfig{5_derivees/der15}{8cm}{Le graphe de la dérivée de $\cos(x)$}
{Les approximations de $f'(x)$ où $f(x)=\cos(x)$ qui se retrouvent
dans le tableau~\ref{TAB_COS_P} sont représentées par des cercles.  La
courbe pleine est le graphe de $-\sin(x)$.}{DER15}

Nous allons démontrer plus loin qu'il est effectivement vrai que
$\displaystyle f'(x) = -\sin(x)$.

Si $g(x) = \sin(x)$, nous pourrions procéder comme nous venons de le faire
pour $f(x)=\cos(x)$ pour montrer numériquement que
$g'(x) = \cos(x)$.  Cependant, il y a une façon plus simple de
démontrer cela en assumant que la dérivée de $\cos(x)$ est
$-\sin(x)$.

Puisque
\[
g(x) = \sin(x) = \cos\left(x-\frac{\pi}{2}\right) =
f\left(x-\frac{\pi}{2}\right) \; ,
\]
nous obtenons le graphe de $g$ par une translation de $\pi/2$
vers la droite du graphe de $f$.  Il s'en suit que les tangentes au
graphe de $g$ sont obtenues par une translation de $\pi/2$
vers la droite des tangentes au graphe de $f$, et il en est de même
pour les pentes de ses tangentes.  Ainsi,
\[
g'(x) = f'\left(x - \frac{\pi}{2}\right) =
-\sin\left(x - \frac{\pi}{2}\right) = \cos(x) \; .
\]
Nous pouvons aussi expliquer cette égalité à l'aide de la définition de la
dérivée.  Pour $a\in \RR$, nous avons
\begin{align*}
g'(a) &= \lim_{h\rightarrow 0} \frac{g(a+h)-g(a)}{h}
= \lim_{h\rightarrow 0} \frac{f\left((a+h)-\frac{\pi}{2}\right) -
f\left(a-\frac{\pi}{2}\right)}{h} \\
&= \lim_{h\rightarrow 0} \frac{f\left(\left(a-\frac{\pi}{2}\right) + h \right)
- f\left(a-\frac{\pi}{2}\right)}{h}
= f'\left(a-\frac{\pi}{2}\right) \ .
\end{align*}

En combinant les deux cas précédents nous obtenons les formules suivantes.

\begin{prop}
\[
\dfdx{\cos(x)}{x} = -\sin(x) \qquad \text{et} \qquad
\dfdx{\sin(x)}{x} = \cos(x) \ .
\]
\end{prop}

Soit $f(x) = \cos(x)$.  Pour démontrer que $f'(x) = -\sin(x)$, il faut
se rappeler que
\[
\cos(a+b) = \cos(a)\cos(b) - \sin(a)\sin(b) \; .
\]
Ainsi,
\begin{align*}
\frac{\cos(x+h)-\cos(x)}{h}
&= \frac{\cos(x)\cos(h) - \sin(x)\sin(h) - \cos(x)}{h} \\
&= \cos(x) \frac{\cos(h)-1}{h} - \sin(x) \frac{\sin{h}}{h} \; .
\end{align*}

Or, nous avons vu à l'exemple~\ref{R_SINXX} que
\[
\lim_{h\rightarrow 0} \frac{\sin(h)}{h} =1 \; .
\]

De plus,
\begin{align*}
\lim_{h\rightarrow 0} \frac{\cos(h)-1}{h}
&= \lim_{h\rightarrow 0} \left(\frac{\cos(h)-1}{h}\right)
\left(\frac{\cos(h)+1}{\cos(h)+1}\right)
= \lim_{h\rightarrow 0}\frac{\cos^2(h)-1}{h(\cos(h)+1)} \\
&= \lim_{h\rightarrow 0} \frac{-\sin^2(h)}{h(\cos(h)+1)}
= -\lim_{h\rightarrow 0} \frac{\sin(h)}{h}
\times \lim_{h\rightarrow 0} \left(\frac{\sin(h)}{\cos(h)+1}\right) \\
&= - 1 \times 0 = 0
\end{align*}
car $\displaystyle \frac{\sin(h)}{\cos(h)+1}$ est une fonction
continue à $h=0$ et donc
\[
\lim_{h\rightarrow 0} \frac{\sin(h)}{\cos(h)+1}=
\frac{\sin(0)}{\cos(0)+1} = \frac{0}{2} = 0 \; .
\]

En conclusion,
\begin{align*}
f'(x) &= \lim_{h\rightarrow 0} \frac{\cos(x+h)-\cos(x)}{h} =
\cos(x) \lim_{h\rightarrow 0}  \frac{\cos(h)-1}{h} 
- \sin(x) \lim_{h\rightarrow 0} \frac{\sin{h}}{h} \\
&= \cos(x) \times 0 - \sin(x) \times 1 = -\sin(x)
\end{align*}
quel que soit $x$.

\section{Règles de dérivation}

Quelle est la dérivée de $x^{10}+5x^3+4$, de $x \cos(x)$, de
$\cos(x^2)$, etc?  Il n'est pas raisonnable de chercher
systématiquement la dérivée de chaque fonction comme nous l'avons fait à la
section précédente.  Il y a des règles de dérivation qui nous
permettront de calculer la dérivée d'un grand nombre de fonctions sans
avoir à utiliser la définition de la dérivée comme nous l'avons fait
pour les fonctions de la section précédente.

\subsection{Dérivée d'une fonction multipliée par une constante}

\begin{egg}
Soit $x(t)$ le nombre de bactéries au temps $t$ en heures dans une
culture A et $y(t)$ le nombre de bactéries au temps $t$ en heures dans
une culture B.  Si nous avons deux bactéries de la culture A pour chaque
bactérie de la culture B en tout temps, alors pour chaque bactérie
qui s'ajoute à la culture B, deux bactéries doivent s'ajouter à la
culture A pour préserver le rapport de $2$ pour $1$.  Par exemple, si
pendant une période de deux heures, le nombre de bactéries de la
culture B augmente de $10,000$ à $11,000$ alors le nombre de bactéries
de la culture A doit augmenter de $20,000$ à $22,000$ pour préserver
le rapport de $2$ pour $1$.  Le taux de croissance de la culture A est
donc deux fois celui de la culture B.

Mathématiquement, si $x(t) = 2 y(t)$ pour tout $t$ alors
$x'(t) = 2 y'(t)$ pour tout $t$ car $x'(t)$ et $y'(t)$ sont les taux
de croissance (taux de variation instantanée) au temps $t$ pour les
cultures A et B respectivement.
\end{egg}

L'exemple précédent suggère la règle suivante.

\begin{theorem}
Si $f:]a,b[\rightarrow \RR$ et $g:]a,b[\rightarrow \RR$ sont deux fonctions
différentiables et $f = c g$ où $c$ est une constante, alors
\[
f'(x) = c g'(x)
\]
pour tout $x \in ]a,b[$.
\end{theorem}

Vérifions à partir de la définition de la dérivée que la règle
précédente est vrai.  Si $f(x) = c g(x)$ pour tout $x\in]a,b[$, alors
\[
\frac{f(x+h)-f(x)}{h} = \frac{c g(x+h) - c g(x)}{h}
= c\; \frac{g(x+h)- g(x)}{h} \; .
\]
Ainsi,
\[
f'(x) = \lim_{h\rightarrow 0} \frac{f(x+h)-f(x)}{h} =
c \; \lim_{h\rightarrow 0} \frac{g(x+h)- g(x)}{h} = c g'(x)
\]
pour tout $x\in]a,b[$.

\begin{egg}
Cette règle nous permet donc de calculer les dérivées suivantes.
\begin{align*}
\dfdx{\big(5x^7\big)}{x} &= 5 \dfdx{\big(x^7\big)}{x}
= 5 ( 7 x^6) = 35 x^6
\intertext{et}
\dfdx{\big(7 \cos(x)\big)}{x} &= 7 \dfdx{\big(\cos(x)\big)}{x}
= 7 ( -\sin(x)) = -7
\sin(x)
\end{align*}
\end{egg}

\subsection{Dérivée d'une somme de fonctions}

\PDFfig{5_derivees/train}{Distance parcourue par un passager de train}
{Distance parcourue par un passager de train qui roule en ligne
droite.}{TRAIN} 

\begin{egg}
Un train se déplace en ligne droite (figure~\ref{TRAIN}).  Si
$x(t)$ est la distance au temps $t$ entre la queue du train et le
début du trajet, la vitesse du train (taux de variation instantané de
sa position) au temps $t$ est donnée par $x'(t)$.

Si $y(t)$ est la distance entre un passager et la queue du train au
temps $t$, alors la vitesse du passager par rapport au train au temps
$t$ est $y'(t)$.  Comme le passager passe normalement la majorité de
son temps assis, nous avons que $y(t)$ est constant et donc $y'(t) = 0$ pour
de long intervalles.  De plus, $y'(t)$ sera plus grand que $0$ si la
personne se déplace vers l'avant du train et plus petit que $0$ si la
personne se déplace vers l'arrière du train.

La distance au temps $t$ entre le passager et le début du trajet est
$z(t) = x(t) + y(t)$, la somme de la distance entre le début du trajet
et la queue du train, et de la distance entre la queue du train et la
position du passager dans le train.

Si, $30$ minutes (i.e. $0.5$ heure) après le début du trajet, le train
se déplace à une vitesse de $130$ km/heure et le passager se déplace 
vers l'avant du train par rapport au train à une vitesse de $3$
km/heure, alors la vitesse du passager par rapport au début du trajet
en ligne droite est de $133$ km/heure.

Mathématiquement, ce que nous venons de dire est que
$z'(0.5) = 133 = 130 + 3 = x'(0.5) + y'(0.5)$.
\end{egg}

L'exemple précédent suggère la règle suivante pour la dérivée.

\begin{theorem}
Si $f:]a,b[\rightarrow \RR$ et $g:]a,b[\rightarrow \RR$ sont deux
fonctions différentiables et $q:]a,b[\rightarrow \RR$ est définie par
$q=f+g$, alors
\[
q'(x) = f'(x) + g'(x)
\]
pour tout $x\in ]a,b[$.
\end{theorem}

Vérifions à partir de la définition de la dérivée que la règle
précédente est vraie.  Par définition de la dérivée de $q$,
\[
q'(x) = \lim_{h\rightarrow 0} \frac{q(x+h)-q(x)}{h} \; .
\]
Or,
\begin{align*}
\frac{q(x+h)-q(x)}{h} &= \frac{f(x+h)+g(x+h) - f(x) - g(x)}{h} \\
&= \frac{f(x+h)- f(x)}{h} + \frac{g(x+h) - g(x)}{h} \; .
\end{align*}
Puisque
\[
\lim_{h\rightarrow 0} \frac{f(x+h)- f(x)}{h} = f'(x)
\quad \text{et} \quad
\lim_{h\rightarrow 0} \frac{g(x+h)- g(x)}{h} = g'(x) \; ,
\]
nous avons
\begin{align*}
q'(x) &= \lim_{h\rightarrow 0} \frac{q(x+h)-q(x)}{h} \\
&= \lim_{h\rightarrow 0} \frac{f(x+h)- f(x)}{h}  + 
\lim_{h\rightarrow 0} \frac{g(x+h)- g(x)}{h} = f'(x) + g'(x) \; .
\end{align*}

\begin{egg}
Soit $p(x) = 5x^{10} + 3x^4 + 2x^2 +7$, alors
\begin{align*}
\dfdx{p(x)}{x} &= \dfdx{\big(5x^{10}\big)}{x}
+ \dfdx{\big(3x^4\big)}{x} + \dfdx{\big(2x^2\big)}{x}
+ \dfdx{\big(7\big)}{x} \\
&= 5\dfdx{\big(x^{10}\big)}{x} + 3\dfdx{\big(x^4\big)}{x}
+ 2\dfdx{\big(x^2\big)}{x} + \dfdx{\big(7\big)}{x} \\
&= 5(10x^9) + 3(4x^3) + 2(2x) + 0 = 50x^9 + 12x^3 + 4x
\end{align*}
où la règle pour calculer la dérivée de $x^n$ a été utilisée.
\end{egg}

\begin{egg}
Pour trouver l'équation de la droite tangente à la courbe
$y = p(x) = 3x^{10} + 5x^4$ lorsque $x=1$, il faut trouver la pente de
la tangente à la courbe $y=p(x)$ au point $(1,p(1)) = (1,8)$.  Cette
pente est donnée par $p'(1)$.  Or
\[
p'(x) = \dfdx{\big(3x^{10}\big)}{x} + \dfdx{\big(5x^4\big)}{x}
= 3 \dfdx{\big(x^{10}\big)}{x} + 5\dfdx{\big(x^4\big)}{x}
= 3 (10x^9) + 5 (4x^3) = 30x^9 + 20x^3 \; .
\]
Donc $p'(1) = 50$.  L'équation de la tangente dans la forme
point-pente est $y - 8 = 50 (x - 1)$, ce qui donne $y = 50x -42$. 
\end{egg}

\subsection{Dérivée du produit de fonctions}

\begin{egg}
Supposons que la longueur des côtés d'un rectangle varient en fonction
du temps.  Soit $x(t)$ la longueur de la base du rectangle en mètres
au temps $t$ en secondes, et soit $y(t)$ la longueur de la hauteur du
rectangle en mètres au temps $t$ en secondes.  L'aire $A(t)$ du
rectangle en mètres carrés au temps $t$ en secondes est donc
$A(t) = x(t) y(t)$ m$^2$.

Si la longueur de la base augmente à une vitesse constante de $2$ m/s
et la longueur de la hauteur augmente à une vitesse constante de $3$
m/s, quel sera le taux de croissance instantané de l'aire du rectangle
en m$^2$/s?

Nous avons que $x'(t) = 2$ m/s et $y'(t) = 3$ m/s pour tout $t$.  Si la
longueur initiale de la base est de $10$ m et la longueur initiale de
la hauteur est de $11$ m, alors la longueur de la base augmente de
$10$ à $12$ m et la longueur de la hauteur augmente $11$ à $14$ m en
une seconde.  L'aire du rectangle augmente donc de
$10\times 11 =110$ m$^2$ à $12\times 14=168$ m$^2$ en une seconde, une
augmentation moyenne de $58$ m$^2$/s.

Est-ce que $A'(t) = 58$ m$^2$/s pour tout $t$?  En particulier, est-ce
que la vitesse de croissance de l'aire du rectangle est constante par
rapport au temps?

Toujours en supposant que la longueur initiale de la base soit de $10$
m et celle de la hauteur soit de $11$ m, après $2$ secondes la
longueur de la base est de $14$ m et celle de la hauteur est de $17$
m.  L'aire du rectangle augmente donc de $10\times 11 = 110$ m$^2$ à
$14\times 17 = 238$ m$^2$ en deux secondes.  Ce qui donne une
augmentation moyenne de $64$ m$^2$/s.  Nous ne pouvons donc pas avoir
$A'(t) = 58$ m$^2$/s pour tout $t$.

Comment pouvons-nous calculer $A'(t)$ à l'aide $x'(t)$ et $y'(t)$?  En
fait, pouvons-nous calculer $A'(t)$ à l'aide $x'(t)$ et $y'(t)$?

Il est clair que la formule $A'(t) = x'(t) y'(t)$ est
{\bfseries fausse}.  Les calculs précédents montrent que le taux de
croissance instantané de l'aire du rectangle varie avec le temps, ce
qui n'est pas le cas pour le produit $x'(t)y'(t)$ qui est égal à $6$
pour tout $t$.  En fait, une étude des unités utilisées montre que
cette formule n'a pas de sens.  Puisque les unités de $x'(t)$ et de
$y'(t)$ sont des m/s, les unités de $x'(t) y'(s)$ sont des m$^2$/s$^2$
et non des m$^2$/s comme nous devons avoir pour $A'(t)$.

Considérons la variation de l'aire sur un très petit intervalle de
temps $\Delta t$.  Durant cet petit intervalle de temps, la longueur
de la base augmente de $x$ à $x+\Delta x$ mètres et la longueur de la
hauteur augmente de $y$ à $y+\Delta y$ mètres (figure~\ref{CARRE}).
L'aire du rectangle augmente donc de $xy$ à
\[
(x+ \Delta x)(y + \Delta y) = x y+ x \Delta y +y \Delta x + \Delta x
\Delta y \; .
\]
Lorsque $\Delta t$ approche $0$, nous avons que
$\Delta x \Delta y / \Delta t$ converge vers $0$ car
\[
\frac{\Delta x \Delta y}{\Delta t} =
\Delta x \; \frac{\Delta y}{\Delta t}
\]
ou $\Delta x$ converge vers $0$ et $\Delta y/\Delta t$
converge vers $y'$ lorsque $\Delta t$ approche $0$.  Ainsi,
\[
\frac{\Delta A}{\Delta t} =
\frac{(x+ \Delta x)(y + \Delta y)-xy}{\Delta t}
= x \frac{\Delta y}{\Delta t} + y \frac{\Delta x}{\Delta y} 
+ \frac{\Delta x \Delta y}{\Delta t} \to x y' + y x'
\]
lorsque $\Delta t$ approche $0$.
\end{egg}

\PDFfig{5_derivees/carre}{L'aire d'un rectangle dont les dimensions
varient avec le temps}{L'aire du rectangle augmente de $xy$ à
$(x+\Delta x)(y+\Delta y) = xy + x\,\Delta y+y\,\Delta x+\Delta x\,\Delta y$.}
{CARRE}

La formule obtenue à l'exemple précédant suggère donc le résultat suivant.

\begin{theorem}
Si $f:]a,b[\rightarrow \RR$ et $g:]a,b[\rightarrow \RR$ sont deux
fonctions différentiables et $q:]a,b[\rightarrow \RR$ est définie par
$q = fg$, alors
\[
q'(x) = f'(x) g(x) + f(x) g'(x)
\]
pour tout $x\in ]a,b[$.
\end{theorem}

Posons $q(x) = f(x) g(x)$ pour tout $x$.  Par définition de la
dérivée,
\[
q'(x) = \lim_{h\rightarrow 0} \frac{q(x+h)-q(x)}{h} \; .
\]
Or,
\begin{align*}
\frac{q(x+h)-q(x)}{h} &= \frac{f(x+h)g(x+h) - f(x)g(x)}{h} \\
&= \frac{f(x+h)g(x+h) - f(x)g(x+h) + f(x)g(x+h) - f(x)g(x)}{h} \\
&= \left(\frac{f(x+h) - f(x)}{h}\right) g(x+h)
+f(x) \left( \frac{g(x+h) - g(x)}{h} \right) \; .
\end{align*}
Comme $g$ est différentiable au point $x$ par hypothèse, $g$ est
continue au point $x$.  Ainsi,
\[
\lim_{h\rightarrow 0} g(x+h) = g(x) \;.
\]
De plus,
\[
\lim_{h\rightarrow 0} \frac{f(x+h) - f(x)}{h} = f'(x)
\quad \text{et} \quad
\lim_{h\rightarrow 0} \frac{g(x+h) - g(x)}{h} = g'(x)
\]
car $f$ et $g$ sont différentiables au point $x$ par hypothèse.  Nous avons
donc
\begin{align*}
q'(x) &= \lim_{h\rightarrow 0} \frac{q(x+h)-q(x)}{h} \\
&= \left( \lim_{h\rightarrow 0} \frac{f(x+h) - f(x)}{h} \right)
\left( \lim_{h\rightarrow 0} g(x+h) \right)
+ f(x) \left( \lim_{h\rightarrow 0} \frac{g(x+h) - g(x)}{h} \right) \\
&=f'(x) g(x) + f(x) g'(x) \; .
\end{align*}
Ce qui démontre la formule pour la dérivée du produit de deux
fonctions.

\begin{egg}
Retournons à l'exemple précédent où nous devions déterminer le taux de
croissance instantané de l'aire d'un rectangle dont la longueur des
côtés varie en fonction du temps.  Notre formule pour la dérivée du
produit de fonctions donne donc
\[
A'(t) = x'(t) y(t) + x(t) y'(t) = 2 y(t) + 3 x(t) \quad \text{m$^2$/s} \ .
\]
Nous observons que le taux de croissance instantané de l'aire du rectangle
n'est pas constant par rapport au temps même si les taux de croissance
instantanés des longueurs des côtés sont constants.  Par exemple, le
taux de croissance instantané au début (à $t=0$) est
\[
A'(0) = x'(0) y(0) + x(0) y'(0) = 2 \times 11 + 3 \times 10
= 52 \quad \text{m$^2$/s}
\]
et celui à $t=1$ est
\[
A'(1) = x'(1) y(1) + x(1) y'(1) = 2 \times 14 + 3 \times 12
= 64 \quad \text{m$^2$/s} \ .
\]

Si nous étudions les unités utilisées, nous obtenons de la formule
$A'(t) = x'(t) y(t) + x(t) y'(t)$ que les unités de $A'(t)$ sont des
m/s $\times$ m + m $\times$ m/s = m$^2$/s comme il se doit.
\end{egg}

\begin{egg}
Soit $q(x) = (x^3+2x)\sin(x)$.  La fonction $q$ est le produit des
fonctions $f(x) = x^3+2x$ et $g(x) = \sin(x)$.  Ainsi,
\[
q'(x) = f'(x) g(x) + f(x) g'(x) = (3x^2 +2)\sin(x) + (x^3+2x)\cos(x) \ .
\]
\end{egg}

\subsection{Dérivée du quotient de fonctions}

Quelle est la dérivée de $g(x) = p(x) / q(x)$?  En multipliant par
$q(x)$ des deux côtés, nous obtenons $g(x) q(x) = p(x)$.  Maintenant, en
dérivant des deux côtés, nous obtenons
\begin{align*}
g'(x) q(x) + g(x) q'(x) = p'(x)
&\Leftrightarrow g'(x) q(x) = p'(x) - g(x) q'(x) \\
&\Leftrightarrow g'(x) = \frac{p'(x)}{q(x)} - \frac{g(x) q'(x)}{q(x)} \; .
\end{align*}
Or $g(x) = p(x) / q(x)$, donc
\[
g'(x) = \frac{p'(x)}{q(x)} - \frac{p(x) q'(x)}{q^2(x)}
= \frac{p'(x) q(x) - p(x) q'(x)}{q^2(x)} \ .
\]

Cette formule pour calculer la dérivée du quotient de deux fonctions
mérite d'être mise en évidence car elle est utile.

\begin{prop} \index{règle de dérivée du quotient}
Soit $p:]a,b[\to \RR$ et $q:]a,b[\to \RR$ deux fonctions
différentiables.  Si $q(x) \neq 0$ pour $a<x<b$, alors
\[
\dfdx{\left(\frac{p(x)}{q(x)}\right)}{x}
= \frac{p'(x) q(x) - p(x) q'(x)}{q^2(x)}
\]
Cette formule est appelée la {\bfseries règle de dérivée du
quotient} de deux fonctions.
\end{prop}

\begin{egg}
Calculons la dérivée de $g(x) = (x^3+2)/(x^2+5x)$.

Il suffit d'appliquer la règle de dérivée du quotient de deux
fonctions ci-dessus avec $p(x) = x^3+2$ et $q(x) = x^2+5x$.  Ainsi,
\[
g'(x) = \frac{p'(x) q(x) - p(x) q'(x)}{q^2(x)}
= \frac{(3x^2)(x^2+5x) - (x^3+2)(2x+5)}{(x^2+5x)^2} \; .
\]
\end{egg}

\begin{egg}
Calculons la dérivée de $\tan(x)$.

Puisque $\tan(x) = \sin(x) / \cos(x)$, nous pouvons utiliser la règle de la
dérivée du quotient de deux fonctions ci-haut avec
$p(x) = \sin(x)$ et $q(x) = \cos(x)$.  Ainsi,
\begin{align*}
\dfdx{\tan(x)}{x} &= \frac{\left(\displaystyle \dfdx{\sin(x)}{x}\right) \cos(x)
- \sin(x) \left(\displaystyle \dfdx{\cos(x)}{x}\right)}{\cos^2(x)} \\
&= \frac{ \cos^2(x) + \sin^2(x)}{\cos^2(x)}
= \frac{1}{\cos^2(x)}
= \left(\frac{1}{\cos(x)}\right)^2 = \sec^2(x)
\end{align*}
où nous avons utilisé l'identité $\cos^2(x) + \sin^2(x) = 1$.
\end{egg}

En procédant de la même façon que dans l'exemple précédent, nous
obtenons le résultat suivant.

\begin{prop}
\[
\dfdx{\tan(x)}{x} = \sec^2(x) \ ,
\quad \dfdx{\cot(x)}{x} = -\csc^2(x) \ ,
\quad \dfdx{\sec(x)}{x} = \tan(x) \sec(x)
\]
et
\[
\dfdx{\csc(x)}{x} = -\cot(x) \csc(x) \; .
\]
\end{prop}

\subsection{Dérivée de fonctions composées}

\begin{egg}
Nous étudions les populations d'ours et de poissons le long d'une section
de $1$ kilomètre d'une rivière du nord de l'Ontario.

Nous estimons que le nombre de poissons augmente de $147$ lorsque la
température de l'eau de la rivière augmente de $2^\circ$C et que le
nombre d'ours augmente de $2$ lorsque le nombre de poissons augmente
de $49$.  Nous pouvons donc dire que le nombre d'ours augmente de $3$
lorsque la température de l'eau augmente de $1^\circ$C car une
augmentation de $2^\circ$C entraîne une augmentation de $147$ poissons
qui entraîne un augmentation de $6$ ours ($2$ ours par tranche de $49$
poissons).

Essayons de reformuler le raisonnement du paragraphe précédent en
termes mathématiques.  Si $P(T)$ est le nombre de poissons lorsque la
température de l'eau est de $T^\circ$C et $N(P)$ est le nombre d'ours
lorsqu'il y a $P$ poissons, alors le nombre d'ours en fonction de la
température $T$ en degrés centigrades de l'eau est donné par
$B(T) = (N\circ P)(T) = N(P(T))$.   Nous avons que $P'(T) = 147/2$
poissons/degré et $N'(P) = 2/49$ ours/poisson.  L'énoncé du paragraphe
précédent se traduit donc par la formule mathématique
\[
B'(T) = N'(P(T)) P'(T) = \frac{2}{49} \times \frac{147}{2} = 3
\quad \text{ours/degré} .
\]
En d'autres mots, le taux de croissance du nombre d'ours par rapport à
la température de l'eau est le produit du taux de croissance du nombre
d'ours par rapport au nombre de poissons ($2/49$ ours/poisson) et du
taux de croissance du nombre de poissons par rapport à la température
de l'eau ($149/2$ poissons/degré).

La formule $B'(T) = N'(P(T)) P'(T)$ est cohérente avec les
unités utilisées.  Les unités de $P'(T)$ sont des poissons/degré et
les unités de $N'(P(T))$ sont des ours/poisson.  Il n'est donc pas 
surprenant que les unités de $B'(T) = N'(P(T)) P'(T)$ soient
des ours/degré.
\end{egg}

L'exemple précédent suggère la règle suivante.

\begin{theorem}
Soit $f:]c,d[\rightarrow\RR$ et $g:]a,b[\rightarrow \RR$ deux
fonctions différentiables telles que $g(x) \in ]c,d[$ pour tout
$x\in]a,b[$.  La dérivée de la fonction $q:]a,b[\rightarrow \RR$
définie par $q = f \circ g$ est
\[
q'(x) = f'(g(x)) g'(x) \; .
\]
\end{theorem}

Vérifions que cette règle est vraie à partir de la définition de la
dérivée.  Par définition de la dérivée,
\[
q'(x) = \lim_{h\rightarrow 0} \frac{q(x+h)-q(x)}{h} \; .
\]
Le quotient dans la limite précédente peut s'écrire
\begin{align*}
\frac{q(x+h)-q(x)}{h} &= \frac{f(g(x+h)) - f(g(x))}{h} \\
&= \left( \frac{f(g(x+h)) - f(g(x))}{g(x+h)-g(x)} \right)
\left( \frac{g(x+h)-g(x)}{h} \right) \; .
\end{align*}
Nous allons calculer la limite lorsque $h$ tend vers $0$ pour chacun
des facteurs de l'expression précédente.  Pour le deuxième facteur, nous
remarquons que
\[
\lim_{h\rightarrow 0} \frac{g(x+h)-g(x)}{h} = g'(x)
\]
car $g$ est différentiable par hypothèse.  Pour le premier facteur,
nous remarquons premièrement que
\[
\lim_{h\rightarrow 0} g(x+h) = g(x)
\]
car $g$ est une fonction différentiable, donc continue, sur
l'intervalle $]a,b[$.  Puisque $f$ est différentiable sur l'intervalle
$]c,d[$ et $g(x) \in ]c,d[$, nous avons
\[
f'(g(x))=\lim_{b\rightarrow g(x)} \frac{f(b) - f(g(x))}{b-g(x)} \; .
\]
Nous en déduisons que
\[
\frac{f(g(x+h)) - f(g(x))}{g(x+h)-g(x)} \to f'(g(x))
\]
lorsque $h \to 0$ car $g(x+h) \to g(x)$ lorsque $h \to 0$.
C'est-à-dire,
\[
\lim_{h\rightarrow 0} \frac{f(g(x +h) - f(g(x))}{g(x+h) - g(x)}
= f'(g(x)) \; .
\]
En conclusion,
\begin{align*}
q'(x) &= \lim_{h\rightarrow 0} \frac{q(x+h)-q(x)}{h}
= \left(\lim_{h\rightarrow 0} \frac{f(g(x+h)) - f(g(x))}{g(x+h)-g(x)}\right)
\left(\lim_{h\rightarrow 0} \frac{g(x+h)-g(x)}{h}\right) \\
&= f'(g(x)) g'(x) \; .
\end{align*}

\begin{rmk}[\theory]
Soit $x \in ]a,b[$ arbitraire mais fixe.  Pour démontrer que
\begin{equation}\label{derComp}
\lim_{h\rightarrow 0} \frac{f(g(x +h) - f(g(x))}{g(x+h) - g(x)}
= f'(g(x)) \; .
\end{equation}
à l'aide de la définition de limite en termes de $\epsilon$ et
$\delta$, nous choisissons $\epsilon >0$.  Puisque
\[
f'(g(x))=\lim_{b\rightarrow g(x)} \frac{f(b) - f(g(x))}{b-g(x)}
\]
pour $g(x) \in ]c,d[$, il existe $\delta_1 >0$ tel que
\[
\left| \frac{f(b) - f(g(x))}{b-g(x)} - f'(g(x))\right| < \epsilon
\]
pour $|b-g(x)| < \delta_1$.   De plus, puisque $g$ est continue
à $x \in ]a,b[$, il existe $\delta$ tel que
$|g(x+h) - g(x)| < \delta_1$ pour $|h|< \delta$.  Donc, pour
$|h|<\delta$, nous avons
\[
\left| \frac{f(g(x+h)) - f(g(x))}{g(x+h)-g(x)} - f'(g(x))\right|
< \epsilon
\]
Puisque $\epsilon$ est arbitraire, nous obtenons (\ref{derComp}).
\end{rmk}

\begin{egg}
Calculons la dérivée de $q(x) = (1+x^2)^{101}$.

Il faut remarquer que
$q = f\circ g$ où $g(x) = 1 + x^2$ et $f(y) = y^{101}$.  Puisque
$g'(x) = 2x$ et $f'(y) = 101 y^{100}$, nous trouvons que
\[
q'(x) = f'(g(x)) g'(x) = 101 (1+x^2)^{100} \times 2 x
= 202 x (1+x^2)^{100} \; .
\]
\end{egg}

\begin{egg}
Calculons la dérivée de $q(x) = \cos(1-x^5)$.

Il faut remarquer que
$q = f\circ g$ où $g(x) = 1-x^5$ et $f(y) = \cos(y)$.  Puisque
$g'(x) = -5x^4$ et $f'(y) = -\sin(y)$, nous trouvons que
\[
q'(x) = f'(g(x)) g'(x) = -\sin(1+x^5) \times (-5x^4)
= 5x^4 \sin(1+x^5) \; .
\]
\end{egg}

\begin{egg}
Nous pouvons utiliser la dérivée de fonctions composées pour obtenir la
formule de dérivation d'un quotient de fonctions.  Soit
$g(x) = p(x) / q(x)$?  La fonction $g$ est le produit des fonctions
$p$ et $r \circ q$ où $r(y) = y^{-1}$.  La règle pour la dérivée du
produit de fonctions donne
\[
g'(x) = p'(x)\; r(q(x)) + p(x)
\left(\dfdx{r(q(x))}{x}\right) \; .
\]
Or,
\[
\dfdx{r(q(x))}{x} = r'(q(x)) q'(x) = - (q(x))^{-2} q'(x)
\]
car $r'(y) = -y^{-2}$.  Nous avons donc
\[
g'(x) = p'(x) (q(x))^{-1} - p(x) (q(x))^{-2} q'(x)
= \frac{p'(x) q(x) - p(x) q'(x)}{q^2(x)} \; .
\]
\end{egg}

\begin{egg}
Calculons la dérivée de $g(x) = (x^3+2)/(x^2+5x)$ sans utiliser la
règle de dérivation du quotient de deux fonctions.

Nous avons que $g(x) = p(x) r(q(x))$ où $p(x) = x^3+2$, $q(x) = x^2+5x$ et
$r(y) = y^{-1}$.  Donc
\begin{align*}
g'(x) &= p'(x) r(q(x)) + p(x) r'(q(x)) q'(x) \\
&= (3x^2)(x^2+5x)^{-1} + (x^3+2) \left( -(x^2+5x)^{-2}\right) (2x + 5)\\
&= \frac{(3x^2)(x^2+5x) - (x^3 + 2)(2x-5)}{(x^2+5x)^2} \; .
\end{align*}
Pour obtenir la première égalité, nous avons utilisé la règle pour la
dérivée du produit de fonctions et celle pour la dérivée des fonctions
composées.
\end{egg}

\section{Encore plus de dérivées de fonctions élémentaires}

Avec nos connaissances des règles de dérivation, nous pouvons maintenant
développer des formules pour calculer la dérivée de fonctions
exponentielles et logarithmiques.

\subsection{Dérivée de $\mathbf{f(x) = b^x}$}\label{SECT_B_EXP}

\begin{table}[t]
{\scriptsize
\begin{center}
\begin{tabular}{r|c|ccr|c|c}
\cline{1-3} \cline{5-7}
\rule[-0.7em]{0ex}{2.5em}
$x$ & $\displaystyle f'(x) \approx \frac{f(x+h)-f(x)}{h}$ & $f'(x)/3^x$ &
\qquad \qquad &
$x$ & $\displaystyle f'(x) \approx \frac{f(x+h)-f(x)}{h}$ & $f'(x)/2^x$
\\[0.8em]
\cline{1-3} \cline{5-7}
$-2.00$ & $0.122069\ldots$ & $1.098618\ldots$ && $-2.00$ & $0.173287\ldots$ & $0.693150\ldots$ \\
$-1.84$ & $0.145527\ldots$ & $1.098618\ldots$ && $-1.84$ & $0.193612\ldots$ & $0.693150\ldots$ \\  
$-1.68$ & $0.173493\ldots$ & $1.098618\ldots$ && $-1.68$ & $0.216320\ldots$ & $0.693150\ldots$ \\ 
$-1.52$ & $0.206834\ldots$ & $1.098618\ldots$ && $-1.52$ & $0.241691\ldots$ & $0.693150\ldots$ \\
$-1.36$ & $0.246582\ldots$ & $1.098618\ldots$ && $-1.36$ & $0.270039\ldots$ & $0.693150\ldots$ \\
$-1.20$ & $0.293969\ldots$ & $1.098618\ldots$ && $-1.20$ & $0.301711\ldots$ & $0.693150\ldots$ \\
$-1.04$ & $0.350462\ldots$ & $1.098618\ldots$ && $-1.04$ & $0.337098\ldots$ & $0.693150\ldots$ \\
$-0.88$ & $0.417811\ldots$ & $1.098618\ldots$ && $-0.88$ & $0.376635\ldots$ & $0.693150\ldots$ \\
$-0.72$ & $0.498103\ldots$ & $1.098618\ldots$ && $-0.72$ & $0.420809\ldots$ & $0.693150\ldots$ \\
$-0.56$ & $0.593826\ldots$ & $1.098618\ldots$ && $-0.56$ & $0.470165\ldots$ & $0.693150\ldots$ \\
$-0.40$ & $0.707943\ldots$ & $1.098618\ldots$ && $-0.40$ & $0.525309\ldots$ & $0.693150\ldots$ \\
$-0.24$ & $0.843991\ldots$ & $1.098618\ldots$ && $-0.24$ & $0.586921\ldots$ & $0.693150\ldots$ \\
$-0.08$ & $1.006183\ldots$ & $1.098618\ldots$ && $-0.08$ & $0.655759\ldots$ & $0.693150\ldots$ \\
$0.08$ & $1.199545\ldots$ & $1.098618\ldots$ && $0.08$ & $0.732672\ldots$ & $0.693150\ldots$ \\
$0.24$ & $1.430066\ldots$ & $1.098618\ldots$ && $0.24$ & $0.818605\ldots$ & $0.693150\ldots$ \\
$0.40$ & $1.704886\ldots$ & $1.098618\ldots$ && $0.40$ & $0.914616\ldots$ & $0.693150\ldots$ \\
$0.56$ & $2.032520\ldots$ & $1.098618\ldots$ && $0.56$ & $1.021889\ldots$ & $0.693150\ldots$ \\
$0.72$ & $2.423116\ldots$ & $1.098618\ldots$ && $0.72$ & $1.141744\ldots$ & $0.693150\ldots$ \\
$0.88$ & $2.888774\ldots$ & $1.098618\ldots$ && $0.88$ & $1.275655\ldots$ & $0.693150\ldots$ \\
$1.04$ & $3.443919\ldots$ & $1.098618\ldots$ && $1.04$ & $1.425273\ldots$ & $0.693150\ldots$ \\
$1.20$ & $4.105749\ldots$ & $1.098618\ldots$ && $1.20$ & $1.592440\ldots$ & $0.693150\ldots$ \\
$1.36$ & $4.894764\ldots$ & $1.098618\ldots$ && $1.36$ & $1.779212\ldots$ & $0.693150\ldots$ \\
$1.52$ & $5.835407\ldots$ & $1.098618\ldots$ && $1.52$ & $1.987891\ldots$ & $0.693150\ldots$ \\
$1.68$ & $6.956816\ldots$ & $1.098618\ldots$ && $1.68$ & $2.221045\ldots$ & $0.693150\ldots$ \\
$1.84$ & $8.293731\ldots$ & $1.098618\ldots$ && $1.84$ & $2.481545\ldots$ & $0.693150\ldots$ \\
$2.00$ & $9.887565\ldots$ & $1.098618\ldots$ && $2.00$ & $2.772598\ldots$ & $0.693150\ldots$ \\
\cline{1-3} \cline{5-7}
\end{tabular}
\end{center}
}
\caption[Approximations de la dérivée de $f(x) = 3^x$ et de $f(x) =2^x$ à
partir de la définition de la dérivées.]{Approximations de la dérivée
de $f(x) = 3^x$ à gauche et de $f(x) = 2^x$ à droite à l'aide du rapport
$\displaystyle \frac{f(x+h)-f(x)}{h}$ où $h = 0.00001$\ .\label{TAB_32_EXP}}
\end{table}

\begin{egg}
Quelle est la dérivée de $f(x) = 3^x$?

Dans la partie gauche du tableau~\ref{TAB_32_EXP}, nous avons
estimé $f'(x)$ pour quelques valeurs de $x$ à l'aide du rapport
$\displaystyle \frac{f(x+h)-f(x)}{h}$ où $h = 0.00001$.

Remarquons que $f'(x) / f(x) \approx 1.098618\ldots$ pour les
quelques valeurs de $x$ dans la partie gauche du tableau~\ref{TAB_32_EXP}.  Si
$C_3 = 1.098618\ldots$, nous semblons avoir
\[
\dfdx{3^x}{x} = C_3 3^x \; .
\]
Le graphe associé aux données de la partie gauche du
tableau~\ref{TAB_32_EXP} ainsi que le graphe de $y=f(x)=3^x$ sont
donnés à la figure~\ref{DER16}.  Il semble bien que $f'(x)$ soit un
multiple plus grand que $1$ de $f(x)$.
\label{EGG_3_EXP}
\end{egg}

\MATHfig{5_derivees/der16}{8cm}{Le graphe de la dérivée de $f(x)=3^x$}
{Les approximations de $f'(x)$ où $f(x)=3^x$ qui se retrouvent dans la
partie gauche du tableau~\ref{TAB_32_EXP} sont représentées par des
cercles.  La courbe pleine est le graphe de $3^x$.}{DER16}

\begin{egg}
Soit $f(x) = 2^x$.  Essayons de voir si, comme à l'exemple précédent,
le rapport $f'(x)/f(x)$ est constant.  Si c'est le cas, est-ce la même
constante qu'à l'exemple précédent?

Dans la partie droite du tableau~\ref{TAB_32_EXP}, nous avons
estimé $f'(x)$ pour quelques valeurs de $x$ à l'aide du rapport
$\displaystyle \frac{f(x+h)-f(x)}{h}$ où $h = 0.00001$.

Remarquons que $f'(x)/f(x) \approx 0.693150\ldots$ pour les quelques
valeurs de $x$ dans la partie droite du tableau~\ref{TAB_32_EXP}.  Si
$C_2 = 0.693150\ldots$, nous semblons avoir
\[
\dfdx{2^x}{x} \approx C_2 2^x \; .
\]
Comme à l'exemple précédent, nous avons que $f'(x)/f(x)$ est constant mais
la constante n'est pas la même qu'à l'exemple précédent.

Le graphe associé aux données de la partie droite du
tableau~\ref{TAB_32_EXP} ainsi que le graphe de $y=f(x)=2^x$ sont
donnés à la figure~\ref{DER17}.  Il semble bien que $f'(x)$ soit un
multiple plus petit que $1$ de $f(x)$.
\label{EGG_2_EXP}
\end{egg}

\MATHfig{5_derivees/der17}{8cm}{Le graphe de la dérivée de $f(x)=2^x$}
{Les approximations de $f'(x)$ où $f(x)=2^x$ qui se retrouvent dans la
partie droite du tableau~\ref{TAB_32_EXP} sont représentées par des
cercles.  La courbe pleine est le graphe de $2^x$.}{DER17}

Les trois questions suivantes découlent des exemples que nous venons de
présenter.
\begin{enumerate}
\item Est-il toujours vrai que, quelle que soit la base $b$ utilisée,
la dérivée de $f(x) = b^x$ est $f'(x) = C_b b^s$ où $C_b$ est une
constante qui dépend seulement du choix de $b$?
\item Si la réponse à la question précédente est affirmative, existe-t-il un
moyen de calculer facilement cette constante $C_b$ en fonction de la base $b$
qui a été choisie?
\item Existe-t-il une base $b$ pour laquelle la dérivée de $f(x)=b^x$
est $f'(x) = f(x) = b^x$?  C'est-à-dire, pour laquelle la constante de
proportionnalité $C_b$ est $1$?
\end{enumerate}

Nous répondons à la première question et donne une réponse partielle à la
deuxième question dans la prochaine sous-section.  La réponse à la
troisième question sera donnée dans la sous-section suivante.  La
réponse complète à la deuxième question dépend de la dérivée de
fonctions composées.  Cette réponse sera donnée dans la dernière
sous-section.

\subsubsection{La constante $\mathbf{C_b}$}

Pour répondre à la première question et aborder la réponse pour la
deuxième question, il faut avoir recours à la définition de la
dérivée.  Si $f(x) = b^x$ pour une base quelconque $b>0$, alors
\[
\frac{f(x+h)-f(x)}{h} = \frac{b^{x+h} - b^x}{h}
= \frac{b^x b^h - b^x}{h} = b^x \frac{b^h-1}{h} \; .
\]
Comme $b^x$ ne dépend pas de $h$, nous avons donc
\[
f'(x) = \lim_{h\rightarrow 0} \frac{f(x+h)-f(x)}{h}
= b^x \underbrace{\lim_{h\rightarrow 0} \frac{b^h-1}{h}}_{= C_b}
\]
si la limite existe.  Pour une base $b$ donnée, l'expression
$\displaystyle C_b \equiv \lim_{h\rightarrow 0} (b^h-1)/h$ est une
constante car elle est indépendante du point $x$ qui est choisi.

Nous avons bien que, quelle que soit la base $b$ utilisée, la dérivée de
$f(x)=b^x$ est $f'(x) = C_b b^x$ pour une constante $C_b$.  Pour 
calculer cette constante $C_b$, nous pouvons utiliser la formule
\begin{equation} \label{COND_EXP}
C_b \equiv \lim_{h\rightarrow 0} \frac{b^h-1}{h} \;.
\end{equation}

\begin{rmk}[\theory]
Dans la discussion précédente, nous avons supposé que la limite
$\displaystyle \lim_{h\rightarrow 0} \frac{b^h-1}{h}$ existait pour
tout $b>0$.  Effectivement, la limite existe et cela sera prouvé à la
section~\ref{DEFofE}; après que nous aurons défini la fonction
exponentielle $e^x$ à l'aide d'une série.
\end{rmk}

\begin{egg}
Soit $f(x) = 3^x$.  Pour estimer $C_3$, nous utilisons (\ref{COND_EXP})
avec $b = 3$ et $h = h_n = 1/n^2$ pour $n=1$, $2$, $3$, \ldots
La suite $\displaystyle \left\{h_n\right\}_{n=1}^\infty$ converge vers
$0$.  Donc $\displaystyle \frac{3^{h_n}-1}{h_n}$ converge vers $C_3$
lorsque $n \to \infty$.  Nous obtenons les résultats suivants.
\[
\begin{array}{c|c|l}
\hline
\rule[-0.4em]{0ex}{1.5em} n & h_n = 1/n^2 &
\rule{0em}{1.7em} \displaystyle \frac{3^{h_n}-1}{h_n} \\[0.8em]
\hline
1 & 1 & 2 \\
2 & 1/4 & 1.26429605180997\ldots \\
3 & 1/9 & 1.16847867518778\ldots \\
4 & 1/16 & 1.13720772916663\ldots \\
5 & 1/25 & 1.12310877860219\ldots \\
\vdots & \vdots & \vdots \\
100 & 1/10000 & 1.09867263832664\ldots \\
\vdots & \vdots & \vdots \\
1000 & 1/1000000 & 1.09861289221413\ldots \\
\vdots & \vdots & \vdots \\
\end{array}
\]
Nous avons donc que $\displaystyle \frac{3^{h_n}-1}{h_n}$ approche la constante
$1.09861\ldots$ lorsque $h_n$ approche $0$.  Comme toute autre
séquence qui tend vers $0$ donnerait le même résultat, nous obtenons
\[
C_3 \equiv \lim_{h\rightarrow 0} \frac{3^h-1}{h} = 1.09861\ldots
\]
qui est la constante trouvée à l'exemple~\ref{EGG_3_EXP}.
\label{COND_EXP_EGG}
\end{egg}

La Formule~\ref{COND_EXP} pour évaluer la constante $C_b$ demande le
calcul d'une limite comme nous venons de le faire à l'exemple précédent
pour $b=3$.  Il existe une façon plus simple (du point de vue de
l'utilisateur d'une calculatrice) de calculer la valeur de $C_b$ que
nous donnerons bientôt.

\subsubsection{Le nombre $\mathbf e$}

Nous répondons maintenant à la troisième question au début de la section.
C'est-à-dire, nous cherchons la base $b$ telle que la dérivée de
$f(x) = b^x$ soit $f'(x) = f(x) = b^x$.

Nous avons refait le travail de l'exemple~\ref{COND_EXP_EGG} en choisissant
des valeurs pour la base $b$ de $f(x) = b^x$ de telle sorte que la
constante de proportionnalité $C_b$ entre $f$ et sa dérivée soit de
plus en plus près de $1$.  Nous résumons dans le tableau~\ref{TAB_EXP}
suivant les résultats que nous avons trouvés.

\begin{table}
{\scriptsize
\begin{center}
\begin{tabular}{|c|c|c|l|}
\hline
$b$ & $C_b \approx $ & & Prochaine valeur de $b$ \\
\hline
$2.0$ & $0.6931470952\ldots$ & $<1$ & \\
$3.0$ & $1.0986123122\ldots$ & $>1$ & $2.5$ entre $2$ et $3$ \\
$2.5$ & $0.9162908209\ldots$ & $<1$ & $2.7$ entre $2.5$ et $3$ \\
$2.7$ & $0.9932517031\ldots$ & $<1$ & $2.8$ entre $2.7$ et $3$ \\
$2.8$ & $1.0296195007\ldots$ & $>1$ & $2.75$ entre $2.7$ et $2.8$ \\
$2.75$ & $1.011600803\ldots$ & $>1$ & $2.72$ entre $2.7$ et $2.75$ \\
$2.72$ & $1.0006317996\ldots$ & $>1$ & $2.71$ entre $2.7$ et $2.72$ \\
$2.71$ & $0.9969487457\ldots$ & $<1$ & $2.715$ entre $2.71$ et $2.72$ \\
$2.715$ & $0.9987919380\ldots$ & $<1$ & $2.718$ entre $2.715$ et $2.72$ \\
$2.718$ & $0.9998963879\ldots$ & $<1$ & $2.719$ entre $2.718$ et $2.72$ \\
$2.719$ & $1.0002640937\ldots$ & $>1$ & $2.7185$ entre $2.718$ et $2.719$ \\
$2.7185$ & $1.0000802408\ldots$ & $>1$ & $2.7182$ entre $2.718$ et $2.7185$ \\
$2.7182$ & $0.9999698846\ldots$ & $<1$ & $2.7183$ entre $2.7182$ et $2.7185$ \\
$2.7183$ & $1.0000067440\ldots$ & $>1$ & $2.71825$ entre $2.7182$ et
$2.7183$ \\
$2.71825$ & $0.9999883143\ldots$ & $<1$ & $2.71828$ entre $2.71825$ et
$2.7183$ \\
$2.71828$ & $0.9999994166\ldots$ & $<1$ & $2.71829$ entre $2.71828$ et
$2.7183$ \\
$2.71829$ & $1.0000029693\ldots$ & $>1$ & $2.718285$ entre $2.71828$ et
$2.71829$ \\
$2.718285$ & $1.0000011929\ldots$ & $>1$ & $2.718282$ entre $2.71828$ et
$2.718285$ \\
 \vdots & \vdots & \vdots & \vdots \\
$2.718281828$ & $0.9999999939\ldots$ & & \\
\hline
\end{tabular}
\end{center}
}
\caption[Approximation de $e$]{$C_b$ est estimé à l'aide de (\ref{COND_EXP}).
Le nombre $e$ est plus grand que les valeurs de $b$ pour lesquelles
$C_b<1$ et plus petit que les valeurs de $b$ pour lesquelles $C_b>1$.  Nous
utilisons ce résultat pour prendre en sandwich le nombre
$e \approx 2.718281828$.  \label{TAB_EXP}} 
\end{table}

Les valeurs de $b$ dans le tableau~\ref{TAB_EXP} approche le nombre
$e = 2.71828182852\ldots$.  Nous avons donc le résultat suivant.

\begin{prop}
Si $e = 2.71828182852\ldots$ et $f(x) = e^x$ pour tout $x$, alors
\[
f'(x) = f(x)
\]
pour tout $x$.
\end{prop}

La formule (\ref{COND_EXP}) avec la base $b = e$ devient donc
\[
C_e = \lim_{h\rightarrow 0} \frac{e^h-1}{h} = 1 \;.
\]
Si $f(x) = e^x$, la formule précédente n'est nul autre que
\[
f'(0) = \lim_{h\rightarrow 0} \frac{f(0+h) - f(0)}{h} =
\lim_{h\rightarrow 0} \frac{e^h-1}{h} = 1 \;.
\]
La pente de la courbe $y=e^x$ à $x=0$ est donc $1$.  La fonction
$f(x)=e^x$ est la seule fonction exponentielle de la forme $b^x$ dont
la valeur de la dérivée à l'origine est $1$.

\begin{rmk}
Nous avons vu la définition suivante de $e$ à la section~\ref{nbrE}.
\begin{equation} \label{LIM_EXP}
e = \lim_{n\rightarrow \infty} \left(1+\frac{1}{n}\right)^n \; .
\end{equation}
Est-ce le même nombre $e$ que celui que nous venons de définir?

Avec la définition de $e^x$ donnée à la section~\ref{DEFofE}, il est
trivial de montrer que le nombre $e$ définie par (\ref{LIM_EXP})
satisfait $\displaystyle \lim_{h\rightarrow 0} \frac{e^h-1}{h} = 1$.

Nous présentons ici un raisonnement qui supporte sans démontrer que le
nombre $e$ défini en (\ref{LIM_EXP}) satisfait
$\displaystyle \lim_{h\rightarrow 0} \frac{e^h-1}{h} = 1$.

Si $\displaystyle \lim_{h\rightarrow 0} \frac{e^h-1}{h} = 1$, il
découle de la définition de limite d'une fonction en un point que
$\displaystyle \frac{e^{x_n}-1}{x_n} \rightarrow 1$ lorsque
$n \rightarrow \infty$ quelle que soit la suite
$\displaystyle \left\{x_n\right\}_{n=1}^\infty$ qui tend vers $0$.  Si
$x_n = \frac{1}{n}$ pour $n=1$, $2$, $3$, \ldots, nous pouvons donc dire que
$\displaystyle 1 \approx \frac{e^{\frac{1}{n}}-1}{\frac{1}{n}}$
pour $n$ très grand.  Si nous résolvons pour $e$, nous trouvons
\[
e \approx \left(1 + \frac{1}{n}\right)^n
\]
pour $n$ très grand.  Le nombre $e$ qui satisfait
$\displaystyle \lim_{h\rightarrow 0} \frac{e^h-1}{h} = 1$ et le nombre
$e$ défini par (\ref{LIM_EXP}) semblent bien être le même nombre.

Il est beaucoup plus facile d'estimer $e$ à l'aide de (\ref{LIM_EXP})
que d'utiliser la formule (\ref{COND_EXP}) pour déterminer la valeur
de $b$ telle que $\displaystyle \lim_{h\rightarrow 0} \frac{b^h-1}{h} = 1$
comme nous l'avons fait au tableau~\ref{TAB_EXP}.
\end{rmk}

\subsubsection{Quelle est la valeur de la constante $\mathbf{C_b}$}

Nous pouvons maintenant donner une réponse complète à la deuxième question
de la section~\ref{SECT_B_EXP}.  À savoir, comment pouvons-nous calculer la
constance $C_b$ dans la relation
\[
\dfdx{b^x}{x} = C_b b^x
\]
sans avoir à calculer une limite comme en (\ref{COND_EXP})?  Soit
$q(x) = b^x$ avec $b>0$.  Puisque
$\displaystyle q(x) = b^x = e^{x\ln(b)}$, 
nous pouvons exprimer $q(x)$ comme la composition $q(x) = f(g(x))$ des
fonctions $g(x) = x \ln(b)$ et $f(y) = e^y$. Comme
$f'(y) = f(y)$ et $g'(x) = \ln(b)$, nous avons
\[
\dfdx{b^x}{x} = q'(x) = f'(g(x)) g'(x) = f(g(x)) g'(x)
= q(x) g'(x) = b^x \ln(b) \;.
\]
Nous obtenons la proposition suivante.

\begin{prop}\label{BexpX}
Si $b>0$, alors
\[
\dfdx{b^x}{x} = b^x \ln(b)
\]
pour tout $x$.
\end{prop}

La constante $C_b$ que nous cherchions est donc $\ln(b)$.

\begin{egg}
Si $f(x) = 5^x$, alors $f'(x) = 5^x \ln(5)$.
\end{egg}

\begin{egg}
Si $f(x) = 5^{x^2}$, la dérivée de $f$ {\em n'est pas}
$5^{x^2}\ln(5)$.  Il faut remarquer que $f(x) = q(p(x))$ où
$q(y) = 5^y$ et $p(x)=x^2$.  Puisque $q'(y) = 5^y \ln(5)$ et $p'(x) = 2 x$,
la règle de la dérivée de fonctions composées donne
\[
f'(x) = q'(p(x)) p'(x) = 5^{x^2}\ln(5) \times 2x 
= 2x\, 5^{x^2} \ln(5)\; .
\]
\end{egg}

\begin{rmk}
Comme nous avons vu à l'exemple précédent, il est facile de se tromper avec
toutes les règles de calcul des dérivées.  Il est souvent plus
avantageux de s'en tenir aux formules les plus importantes.

À l'exemple précédent, nous aurions pu simplement écrire $f(x) = 5^{x^2}$
comme $f(x) = e^{x^2\ln(5)}$.  Il est alors clair que $f$ est la
composition de deux fonctions.  En effet, $f(x) = q(p(x))$ où $q(y) = e^y$
et $p(x) = x^2 \ln(5)$.  Puisque $q'(y) = e^y = q(y)$ et $p'(x) = 2x \ln(5)$,
nous obtenons
\[
f'(x) = q'(p(x))p'(x) = q({p(x)}) ( 2x \ln(5) )
= 5^{x^2} (2x \ln(5)) = 2x\, 5^{x^2}\ln(5) \; .
\]
\end{rmk}

\subsection{Dérivée de $\mathbf{\log_b(x)}$}

Puisque $\log_b(x) = \ln(x) / \ln(b)$ pour tout $x>0$, il suffit de
trouver une formule pour la dérivée de $\ln$.  Pour trouver cette
formule, nous utilisons le fait que $g(x)=\ln(x)$ est la fonction inverse
de $f(x) = e^x$.

Si nous dérivons les deux côtés de l'égalité $f(g(x))= x$ pour $x>0$,
nous trouvons
\[
f'(g(x))g'(x) = 1 \; .
\]
Or $f'(x) = f(x)$ et $f(g(x))=x$, donc
\[
\dfdx{\ln(x)}{x} = g'(x) = \frac{1}{f'(g(x))}
= \frac{1}{f(g(x))} = \frac{1}{x} \; .
\]

Nous venons de démontrer la règle suivante.

\begin{prop}
Pour $x>0$, alors
\[
\dfdx{\ln(x)}{x} = \frac{1}{x} \; .
\]
\end{prop}

Quelle est la dérivée de $\ln |x|$ pour tout $x \neq 0$?  Puisque
$|x|=x$ pour $x>0$, nous avons
\[
\dfdx{\ln|x|}{x} = \dfdx{\ln(x)}{x} = \frac{1}{x} \quad \text{pour}
\quad x>0 \;.
\]
Puisque $|x|=-x$ pour $x<0$, nous avons $\ln|x| = \ln(-x)$ pour $x<0$.
Ainsi, pour $x<0$, $\ln|x| = g(h(x))$ où $g(y) = \ln(y)$ et
$h(x) = -x$.  Puisque $g'(y) = 1/y$ et $h'(x) = -1$, nous obtenons donc
\[
\dfdx{\ln|x|}{x} = g'(h(x)) h'(x) = \frac{1}{h(x)} \ h'(x) =
\left(\frac{1}{-x}\right) (-1) = \frac{1}{x} \quad \text{pour} \quad x< 0 \; .
\]
Nous pouvons donc généraliser le théorème précédent de la façon suivante.

\begin{prop}
Si $x\neq 0$, alors
\[
\dfdx{\ln|x|}{x} = \frac{1}{x} \; .
\]
\end{prop}

\begin{egg}
Quelle est la dérivée de $\log_5(x)$ pour tout $x>0$?

Puisque $\log_5(x) = \ln(x)/\ln(5)$, nous avons
\[
\dfdx{\log_5(x)}{x} = \dfdx{\left(\frac{\ln(x)}{\ln(5)}\right)}{x}
= \frac{1}{\ln(5)} \dfdx{\ln(x)}{x}
= \frac{1}{x\ln(5)} \; .
\]
\end{egg}

\begin{egg}
Quelle est la dérivée de $q(x) = \ln( \sin(x) )$ pour $0 < x < \pi$?

Notez que nous assumons $0<x<\pi$ pour que $\sin(x)$ soit positif et
donc que $\ln(\sin(x))$ soit bien définie.

Puisque $q = f\circ g$ où $g(x) = \sin(x)$ et $f(y) = \ln(y)$, nous
obtenons de la règle pour la dérivée de fonctions composées que
\[
q'(x) = f'(g(x)) g'(x) = \frac{1}{\sin(x)} \, \cos(x)
= \cot(x)
\]
pour $0 < x < \pi$.
\end{egg}

\begin{egg}
Calculons la dérivée de $f(x) = (x+1)^5(x+5)^7/(x+2)^4$.

Il serait tentant d'utiliser la règle de la dérivée du quotient de
deux fonctions mais cela risque d'être très long.

En calcul numérique, nous utilisons très souvent la formule
$\ln(ab) = \ln(a)+\ln(b)$ pour remplacer le produit de très
grands nombres par la somme de leurs logarithmes.  Cela permet
d'éviter les nombres trop grands pour l'ordinateur.
La même idée peut être utilisée ici.  Nous avons
\[
\ln|f(x)| = 5\ln|x+1| + 7\ln|x+5| - 4\ln|x+2|
\]
Si nous dérivons des deux côtés de l'égalité, nous trouvons
\[
\frac{f'(x)}{f(x)} = \frac{5}{x+1} + \frac{7}{x+5} - \frac{4}{x+2}
\]
et ainsi
\[
f'(x) = f(x) \left(\frac{5}{x+1} + \frac{7}{x+5} - \frac{4}{x+2} \right)
= \frac{(x+1)^5(x+5)^7}{(x+2)^4}
\left(\frac{5}{x+1} + \frac{7}{x+5} - \frac{4}{x+2} \right) \; .
\]
Calculez la dérivée de $f$ avec la règle de la dérivée du quotient de
deux fonctions pour comparer cette méthode avec celle que nous venons
d'introduire.
\end{egg}

\begin{egg}
Calculons la dérivée de
$\displaystyle g(x) = \frac{(x^4+2)e^{2x}}{(x^2+3x)\sqrt{x^2+3}}$.

Nous pourrions adresser ce problème directement avec la règle de
dérivation du quotient de deux fonctions mais, comme la fonction $g$
est assez complexe, ce ne serait pas une bonne idée.  Nous invitons le
lecteur à essayer.

Nous allons faire appel à l'ensemble de nos connaissances des méthodes de
dérivation.  Commençons par calculer le logarithme de $g$.  Nous
obtenons
\begin{align*}
\ln|g(x)| &= \ln\left|\frac{(x^4+2)e^{2x}}{(x^2+3x)\sqrt{x^2+3}}\right|
= \ln\left(\frac{(x^4+2)e^{2x}}{|x^2+3x|\sqrt{x^2+3}}\right) \\
& = \ln(x^4+2) + \ln(e^{2x}) - \ln|x^2+3x| - \ln\left((x^2+3)^{1/2}\right) \\
& = \ln(x^4+2) + 2x - \ln|x^2+3x| -\frac{1}{2} \ln(x^2+3)
\end{align*}
Nous dérivons maintenant des deux côtés
\[
\dfdx{\ln|g(x)|}{x} = \dfdx{\ln(x^4+2)}{x} + \dfdx{(2x)}{x} -
\dfdx{\ln|x^2+3x|}{x} -\frac{1}{2} \dfdx{\ln(x^2+3)}{x}
\]
pour obtenir
\[
\frac{g'(x)}{g(x)} = \frac{4x^3}{x^4+2} + 2 -
\frac{2x + 3}{x^2+3x} - \frac{x}{x^2+3} \ .
\]
Ainsi,
\begin{align*}
g'(x) &= \left(\frac{4x^3}{x^4+2} + 2 -
\frac{2x + 3}{x^2+3x} - \frac{x}{x^2+3}\right) g(x) \\
&= \left(\frac{4x^3}{x^4+2} + 2 -
\frac{2x + 3}{x^2+3x} - \frac{x}{x^2+3}\right)
\left(\frac{(x^4+2)e^{2x}}{(x^2+3x)\sqrt{x^2+3}}\right)
\end{align*}
\end{egg}

\subsection{Dérivée de $\mathbf{x^\alpha}$ où $\mathbf{\alpha}$ est réel}

Nous avons vu que la dérivée de $h(x) = x^n$ où $n$ est un entier positive
est $h'(x) = nx^{n-1}$.  Mais, quelle est la dérivée de
$h(x) = x^{\alpha}$ si $\alpha$ n'est pas un entier positif.  Par
exemple, quelle est la dérivée de $\sqrt{x} = x^{1/2}$ ou de $x^\pi$?
Commençons par un exemple simple.

\begin{egg}
Soit $f(x)= x^{-1} = 1/x$ pour $x\neq 0$.  Si nous estimons $f'(x)$ pour
quelques valeurs de $x$ à l'aide du rapport
$\displaystyle \frac{f(x+h)-f(x)}{h}$ où
$h = 0.00001$, nous obtenons les résultats suivants.
\[
\begin{array}{l|l|l}
\hline
\rule[-0.7em]{0ex}{2.5em} x &
f'(x) \approx \displaystyle \frac{f(x+0.00001)-f(x)}{0.00001}
 & |-x^{-2} - f'(x) | \\[0.8em]
\hline
-1 & -1.00001000009\ldots & 0.1000009\ldots \times 10^{-4} \\
 1 & -0.99999000010\ldots & 0.999989\ldots \times 10^{-5} \\
 2 & -0.24999875000\ldots & -0.124999\ldots \times 10^{-5} \\
\pi & -0.10132086112\ldots & -0.32251\ldots \times 10^{-6} \\
3.5 & -0.08163241982\ldots & -0.23323\ldots \times 10^{-6} \\
\hline
\end{array}
\]
Remarquons que $f'(x) \approx -x^{-2}$ pour les valeurs de $x$
considérées dans le tableau.  Nous pouvons donc supposer que
$f'(x) = -x^{-2}$ pour tout $x\neq 0$.

Montrons à partir de la définition de la dérivée que
$f'(x) = -x^{-2}$ pour tout $x\neq 0$.  En effet, si $x\neq 0$, alors
\begin{align*}
f'(x) &= \lim_{h\to 0} \frac{f(x+h)-f(x)}{h}
= \lim_{h\to 0} \frac{1/(x+h) - 1/x}{h} \\
&= \lim_{h\to 0} \frac{-h}{h(x+h)x}
= \lim_{h\to 0} \frac{-1}{x^2 + x h} = -\frac{1}{x^2}
\end{align*}
car $x^2 + x h \to x^2$ lorsque $h \to 0$.
\label{XminusONE}
\end{egg}

L'exemple précédent semble indiquer que la règle qui dit que la
dérivée de $f(x) = x^n$ est $f'(x) = nx^{n-1}$ pourrait s'appliquer
pour tous les nombres entiers $n$.

\begin{egg}
Calculons la dérivée de $f(x) = \sqrt{x}$ à partir de la définition.
\begin{align*}
f'(x) &= \lim_{h\to 0} \frac{f(x+h)-f(x)}{h}
= \lim_{h\to 0} \frac{\sqrt{x+h}-\sqrt{x}}{h} \\
&= \lim_{h\to 0} \left(\frac{\sqrt{x+h}-\sqrt{x}}{h}\right)
\underbrace{\left(\frac{\sqrt{x+h}+\sqrt{x}}{\sqrt{x+h}+\sqrt{x}}\right)}_{=1}
= \lim_{h\to 0} \frac{h}{h(\sqrt{x+h}+\sqrt{x})} \\
&= \lim_{h\to 0} \frac{1}{(\sqrt{x+h}+\sqrt{x})} = \frac{1}{2\sqrt{x}} \ .
\end{align*}
\end{egg}

Encore une fois, l'exemple précédent semble indiquer que la règle qui
dit que la dérivée de $f(x) = x^n$ est $f'(x) = nx^{n-1}$ pourrait
s'appliquer pour tous les nombres rationnels $n$.  En fait,
nous avons plus.

Pour montrer que la règle est vraie pour tout exposant réel non nul,
il faut se rappeler que $\displaystyle x^\alpha = e^{\alpha \ln(x)}$
pour $x>0$.  Ainsi, $x^\alpha = f(g(x))$ où $g(x) = \alpha \ln(x)$ et
$f(y) = e^y$.   Puisque $f'(y) = f(y)$ et $g'(x) = \alpha / x$,
nous déduisons de la règle de la dérivée de fonctions composées que
\[
\dfdx{x^\alpha}{x} = f'(g(x)) g'(x)
= f(g(x)) g'(x) = x^{\alpha} \left(\frac{\alpha}{x}\right)
= \alpha x^{\alpha-1}
\]
pour $x>0$.  Nous venons de démontrer la règle suivante.

\begin{prop}
Si $\alpha \neq 0$ est un nombre réel, alors
\[
\dfdx{x^\alpha}{x} = \alpha x^{\alpha-1}
\]
pour tout $x>0$.
\end{prop}

\begin{egg}
Si $f(x) = x^{-1/2}$, alors
$\displaystyle f'(x) = -\frac{1}{2} x^{-3/2} = \frac{-1}{2 x^{3/2}}$
pour $x>0$.
\end{egg}

Avant de terminer cette section, il y a un exemple important que nous
nous devons de présenter.

\begin{egg}
Que devons-nous faire pour calculer la dérivée d'une fonction comme
$f(x) = x^x$?  La variable $x$ apparaît à la base et à l'exposant.
La règle ci-dessus pour dérivée $x^\alpha$ ne s'applique pas car
l'exposant n'est pas une constante.  La règle donnée à la
proposition~\ref{BexpX} pour dérivée $b^x$ ne s'applique pas car
c'est la base qui n'est pas une constante.

Nous devons utiliser les fonctions exponentielles et logarithmiques.

Puisque
\[
f(x) = x^x = e^{\ln((x^x)}  = e^{x\ln(x)} \ ,
\]
il découle de la règle de dérivation des fonctions composées que
\begin{align*}
f'(x) &= \left(\dfdx{e^y}{y} \bigg|_{y=x\ln(x)} \right)
\dfdx{\left( x\ln(x)\right)}{x}
= \left( e^y \bigg|_{x\ln(x)} \right) ( \ln(x) + 1 ) \\
&= e^{x\ln(x)} ( \ln(x) + 1)
= x^x( \ln(x) + 1) \ .
\end{align*}
\end{egg}

\subsection{Dérivée des fonction trigonométriques inverses \life \eng}

Si nous dérivons les deux côtés de l'identité $x = \sin (\arcsin(x))$ pour
$-1 < x < 1$, nous obtenons grâce à la dérivée de fonctions composées que
\[
1 = \cos(\arcsin(x)) \dfdx{\arcsin(x)}{x} \; .
\]
Donc
\[
\dfdx{\arcsin(x)}{x} = \frac{1}{\cos(\arcsin(x))} \; .
\]

Posons $\theta = \arcsin(x)$.  Donc $\sin(\theta) = x$ par définition
de l'arcsinus et nous obtenons la figure suivante par définition des
fonctions trigonométriques à partir d'un triangle droit.
\PDFgraph{5_derivees/invtrig1}
Nous en déduisons que
\[
\cos(\arcsin(x)) = \cos(\theta) = \sqrt{1-x^2} \; .
\]
Nous obtenons donc la formule suivante.

\begin{prop}
\[
\dfdx{\arcsin(x)}{x} = \frac{1}{\sqrt{1-x^2}}
\]
pour $-1<x<1$.
\end{prop}

De façon semblable, le lecteur peut démontrer la proposition suivante.

\begin{prop}
\[
\dfdx{\arccos(x)}{x} = \frac{-1}{\sqrt{1-x^2}}
\]
pour $-1<x<1$.
\end{prop}

Si maintenant nous dérivons les deux côtés de l'identité
$x = \tan (\arctan(x))$
où $x$ est réel, nous obtenons grâce à la dérivée de fonctions composées que
\[
1 = \sec^2(\arctan(x)) \dfdx{\arctan(x)}{x}\; .
\]
Donc
\[
\dfdx{\arctan(x)}{x} = \frac{1}{\sec^2(\arctan(x))} \; .
\]

Posons $\theta = \arctan(x)$.  Donc $\tan(\theta) = x$ par définition
de l'arctangente et nous obtenons la figure suivante par définition des
fonctions trigonométriques à partir d'un triangle droit.
\PDFgraph{5_derivees/invtrig2}
Nous en déduisons que
\[
\sec(\arctan(x)) = \sec(\theta) = \sqrt{1+x^2} \ .
\]
Nous obtenons donc la formule suivante.

\begin{prop}
\[
\dfdx{\arctan(x)}{x} = \frac{1}{1+x^2}
\]
pour $x\in \RR$.
\end{prop}

}  % End of theory

\section{Exercices}

\subsection{Taux de variation}

\begin{question}
Pour chacune des fonctions suivantes et des points $t_0$ suivants:

\subQ{i} Calculez la moyenne de la fonction $f$ entre les points $t_0$
et $t_0+\Delta t$ pour $\Delta t = 1$, $0.5$, $0.1$ et
$0.01$.\\
\subQ{ii} Donnez l'équation de la sécante qui passe par les points
$(t_0, f(t_0))$ et $(t_0 + \Delta t, f(t_0+\Delta t))$
pour $\Delta t = 1$, $0.5$, $0.1$ et $0.01$.\\
\subQ{iii} Dessiner, sur une même figure, le graphe de la fonction
avec ses quatre sécantes.\\
\subQ{iv} Donner, en vous basant sur les valeurs calculées en
(i), une approximation de la pente de la droite tangente à la
courbe $y=f(t)$ au point $(t_0,f(t_0))$.\\
\subQ{v} Donnez l'équation de la droite tangente à la courbe $y=f(t)$ au
point $(t_0, f(t_0))$ en vous basant sur votre réponse en (iv).

\begin{center}
\begin{tabular}{*{1}{l@{\hspace{0.5em}}l@{\hspace{3em}}}l@{\hspace{0.5em}}l}
\subQ{a} & $f(t) = 2 t^2$ et $t_0 = 1$. &
\subQ{b} & $f(t) = e^{2t}$ et $t_0 = 0$.
\end{tabular}
\end{center}
\label{5Q1}
\end{question}

\begin{question}
Supposons que le nombre d'individus au temps $t$ (en heures) pour une
certaine population animale soit donnée par $p(t) = 1.5^t$.

\subQ{a} Quelle est le taux de croissance moyen entre $0$ et $1$?\\
\subQ{b} Quelle est le taux de croissance moyen entre $0$ et $0.1$?\\
\subQ{c} Quelle est le taux de croissance moyen entre $0$ et $0.01$?\\
\subQ{d} Quelle est le taux de croissance moyen entre $0$ et $0.0001$?\\
\subQ{e} Quelle est le taux de croissance instantané à $t=0$?\\
\subQ{f} Donnez l'équation de la droite tangente à la courbe $y=p(t)$ au point
  $(0,p(0))$.
\label{5Q2}
\end{question}

\begin{question}[\life]
Les données du tableau suivant décrivent la hauteur $H$ (en mètres) d'un
arbre en fonction du temps $t$ (en années).
\[
\begin{array}{c|ccccccccccc}
t & 0 & 1 & 2 & 3 & 4 & 5 & 6 & 7 & 8 & 9 & 10 \\
\hline
H & 10.11 & 11.18 & 12.40 & 13.74 & 15.01 & 16.61 & 18.27 &
20.17 & 22.01 & 24.45 & 26.85
\end{array}
\]
\subQ{a} Estimez le taux de croissance instantané à $t=0$, $1$, $2$,
\ldots, $10$ ans à l'aide de la moyenne des taux de croissance que
vous pouvez estimer à partir des données de l'année qui précède et de
l'année qui suit immédiatement (si possible).\\
\subQ{b} Tracez sur un graphe les points qui représentent le taux de
croissance instantané pour chacune des années en (a).  Est-ce que les
points semblent tracer une courbe quelconque?\\
\subQ{c} Estimez le taux de croissance relatif à $t=0$, $1$, $2$,
\ldots, $10$ ans.\\
\subQ{d} Tracez sur un graphe les points qui représente le taux de
croissance relatif pour chacune des années en (c).  Est-ce que les
points semblent tracer une courbe quelconque?\\
\subQ{e} Comparez vos deux graphes.  Que pouvez-vous conclure?
\label{5Q3}
\end{question}

\begin{question}[\life]
Le nombre d'individus dans une population est donné par la formule
$p(t) = 2^t$ où $t$ est le temps en heures. Trouvez le taux de
croissance moyen entre $0$ et $1$ heure, $0$ et $0.1$ heure, $0$ et
$0.01$ heure, et $0$ et $0.001$ heure.

Quelle est la limite?  Que représente cette limite?  Tracez le graphe
de la fonction $p$ et la droite tangent à ce graphe au point
$(t,p) = (0,1)$.  Donnez l'équation de cette droite tangente.
\label{5Q4}
\end{question}

\begin{question}
Le nombre $N$ de visites au nouveau site Internet du ministère de
l'environnement est donné dans le tableau suivant.
\begin{center}
\begin{tabular}{c|c|c|c|c|c}
nombre d'heures après l'inauguration du site & 1 & 2 & 3 & 4 & 5 \\
\hline
$N$ & 30 & 57 & 87 & 151 & 246
\end{tabular}
\end{center}

\subQ{a} Trouvez le taux moyen de croissance des visites de la
deuxième à la troisième heure, de la troisième à la quatrième heure,
et de la troisième à la cinquième heure.  Donnez les unités de vos
réponses.\\
\subQ{b} Donnez une approximation du taux instantané de croissance des
visite après trois heures en prenant la moyenne de deux taux de
croissance moyens. Donnez les unités de votre réponse.\\
\subQ{c} Est-ce qu'il est réaliste d'estimer le taux de croissance
instantané des visites après trois heures sur la base des données que
nous possédons?
\label{5Q5}
\end{question}

\subsection{Dérivée d'une fonction}

\begin{question}
Le graphe de la fonction $f$ est donné ci-dessous.
\PDFgraph{5_derivees/cont1}
\subQ{a} Identifiez les points où la fonction n'est pas continue.\\
\subQ{b} Identifiez les points où la fonction n'est pas différentiable
et dite pourquoi.\\
\subQ{c} Identifiez les points où la dérivée de la fonction est nulle.
\label{5Q6}
\end{question}

\begin{question}
Le graphe de la fonction $f$ est donné ci-dessous.
\PDFgraph{5_derivees/der27}
\subQ{a} Identifiez les points où la dérivée est positive.\\
\subQ{b} Identifiez les points où la dérivée est négative.\\
\subQ{c} Identifiez les points où la dérivée est nulle.
\label{5Q7}
\end{question}

\begin{question}
Un chien court après un cycliste.  Tracez sur un même système de
coordonnées la position du chien et du cycliste en fonction du temps
pour chacun des scénarios suivants.

\subQ{a} Le chien et le cycliste se déplacent à vitesse constante mais
le chien se déplace plus rapidement que le cycliste et, après un
certain temps, rattrape le cycliste.\\
\subQ{b} La vitesse du chien augmente et celle du cycliste diminue.
Le chien rattrape le cycliste.\\
\subQ{c} La vitesse du chien et du cycliste diminue, et la distance
entre les deux augment.
\label{5Q8}
\end{question}

\begin{question}
Une voiture est remorquée par une dépanneuse à l'aide d'une tige
rigide de $10$ mètres de long.  Tracez les graphes possibles de la
position et de la vitesse des deux véhicules en fonction du temps dans
chacun des cas suivants.

\subQ{a} La dépanneuse part du repos, recule lentement pour une courte
période de temps, arrête, avance lentement pour une autre courte
période de temps et finalement avance plus rapidement.\\
\subQ{b} La dépanneuse est initialement au repos, accélère lentement,
garde une vitesse constante pour un certain temps, et finalement
arrête brusquement.
\label{5Q9}
\end{question}

\begin{question}
La figure ci-dessous contient le graphe d'une fonction et le graphe de la
dérivée de cette fonction.  Quelle est le graphe de la dérivée?
\PDFgraph{5_derivees/der28}
\label{5Q10}
\end{question}

\begin{question}
Considérons les graphes de deux fonctions donnés dans la figure ci-dessous.
\PDFgraph{5_derivees/ass3D}
Identifiez la courbe qui représente le graphe de la dérivée d'un
fonction dont le graphe est donné par l'autre courbe.
\label{5Q11}
\end{question}

\begin{question}
Trois des graphes ci-dessous représentent les graphes de fonctions et
les trois autres graphes représentent les graphes des dérivées de ces
fonctions.  Donner les pairs $(n_1, n_2)$ où $n_1$ est le numéro d'un
graphe associé au graphe d'une fonction et $n_2$ est le numéro du
graphe associé au graphe de la dérivée de cette fonction.
{\bfseries Vous ne pouvez pas utiliser une image plus d'une fois}.
\MATHgraph{5_derivees/graphs}{10cm}
\label{5Q12}
\end{question}

\begin{question}
La température en fonction du temps d'une substance chimique est
donnée par le graphe suivant.
\PDFgraph{5_derivees/der29}
Tracez le graphe du taux de variation instantané de la température en
fonction du temps.  Sur le graphe que vous avez dessiner, indiquer
lorsque la température augmente et lorsqu'elle diminue.
\label{5Q13}
\end{question}

\begin{question}
Le graphe du volume $V(t)$ en mètres cubes d'un objet en fonction du
temps $t$ en heures est donné ci-dessous.
\PDFgraph{5_derivees/ass3C}
\subQ{a} Identifiez un point $t$ où la dérivée est positive.\\
\subQ{b} Identifiez un point $t$ où la dérivée est négative.\\
\subQ{c} Identifiez le point $t$ avec la plus grande dérivée.\\
\subQ{d} Identifiez le point $t$ avec la plus petite dérivée.\\
\subQ{e} Identifiez les points $t$ où la dérivée est nulle.\\
\subQ{f} Tracez le graphe de la dérivée $V'$ de $V$.
\label{5Q14}
\end{question}

\begin{question}
Le graphe d'une fonction $f$ est donné ci-dessous.
\PDFgraph{5_derivees/ass3B}
Donnez les points $x$ qui satisfont chacun des cas suivants.

\subQ{a} La fonction $f$ n'est pas continue.\\
\subQ{b} La fonction $f$ n'est pas différentiable et dite pourquoi.\\
\subQ{c} La dérivée $f'(x)$ est nulle.
\label{5Q15}
\end{question}

\begin{question}
Utilisez la définition de la dérivée pour calculer la dérivée de
$f(x) = 4 - x^2$.  Tracez le graphe de $f$ et $f'$.  Trouvez les
points où la dérivée de $f$ est nulle.  Déterminez les intervalles où
la fonction est strictement croissante et ceux où elle est strictement
décroissante.
\label{5Q16}
\end{question}

\begin{question}
Soit $g(x) = x+2x^2$ une fonction quadratique.  Donnez la pente
de la sécante à la courbe $y=g(x)$ entre les points $x$ et $x+h$.
Donnez la pente de la tangente à la courbe $y=g(x)$ au point $x$ en
passant à la limite lorsque $h$ approche $0$.  Donnez la dérivée de
$g$ au point $x$.
\label{5Q17}
\end{question}

\begin{question}
Utilisez la définition de la dérivée pour calculer la dérivée de
chacune des fonctions suivantes.
\begin{center}
\begin{tabular}{*{2}{l@{\hspace{0.5em}}l@{\hspace{3em}}}l@{\hspace{0.5em}}l}
\subQ{a} & $f(x) = (x+1)^2$ & \subQ{b}
& $\displaystyle f(x) = \frac{1}{2x+5}$ & \subQ{c} & $f(x) = x^2+x$ \\[0.9em]
\subQ{d} & $\displaystyle f(x) = \frac{x}{3x-1}$ &
\subQ{e} & $\displaystyle f(x) = \frac{1}{x(x+1)}$ & &
\end{tabular}
\end{center}
\label{5Q18}
\end{question}

\begin{question}[\eng]
La fonction de Heaviside est définie par
\[
H(x) =
\begin{cases}
1 & \quad \text{si} \quad x \geq 0 \\
0 & \quad \text{si} \quad x < 0
\end{cases} \ .
\]
Montrez qu'il n'existe pas de solution à l'équation $H(x) = 0.5$ même si
$H(-1)=0$ et $H(1)=1$.  Ceci démontre que le Théorème des valeurs
intermédiaires est valable seulement pour les fonctions continues.  

Montrez qu'il n'existe pas de valeur $c$ telle que $H'(c)$ soit égale à la
pente $\displaystyle \frac{H(1)-H(-1)}{2}$ de la sécante entre les points
$(-1,H(-1))$ et $(1,H(1))$.  Ceci démontre que le Théorème de la moyenne est
valable seulement pour les fonctions qui sont différentiable.
\label{5Q19}
\end{question}

\begin{question}[\life]
Un animal passe de $4$ kg à $60$ kg en $14$ ans.  Si nous supposons
que la masse de l'animal est une fonction différentiable par rapport
au temps, pourquoi pouvons-nous dire que le taux de croissance
(instantané) a été de $4$ kg/année à un moment durant les $14$ années?
\label{5Q20}
\end{question}

\subsection{Calcul des dérivées}

\begin{question}
Calculez la dérivée de $g(y) = -3 y + 5$ et déterminez les
intervalles, s'il y en a, où la fonction est strictement croissante.
\label{5Q21}
\end{question}

\begin{question}
Une quantité $T(t)$ au temps $t$ est donnée par le produit de deux
quantités: $p(t) = 2\times 10^6 + 10^3 t^2$ et $M(t) = 80 - 0.5 t$.

\subQ{a} Donnez $T$ en fonction du temps.\\
\subQ{b} Calculez la dérivée de $T$ en fonction du temps.\\
\subQ{c} Quelle est la valeur de $T$ au moment où $T'(t) = 0$?
Quelle est la valeur de $p$ à ce moment?  Quelle est la valeur de $M$
à ce moment?
\label{5Q22}
\end{question}

\begin{question}
Un train se déplace à $110$ km/h.  Un passager de ce train se déplace
à un vitesse de $3$ km/h vers l'arrière du train.  Quelle est la
vitesse de ce passager par rapport au sol?
\label{5Q23}
\end{question}

\begin{question}
Pour chacune des questions suivantes, Calculez la dérivée à l'aide de
la formule de dérivation des fonctions composées.

\subQ{a} Soit $T(W) = 30 -0.2W$, $S(T) = 9 -T/5$ et $H(W) = S(T(W))$.
Calculez $\displaystyle \dydx{H}{W}$.\\
\subQ{b} Soit $L(T) = 10+T/10$, $V(L) = 2L^3$, $M(V) = 1.3V$
$H(T) = M(V((L(T))))$.  Calculez $\displaystyle \dydx{H}{T}$.\\
\subQ{c} Soit $M(G) = 5G+2$, $P(M) = 0.5 M$ et $H(G)=P(M(G))$.
Calculer $\displaystyle \dydx{H}{G}$.\\
\subQ{d} Soit $V(I) = 5 I^2$, $F(V) = 37 + 0.4 V$ et $H(I) = F(V(I))$.
Calculer $\displaystyle \dydx{F}{I}$.  
\label{5Q24}
\end{question}

\begin{question}
Soit $B=0.007 W^{2/3}$ et $W=0.12 L^{2.53}$.  De plus, supposons que
$L$ soit une fonction de $t$ et que $L'(t)$ soit constant.  Si nous
savons que $L$ augmente de $5$ lorsque $t$ augmente de $10$, quelle
est la dérivée de $H(t) = B(W(L(t)))$ par rapport à $t$ lorsque $L = 18$?
\label{5Q25}
\end{question}

\begin{question}
Calculez la dérivée de chacune des fonctions suivantes.
\begin{center}
\begin{tabular}{*{2}{l@{\hspace{0.4em}}l@{\hspace{2.7em}}}l@{\hspace{0.4em}}l}
\subQ{a} & $\displaystyle f(x) = x^{1/5}$ &
\subQ{b} & $\displaystyle h(t) = x^{1/e}$ &
\subQ{c} & $\displaystyle g(z) = 3z^3 + 2z^2$ \\[0.7em]
\subQ{d} & $\displaystyle p(z) = (1+3z)^2(1+2z)^3$ &
\subQ{e} & $\displaystyle f(x) = (2x+1)^3$ &
\subQ{f} & $\displaystyle h(x) = (x^2-5)^{135/2}$ \\
\subQ{g} & $\displaystyle G(x) = \frac{(1+x)(2+x)}{(3+x)}$ &
\subQ{h} & $\displaystyle f(t) = \frac{\sqrt{t}}{1+\sqrt{t}}$ &
\subQ{i} & $\displaystyle f(x) = \log_2(1-3x)$ \\[0.7em]
\subQ{j} & $\displaystyle f(y) = (5y-3)^7(y^2-1)$ &
\subQ{k} & $\displaystyle h(t) = \frac{1+t}{2-t}$ &
\subQ{l} & $\displaystyle g(z) = \frac{1+z^2}{1+2z^3}$ \\[0.7em]
\subQ{m} & $\displaystyle f(x) = \frac{\ln(2x+ e^x)}{2x+e^x}$ &
\subQ{n} & $\displaystyle h(x) = \frac{(1+3x)^2}{(1+2x)^3}$ &
\subQ{o} & $\displaystyle f(x) = e^{-7x}$ \\[1em]
\subQ{p} & $\displaystyle g(z) = \left(1 + \frac{2}{1+z}\right)^7$ &
\subQ{q} & $\displaystyle f(t) = (1+3t)^{33}$ &
\subQ{r} & $\displaystyle h(x) = \ln|\ln(x)|$ \\[1em]
\subQ{s} & $\displaystyle g(t) = \ln\left(\frac{t^2}{(t-2)^3}\right)$ &
\subQ{t} & $\displaystyle f(x) = x^{\ln(x)}$ &
\subQ{u} & $\displaystyle f(x) = x^{x^2}$ \\[1em]
\subQ{v} & $\displaystyle g(t) = \ln\left(\frac{t^7(t^2-5)^8}{(t-2)^5}\right)$ &
 &  & &
\end{tabular}
\end{center}
\label{5Q26}
\end{question}

\begin{question}[\eng \life]
Évaluez les dérivées suivantes.
\begin{center}
\begin{tabular}{*{1}{l@{\hspace{0.5em}}l@{\hspace{6em}}}l@{\hspace{0.5em}}l}
\subQ{a} & $\displaystyle h(\theta) =\sec(\theta)\tan(\theta)$ &
\subQ{b} & $\displaystyle G(t) = 1 + 2
\cos\left( \frac{2\pi(t-3)}{5} \right)$ \\[1em]
\subQ{c} & $\displaystyle f(x) = \ln(4+\sin(x))$ &
\subQ{d} & $\displaystyle g(\theta) = \sqrt{2\theta\, \sin(\theta)}$ \\
\subQ{e} & $\displaystyle F(\theta) = \tan^2(\sin(\theta))$ &
\subQ{f} & $\displaystyle g(x) = x \arctan(x^2)$\\
\subQ{g} & $\displaystyle h(\theta) = \sin(\theta)\cos(\theta)$ &
\subQ{h} & $\displaystyle h(x) = 3 + \cos(2x-1)$ \\
\subQ{i} & $\displaystyle g(z) = e^{\cos(z)}$ &
\subQ{j} & $\displaystyle f(\theta) = \sec(\theta)$ \\
\subQ{k} & $\displaystyle f(x)=e^{\sin(x^{100}+1)}$ &
\subQ{l} & $\displaystyle g(x) = x^{\sin(x)}$ \\
\subQ{m} & $\displaystyle f(t) = \ln\left(\frac{2t-\cos(6t)}{t}\right)$  &
         &
\end{tabular}
\end{center}
\label{5Q27}
\end{question}

\begin{question}
Calculez la dérivée de $\displaystyle f(x) = \frac{1}{1+e^x}$ de deux
façons:

\subI{i} Avec la règle de la dérivée d'un quotient.\\
\subI{ii} Avec la règle de la dérivée de fonctions composées.
\label{5Q28}
\end{question}

\begin{question}
Calculez la dérivée de $\displaystyle f(x) = \ln(7x)$ de deux façons:

\subI{i} Avec une des identités satisfaites par $\ln()$.\\
\subI{ii} Avec la règle de la dérivée de fonctions composées.
\label{5Q29}
\end{question}

\begin{question}
Calculez la dérivée de $\displaystyle f(x) = 7^x$ de deux façons:

\subI{i} Avec la règle introduite en classe pour dériver ce type
de fonctions.\\
\subI{ii} Avec le fait que $7^x$ peut s'exprimer sous la forme
$e^{h(x)}$ pour une certaine fonction $h$.
\label{5Q30}
\end{question}

\begin{question}[\eng \life]
Calculez la dérivée de $\displaystyle f(\theta) = \cos(2\theta)$ de deux
façons:

\subI{i} Avec les formules d'addition pour le cosinus et le sinus.\\
\subI{ii} Avec la règle de la dérivée de fonctions composées.

\noindent Montrez que vos deux réponses sont égales.
\label{5Q31}
\end{question}

\begin{question}
Un tableau contenant certaines valeurs de $f$, $f'$, $g$ et $g'$ est
donné ci-dessous.
\begin{center}
\begin{tabular}{c|c|c|c|c|c}
$x$ & $f(x)$ & $g(x)$ & $f'(x)$ & $g'(x)$ \\
\hline
1 & 3 & 2 & 4 & 6 \\
2 & 1 & 8 & 5 & 7 \\
3 & 7 & 2 & 7 & 9
\end{tabular}
\end{center}
\begin{center}
\begin{tabular}{*{1}{l@{\hspace{0.5em}}l@{\hspace{3em}}}l@{\hspace{0.5em}}l}
\subQ{a} & Si $h(x) = f(g(x))$, calculez $h'(1)$. &
\subQ{b} & Si $h(x) = g(f(x))$, calculez $h'(1)$. \\
\subQ{c} & Si $\displaystyle h(x) = \frac{f(x)}{g(x)}$, calculez $h'(1)$. & 
\subQ{d} & Si $h(x) = f^{-1}(x)$, calculez $h'(1)$
\end{tabular}
\end{center}
\noindent Suggestion pour (d): $f(f^{-1}(x))=x$.  Nous assumons que $f$
est injective.
\label{5Q32}
\end{question}

\begin{question}
Soit $f$ une fonction strictement croissante positive.  Une fonction
$f$ est positive si $f(x)>0$ pour tout $x$.  Montrez que la fonction
$g$ définie par $g(x) = 1/f(x)$ pour tout $x$ est strictement
décroissante.
\label{5Q33}
\end{question}

\begin{question}
Pour chacune des fonctions suivantes, calculez la dérivée de la fonction
inverse $f^{-1}$ en procédant de deux façons:

\subI{i} Trouvez la fonction inverse et dérivez cette fonction.\\
\subI{ii} Calculez la dérivée de la fonction composée
$(f\circ f^{-1})(x) = f(f^{-1}(x)) = x$.

\begin{center}
\begin{tabular}{*{1}{l@{\hspace{0.5em}}l@{\hspace{6em}}}l@{\hspace{0.5em}}l}
\subQ{a} & $\displaystyle f(x) = 2 + x^3$ &
\subQ{b} & $f(x) = 1 - e^{-x}$
\end{tabular}
\end{center}
\label{5Q34}
\end{question}

\begin{question}[\eng \life]
Pour chacune des courbes ci-dessous, trouvez l'équation de la droite
tangente à la courbe au point donné.
\begin{center}
\begin{tabular}{*{1}{l@{\hspace{0.5em}}l@{\hspace{3em}}}l@{\hspace{0.5em}}l}
\subQ{a} & $y = \ln(x^3-7)$ au point $(2,0)$ &
\subQ{b} & $y=\sin(\sin(x))$ au point $(x,y) = (\pi,0)$.
\end{tabular}
\end{center}
\label{5Q35}
\end{question}

\begin{question}
\subQ{a} Trouvez la pente de la droite tangente à la courbe
$y = f(x) = x^3 + 8$ au point où cette courbe coupe l'axe des $x$.\\
\subQ{b} Trouvez l'équation de la droite tangente à la courbe
$y = f(x)$ au point d'intersection trouvé en (a).\\
\subQ{c} Trouvez l'équation de la droite perpendiculaire à la droite
tangente en (b) qui passe par le point d'intersection en (a).
\label{5Q36}
\end{question}

\begin{question}[\eng \life]
Pour quelle valeur de $x$ avons-nous que la tangente à la courbe 
$y = f(x) = \sqrt{3}\,x + 2\sin(x)$ au point $(x,f(x))$ est horizontale?
\label{5Q37}
\end{question}

\begin{question}
L'aire d'un disque en fonction de son rayon est donné par la formule
$A(r) = \pi r^2$, où $r$ est mesuré en mètres et $A$ en mètres
carrés.  Calculez la dérivée de l'aire en fonction du rayon.
Illustrez par un dessin la région représentée par la différence entre
un disque de rayon $r+\Delta r$ et un disque de rayon $r$.  Donnez une
représentation géométrique de la dérivée.  Est-ce que les unités sont
cohérentes.
\label{5Q38}
\end{question}

\begin{question}[\life \eng]
Soit $f(x) = x^2$.  Trouvez le point sur l'intervalle $[0,2]$ où
$\displaystyle f'(x) = \frac{f(2)-f(0)}{2-0}$.  
\label{5Q39}
\end{question}

%%% Local Variables: 
%%% mode: latex
%%% TeX-master: "notes"
%%% End: 

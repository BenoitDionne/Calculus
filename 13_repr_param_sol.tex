\section{Représentations paramétriques des courbes}

\subsection{Droite tangente à une courbe}

\subsection{Longueur d'une courbe}

\compileSOL{\SOLUb}{\ref{13Q2}}{
Pour aller de $(0,1)$ à $(4,9)$ le long de la courbe, il faut que
$0 \leq t \leq 2$.  Donc
\[
  L = \int_a^b \sqrt{  \left(x'(t)\right)^2 + \left(y'(t)\right)^2} \dx{t}
= \int_0^2 \sqrt{ (2t)^2+(3t^2)^2} \dx{t}
= \int_0^2 t\sqrt{4+9t^2}\,\dx{t} \ .
\]
Si $u=4+9t^2$, alors $\dx{u} = 18 t \dx{t}$, $u = 4$
pour $t=0$ et $u = 40$ pour $t=2$.  Donc
\[
L = \frac{1}{18} \int_4^{40} u^{1/2} \dx{u}
= \frac{1}{18} \left( \frac{2}{3}u^{3/2} \right)\bigg|_4^{40}
= 9.07\ldots
\]
}

\compileSOL{\SOLUb}{\ref{13Q4}}{
La longueur de la courbe est donnée par
\[
L = \int_1^3 \sqrt{1 + (y')^2} \dx{x}
= \int_1^3 \sqrt{1 + (3x^{1/2})^2} \dx{x}
= \int_1^3 \sqrt{1 + 9x} \dx{x} \ .
\]
Si $u=1+9x$, alors $\dx{u} = 9 \dx{x}$, $u=28$
lorsque $x=3$ et $u=10$ lorsque $x=1$.  Donc
\[
L = \frac{1}{9} \int_{10}^{28} u^{1/2} \dx{u}
= \frac{1}{9} \left(\frac{2}{3} u^{3/2}\right) \bigg|_{10}^{28}
= \frac{2}{27}\left( 56 \sqrt{7} - 10 \sqrt{10}\right) \ .
\]
}

%%% Local Variables: 
%%% mode: latex
%%% TeX-master: "notes"
%%% End: 

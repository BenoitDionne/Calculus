\section{Suites et séries}

\subsection{Suites}

\compileSOL{\SOLUb}{\ref{3Q1}}{
\subQ{a} Nous avons
\[
\lim_{n\to \infty} \frac{2^n}{3^{n+1}}
  = \lim_{n\to \infty} \frac{1}{3}\left(\frac{2}{3}\right)^n
  = \frac{1}{3} \lim_{n\to \infty} \left(\frac{2}{3}\right)^n = 0 \ .
\]
La deuxième égalité provient du théorème~\ref{suite3} et la troisième
égalité de la proposition~\ref{suite1}.

\subQ{b} Puisque
$\displaystyle \ln(n+3) - \ln(n) = \ln\left(\frac{n+3}{n}\right)
= \ln\left(1 + \frac{3}{n}\right)$, nous avons
\[
\lim_{n\to \infty} \left( \ln(n+3) - \ln(n) \right)
= \lim_{n\to \infty} \ln\left(1 + \frac{3}{n}\right)
= \ln\left(\lim_{n\to \infty} \left(1 + \frac{3}{n}\right)\right)
= \ln(1) = 0 \ .
\]
Nous verrons lors de l'étude des fonctions continues que la deuxième
égalité, lorsque nous passons la limite à l'intérieure de la fonction
$\ln$, est valide.

\subQ{d}
\[
  \left\{ 3 + \sin\left(n\frac{\pi}{2}\right) \right\}_{n=1}^\infty
=  \left\{4, 3, 2, 3, 4, 3, 2, 3, \ldots \right\} \ .
\]
La suite $4, 3, 2, 3$ est répétée un nombre infini de fois.  Donc, le
terme générale de la suite n'approche pas une valeur unique et fixe.
}

\subsection{Séries}

\compileSOL{\SOLUb}{\ref{3Q4}}{
\subQ{a}
Après $n$ années, le nombre d'unités produites dans la première année
et qui sont toujours en usage est de $5000 ( 1 - 0.01)^n$, le nombre
d'unités produites dans la deuxième année et qui sont toujours en
usage est de $5000 ( 1- 0.01)^{n-1}$, \ldots, le nombre d'unités
produites dans l'avant dernière année et qui sont toujours en usage
est de $5000( 1 - 0.01)^2$, et le nombre d'unités produites dans la
dernière année et qui sont toujours en usage est de $5000( 1 - 0.01)$.
Ainsi, après $n$ années, le nombre d'unités produites et qui sont
toujours en usage est
\[
P_n = \sum_{i=1}^n 5000 (1- 0.1)^i = 5000 \sum_{i=1}^n 0.9^i
= 5000 \left(\frac{ 1 - 0.9^{n+1}}{ 1- 0.9}\right) = 50000 (1-0.9^{n+1}) \ .
\]

\subQ{b} Le niveau de stabilisation du marché pour ces cerfs-volants
est de
\[
\lim_{n\rightarrow \infty} 50000 ( 1- 0.9^n) = 50000
\]
unités.
}

\compileSOL{\SOLUb}{\ref{3Q5}}{
\subQ{b} Puisque
\[
  \sum_{n=1}^\infty \frac{2^n+1}{6^n} =
\sum_{n=1}^\infty \left( \left(\frac{1}{3}\right)^n +
  \left(\frac{1}{6}\right)^n \right)
\]
est la somme de deux séries géométriques convergente,
\[
\sum_{n=0}^\infty \left(\frac{1}{3}\right)^n =
\frac{1}{1- 1/3} = \frac{3}{2}
\qquad \text{et} \qquad 
\sum_{n=0}^\infty \left(\frac{1}{6}\right)^n =
\frac{1}{1- 1/6} = \frac{6}{5} \ ,
\]
la série converge et sa somme est
$\displaystyle \frac{3}{2}+\frac{6}{5} = \frac{27}{10}$.

\subQ{c} Nous avons que
\[
\frac{2^n - 3^{n+2}}{5^{n+1}} = \frac{2^n}{5^{n+1}}
- \frac{3^{n+2}}{5^{n+1}}
= \frac{1}{5} \left(\frac{2}{5}\right)^n
- \frac{3^2}{5} \left(\frac{3}{5}\right)^n \ .
\]
De plus, $\displaystyle \sum_{n=0}^\infty \left(\frac{2}{5}\right)^n$ et
$\displaystyle \sum_{n=0}^\infty \left(\frac{3}{5}\right)^n$ sont deux
séries géométriques convergentes.  La première séries est de raison
$\displaystyle r= \frac{2}{5}$ et converge vers
$\displaystyle \frac{1}{1-2/5} = \frac{5}{3}$.
La seconde série est de raison $\displaystyle r= \frac{3}{5}$ et
converge vers $\displaystyle \frac{1}{1-3/5} = \frac{5}{2}$.  Nous avons
donc que
\[
\sum_{n=0}^\infty \frac{2^n - 3^{n+2}}{5^{n+1}}
= \frac{1}{5} \sum_{n=0}^\infty \left(\frac{2}{5}\right)^n
- \frac{3^2}{5} \sum_{n=0}^\infty \left(\frac{3}{5}\right)^n
= \frac{1}{5} \left(\frac{5}{3}\right)
- \frac{3^2}{5} \left(\frac{5}{2}\right)
 = -\frac{25}{6} \ .
\]

\subQ{e} Puisque $\displaystyle 5^n 2^{-2n} =
\left(\frac{5}{4}\right)^n$ ne converge pas vers $0$ lorsque $n$ tend
vers l'infini, la série diverge.

\subQ{f} Nous avons que
\[
\frac{30}{n^2 + 3 n +2} = \frac{30}{(n+2)(n+1)}
= \frac{A}{n+2} + \frac{B}{n+1} = \frac{A(n+1) + B(n+2)}{(n+2)(n+1)} \ .
\]
Pour que l'égalité soit satisfaite, nous devons avoir
$30 = A(n+1) + B(n+2)$.  Si $n=-1$, nous obtenons $30 = B$.  Si 
$n=-2$, nous obtenons $30 = -A$.  Donc
\[
\frac{30}{(n+2)(n+1)} = \frac{-30}{n+2} + \frac{30}{n+1} \ .
\]
Nous avons la série télescopique
\begin{align*}
\sum_{n=0}^N \frac{30}{(n+2)(n+1)}
&= \sum_{n=0}^N \left(\frac{-30}{n+2} + \frac{30}{n+1}\right) \\
&= \left(-\frac{30}{2} + 30\right)
+ \left(\frac{-30}{3} + \frac{30}{2}\right)
+ \left(\frac{-30}{4} + \frac{30}{3}\right)
+ \left(\frac{-30}{5} + \frac{30}{4}\right) \\
& \qquad + \ldots + \left(\frac{-30}{N+2} + \frac{30}{N+1}\right)
= 30 - \frac{30}{N+2} \ .
\end{align*}
Puisque $\displaystyle \lim_{N\to \infty}\frac{30}{N+2} = 0$, nous
obtenons
\[
  \sum_{n=0}^\infty \frac{30}{(n+2)(n+1)} =
\lim_{N\to \infty}\left( 30 - \frac{30}{N+2}\right) =  30 \ .
\]

\subQ{g} Nous avons que
\[
\frac{1}{n(n+5)} = \frac{A}{n} + \frac{B}{n+5} = \frac{A(n+5) + Bn}{n(n+5)} \ .
\]
Pour que l'égalité soit satisfaite, nous devons avoir
$1 = A(n+5) + Bn$.  Donc, $5A=1$ et $A+B = 0$.  Nous obtenons
$A = -B = 1/5$.
\[
\frac{1}{n(n+5)} = \frac{1}{5n} - \frac{1}{5(n+1)} \ .
\]
Nous avons la série télescopique
\begin{align*}
\sum_{n=1}^N \frac{1}{n(n+5)}
&= \sum_{n=1}^N \left(\frac{1}{5n} - \frac{1}{5(n+1)}\right) \\
&= \left(\frac{1}{5} - \frac{1}{10}\right)
+ \left(\frac{1}{10} - \frac{1}{15}\right)
+ \left(\frac{1}{15} - \frac{1}{20}\right)
+ \left(\frac{1}{20} - \frac{1}{25}\right) \\
& \qquad + \ldots + \left(\frac{1}{5N} - \frac{1}{5(N+1)}\right)
= \frac{1}{5} - \frac{1}{5(N+1)} \ .
\end{align*}
Puisque $\displaystyle \lim_{N\to \infty}\frac{1}{5(N+1)} = 0$, nous
obtenons
\[
  \sum_{n=1}^\infty \frac{1}{n(n+1)} =
\lim_{N\to \infty}\left( \frac{1}{5} - \frac{1}{5(N+1)}\right) =  \frac{1}{5} \ .
\]
}

\subsection{Tests de convergence}

\compileSOL{\SOLUb}{\ref{3Q6}}{
\subQ{b} Puisque
\[
\lim_{n\rightarrow \infty} \left(1+\frac{2}{n}\right)^n =
\lim_{n\rightarrow \infty} \left( \left(1+\frac{2}{n}\right)^{n/2} \right)^2
= \left( \lim_{n\rightarrow \infty} \left(1+\frac{2}{n}\right)^{n/2} \right)^2
= e^2 \neq 0 \ ,
\]
la série $\displaystyle \sum_{n=1}^\infty \left(1+\frac{2}{n}\right)^n$
diverge.

\subQ{f} Nous utilisons le test du quotient avec
$\displaystyle a_n = \frac{(n+1)^2}{n!}$.  Puisque
\begin{align*}
\lim_{n\rightarrow \infty} \frac{|a_{n+1}|}{|a_n|}
&= \lim_{n\rightarrow \infty} \frac{(n+2)^2 n!}{(n+1)^2 (n+1)!}
= \lim_{n\rightarrow \infty} \left(\frac{n+2}{n+1}\right)^2\;\frac{1}{n+1}  \\
& = \lim_{n\rightarrow \infty} \left(\frac{1+2/n}{1+1/n}\right)^2
\lim_{n\rightarrow \infty} \frac{1}{n+1} = 1 \times 0 = 0 < 1 \ ,
\end{align*}
la série converge.

\subQ{g} Il y a plusieurs façons de démontrer que cette série converge.

Nous pouvons comparer avec la série 
$\displaystyle \sum_{n=1}^\infty \frac{1}{n^2}$ qui converge car c'est
une série de la forme
$\displaystyle \sum_{n=1}^\infty \frac{1}{n^p}$ avec $p>1$
(proposition~\ref{Pseries}).
Notons que
\[
 \frac{1}{n(n^3+5)^{1/3}} \leq \frac{1}{n(n^3)^{1/3}} = \frac{1}{n^2}
\]
pour tout $n$.  Nous obtenons donc que la série
$\displaystyle \sum_{n=1}^\infty \frac{1}{n(n^3+5)^{1/3}}$ converge
grâce au Test de comparaison.

Nous aurions aussi pu utiliser le Test de comparaison à la
limite (théorème~\ref{comp_lim_theo}) avec la série
convergente $\displaystyle \sum_{n=1}^\infty \frac{1}{n^2}$ pour
obtenir la même conclusion car
\[
\lim_{n\rightarrow \infty} \frac{1/(n(n^3+5)^{1/3})}{1/n^2} 
= \lim_{n\rightarrow \infty} \frac{n^2}{n(n^3+5)^{1/3}} 
= \lim_{n\rightarrow \infty} \left(\frac{n^3}{n^3+5}\right)^{1/3}
= \lim_{n\rightarrow \infty} \left(\frac{1}{1+5/n^3}\right)^{1/3}
= 1
\]
un nombre réel positif.

\subQ{i} Nous avons la séries $\displaystyle \sum_{n=1}^\infty a_n$ avec
$\displaystyle a_n = \frac{n}{5^n}$.  Puisque
\[
\lim_{n \to \infty} \left|\frac{a_{n+1}}{a_n}\right|
= \lim_{n\rightarrow \infty} \left| \frac{(n+1)5^n}{n5^{n+1}} \right|
= \lim_{n\rightarrow \infty} \frac{n+1}{5n}
= \frac{1}{5} \lim_{n\rightarrow \infty} \left(1+\frac{1}{n}\right)
 = \frac{1}{5} < 1 \ ,
\]
la série converge grâce au théorème~\ref{Alembert}.

\subQ{m} Puisque $-3 \leq 3\sin(n) \leq 3$ pour tout $n$, nous avons
$2 \leq 5+3\sin(n) \leq 8$ pour tout $n$.   Puisque
$3n^3 < 3n^3 + n +4$ pour tout $n>0$, nous avons
$\displaystyle \frac{1}{3n^3} > \frac{1}{3n^3 + n +4}$ pour tout
$n>0$.  Ainsi
\[
0 < \frac{5+3\sin(n)}{3n^3+n+4} \leq \frac{8}{3n^3}  \ .
\]
Or la série
$\displaystyle \sum_{n=1}^\infty \frac{8}{3n^3} = 
\frac{8}{3} \sum_{n=1}^\infty \frac{1}{n^3}$ converge grâce à la
proposition~\ref{Pseries} avec $p=3>1$.  Donc, nous obtenons du Test de
comparaison que la séries
$\displaystyle \sum_{n=1}^\infty \frac{5+3\sin(n)}{3n^3+n+4}$
converge.
}

\subsection{Convergence absolue et séries alternées}

\compileSOL{\SOLUb}{\ref{3Q7}}{
\subQ{b} La séries
$\displaystyle \sum_{n=2}^\infty \frac{(-1)^n}{\ln(n)}$
est une série alternée de la forme
$\displaystyle \sum_{n=2}^\infty (-1)^na_n$
avec $\displaystyle a_n = \frac{1}{\ln(n)}$.
Nous avons $a_n >0$ et $a_{n+1} < a_n$ pour tout $n$, et
$\displaystyle \lim_{n\to \infty} a_n = 0$.  Donc, grâce au
théorème~\ref{altTest}, la série converge.

Par contre, $\displaystyle \frac{1}{\ln(n)} > \frac{1}{n}$ pour tout
$n > 1$ et $\displaystyle \sum_{n=2}^\infty \frac{1}{n}$ diverge (la série
harmonique).  Donc, par le test de comparaison,
$\displaystyle \sum_{n=2}^\infty \frac{1}{\ln(n)}$ diverge.

En conclusion, $\displaystyle \sum_{n=2}^\infty \frac{(-1)^n}{\ln(n)}$
converge seulement conditionnellement.

\subQ{c} Nous pouvons montrer à l'aide du Test de d'Alembert que
$\displaystyle \sum_{n=1}^\infty \frac{(-1)^n n!}{(2n+1)!}$
converge absolument.  En effet, avec
$\displaystyle a_n = \frac{(-1)^n n!}{(2n+1)!}$,
\[
\lim_{n\to\infty} \frac{|a_{n+1}|}{|a_n|}
= \lim_{n\to\infty} \left|\frac{(-1)^{n+1} (n+1)!}{(2n+3)!}\right|
\, \left|\frac{(2n+1)!}{(-1)^n n!}\right|
= \lim_{n\to\infty} \frac{(n+1)}{(2n+2)(2n+3)} = 0 < 1 \ .
\]

\subQ{d} La séries
$\displaystyle \sum_{n=1}^\infty \frac{(-1)^n}{\sqrt{n(n+2)}}$
est une série alternée de la forme
$\displaystyle \sum_{n=2}^\infty (-1)^na_n$ avec
$\displaystyle a_n = \frac{1}{\sqrt{n(n+2)}}$.
Nous avons $a_n >0$, $a_{n+1} < a_n$ pour tout $n$, et
$\displaystyle \lim_{n\to \infty} a_n = 0$.  Donc, grâce au
théorème~\ref{altTest}, la série converge.

Par contre,
\[
  \lim_{n\to \infty} \frac{1/n}{1/\sqrt{n(n+2)}}
  = \lim_{n\to \infty} \frac{1}{n}\sqrt{n(n+2)}
  = \lim_{n\to \infty} \sqrt{1 + \frac{2}{n}} = \sqrt{1} = 1
\]
et $\displaystyle \sum_{n=2}^\infty \frac{1}{n}$ diverge (série
harmonique).  Donc, par le test de comparaison à la limite, \\
$\displaystyle \sum_{n=2}^\infty \frac{1}{\sqrt{n(n+2)}}$ diverge.

En conclusion, $\displaystyle \sum_{n=2}^\infty \frac{(-1)^n}{\sqrt{n(n+2)}}$
converge seulement conditionnellement.

\subQ{e} Puisque $\displaystyle \sum_{n=1}^\infty \frac{1}{n^{3/2}}$ et 
$\displaystyle \sum_{n=1}^\infty \frac{1}{n^2}$ converge
(proposition~\ref{Pseries} avec $p=3/2$ et $p=2$ respectivement), nous avons
que
$\displaystyle \sum_{n=1}^\infty \left(\frac{1}{n^{3/2}} + \frac{1}{n^2}\right)$
converge (théorème~\ref{series_linear}).  Donc\\
$\displaystyle \sum_{n=1}^\infty (-1)^n\left(\frac{1}{n^{3/2}}
  + \frac{1}{n^2}\right)$ converge absolument.

\subQ{f} la suite
$\displaystyle \sum_{n=1}^\infty (-1)^n \frac{n^2}{(2n+1)(2n+3)}$
ne converge pas absolument car \\
$\displaystyle \lim_{n\rightarrow \infty} \frac{n^2}{(2n+1)(2n+3)} = 1/4 \neq 0$.
Elle ne converge même pas simplement car \\
$\displaystyle \left\{(-1)^n \frac{n^2}{(2n+1)(2n+3)}\right\}_1^\infty$
ne converge pas vers $0$.
}

\compileSOL{\SOLUb}{\ref{3Q8}}{
Nous avons une série alternée
$\displaystyle \sum_{n=1}^\infty (-1)^{n+1} a_n$
avec $\displaystyle a_n = \frac{1}{3^n}$.  Vérifions les conditions du
test des séries alternées.  Nous avons $a_n>0$ sans aucun doute.  De plus
$\displaystyle \lim_{n\to \infty} a_n = 0$ grâce à la 
proposition~\ref{suite1}.  Finalement,
$\displaystyle a_n = \frac{1}{3^n} > \frac{1}{3^{n+1}} = a_{n+1}$. 
Donc la série alternée
$\displaystyle \sum_{n=1}^\infty \frac{(-1)^{n+1}}{3^n}$ converge.

Il faut trouver $N$ tel que $|S_n - S |<10^{-3}$ pour tout $n\geq N$.
Or, pour une série alternée, nous avons que $|S_n - S |< a_{n+1}$.  Il
suffit donc de trouver $n$ tel que
$\displaystyle \frac{1}{3^{n+1}} \leq 0.001$.
\begin{align*}
\frac{1}{3^{n+1}} \leq 0.001 & \Rightarrow 1000 \leq 3^{n+1}
\Rightarrow \ln(1000) = 3\ln(10) \leq (n+1) \ln(3) \\
&\Rightarrow \frac{3\ln(10)}{\ln(3)} -1 \leq n
\Rightarrow 5.2877\ldots \leq n \ .
\end{align*}
Donc six termes ($N=6$) est suffisant.
}

\compileSOL{\SOLUa}{\ref{3Q9}}{
Nous avons une série alternée $\displaystyle \sum_{n=1}^\infty (-1)^n a_n$
avec $a_n = n/4^n$.  Vérifions les conditions du test des séries
alternées.  Nous avons évidemment $a_n >0$.  Puisque
\[
  2^n > n \Rightarrow
  \left(\frac{1}{2}\right)^n = \frac{2^n}{4^n} > \frac{n}{4^n} \geq 0
\]
et $\displaystyle \lim_{n\to \infty}\left(\frac{1}{2}\right)^n = 0$
(proposition~\ref{suite1}), nous avons grâce au Théorème des gendarmes
(théorème~\ref{gendarmeS}) que
$\displaystyle \lim_{n\rightarrow \infty} a_n = 0$.  Finalement,
\[
  \frac{3n}{4} > \frac{1}{4} \Rightarrow n > \frac{n+1}{4}
  \Rightarrow \frac{n}{4^n} > \frac{n+1}{4^{n+1}}
  \Rightarrow a_n > a_{n+1}
\]
pour $n>0$.  Donc la série alternée
$\displaystyle \sum_{n=1}^\infty \frac{(-1)^n \, n}{4^n}$ converge.

Pour déterminer $N$ tel que $|S_n - S |<10^{-3}$ pour tout $n\geq N$,
il suffit de trouver $n$ tel que $|S_n - S |< a_{n+1} < 10^{-3}$.
Pour simplifier les calculs, nous choisissons $n$ tel que
\[
a_{n+1} = \frac{n+1}{4^{n+1}} < \left(\frac{1}{2}\right)^{n+1} <
10^{-3} \ .
\]
Si nous résolvons pour $n$, nous trouvons
\begin{align*}
\left(\frac{1}{2}\right)^{n+1} < 10^{-3}
&  \Rightarrow 10^3 < 2^{n+1}
\Rightarrow \log_{10}(10^3) < \log_{10}(2^{n+1}) = (n+1)\log_{10}(2) \\
& \Rightarrow 3 - \log_{10}(2) < n\log_{10}(2)
  \Rightarrow n> \frac{3 -\log_{10}(2)}{\log_{10}(2)} \approx 8.966 \ .
\end{align*}
Donc $N=9$ est un choix acceptable et nous avons
\[
  \sum_{n=1}^\infty \frac{(-1)^n \, n}{4^n} \approx
  \sum_{n=1}^9 \frac{(-1)^n \, n}{4^n} \approx -0.1600075
\]
avec une erreur d'au plus $10^{-3}$.

\noindent Note: Nous pouvons vérifier en calculant directement $a_{n+1}$ que
$n = 6$ est la plus petite valeur qui donne $a_{n+1} < 10^{-3}$.
Donc, nous avons surestimé la valeur de $n$. 
}

%%% Local Variables: 
%%% mode: latex
%%% TeX-master: "notes"
%%% End: 

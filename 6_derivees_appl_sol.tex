\section{Applications de la dérivée}

\subsection{Dérivées d'ordres supérieures}

\compileSOL{\SOLUb}{\ref{6Q1}}{
\subQ{a} $h'(y) = 10y^9 - 9 y^8$ et $h''(y) = 90y^8 - 72 y^7$.

\subQ{b}  Puisque $\displaystyle f(x) = \frac{3+x}{2x}
= \frac{3}{2}x^{-1} + \frac{1}{2}$, nous avons
$\displaystyle f'(x) = -\frac{3}{2} x^{-2}$ et
$\displaystyle f''(x) = 3 x^{-3}$.

\subQ{c} Si nous utilisons la règle de la dérivée d'un produit,
nous obtenons
\[
f'(x) = \left(\dfdx{x^2}{x}\right) e^x + x^2\left(\dfdx{e^x}{x}\right)
= 2x e^x + x^2 e^x = (2x+x^2)e^x
\]
et
\begin{align*}
f''(x) &= \left(\dfdx{(2x+x^2)}{x}\right) e^x + (2x +x^2)
\left(\dfdx{e^x}{x}\right) \\
&= (2+2x) e^x + (2x +x^2) e^x = (2 + 4x + x^2)e^x \; .
\end{align*}

\subQ{d} Si nous utilisons seulement la règle de la dérivée d'un quotient,
nous obtenons
\[
f'(x) = \frac{\displaystyle \left(\dfdx{(1+x)}{x}\right) e^x  -
(1+x)\left(\dfdx{e^x}{x}\right)}{(e^x)^2} =
\frac{e^x  - (1+x)e^x}{e^{2x}} = \frac{-x}{e^x}
\]
et
\[
f''(x) = \frac{\displaystyle \left(\dfdx{(-x)}{x}\right) e^{x}
+ x \left(\dfdx{e^x}{x}\right)}{e^{2x}} =
\frac{-e^x + x e^x}{e^{2x}} = \frac{x-1}{e^x} \ .
\]

\subQ{e} Si nous utilisons la règle de la dérivée d'un produit,
nous obtenons
\[
g'(z) = \left(\dfdx{(z+4)}{z}\right) \ln(z) +
(z+4)\left(\dfdx{\ln(z)}{z}\right) = \ln(z) + (z+4)\left(\frac{1}{z}\right)
= \ln(z) + 1 +\frac{4}{z}
\]
et
\[
  g''(z) = \frac{1}{z} - \frac{4}{z^2} \quad .
\]

\subQ{f} Si nous utilisons la règle de la dérivée d'un produit,
nous obtenons
\[
F'(w) = \left(\dfdx{e^w}{w}\right) \ln(w) +
e^w \left(\dfdx{\ln(w)}{w}\right) = e^w \ln(w) + e^w \left(\frac{1}{w}\right) =
e^w \left( \ln(w) + \frac{1}{w}\right)
\]
et
\begin{align*}
F''(w) &= \left(\dfdx{e^w}{w}\right) \left( \ln(w) + \frac{1}{w}\right) +
e^w \left(\dfdx{\left( \ln(w) + \frac{1}{w}\right)}{w}\right) \\
&= e^w \left( \ln(w) + \frac{1}{w}\right) + e^w
\left( \frac{1}{w} - \frac{1}{w^2}\right)
= e^w \left( \ln(w) + \frac{2}{w} - \frac{1}{w^2} \right) \ .
\end{align*}

\subQ{g}
Puisque $f(x) = \ln(x^7) = 7 \ln(x)$, nous avons
$\displaystyle f'(x) = \frac{7}{x}$ et
$\displaystyle f''(x) = -\frac{7}{x^2}$.

\subQ{h}  Il faut premièrement remarquer que $e^{-z} = g_1(g_2(z))$ où
$g_1(w)=e^w$ et $g_2(z) = -z$.  Puisque $g_1'(w) = e^w$ et
$g_2'(z) = -1$, nous obtenons
\[
  \dfdx{\left( e^{-z} \right)}{z} = g_1'(g_2(z)) g_2'(z)
= e^{-z} \times (-1) = - e^{-z} \ .
\]
Ainsi
\begin{align*}
f'(z) &= \frac{\displaystyle \left(\dfdx{(1+e^{-z})}{z}\right)
(1+e^z) -(1+e^{-z})\left(\dfdx{(1+e^z)}{z}\right)}{(1+e^z)^2} \\
&= \frac{-e^{-z}(1+e^z) -(1+e^{-z})e^z}{(1+e^z)^2}
= -\, \frac{e^{-z} +2 + e^z}{1+2 e^z + e^{2z}} \ .
\end{align*}

Nous avons que $e^{2z} = g_1(g_2(z))$ où $g_1(w)=e^w$ et $g_2(z) = 2z$.
Puisque $g_1'(w) = e^w$ et $g_2'(z) = 2$, nous obtenons
\[
  \dfdx{\left( e^{2z} \right)}{z} = g_1'(g_2(z)) g_2'(z)
= e^{2z} \times (2) = 2 e^{2z} \ .
\]
Ainsi,
\begin{align*}
f''(z) &= -\frac{\displaystyle \left(\dfdx{(e^{-z}+2+e^z)}{z}\right)
(1+2e^z+e^{2z}) -(e^{-z} +2 + e^z)\left(\dfdx{(1+2e^z+e^{2z})}{z}\right)}
{(1+2e^z+e^{2z})^2} \\
&= -\frac{(-e^{-z} +e^z)(1+2e^z+e^{2z}) -(e^{-z}+2+e^z)(2e^x +2e^{2z})}
{(1+2e^z+e^{2z})^2} \\
&= \frac{e^{-z} +4 +6 e^z +4 e^{2z} + e^{3z}}{(1+2e^z+e^{2z})^2} \ .
\end{align*}

\subQ{k}
Grâce à la règle de la dérivée d'un quotient, nous avons
\begin{align*}
f'(x) &=
\frac{\displaystyle \left(\dfdx{(2+x^3)}{x}\right)
(1+x^2) -(2+x^3)\left(\dfdx{(1+x^2)}{x}\right)}{(1+x^2)^2} \\
&= \frac{(3x^2)(1+x^2)-(2+x^3)(2x)}{(1+x^2)^2}
= \frac{x^4+3x^2-4x}{1+2x^2 +x^4}
\end{align*}
et
\begin{align*}
f''(x) &=
\frac{\displaystyle \left(\dfdx{(x^4+3x^2-4x)}{x}\right)
(1+2x^2+x^4) -(x^4+3x^2-4x)\left(\dfdx{(1+2x^2+x^4)}{x}\right)}
{(1+2x^2+x^4)^2} \\
&= \frac{(4x^3+6x-4)(1+2x^2+x^4)-(x^4+3x^2-4x)(4x+4x^3)}{(1+2x^2+x^4)^2} \\
&= \frac{-2x^5+12x^4+4x^3-8x^2+22x-4}{(1+2x^2 +x^4)^2} \ .
\end{align*}
}

\compileSOL{\SOLUb}{\ref{6Q2}}{
$H'(\theta) = \theta\cos(\theta) + \sin(\theta)$ et
$H''(\theta) = 2\cos(\theta) -\theta \sin(\theta)$.
}

\compileSOL{\SOLUb}{\ref{6Q3}}{
Assumons que la distance positive est vers le haut.
Si $d(t)$ est la distance entre le sol et l'objet au temps $t$ , alors
$d''(t) = -22.88$.  Le signe négatif devant $22.88$ provient du fait
que la direction positive est vers le haut.  Donc
$d'(t) = -22.88 t + C_1$ et $d(t) = -11.44 t^2 + C_1 t + C_2$ où $C_1$
et $C_2$ sont des constantes qui sont déterminées par les conditions
initiales.  Nous laissons tomber un objet d'une hauteur de $100$ m.  Donc
$d(0)=100$ et $d'(0)=0$.  $d'(0)=0$ donne $C_1=0$ et $d(0)=100$ donne
$C_2=100$.  Nous obtenons $\displaystyle d(t) = -11.44 t^2 + 100$.

Nous cherchons $t$ tel que $d(t) = 0$; c'est-à-dire,  tel que
$-11.44 t^2 + 100 = 0$.  La solution positive de cet équation est
$t \approx 2.9566$ s.  La vélocité
à ce moment est $d'(2.95656198) \approx -67.6461$ m/s.  Comme la
direction positive est vers le haut, cela veut dire que l'objet frappe
le sol à une vitesse de $67.6461$ m/s.
}

\compileSOL{\SOLUa}{\ref{6Q4}}{
Assumons que la distance positive est vers le haut.
Si $d(t)$ est la distance entre le sol et l'objet au temps $t$, alors
$d''(t) = -22.88$.  Le signe négatif devant $22.88$ provient du fait
que la direction positive est vers le haut.  Donc
$d'(t) = -22.88 t + C_1$ et $d(t) = -11.44 t^2 + C_1 t + C_2$ où $C_1$
et $C_2$ sont des constantes qui sont déterminées par les conditions
initiales.  Nous avons que $d'(0)=10$ m/s car nous lançons l'objet vers le haut.
Donc $C_1 = 10$.  De plus, $d(0)=100$.  Donc $C_2=100$.  Ainsi,
\[
d(t) = 100 + 10 t - 11.44 t^2 \  \text{m},
\]
où le temps $t$ est en secondes.  

Au moment où l'objet atteint sa hauteur maximale, la vélocité est
nulle.  C'est-à-dire, $d'(t)= -22.88 t + 10 = 0$.  Donc
$t \approx 0.43703$ s.  Il s'écoule $0.43703$ s avant que l'objet
atteigne sa hauteur maximale. À ce moment, l'objet est à
$d(0.43703) \approx 102.1853$ m.

L'objet frappe le sol lorsque $d(t)=0$.  Les deux racines du polynôme
$100 + 10 t - 11.44 t^2$ sont
\begin{align*}
t_1 &= \frac{-10 - \sqrt{10^2 + 4\times 100\times 11.44}}{-2\times 11.44}
\approx 3.425755
\intertext{et}
t_2 &= \frac{-10 + \sqrt{10^2 + 4\times 100\times 11.44}}{-2\times 11.44}
\approx -2.551629 \ .
\end{align*}
Naturellement, la seule réponse plausible est $t=3.425755$ s.  À ce
moment, l'objet a une vélocité de $d'(3.425755)\approx -68.3813$ m/s.
Comme la direction positive est vers le haut, cela veut dire que
l'objet frappe le sol à une vitesse de $68.3813$ m/s.
}

\subsection{Graphes de fonctions}

\compileSOL{\SOLUb}{\ref{6Q5}}{
Nous savons que la valeur de $f'$ en un point est donné par la pente
de la droite tangente à la courbe en ce point.  En particulier, $f'>0$
sur un intervalle si et seulement si $f$ est croissante sur cet
intervalle, et $f'<0$ sur un intervalle si et seulement si $f$ est
décroissante sur cet intervalle.  De plus, $f''<0$ sur un
intervalle si et seulement si $f$ est concave sur cet intervalle,
et $f''>0$ sur un intervalle si et seulement si $f$ est convexe
sur cette intervalle.  Avec cette information, nous obtenons les résultats
suivants.

\subQ{a} Les points critiques sont $-3$, $-2$, $-1$, $-0.5$, $1$, $2$,
$4$ et $5$.

\subQ{b} $f'>0$ ( la fonction est strictement croissante) sur les intervalles
$]-\infty, -3[$, $]-2,-1[$, $]-0.5,1[$, $]2,4[$ et $]5,\infty[$.

\subQ{c} $f'<0$ ( la fonction est strictement décroissante) sur les intervalles
$]-3, -2[$, $]-1, -0.5[$, $]1,2[$ et $]4,5[$.

\subQ{d} $f''>0$ (la fonction est convexe) sur les intervalles
$]-2.5, -1.5[$, $]-0.8, 0.2[$, $]1.5, 3[$ et $]4.5, \infty[$.

\subQ{e} $f''<0$ (la fonction est concave) sur les intervalles
$]-\infty,-2.5[$, $]-1.5,-0.8[$, $]0.2, 1.5[$ et $]3, 4.5[$.
}

\compileSOL{\SOLUa}{\ref{6Q6}}{
\subQ{a} Si la dérivée $f'$ est strictement croissante alors $f''>0$
et donc la fonction est convexe.
\PDFgraph{6_derivees_appl/graph_funct2}

\subQ{b} Si $f'< 0$ alors la fonction $f$ est strictement
décroissante.  Si la dérivée $f'$ est strictement croissante alors
$f''>0$ et la fonction est convexe. 
\PDFgraph{6_derivees_appl/graph_funct3}
}

\compileSOL{\SOLUb}{\ref{6Q7}}{
La voiture se déplace le plus rapidement lorsque $t\approx 15$ s et
$t\approx 45$ s.  C'est lorsque la pente de la tangente à la courbe est la
plus grande (i.e.\ $x'(t)$ a sa plus grande valeur). 

La voiture accélère le plus rapidement au voisinage de $t=6$ s et
$t=42$ s.  C'est lorsque la pente de la tangente à la courbe augmente le plus
rapidement tout en étant positive (i.e.\ $x''(t)>0$ a sa plus grande
valeur et $x'(t)>0$).

La voiture décélère le plus rapidement au voisinage de $t=22$ s.
C'est lorsque la pente de la tangente à la courbe diminue le plus
rapidement tout en étant positive (i.e.\ $x''(t)<0$ a sa plus petite
valeur et $x'(t)>0$).
}

\compileSOL{\SOLUa}{\ref{6Q8}}{
\subQ{a} La vélocité au temps $t$ est $p'(t) = -10.4 t -2$ m/s et
l'accélération au temps $t$ est $p''(t) = -10.4$ m/s$^2$.

\subQ{b} $p(t)$ est un simple polynomial de degré deux.  Nous pourrions
facilement tracer le graphe de $p$ sans faire appel aux dérivées de
$p$.  Mais, comme nous voulons nous pratiquer à tracer des graphes à l'aide
de l'information fournie par les dérivées d'ordre un et deux, nous
ignorerons le fait que $p(t)$ est un simple polynomial.

Les solutions de $p(t)= 0$ sont $t=t_1=2.91451\ldots$ et
$t=t_2=-3.299133\ldots$.  Nous pouvons ignorer $t_2$ puisque nous sommes
simplement intéressé aux valeurs positives de $t$.

Il n'y a qu'un seul point critique; c'est $t = -0.1923\ldots$ s.
Comme le point critique est négatif, nous pouvons nous douter que l'objet a
été lancé vers le sol.  Nous pouvons ignorer ce point critique puisqu'il
est négatif.  Pour $t\geq 0 $, nous aurons toujours $p'(t)<0$.  Donc la
fonction est strictement décroissante.

Puisque $p''(t) = -10.4 <0$ pour tout $t$, la fonction sera concave
sur toute la droite réelle.

Le graphe de $p$ est
\MATHgraph{6_derivees_appl/graph_funct5}{8cm}

\subQ{c} La hauteur de la tour est $p(0)=50$ m.  Puisque $p'(0)= - 2$
m/s, l'objet est lancé vers le bas à une vitesse de $2$ m/s.  Puisque
$p''(t)=-10.4$ m/s$^2$, l'accélération dû à la gravité sur cette
planète est $10.4$ m/s$^2$.
}

\compileSOL{\SOLUb}{\ref{6Q9}}{
Pour cette question, assumons que la direction positive est vers le haut.
\subQ{a}
Soit $d(t)$ la distance entre le sol et l'objet au temps $t$.  Nous avons
que $d''(t) = -2.15\times 10^{-3}$.  l'accélération est négative car
nous assumons que la direction positive est vers le haut.
Donc $d'(t) = -2.15\times 10^{-3} t + C_1$ et
$d(t) = -1.075\times 10^{-3} t^2 + C_1 t + C_2$ où $C_1$ et $C_2$ sont
des constantes qui sont déterminées par les conditions initiales.

L'objet a été lancé vers le haut à une vitesse de $5$ m/s.  Donc
$d'(0)=5$ m/s.  Ce qui donne $C_1 = 5$.  L'objet est lancé d'une
hauteur de $100$ m. Donc $d(0)=100$ m.  Ce qui donne $C_2 = 100$.

La vélocité de l'objet en fonction du temps est
$\displaystyle v(t) = d'(t) = -2.15\times 10^{-3} t + 5$ m/s et le
déplacement en fonction du temps est
$\displaystyle d(t) = -1.075\times 10^{-3} t^2 + 5 t + 100$ m.

\subQ{b} Pour déterminer la hauteur maximale, il faut trouver les
points critiques de $d$; c'est-à-dire, les valeurs de $t$ telles que
$v(t)=-2.15\times 10^{-3} t + 5=0$. Donc
$\displaystyle t = 5/( 2.15 \times 10^{-3}) \approx 2,325.581$ s.
Puisque $d''(t) < 0$ pour tout $t$, la fonction $d$ atteint son
maximum absolu à $t = t_1 \approx 2,325.581$ qui est
$d(t_1) \approx 5,913.95$ m.

\subQ{c} Nous cherchons $t>0$ tel que
$\displaystyle d(t)= -1.075\times 10^{-3} t^2 + 5 t + 100 = 100$ où,
si vous préférez, $\displaystyle -1.075\times 10^{-3} t^2 + 5 t = 0$.
La seule solution positive est $t = t_2 \approx 4,651.16$ s.  À ce
moment, la vélocité est $v(t_2) = -5$ m/s.  Ceci n'est pas surprenant
car le déplacement de l'objet est décrit par une parabole.  À cause de
la symétrie de la parabole par rapport à son axe, la pente de la
tangente à la parabole au point $(0,100)$ est l'inverse additif de la
pente à la parabole au point $(t_2,100)$.

\subQ{d} Nous cherchons $t>0$ tel que
$d(t)=-1.075\times 10^{-3} t^2 + 5 t + 100 = 0$.  La seule solution
positive est $t_3 \approx 4,671.0775$ s.  La vélocité lorsque l'objet
frappe le sol est donc $v(t_3) \approx -5.0428$ m/s

\subQ{e}
\MATHgraph{6_derivees_appl/lune}{8cm}
}

\compileSOL{\SOLUb}{\ref{6Q10}}{
Nous avons l'information suivante pour la fonction $g$.
\[
\begin{array}{c|c|c|c|c|c|}
t & 0 < t < 0.5 & 0.5 & 0.5 < t < 1 & 1 & 1 < t < 2 \\
\hline
g'(t) & - & 0 & + & + & + \\
\hline
& \text{décroît} & \text{min. local} & \text{croît} & \text{point} &
\text{croît} \\
& \text{et convexe} & \text{et convexe} & \text{et convexe}
& \text{d'inflexion} & \text{et concave}
\end{array} \quad \ldots
\]
\[
\ldots \quad \begin{array}{|c|c|c|c|c|c|}
2 & 2 < t < 3  & 3 & 3 <t<4 & 4 & 4 < t < 5 \\
\hline
0 & -  & - & - & 0 & + \\
\hline
\text{max. local} & \text{décroît} & \text{point} & \text{décroît} & \text{min. local} & \text{croît} \\
\text{et concave} & \text{et concave} & \text{d'inflexion} &
\text{et convexe} & \text{et convexe} & \text{et convexe}
\end{array} \quad \ldots
\]
\[
\ldots \quad \begin{array}{|c|c|c|c}
5 & 5 < t < 6 & 6 & t > 6 \\
\hline
+ & + & 0 & - \\
\hline
\text{point} & \text{croît} & \text{max. local} & \text{décroît} \\
\text{d'inflexion} & \text{et concave} & \text{et concave} & \text{et concave}
\end{array}
\]
Ne pas oublier que $g$ est convexe lorsque $g'$ est croissante et
$g$ est concave lorsque $g'$ est décroissante.

Nous obtenons le graphe de $g$ suivant.
\PDFgraph{6_derivees_appl/derivative2_sol}
}

\compileSOL{\SOLUa}{\ref{6Q11}}{
Nous avons l'information suivante pour la fonction $w$.
\[
\begin{array}{c|c|c|c|c|c}
t & 0 < t < 0.5 & 0.5 & 0.5 < t < 1 & 1 & \ldots \\
\hline
w'(t) & + & 0 & - & - & \ldots \\
\hline
& \text{croît} & \text{max. local} & \text{décroît} & \text{plus petite}
&\ldots \\
& \text{et concave} & \text{et concave} & \text{et concave}
& \text{pemte} & \ldots
\end{array}
\]
Ne pas oublier que $w$ est convexe lorsque $w'$ est croissante et
$w$ est concave lorsque $w'$ est décroissante.

Nous obtenons le graphe suivant.
\PDFgraph{6_derivees_appl/derivative3_sol}
}

\compileSOL{\SOLUb}{\ref{6Q12}}{
\subQ{a} Le volume total des déchets au temps $t$ est
$V(t) = V_a(t)+V_b(t) = t^2 + 36$ m$^3$.  La masse du produit toxique
au temps $t$ sera $M(t) = V(t) \rho(t) = (t^2 + 36)(1.2 - 0.1 t)$ g.

\subQ{b} le taux de variation de la masse en fonction du temps est
\begin{align*}
M'(t) &= V'(t) \rho(t) + V(t)\rho'(t) =
(2t)(1.2-0.1 t) -0.1 (t^2+36) \\
&= -0.3 t^2 + 2.4 t - 3.6 \text{ g/année.}
\end{align*}

\subQ{c}
La seul solution de $M(t)= 0$ est $t=12$.  Les points critiques de $M$ sont
les points où
\[
M'(x) = -0.3 t^2 + 2.4 t - 3.6 = -0.3( t^2 -8t +12) = -0.3( t-2)(t-6) = 0 \ .
\]
Nous trouvons $x=2$ et $6$.
De plus, $\displaystyle \lim_{t\rightarrow \infty} M(t) = -\infty$.

Nous avons l'information suivante pour la fonction $M$.
\[
\begin{array}{c|c|c|c|c|c|c|}
t & 0 & 0<t<2 & 2& 2 < t <6 & 6 & 6<x<12 \\
\hline
M(t) & 43.2 & + & 40 & + & 43.2 & + \\
M'(t) & - & - & 0 & + & 0 & - \\
\hline
& \text{décroît} & \text{décroît} & \text{min. loc.} & \text{croît} &
\text{max. loc.} & \text{décroît} \\
\end{array} \quad \ldots
\]
\[
\qquad \ldots \quad 
\begin{array}{|c|c|c}
12 & 12 < x & +\infty  \\
\hline
0 & - & -\infty \\
- & - &  \\
\hline
\text{décroît} & \text{décroît} &
\end{array}
\]

Le graphe de $M$ a l'allure suivante.
\PDFgraph{6_derivees_appl/graph_funct7}

Les efforts ne sont pas encourageants au début mais le produit toxique
sera complètement éliminé des déchets après 12 ans.
}

\compileSOL{\SOLUa}{\ref{6Q13}}{
\subQ{a} La densité au temps $t$ est
$\displaystyle \rho(t) = \frac{M(t)}{V(t)} = \frac{1 + t^2}{1+t}$ g/cm$^3$.

\subQ{b}
\begin{align*}
\rho'(t) &= \frac{\displaystyle \left(\dfdx{(1+t^2)}{t}\right) (1+t) - (1+t^2)
\left(\dfdx{(1+t)}{t}\right)}{(1+t)^2}
= \frac{2t(1+t) - (1+t^2)}{(1+t)^2}\\
&=\frac{t^2+2t-1}{(1+t)^2}
\end{align*}

\subQ{c} Nous cherchons les valeurs de $t$ pour lesquelles $\rho'(t)>0$.
Puisque $(1+t)^2>0$ pour tous $t\neq -1$, il suffit de trouver les valeurs de
$t$ pour lesquelles $t^2+2t-1>0$.

Les racines du polynôme $t^2+2t-1$ sont
$t_1 = -1+\sqrt{2} = 0.41421356\ldots$ et
$t_2 = -1 - \sqrt{2} = -2.41421356\ldots$.  Puisque le coefficient
de $t^2$ dans le polynôme $t^2+2t-1$ est positif, nous avons que
$t^2+2t-1>0$ pour $t<t_2$ et $t>t_1$.

Si nous considérons seulement les valeurs de $t\geq 0$, nous avons que
$\rho'(t)>0$ pour $t>t_1$.  La densité $\rho$ est donc strictement
croissante sur l'intervalle $]t_1,+\infty[$.

\subQ{d} Nous traçons le graphe de $\rho$ seulement pour $t\geq 0$.
C'est l'intervalle où $\rho(t)>0$.  Nous avons calculé les
points critiques de $\rho$ en (c); c'est-à-dire, les points $t$ tels que
$\rho'(t)=0$.  Le seul point critique positif et est $t=t_1$.  De plus,
$\displaystyle \lim_{t\rightarrow \infty} \rho(t)
= \lim_{t\rightarrow \infty} \frac{1 + t^2}{1+t} = +\infty$.

Nous obtenons les données suivantes.
\[
\begin{array}{c|c|c|c|c|c}
t & 0 & 0<t<t_1 & t_1& t_1\approx 0.4142 < t & \infty \\
\hline
\rho(t) & 1 & + & \approx 0.828 & + & \infty \\
\rho'(t) & - & - & 0 & + &  \\
\hline
& \text{décroît} & \text{décroît} & \text{min. loc.} & \text{croît} & \\
\end{array}
\]
Le graphe de $\rho$ a l'allure suivante.
\PDFgraph{6_derivees_appl/graph_funct8}
}

\compileSOL{\SOLUb}{\ref{6Q14}}{
\subQ{a} La taille de la population après un an est la somme de la
population initiale et de la production annuelle pour l'ensemble de la
population.  Donc
\[
  T(p) = p + pf(p) = p + 2p\left( 1- \frac{p}{1000}\right)
  = 3p - \frac{p^2}{500} \ .
\]

\subQ{b} $\displaystyle T'(p) = 3 -\frac{p}{250}$.

\subQ{c} $T(p)$ est un simple polynomial de degré deux.  Nous pourrions
facilement tracer le graphe de $T$ sans faire appel à la dérivée de $T$.
Mais, comme nous voulons nous pratiquer à tracer des graphes à l'aide
de la dérivée, nous ignorerons le fait que $T(p)$ est un simple polynomial.

Les solutions de $T(p)= 0$ sont $p=0$ et $p=1500$.  Il n'y a qu'un seul
point critique pour $T$ (lorsque $T'(p) = 0$) et c'est $p=750$.
De plus, $p$ et $T(p)$ doivent être positifs car $p$ est le nombre
d'individus dans la population initialement et $T(p)$ est la  
taille de la population après un an.  Nous devons donc assumer que
$0 \leq p \leq 1500$.

Nous obtenons le tableau suivant.
\[
\begin{array}{c|c|c|c|c|c}
p & 0 & 0<p<750 & 750 & 750<p<1500 & 1500\\
\hline
T(p) & 0 & + & 1125 & + & 0 \\
T'(p) & + & + & 0 & - & - \\
\hline
& \text{croît} & \text{croît} & \text{max. loc.} & \text{décroît} &
\text{décroît} \\
\end{array}
\]
Le graphe de $T$ a l'allure suivante.
\PDFgraph{6_derivees_appl/graph_funct9}
}

\compileSOL{\SOLUa}{\ref{6Q15}}{
$f\,'(x)>0$ si $-2<x<1$ veut dire que $f$ est (strictement)
croissante pour $x$ entre $-2$ et $1$.
$f\,'(x)<0$ si $x<-2$ ou si $x>1$ veut dire que $f$ est (strictement)
décroissante pour $x<-2$ et $x>1$.

Nous obtenons le graphe suivant.
\PDFgraph{6_derivees_appl/graphQ1}
}

\compileSOL{\SOLUb}{\ref{6Q16}}{
Nous avons
\[
f'(x) = \frac{4x^2 e^{x^2} - e^{x^2}}{x^2}
= \frac{4e^{x^2}(x^2 - 1/4)}{x^2}  \ .
\]
Nous résumons dans le tableau suivant l'information que nous pouvons
déduire de $f$ et de sa dérivée.
\[
\begin{array}{c|c|c|c|c|c|c|c}
x & x<-1/2 & -1/2 & -1/2<x<0 & 0 & 0<x<1/2 & x>1/2 \\
\hline
f(x) & - & - & - & \text{N.D.} & + & + & + \\
\hline
f'(x) & + & 0 & - & \text{N.D.} & - & 0 & + \\
\hline
& & \text{max.} & & \text{asymptote} & & \text{min.} & \\
& & \text{local} & & \text{vertical} & & \text{local} & \\
\end{array}
\]
}

\compileSOL{\SOLUb}{\ref{6Q17}}{
Pour répondre à cette question, il faut savoir que 
\[
  f'(x) = \frac{10-3x}{2\sqrt{5-x}} \qquad \text{et} \qquad
  f''(x) = \frac{3x-20}{4(5-x)^{3/2}} \ .
\]
Nous résumons dans le tableau suivant l'information que nous pouvons
déduire de $f$, $f'$ et $f''$.
\[
\begin{array}{c|c|c|c|c|c}
x & x<0 & 0 & 0<x<10/3 & 10/3 & 10/5<x<5 \\
\hline
f(x) & - & 0 & + & + & + \\
\hline
f'(x) & + & + & + & 0 & - \\
\hline
f''(x) & - & - & - & - & - \\
\hline
& \multicolumn{3}{|c|}{\text{croît et}} & \text{max.} & \text{décroît et} \\
& \multicolumn{3}{|c|}{\text{concave}} & \text{local} & \text{concave} \\
\end{array}
\]
\subQ{a} La fonction est strictement croissante sur $]-\infty, 10/3[$
et strictement décroissante sur $]10/3,5[$.

\subQ{b} Le seul maximum local est $f(10/3) = 10\sqrt{15}/9$.

\subQ{c} La fonction est concave sur $]-\infty,5[$.

\subQ{d}
\MATHgraph{6_derivees_appl/figFX1}{8cm}
}

\compileSOL{\SOLUa}{\ref{6Q18}}{
Pour répondre à cette question, il faut utiliser
\[
  f'(x) = 1-2\cos(x) \qquad \text{et} \qquad
  f''(x) = 2\sin(x)
\]
pour produire un tableau contenant l'information sur la croissance, la
concavité, les points critiques, \ldots de $f$.  Nous laissons aux
lecteurs le soin de produire ce tableau.

\subQ{a} La fonction est strictement croissante sur $]\pi/3, 5\pi/3[$
et $]7\pi/3,3\pi[$ et strictement décroissante sur $]0,\pi/3[$ et
$]5\pi/3,7\pi/3[$.

\subQ{b} Le seul maximum local est $f(5\pi/3) = 5\pi/3 + \sqrt{3}$.
Il y a deux minimums locaux: $f(\pi/3) = \pi/3-\sqrt{3}$ et
$f(7\pi/3) = 7\pi/3+\sqrt{3}$.

\subQ{c} La fonction est convexe sur $]0,\pi[$ et $]2\pi,3\pi[$ et
concave sur $]\pi,2\pi[$.  Les points $(\pi,\pi)$ et
$(2\pi,2\pi)$. sont des points d'inflexion.

\subQ{d}
\MATHgraph{6_derivees_appl/figFX2}{8cm}
}

\compileSOL{\SOLUb}{\ref{6Q19}}{
  \PDFgraph{6_derivees_appl/graphQ2}
}

\compileSOL{\SOLUa}{\ref{6Q20}}{
\subQ{d}
\PDFgraph{6_derivees_appl/graph_funct27}
}

\compileSOL{\SOLUb}{\ref{6Q21}}{
\subQ{a}
Les solutions de $h(x)= x^3-3x + 3 = 0$ sont les points où $h$ coupe
l'axe des $x$.  Malheureusement, nous n'avons pas de formule pour les
calculer comme c'est le cas pour les équation quadratiques.  Les points
critiques de $h$ sont les points où $h'(x) = 3x^2 -3 = 3(x^2-1) =0$.  nous
trouvons $x=1$ et $-1$.

Les points critiques de $h'(x)$ sont les solutions de $h''(x) = 6x = 0$.
Donc $x=0$ est un candidat pour un point d'inflexion.  Finalement,
$\displaystyle \lim_{x\rightarrow \infty} h(x) = +\infty$ et
$\displaystyle \lim_{x\rightarrow -\infty} h(x) = -\infty$.

Nous avons de plus que $h(-3) = - 15 < 0$ et $h(-2) = 1 >0$.  Comme la
fonction $h$ change de signe entre $-3$ et $-2$, il doit donc y avoir
un point $c$ entre $-3$ et $-2$ où $h(c) = 0$ selon le Théorème des
valeurs intermédiaires.  Comme nous pouvons voir dans le tableau
ci-dessous, le seul minimum local pour $x > c$ est positif.  Donc $h$
coupe l'axe des $x$ en un seul point.

Nous résumons dans le tableau suivant l'information que nous pouvons
déduire de $h$, $h'$ et $h''$.
\[
\begin{array}{c|c|c|c|c|c|c|}
x & -\infty & x<c & c & c < x -1 & -1 & -1< x <0  \\
\hline
h(x) & -\infty & - & 0 & + & 5 & + \\
h'(x) & & + & + & + & 0 & - \\
h''(x) & & - & - & - & - & - \\
\hline
& & \multicolumn{3}{|c|}{\text{croît et}} & \text{max. loc.}
& \text{décroît et}\\
& & \multicolumn{3}{|c|}{\text{concave}} & \text{et concave} & \text{concave}
\end{array} \quad \ldots
\]
\[ \ldots \quad 
\begin{array}{|c|c|c|c|c}
0 & 0<x<1 & 1 & 1 < x & +\infty  \\
\hline
0 & - & 1 & - &  +\infty \\
- & - & 0 & + & \\
0 & + & + & + & \\
\hline
\text{point} & \text{décroît et} & \text{min. loc.} &
\text{croît et} & \\
\text{d'inflexion} & \text{et convexe} & \text{et convexe} &
\text{convexe} &  
\end{array}
\]
Le graphe de $h$ est
\MATHgraph{6_derivees_appl/graph_funct10}{8cm}

\subQ{b} Nous avons $\displaystyle h'(x) = 3x^2 -12 x -15 = 3(x-5)(x+1)$ et
$\displaystyle h''(x)=6x-12$.

$h'(x)$ est définie pour tout $x$ et $h'(x) = 0$ pour $x=-1$ et $x=5$.
Les points critiques sont $x=-1$ et $x=5$.  $h''(x) = 0$ pour $x=2$.  Donc
$x=2$ est un candidat pour un point d'inflexion.  Finalement,
$\displaystyle \lim_{x\rightarrow \infty} h(x) = +\infty$ et
$\displaystyle \lim_{x\rightarrow -\infty} h(x) = -\infty$.

Nous résumons l'information que nous avons obtenues dans le tableau suivant.
\[
\begin{array}{c|c|c|c|c|c|}
x & -\infty & x<-1 & -1 & -1<x<2 & 2 \\
\hline
f(x) & -\infty & -|+ & 11 & +|- & -43 \\
f'(x) &  & + & 0 & - & - \\
f''(x) &  & - & - & - & 0 \\
\hline
& & \text{croît et} & \text{max. local} & \text{décroît et} & \text{point} \\
& & \text{concave} & \text{et concave} & \text{concave} & \text{d'inflexion}
\end{array} \quad \ldots
\]
\[
\qquad \ldots \begin{array}{|c|c|c|c}
2<x<5 & 5 & 5<x & +\infty \\
\hline
- & -97 & -|+ & +\infty \\
- & 0 & + & \\
+ & + & + & \\
\hline
\text{décroît et} & \text{min. loc.} & \text{croît} &  \\
\text{convexe} & \text{et convexe} & \text{convexe} &
\end{array}
\]
La fonction $f$ change de signe dans les intervalles où nous retrouvons un des
symboles $+|-$ ou $-|+$.  Comme $f$ est continue, il existe un point dans
un tel intervalle où la fonction $f$ est nulle.

Le graphe de $f$ est
\MATHgraph{6_derivees_appl/graph_funct6}{8cm}

\subQ{c} Nous avons
\[
f(x) = x + \frac{4}{x^2} \quad , \quad
f'(x) = 1 - \frac{8}{x^3} \quad \text{et} \quad f''(x) = \frac{24}{x^4} \; .
\]
Ainsi, la fonction $f$ n'est pas définie en $x=0$, elle est
positive pour $x > -4^{1/3}$ et négative pour $x<-4^{1/3}$.

La fonction $f'$ n'est pas définie en $x=0$.  Elle est égale à $0$ au point
$x=2$.  De plus, sauf au point $x=0$ où la dérivée n'existe pas,
$f'(x)$ est négatif pour $0<x<2$ et positif pour $x<0$ et $x>2$.

La fonction $f''$ n'est pas définie en $x=0$ et est positive pour tout
$x\neq 0$.

Puisque
\[
\lim_{x\rightarrow \infty} f(x) = \infty  \quad \text{et} \quad
\lim_{x\rightarrow -\infty} f(x) = -\infty \; ,
\]
il n'y a aucune asymptote horizontale.  Par contre,
\[
\lim_{x\rightarrow 0^-} f(x) = \lim_{x\to 0^+} f(x) = +\infty  \ .
\]
La partie positive de l'axe des $x$ est donc une asymptote verticale.

Nous résumons l'information que nous avons obtenues le tableau suivant.
\[
\begin{array}{c|c|c|c|c|c|}
x & -\infty & -\infty<x<-4^{1/3} & -4^{1/3} & -4^{1/3}<x<0 & 0 \\
\hline
f(x) & -\infty & - & 0 & + & \text{N.D.} \\
f'(x) & & + & + & + & \text{N.D.} \\
f''(x) &  & + & + & + & \text{N.D.} \\
\hline
& & \multicolumn{3}{|c|}{\text{croît et}} & \text{asymptote} \\
& & \multicolumn{3}{|c|}{\text{convexe}} & \text{verticale}
\end{array} \quad \ldots
\]
\[ \quad \ldots  \quad
\begin{array}{|c|c|c|c}
 0<x<2 & 2 & 2<x<\infty & \infty \\
\hline
+ & + & + & +\infty \\
- & 0 & + & \\
+ & + & + & \\
\hline
\text{décroît et} & \text{min. local} & \text{croît et} & \\
\text{convexe} & \text{et convexe} & \text{convexe} &
\end{array}
\]
Le graphe de $f$ est donné ci-dessous.
\MATHgraph{6_derivees_appl/figFX3}{8cm}

\subQ{d} Nous avons $\displaystyle f'(x) = -e^x + (1-x)e^x = - x e^x$ et
$\displaystyle f''(x) = - e^x - x e^x = -(1+x)e^x$.

La fonction $f(x)$ est nulle pour $x=1$, positive pour $x<1$ et
négative pour $x>1$.

La fonction $f'$ est définie sur toute la droite réelle et $f'(x)=0$
seulement pour $x=0$.  Donc $x=0$ est le seul point critique de $f$.
Nous avons $f'(x) < 0$ pour $x>0$ et $f'(x) >0$ pour $x<0$.

La fonction $f''$ est définie sur toute la droite réelle et $f''(x) = 0$
seulement pour $x=-1$.  Donc $x=-1$ est un candidat pour un point
d'inflexion.  Nous avons que $f''(x) < 0$ pour $x>-1$ et $f''(x)>0$ pour $x<-1$.

Finalement,
\[
\lim_{x\rightarrow \infty} f(x) = \lim_{x\rightarrow \infty} (1-x)e^x =
-\infty
\quad \text{et} \quad 
\lim_{x\rightarrow -\infty} f(x) = \lim_{x\rightarrow -\infty}
(1-x)e^x = 0 \ .
\]
Nous pouvons vérifier numériquement ces limites à l'aide de suites
$\displaystyle \left\{x_n\right\}_{n=0}^\infty$ qui tendent vers plus ou
moins l'infini selon le cas.

Nous résumons l'information que nous avons obtenues dans le tableau suivant.
\[
\begin{array}{c|c|c|c|}
x & -\infty & x<-1 & -1 \\
\hline
f(x) & 0 & + & 2e^{-1} \\
f'(x) &  & + & + \\
f''(x) & & + & 0 \\ 
\hline
& \text{asymptote} & \text{croît et} & \text{point} \\
& \text{horizontale} & \text{convexe} & \text{d'inflexion}
\end{array} \quad \ldots
\]
\[
\qquad \ldots  \quad 
\begin{array}{|c|c|c|c|c|c}
-1<x<0 & 0 & 0<x<1 & 1 & 1<x & +\infty \\
\hline
+ & 1 & + & 0 & - & -\infty \\
+ & 0 & - & - & - & \\
- & - & - & - & - & \\
\hline
\text{croît et} & \text{max. local} &
\multicolumn{3}{|c|}{\text{décroît et}} \\
\text{concave} & \text{et concave} &
\multicolumn{3}{|c|}{\text{concave}}
\end{array}
\]
Le graphe de $f$ est ci-dessous.
\MATHgraph{6_derivees_appl/graph_funct11}{8cm}

\subQ{e} Nous avons $\displaystyle g(z) = \frac{e^z}{z^2} = z^{-2} e^z$.
Ainsi,
\begin{align*}
g'(z) &= -2z^{-3}e^x + z^{-2}e^z = \left(z^{-2} -2z^{-3}\right) e^z
= \left( z-2\right) z^{-3} e^z
\intertext{et}
g''(z) &= \left( -2z^{-3} + 6z^{-4}\right)e^z + \left(z^{-2} -2z^{-3}\right) e^z
= \left(z^{-2} - 4z^{-3} + 6z^{-4}\right)e^z \\
&= \left(z^2 - 4z + 6\right) z^{-4}e^z \ .
\end{align*}
La fonction $g$ est définie sur toute la droite réelle sauf à l'origine.
De plus, $g(z)>0$ pour tout $z \neq 0$.

La fonction $g'$ est aussi définie sur toute la droite réelle sauf à
l'origine.  De plus $g'(z) = 0$ si $z=2$.  Nous avons donc un point critique
à $x=2$.

Comme $g$ et $g'$, la fonction $g''$ est aussi définie sur toute la droite
réelle sauf à l'origine.  De plus, $g''(z)\neq 0$ pour tout $z \neq 0$
car $z^2 - 4 z + 6$ n'a pas de racines réelles.  Comme $g''(z)$  est
positif pour $z = 0$, nous avons donc que $g''(z)>0$ pour toutes valeurs de
$z$.

À l'aide de suites de nombres, nous pouvons vérifier que
\begin{align*}
\lim_{x\rightarrow \infty} g(x) &= \lim_{x\rightarrow \infty} \frac{e^z}{z^2}
= \infty
\intertext{et}
\lim_{x\rightarrow -\infty} f(x) &= \lim_{x\rightarrow -\infty}
\frac{e^z}{z^2} = 0 \ .
\end{align*}
De plus,
\[
\lim_{x\rightarrow 0} g(x) = \lim_{x\rightarrow 0} \frac{e^z}{z^2}
= \infty \ .
\]

Nous résumons l'information que nous avons obtenues dans le tableau suivant.
\[
\begin{array}{c|c|c|c|c|c|c|c}
z & -\infty & x<0 & 0 & 0 < x < 2 & 2 & 2 < x & \infty \\
\hline
g(z) & 0 & + & \text{N.D.} & + & e^2/4 & + & +\infty \\
g'(z) & & + & \text{N.D.} & - & 0 & + & \\
g''(z) & & + & \text{N.D.} & + & + & + &  \\ 
\hline
& \text{asymptote} & \text{croît et} & \text{asymptote} &
\text{décroît et} & \text{min. loc.} & \text{croît et} & \\
& \text{horizontale} & \text{convexe} & \text{verticale} & 
\text{convexe} & \text{convexe} & \text{convexe} &
\end{array}
\]

Le graphe de $g$ est ci-dessous.
\MATHgraph{6_derivees_appl/graph_funct12}{8cm}

\subQ{g} Nous avons
\begin{align*}
T'(t) &= -2 t e^t + (1-t^2)e^t = (1-2t-t^2) e^t
\intertext{et}
T''(t) &= (-2-2t)e^t +(1-2t-t^2) e^t = (-1-4t-t^2)e^x \ .
\end{align*}
La fonction $T$ ainsi que ses dérivées sont définies sur toute la droite
réelle.  Nous avons que $T'(t)=0$ pour $t_1 = -1+\sqrt{2} \approx 0.41421$ et
$t_2 = -1 - \sqrt{2}$.  Puisque $t_2 < -1$, le point $t_1$ est le seul
point critique dans l'intervalle $[-1,1]$.  Nous avons que
$T''(t)=0$ pour $s_1 = -2+\sqrt{3} \approx -0.267949$ et
$s_2 = -2 - \sqrt{3}$.  Puisque $s_2<-1$, le point.$s_1$ est le
seul point dans l'intervalle $[-1,1]$ où $T''(s_1)=0$.  C'est un
candidat pour être un point d'inflexion.

Nous résumons l'information que nous avons obtenues dans le tableau suivant.
\[
\begin{array}{c|c|c|c|c|c|c|c}
t & -1 & -1<t<s_1 & s_1 & s_1 < t < t_1 & t_1 & t_1 < t <1 & 1 \\
\hline
T(t) & 0 & + & (1-s_1^2)e^{s_1} & + & (1-t_1^2)e^{t_1} & + & 0 \\
T'(t) & + & + & + & + & 0 & + & + \\
T''(t) & + & + & 0 & - & - & - & -  \\ 
\hline
& \multicolumn{2}{c|}{\text{croît et}} & \text{croît et} & \text{croît et} &
\text{max. local} & \multicolumn{2}{|c}{\text{décroît et}} \\
& \multicolumn{2}{c|}{\text{convexe}} & \text{point} & 
\text{concave} & \text{et concave} & \multicolumn{2}{c}{\text{concave}} \\
& \multicolumn{2}{c|}{} & \text{d'inflexion} & & & \multicolumn{2}{|c}{}
\end{array}
\]
Le graphe de $T$ est ci-dessous.
\MATHgraph{6_derivees_appl/graph_funct14}{8cm}

\subQ{h} Puisque $G(x) = \sqrt{x} \, e^{-x} = x^{1/2} e^{-x}$, nous obtenons
\begin{align*}
G'(x) &= \frac{1}{2} x^{-1/2} e^{-x} - x^{1/2} e^{-x} =
\left( \frac{1}{2} x^{-1/2} - x^{1/2}\right) e^{-x} =
\left( \frac{1}{2} -x \right) x^{-1/2} e^{-x}
\intertext{et}
G''(x) &= \left( -\frac{1}{4} x^{-3/2} - \frac{1}{2} x^{-1/2} \right) e^{-x}
- \left( \frac{1}{2} x^{-1/2} - x^{1/2}\right) e^{-x}
= \frac{1}{4} \left( -1 - 4x + 4x^2\right) x^{-3/2} e^{-x} \ .
\end{align*}

La fonction $G$ est définie sur l'intervalle $[0,\infty[$ alors que les
fonctions $G'$ et $G''$ sont définie sur $]0,\infty[$.  Nous allons
tracer le graphe de $G$ pour $x\geq 0$.

Nous avons que $G(x) \geq 0$ pour tout $x\geq 0$.

$G'(x)=0$ pour $x = 1/2$ et $G'$ n'est pas définie à l'origine.
Le seul point critique est $x=1/2$.  De plus,
$G''(x)=0$ pour $x = x_1 = (1+\sqrt{2})/2 \approx 1.20711$ (la racine
positive du polynôme $4x^2- 4x-1$.  C'est un candidat pour être un point
d'inflexion.

À l'aide de suites de nombres, nous pouvons vérifier que
\[
\lim_{x\rightarrow \infty} G(x) = \lim_{x\rightarrow \infty} \sqrt{x}\,
e^{-x} = 0 \ .
\]
Nous résumons l'information que nous avons obtenues dans le tableau suivant.
\[
\begin{array}{c|c|c|c|c|c|c|c}
x & 0 & 0<x<1/2 & 1/2 & 1/2 < x < x_1 & x_1 & x_1 < x & +\infty \\
\hline
G(x) & 0 & + & \rule{0em}{1em} e^{-1/2}/\sqrt{2} & + &
\sqrt{x_1} \, e^{-x_1} & + & 0 \\
G'(x) & \text{N.D.} & + & 0 & - & - & - & \\
G''(x) & \text{N.D.} & - & - & - & 0 & + & \\ 
\hline
& & \text{croît et} & \text{max. local} & \text{décroît et} & \text{décroît} &
\text{décroît et} & \\
& & \text{concave} & \text{et concave} & \text{concave} & \text{et point}
& \text{convexe} & \\
& & & & & \text{d'inflexion} & &
\end{array}
\]
Le graphe de $G$ est ci-dessous.
\MATHgraph{6_derivees_appl/graph_funct15}{8cm}

\subQ{i} Nous avons $\displaystyle f'(x) = \frac{3x^2}{(1+x^3)^2}$ et
$\displaystyle f''(x) = \frac{6x(1-2x^3)}{(1+x^3)^3}$.

La fonction $f$ est définie sur toute la droite réelle sauf à $-1$.
$f(x)>0$ pour $x>0$ et $x<-1$, $f(x)<0$ pour $-1<x<0$ et $f(x)=0$
seulement pour $x=0$.

La fonction $f'$ est aussi définie sur toute la droite réelle sauf à $-1$.
$f'(x) = 0$ à $x=0$ et $f'(x) > 0$ pour tout $x$ différent de $-1$ et $0$.

La fonction $f''$ est définie sur toute la droite réelle sauf à $-1$.
$f''(x)=0$ pour $x=0$ et $x=2^{-1/3}$.  $f''(x)>0$ pour $x<-1$ et
$0<x< 2^{-1/3}$.  $f''(x)<0$ pour $-1<x<0$ et $x>2^{-1/3}$.

De plus,
\begin{align*}
\lim_{x\rightarrow \pm\infty} f(x) &=\lim_{x\rightarrow \pm\infty} \frac{x^3}{1+x^3}
=\lim_{x\rightarrow \pm\infty} \frac{1}{(1/x^3) + 1}
=\frac{1}{\displaystyle \lim_{x\rightarrow \pm\infty}(1/x^3) + 1} = 1 \ , \\
\lim_{x\rightarrow -1^+} f(x) &= +\infty
\intertext{et}
\lim_{x\rightarrow -1^-} f(x) &= -\infty \ .
\end{align*}
Nous ne pouvons pas utiliser la Règle de l'Hospital pour évaluer les
limites $\displaystyle \lim_{x\rightarrow -1^+} f(x)$ et
$\displaystyle \lim_{x\rightarrow -1^-} f(x)$.  Pourquoi?

Nous résumons l'information que nous avons obtenues dans le tableau suivant.
\[
\begin{array}{c|c|c|c|c|c|}
x & -\infty & x<-1 & -1 & -1 < x < 0 & 0 \\
\hline
f(x) & 1 & + & \text{N.D.} & - & 0 \\
f'(x) &  & + & \text{N.D.} & + & 0 \\
f''(x) & & + & \text{N.D.} & - & 0 \\
\hline
& \text{asymptote} & \text{croît et} & \text{asymptote} & \text{croît et} &
\text{point} \\
& \text{horizontale} & \text{convexe} & \text{verticale} & \text{concave}
& \text{critique et} \\
& & & & & \text{d'inflexion}
\end{array} \quad \ldots
\]
\[
\qquad \ldots  \quad
\begin{array}{|c|c|c|c}
0<x<2^{-1/3} & 2^{-1/3} & 2^{-1/3} <x & +\infty \\
\hline
+ & 1/3 & + & 1 \\
+ & + & + & \\
+ & 0 & - & \\
\hline
\text{croît et} & \text{point} & \text{croît et} & \text{asymptote} \\
\text{convexe} & \text{d'inflexion} & \text{concave} & \text{horizontale}
\end{array}
\]
Le graphe de $f$ est ci-dessous.
\MATHgraph{6_derivees_appl/graph_funct16}{8cm}

\subQ{j} Le domaine de la fonction est $\RR\setminus \{0\}$.
Nous avons que
\[
\lim_{x\rightarrow \infty} f(x)
= \lim_{x\rightarrow \infty} \frac{2}{x^2} - \lim_{n\to \infty} \frac{3}{x^3}
= 0
\quad \text{et} \quad
\lim_{x\rightarrow -\infty} f(x) =
\lim_{x\rightarrow -\infty} \frac{2}{x^2} - \lim_{n\to -\infty} \frac{3}{x^3} = 0
\ .
\]
L'axe des $x$ est une asymptote horizontale lorsque $x\to \infty$ et
$x\to -\infty$.  De plus, à l'aide d'un tableau de valeurs, nous pouvons
vérifier que
\[
\lim_{x\rightarrow 0^+} f(x) = 
\lim_{x\rightarrow 0^+} \left( \frac{2x-3}{x^3} \right) = -\infty
\quad \text{et} \quad
\lim_{x\rightarrow 0^-} f(x) = 
\lim_{x\rightarrow 0^-} \left( \frac{2x-3}{x^3} \right) = +\infty \ .
\]
La partie positive de la droite $x=0$ est donc une asymptote verticale
lorsque $x$ converge vers $0$ par la gauche et la partie négative de la
droite $x=0$ est une asymptote verticale lorsque $x$ converge vers $0$
par la droite.

Nous avons que $\displaystyle f(x)= \frac{2x-3}{x^3} = 0$ pour $x = 3/2$.
Il découle de
\[
f'(x) = -4x^{-3} + 9 x^{-4} = -\frac{4x-9}{x^4}
\quad \text{et} \quad
f''(x) = 12x^{-4} - 36 x^{-5} = 12 \left(\frac{x-3}{x^5}\right)
\]
que le point $x=9/4$ où $f'(x)=0$ est le seul point critique pour $f$
et le point $x=3$ où $f''(x)=0$ est le seul point critique pour $f'$.
Ce dernier point est possiblement un point d'inflexion.

Nous résumons dans les tableaux suivants l'information que nous avons pu
déduire de $f$, $f'$ et $f''$.
\[
\begin{array}{c|c|c|c|c|c|c|}
x & -\infty & x<0 & 0 & 0<x<3/2 & 3/2 & 3/2<x<9/4 \\
\hline
f(x) & 0 & + & \text{N.D.} & - & 0 & + \\
\hline
f'(x) & & + & \text{N.D.} & + & + & + \\
\hline
f''(x) & & + & \text{N.D.} & - & - & - \\
\hline
& \text{asymptote} & \text{croît et} & \text{asymptote} &
\multicolumn{3}{|c|}{\text{croît et}} \\
& \text{horizontale} & \text{convexe} & \text{verticale} &
\multicolumn{3}{|c|}{\text{concave}}
\end{array} \quad \ldots
\]
\[
\ldots \quad 
\begin{array}{|c|c|c|c|c}
9/4 & 9/4<x<3 & 3 & 3<x<\infty & \infty \\
\hline
+ & + & + & + & 0 \\
\hline
0 & - & - & - & \\
\hline
- & - & 0 & + & \\
\hline
\text{max. local} & \text{décroît et} & \text{point} &
\text{décroît et} & \text{asymptote} \\
\text{et concave} & \text{concave} & \text{d'inflexion} &
\text{convexe} & \text{horizontale}
\end{array}
\]
\MATHgraph{6_derivees_appl/graph_funct23}{8cm}

\subQ{k} Le domaine de la fonction est $\RR$ au complet.
Nous avons que $f(x)=0$ seulement à $x=-4$ et $f(0) = 4$.

Il découle de
\[
f'(x) = e^{-x}-(x+4)e^{-x} = -(x+3)e^{-x} \quad \text{et} \quad
f''(x) = -e^{-x} + (x+3)e^{-x} = (x+2)e^{-x}
\]
que la fonction $f$ a un seul point critique à $x=-3$ et la fonction $f'$
a un seul point critique à $x=-2$.  Ce dernier point est un candidat
possible pour un point d'inflexion.

Grâce à la Règle de l'Hospital, nous avons
\[
\lim_{x\rightarrow \infty} f(x)
= \underbrace{\lim_{x\rightarrow \infty} \frac{x+4}{e^x}
}_{\text{type }\infty/\infty}
= \lim_{x\rightarrow \infty} \frac{1}{e^x} = 0 \ .
\]
De plus,
\[
\lim_{x\rightarrow -\infty} f(x) = \lim_{x\rightarrow -\infty} (x+4)e^{-x}
= -\infty
\]
car
\[
\lim_{x\rightarrow -\infty} e^{-x} = \lim_{x\rightarrow \infty} e^x
= \infty \quad \text{et} \quad
\lim_{x\rightarrow -\infty} x+4 = -\infty \ .
\]

Nous résumons dans le tableau suivant l'information que nous avons pu
déduire de $f$, $f'$ et $f''$
\[
\begin{array}{c|c|c|c|c|c|c|}
x & -\infty & x<-4 & -4 & -4<x<-3 & -3 & -3<x<-2 \\
\hline
f(x) & -\infty & - & 0 & + & e^3 & + \\
\hline
f'(x) & & + & + & + & 0 & - \\
\hline
f''(x) & & - & - & - & - & - \\
\hline
& & \multicolumn{3}{|c|}{\text{croît et}} & \text{max. local} &
\text{décroît et} \\
& & \multicolumn{3}{|c|}{\text{concave}} & \text{et concave} &
\text{concave}
\end{array}\quad \ldots
\]
\[
\ldots \quad \begin{array}{|c|c|c}
-2 & -2<x & +\infty \\
\hline
2e^2 & + & 0 \\
\hline
- & - & \\
\hline
0 & + & \\
\hline
\text{point} & \text{décroît et} & \text{asymptote} \\
\text{d'inflexion} & \text{convexe} & \text{horizontale}
\end{array}
\]

Nous obtenons le graphe suivant.
\MATHgraph{6_derivees_appl/graph_funct21}{8cm}

\subQ{l} Le domaine de la fonction est $\RR \setminus \{0\}$.
Nous avons que
\begin{align*}
\lim_{x\rightarrow \infty} f(x) &= 
\lim_{x\rightarrow \infty} \frac{2}{x} - \lim_{x\rightarrow \infty} \frac{6}{x^3}
= 0 - 0 = 0
\intertext{et}
\lim_{x\rightarrow -\infty} f(x) &= 
\lim_{x\rightarrow -\infty} \frac{2}{x} - \lim_{x\rightarrow -\infty} \frac{6}{x^3}
= 0 - 0 = 0 \quad.
\end{align*}
L'axe des $x$ est donc une asymptote horizontale lorsque $x$ converge
vers plus ou moins l'infini.
De plus,
\[
\lim_{x\rightarrow 0^+} f(x) =
\lim_{x\rightarrow 0^+} \frac{2(x^2-3)}{x^3} = -\infty
\quad \text{et} \quad
\lim_{x\rightarrow 0^-} f(x) = 
\lim_{x\rightarrow 0^-} \frac{2(x^2-3)}{x^3} = +\infty \ .
\]
La partie positive de l'axe des $y$ est donc une asymptote verticale
lorsque $x$ converge vers $0$ par la droite et la partie négative de
l'axe des $y$ est une asymptote verticale lorsque $x$ converge vers
$0$ par la gauche.

Nous avons que $\displaystyle f(x)= \frac{2(x^2-3)}{x^3} = 0$ à
$x=\pm\sqrt{3}$.  Il découle de
\[
f'(x) = -\frac{2}{x^2} + \frac{18}{x^4} = -\frac{2(x^2-9)}{x^4} \quad
\text{et} \quad 
f''(x) = \frac{4}{x^3} - \frac{72}{x^5}
= \frac{4(x^2-18)}{x^5}
\]
que les points $x=\pm 3$ où $f'(x)=0$ sont les seuls points critiques
de $f$ et les points $x=\pm 3\sqrt{2}$ où $f''(x)=0$ sont les seuls
points critiques de $f'$.  C'est deux derniers points sont possiblement
des points d'inflexion. 

Nous résumons dans les tableaux suivants l'information que nous avons pu
déduire de $f$, $f'$ et $f''$.  Notons que la fonction $f$ est impaire
$f(-x)=-f(x)$.  Il suffit donc de tracer le graphe de $f(x)$ pour
$x>0$.  Le graphe de $f(x)$ pour $x<0$ est la réflection par rapport à
l'origine du graphe de $f(x)$ pour $x > 0$.
\[
\begin{array}{c|c|c|c|c|c|}
x & -\infty & x< -3\sqrt{2} & -3\sqrt{2} & -3\sqrt{2} <x< -3 & -3 \\
\hline
f(x) & 0 & - & 0 & - & - \\
\hline
f'(x) & & - & - & - & 0  \\
\hline
f''(x) & & - & 0 & + & + \\
\hline
& \text{asymptote} & \text{décroît et} & \text{point} & \text{décroît et} &
\text{min. local}  \\
& \text{horizontale} & \text{concave} & \text{d'inflexion} & \text{convexe} &
\text{et convexe}
\end{array} \quad \ldots
\]
\[
\ldots \begin{array}{|c|c|c|c|c|c|c|}
-3 <x< -\sqrt{3} & -\sqrt{3} & \sqrt{3} <x< 0 & 0 & 0 <x< \sqrt{3}
& \sqrt{3} & \sqrt{3} <x< 3  \\
\hline
- & 0 & + & \text{N.D.} & - & 0 & + \\
\hline
+ & + & + & \text{N.D.} & + & + & + \\
\hline
+ & + & + & \text{N.D.} & - & - & - \\
\hline
\multicolumn{3}{|c|}{\text{croît et}} & \text{asymptote}
& \multicolumn{3}{|c|}{\text{croît et}} \\
\multicolumn{3}{|c|}{\text{convexe}} & \text{verticale}
& \multicolumn{3}{|c|}{\text{concave}}
\end{array} \ldots
\]
\[
\ldots \quad \begin{array}{|c|c|c|c|c}
3 & 3 <x< 3\sqrt{3} & 3\sqrt{2} & 3\sqrt{2} <x & +\infty \\
\hline
+ & + & + & + & 0 \\
\hline
0 & - & - & - & \\
\hline
- & - & 0 & + & \\
\hline
\text{max. local} & \text{décroît et} & \text{point} & \text{décroît et} &
\text{asymptote} \\
\text{et concave} & \text{concave} & \text{d'inflexion} & \text{convexe} &
\text{horizontale}
\end{array}
\]

Le graphe de la fonction est donné ci-dessous.
\MATHgraph{6_derivees_appl/graph_funct22}{8cm}

\subQ{m} Le domaine de la fonction est $\RR\setminus \{1\}$.
À l'aide de la Règle de l'Hospital, nous avons que
\[
\lim_{x\rightarrow \infty} f(x) =
\underbrace{\lim_{x\rightarrow \infty} \frac{e^x}{x-1}}_{\text{type }\infty/\infty}
= \lim_{x\rightarrow \infty} \frac{e^x}{1} = +\infty \ .
\]
De plus,
\[
\lim_{x\rightarrow -\infty} f(x)
= \left(\lim_{x\rightarrow -\infty} \frac{1}{x}\right)
\left(\lim_{x\rightarrow -\infty} e^x\right) = 0 \times 0 = 0 \ .
\]
L'axe des $x$ est une asymptote horizontale lorsque $x\to -\infty$.

À l'aide d'un tableau de valeurs, nous pouvons vérifier que
\[
\lim_{x\rightarrow 1^+} f(x) = +\infty \quad \text{et} \quad
\lim_{x\rightarrow 1^-} f(x) = -\infty \ .
\]
La partie positive de la droite $x=1$ est donc une asymptote verticale
lorsque $x$ converge vers $1$ par la droite et la partie négative de la
droite $x=1$ est une asymptote verticale lorsque $x$ converge vers $0$
par la gauche.

Nous avons que $\displaystyle f(x)= \frac{e^x}{x-1} \neq 0$ pour tout $x$.
Il découle de
\begin{align*}
f'(x) &= \frac{e^x (x-1) - e^x}{(x-1)^2} = \frac{(x-2)e^x}{(x-1)^2}
\intertext{et}
f''(x) &= \frac{(e^x+(x-2)e^x)(x-1)^2 - 2(x-2)(x-1)e^x}{(x-1)^4} \\
&= \frac{(x-1)^3 e^x - 2(x-2)(x-1)e^x}{(x-1)^4}
= \frac{(x-1)^2 e^x - 2(x-2)e^x}{(x-1)^3}
= \frac{(x^2-4x+5) e^x}{(x-1)^3}
\end{align*}
que $x=2$ est le seul point critique pour $f$ et que
$f'$ a aucun point critique.

Nous résumons dans le tableau suivant l'information que nous avons pu
déduire de $f$, $f'$ et $f''$.
\[
\begin{array}{c|c|c|c|c|c|c|c}
x & -\infty & x<1 & 1 & 1<x<2 & 2 & 2<x & +\infty \\
\hline
f(x) & 0 & - & \text{N.D.} & + & + & + & +\infty \\
\hline
f'(x) & & - & \text{N.D.} & - & 0 & + & \\
\hline
f''(x) & & - & \text{N.D.} & + & + & + & \\
\hline
& \text{asymptote} & \text{décroît et} & \text{asymptote} &\text{décroît et} &
\text{min. local} & \text{croît et} & \\
& \text{horizontal} & \text{concave} & \text{vertical} & \text{convexe}
& \text{et convexe} & \text{convexe} &
\end{array}
\]

\MATHgraph{6_derivees_appl/graph_funct24}{8cm}

\subQ{n}  Le domaine de la fonction est $\RR\setminus \{-1\}$.
Puisque
\[
\lim_{x\to \infty} f(x) = \lim_{x\to \infty} \frac{x}{x^3+1}
= \lim_{x\to \infty} \frac{1/x^2}{1+1/x^3} = 0
\]
car $\displaystyle \lim_{x\to \infty} 1/x^r = 0$ pour $r>0$, nous avons que
l'axe des $x$ est une asymptote horizontale pour $x>0$.  De même, nous
avons que l'axe des $x$ est aussi une asymptote horizontale pour $x<0$.

De plus, à l'aide de suites numériques, nous pouvons voir que
$\displaystyle \lim_{x\to -1^-} f(x)
= \lim_{x\to -1^-} \frac{x}{x^3+1} = +\infty$
car $x$ approche $-1$ et $x^3+1<0$ approche $0$ lorsque $x \to -1^-$.
De même, $\displaystyle \lim_{x\to -1^+} f(x)
= \lim_{x\to -1^+} \frac{x}{x^3+1} = -\infty$
car $x$ approche $-1$ et $x^3+1>0$ approche $0$ lorsque $x \to -1^+$.
La droite $x=1$ est donc une asymptote verticale.

Il découle de $\displaystyle f'(x) = \frac{-2x^3+1}{(x^3+1)^2}$ et
$\displaystyle f''(x) = \frac{6x^2(x^3-2)}{(x^3+1)^3}$ que
$x = 2^{-1/3}$ est le seul point critique pour $f$ et que
$x=0$ et $x = 2^{1/3}$ sont les seuls points critiques pour
$f'$.  C'est deux dernier points sont des candidats possibles pour
être des points d'inflexion.

Nous résumons dans le tableau suivant l'information que nous avons pu
déduire de $f$, $f'$ et $f''$.
\[
\begin{array}{c|c|c|c|c|c|c|c|}
x & -\infty & x< -1&-1& -1 <x< 0 & 0 & 0 <x< 2^{-1/3} & 2^{-1/3} \\
\hline
f(x) & 0 &+ & \text{N.D.} & - & 0 &+ & + \\
\hline
f'(x) & & + & \text{N.D.} & + & + & + & 0  \\
\hline
f''(x) & & + & \text{N.D.} & - & 0 & - & - \\
\hline
& & \text{croît et} & \text{asymptote} & \text{croît et} & \text{croît}
& \text{croît et} & \text{max. local}  \\
& & \text{convexe} & \text{verticale} & \text{concave} & &
\text{concave} & \text{et concave}
\end{array} \quad \ldots
\]
\[
\ldots \quad \begin{array}{|c|c|c|c}
2^{-1/3} <x< 2^{1/3} & 2^{1/3}& 2^{1/3} <x & +\infty\\
\hline
 + & + & + & 0\\
\hline
 - & - & - & \\
\hline
 - & 0 & + & \\
\hline
\text{décroît et} & \text{point} & \text{décroît et} & \text{asymptote} \\
\text{concave} & \text{d'inflexion} & \text{convexe} & \text{horizontale}
\end{array}
\]
\MATHgraph{6_derivees_appl/graph_funct25}{8cm}

\subQ{o} Le domaine de $f$ est $]0,\infty[$ car $\ln(x)$
est seulement définie pour $x>0$.

Puisque $x>0$, nous avons que $f(x) = x \ln(x) = 0$ seulement lorsque
$\ln(x) = 0$; c'est-à-dire, lorsque $x=1$.

Nous avons que $\displaystyle \lim_{x \to +\infty} f(x) = \infty$ car
$x \to \infty$ et $\ln(x) \to \infty$ lorsque $x \to \infty$.
Nous avons une limite du type $(0\cdot\infty)$ lorsque $x\to 0^+$.  Or
\[
\lim_{x\to 0^+}  x \ln(x) = \lim_{x\to 0^+} \frac{\ln(x)}{1/x} \ .
\]
Nous avons maintenant une limite du type $(\infty/\infty)$ pour laquelle nous
pouvons utiliser la Règle de l'Hospital pour obtenir
\[
\lim_{x\to 0^+} \frac{\ln(x)}{1/x}
= \lim_{x\to 0^+} \frac{1/x}{-1/x^2}
= \lim_{x\to 0^+} -x = 0 \ .
\]

Il découle de $f'(x) = \ln(x) + 1$ et $f''(x) = 1/x$ que $x= 1/e$
est le seul point critique pour $f$.  Il n'y a pas de point critique
pour $f'$.

Nous résumons dans le tableau suivante l'information que nous avons pu obtenir
de $f'$ et $f''$.
\[
\begin{array}{c|c|c|c|c|c|c|c}
x & 0 & x< 1/e & 1/e & 1/e <x< 1 & 1 & 1 < x & +\infty\\
\hline
f(x) & 0 & - & - & -  & 0 & + & \infty\\
\hline
f'(x) & & - & 0 & + & + & + &  \\
\hline
f''(x) & & + & + & + & + & + &  \\
\hline
& & \text{décroît et} & \text{min local.} &
\multicolumn{3}{|c|}{\text{croît et}} & \\
& & \text{convexe} & \text{et convexe} & \multicolumn{3}{|c|}{\text{convexe}} &
\end{array}
\]

Le graphe de $f$ est donné ci-dessous.
\MATHgraph{6_derivees_appl/graph_funct26}{8cm}

Le point $(0,0)$ ne fait pas parti du graphe de $f$.
}

\compileSOL{\SOLUb}{\ref{6Q22}}{
\subQ{a} Puisque $f(x) = \sin^2(x) - 2 \cos(x)$ est une fonction de
période $2\pi$, il suffit d'étudier la fonction entre $0$ et $2\pi$.
\[
  f'(x) = 2 \sin(x) \cos(x) + 2\sin(x) = 2\sin(x) (\cos(x) + 1)
\]
et
\begin{align*}
  f''(x) &= 2 \cos^2(x) -2\sin^2(x) + 2\cos(x)
  = 2\cos^2(x) -2(1-\cos^2(x)) + 2\cos(x) \\
 & = 4 \cos^2(x) + 2\cos(x) - 2 \ .
\end{align*}
Nous avons que
\[
  f'(x)=0 \Rightarrow \sin(x) = 0 \text{ ou } \cos(x) = -1
  \Rightarrow x=0, \pi \text{ ou } x = \frac{3\pi}{2} \ .
\]
Il y a trois points critiques dans l'intervalle $[0,2\pi[$: $0$, $\pi$
et $3\pi/2$.  Nous avons que $f'(x) > 0$ pour $0 < x < \pi$ 
et $f'(x) < 0$ pour $\pi < x < 3\pi/2$ et $3\pi/2 <x < 2\pi$.
Donc $f$ a un maximum local à $x=\pi$ et un minimum local à
$x = 2\pi$, alors que $x=3\pi/2$ n'est ni un maximum ou un minimum local.

Les points critiques de $f'$ sont les solutions de $z = \cos(x)$ pour
$x$ dans l'intervalle $[0,2\pi[$ où $z$ est une racine de
$2z^2 + z -1$.  Les racines de ce polynôme sont $z=-1$ ou $1/2$.
Ainsi, $\cos(x) = -1$ pour $x=\pi$, et $\cos(x) = 1/2$ pour $x=\pi/3$
et $5\pi/3$.  Ce sont nos candidats pour les points d'inflexion.  Nous avons
$f''(x) > 0$ pour $0 < x < \pi/3$ et $5\pi/3 < x < 2\pi$, et $f'(x)<0$
pour $\pi/3 < x < \pi$ et $\pi < x < 5\pi/3$.  Nous avons deux points
d'inflexion: $x=\pi/3$ et $x = 5\pi/3$.

Nous résumons l'information que nous venons de déduire dans le tableau suivant.
\[
\begin{array}{c|c|c|c|c|c|c|}
\rule[-0.5em]{0em}{1em}
x & 0 & 0<x<\frac{\pi}{3} & \frac{\pi}{3} & \frac{\pi}{3} < x < \pi
& \pi & \pi < x < \frac{3\pi}{2} \\
\hline
f(x) & -2 & - & - & -|+ & + & 2 \\
f'(x) & 0 & + & + & + & 0 & - \\
f''(x) & + & + & 0 & - & 0 & - \\ 
\hline
& \text{min. local} & \text{croît et} & \text{croît}
& \text{croît et} & \text{max. local} & \text{décroît et} \\
& \text{et convexe} & \text{convexe} & \text{et point}
& \text{concave} & \text{et concave} & \text{concave} \\
& & & \text{d'inflexion} & & &
\end{array}
\ldots
\]
\[
\ldots
\begin{array}{|c|c|c|c|c}
\rule[-0.5em]{0em}{1em}
\frac{3\pi}{2} & \frac{3\pi}{2} < x < \frac{5\pi}{3} & \frac{5\pi}{3}
  & \frac{5\pi}{3} < x < 2\pi & 2\pi \\
\hline
+  & +|- & - & - & -2 \\
0 & - & - & - &  0 \\
- & - & 0 & + & + \\
\hline
& \text{décroît et} & \text{décroît} &
\text{décroît et} & \text{min. local} \\
\text{concave} & \text{concave} & \text{et point} &
\text{convexe} & \text{et convexe} \\
& & \text{d'inflexion} & &
\end{array}
\]
La fonction $f$ change de signe dans les intervalles où nous retrouvons un des
symboles $+|-$ ou $-|+$.  Comme $f$ est continue, il existe un point dans
un tel intervalle où la fonction $f$ est nulle.

Il y a aucune asymptote.

La fonction est strictement croissante sur les intervalles
$]2n\pi,(2n+1)\pi[$ pour $n \in \ZZ$.  Elle est
strictement décroissante sur les intervalles $](2n+1)\pi,(2n+2)\pi[$
pour $n \in \ZZ$.   Les maximums locaux sont aux points $x= (2n+1)\pi$
où nous avons $f( (2n+1)\pi) = 2$.  Les minimums locaux sont aux
points
$x= 2n\pi$ où nous avons $f(2n\pi) = -2$.  La fonction est convexe sur les
intervalles $]2n\pi - \pi/3, 2n\pi + \pi/3[$ pour
$n \in \ZZ$.  Elle est concave sur les intervalles
$]2n\pi + \pi/3, (2n+1)\pi[$ et
$](2n+1)\pi, 2n\pi + 5\pi/3[$ pour $n\in \ZZ$.

\MATHgraph{6_derivees_appl/figFX4}{8cm}

\subQ{b}
Nous avons $\displaystyle f'(\theta) = 1 - 3\sin(\theta)$ et
$\displaystyle f''(\theta)= -3\cos(\theta)$.  La fonction $f$ et ses dérivées
sont définies sur toute la droite réelle.

Les valeurs de $\theta \in [0, 4\pi]$ pour lesquelles $f'(\theta) = 0$ sont
$\theta_1 = \arcsin(1/3) \approx 0.3398$, $\theta_2 = \theta_1+2\pi$,
$\theta_3 = (\pi -\arcsin(1/3)) \approx 2.8018$ et
$\theta_4 = \theta_3 + 2\pi$ comme nous pouvons voir dans la figure
ci-dessous. Les points $\theta_i$'s sont les points critiques de $f$
sur l'intervalle $[0,4\pi]$.
\PDFgraph{6_derivees_appl/problTrig}

Les valeurs de $\theta \in [0,4\pi]$ pour lesquelles $f''(\theta)=0$ sont
$\pi/2$, $3\pi/2$, $5\pi/2$ et $7\pi/2$.  Ce sont donc nos candidats pour
être des points d'inflexion.

Nous résumons l'information que nous venons de déduire dans le tableau suivant.
\[
\begin{array}{c|c|c|c|c|c|}
\rule[-0.5em]{0em}{1em}
\theta & 0 & 0<\theta < \theta_1 & \theta_1 &
\theta_1 < \theta < \frac{\pi}{2} & \frac{\pi}{2} \\
\hline
f(\theta) & 3 & + & + & + & + \\
f'(\theta) & 1 & + & 0 & - & - \\
f''(\theta) & -3 & - & - & - & 0 \\
\hline
& \multicolumn{2}{|c|}{\text{croît et}} & \text{max. local} &
\text{décroît et} & \text{décroît et} \\
& \multicolumn{2}{|c|}{\text{concave}} & \text{et concave} &
\text{concave} & \text{et point} \\
& \multicolumn{2}{|c|}{} & & & \text{d'inflexion}
\end{array} \quad \ldots
\]
\[
\qquad \ldots \quad 
\begin{array}{|c|c|c|c|}
\rule[-0.5em]{0em}{1em}
\frac{\pi}{2} < \theta < \theta_3 & \theta_3 &
\theta_3 < \theta < \frac{3\pi}{2} & \frac{3\pi}{2} \\
\hline
+|- & - & -|+ & + \\
- & 0 & + & 4 \\
+ & + & + & 0 \\
\hline
\text{décroît et} & \text{min. local} & \text{croît} & \text{croît} \\
\text{convexe} & \text{et convexe} & \text{convexe} & \text{et point} \\
& & & \text{d'inflexion}
\end{array} \quad \ldots
\]
\[
\qquad \ldots \quad
\begin{array}{|c|c|c|c|}
\rule[-0.5em]{0em}{1em}
\frac{3\pi}{2} <\theta < \theta_2 & \theta_2 &
\theta_2 < \theta < \frac{5\pi}{2} & \frac{5\pi}{2} \\
\hline
+ & + & + & + \\
+ & 0 & - & - \\
- & - & - & 0 \\
\hline
\text{croît et} & \text{max. local} & \text{décroît} & \text{décroît} \\
\text{concave} & \text{et concave} & \text{concave} & \text{et point} \\
& & & \text{d'inflexion}
\end{array} \quad \ldots
\]
\[
\qquad \ldots \quad
\begin{array}{|c|c|c|c|c|c}
\rule[-0.5em]{0em}{1em}
\frac{5\pi}{2} < \theta < \theta_4 & \theta_4 &
\theta_4 < \theta < \frac{7\pi}{2} & \frac{7\pi}{2} &
\frac{7\pi}{2} < \theta < 4\pi & 4\pi\\
\hline
+ & + & + & + & + & + \\
- & 0 & + & + & + & + \\
+ & + & + & 0 & - & - \\
\hline
\text{décroît et} & \text{min. local} & \text{croît et} &
\text{croît et} & \multicolumn{2}{|c}{\text{croît et}} \\
\text{convexe} & \text{et convexe} & \text{convexe} & \text{et point} &
\multicolumn{2}{|c}{\text{concave}} \\
& & & \text{d'inflexion} & \multicolumn{2}{|c}{}
\end{array}
\]
La fonction $f$ change de signe dans les intervalles où nous retrouvons un des
symboles $+|-$ ou $-|+$.  Comme $f$ est continue, il existe un point dans
un tel intervalle où la fonction $f$ est nulle.

Le graphe de $f$ est
\MATHgraph{6_derivees_appl/graph_funct17}{8cm}

\subQ{c}
Nous avons $\displaystyle h'(t) = \left(\cos(t)- \sin(t)\right) e^{-t}$ et
$\displaystyle h''(t)= -2\cos(t)e^{-t}$.  La fonction $h$ et ses dérivées
sont définies sur toute la droite réelle.

Les valeurs de $t \in [0, 4\pi]$ pour lesquelles $h(t) = 0$ sont les valeurs
de $t$ telles que $\sin(t) = 0$; c'est-à-dire, $n \pi$ pour $n=0$, $1$, $2$,
$3$ et $4$.

Les valeurs de $t \in [0, 4\pi]$ pour lesquelles $h'(t) = 0$ sont les valeurs
de $t$ telles que $\tan(t) = 1$ puisque $\cos(t)-\sin(t) \neq 0$ pour
$t=\pi/2$, $3\pi/2$, $5\pi/2$ et $7\pi/2$.  Les points critiques de $h$
sur l'intervalle $[0,4\pi]$ sont donc $\pi/4$, $5\pi/4$, $9\pi/4$ et $13\pi/4$.
Le valeurs de $t \in [0,4\pi]$ pour lesquelles $h''(t)=0$ sont
$(2n+1)\pi/2$ pour $n=0$, $1$, $2$ et $3$.   Ce sont donc nos candidats pour
être des points d'inflexion.

Nous résumons l'information que nous venons de déduire dans le tableau suivant.
\[
\begin{array}{c|c|c|c|c|c|c|c|}
\rule[-0.5em]{0em}{1em}
t & 0 & 0<t<\frac{\pi}{4} & \frac{\pi}{4} &
\frac{\pi}{2}<t<\frac{\pi}{2} & \frac{\pi}{2} &
\frac{\pi}{2}<t<\pi & \pi \\
\hline
h(t) & 0 & + & + & + & + & + & 0 \\
h'(t) & 1 & + & 0 & - & - & - & - \\
h''(t) & -2 & - & - & - & 0 & + & + \\
\hline
& \multicolumn{2}{|c|}{\text{croît et}} & \text{max. local} &
\text{décroît et} & \text{décroît} &
\multicolumn{2}{|c|}{\text{décroît et}} \\
& \multicolumn{2}{c|}{\text{concave}} & \text{et concave} &
\text{et concave} & \text{et point} & \multicolumn{2}{|c|}{\text{convexe}} \\
& \multicolumn{2}{|c|}{} & & & \text{d'inflexion} &
\multicolumn{2}{|c|}{}
\end{array} \ \ldots
\]
\[
\qquad \ldots \quad 
\begin{array}{|c|c|c|c|c|c|}
\rule[-0.5em]{0em}{1em}
\pi<t<\frac{5\pi}{4} & \frac{5\pi}{4} &
\frac{5\pi}{4}<t<\frac{3\pi}{2} & \frac{3\pi}{2} &
\frac{3\pi}{2}<t<2\pi & 2\pi \\
\hline
- & - & - & - & - & 0 \\
- & 0 & + & + & + & + \\
+ & + & + & 0 & - & - \\
\hline
\text{décroît et} & \text{min. local} & \text{croît et} & \text{croît} &
\multicolumn{2}{|c|}{\text{croît et}} \\
\text{convexe} & \text{convexe} & \text{convexe} & \text{et point} &
\multicolumn{2}{|c|}{\text{concave}}  \\
& & & \text{d'inflexion} & \multicolumn{2}{|c|}{}
\end{array} \quad \ldots
\]
\[
\qquad \ldots \quad
\begin{array}{|c|c|c|c|c|c|}
\rule[-0.5em]{0em}{1em}
2\pi<t<\frac{9\pi}{4} & \frac{9\pi}{4} & \frac{9\pi}{4}<t<\frac{5\pi}{2} &
\frac{5\pi}{2} & \frac{5\pi}{2}<t<3\pi & 3\pi \\
\hline
+ & + & + & + & + & 0 \\
+ & 0 & - & - & - & - \\
- & - & - & 0 & + & + \\
\hline
\text{croît et} & \text{max. local} & \text{décroît et} & \text{décroît} &
\multicolumn{2}{|c|}{\text{décroît}} \\
\text{concave} & \text{concave} & \text{concave} & \text{et point} &
\multicolumn{2}{|c|}{\text{convexe}} \\
 & & & \text{d'inflexion} & \multicolumn{2}{|c|}{}
\end{array} \quad \ldots
\]
\[
\qquad \ldots \quad
\begin{array}{|c|c|c|c|c|c}
\rule[-0.5em]{0em}{1em}
3\pi<t<\frac{13\pi}{4} & \frac{13\pi}{4} &
\frac{13\pi}{4}<t<\frac{7\pi}{2} & \frac{7\pi}{2} & \frac{7\pi}{2}<t<4\pi &
4\pi \\
\hline
- & - & - & - & - & 0 \\
- & 0 & + & + & + & + \\
+ & + & + & 0 & - & - \\
\hline
\text{décroît et} & \text{min. local} & \text{croît et} & \text{croît}
& \multicolumn{2}{|c}{\text{croît}} \\
\text{convexe} & \text{convexe} & \text{convexe} & \text{et point}
& \multicolumn{2}{|c}{\text{concave}} \\
 & & & \text{d'inflexion} & \multicolumn{2}{|c}{\text{concave}}
\end{array}
\]
La fonction $f$ change de signe dans les intervalles où nous retrouvons un des
symboles $+|-$ ou $-|+$.  Comme $f$ est continue, il existe un point dans
un tel intervalle où la fonction $f$ est nulle.

Le graphe de $f$ est donné ci-dessous.  Puisque $e^{-t}$ tend très
rapidement vers $0$ lorsque $t$ tend vers plus l'infini, nous avons aussi tracé
le graphe de $f$ sur des intervalles de plus en plus éloigné de l'origine.
Portez attention à l'amplitude de $f$ dans chacun des graphes
ci-dessous.
\MATHgraph{6_derivees_appl/graph_funct18}{9cm}
}

\subsection{Optimisation}

\compileSOL{\SOLUb}{\ref{6Q24}}{
\begin{center}
\begin{tabular}{c|c|c|c}
\hline
$x$ (nombre d'oeufs pondus) & $5$ & $10$ & $20$ \\
\hline
$S(x)=xP(x)$ (nombre & & & \\
de poussins qui survivent) & $0.\overline{370}$ & $0.196\ldots$ &
$0.0995\ldots$ \\
\hline
\end{tabular}
\end{center}
Nous avons $\displaystyle S(x) = \frac{x}{1+0.5x^2}$ et
$\displaystyle S'(x) = \frac{1-0.5x^2}{(1+0.5x^2)^2}$.
La seul solution de $S(x)= 0$ est $x=0$.  En fait, $S(x)>0$ pour tout
$x>0$.  Le seul point critique (positif) de $S$ est $x = \sqrt{2}$.
De plus,
\[
\lim_{x\rightarrow \infty} S(x) = \lim_{x\rightarrow \infty} \frac{x}{1+0.5x^2}
= \lim_{x\rightarrow \infty} \frac{(1/x)}{(1/x^2)+0.5}
= \frac{ \big(\lim_{x\rightarrow \infty} (1/x)\big)}
{\big(\lim_{x\rightarrow \infty} (1/x^2)\big)+0.5} = \frac{0}{0+5} = 0 \ .
\]
Nous obtenons le tableau suivant.
\[
\begin{array}{c|c|c|c|c|c}
x & 0 & 0<x< \sqrt{2} & \sqrt{2} & \sqrt{2} < x & \infty \\
\hline
S(x) & 0 & + & \sqrt{2}/2 & + & 0 \\
S'(x) & + & + & 0 & - & \\
\hline
& \text{croît} & \text{croît} & \text{max. loc.} & \text{décroît} \\
\end{array}
\]
Le graphe de $S$ a l'allure suivante.
\PDFgraph{6_derivees_appl/graph_funct13}

La meilleure stratégie est de pondre $\sqrt{2}$ oeufs en moyenne car
cela donne la valeur maximale de $S$.
}

\compileSOL{\SOLUa}{\ref{6Q25}}{
Le nombre $N$ de pommiers vendus est une fonction linéaire du prix $x$
de vente d'un pommier.  Nous avons $N(50) = 100$, $N(49)=104$ et
$N(51) = 96$.  C'est plus d'information que nécessaire pour trouver
l'équation de la droite représentant le graphe de $N(x)$.  Donc
$N(x) = m x + b$ où
$\displaystyle m = \frac{104-100}{49-50} = -4$ et $b=300$ (pour avoir
$N(50)=100$).

Ainsi, le revenu est donné par la formule $R(x) = x N(x) = -4x^2 + 300x$
où $x$ est le prix de vente d'un pommier.  Le coût de production pour
les $N(x)$ pommiers vendus est $C(x) = 9 N(x) = -36x + 2700$.  Le
profit est donc donné par $P(x) = R(x) - C(x) = -4x^2 + 336 x - 2700$.

Il y a un seul point critique qui est donné par
$P'(x) = -8x + 336 = -8 (x - 42) = 0$.  Donc $x= 42$.
Puisque $P''(x) = -8 < 0$, la fonction est concave pour tout $x$. Ce
n'est pas surprenant car $P(x)$ est un polynôme de degré $2$ avec le
coefficient de $x^2$ négatif.  Donc $P(x)$ a un maximum absolu 
lorsque le jardinier vent ses pommiers à \$42.00 l'unité.
}

\compileSOL{\SOLUa}{\ref{6Q26}}{
\PDFgraph{6_derivees_appl/graph_funct19}
La fonction en (b) est strictement décroissante.
}

\compileSOL{\SOLUb}{\ref{6Q27}}{
\subQ{a} Puisque $f$ est une belle fonction continue sur un intervalle
fermé et borné, nous pouvons utiliser le Théorème des valeurs extrêmes.

Le seul point critique de $f$, lorsque $f'(x) = (1-x) e^{-x} = 0$, est
$x=1$.  C'est un point dans l'intervalle $[0.5,2]$.  Nous devons donc
le considérer.

Nous comparons les valeurs de $f(x)$ au points $x=0.5$, $x=1$ et $x=2$.
Puisque $f(0.5)= 2.303265330$, $f(1) = 2.367879441$ et $f(2) =
2.270670566$, le maximum absolue est $2.367879441$ et le minimum
absolue est $2.270670566$.

\subQ{b} Comme $f$ est une fonction continue sur l'intervalle borné et
fermé $[-2,2]$, nous pouvons utiliser le Théorème des valeurs extrêmes pour
trouver le maximum absolu et le minimum absolu de $f$ sur $[-2,2]$.

Les points critiques de $f$, lorsque $f'(x) = 3x^2-3 = 3(x^2-1)=0$,
sont $x=\pm 1$.  Ces points critiques sont bien dans l'intervalle $[-2,2]$.
Nous avons $f(-2) = -2$, $f(-1) = 2$, $f(1) = -2$ et $f(2) = 2$.
Ainsi, le maximum absolu est $2$ que $f$ atteint à $x=-1$ et $x=2$.
Le minimum absolu est $-2$ que $f$ atteint à $x=-1$ et $x=2$.

\subQ{c} Comme $f$ est une fonction continue sur l'intervalle borné et
fermé $[-2,0]$, nous pouvons utiliser le Théorème des valeurs extrêmes pour
trouver le maximum absolu et le minimum absolu de $f$ sur $[-2,2]$.

Les points critiques de $f$, lorsque
$\displaystyle f'(x) = \frac{1-x^2}{(x^2+x+1)^2}=0$, sont $x=\pm 1$.
Le point $1$ n'est pas dans l'intervalle $[-2,0]$,  Nous considérons
seulement $x = -1$.  Nous avons $f(-2) = -2/3$, $f(-1) = -1$ et
$f(0) = 0$.  Ainsi, le maximum absolu est $f(0)=0$.  Le minimum absolu
est $f(-1)=-1$.  C'est aussi un minimum local.

\subQ{d} Puisque $f$ est une fonction continue sur l'intervalle fermé
$[0,1]$, nous pouvons utiliser le Théorème des valeurs extrêmes
pour conclure qu'il existe $c \in [0,1]$ tel que $f(c) \geq f(x)$ pour
tout $x \in [0,1]$ et $d \in [0,1]$ tel que $f(d) \leq f(x)$ pour tout
$x \in [0,1]$.  $f(c)$ est le maximum absolu de $f$ sur l'intervalle
$[0,1]$ et $f(d)$ est le minimum absolu.

Nous avons $\displaystyle f'(x) = 15x^2 - 30x + 10$.  Les points
critiques de $f$ sont $x= 1 \pm 1/\sqrt{3}$.   Le point $1+1/sqrt{3}$
n'est pas dans l'intervalle $[0,1]$,  Par contre
$1-1/\sqrt{3} \approx 0.42265$ est dans l'intervalle $[0,1]$. 
Nous avons $f(0) = -1$, $f(1-1/\sqrt{3}) \approx 0.9245$ et $f(1) = -1$.
Ainsi, le maximum absolu est $f(1-1/sqrt{3})$.  Le minimum absolu est
$f(0)=f(1) =-1$.

\subQ{e} Puisque $f$ est une fonction continue sur l'intervalle fermé
$[0,1]$, nous pouvons utiliser le Théorème des valeurs extrêmes.
Il découle de
\begin{align*}
f'(x) &= \frac{1}{2} \, x^{-1/2}(x-4)^2 + 2 x^{1/2} (x-4)
= \frac{1}{2} \, x^{-1/2} (x-4) \left( (x-4) + 4 x \right) \\
&= \frac{1}{2} \, x^{-1/2} (x-4)(5x-4) \ .
\end{align*}
que les points critiques de $f$ sur l'intervalle $[0.6, 6]$ sont
$4/5$ et $4$.

Puisque $f(0.5) = 8.662\ldots$, $f(0.8)=9.1589\ldots$, $f(4)=0$ et
$f(6)=9.7979\ldots$, le maximum global est $9.7979$ et le minimum
global est $0$.
}

\compileSOL{\SOLUb}{\ref{6Q30}}{
Il faut minimiser la distance $\sqrt{x^2+y^2}$ lorsque $y=4x+7$.  Comme le
point $(x,y)$ où $\sqrt{x^2+y^2}$ atteint son minimum est aussi le point où
$x^2+y^2$ atteint son minimum (car la racine carrée est une fonction
strictement croissante), nous allons minimiser $D(x) = x^2 + y^2$
lorsque $y=4x+7$.
Donc $D(x) = x^2 + (4x+7)^2 = 17x^2 + 56x + 49$,  $D'(x) = 34x+56$ et
$D''(x) = 34$.  Nous avons un seul point critique lorsque $x=-28/17$.  Puisque
$D''(x) > 0$ pour tout $x$, la fonction est convexe pour tout
$x$ et le point critique est un minimum absolu.

La distance minimum est donc atteinte au point
\[
(x,y) = (-28/17 , 4(-28/17)+7) = (-28/17 , 7/17 ) \ .
\]
La distance minimal est $\sqrt{(-28/17)^2+(7/17)^2} =  7/\sqrt{17} $.
}

\compileSOL{\SOLUb}{\ref{6Q32}}{
\PDFgraph{6_derivees_appl/fence_sol}

L'aire $A$ de l'enclos est donné par $A = x y$.  Nous avons la contrainte que
$4 x + 2 y = 3\,600$.  Donc $y = 1\,800 - 2x$ avec $0 \leq x \leq 900$
car $x$ et $y$ doivent être non négatifs.

Ainsi, $A(x) = x(1\,800-2x) = 1\,800x - 2 x^2$.
Puisque $A$ est une fonction continue sur l'intervalle fermé
$[0,900]$, nous pouvons utiliser le Théorème des valeurs extrêmes.
Le seul point critique est donné par $A'(x) = 1\,800 -4 x = 0$.  Donc
$x = 450$.

Puisque $A(0) = A(900) = 0$ et $A(450) = 405\,000$ m$^2$.  L'aire
maximale est $405\,000$ m$^2$ lorsque $x= 450$ m et $y = 900$ m, et
chaque section est d'aire $135\,000$ m$^2$.
}

\compileSOL{\SOLUb}{\ref{6Q36}}{
Soit $\displaystyle f(q) = \frac{q}{P(q)} = \frac{q}{1+q^2}$.  Nous avons
$q \geq 0$.  Nous ne pouvons pas utiliser le théorème des valeurs extrême
car l'intervalle n'est pas borné.  Nous devons donc étudier le graphe de
$f$ sur l'intervalle $[0,\infty[$.

Nous avons $\displaystyle f'(q) = \frac{1-q^2}{(1+q^2)^2}$.  La fonction $f$
a un seul point critique positif, c'est $q=1$.  De plus,
$\displaystyle \lim_{q\to \infty} f(q) = 0$.  Nous obtenons le tableau suivant.
\[
\begin{array}{c|c|c|c|c|c}
q & 0 & 0<q<1 & 1 & 1<q & \infty \\
\hline
f(q) & 0 & + & 1/2 & + & 0 \\
f'(q) & + & + & 0 & - &  \\
\hline
& \text{croît} & \text{croît} & \text{max. local} & \text{décroît} & \\
\end{array}
\]
Puisqu'il n'y a pas de points critiques autres que $1$, le maximum
local est aussi le maximum absolu (tracez le graphe de $f$ pour vous
en convaincre).
}

\compileSOL{\SOLUa}{\ref{6Q37}}{
Si un cylindre a une hauteur de $H$ unités et un rayon de $R$ unités, l'aire
de sa surface est la somme de l'aire du dessus, l'aire du dessous et
l'aire du côté circulaire du cylindre.  C'est-à-dire,
$S(H) = 2\pi R^2 + 2\pi R H$ où $0 \leq H \leq 2r$ car le cylindre est à
l'intérieur de la sphère de rayon $r$.

Or, si nous regardons une coupe transversal de la sphère passant par l'axe du
cylindre, nous voyons (encore et toujours par le théorème de Pythagore) que
$(H/2)^2 + R^2 = r^2$ où, si vous préférez, $R^2 = r^2 - (H/2)^2$.  Donc
\begin{align*}
S(H) &= 2\pi R^2 + 2\pi R H
= 2\pi\left(r^2-\left(\frac{H}{2}\right)^2\right) +
2\pi H \sqrt{r^2-\left(\frac{H}{2}\right)^2} \\
&= \frac{1}{2} \pi(4 r^2-H^2) + \pi H\sqrt{4r^2-H^2} \ .
\end{align*}
If faut trouver la valeur de $H$ dans l'intervalle fermé $0 \leq H \leq 2r$
où $S(H)$ est maximal.  Les points critiques de $S(H)$ sont donnés par
\[
S'(H) = -\pi H - \frac{\pi H^2}{\sqrt{4 r^2 - H^2}} + \pi \sqrt{4r^2-H^2}
=  -\pi H + \frac{\pi (4r^2-2H^2)}{\sqrt{4 r^2 - H^2}} = 0 \ .
\]
Donc
\begin{align*}
\pi H = \frac{\pi (4r^2-2H^2)}{\sqrt{4 r^2 - H^2}} & \Rightarrow
H^2 = \frac{(4r^2-2H^2)^2}{4 r^2 - H^2}
\Rightarrow (4 r^2 - H^2) H^2 = (4r^2-2H^2)^2 \\
&\Rightarrow 0 = 5(H^2)^2 -20 r^2 (H^2) + 16 r^4 \ .
\end{align*}
C'est une équation quadratique en $H^2$.  Les racines sont
\[
H^2 = \frac{ 20 r^2 \pm \sqrt{(-20r^2)^2 - 4\times 5 \times
    16r^4}}{10}
= \left(2 \pm \frac{2}{\sqrt{5}}\right)r^2 \ .
\]
Nous trouvons deux points critiques:
$\displaystyle H_1 = r\sqrt{2 + \frac{2}{\sqrt{5}}} = 1.7013\ldots r$ et
$\displaystyle H_2 = r\sqrt{2 - \frac{2}{\sqrt{5}}} = 1.05146\ldots r$.

Puisque $S(0) = 2\pi r^2$, $S(H_1) \approx 7.35648 r^2$,
$S(H_2) \approx 10.16640 r^2$ et $S(2r)=0$, la surface maximal est de
$10.16640\ldots r^2$ lorsque la hauteur du cylindre est de
$\displaystyle r\sqrt{2 - \frac{2}{\sqrt{5}}}$ unités.
}

\compileSOL{\SOLUb}{\ref{6Q38}}{
L'illumination totale sur l'objet au point $x$ est
\[
I(x) = \frac{k P}{x^2} + \frac{3kP}{(3-x)^2}
\]
où $k$ est la constance de proportionnalité.  Les points critiques sont
donnés par
\[
I'(x) = 2kP\left( \frac{-1}{x^3} + \frac{3}{(3-x)^3} \right) = 0 \ .
\]
Donc $\displaystyle \frac{1}{x^3} = \frac{3}{(3-x)^3}$.  Si nous
prenons la racine cubique des deux côtés de l'égalité, nous trouvons
$\displaystyle \frac{1}{x} = \frac{\sqrt[3]{3}}{(3-x)}$ et ainsi
$\displaystyle x = \frac{3}{1+\sqrt[3]{3}}$.

Grâce aux données du tableau suivant, nous pouvons conclure que $I$ atteint un
maximum absolu à $\displaystyle x = \frac{3}{1+\sqrt[3]{3}}$.
\[
\begin{array}{c|c|c|c}
\rule[-0.7em]{0em}{1.5em}
x & x < \frac{3}{1+\sqrt[3]{3}} & x=\frac{3}{1+\sqrt[3]{3}} &
\frac{3}{1+\sqrt[3]{3}}<x <3 \\
\hline
I'(x) & - & 0 & +
\end{array}
\]
L'illumination maximale est lorsque
$\displaystyle x=\frac{3}{1+\sqrt[3]{3}} = 1.228375690$ et la valeur de
l'illumination est
$\displaystyle I\left(\frac{3}{1+\sqrt[3]{3}}\right) = 1.618555576\,kP$.
}

\compileSOL{\SOLUb}{\ref{6Q39}}{
Nous avons $\displaystyle F(t) = \frac{\beta t}{t+\alpha}$ avec $\beta=1$ et
$\alpha = 0.5$.  De plus, nous avons montré dans le manuel que
$T=\sqrt{\alpha \tau}$ où $\tau$ est le temps en minutes que prend une
abeille pour se rendre d'une fleur à un autre fleur.  Comme $T=1$,
nous obtenons $1=\sqrt{0.5\tau}$ et donc $\tau = 2$ minutes.
\MATHgraph{6_derivees_appl/bees5}{8cm}
}

\compileSOL{\SOLUb}{\ref{6Q40}}{
La quantité de nectar récoltée par une abeille après $t$ minutes sur une même
fleur est $\displaystyle F(t) = \frac{t^2}{1+t^2}$
La vitesse moyenne à laquelle l'abeille aspire le nectar d'une fleur sur
laquelle elle demeure pendant $t$ minutes est
$\displaystyle R(t) = \frac{F(t)}{t+1}$.  Puisque
\[
R(t) = \frac{t^2}{(1+t^2)(t+1)} \; ,
\]
nous obtenons
\[
R'(t) = \frac{-t(t^3-t-2)}{(1+t^2)^2(1+t)^2} \; .
\]
Les points critiques positifs de $R$ sont les racines du polynôme
$p(t) = t^3-t-2$.  Comme nous n'avons pas de formule exacte pour
trouver ces racines, il faut utiliser la méthode de Newton pour les
estimer.  Puisque $p(1)= -2 <0$ et $p(2) = 4>0$, il 
y a une racine entre $1$ et $2$.  Avec $t_0=1.5$, nous obtenons
\[
\begin{array}{c|c|c}
n & t_n & \displaystyle t_{n+1} \approx t_n - \frac{p(t_n)}{p'(t_n)} \\[0.7em]
\hline
0 & 1.5 & 1.5217391304 \\
1 & 1.5217391304 & 1.5213798060 \\
2 & 1.5213798060 & 1.5213797068 \\
3 & 1.5213797068 & 1.5213797068
\end{array}
\]
Un point critique de $p$ est $T\approx 1.5213797068$\; .
Ils n'existent pas d'autres racines positives de $p(t)$ et donc
d'autres points critiques positifs de $R$ car $p'(t) = 3t^2-1$ nous donne le
tableau suivant.
\[
\begin{array}{c|c|c|c|c|c|c|c}
t & 0 & 0<t< 1/\sqrt{3} & 1/\sqrt{3} & 1/\sqrt{3} < t <T & T & T< t & \infty \\
\hline
p(t) & -1 & - & - & - & 0 & + & +\infty \\
p'(t) & - & - & 0 & + & + & + & \\
\hline
& \multicolumn{2}{|c|}{\text{décroît}} & \text{min. local}
& \multicolumn{3}{|c|}{\text{croît}}
\end{array}
\]
Nous obtenons le graphe de $p$ qui suit.
\MATHgraph{6_derivees_appl/bees6_a}{8cm}

Montrons que $R$ a un maximum absolu à $t=T$.  Nous avons le tableau suivant.
\[
\begin{array}{c|c|c|c|c|c}
t & 0 & 0 < t < T & T & T < t < +\infty & +\infty \\
\hline
R(t) & 0 & + & T^2(1+T^2)^{-1}(1+T)^{-1} & + & 0 \\
R'(t) & 0 & + & 0 & - & \\
\hline
& \text{point} & \text{croît} & \text{max. local} & \text{décroît}
& \text{asymptote} \\
&\text{critique} & & & & \text{horizontal}
\end{array}
\]
Donc $t=T$ est le temps optimal pour maximiser la récolte de nectar.
\MATHgraph{6_derivees_appl/bees6_b}{8cm}
Comme il est prédit par la règle des valeurs marginales, nous avons que
$R(T)= F'(T)$.
}

\subsection{Taux liées}

\compileSOL{\SOLUb}{\ref{6Q41}}{
La question peut se résumer en une figure.
\PDFgraph{6_derivees_appl/quai}
D'après le théorème de Pythagore, nous avons $y^2 + 2^2 = x^2$ où $x$ et $y$
sont des fonctions de $t$.  Si nous dérivons cette dernière équation par
rapport à $t$, nous obtenons
\[
2 y \dydx{y}{t} = 2 x \dydx{x}{t} \ .
\]
Ainsi,
\[
\dydx{y}{t} = \frac{x}{y}\; \dydx{x}{t} \ .
\]
Au temps $t=\tau$ où $y(\tau) = 6$ m, nous avons
$x(\tau) = \sqrt{y^2(\tau) + 2^2} = \sqrt{6^2 + 2^2} = 2\sqrt{10}$ m.
Puisque $\displaystyle \dydx{x}{t}(t) = 0.5$ m/s pour tout $t$, nous avons
\[
\dydx{y}{t}(\tau) = \frac{x(\tau)}{y(\tau)}\; \dydx{x}{t}(\tau)
= \frac{2\sqrt{10}}{6} \times \frac{1}{2} = \frac{\sqrt{10}}{6} \approx
0.527 \ \text{m/s} \ .
\]

La vitesse d'approche du bateau n'est pas constante car
\[
\frac{x(t)}{y(t)} = \frac{\sqrt{y^2(t)+2^2}}{y(t)}
\]
n'est pas constant.  Par exemple, lorsque $y(\tau) = 10$ m, nous obtenons
$x(\tau) = \sqrt{y^2(\tau) + 2^2} = \sqrt{10^2 + 2^2} = 2\sqrt{26}$ m et
\[
\dydx{y}{t}(\tau) = \frac{x(\tau)}{y(\tau)}\; \dydx{x}{t}(\tau)
= \frac{2\sqrt{26}}{10} \times \frac{1}{2} = \frac{\sqrt{26}}{10} \approx
0.5099 \ \text{m/s} \ .
\]
}

\compileSOL{\SOLUa}{\ref{6Q42}}{
Il faut trouver une relation entre le volume d'eau dans la piscine et la
profondeur $h$ de l'eau à l'endroit le plus profond.

Considérons une section transversale de la piscine.
\PDFgraph{6_derivees_appl/piscine2}
Pour déterminer la longueur du segment $\overline{eb}$, comparons les
triangles semblables $\triangle abe$ et $\triangle acd$.  Nous avons
$\displaystyle \frac{|\overline{eb}|}{|\overline{ab}|} =
\frac{|\overline{dc}|}{|\overline{ac}|}$; c'est-à-dire,
$\displaystyle \frac{|\overline{eb}|}{h} = \frac{6}{4}$.  Donc
$|\overline{eb}| = 3h/2$.

Le volume d'eau dans la piscine est alors donnée par la formule
\[
V = \underbrace{\frac{(3h/2)\times h\times 10}{2}}
_{(\text{aire du triangle }A)\times 10} +
\underbrace{h\times 5 \times 10}_{(\text{aire du rectangle }B) \times 10}
= \frac{15h^2}{2} + 50 h
\]
lorsque $h<4$ m.  Donc
\[
\dydx{V}{t}(t) = 15h(t)\; \dydx{h}{t}(t) + 50 \dydx{h}{t}(t)
= 5\left( 3h(t)+ 10\right) \dydx{h}{t}(t) \ .
\]
Au temps $t=\tau$ où $h(\tau) = 3$ m, nous avons
\[
0.1 = 5\left( 3\times 3 + 10\right) \dydx{h}{t}(\tau)
\]
puisque $\displaystyle \dydx{V}{t}(t) = 0.1$ m$^3$/min pour tout $t$.
Ainsi, $\displaystyle \dydx{h}{t}(\tau) \approx 0.00105$ m/min.
}

\compileSOL{\SOLUa}{\ref{6Q43}}{
$5$ révolutions par minute se traduit par
$\displaystyle \dydx{\theta}{t}(t) = 10\pi$ radians/min pour tout $t$.

Si nous dérivons la relation $\displaystyle \tan(\theta) = \frac{x}{4}$,
nous obtenons
\[
\sec^2(\theta(t)) \; \dydx{\theta}{t}(t) = \frac{1}{4} \dydx{x}{t}(t) \ .
\]
Donc
\[
\dydx{x}{t}(t) = 4 \sec^2(\theta(t)) \; \dydx{\theta}{t}(t) 
= 4\left( \frac{\sqrt{x^2(t) + 4^2}}{4}\right)^2 \; \dydx{\theta}{t}(t) \ .
\]
Au temps $t=\tau$ lorsque $x(\tau) = 1$, nous avons
\[
\dydx{x}{t}(\tau) = 4\left( \frac{\sqrt{x^2(\tau) + 4^2}}{4}\right)^2
\; \dydx{\theta}{t}(\tau) 
= 4\left( \frac{\sqrt{1 + 4^2}}{4}\right)^2 (10\pi)
= \frac{85\pi}{2} \approx 133.518 \ \text{km/min} \ .
\]
}

\subsection{Dérivées implicites}

\compileSOL{\SOLUb}{\ref{6Q44}}{
Si nous dérivons des deux côtés de l'égalité par rapport à $x$, nous obtenons
\[
4 (f(x))^3 f'(x) + 6 f'(x) = 2x
\]
Pour $x=3$, nous obtenons
\[
4 (f(3))^3 f'(3) + 6 f'(3) = 2
\Rightarrow f'(3) (4\times 1^3 + 6) = 2
\Rightarrow f'(3) = \frac{2}{10} = \frac{1}{5} \ .
\]
}

\compileSOL{\SOLUb}{\ref{6Q45}}{
\subQ{a} $\displaystyle y' = \frac{3-2xy-y^2}{x^2+2xy}$

\subQ{b} $\displaystyle y' = \frac{4xy\sqrt{xy} - y}{x-2x^2\sqrt{xy}}$

\subQ{c} $\displaystyle y' = \frac{1-y^4-2xy}{4xy^3 + x^2 -3}$
}

\subsection{Approximation locale des fonctions}

\compileSOL{\SOLUb}{\ref{6Q46}}{
\subQ{a} Considérons $f(x) = x^{2001}$.  Nous cherchons à estimer $f(1.002)$.
L'approximation linéaire de $f$ au voisinage de $x=1$ est
$p(x) = f(1) + f'(1) ( x -1 ) = 1 + 2001(x-1)$.  Ainsi,
\[
f(1.002) \approx p(1.002) =2001(1.002 - 1) +1 = 5.002 \ .
\]
Ce n'est pas une bonne approximation car 
$1.002^{2001} \approx 54.489244196998690$.

\subQ{b} Considérons $f(x) = \sin(x)$.  Nous cherchons à estimer $f(0.02)$.
L'approximation linéaire de $f$ au voisinage de l'origine est
$p(x) = f(0) + f'(0) ( x -0 ) = x$.  Ainsi,
\[
f(0.02) \approx p(0.02) = 0.02 \ .
\]
C'est une bonne approximation car $\sin(0.02) \approx 0.019998667$.
}

\compileSOL{\SOLUb}{\ref{6Q47}}{
Nous utilisons l'approximation linéaire de $f$ au voisinage de $x=1$.
\[
f(x) \approx p(x) = f(1) + f'(1) (x-1) = 1 + 2(x-1) \ .
\]
Ainsi $f(1.1) \approx 1 + 2(1.1-1) = 1.2$ et
$f(0.9) \approx 1 + 2(0.9-1) = 0.8$.
\MATHgraph{6_derivees_appl/approxlin1}{8cm}

Les approximations obtenues de l'approximation linéaire sous-estime les
valeurs exactes de $f(0.9)$ et $f(1.1)$ car $f$ est
convexe.  La droite tangente à la courbe $y=f(x)$ est donc
en dessous de la courbe $y=f(x)$.
}

\compileSOL{\SOLUb}{\ref{6Q50}}{
\subQ{a} L'approximation linéaire de $g(x) = \ln(x)$ au voisinage
du point $x=1$ est $p_1(x) = g(1) + g'(1) (x-1)$.  Or
$\displaystyle g'(1) = \frac{1}{x}\bigg|_{x=1} = 1$ et $g(1) = \ln(1) = 0$.
Ainsi $p_1(x) = x-1$.

\subQ{b} L'approximation linéaire de $h(x) = x^2-1$ au voisinage
du point $x=1$ est $q_1(x) = h(1) + h'(1) (x-1)$.  Or
$\displaystyle h'(1) = 2x\big|_{x=1} = 2$ et $h(1) = 0$.
Ainsi $q_1(x) = 2(x-1)$.

\subQ{c} Nous avons
\[
\lim_{x\to 1} \frac{p_1(x)}{q_1(x)} = \lim_{x\to 1} \frac{x-1}{2x-2}
= \lim_{x\to 1} \frac{1-(1/x)}{2-(2/x)}
= \frac{\displaystyle 1-\lim_{x\to 1} (1/x)}
{\displaystyle 2-\lim_{x\to 1} (2/x)} = \frac{1}{2} \ .
\]
De plus, grâce à la Règle de l'Hospital, nous obtenons
\[
\underbrace{\lim_{x\to 1} \frac{\ln(x)}{x^2-1}}_{\text{forme $0/0$}}
= \lim_{x\to 1} \frac{1/x}{2x} = \lim_{x\to 1} \frac{1}{2x^2} 
= \frac{1}{2} \ .
\]
Les deux limites sont biens égales.
}

\compileSOL{\SOLUb}{\ref{6Q51}}{
L'approximation linéaire de $f$ pour $x$ près de $x=1$ est
\[
f(x) \approx p_1(x) = f(1) + f'(1) (x-1) \ .
\]
Puisque $f'(x) = -3 x^{-4}$, nous obtenons $f'(1)= -3$.
Donc
\[
f(x) \approx p_1(x) = 1 -3 (x-1) = -3 x + 4
\]
pour $x$ près de $1$ est l'approximation linéaire cherchée.

Nous cherchons un polynôme $p_2(x) = a + bx + c x^2$ tel que
$p_2(1) = f(1)$, $p_2'(1) = f'(1)$ et $p_2''(1) = f''(1)$.  Puisque
$f(1)=1$, $f'(1)=-3$ et $f''(1)= 12$, nous obtenons le système
d'équations linéaires
\begin{align*}
  p_2(1) = f(1) &\Rightarrow a+b+c = 1 \\
  p_2'(1) = f'(1) &\Rightarrow b+2c = -3 \\
  p_2''(1) = f''(1)&\Rightarrow 2c = 12
\end{align*}
La solution de ce système est $a= 10$, $b=-15 $ et $c=6$.  Donc
\[
p_2(x) = 10 -15 x + 6x^2 \ .
\]
\MATHgraph{6_derivees_appl/taylor1}{8cm}
}

\compileSOL{\SOLUb}{\ref{6Q52}}{
\subQ{a} Considérons $f(x) = x^{2001}$.  Nous cherchons à estimer $f(1.002)$.
L'approximation quadratique de $f$ pour $x$ près de $x=1$ est
\begin{align*}
p(x) &= f(1) + f'(1) ( x -1 ) + \frac{1}{2} f''(1)(x-1)^2
= 1 + 2001(x-1) + \frac{2001\times 2000}{2}\, (x-1)^2 \\
&= 1 + 2001(x-1) + 2.001\times 10^6 (x-1)^2 \ .
\end{align*}
Ainsi,
\[
f(1.002) \approx p(1.002) = 1 + 2001(1.002 - 1) +2.001\times 10^6 (1.002-1)^2
\approx 13.006 \ .
\]
Ce n'est pas une bonne approximation de
$1.002^{2001} \approx 54.489244196998690$ mais elle est meilleure que celle 
obtenue avec l'approximation linéaire.

\subQ{b} Considérons $f(x) = \sin(x)$.  Nous cherchons à estimer $f(0.02)$.
L'approximation quadratique de $f$ pour $x$ près de $x=0$ est
\[
p(x) = f(0) + f'(0) ( x -0 ) + \frac{1}{2} f''(0)(x-1)^2 = x
\]
Il n'y a pas d'approximation quadratique de $\sin(x)$ pour $x$ près de
l'origine.

\subQ{c} Le polynôme de Taylor de degré deux d'une fonction $f$ pour
$x$ près du point $a$ est
\[
p(x) = f(a) + f'(a) (x-a) + \frac{1}{2} f''(a) (x-a)^2 \ .
\]
Dans le cas présent, nous avons $f(x) = x^{3/4}$ et nous choisissons la valeur
$a=16$ qui est près de $16.2$ et pour laquelle il est facile de
calculer $f(a)$, $f'(a)$ et $f''(a)$.  Puisque
$\displaystyle f'(x) = \frac{3}{4} x^{-1/4}$ et
$\displaystyle f''(x) = -\frac{3}{16} x^{-5/4}$, nous obtenons
\[
p_2(x) = 16^{4/3} + \frac{3}{4} 16^{-1/4}(x-16) - \frac{3}{32} 16^{-5/4} (x-16)^2
= 8 + \frac{3}{8} (x-16) - \frac{3}{32^2}(x-16)^2 \ .
\]
Donc
\[
16.2^{4/3} = f(16.2) \approx p_2(16.2)
= 8 + \frac{3}{8} (16.2-16) - \frac{3}{32^2}(16.2-16)^2
= 8.07488281 \ldots
\]
La valeur exacte est $16.2^{4/3} \approx 8.074883419$.
}

\compileSOL{\SOLUb}{\ref{6Q53}}{
\subQ{a} Le polynôme de Taylor de degré trois de $f$ pour $x$ près de
$x=1$ est
\[
p_3(x) = f(1) + f'(1) (x-1) + \frac{f''(1)}{2}\, (x-1)^2 +
\frac{f'''(1)}{3!}(x-1)^3 \ .
\]
Puisque $f(x) = \ln(x)$, $f'(x) = 1/x$, $f''(x) = -1/x^2$ et
$f'''(x)= 2/x^3$, nous obtenons
\[
p_3(x) = 0 + (x-1) - \frac{1}{2}\, (x-1)^2 + \frac{2}{3!}(x-1)^3 
= (x-1) - \frac{1}{2}\, (x-1)^2 + \frac{1}{3}(x-1)^3 \ .
\]
Ainsi,
\[
\ln(1.2) = f(1.2) \approx p_3(1.2) = (1.2-1) - \frac{1}{2}\, (1.2-1)^2
+ \frac{1}{3}\,(1.2-1)^3 = 0.1826666\ldots
\]

\subQ{b}  Puisque $f(x) = x^{1/4}$, nous avons
\[
f'(x) =  \frac{1}{4} x^{-3/4} = \frac{1}{4\,x^{3/4}} \ , \ 
f''(x) =  \frac{-3}{16} x^{-7/4} = \frac{-3}{16\,x^{7/4}} \ 
\text{et} \ 
f'''(x) =  \frac{21}{64} x^{-11/4} = \frac{21}{64\,x^{11/4}} \ .
\]
Le polynôme de Taylor de degré trois de $f$ pour $x$ près de
$1$ est
\begin{align*}
p_3(x) &= f(1) + f'(1)(x-1) + \frac{1}{2!} f''(1) (x-1)^2 + \frac{1}{3!}
f'''(1) (x-1)^3 \\
&= 1 + \frac{1}{4} \, (x-1) - \frac{3}{32} \,(x-1)^2 +
\frac{21}{384}\,(x-1)^3 \ .
\end{align*}
Ainsi,
\[
1.4^{1/4} = f(1.4) \approx p_3(1.4)
= 1 + \frac{1}{4} (0.4)  - \frac{3}{32} (0.4^2)
+ \frac{21}{384} (0.4^3) = 1.0885 \ .
\]

\subQ{c} Puisque $\displaystyle f(x) = (1+x)^{-1/2}$,
$\displaystyle f'(x) = -\frac{1}{2}(1+x)^{-3/2}$,
$\displaystyle f''(x) = \frac{3}{4}(1+x)^{-5/2}$ et
$\displaystyle f'''(x) = -\frac{15}{8}(1+x)^{-7/2}$, le polynôme
de Taylor de degré trois de $f$ pour $x$ près de $3$ est
\begin{align*}
p_3(x) &= f(3) + f'(3) (x-3) + \frac{f''(3)}{2} (x-3)^2 +
\frac{f'''(3)}{3!} (x-3)^2 \\
& =\frac{1}{2} - \frac{1}{16} (x-3) + \frac{3}{128} (x-3)^2
- \frac{15}{1024} (x-3)^3 \ .
\end{align*}
Ainsi,
\[
f(3.1) \approx p(3.1) = \frac{1}{2} - \frac{1}{16} (3.1-3) +
\frac{3}{128} (3.1-3)^2 - \frac{15}{1024} (3.1-3)^3 \approx
0.49396973 \ .
\]

\subQ{d} Le polynôme de Taylor de degré trois de $f$ pour $x$ près de
$a$ est 
\[
p_3(x) = f(a) + f'(a)(x-a) + \frac{1}{2!} f''(a)(x-a)^2
+ \frac{1}{3!} f''(a)(x-a)^3 \ .
\]
Dans le cas présent, $f(x) = x^{4/3}$ et $a=8$.  Puisque
\[
f'(x) = \frac{4}{3}x^{1/3} \ , \ f''(x) = \frac{4}{9} x^{-2/3} \
\text{et} \ f'''(x) = - \frac{8}{27} x^{-5/3} \quad ,
\]
nous obtenons
\[
x^{4/3} = f(x) \approx p_3(x) = 16 + \frac{8}{3}\,(x-8) +
\frac{1}{18}\, (x-8)^2  - \frac{1}{648} \, (x-8)^3 \ .
\]
Ainsi,
\[
8.1^{4/3} = f(8.1) \approx p(8.1) = 16 + \frac{8}{30} + \frac{1}{1800}
- \frac{1}{648000} = 16.2672206790\ldots
\]

\subQ{e} Puisque
\begin{align*}
f'(x) &= \left(\dfdx{\arcsin(y)}{y} \bigg|_{y=x^2-1}\right) \dfdx{(x^2-1)}{x}
= \left(\frac{1}{\sqrt{1-y^2}} \bigg|_{y=x^2-1}\right) (2x) \\
&= \frac{2x}{\sqrt{1-(x^2-1)^2}}
= \frac{2x}{\sqrt{2 x^2 - x^4}}
= \frac{2}{\sqrt{2 - x^2}} \ , \\
f''(x) &= \dfdx{ \left(2(2 - x^2)^{-1/2}\right)}{x} = 2x (2-x^2)^{-3/2}
\intertext{et}
f'''(x) &= 2 (2-x^2)^{-3/2} + 2x \left( 3 x (2-x^2)^{-5/2}\right) \\
&= \left( 2 (2-x^2) + 6x^2\right) (2-x^2)^{-5/2}
= (4 + 4 x^2) (2-x^2)^{-5/2}
\end{align*}
pour $0<x<\sqrt{2}$, le polynôme de Taylor de $f$ de degré trois pour
$x$ près de $1$ est
\begin{align*}
p_3(x) &= f(1) + f'(1) (x-1) + \frac{f''(1)}{2!} (x-1)^2 + 
\frac{f'''(1)}{3!})x-1)^3 \\
&= 2(x-1) + (x-1)^2 + \frac{4}{3} (x-1)^3 \ .
\end{align*}
Ainsi,
\[
\arcsin(0.21) = f(1.1) \approx p_3(1.1)
= 2(1.1-1) + (1.1-1)^2 + \frac{4}{3} (1.1-1)^3
= 0.211\overline{3} \ .
\]
La valeur exacte est $\arcsin(0.21) = 0.21157496\ldots$
}

\compileSOL{\SOLUb}{\ref{6Q54}}{
Par définition, nous avons
\[
p_4(x) = f(3) + f'(3) (x-3) + \frac{f''(3)}{2} (x-3)^2 +
\frac{f'''(3)}{6} (x-3)^3 + \frac{f^{(4)}(3)}{24} (x-3)^4 \ .
\]
Donc $\displaystyle \frac{f''(3)}{2} = 0$ et
$\displaystyle \frac{f^{(4)}(3)}{24} = \frac{1}{3}$.  nous obtenons
$f''(3) = 10$ et $f^{(4)}(3) = 8$.
}

\compileSOL{\SOLUb}{\ref{6Q55}}{
Comme le polynôme de Taylor d'une fonction au voisinage d'un point donné est
unique, le polynôme de Taylor de $p$ au voisinage de l'origine est $p$
lui-même.
}

\compileSOL{\SOLUb}{\ref{6Q56}}{
Puisque $\displaystyle \dfdxn{\left(x^k\right)}{x}{9} = 0$ pour $k < 9$ et
$\displaystyle \dfdxn{\left(x^9\right)}{x}{9} = 9!$, nous avons que
$\displaystyle \dydxn{p}{x}{9}(x) = 7\times 9! >0$ pour tout $x$.
}

\subsection{Comportement asymptotique}

\compileSOL{\SOLUb}{\ref{6Q63}}{
Pour répondre à cette question, nous utilisons la Règle de l'Hospital.

\subQ{a}
Nous avons $\displaystyle \lim_{x\to\infty} f(x) = +\infty$ et
$\displaystyle \lim_{x\to\infty} g(x) = +\infty$.  Ainsi,
\[
\lim_{x\to \infty} \frac{f(x)}{g(x)} 
= \underbrace{\lim_{x\to \infty} \frac{0.1 x^{0.5}}{30\ln(x)}}_{
\text{forme $\infty/\infty$}}
= \lim_{x\to \infty} \frac{0.05 x^{-0.5}}{30/x}
= \lim_{x\to \infty} \frac{0.05 x^{0.5}}{30} = +\infty \ .
\]
Donc $f$ croît plus rapidement que $g$.

\subQ{b}
Nous avons $\displaystyle \lim_{x\to\infty} f(x) = 0$ et
$\displaystyle \lim_{x\to\infty} g(x) = 0$.  Ainsi,
\[
\lim_{x\to \infty} \frac{f(x)}{g(x)} 
= \lim_{x\to \infty} \frac{e^{-2x}}{x^{-2}}
= \underbrace{\lim_{x\to \infty} \frac{x^2}{e^{2x}}}_
{\text{forme $\infty/\infty$}}
= \underbrace{\lim_{x\to \infty} \frac{2x}{2e^{2x}}}_
{\text{forme $\infty/\infty$}}
= \lim_{x\to \infty} \frac{2}{4e^{2x}}
= \lim_{x\to \infty} \frac{1}{2} e^{-2x} = 0
\]
Donc $f$ tend vers $0$ plus rapidement que $g$.

\subQ{c}
Nous avons $\displaystyle \lim_{x\to 0^+} f(x) = +\infty$ et
$\displaystyle \lim_{x\to 0^+} g(x) = +\infty$.  Ainsi,
\[
\lim_{x\to 0^+} \frac{f(x)}{g(x)} 
= \underbrace{\lim_{x\to 0^+} \frac{x^{-1}}{-\ln(x)}}_
{\text{forme $\infty/\infty$}}
= \lim_{x\to 0^+} \frac{-2x^{-2}}{-1/x}
= \lim_{x\to 0^+} \frac{2}{x} = \infty
\]
Donc $f$ tend vers $+\infty$ plus rapidement que $g$.
}

\compileSOL{\SOLUb}{\ref{6Q64}}{
\subQ{a} La limite est du type $\infty/\infty$.  Nous pouvons appliquer la
Règle de l'Hospital.
\[
\lim_{x\rightarrow \infty} \frac{\ln(x)}{e^{3x}}
= \lim_{x\rightarrow \infty} \frac{1/x}{3e^{3x}}
= \lim_{x\rightarrow \infty} \frac{1}{3xe^{3x}} = 0
\]
car $3xe^{3x} \rightarrow \infty$ lorsque $x\rightarrow \infty$.

\subQ{b}  Puisque
$\displaystyle \lim_{t\rightarrow \infty} \frac{1}{t^n} = 0$ pour
$n>0$, nous avons
\[
\lim_{t\rightarrow \infty} \frac{1+t}{1+t+t^2}
= \lim_{t\rightarrow \infty} \frac{(1/t^2)+(1/t)}{(1/t^2)+(1/t)+1}
= \frac{\displaystyle
  \lim_{t\rightarrow \infty} (1/t^2)+\lim_{t\rightarrow \infty} (1/t)}
{\displaystyle
  \lim_{t\rightarrow \infty} (1/t^2)+\lim_{t\rightarrow \infty} (1/t)+1}
= \frac{0}{1} = 0 \ .
\]

\subQ{c} La limite est du type $0/0$.  Nous pouvons utiliser la
Règle de l'Hospital.
\[
\lim_{x\rightarrow 0} \frac{\sin(x) - x}{\cos(x) -1}
= \lim_{x\rightarrow 0} \frac{\cos(x) - 1}{-\sin(x)} \ .
\]
La limite à droite est encore une limite du type $0/0$.  Nous pouvons
encore utiliser la Règle de l'Hospital.
\[
\lim_{x\rightarrow 0} \frac{\sin(x) - x}{\cos(x) -1}
= \lim_{x\rightarrow 0} \frac{\cos(x) - 1}{-\sin(x)}
= \lim_{x\rightarrow 0} \frac{-\sin(x)}{-\cos(x)}
= \lim_{x\rightarrow 0} \tan(x) = 0 \ .
\]

\subQ{d} La limite est du type $\infty/\infty$. Nous pouvons utiliser
la Règle de l'Hospital.
\[
\underbrace{\lim_{z\rightarrow \infty} \frac{3z}{1+\ln(1+z)}}_
{\text{forme $\infty/\infty$}}
= \lim_{z\rightarrow \infty} \frac{3}{1/(1+z)}
= \lim_{z\rightarrow \infty} 3(1+z) = \infty  \ .
\]

\subQ{e} Nous avons une limite du type $0/0$.  Nous pouvons donc utiliser la
Règle de l'Hospital pour calculer cette limite.
\[
\lim_{x\rightarrow 0} \frac{\sin(3x^2)}{x^2}
= \lim_{x\rightarrow 0} \frac{6x\cos(3x^2)}{2x}
= \lim_{x\rightarrow 0} 3\cos(3x^2) = 3 \cos(0) = 3 \ .
\]

\subQ{f} $\displaystyle \lim_{x\to \pi/2} \frac{\cos^2(x)}{(x-\pi/2)^2}$
est une limite du type $0/0$.  Nous utilisons la Règle de l'Hospital pour
obtenir
\[
\lim_{x\to \pi/2} \frac{\cos^2(x)}{(x-\pi/2)^2} =
\lim_{x\to \pi/2} \frac{\dfdx{\cos^2(x)}{x}}{\dfdx{(x-\pi/2)^2}{x}} =
\lim_{x\to \pi/2} \frac{-2\cos(x)\sin(x)}{2(x-\pi/2)} =
\lim_{x\to \pi/2} \frac{-\cos(x)\sin(x)}{(x-\pi/2)} \ .
\]
Nous avons encore une limite du type $0/0$.  Nous utilisons une seconde
fois la Règle de l'Hospital pour obtenir
\begin{align*}
\lim_{x\to \pi/2} \frac{\cos^2(x)}{(x-\pi/2)^2}
&= \lim_{x\to \pi/2} \frac{-\cos(x)\sin(x)}{(x-\pi/2)}
= \lim_{x\to \pi/2} \frac{\dfdx{\left(-\cos(x)\sin(x)\right)}{x}}
{\dfdx{(x-\pi/2)}{x}} \\
&= \lim_{x\to \pi/2} \frac{\sin^2(x)-\cos^2(x)}{1} =
\sin^2(\pi/2)-\cos^2(\pi/2) = 1 \ .
\end{align*}

\subQ{g}
$\displaystyle \lim_{x\to \infty} \frac{e^{1/x}-1}{1/x}$ est une limite du
type $0/0$.  Nous utilisons la Règle de l'Hospital pour obtenir
\[
\lim_{x\to \infty} \frac{e^{1/x}-1}{1/x} =
\lim_{x\to \infty} \frac{\dfdx{(e^{1/x}-1)}{x}}{\dfdx{(1/x)}{x}} =
\lim_{x\to \infty} \frac{ e^{1/x} (-1/x^2)}{-1/x^2} =
\lim_{x\to \infty} e^{1/x} = 1 \ .
\]

\subQ{h} Nous avons une limite du type $\infty/\infty$.  Nous pouvons donc
utiliser la Règle de l'Hospital.
\[
\lim_{x\to 0^+} \frac{\ln(\sin(x))}{\ln(x^4)}
= \underbrace{\lim_{x\to 0^+} \frac{\ln(\sin(x))}{4\ln(x)}}_{\infty/\infty}
= \lim_{x\to 0^+} \frac{\cos(x)/\sin(x)}{4/x}
= \lim_{x\to 0^+} \frac{x\cos(x)}{4\sin(x)} \ .
\]
Nous avons maintenant une limite du type $0/0$.  Nous pouvons encore utiliser la
Règle de l'Hospital.
\[
\lim_{x\to 0^+} \frac{\ln(\sin(x))}{\ln(x^4)}
= \underbrace{\lim_{x\to 0^+} \frac{x\cos(x)}{4\sin(x)}}_{0/0}
= \lim_{x\to 0^+} \frac{\cos(x)-x\sin(x)}{4\cos(x)}
= \frac{1-0}{4} = \frac{1}{4} \ .
\]
}

\compileSOL{\SOLUb}{\ref{6Q65}}{
\subQ{a}
La limite est du type $0\cdot \infty$.  Par contre,
\[
\lim_{x\rightarrow \pi^+} \csc(5x)\sin(3x)
= \lim_{x\rightarrow \pi^+} \frac{\sin(3x)}{\sin(5x)}
\]
où la limite de droite est du type $0/0$.  Nous pouvons donc utiliser
la Règle de l'Hospital pour obtenir.
\[
\lim_{x\rightarrow \pi^+} \csc(5x)\sin(3x)
= \lim_{x\rightarrow \pi^+} \frac{\sin(3x)}{\sin(5x)}
= \lim_{x\rightarrow \pi^+} \frac{3\cos(3x)}{5\cos(5x)}
= \frac{3\cos(3\pi)}{5\cos(5\pi)} = \frac{3}{5} \ .
\]

\subQ{b}
Nous avons une limite du type $\infty - \infty$.  Par contre,
\[
\lim_{x\rightarrow 0^+} \left( \frac{1}{\sin(x)} -\frac{1}{x} \right) =
\lim_{x\rightarrow 0^+} \frac{x-\sin(x)}{x\sin(x)}
\]
où la limite de droite est du type $0/0$.  Nous pouvons utiliser la Règle de
l'Hospital pour obtenir
\[
\lim_{x\rightarrow 0^+} \left( \frac{1}{\sin(x)} -\frac{1}{x} \right)
= \lim_{x\rightarrow 0^+} \frac{x-\sin(x)}{x\sin(x)}
= \lim_{x\rightarrow 0^+} \frac{1-\cos(x)}{\sin(x) +x\cos(x)} \ .
\]
Cette dernière limite est encore du type $0/0$.  Un deuxième appel à la Règle
de l'Hospital donne
\begin{align*}
\lim_{x\rightarrow 0^+} \left( \frac{1}{\sin(x)} -\frac{1}{x} \right)
&= \lim_{x\rightarrow 0^+} \frac{1-\cos(x)}{\sin(x) +x\cos(x)}
= \lim_{x\rightarrow 0^+} \frac{\sin(x)}{2\cos(x) -x\sin(x)} \\
&= \frac{\sin(0)}{2\cos(0) -0\sin(0)} = 0 \ .
\end{align*}

\subQ{c} Nous avons une limite du type $\infty - \infty$.  Par contre
\[
\left(\csc(x) - \cot(x)\right) = \left(\frac{1}{\sin(x)}
- \frac{\cos(x)}{\sin(x)}\right) = \frac{1-\cos(x)}{\sin(x)}
\]
transforme la limite en une limite du type $0/0$.  Ainsi,
\[
\lim_{x\rightarrow 0^+} \left(\csc(x) - \cot(x)\right)
= \lim_{x\rightarrow 0^+} \frac{1-\cos(x)}{\sin(x)}
= \lim_{x\rightarrow 0^+} \frac{\sin(x)}{\cos(x)} 
= \frac{\sin(0)}{\cos(0)} = 0
\]
où la deuxième égalité est une conséquence de la Règle de l'Hospital.

\subQ{d} Après une petite manipulation algébrique, nous pouvons
utiliser la Règle de l'Hospital.
\begin{align*}
\underbrace{\lim_{x\to 0^+} 7x \cot(3x)}_{\text{type }0\cdot \infty} &
= \lim_{x\to 0^+} 7\cos(3x)\ \frac{x}{\sin(3x)}
= \left(\lim_{x\to 0^+} 7\cos(3x) \right) 
\underbrace{\left( \lim_{x\to 0^+}\frac{x}{\sin(3x)}\right)}_{ \text{type }(0/0)} \\
&= 7 \lim_{x\to 0^+} \frac{1}{3\cos(3x)} = \frac{7}{3\cos(0)} =
\frac{7}{3} \ .
\end{align*}
Nous avons utilisé le fait que $\cos(3x)$ est une fonction continue à
l'origine et donc que $\displaystyle \lim_{x\to 0}\cos(3x) = \cos(0) = 1$.

Un autre approche est d'utiliser
$\displaystyle \lim_{y\to 0} \frac{y}{\sin(y)} = 1$.  Donc
\begin{align*}
\lim_{x\to 0^+} 7x \cot(3x) &
= \lim_{x\to 0^+} \frac{7\cos(x)}{3}\ \frac{3x}{\sin(3x)}
= \left(\lim_{x\to 0^+} \frac{7\cos(x)}{3}\right)
\left( \lim_{x\to 0^+}\frac{3x}{\sin(3x)}\right) \\
&= \frac{7}{3} \underbrace{\left(\lim_{y\to 0^+}\frac{y}{\sin(y)}\right)}_{y=3x}
= \frac{7}{3} \ .
\end{align*}

\subQ{f} Après une petite manipulation algébrique, nous pouvons
utiliser la Règle de l'Hospital.
\begin{align*}
\underbrace{\lim_{x\to \pi} \cot^2(x) (x-\pi)^2}_{\infty \cdot 0} &
= \lim_{x\to \pi} \cos^2(x) \frac{(x-\pi)^2}{\sin^2(x)}
= \left( \lim_{x\to \pi} \cos^2(x) \right)
\underbrace{\left(\lim_{x\to \pi}\frac{(x-\pi)^2}{\sin^2(x)
}\right)}_{ \text{type }(0/0)} \\
&= \underbrace{\lim_{x\to  \pi}
\frac{2(x-\pi)}{2\sin(x)\cos(x)}}_{\text{type }(0/0)}
= \lim_{x\to \pi}\frac{2}{2\cos^2(x) - 2\sin^2(x)} = \frac{2}{2} = 1 \ . 
\end{align*}
Nous avons utilisé le fait que $\cos(x)$ et $\sin(x)$ sont des fonctions
continues à $x= \pi$ et donc
que $\displaystyle \lim_{x\to \pi}\cos(x) = \cos(\pi) = -1$ et
$\displaystyle \lim_{x\to \pi}\sin(x) = \sin(\pi) = 0$.

\subQ{g} Nous avons une limite du type $(\infty - \infty)$.  Nous
avons cependant que
\[
\lim_{x\to +\infty} \left( x  \cos\left(\frac{1}{x}\right) - x \right)
= \lim_{x\to +\infty} \frac{\cos(1/x)-1}{1/x}
\]
où la limite à droite est une limite du type $0/0$.  Nous pouvons donc
utiliser la Règle de l'Hospital.
\[
\lim_{x\to +\infty} \frac{\cos(1/x)-1}{1/x}
= \lim_{x\to +\infty} \frac{(1/x^2) \sin(1/x)}{-1/x^2}
= \lim_{x\to +\infty} -\sin\left(\frac{1}{x}\right) = -\sin(0) = 0 \ .
\]

\subQ{h} Puisque
\[
(\cos(x))^{1/x^2} = e^{\ln( (\cos(x))^{1/x^2})} =  e^{(1/x^2)\ln(\cos(x))}
\]
pour $-\pi/2 < x < \pi/2$, et la fonction exponentielle est continue
sur la droite réelle, nous avons 
\[
\lim_{x\to 0} (\cos(x))^{1/x^2}
= e^{\lim_{x\to 0} (1/x^2)\ln(\cos(x))} \ .
\]
Il suffit donc de calculer la limite que nous retrouvons comme exposant.
Pour ce faire, nous utilisons la Règle de l'Hospital.
\begin{align*}
\underbrace{\lim_{x\to 0} \frac{\ln(\cos(x))}{x^2}}_{\text{type }0/0}
&= \lim_{x\to 0} \frac{-\sin(x) / \cos(x)}{2x}
= \underbrace{\lim_{x\to 0} \frac{-\sin(x)}{2x \cos(x)}}_{\text{type }0/0} \\
&= \lim_{x\to 0} \frac{-cos(x)}{2\cos(x) - 2x \sin(x)}
= \frac{-1}{2}
\end{align*}
car la fonction impliquée dans la dernière limite est continue à
$x=0$, il suffit donc d'évaluer cette fonction pour 
obtenir la limite.  Finalement
\[
\lim_{x\to 0} (\cos(x))^{1/x^2} = e^{\lim_{x\to 0} (1/x^2)\ln(\cos(x))}
= e^{-1/2} \ .
\]
}

\subsection{Méthode de Newton}

\compileSOL{\SOLUb}{\ref{6Q66}}{
La première itération est
\[
x_1 = x_0 - \frac{f(x_0)}{f'(x_0)} = x_0 - \frac{x_0^4 -20}{4x_0^3}
= 2.125
\]
Puisque $|x_1 - x_0| = 0.125 > 10^{-4}$, nous calculons l'itération
suivante.
\[
x_2 = x_1 - \frac{f(x_1)}{f'(x_1)} = x_1 - \frac{x_1^4 -20}{4x_1^3}
\approx 2.11481856
\]
Puisque $|x_2 - x_1| \approx 0.01018 > 10^{-4}$, nous calculons l'itération
suivante.
\[
x_3 = x_2 - \frac{f(x_2)}{f'(x_2)} = x_2 - \frac{x_2^4 -20}{4x_2^3}
\approx 2.11474253
\]
Puisque $|x_3 - x_2| \approx 5.40273 \times 10^{-5} < 10^{-4}$, nous
pouvons utiliser $x_3 \approx 2.11474253$ comme approximation de la
solution de $x^4 -20 = 0$.

La solution exacte est $2.11474253\ldots$
}

\compileSOL{\SOLUb}{\ref{6Q68}}{
\subQ{a} Nous pourrions tracer le graphe de $f$.  Nous invitons le lecteur à
le faire.  Puisque nous cherchons la solution de $e^x = x+2$, il est
préférable de tracer les graphes de $e^x$ et de $x +2$ pour trouver
approximativement où elles se coupent.  De plus, comme ce sont deux
graphes bien connus, nous pouvons utiliser cette information pour confirmer
le nombre de points d'intersection.
\MATHgraph{6_derivees_appl/newton2}{8cm}

En regardant le graphe ci-dessus, il est claire qu'il y a une seul
solution positive entre $1$ et $1.5$\;.

\subQ{b} $x=c$ est une solution de $e^x = x+2$ si et seulement si
$x=c$ est une solution de $f(x) = e^x-x-2 = 0$.  Pour répondre à la
question. il suffit donc de trouver la solution de $f(x)=0$.

Pour choisir $x_0$, nous remarquons que $f(1) \approx -0.2817 <0$ et
$f(1.5) \approx 0.981689 >0$.  Puisque $f$ est une fonction continue sur
l'intervalle $[1, 1.5]$, le Théorème des valeurs intermédiaires
garantit qu'il existe une valeur $c$ entre $1$ et 
$1.5$ telle que $f(c)=0$.  C'est effectivement ce que nous avions
observé à l'aide de la figure ci-dessus.  Nous choisissons donc
$x_0 \in [1, 1.5]$.

\subQ{c} Prenons $x_0 = 1.5$ pour résoudre numériquement $f(x) = 0$
à l'aide de la méthode de Newton
\[
x_{n+1} = x_n - \frac{f(x_n)}{f'(x_n)} \quad , \quad n=0,1,2,3,\ldots 
\]
Nous avons les quatre itérations suivantes:
\begin{align*}
x_1 &= x_0 - \frac{f(x_0)}{f'(x_0)} = x_0 - \frac{e^{x_0}-x_0-2}{e^{x_0} -1}
\approx 1.218042\\
x_2 &= x_1 - \frac{f(x_1)}{f'(x_1)} = x_1 - \frac{e^{x_1}-x_1-2}{e^{x_1} -1}
\approx 1.1497723\\
x_3 &= x_2 - \frac{f(x_2)}{f'(x_2)} = x_2 - \frac{e^{x_2}-x_2-2}{e^{x_2} -1}
\approx 1.1462025\\
x_4 &= x_3 - \frac{f(x_3)}{f'(x_3)} = x_3 - \frac{e^{x_3}-x_3-2}{e^{x_3} -1}
\approx 1.1461932
\end{align*}
Donc $c \approx 1.1461932$.
}

\compileSOL{\SOLUb}{\ref{6Q69}}{
\subQ{a}
L'équation de la sécante qui passe par $(x_i, f(x_i)$ et
$(x_{i-1}, f(x_{i-1})$ est
\[
y = \frac{f(x_{i-1}) - f(x_i)}{x_{i-1} - x_i}\, (x-x_i) + f(x_i) \ .
\]
Lorsque cette droite coupe l'axe des $x$ (c'est-à-dire, lorsque $y=0$),
nous avons
\[
0 = \frac{f(x_{i-1}) - f(x_i)}{x_{i-1} - x_i}\, (x-x_i) + f(x_i) \ .
\]
Si nous résolvons cette équation pour $x$, nous trouvons
\[
x = x_i - f(x_i) \left(\frac{f(x_{i-1}) - f(x_i)}{x_{i-1} - x_i}\right)^{-1}
\ .
\]
C'est la formule qui donne $x_{i+1}$.
\PDFgraph{6_derivees_appl/secante1}

\subQ{b}
Comme à la question~\ref{6Q68}, nous considérons la fonction
$f(x) = e^x-x-2$.  Nous choisissons $x_0$ et $x_1$ dans l'intervalle
$[1,2]$ où $f$ change de signe.  Soit $x_0=1.6$ et $x_1=1.7$.  Nous
obtenons les quatre itérations suivantes:
\begin{align*}
x_2 & = x_1 - f(x_1)\left(\frac{f(x_1)- f(x_0)}{x_1-x_0}\right)^{-1}
= f(1.7) - f(1.7) \left(\frac{f(1.7)-f(1.6)}{1.7-1.6}\right)^{-1} \\
&\approx 1.2785497 \\
x_3 & = x_2 - f(x_2)\left(\frac{f(x_2)- f(x_1)}{x_2-x_1}\right)^{-1}
\approx 1.1882991 \\
x_4 & = x_3 - f(x_3)\left(\frac{f(x_3)- f(x_2)}{x_3-x_2}\right)^{-1}
\approx 1.150012 \\
x_5 & = x_4 - f(x_4)\left(\frac{f(x_4)- f(x_3)}{x_4-x_3}\right)^{-1}
\approx 1.1463089
\end{align*}
Donc $e^c = c+ 2$ pour $c \approx 1.1463089$.  Remarquons que la méthode de
la sécante tend vers la solution de $f(x)=0$ plus lentement que la méthode
de Newton.
}

\subsection{Systèmes dynamiques discrets}

\compileSOL{\SOLUb}{\ref{6Q70}}{
La solution du système dynamique discret est
$v_n = 1.5^n v_0 = 1350 \times 1.5^n$.  Nous cherchons le plus petit entier $n$
tel que $v_n \geq 3250$.
\begin{align*}
1350 \times 1.5^n \geq 3250
&\Leftrightarrow 1.5^n \geq \frac{3250}{1350} = \frac{65}{27}
\Leftrightarrow \ln(1.5^n) \geq \ln\left(\frac{65}{27}\right) \\
&\Leftrightarrow n \ln(1.5) \geq \ln(65)-\ln(27)
\Leftrightarrow n \geq \frac{\ln(65)-\ln(27)}{\ln(1.5)} = 2.1667719\ldots
\end{align*}
car $\ln(1.5)>0$.  Puisque $n$ doit être un entier, nous choisissons $n=3$
pour être certain que le volume sera plus grand que $3250\ \mu m^3$.
}

\compileSOL{\SOLUa}{\ref{6Q71}}{
\subQ{a}  Nous cherchons $p_9$ tel que $10^8 < p_{10} = 2 p_9 < 10^9$.
Donc
\[
5 \times 10^7 = \frac{10^8}{2} < p_9 < \frac{10^9}{2} = 5 \times 10^8 \ .
\]

\subQ{b} La solution du système dynamique discret est de la forme
$p_n = 2^n p_0$ où $p_0$ est le nombre initial d'individus.
Nous cherchons $p_0$ tel que $10^8 < p_{10} = 2^{10} p_0 < 10^9$.
Donc
\[
\frac{5^8}{4} = \frac{10^8}{2^{10}} < p_0 < \frac{10^9}{2^{10}}
= \frac{5^9}{2} \ .
\]
}

\compileSOL{\SOLUb}{\ref{6Q72}}{
La solution est $y_i = 0.5^i y_0 =1200 \times 0.5^i$ pour $i \geq 0$ et
$y_{20} = 1200\times 0.5^{20} = 0.00114440\ldots$ 

Voici le graphe de la solution avec la condition initiale $y_0 = 1$.
\MATHgraph{6_derivees_appl/linear3_a}{8cm}

Voici le graphe de la fonction itérative $z = f(y) = 0.5 y$.
\MATHgraph{6_derivees_appl/linear3_b}{8cm}

Notons que la fonction itérative est strictement croissante alors que la
solution est strictement décroissante.
}

\compileSOL{\SOLUa}{\ref{6Q73}}{
La solution générale est de la forme
$b_i = 0.7^i b_0 = 5.0\times 10^5 \times 0.7^i$ pour $i = 0$, $1$, $2$, \ldots

Si  $b_i \approx 10^5$, nous obtenons
\begin{align*}
0.7^i \times 5 \times 10^5 \approx 10^5
&\Leftrightarrow 0.7^i \times 5 \approx 1
\Leftrightarrow 0.7^i \approx \frac{1}{5} \\
&\Leftrightarrow i\ln(0.7) = \ln(0.7^i) \approx \ln\left(\frac{1}{5}\right)
= - \ln(5) \ .
\end{align*}
Donc $\displaystyle i \approx \frac{-\ln(5)}{\ln(0.7)} = 4.512338026$.
Nous choisissons l'entier le plus près de $4.512338026$, soit $i=5$.

Le graphe de la solution du système dynamique
$b_{i+1} = 0.7 b_i$ est donné ci-dessous.
\MATHgraph{6_derivees_appl/linear6}{8cm}
}

\compileSOL{\SOLUb}{\ref{6Q74}}{
La solution générale du premier système est
$x_n = 10^2\times 2.5^n$ pour $n \geq 0$ et la solution générale du
second système $y_n = 10^3 \times 2^n$ pour $n\geq 0$.  Ainsi,
\[
\lim_{n\to \infty} \frac{y_n}{x_n}
= \lim_{n\to \infty} \frac{10^3 \times 2^n}{10^2\times 2.5^n} 
= \lim_{n\to \infty} 10\,\left(\frac{2}{2.5}\right)^n
= \lim_{n\to \infty} 10\,\left(\frac{4}{5}\right)^n = 0
\]
car $\displaystyle \lim_{n\to \infty} r^n = 0$ pour $|r|<1$.  La
population associée au système (\ref{sect26eq1}) tend donc plus rapidement
vers plus l'infini que la population associée au système (\ref{sect26eq2}).
}

\compileSOL{\SOLUa}{\ref{6Q75}}{
La fonction itérative est $f(x) = 4-x$.  Nous avons
$x_0 = 1$, $x_1 = 4-x_0 = 3$, $x_2 = 4-x_1 = 1$, $x_3 = 4-x_2 = 3$, \ldots
La solution est
\[
x_i = 
\begin{cases}
1 & \quad \text{si $i$ est pair} \\
3 & \quad \text{si $i$ est impair}
\end{cases}
\]
}

\compileSOL{\SOLUb}{\ref{6Q76}}{
Nous avons $h_0 = 1$, $h_1 = h_0 +1 =2$, $h_2 = h_1 +1 =3$, $h_3 = h_2 +1 =4$,
\ldots\ La solution est donc $h_i = i+1$ pour $i=0$, $1$, $2$,
\ldots\ En effet, 
\[
h_i + 1 = (i+1)+1 = i + 2 = h_{i+1} \quad \text{pour}\quad
 i = 0, 1, 2, \ldots
\]
Ainsi, $h_{20} = 21$.

Le modèle est plausible pour les premières années mais n'est pas
valable après plusieurs années.  La croissance de l'arbre devrait
ralentir avec le temps.  L'arbre devrait arrêter de grandir après un
certain nombre d'années.
}

\compileSOL{\SOLUa}{\ref{6Q77}}{
La fonction itérative est $f(x) = 2x + 30$.  Pour obtenir la solution, il
faut trouver le point d'équilibre; c'est-à-dire, $p$ tel que $p=f(p)$.
\[
p = f(p) \Leftrightarrow p = 2 p + 30 \Leftrightarrow p = -30 \ .
\]
La solution est
\[
x_i = 2^i(x_0 -p) + p = 2^i(10 + 30) - 30 = 40\times 2^i - 30 \ .
\]
}

\compileSOL{\SOLUb}{\ref{6Q78}}{
Le tableau nous donne plusieurs valeurs $v_0$ (colonne de gauche) et $v_1$
(colonne de droite).  Ces valeurs doivent satisfaire
$v_1 = a v_0 + b$.  Si nous prenons les deux premières pairs, nous
obtenons le système d'équations linéaires suivant.
\begin{align*}
1830 & = 1220 a  + b \\
2790 &= 1860 a + b
\end{align*}
La solution pour $a$ et $b$ de ce système est $a=3/2$ et
$b=0$. Nous avons donc $\displaystyle v_{i+1} = (3/2) v_i$.

Pour compléter le tableau, notons que
$\displaystyle v_1 = (3/2) \times 1420 = 2130$.

Si le valeur initial est $v_0 = 1420$, le valeur après une heure sera
donné par $v_6$ car chaque itération représente $10$ minutes.  La
solution du système dynamique discret $\displaystyle v_{i+1} = (3/2) v_i$ est
$\displaystyle v_i = (3/2)^i v_0 = 1420 \times (3/2)^i$ pour $i\geq 0$.  Donc
$\displaystyle v_6 = 1420 \times (3/2)^6 = 16174.6875$.

Le modèle est valide pour une certaine période de temps mais il est
clair qu'il n'est pas valide à long terme car $v_n \to \infty$ lorsque
$n\to \infty$.  Le nombre de moyen bactéries ne peut pas augmenter
indéfiniment.
}

\compileSOL{\SOLUa}{\ref{6Q79}}{
Le tableau nous donne $p_0$, $p_1$, $p_2$ et $p_3$.  Ces valeurs doivent
satisfaire $p_{i+1} = a p_i + b$.  Si nous utilisons les valeurs $p_0$, $p_1$,
$p_2$, nous obtenons le système d'équations linéaires suivant.
\begin{align*}
p_1 = a p_0 + b &\Rightarrow 16 = 20 a  + b \\
p_2 = a p_1 + b &\Rightarrow 13 = 16 a + b
\end{align*}
La solution de ce système pour $a$ et $b$ est $a=3/4$ et $b=1$.
Nous avons donc $\displaystyle p_{i+1} = (3/4) p_i + 1$.

La fonction itérative est $\displaystyle f(x) = (3/4) x + 1$.  Pour
obtenir la solution, il faut trouver le point d'équilibre;
c'est-à-dire, le point $p$ tel que $p=f(p)$.
\[
p = f(p) \Leftrightarrow p = \frac{3}{4} p + 1 \Leftrightarrow p = 4 \ .
\]
La solution est
\[
p_i = \left(\frac{3}{4}\right)^i(p_0 -p) + p
= \left(\frac{3}{4}\right)^i(20 -4 ) + 4
= 16\times \left(\frac{3}{4}\right)^i + 4 \ .
\]
\MATHgraph{6_derivees_appl/linear4}{8cm}

Le modèle est réaliste.  La concentration peut devenir nulle.
}

\compileSOL{\SOLUb}{\ref{6Q80}}{
\subQ{a} Nous trouvons le point fixe $M$ pour le système dynamique discret
(\ref{sdd18}); c'est-à-dire, nous cherchons $M$ tel que $M=0.75 M + 2$.  La
solution de cette équation est $M=8$.  Ainsi la solution générale est
\[
M_i = 0.75^i(M_0-8)+8 \ .
\]

% Les premières valeurs de $M_i$ sont
% \begin{align*}
% M_0 &= 16 \\
% M_1 &= 0.75 M_0 + 2 = 0.75(16-8)+8 = 14\\
% M_2 &= 0.75 M_1 + 2 = 0.75 (0.75 M_0 + 2) + 2
% = 0.75^2 M_0 + 0.75 \times 2 + 2 \\
% M_3 &= 0.75 M_2 + 2 = 0.75 (0.75^2 M_0 + 0.75\times 2 + 2) + 2\\
% &= 0.75^3 M_0 + 0.75^2 \times 2 + 0.75 \times 2 + 2 \\
% M_4 &= 0.75 M_3 + 2 = 0.75 (0.75^3 M_0 + 0.75^2 \times 2 +0.75\times 2 + 2)
% + 2 \\
% &= 0.75^4 M_0 + 0.75^3 \times 2 + 0.75^2 \times 2 + 0.75 \times 2 + 2 \\
% & \vdots
% \end{align*}
% Par induction, nous trouvons que
% \begin{align*}
% M_i &= 0.75^i M_0 + 2 \sum_{k=0}^{i-1} 0.75^k 
% = 0.75^i M_0 + 2 \left( \frac{1-0.75^i}{1-0.75}\right) \\
% &= 0.75^i (M_0 - 8) + 8
% \end{align*}
% Puisque $0.75^i$ approche $0$ lorsque $i$ devient de plus en plus
% grand, nous avons que $M_i$ approche $8$ lorsque $i$ tend vers plus infini.

\subQ{b} Nous avons
\begin{align*}
M_0 & =16  \\
M_1 &= 0.75(16-8)+8 = 14 \\
M_2 &= 0.75^2(16-8)+8 = 12.5 \\
M_3 &= 0.75^3(16-8)+8 = 11.375 \\
M_4 &= 0.75^4(16-8)+8 = 10.53125 \\
M_5 &= 0.75^5(16-8)+8 = 9.8984375
\end{align*}

\subQ{c}
Le dessin à droite dans la figure ci-dessous représente la solution du
système (\ref{sdd18}) pour $M_0 = 16$.

\subQ{d}
Le dessin à gauche dans la figure ci-dessous représente le graphe de
la fonction itérative du système dynamique discret.
\PDFgraph{6_derivees_appl/linear7}

\subQ{e} Nous utilisons le résultat en (a) pour obtenir
\[
M_{60} = 0.75^{60} (10 - 8) + 8 = 0.75^{60}\times 2 + 8 \approx
8.000000064 \ .
\]
}

\compileSOL{\SOLUa}{\ref{6Q81}}{
La fonction itérative est $y=f(x) = 2x-5$.  Nous avons dessiné le graphe de la
fonction itérative ci-dessous.
\PDFgraph{6_derivees_appl/linear18}

Le point d'équilibre $p$ du système dynamique discret est la solution
de l'équation $p = 2p-5$; c'est-à-dire, $p = 5$.  Le graphe de la
fonction itérative est au-dessus de la droite $y=x$ pour $x>5$ et
en dessous de la droite $y=x$ pour $x<5$.
}

\compileSOL{\SOLUb}{\ref{6Q82}}{
\MATHgraph{6_derivees_appl/linear10}{8cm}
}

\compileSOL{\SOLUb}{\ref{6Q83}}{
La fonction itérative est $f(w) = -0.5 w + 3$.

Pour obtenir la solution, il faut trouver le point d'équilibre;
c'est-à-dire, le point $p$ tel que $p=f(p)$.
\[
p = f(p) \Leftrightarrow p = -0.5 p + 3 \Leftrightarrow p = 2 \ .
\]
La solution est
\[
  w_i = (-0.5)^i(w_0 -p) + p = (-0.5)^i(0.2 - 2 ) + 2
  = -1.8 \times (-0.5)^i + 2 \ .
\]
Le graphe de la fonction itérative est la droite de pente $-0.5$ et
d'ordonnée à l'origine $3$.  Le graphe en forme de toile d'araignée
est donnée ci-dessous.
\MATHgraph{6_derivees_appl/linear11}{8cm}

Puisque $(-0.5)^i \rightarrow 0$ lorsque $i\rightarrow \infty$, il n'est pas
surprenant que $w_i \rightarrow 2$ lorsque $i\rightarrow \infty$.
Nous avons que $-1.8 \times (-0.5)^i$ est positif pour $i$ impair et
négatif pour $i$ pair.  Cela explique pourquoi les $x_i$ alternent
entre plus petit et plus grand que $2$.
}

\compileSOL{\SOLUb}{\ref{6Q84}}{
La fonction itérative est $\displaystyle f(z) = 0.5 z + 8$.

Pour obtenir la solution, il faut trouver le point d'équilibre;
c'est-à-dire, le point $p$ tel que $p=f(p)$.
\[
p = f(p) \Leftrightarrow p = 0.5 p + 8 \Leftrightarrow p = 16 \ .
\]
La solution est
\[
z_i = 0.5^i(z_0 -p) + p = 0.5^i(2 - 16 ) + 16 = -14\times 0.5^i + 16 \ .
\]
Le graphe de la fonction itérative est la droite de pente 0.5 et
d'ordonnée à l'origine 8 qui est tracé en noire dans la figure
ci-dessous.  La figure contient aussi le graphe en forme de toile
d'araignée.
\MATHgraph{6_derivees_appl/linear12}{8cm}
}

\compileSOL{\SOLUa}{\ref{6Q85}}{
Nous avons trouver à la question~\ref{6Q79} que la concentration
$p_n$ d'un médicament après $n$ heures satisfait le système dynamique
discret $p_{n+1} = 0.75 p_n + 1$.  Nous avons aussi montré que $p=4$
était le point d'équilibre et que la solution était
$\displaystyle p_n = 16\times 0.75^n + 4$.

Le graphe en forme de toile d'araignée est donné ci-dessous.
\MATHgraph{6_derivees_appl/linear13}{8cm}
}

\compileSOL{\SOLUb}{\ref{6Q86}}{
Pour la première partie de la question, le nombre de bactéries $x_i$
que nous avons après $i$ heures satisfait le système dynamique discret
\[
x_{i+1}  = 2 x_i - 10^6 \quad \text{pour} \quad i=0.1.2.3.\ldots
\]
Le point d'équilibre de ce système est donné par la solution de
l'équation $\displaystyle x = 2 x - 10^6$, soit $x= 10^6$.  La solution du
système dynamique discret est donc
\[
x_i = 2^i \left(x_0 - 10^6\right) + 10^6
\quad \text{pour} \quad i=0, 1, 2, 3, \ldots
\]
Le graphe en forme de toile d'araignée est donné ci-dessous.
\MATHgraph{6_derivees_appl/linear14}{8cm}

Pour la deuxième partie de la question, le système dynamique discret est
\[
x_{i+1} = 2 (x_i - 10^6) \ .
\]
En d'autres mots, avant de doubler le nombre de bactéries, nous retirons
$10^6$ bactéries.  Le point d'équilibre de ce système est donné par la
solution de l'équation $\displaystyle x = 2 ( x - 10^6)$, soit
$x= 2 \times 10^6$.  La solution du système dynamique discret est donc
\[
x_i = 2^i \left(x_0 - 2\times 10^6\right) + 2\times 10^6
\quad \text{pour} \quad i=0, 1, 2, 3, \ldots
\]
Le graphe en forme de toile d'araignée est donné ci-dessous.
\MATHgraph{6_derivees_appl/linear15}{8cm}

Les deux graphes en forme de toile d'araignée sont légèrement
différents si nous comparons les valeurs de $x_i$ mais, qualitativement,
ils sont identique.  Dans les deux cas, le nombre de bactéries croit
sans borne supérieure; c'est-à-dire, $x_i\rightarrow \infty$ lorsque
$i\rightarrow \infty$.

Dans les deux cas, le système n'est pas vraiment réaliste à long terme
même si la croissance est exponentielle au départ.
}

\compileSOL{\SOLUa}{\ref{6Q87}}{
Le système dynamique discret est de la forme
\[
x_{i+1} = f(x_i) = m x_i + b \quad \text{pour} \quad i=0, 1, 2, 3, \ldots
\]
où $f(x) = mx + b$ est la fonction itérative.   Nous utilisons les deux
premiers itérations pour déterminer $m$ et $b$;
c'est-à-dire, nous utilisons $x_1 = f(x_0) = m x_0 + b$ et
$x_2 = f(x_1) = m x_1 + b$.  Il faut donc résoudre le système d'équations
linéaires suivant.
\begin{align*}
2 &= m \times 0 + b \\
3.2 &= m \times 2 + b
\end{align*}
La solution pour $m$ et $b$ est $m= 0.6$ et $b=2$.
La fonction itérative est donc $f(x) = 0.6 x + 2$.  Le système
dynamique discret est
\[
x_{i+1} = 0.6 x_i + 2  \quad \text{pour} \quad i=0, 1, 2, 3, \ldots
\]
Le point d'équilibre du système dynamique discret
\[
x_{i+1} = 0.6 x_i + 2 \quad \text{pour} \quad i=0,1,2,3,\ldots
\]
est la solution de l'équation $p=f(p) = 0.6p+2$ soit $p=5$.

Le graphe de la fonction itérative (en noire) du système dynamique
discret $x_{i+1} = 0.6 x_i + 2$ avec le graphe en forme de toile
d'araignée associé à la condition initiale $x_0 =0$ se retrouve dans
la figure suivante.
\PDFgraph{6_derivees_appl/linear16}
}

\compileSOL{\SOLUb}{\ref{6Q88}}{
La fonction itérative du système dynamique discret est $f(x) = 0.9 x + 8$.
le point d'équilibre est la solution de $p=0.9 p + 8$, soit $p = 80$.

Puisque la pente de la droite $y=f(x)$ est entre $-1$ et $1$, le point
d'équilibre sera (asymptotiquement) stable d'après le théorème sur la
stabilité des points d'équilibre.

Nous avons tracé le graphe en forme de toile d'araignée pour $x_0=10$
ci-dessous.  Nous voyons bien que les orbites tendent vers le point
d'équilibre.
\MATHgraph{6_derivees_appl/linear17}{8cm}
}

\compileSOL{\SOLUb}{\ref{6Q90}}{
Si $45$\% du médicament est éliminé en un jour, alors, après un jour,
nous avons $55$\% de la quantité du médicament que nous avions au début de la
journée.  Nous ajoutons $50$ mg/l à chaque jour avant de mesurer la
quantité du médicament dans l'organisme du patient.
Donc $x_{t+1} = 0.55 x_t + 50$ pour $t=0$, $1$, $2$, \ldots
La condition initiale est $42$ mg/l.  Donc $x_0 = 42$.
}

\compileSOL{\SOLUa}{\ref{6Q91}}{
\subQ{a}  S'il y a $p_n$ g de graines au début de la semaine et les
oiseaux en manche 75\%, il en reste donc $0.25 p_n$ g à la fin de la
semaine.  Puis nous ajoutons $30$ g.  Nous avons donc $0.25 p_n + 30$ g au
début de la semaine suivante.  Ainsi, le système dynamique discret est
\[
  p_{n+1} = 0.25 p_n + 30 \quad \text{pour} \quad n = 0, 1, 2, 3, \ldots
\]

\subQ{b} La fonction génératrice est $f(x) = 0.25 x + 30$.

\subQ{c} Le point d'équilibre du système dynamique discret est la
solution de l'équation $f(p)= 0.25 p + 30 = p$, soit
$p = 30/0.75 = 40$.

\subQ{d} La solution générale du système dynamique discret est
\[
p_n = 0.25^n(p_0-p)+p = 0.25^n(5-40)+40 = -35 \times 0.25^n + 40 \ .
\]

\subQ{e} Nous avons tracé le graphe de la fonction génératrice et le graphe
en forme de toiles d'araignées dans la figure ci-dessous.
\MATHgraph{6_derivees_appl/linear19}{8cm}

\subQ{f} Puisque $|f'(40)| = |0.25| < 1$, nous avons que $p=40$ est une
point d'équilibre (asymptotiquement) stable grâce au théorème de
stabilité des points d´équilibre de système dynamique linéaire.

Il est aussi évident à partir du graphe en forme de toile d'araignée
que le point d'équilibre $p = 40$ est (asymptotiquement) stable car
les orbites convergent tous vers $p$.
}

\compileSOL{\SOLUb}{\ref{6Q92}}{
Nous avons $x_0 = 1$, $\displaystyle x_1 = f(x_0) = \frac{1}{2}$, 
$\displaystyle x_2 = f(x_1) = \frac{1}{3}$,
$\displaystyle x_3 = f(x_2) = \frac{1}{4}$, \ldots\
Nous pouvons conjecturer que la solution est
$\displaystyle x_i = \frac{1}{i+1}$.  Pour démontrer que notre
conjecture est vrai, il faut vérifier que l'équation
$\displaystyle x_{i+1} = \frac{x_i}{1+x_i}$ est satisfaite avec
$\displaystyle x_i = \frac{1}{i+1}$.  Effectivement,
\[
\frac{x_i}{1+x_i} = \frac{1/(i+1)}{1 + 1/(i+1)} =
\frac{1/(i+1)}{(i+2)/(i+1)} = \frac{1}{i+2} = x_{i+1} \ .
\]
}

\compileSOL{\SOLUb}{\ref{6Q93}}{
Les points d'équilibre $p$ sont les solutions de l'équation
$y = f(y) = e^{-y}$.  Malheureusement, nous ne pouvons pas résoudre
cette équation pour $y$.  Par contre, nous pouvons déduire de
l'intersection de la droite $y=x$ avec la courbe $y=e^{-x}$ qu'il
existe un point d'équilibre.
\PDFgraph{6_derivees_appl/graph_funct20}
Nous pourrions utiliser la méthode de Newton pour estimer numériquement
la valeur de $p$.
}

\compileSOL{\SOLUa}{\ref{6Q94}}{
Le graphe de la fonction itérative est donné ci-dessous.
\PDFgraph{6_derivees_appl/linear9}

Le point d'équilibre $p$ est donné par l'intersection du graphe de la
fonction itérative $z = f(y)$ avec la droite $z=y$.  Le point
d'équilibre $p$ que nous cherchons est donc la solution de l'équation 
$p=f(p)=p^2-1$ avec $0 \leq p \leq 2$.  C'est-à-dire qu'il faut trouver les
racines du polynôme $p^2- p -1=0$ avec $0 \leq p \leq 2$.  Grâce à la formule
pour trouver les racines d'un polynôme de degré deux, nous obtenons
\[
p_\pm = \frac{1 \pm \sqrt{1+4}}{2} = \frac{1 \pm \sqrt{5}}{2} \ .
\]
La seule racine $p$ telle que $0 \leq p \leq 2$ est
$\displaystyle p = \frac{1 + \sqrt{5}}{2}$.
}

\compileSOL{\SOLUa}{\ref{6Q95}}{
Soit $f(x)= x-\cos(x)$.  $f$ est une fonction continue sur la droite réelle
telle que $f(0) = 0 -\cos(0) = -1 <0$ et 
$f(\pi/2) = \pi/2 - \cos(\pi/2) = \pi/2 > 0$.  Selon le théorème des
valeurs intermédiaires, il existe donc $p$ entre $0$ et $\pi/2$ tel que
$f(p)= p - \cos(p)= 0$; c'est-à-dire, $\cos(p) = p$.  La figure suivante
illustre ce fait.  Nous pouvons montrer à l'aide d'une méthode numérique
(e.g. méthode de Newton) que $p \approx 0.73908513321$.
\MATHgraph{6_derivees_appl/nonlinear1}{8cm}
}

\compileSOL{\SOLUb}{\ref{6Q96}}{
La fonction itérative est $\displaystyle f(x) = \frac{x}{x-1}$ pour
$x>1$.  Les point d'équilibre positifs du système dynamique discret
sont les solutions de l'équation $\displaystyle p = f(p) = \frac{p}{p-1}$.
Nous pouvons récrire cette équation de la façon suivante.
\[
p^2 - 2 p = p(p-2) = 0 \ .
\]
La seule solution pour $p>1$ est $p=2$.  C'est le point d'intersection
du graphe de la fonction itérative $f$ avec la droite $y=x$ qui
représenté dans la figure ci-dessous.
\PDFgraph{6_derivees_appl/nonlinear2}
}

\compileSOL{\SOLUa}{\ref{6Q97}}{
La fonction itérative est $\displaystyle f(x) = \frac{x}{x +1}$.  Le
graphe de cette fonction est en noir dans la figure ci-dessous.  La
figure ci-dessous contient aussi le graphe en forme de toile
d'araignée.
\MATHgraph{6_derivees_appl/nonlinear3}{8cm}
}

\compileSOL{\SOLUb}{\ref{6Q98}}{
Le point d'équilibre est $p$ car $f(p) = p$.
\PDFgraph{6_derivees_appl/nonlinear4_b}
}

\compileSOL{\SOLUa}{\ref{6Q99}}{
$p$ est un point d'équilibre si
$\displaystyle p = f(p) = \frac{\alpha p}{p+1}$.  Or,
\[
p = \frac{\alpha p}{p+1} \Leftrightarrow 
p(p+1) = \alpha p \Leftrightarrow  p^2 + (1-\alpha) p = 0
\Leftrightarrow p ( p+1-\alpha) = 0 \ .
\]
Les points d'équilibre sont donc $p=0$ et $p = \alpha -1$.  le
système dynamique a toujours au moins un point d'équilibre; c'est le
point $p=0$.

\subQ{a} Il n'y pas de point d'équilibre autre que $0$ quand
$\alpha = 1$.

\subQ{b} le système dynamique discret a un point d'équilibre négatif
lorsque $\alpha < 1$; c'est le point $p=\alpha - 1$.

\subQ{c} le système dynamique discret a un point d'équilibre positif
lorsque $\alpha > 1$; c'est le point $p=\alpha - 1$.
}

\compileSOL{\SOLUb}{\ref{6Q100}}{
\subQ{a} Le graphe en forme de toile d'araignée lorsque la fonction
itérative $f$ est dans sa position initiale.
\PDFgraph{6_derivees_appl/nonlinear6_a}

Il y a deux points d'équilibre: $0$ et $p$.  L'origine est stable.
Par contre, $p$ n'est pas stable.  Il est vrai que les orbites
convergent vers $p$ si la condition initiale $x_0$ est plus grande que
$p$.  Cependant, si la condition initiale $x_0$ satisfait
$0 < x_0 < p$, alors les orbites convergent vers l'origine et non pas
vers $p$.

\subQ{b} Le graphe en forme de toile d'araignée lorsque le graphe de
$f$ est légèrement pivoté dans le sens des aiguilles d'une montre.
\PDFgraph{6_derivees_appl/nonlinear6_b}

Il n'y a maintenant qu'un seul point d'équilibre, l'origine, qui est
stable.  Toutes les orbites convergent vers l'origine.

\subQ{c} Le graphe en forme de toile d'araignée lorsque le graphe de
$f$ est légèrement pivoté dans le sens contraire aux aiguilles d'une
montre.
\PDFgraph{6_derivees_appl/nonlinear6_c}

Il y a maintenant trois points d'équilibre: $0$, $p_1$ et $p_2$.
L'origine est toujours stable.  Le point $p_1$ est un point d'équilibre
instable alors que le point $p_2$ est un point d'équilibre stable.
}

\compileSOL{\SOLUa}{\ref{6Q101}}{
\PDFgraph{6_derivees_appl/nonlinear7_b}
Le point d'intersection (autre que l'origine) de la courbe $y=f(x)$ avec la
droite $y=x$ est $(700,700)$.  Donc $p=700$ est le point d'équilibre du
système dynamique discret $x_{n+1} = f(x_n)$ pour $n=0$, $1$, $2$, \ldots 

La pente de la droite tangente à la courbe $y=f(x)$ au point $(700,700)$ est
approximativement $m = 3/16$.  Puisque $|m|<1$, nous obtenons du
théorème sur la stabilité des points d'équilibre que le point
d'équilibre $p=700$ est (asymptotiquement) stable .
}

\compileSOL{\SOLUb}{\ref{6Q102}}{
\subQ{a}
\PDFgraph{6_derivees_appl/nonlinear8_a}
Puisque la courbe $y=f(x)$ est tangente à la droite $y=x$ au point
$(p,p)$ par construction, nous avons $f'(p) = 1$.  Nous ne pouvons rien conclure
à l'aide du théorème de stabilité pour les points d'équilibre.  Par
contre, le graphe en forme de toile d'araignée indique que le point
d'équilibre $p$ est stable.

\subQ{b}
\PDFgraph{6_derivees_appl/nonlinear8_b}
Comme précédemment, puisque la courbe $y=f(x)$ est tangente à la
droite $y=x$ au point $(p,p)$ par construction, nous avons $f'(p) = 1$.  Nous
ne pouvons rien conclure à l'aide du théorème de stabilité pour les
points d'équilibre.  Par contre, le graphe en forme de toile
d'araignée indique que le point d'équilibre $p$ est instable.
}

\compileSOL{\SOLUb}{\ref{6Q103}}{
Le système dynamique discret est de la forme $p_{n+1} = f(p_n)$ où
$f(x)=6xe^{-2x}-2x$.  Les points d'équilibres sont donc les points
$p$ qui satisfont $\displaystyle p = f(p) = 6pe^{-2p} - 2p$.
Le point $p=0$ est un point d'équilibre.  Si $p\neq 0$, nous pouvons
diviser des deux côtés de l'équation par $p$ pour obtenir 
$\displaystyle 1 = 6e^{-2p}-2$.  Donc $e^{-2p} = 0.5$.  En appliquant
$\ln$ des deux côtés de cette dernière égalité, nous obtenons un
deuxième point d'équilibre, $p = \ln(2)/2$.

Pour déterminer la stabilité de ces deux points d'équilibre, nous avons
besoin de la dérivée de $f$; c'est-à-dire,
$f'(x) = 6e^{-2x} - 12 xe^{-2x} -2 = 6(1-2x)e^{-2x} - 2$.

Puisque $|f'(0)| = 4 > 1$, le point $p=0$ est instable.  De même,
$|f'(\ln(2)/2)| = |6(1 - \ln(2))/2 -2| = 1.07944\ldots > 1$ et le point
$p=\ln(2)/2$ est aussi instable.

\noindent Note: le comportement de ce système dynamique discret est
beaucoup plus complexe que ceux que nous avons normalement dans nos
exemples.  Voir la section sur les orbites périodiques.
}

\compileSOL{\SOLUa}{\ref{6Q104}}{
La fonction itérative est
$\displaystyle f(x) = \frac{1.5x}{1.5x + 2(1-x)} = \frac{1.5x}{2-0.5x}$.
Nous avons $\displaystyle f'(x) = \frac{3}{(2-0.5x)^2}$.

Notons que $p=0$ et $p=1$ sont des points d'équilibre car $f(0)=0$ et
$f(1)=1$.  Puisque $|f'(0)| = 3/4 < 1$, nous avons que $0$ est un point
d'équilibre asymptotiquement stable.  Puisque $|f'(1)| = 4/3 >1$, nous avons
que $1$ est un point d'équilibre instable.
}

\compileSOL{\SOLUb}{\ref{6Q105}}{
La fonction itérative est $f(x) = 2x(1-x)$.  Les points d'équilibre sont
les solutions de $f(x)=x$, soit $x=0$ ou $x=1/2$.  Le point d'équilibre
non nul est $p=1/2$.  Puisque $f'(x)= 2-4x$, nous obtenons que
$f'(p) = 0$.  Le point d'équilibre $p=1/2$ est donc asymptotiquement
stable car $|f'(p)| = 0 <1$.

Le graphe en forme de toile d'araignée pour $x_0 = 0.2$ est ci-dessous.
\MATHgraph{6_derivees_appl/logistic1}{8cm}

Comme nous pouvons le voir dans le graphe en forme de toile d'araignée, la
convergence est très rapide.  À partir de $0.2$, il faut seulement quatre
itérations pour être très près du point d'équilibre $p=1/2$.
}

\compileSOL{\SOLUa}{\ref{6Q106}}{
\subQ{a} Nous avons le système dynamique discret
\[
M_{i+1} = M_i - \frac{M_i^2}{2+M_i} + 1
= \frac{2M_i}{2+M_i} + 1  \quad \text{pour} \quad i =0, 1, 2, 3, \ldots ,
\]
La fonction itérative est $\displaystyle g(M) = \frac{2M}{2+M} + 1$.
Notons que $G(2)=2$.  Ce qui confirme que $M=2$ est un point d'équilibre.

Nous déduisons de $\displaystyle g'(M) = \frac{4}{(2+M)^2}$ que
$g'(2) = 1/4$.  Puisque que $|g'(2)|<1$, le point d'équilibre $M=2$ est
asymptotiquement stable.  De plus, puisque $g'(2)>0$, les orbites n'oscillent
pas autour de $M=2$.
}

\compileSOL{\SOLUb}{\ref{6Q107}}{
\subQ{a} La fonction génératrice du système dynamique discret est
$f(x) = r x (1-x) - 0.75 x$.  Les points d'équilibre sont les
solutions de l'équation $x = f(x) = r x (1-x) - 0.75 x$.  Il est
facile de voir que $x=0$ est une solution.  Donc, $p_1 = 0$ est un point
d'équilibre.  Si $x\neq 0$, nous pouvons diviser les deux
côtés de l'égalité par $x$ pour obtenir $1 = r(1-x) - 0.75$.  Si nous
résolvons pour $x$ cette dernière équation, nous obtenons le deuxième point
d'équilibre $\displaystyle p_2  = 1 - \frac{1.75}{r}$.

\subQ{b} Le point $p_1=0$ a toujours un sens biologique; il n'y a pas
d'individus.  Le point d'équilibre
$\displaystyle p_2 = 1- \frac{1.75}{r}$ a un sens
biologique seulement s'il est plus grand ou égale à $0$.  Ainsi
\[
  1 - \frac{1.75}{r} \geq 0 \Rightarrow r \geq 1.75 \ .
\]

\subQ{c} La dérivée de la fonction génératrice est
$f'(x) = r - 2r x -0.75$. Le point d'équilibre $p_2=1- 1.75/r$ sera
stable si
\[
\left| f'(p_2) \right| = \left| r -2r \left(1 - \frac{1.75}{r}\right)
  - 0.75 \right| = | -r + 2.75 | = |r-2.75|<1 \ ;
\]
c'est-à-dire, si $-1 < r - 2.75 < 1$ ou, plus précisément, si
$1.75 < r < 3.75$.  Le point d'équilibre $p_2$ sera instable si
$\left| f'(p_2) \right| > 1$; c'est-à-dire si $r > 3.75$ car
nous considérons seulement $r > 1.75$ pour obtenir un point d'équilibre
$p_2$ qui a un sens biologique.

Le point d'équilibre $p_1=0$ sera stable si
$\left| f'(p_1) \right| = |r - 0.75|<1$; c'est-à-dire si
$-1 < r-0.75 < 1$ ou, plus précisément, si $-0.25 < r < 1.75$.
Le point d'équilibre $p_1$ sera instable si
$\left| f'(p_1) \right| > 1$; c'est-à-dire, si $r> 1.75$ ou $r<-0.25$.
}

\compileSOL{\SOLUa}{\ref{6Q108}}{
\subQ{a} La fonction itérative est $f(x) = 2 x(1-x^2)$.

\subQ{b} Les points d'équilibre sont les solutions de $x=f(x)$.
Il découle de $2 x(1-x^2) = x$ que $x=0$ ou $2 (1-x^2) = 1$ si $x\neq 0$.
De plus,
\[
2 (1-x^2) = 1 \Leftrightarrow x^2 = \frac{1}{2} \Leftrightarrow 
x = \pm \frac{1}{\sqrt{2}} \ .
\]
Les trois points d'équilibre sont donc $0$, $1/\sqrt{2}$
et $-1/\sqrt{2}$.

\subQ{c} Nous avons $f'(x) = 2 - 6 x^2$.  Ainsi,
$f'(-1/\sqrt{2})= -1$, $f'(0)=2$ et $f'(1/\sqrt{2}) = -1$.
Puisque $|f'(0)|=2>0$, le point d'équilibre $x=0$ est instable.

Puisque $|f'(-1/\sqrt{2})| = |f'(1/\sqrt{2})| = 1$, nous ne pouvons
rien conclure à partir du théorème de stabilité des points
d'équilibre.  Il faut faire une analyse plus poussée pour déterminer
la stabilité de ces points d'équilibre.

Nous retrouvons ci-dessous deux graphe en forme de toile d'araignée.  Dans le
deuxième graphe, nous avons tracer le graphe au environ du point d'équilibre
$1/\sqrt{2}$.  Nous voyons que les points d'équilibre $-1/\sqrt{2}$ et
$1/\sqrt{2}$ sont stables mais la stabilité est très faible.  Il faut
un très grand nombre d'itérations pour approcher légèrement le point
d'équilibre.
\MATHgraph{6_derivees_appl/logisticlike1_a}{8cm}
\MATHgraph{6_derivees_appl/logisticlike1_b}{8cm}
}

\compileSOL{\SOLUb}{\ref{6Q109}}{
\subQ{a} La fonction itérative du système dynamique discret est
$f(x) = \mu x(1-x^2)$.  Ainsi, $\displaystyle f'(x) = \mu (1-3x^2)$ et
$f'(0)= \mu$.  Le point d'équilibre $x=0$ est asymptotiquement stable
si $|f'(0)|= |\mu|<1$.  Si $|\mu|>1$, le point d'équilibre $x=0$ est
instable. Si $\mu=1$ ou $\mu=-1$, il faut utiliser un graphe en forme
de toile d'araignée pour déterminer la stabilité du point d'équilibre
$x=0$.

Pour $\mu=1$, nous obtenons le graphe en forme de toile d'araignée suivant.
\MATHgraph{6_derivees_appl/logisticlike2_a}{8cm}

Pour $\mu=-1$, nous obtenons le graphe en forme de toile d'araignée suivant.
\MATHgraph{6_derivees_appl/logisticlike2_b}{8cm}

Dans les deux cas, l'origine est \lgm faiblement\rgm\ stable.  Dans le cas
$\mu=-1$, les orbites oscillent autour de l'origine.

\subQ{b} Les points d'équilibre non nuls sont les solutions non nuls de
l'équation $x = f(x) = \mu x(1-x^2)$.  Si $x\neq 0$, nous pouvons
diviser pas $x$ des deux côtés de l'égalité pour obtenir
$1 = \mu (1-x^2)$.   La solution pour $x$ de cette équation est
$\displaystyle x = \pm \sqrt{ 1 - 1/\mu}$ pour $\mu\geq 1$.

Le point d'équilibre positif est donc
$\displaystyle p = \sqrt{ 1 - 1/\mu}$ pour $\mu > 1$.

Nous avons toujours $\displaystyle f'(x) = \mu (1-3x^2)$.  Ainsi,
\[
f'(p) = \mu \left(1 - 3 p^2\right)
= \mu \left( 1 - 3\left(1 -\frac{1}{\mu}\right)\right) 
= -2\mu + 3 \ .
\]
Le point d'équilibre $x=p$ est asymptotiquement stable si
$|f'(p)|= |3-2\mu|<1$; c'est-à-dire, si
\[
|3-2\mu|<1 \Leftrightarrow -1 < 3-2\mu < 1 \Leftrightarrow
1 > 2\mu -3 > -1 \Leftrightarrow 4 > 2\mu > 2 \Leftrightarrow
2>\mu > 1
\]
Le point d'équilibre $x=p$ est instable si
$|f'(p)|= |3-2\mu|>1$; c'est-à-dire, si
\begin{align*}
|3-2\mu|>1 &\Leftrightarrow  3-2\mu < -1 \text{ ou } 3-2\mu>1
\Leftrightarrow 2\mu -3 > 1 \text{ ou } 2\mu -3 < -1 \\
&\Leftrightarrow 2\mu > 4 \text{ ou } 2\mu  < 2 \Leftrightarrow
 \mu > 2 \text{ ou } \mu < 1
\end{align*}
Comme $p$ est défini seulement pour $\mu>1$, nous conservons seulement
$\mu > 2$.  Si $\mu=2$, il faut utiliser un graphe en forme de toile
d'araignée pour déterminer la stabilité du point d'équilibre $x=p$.
C'est ce qui a été fait à la question~\ref{6Q108}.

Nous tirons la même conclusion pour le point d'équilibre négatif
$\displaystyle p = -\sqrt{ 1 - 1/\mu}$ pour $\mu > 1$.
}

\compileSOL{\SOLUb}{\ref{6Q110}}{
\subQ{a} Nous avons que
\[
x_{n+1} = x_n t(x_n) = \frac{2 x_n^2}{1+x_n^2} \quad \text{pour}
\quad n=0, 1, 2, 3, \ldots 
\]
C'est le taux de reproduction par individus multiplié par le nombre
d'individus par cm$^2$.

\subQ{b}  La fonction itérative est
$\displaystyle f(x) = \frac{2 x^2}{1+x^2}$.  Les points d'équilibre sont les
solutions de $f(x)=x$.  Donc
\[
\frac{2 x^2}{1+x^2} = x \Leftrightarrow 2x^2 = x+ x^3 \Leftrightarrow
x^3 -2x^2 + x = x(x-1)^2 = 0 \ .
\]
Il y a deux points d'équilibre $p=0$ et $p=1$.

\subQ{c} Le graphe de la fonction itérative (en noire) et le graphe en
forme de toile d'araignée sont donnés dans la figure ci-dessous.
\MATHgraph{6_derivees_appl/nonlinear9}{8cm}

\subQ{d} Puisque $\displaystyle f'(x) = \frac{4x}{(1+x^2)^2}$, nous avons que
$f'(0)= 0$ et $f'(1) = 1$.  Selon le théorème de stabilité des points
d'équilibre, le point d'équilibre $p=0$ est
asymptotiquement stable car $|f'(0)|=0 <1$.  Par contre, nous ne pouvons
rien conclure pour $p=1$.

\subQ{e}
Si initialement le nombre $x_0$ d'individus par cm$^2$ est supérieur à
$1$, alors le nombre $x_n$ d'individus par cm$^2$ après $n$ heures
converge vers $1$ lorsque $n \to \infty$.  Par contre, si initialement
le nombre $x_0$ d'individus par cm$^2$ est supérieur à $0$ mais
inférieure à $1$, alors le nombre $x_n$ d'individus par cm$^2$
converge vers $0$ lorsque $n \to \infty$; la population va
disparaître.
}

\compileSOL{\SOLUb}{\ref{6Q112}}{
\subQ{a} La fonction génératrice est
$\displaystyle f(x) = \frac{6 x^2}{7+x^2}$ car $x_{n+1} = f(x_n)$.

\subQ{b} Les points d'équilibre sont les solutions de l'équation
$\displaystyle f(x) = \frac{6 x^2}{7+x^2} = x$.  Nous avons
\[
\frac{6 x^2}{7+x^2} = x \Leftrightarrow
6 x^2 = 7x +x^3 \Leftrightarrow x^3 - 6 x^2 + 7x
= x(x^2 - 6x + 7) = 0 \ .
\]
Les racines du polynôme $x^2- 6x + 7$ sont
$\displaystyle x_{\pm}
= \big(6 \pm \sqrt{(-6)^2 - 4 \times 7}\,\big)/2 = 3 \pm \sqrt{2}$.
Les trois points d'équilibre sont donc
$x_1^\ast = 0$, $x_2^\ast = 3-\sqrt{2} \approx  1.5858$ et
$x_3^\ast = 3+\sqrt{2} \approx 4.4142$.

\subQ{c} Nous avons $\displaystyle f'(x) = \frac{84x}{(7+x^2)^2}$.
Puisque $|f'(x_1^\ast)| = |f'(0)| = 0 < 1$, le
point d'équilibre $x_1^\ast$ est (asymptotiquement) stable.  Puisque
$|f'(x_2^\ast)| = |f'(3-\sqrt{2})| \approx 1.4714 > 1$, le point
d'équilibre $x_2^\ast$ est instable.  Finalement, puisque
$|f'(x_3^\ast)| = |f'(3+\sqrt{2})| =  0.5286 < 1$, le
point d'équilibre $x_3^\ast$ est (asymptotiquement) stable.

\subQ{d} Si $x_0 = 1$, alors $x_1 = f(x_0) = 3/4$,
$x_2 = f(x_1) \approx 0.4463$ et $x_3 = f(x_2) \approx  0.1660$.

Nous obtenons le graphe en forme de toile d'araignée suivant.
\MATHgraph{6_derivees_appl/nonlinear11}{8cm}

\subQ{e} Si $x_0=1$, alors $x_n \to 0$ lorsque $n\to \infty$.
}

\compileSOL{\SOLUb}{\ref{6Q114}}{
\subQ{a} Si les plantes ont produites $N_0 = 20$ graines dans l'année
$0$, alors la taille moyenne des plantes l'année suivante (l'année
$1$) sera $\displaystyle S_1= \frac{100}{N_0} = \frac{100}{20} = 5$.
Ces plantes produiront donc en moyenne $N_1=S_1-1 = 5-1 = 4$ nouvelles
graines pour le première année (l'année $1$).  

la taille moyenne des plantes pour l'année $2$ sera
$\displaystyle S_2= \frac{100}{N_1} = \frac{100}{4} = 25$.
Ces plantes produiront donc en moyenne $N_2=S_2-1 = 25-1 = 24$
nouvelles graines pour l'année $2$.

la taille moyenne des plantes pour l'année $3$ sera
$\displaystyle S_3= \frac{100}{N_2} = \frac{100}{24} = 4.1\overline{6}$.
Ces plantes produiront donc en moyenne
$N_3=S_3-1 = 4.1\overline{6}-1 = 3.1\overline{6}-1$
nouvelles graines pour l'année $3$.

\subQ{b} Nous pouvons déduire le système dynamique discret suivant à
partir du raisonnement en (a) pour obtenir $N_1$, $N_2$ et $N_3$ à
partir de $N_0$.
\[
N_{n+1} = \frac{100}{N_n} - 1 \quad \text{pour} \quad n =0, 1, 2, 3, \ldots 
\]

\subQ{c} La fonction itérative de notre système dynamique discret est
$\displaystyle f(N) = \frac{100}{N} - 1$.  Les points d'équilibre sont les
solutions de $f(N) = N$.
\[
\frac{100}{N} - 1 = N \Leftrightarrow 100 - N = N^2 
\Leftrightarrow N^2+N-100 = 0 \ .
\]
Les racines du polynôme $N^2+N-100$ sont
$\displaystyle N_1 = \frac{1}{2}\left(-1 + \sqrt{401}\right) \approx 9.5124922$
et $\displaystyle N_2 = \frac{1}{2}\left(-1 - \sqrt{401}\right)$.  Puisque
$N_2<0$, nous pouvons l'ignorer car nous ne pouvons pas avoir un
nombre négatif de graines.

\subQ{d} Nous avons $\displaystyle f'(N) = -\frac{100}{N^2}$, donc
$\displaystyle f'(N_1) = - \frac{100}{N_1^2} \approx -1.1051249$.  Puisque
$|f'(N_1)|>1$, le point d'équilibre $N_1$ est instable.

\subQ{e} Le graphe de la fonction itérative ainsi qu'un graphe en forme de
toile d'araignée sont donnés dans la figure ci-dessous.  Le graphe en forme
de toile d'araignée montre que le point d'équilibre $N_1$ est instable.
De plus, les orbites oscillent autour de ce point d'équilibre.  Ce qui n'est
pas surprenant car $f'(N_1)<0$.
\MATHgraph{6_derivees_appl/nonlinear10}{8cm}
}

\compileSOL{\SOLUb}{\ref{6Q115}}{
\subQ{a} La fonction itérative du système dynamique discret est
$f(x) = 2 x (1-x) - h x$.  Les points d'équilibre sont les solutions de
$f(x)=x$.  L'origine $x=0$ est une point d'équilibre.  Si $x\neq 0$,
nous pouvons alors diviser des deux côtés de l'égalité $2 x (1-x) - h x=x$
par $x$ pour obtenir $2(1-x) - h=1$.  La solution pour $x$ de cette
dernière équation donne le point d'équilibre $x = p(h) = (1-h)/2$.

Pour que le point d'équilibre $p(h) = (1-h)/2$ soit non négatif, il faut avoir
$h \leq 1$.  Naturellement, le facteur d'efficacité $h$ ne peut pas être
négatif.

\subQ{b} La récolte à long terme est
\[
R(h) = h\,p(h) = h\left(\frac{1-h}{2}\right) =
\frac{1}{2} \left( h-h^2\right)
\]
pour $0\leq h \leq 1$.

\subQ{c} Comme $R$ est une fonction continue définie sur l'intervalle fermé
$[0,1]$, nous pouvons utiliser le Théorème des valeurs extrêmes pour trouver le
maximum absolu de $R$ sur l'intervalle $[0,1]$.

Puisque $\displaystyle R'(h) = \frac{1}{2} \left( 1 - 2h\right)$, il n'y a
qu'un seul point critique et il est donné par la solution de $R'(h)=0$.  Ce
point critique est $h = 1/2$.

Puisque $R(0)=R(1)= 0$ et $R(1/2) = 1/8$, le maximum de $R$ sur
l'intervalle $[0,1]$ est donc $1/8$ lorsque $h=1/2$.

\subQ{d} La récolte est $R(1/2)=1/8$.
}

\compileSOL{\SOLUa}{\ref{6Q116}}{
\subQ{a} La fonction itérative du système dynamique discret est
$f(x) = 2.5 x (1-x) - h x$.  Les points d'équilibre sont les solutions de
$f(x)=x$.  l'origine $x=0$ est une point d'équilibre.  Si $x\neq 0$,
nous pouvons alors diviser les deux côtés de l'égalité $2.5 x (1-x) - h x=x$
par $x$ pour obtenir $2.5(1-x) - h=1$.  La solution de cette équation 
nous donne le point d'équilibre $x = p(h) = (3-2h)/5$.

Pour que le point d'équilibre $p(h)=(3-2h)/5$ soit non négatif, il faut avoir
$h \leq 3/2$.  Naturellement, le facteur d'efficacité $h$ ne peut pas être
négatif.

\subQ{b} La récolte à long terme est
\[
R(h) = h\, p(h) = h\left(\frac{3-2h}{5}\right) =
\frac{1}{5} \left( 3h-2h^2\right)
\]
pour $0\leq h \leq 3/2$.

\subQ{c} Comme $R$ est une fonction continue définie sur l'intervalle fermé
$[0,3/2]$, nous pouvons utiliser le Théorème des valeurs extrêmes pour
trouver le maximum absolu de $R$ sur l'intervalle $[0,3/2]$.

Puisque $\displaystyle R'(h) = \left( 3 - 4h\right)/5 = 0$ seulement
pour $h=3/4$, le seul point critique est $h=3/4$.   Puisque
$R(0)=R(3/2)= 0$ et $R(3/4) = 9/40$, le maximum de $R$ sur
l'intervalle $[0,3/2]$ est $9/40$ lorsque $h=3/4$.

\subQ{d} Pour $h= 3/4 = 0.75$, la fonction itérative est
\[
f(x) = 2.5 x (1-x) - h x = 2.5 x (1-x) - 0.75x = 1.75 x - 2.5x^2
\]
et le point d'équilibre positif du système dynamique discret
$N_{i+1} = f(N_i) = 1.75 N_1 -2.5 N_i^2$ est
$\displaystyle p(3/4) = \left(3- 2(3/4)\right)/5 = 3/10 = 0.3$.  

La dérivée de la fonction itérative est $f'(x) = 1.75 - 5 x$.  Ainsi,
\[
| f'(0,3) | = |1.75 - 5 \times 0.3| = 0.25 < 1
\]
et le point d'équilibre $0.3$ est (asymptotiquement) stable.

\subQ{e} Le graphe en forme de toile d'araignée pour le système dynamique
$N_{n+1} = f(N_n) = 2.5N_n(1-N_n) - 0.75 N_n$ est donné ci-dessous.
\MATHgraph{6_derivees_appl/harvest1}{8cm}
}

\compileSOL{\SOLUb}{\ref{6Q118}}{
\subQ{a} Le système dynamique discret est $p_{n+1} = f(p_n)$
où $f(x) = 1.5 x(1-x^2) - h x$.  Les points d'équilibre sont
les solutions $p$ de
\[
p= f(p) = 1.5\,p(1-p^2) - h\;p \ .
\]
$p=0$ est une solution et donc un point d'équilibre.  Si $p\neq 0$,
nous pouvons diviser des deux côtés de l'égalité par $p$ pour obtenir
$1=1.5(1-p^2) -h$.  D'où $p^2 = (1-2h)/3$.  Si $h < 1/2$, la solution
positive de cette dernière équation nous donne
le point d'équilibre $\displaystyle p = \sqrt{\frac{1-2h}{3}}$.

\subQ{b} Le point d'équilibre $p=0$ va donner une récolte de $0$
insecte.  Ce n'est certainement pas le maximum.  Nous devons trouver $h$
qui maximise $\displaystyle R(h) = h\; p = h\sqrt{\frac{1-2h}{3}}$
pour $0\leq p \leq 1/2$.  Pour que la récolte soit positive, il faut
que $p$ soit plus grand que $0$.

Les points critiques de $R(h)$ sont les valeurs de $h$ telles que
$R'(h)=0$.  Or
\[
R'(h) = \frac{1}{\sqrt{3}} \left( \sqrt{1-2h} 
- \frac{h}{\sqrt{1-2h}}\right)
= \frac{1}{\sqrt{3}}\left( \frac{1-3h}{\sqrt{1-2h}} \right)  \ .
\]
Donc $R'(h)=0$ pour $h=1/3$.  Le tableau suivant confirme que nous avons
bien un maximum absolu positif pour $0 \leq h < 1/2$.
\[
\begin{array}{c|c|c|c|c|c}
h & 0 & 0<h< 1/3 & 1/3 & 1/3<h<1/2 & 1/2 \\
\hline
R(h) & 0 & + & 1/9 & + & 0 \\
\hline
R'(h) & + & + & 0 & - &
\end{array}
\]
Avec un facteur d'efficacité de $h=1/3$, les grenouilles obtiennent la
récolte maximale à long terme de $1/9$ de la population
maximale d'insectes que le lac peut supporter.

\subQ{c} Il reste à démontrer que le point d'équilibre $p$ avec $h=1/3$
est stable; c'est-à-dire, que $p = 1/3$ est stable.

Pour $h=1/3$, nous avons la fonction itérative
$\displaystyle f(x) = 1.5 x(1-x^2) - x/3$.  Donc
$\displaystyle f'(x) = 7/6 - 9x^2/2$.
Puisque $|f'(p)| = 2/3 < 1$, le point d'équilibre
$\displaystyle p = 1/3$ est stable.  À long
terme, notre récolte maximale sera donc bien $1/9$ de la population.
}

\compileSOL{\SOLUa}{\ref{6Q119}}{
\subQ{a} Le système dynamique discret est
$x_{n+1} = f(x_n)$ où $f(x) = 2 x(1-x^3) - h x$ est la fonction
génératrice.  Les points d'équilibre sont les solutions $p$ de
$p=f(p) = 2\,p(1-p^3) - h\,p$.  Un premier point d'équilibre est $p=0$.
Si $p\neq 0$, nous pouvons diviser les deux côtés de l'égalité par $p$ pour
obtenir $1 = 2(1-p^3) -h$.  D'où $p^3 = (1-h)/2$.  Si $h<1$, la
solution positive de cette dernière équation est le point d'équilibre
$\displaystyle p = \sqrt[3]{(1-h)/2}$.

Notons que nous avons un point d'équilibre non négatif seulement si
$h\leq 1$.  De plus, $h$ ne peut pas être négatif car nous ne pouvons pas
avoir un facteur d'efficacité négatif.

\subQ{b} Le point d'équilibre $p=0$ va donner une récolte de $0$
grenouille et ce n'est certainement pas le maximum.  Il faut trouver $h$
qui maximise
$\displaystyle R(h) = h\, p(h) = \frac{h(1-h)^{1/3}}{\sqrt[3]{2}}$ pour
$0\leq h \leq 1$.

Les points critiques de $R(h)$ sont les valeurs de $h$ telles que
$R'(h)=0$.  Or
\[
R'(h) = \frac{1}{\sqrt[3]{2}} \left( (1-h)^{1/3} - \frac{h}{3(1-h)^{2/3}}
\right) = \frac{1}{\sqrt[3]{2}}\left( \frac{3(1-h) -h}{3(1-h)^{2/3}}\right)
= \frac{3-4h}{3(2(1-h)^2)^{1/3}} \ .
\]
Donc $R'(h)=0$ pour $h=3/4$.  Le tableau suivant confirme que nous avons
bien un maximum absolu positif lorsque $h=3/4$.
\[
\begin{array}{c|c|c|c|c|c}
h & 0 & 0<h<3/4 & 3/4 & 3/4<h<1 & 1 \\
\hline
R(h) & 0 & + & 3/8 & + & 0 \\
\hline
R'(h) & + & + & 0 & - & -\infty
\end{array}
\]
Avec un facteur d'efficacité de $h=3/4$, nous obtenons le point
d'équilibre $\displaystyle p(3/4) = 1/2$ pour le système
dynamique discret
\[
x_{n+1} = 2 x_n (1 - x_n^3) - \frac{3x_n}{4} \quad \text{pour} \quad n=0, 1,
2, \ldots
\]
La fonction génératrice de ce système est
$\displaystyle f(x) = 2x(1-x^3)- 3x/4$.

\subQ{c} Pour déterminer la stabilité du point d'équilibre, nous évaluons
$|f'(x)|$ au point $p = 1/2$.  Puisque
$\displaystyle f'(x) = 2 - 8 x^3 - 3/4 = 5/4- 8 x^3$,
nous obtenons
\[
\left| f\left( \frac{1}{2} \right) \right| 
= \left| \frac{5}{4} - 1 \right| = \frac{1}{4} < 1 \ .
\]
Le point d'équilibre $p=1/2$ est stable et il sera possible d'obtenir
une récolte optimale à long termes.
}

\compileSOL{\SOLUb}{\ref{6Q120}}{
\subQ{a} La fonction itérative du système dynamique discret est
$\displaystyle f(x) = \frac{2.5 x}{1+x} - h x$.  Les points d'équilibre sont
les solutions de
\[
  f(x)=\displaystyle \frac{2.5 x}{1+x} - h x=x \ .
\]
L'origine $x=0$ est une point d'équilibre.  Si $x\neq 0$, nous pouvons
alors diviser les deux côtés de l'égalité précédente pour obtenir
$\displaystyle \frac{2.5}{1+x} - h=1$.  La solution
$\displaystyle x = p(h) = \frac{3-2h}{2+2h}$ de cette dernière
équation est le deuxième point d'équilibre.

Pour que le point d'équilibre $\displaystyle p(h) = \frac{3-2h}{2+2h}$ soit
non négatif, il faut avoir $h \leq 3/2$.  Naturellement, le facteur
d'efficacité $h$ ne peut pas être négatif.   En fait, pour que le
problème ait un sens biologique, il faut aussi que $h\leq 1$ car nous ne
pouvons pas récolter plus de 100\% de la population.

\subQ{b} La récolte à long terme est
\[
R(h) = h\, p(h) = h\left(\frac{3-2h}{2+2h}\right) =
\frac{3h-2h^2}{2+2h}
\]
pour $0\leq h \leq 3/2$.

\subQ{c} Comme $R$ est une fonction continue définie sur l'intervalle fermé
$[0,3/2]$, nous pouvons utiliser le Théorème des valeurs extrêmes pour
trouver le maximum absolu de $R$ sur l'intervalle $[0,3/2]$.

Puisque $\displaystyle R'(h) = \frac{-2h^2-4h+3}{2(1+h)^2} = 0$
si et seulement si $-2h^2-4h+3=0$, les solutions de $R'(h)=0$ sont les
racines $-1 \pm \sqrt{10}/2$ de ce polynôme.
La seule racine qui appartient à l'intervalle
$[0,3/2]$ est $h_1 = -1 + \sqrt{10}/2 \approx 0.5811388$\ .  C'est
le seul point critique dans l'intervalle $[0,3/2]$.

Puisque $R(0)=R(3/2)= 0$ et $R(h_1) \approx 0.3377$, le maximum de $R$ sur
l'intervalle $[0,3/2]$ est approximativement $0.3377$ lorsque $h=h_1$.

\subQ{d} Pour $h= h_1$, la fonction itérative est
$\displaystyle f(x) = \frac{2.5 x}{1+x} - h_1 x$
et le point d'équilibre positif du système dynamique discret
$\displaystyle N_{i+1} = f(N_i) = \frac{2.5N_1}{1+N_i} - h_1 N_i$ est
$\displaystyle p(h_1) = \frac{3-2h_1}{2+2h_1} \approx 0.58113883$.

La dérivée de la fonction itérative est
$\displaystyle f'(x) = \frac{2.5}{(1+x)^2} - h_1$.  Ainsi,
\[
| f'(p(h_1)) | = \left| \frac{2.5}{(1+p(h_1))^2} - h_1 \right|
\approx \left| \frac{2.5}{(1+0.58113883)^2} - 0.5811388 \right|
\approx 0.418861 < 1
\]
et le point d'équilibre $p(h_1) \approx 0.58113883$ est
(asymptotiquement) stable. 
}

\compileSOL{\SOLUa}{\ref{6Q121}}{
La fonction génératrice est
$\displaystyle f(x) = \frac{a x}{1+x} - \frac{x}{2}$.

\subQ{a} Les point d'équilibre sont les solutions de
\[
p = f(p) = \frac{a p}{1+p} - \frac{p}{2} \ .
\]
$p=0$ est une solution possible.  Si $p\neq 0$, nous pouvons diviser des deux
côtés de l'égalité par $p$ pour obtenir
\[
1 = \frac{a}{1+p} - \frac{1}{2}  \Rightarrow  \frac{3}{2} = \frac{a}{1+p}
\Rightarrow \frac{3}{2}\, (1+p) = a \Rightarrow \frac{3p}{2} = a -
\frac{3}{2} \Rightarrow p = \frac{2a-3}{3} \ .
\]
Il y a deux points d'équilibre $p_1=0$ et $p_2 = (2a-3)/3$

\subQ{b} Nous avons
\[
f'(x) = \frac{a(1+x) - ax}{(1+x)^2} - \frac{1}{2}
= \frac{2a - (1+x)^2}{2(1+x)^2} \ . 
\]

Pour $p_1$, nous avons $|f'(p_1)| = |f'(0)| = |a - 1/2|$.  Ainsi, $p_1$ est
stable si $|a - 1/2|<1$; c'est-à-dire, $-1/2 < a < 3/2$.  De plus,
$p_1$ est instable si $a < -1/2$ ou $a>3/2$.  Nous ne pouvons rien conclure
à partir du Théorème de stabilité des points d'équilibre pour $a=-1/2$
ou $a=3/2$.  Il faut tracer le graphe en forme de toile d'araignée
pour déterminer la stabilité dans ces deux cas.

Pour $p_2$, nous avons
\[
|f'(p_2)| = \left|f'\left(\frac{2a-3}{3}\right)\right|
= \left| \frac{2a - (1+(2a-3)/3)^2}{2(1+(2a-3)/3)^2} \right|
= \left| \frac{2a - (2a/3)^2}{2(2a/3)^2} \right|
= \left| \frac{9}{4a} - \frac{1}{2} \right| \ .
\]
Si $a> 0$, alors $p_2$ est stable si
\[
\left| \frac{9}{4a} - \frac{1}{2} \right| < 1
\Rightarrow -1 < \frac{9}{4a} - \frac{1}{2} < 1
\Rightarrow -\frac{1}{2} < \frac{9}{4a} < \frac{3}{2}
\Rightarrow a > \frac{3}{2} \ .
\]
Le point d'équilibre $p_2$ est instable si
$0<a < 3/2$.   Nous ne pouvons rien conclure à partir du Théorème de
stabilité des points d'équilibre pour $a=3/2$.  Il faut tracer le
graphe en forme de toile d'araignée pour déterminer la stabilité dans
ce cas.  Pour $a = 0$, nous avons le système dynamique discret linéaire
$x_{n+1} = -0.5 x_n$ qui possède seulement le point d'équilibre
stable $p_1$.
}

\compileSOL{\SOLUb}{\ref{6Q122}}{
\subQ{a} La fonction itérative du système dynamique discret est
$f(x) = 2.5 x e^{-x} - h x$.  Les points d'équilibre sont les solutions de
$f(x)=x$.  Le point $x=0$ est une des solutions et donc un point
d'équilibre.  Si $x\neq 0$, nous pouvons diviser les deux côtés de
l'égalité $2.5 x e^{-x} - h x=x$ par $x$ pour obtenir $2.5 e^{-x} - h=1$.
La solution de cette dernière équation est le deuxième point d'équilibre
$\displaystyle x = p(h) = \ln\left(\frac{5}{2+2h}\right)$.

Pour que le point d'équilibre
$\displaystyle p(h)=\ln\left(\frac{5}{2+2h}\right)$ soit non négatif,
il faut avoir $\displaystyle \frac{5}{2+2h} \geq 1$.  Ce qui donne
$h \leq 3/2$. Naturellement, le facteur d'efficacité $h$ ne peut pas
être négatif.   De plus, pour que le problème ait un sens biologique,
il faut aussi que $h\leq 1$ car nous ne pouvons pas récolter plus de 100\%
de la population.

\subQ{b} La récolte à long terme est
\[
R(h) = h\,p(h) = h \ln\left(\frac{5}{2+2h}\right)
\]
pour $0\leq h \leq \frac{3}{2}$.

\subQ{c} Comme $R$ est une fonction continue définie sur l'intervalle fermé
$[0,3/2]$, nous pouvons utiliser le Théorème des valeurs extrêmes pour
trouver le maximum absolu de $R$ sur l'intervalle $[0,3/2]$.  Nous avons
\[
R'(h) = \ln\left(\frac{5}{2+2h}\right) - \frac{h}{1+h} \ .
\]
Puisque $R'(0) = \ln(5/2)>0$ et $R'(3/2) = -3/5 < 0$, il y a au
moins un point critique positif pour $R$ (i.e.\ une solution de
$R'(h)=0$) entre $0$ et $3/2$ d'après le Théorème des
valeurs intermédiaires.  Puisque
\[
R''(h) = - \frac{1}{1+h} - \frac{h}{(1+h)^2} = \frac{-2-h}{(1+h)^2} < 0
\]
pour tout $h \in [0,3/2]$, la fonction $R'$ est strictement
décroissante.  Le point critique est donc unique; c'est-à-dire, $R'$
ne peut pas couper l'axe des $h$ plus d'une fois.  Il est impossible
de résoudre algébriquement $R'(h) = 0$.  Nous devons donc utiliser la
méthode de Newton avec $g(h) = R'(h)$.  Pour $h_0 =1 \in [0,3/2]$,
nous obtenons les résultats suivants.
\[
\begin{array}{c|c|c}
n & h_n & h_{n+1} = h_n - \frac{g(h_n)}{g'(h_n)} \\
\hline
0 & 1 & 0.630858068418946 \\
1 & 0.630858068418946 & 0.671659064055244 \\
2 & 0.671659064055244 & 0.672372670703419 \\
3 & 0.672372670703419 & 0.672372880075553 \\
4 & 0.672372880075553 & 0.672372880075571 \\
5 & 0.672372880075571 & 0.672372880075571 \\
\end{array}
\]
% voir NewtonHarvest.m pour calculer les données
Ainsi, le point critique est $H\approx 0.672372880075571$\ .
Puisque $R(0)= R(3/2)=0$ et $R(H) \approx 0.270325652399176>0$, La récolte à
long terme est maximale lorsque $h=H$.  Le point d'équilibre à long
terme est $P(H) \approx 0.402047228$.
}

%%% Local Variables: 
%%% mode: latex
%%% TeX-master: "notes"
%%% End: 

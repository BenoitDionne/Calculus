\chapter[Suites et séries]{Suites et séries}\label{ChapSuitesSeries}

\compileTHEO{

Au chapitre précédent, nous avons introduit les fonctions; les objets
sur lesquels nous travaillerons tout au long de cet ouvrage.  Ce ne
sont pas toutes les fonctions qui nous intéressent.  Par exemple, nous
sommes intéressé aux fonctions qui peuvent représenter des phénomènes
naturels qui évoluent de façon continue dans le temps et qui ont des
comportements prévisibles.  Les fonctions qui représentent ce genre de
phénomènes sont généralement {\em continues} ou même
{\em différentiables}.  Pour définir ces propriétés, nous aurons
besoin des concepts de {\em limite d'une suite} et de
{\em limite d'une fonction en un point}.  Nous étudions la limite
d'une suite dans ce chapitre alors que la limite d'une fonction en un
point sera présentée au prochain chapitre.

Certaines fonctions ne peuvent être définies de façon simple comme,
par exemple, à l'aide d'une expression algébrique.  Nous aurons besoin
du concept de {\em séries} pour définir ces fonctions.  C'est le cas
en particulier de la fonction $e^x$.  Ce sont ces concepts que nous
présentons dans la deuxième partie du chapitre.

\section{Suites}

\begin{focus}{\dfn}
Une {\bfseries suite}\index{Suite} est un ensemble infini et ordonné
de nombres.  Les nombres qui forment une suite sont appelés les
{\bfseries termes}\index{Suite!terme} de la suite.  Par exemple,
$\{ a_1, a_2, a_3, \ldots \}$ représente une suite de nombres où $a_1$
est le {\bfseries premier terme} de la suite, $a_2$ est le
{\bfseries deuxième terme} de la suite, ainsi de suite.

Les termes de la suite sont généralement donnés par une fonction
$f:\NN^+\rightarrow \RR$; c'est-à-dire, $a_n = f(n)$ pour $n \in \NN^+$.
La suite est alors dénoté  $\{ f(n) \}_{n=1}^\infty$.
\end{focus}

\begin{egg}
La suite $\displaystyle \left\{\frac{n}{n+2}\right\}_{n=1}^\infty$ est
la suite formée des termes
\[
\left\{\frac{1}{3}, \frac{2}{4}, \frac{3}{5}, \ldots \right\}
\]
alors que la suite
$\displaystyle \left\{\frac{(-1)^n}{n^2}\right\}_{n=1}^\infty$ est
la suite formée des termes
\[
\left\{-1, \frac{1}{4}, -\frac{1}{9}, \frac{1}{16}, \ldots \right\} \; .
\]
\end{egg}

\begin{egg}
La procédure pour générer certaines suites demande plus qu'une simple
fonction $f:\NN^+ \rightarrow \RR$.  C'est le cas de la
{\bfseries suite de Fibonnacci}.  Après avoir choisis $a_1$ et $a_2$, les
autres termes de la suites sont générés par $a_{n+1} = a_n + a_{n-1}$
pour $n=2$, $3$, $4$, \ldots.  En d'autres mots, chaque terme de la
suite est le résultat de la somme des deux termes qui le précède.

La suite de Fibonnacci la plus classique est lorsque $a_1=a_2=1$.  Nous
obtenons la suite
\[
\{ 1, 1, 2, 3, 5, 8, 13, 21, \ldots \}\; .
\]
\end{egg}

Nous sommes souvent intéressés par le comportement asymptotique des
termes de la suite lorsque $n\rightarrow \infty$.

\begin{focus}[\theory]{\dfn} \index{Suite!convergence}
Nous disons que la suite $\{ a_n\}_{n=1}^\infty$ {\bfseries converge}
(ou {\bfseries tend}) vers un nombre $L$
lorsque $n$ tend vers l'infini si,
{\em pour chaque valeur $\epsilon >0$}, il existe un 
entier $N$ (qui peut dépendre de $\epsilon$) tel que
\[
|a_n-L| < \epsilon \qquad \text{si} \qquad n \geq N \; .
\]

En d'autre mots, Quel que soit le nombre $\epsilon$ qui est donné, nous
avons que la distance entre $L$ et $a_n$ est plus petite que $\epsilon$ à
partir d'un indice $N$.  Nous écrivons
\[
\lim_{n\rightarrow \infty} a_n = L \; .
\]
Le nombre $L$ est appelé la {\bfseries limite}\index{Suite!limite} de la
suite.

S'il n'existe pas de valeur $L$ qui satisfait la définition ci-dessus,
nous disons que la suite {\bfseries diverge}.
\end{focus}

\begin{egg}[\theory]
Il est intuitivement claire que la suite
\[
\left\{ \frac{1}{n} \right\}_{n=1}^\infty
= \left\{ 1 , \frac{1}{2}, \frac{1}{3}, \frac{1}{4}, \ldots \right\}
\]
tend vers $0$.  Nous pouvons vérifier que la définition de limite est
satisfaite par $L=0$.  En effet, pour $\epsilon >0$ donné, il suffit
de prendre $N > 1/\epsilon$\footnote{L'existence de $N$ est une
conséquence du principe d'Archimède} pour obtenir
$|a_n - 0| = 1/n \leq 1/N < \epsilon$ pour $n \geq N$.
\end{egg}

\begin{egg}[\theory]
La suite
\[
\left\{\frac{n}{n+2}\right\}_{n=1}^\infty
= \left\{ \frac{1}{3}, \frac{2}{4}, \frac{3}{5}, \frac{4}{6}, \ldots
\right\}
\]
tend vers $1$.

En effet, pour $\epsilon$ donné, il suffit de prendre
$N > 2(1-\epsilon)/\epsilon$ pour obtenir
$|a_n-1| < \epsilon$ lorsque $n\geq N$.  En effet,
\[
n > \frac{2(1-\epsilon)}{\epsilon} \Rightarrow
n\epsilon > 2 - 2\epsilon \Rightarrow \epsilon(n+2) > 2
\Rightarrow \epsilon > \frac{2}{n+2} = 1 - \frac{n}{n+2}  
= | a_n - 1|
\]

Par exemple, si $\epsilon = 10^{-5}$, il faut prendre
$N > 2(1-10^{-5})/10^{-5} = 199998$.
\end{egg}

\begin{rmkList}
\begin{itemize}
\item Nous utilisons aussi l'expression \flqq\ $a_n \rightarrow L$ lorsque
  $n\rightarrow \infty$\ \frqq\ pour dire que
  $\displaystyle \lim_{n\rightarrow \infty} a_n = L$. 
\item Il est souvent très utile d'explorer numériquement le
  comportement des suites $\{ a_n\}_{n=1}^\infty$; c'est-à-dire,
  d'évaluer $a_n$ pour de très grandes valeurs de $n$ dans l'espoir
  d'obtenir des termes qui approchent une constante $L$ quelconque.
  Cela ne démontre pas que la suite $\{ a_n\}_{n=1}^\infty$ tend vers
  $L$ mais cela nous permet de conjecturer la limite de la suite.  Il
  faut vérifier que la définition de limite ci-dessus est satisfaite
  pour pouvoir dire que la suite $\{ a_n\}_{n=1}^\infty$ tend vers la
  constante $L$ obtenue numériquement.
\item La définition de limite ci-dessus revient à dire que la suite
  $\{ a_n\}_{n=1}^\infty$ tend vers $L$ si et seulement si la suite
  $\{ |a_n - L|\}_{n=1}^\infty$ tend vers $0$ lorsque $n$ tend vers
  l'infini.  Nous écrivons
  \[
    \lim_{n\rightarrow \infty} |a_n-L| = 0 \; .
  \]
  En d'autres mots, la distance $|a_n - L|$ entre $a_n$ et $L$ approche
  $0$ lorsque $n$ augmente. 
\end{itemize}
\end{rmkList}

La limite de suites possède les propriétés suivantes qui
permettront d'évaluer facilement les limites de certaines suites.

\begin{focus}{\thm} \label{suite3}
Si $\displaystyle \{ a_n\}_{n=1}^\infty$ est une suite qui tend vers
$L_a$ et $\displaystyle \{ b_n\}_{n=1}^\infty$ est une suite qui tend
vers $L_b$, alors:
\begin{enumerate}
\item La suite $\displaystyle \{ a_n + b_n \}_{n=1}^\infty$ tend vers
$L_a + L_b$.  Nous écrivons
\[
\lim_{n\rightarrow \infty} (a_n + b_n) =
\lim_{n\rightarrow \infty} a_n + \lim_{n\rightarrow \infty} b_n \; .
\]
\item La suite $\displaystyle \{ a_n b_n \}_{n=1}^\infty$ tend vers
$L_a L_b$.  Nous écrivons
\[
\lim_{n\rightarrow \infty} (a_n b_n) =
\left(\lim_{n\rightarrow \infty} a_n \right)
\left(\lim_{n\rightarrow \infty} b_n \right) .
\]
\item Si $b_n \neq 0$ pour tout $n$ et $L_b\neq 0$, alors la suite
$\displaystyle \left\{ \frac{a_n}{b_n} \right\}_{n=1}^\infty$
tend vers $\displaystyle \frac{L_a}{L_b}$.  Nous écrivons
\[
\lim_{n\rightarrow \infty} \frac{a_n}{b_n} =
\frac{\displaystyle \lim_{n\rightarrow \infty} a_n}
{\displaystyle \lim_{n\rightarrow \infty} b_n} \; .
\]
\end{enumerate}
\end{focus}

\begin{egg}
Soit $c$ une constante quelconque.  Il est facile de voir que la
limite de la suite $\displaystyle \{ a_n\}_{n=1}^\infty$ où $a_n = c$
pour tout $n$ est $c$.  Nous pouvons utiliser cette information pour
trouver la limite de la suite
$\displaystyle \left\{ \frac{5n+2}{n} \right\}_{n=1}^\infty$.  En
effet, grâce au théorème précédent, nous avons
\[
\lim_{n\rightarrow \infty} \frac{5n+2}{n}
= \lim_{n\rightarrow \infty} \left(5 + 2\,\frac{1}{n} \right)
= \lim_{n\rightarrow \infty} 5 + 2\,\lim_{n\rightarrow \infty}\frac{1}{n}
= 5 + 2 \times 0 = 5 \; .
\]
\end{egg}

\begin{focus}[][des gendarmes ou sandwich]{\thm} \label{gendarmeS}
\index{Théorème des gendarmes} \index{Théorème sandwich}
Soit
$\displaystyle \{ a_n\}_{n=1}^\infty$ et $\displaystyle \{c_n\}_{n=1}^\infty$
deux suites qui tendent vers $L$.  Si
$\displaystyle \{ b_n\}_{n=1}^\infty$ est une suite telle que 
$a_n \leq b_n \leq c_n$ pour tout $n$, alors la
suite $\displaystyle \{ b_n\}_{n=1}^\infty$ tend aussi vers $L$.
\end{focus}

\begin{proof}[\theory]
Soit $\epsilon>0$.  Puisque $\displaystyle \{ a_n\}_{n=1}^\infty$ et
$\displaystyle \{c_n\}_{n=1}^\infty$ tendent vers $L$, il existe $N\geq 0$
tel que $|a_n-L|<\epsilon$ et $|c_n - L|<\epsilon$ pour $n\geq N$.  Ainsi,
\[
L-\epsilon < a_n \leq b_n \leq c_n < L+\epsilon
\]
pour $n\geq N$.  C'est-à-dire, $|b_n-L|<\epsilon$ pour $n\geq N$.
\end{proof}

\begin{egg}[\theory\ \eng]
Montrons que la limite de la suite
$\displaystyle \{ b_n\}_{n=1}^\infty$ où $b_n = n^{1/n}$ est $1$.

Il est clair que la suite
$\displaystyle \{ a_n\}_{n=1}^\infty$ où $a_n = 1$ pour tout
$n$ satisfait $a_n \leq b_n$ pour tout $n$.  De plus, la suite
$\displaystyle \{ c_n\}_{n=1}^\infty$ où $c_n = 1+\sqrt{2/n}$ pour
tout $n$ satisfait $b_n \leq  c_n$ pour tout $n$.

Pour démontrer ce dernier énoncé, posons $x_n = b_n-1$ et
notons que
\[
n = b_n^n = (1+x_n)^n
= 1 + nx_n + \frac{n(n-1)}{2}\,x_n^2 + \ldots + x_n^n \; .
\]
Comme c'est une somme de termes positifs, nous obtenons
\[
n \geq 1 + \frac{n(n-1)}{2}\,x_n^2 \; .
\]
Si nous isolons $x_n$, nous trouvons
\[
x_n \leq \sqrt{2/n} \quad , \quad n \geq 1 \; .
\]
Ainsi,
\[
b_n = 1 + x_n \leq 1 + \sqrt{2/n} = c_n
\]
pour tout $n$.

Nous avons $\displaystyle \lim_{n\rightarrow \infty} a_n = 1$ et
\[
\lim_{n\rightarrow \infty} c_n = \lim_{n\rightarrow \infty}
\left(1+ \sqrt{2/n} \right) = 1 + \lim_{n\rightarrow \infty} \sqrt{2/n} = 1
\]
car nous pouvons facilement montrer à partir de la définition de limite que
la suite $\displaystyle \left\{ \sqrt{2/n} \right\}_{n=0}^\infty$ tend
vers $0$.  Puisque $a_n \leq b_n \leq c_n$ pour tout $n \geq 1$, nous
obtenons du Théorème des gendarmes que
$\displaystyle \{ b_n\}_{n=1}^\infty$ tend vers $1$.
\end{egg}

Le résultat suivant est quelques fois utile.

\begin{focus}{\thm}
La suite $\displaystyle \{ a_n\}_{n=1}^\infty$ tend vers $0$ si et
seulement si la suite $\displaystyle \{ |a_n| \}_{n=1}^\infty$ tend vers
$0$. \label{suiteAbs}
\end{focus}

\begin{proof}[\theory]
\subQ{$\Rightarrow$} Soit $\epsilon>0$. Puisque que
$\displaystyle \{ a_n\}_{n=1}^\infty$ 
tend vers $0$, il existe $N>0$ tel que $|a_n- 0 | = |a_n| <\epsilon$
pour $n\geq N$.  Ainsi, il existe $N>0$ tel que
$\big| |a_n|- 0 \big| = |a_n| <\epsilon$ pour $n\geq N$.  Comme
$\epsilon$ est arbitraire, cela implique que
$\displaystyle \{|a_n|\}_{n=1}^\infty$ tend vers $0$.

\subQ{$\Leftarrow$} Soit $\epsilon>0$. Puisque que
$\displaystyle \{ |a_n|\}_{n=1}^\infty$ tend vers $0$, il existe $N>0$
tel que $\big| |a_n|- 0 \big| = |a_n| <\epsilon$ pour $n\geq N$.
Ainsi, il existe $N>0$ tel que $|a_n- 0| = |a_n| <\epsilon$ pour
$n\geq N$.  Comme $\epsilon$ est arbitraire, cela implique que
$\displaystyle \{a_n\}_{n=1}^\infty$ tend vers $0$.
\end{proof}

Il est très utile de pouvoir caractériser le comportement des suites
lorsque $n$ tend vers l'infini.

\begin{focus}{\dfn}
\begin{enumerate}
\item Une suite $\{a_n\}_{n=1}^\infty$ est {\bfseries croissante}
\index{Suite!croissante} si $a_n < a_{n+1}$ pour tout $n$.
\item Une suite $\{a_n\}_{n=1}^\infty$ est {\bfseries décroissante}
\index{Suite!décroissante} si $a_n > a_{n+1}$ pour tout $n$.
\item Une suite $\{a_n\}_{n=1}^\infty$ est
{\bfseries bornée supérieurement}\index{Suite!bornée supérieurement}
s'il existe un nombre réel $M$ tel que $a_n \leq M$ pour tout $n$.  Le
nombre $M$ est appelé une {\bfseries borne supérieure} pour la suite.
\item Une suite $\{a_n\}_{n=1}^\infty$ est
{\bfseries bornée inférieurement}\index{Suite!bornée inférieurement}
s'il existe un nombre réel $m$ tel que $a_n \geq m$ pour tout $n$.  Le
nombre $m$ est appelé une {\bfseries borne inférieure} pour la suite.
\item Une suite $\{a_n\}_{n=1}^\infty$ est
{\bfseries bornée}\index{Suite!bornée} si elle est bornée
inférieurement et supérieurement.
\end{enumerate}
\end{focus}

Le concept de convergence peut aussi être adapté aux suites non bornées.

\begin{focus}[\theory]{\dfn}
\index{Suite!convergence à l'infini} 
Nous disons qu'une suite $\displaystyle \{ a_n\}_{n=1}^\infty$
{\bfseries converge} (ou {\bfseries tend}) vers $+\infty$ (plus
l'infini) si, pour chaque $M>0$, il existe $N$ (qui dépend du $M$ donné)
tel que $a_n > M$ pour tout $n \geq N$.  Nous écrivons
\[
\lim_{n\rightarrow \infty} a_n = +\infty \; .
\]
Nous disons qu'une suite $\displaystyle \{ a_n\}_{n=1}^\infty$
{\bfseries converge} (ou {\bfseries tend}) vers $-\infty$ (moins
l'infini) si, pour chaque $M<0$, il existe $N$ (qui dépend du $M$ donné)
tel que $a_n < M$ pour tout $n \geq N$.  Nous écrivons
\[
\lim_{n\rightarrow \infty} a_n = -\infty \; .
\]
\label{def_of_lim_at_inf}
\end{focus}

\begin{egg}[\theory]
La suite $\displaystyle \{ a_n\}_{n=1}^\infty$ où $a_n = n^2+1$ tend
vers plus l'infini.  En effet, si $M\geq 1$ est un nombre quelconque,
il suffit de prendre $N > \sqrt{M-1}$ pour obtenir
\[
a_n = n^2+1 \geq N^2+1 > \left(\sqrt{M-1}\right)^2 + 1 = M
\]
pour tout $n\geq N$.
\end{egg}

\begin{egg}
Le fait qu'une suite ne soit pas bornée n'implique pas qu'elle tende
vers plus ou moins l'infini.  Par exemple, la suite 
$\{a_n\}_{n=1}^\infty$ où
\[
a_n =
\begin{cases}
n & \qquad \text{si $n$ est pair}\\
1/n & \qquad \text{si $n$ est impair}
\end{cases}
\]
n'est pas bornée car il y a des termes qui sont aussi grands que nous
voulons.  Par contre, cette suite ne tend pas vers plus l'infini car il
est impossible de trouver $N>0$ tel que $a_n> 1$ pour tout $n\geq N$.
Les termes $a_n$ avec $n$ impair et plus grand que $1$ sont tous plus
petits que $1$.
\end{egg}

Les deux propositions suivantes nous donnent la limite de certaines suites
que nous retrouvons fréquemment dans les applications.

\begin{focus}{\prp}
Soit $r$ un nombre réel.  Alors
\[
\lim_{n\rightarrow \infty} r^n=
\begin{cases}
\infty &\qquad \text{pour $r>1$} \\
1 &\qquad \text{pour $r=1$} \\
0 &\qquad \text{pour $-1 < r < 1$}
\end{cases} \; .
\]
La suite $\{r^n\}_{n=1}^\infty$ ne converge pas pour $r\leq -1$.
\label{suite1}
\end{focus}

\begin{focus}{\prp}
Soit $r$ un nombre réel.  Alors
\[
\lim_{n\rightarrow \infty} \frac{1}{n^r}=
\begin{cases}
0 &\qquad \text{pour $r>0$} \\
1 &\qquad \text{pour $r=0$} \\
\infty &\qquad \text{pour $r<0$}
\end{cases} \; .
\]
\label{suite2}
\end{focus}

\begin{egg}
Évaluons les limites suivantes.
\begin{center}
\begin{tabular}{*{1}{l@{\hspace{1em}}l@{\hspace{5em}}}l@{\hspace{1em}}l}
\subQ{a} & $\displaystyle \lim_{n\rightarrow \infty} \frac{n}{n^2+2n+4}$ &
\subQ{b} & $\displaystyle \lim_{n\rightarrow \infty} \frac{\cos(n)}{n^2}$
\end{tabular}
\end{center}

Pour calculer la limite en (a), nous utilisons les
théorèmes~\ref{suite3} et \ref{suite2} pour obtenir
\begin{align*}
\lim_{n\rightarrow \infty} \frac{n}{n^2+2n+4}
&= \lim_{n\rightarrow \infty} \frac{1/n}{1+2/n+4/n^2} \\
&= \frac{\displaystyle \lim_{n\rightarrow \infty} 1/n}
{\displaystyle 1 + \lim_{n\rightarrow \infty} 2/n
+ 4 \lim_{n\rightarrow \infty} 1/n^2} = \frac{0}{1+0+0} = 0 \; .
\end{align*}

Pour évaluer la limite en (b), nous utilisons le
Théorème des gendarmes.  Puisque
\[
0 \leq \left| \frac{\cos(n)}{n^2} \right| \leq \frac{1}{n^2}
\]
pour tout $n$, la suite $\{b_n\}_{n=1}^\infty$ avec $b_n = |\cos(n)/n^2|$
tend vers $0$ car les suites $\{a_n\}_{n=1}^\infty$ avec $a_n = 0$ et
$\{c_n\}_{n=1}^\infty$ avec $c_n = 1/n^2$ tendent vers $0$.  Puisque la
suite
$\displaystyle \left\{ \left| \frac{\cos(n)}{n^2} \right|\right\}_{n=1}^\infty$
tend vers $0$, il en est de même de la suite 
$\displaystyle \left\{\frac{\cos(n)}{n^2}\right\}_{n=1}^\infty$ grâce
au théorème~\ref{suiteAbs}.  Nous déduisons donc que
\[
\lim_{n\rightarrow \infty} \frac{\cos(n)}{n^2} = 0 \; .
\]
\end{egg}

\begin{focus}[\theory]{\thm}
Toute suite croissante et bornée supérieurement converge.  De même,
toute suite décroissante et bornée inférieurement converge.  Dans le
premier cas, la limite est la plus petite borne
supérieure\footnotemark\ de la suite alors que dans le deuxième cas
c'est la plus grande borne inférieure de la suite.
\label{suiteBORN}
\end{focus}

\footnotetext{{\bfseries L'Axiome de complétude} des nombres
réels dit que tout ensemble non vide et borné supérieurement
possède une plus petite borne supérieure.  Cette borne ne fait pas
nécessairement partie de l'ensemble.  Par exemple, $\sqrt{2}$ est
la plus petite borne supérieure de l'ensemble
$\{ x \in \RR : x^2 <4\}$ mais ne fait pas partie de cet ensemble.  Il
en découle aussi que tout ensemble non vide et borné inférieurement
possède une plus  grande borne inférieure.}

\begin{proof}[\theory]
Nous considérons le cas où $\displaystyle \{ a_n \}_{n=0}^\infty$ est une
suite décroissante et bornée inférieurement.  La démonstration pour
une suite croissante et bornée supérieurement est semblable.  Soit $m$
la plus grande borne inférieure de
$\displaystyle \{ a_n \}_{n=0}^\infty$.  Montrons que
$\displaystyle \lim_{n\to \infty} a_n = m$.

Soit $\epsilon >0$.  Puisque $m+\epsilon > m$ n'est pas une borne
inférieure de la suite $\displaystyle \{ a_n \}_{n=0}^\infty$, il
existe $N>0$ tel que $m \leq a_N < m+\epsilon$.  Puisque la suite
$\displaystyle \{ a_n \}_{n=0}^\infty$ est décroissante, nous avons
$m \leq a_n < m+\epsilon$ pour $n\geq N$.  C'est-à-dire,
$|a_n-m|<\epsilon$ pour $n\geq N$.  Comme $\epsilon$ est arbitraire,
$\displaystyle \lim_{n\to \infty} a_n = m$.
\end{proof}

\begin{egg}[\theory]
Quel est la limite de la suite
\[
\left\{ \sqrt{2}, \sqrt{2\sqrt{2}}, \sqrt{2\sqrt{2\sqrt{2}}}, \ldots
\right\} \; ?
\]
Cette suite est produite de façon itérative. C'est la suite
$\{a_n\}_{n=1}^\infty$ où $a_1=\sqrt{2}$ et $a_{n+1} = \sqrt{2a_n}$
pour $n=1$, $2$, \ldots.

Montrons que la suite $\{a_n\}_{n=1}^\infty$ est bornée
supérieurement par $2$.  Cela se démontre par induction.  Nous avons
$a_1<2$.  Supposons que $a_n < 2$.  Alors $2a_n < 4$ et
\[
a_{n+1} = \sqrt{2a_n} < \sqrt{4} = 2 \; .
\]
Donc $a_n <2$ pour tout $n$ par induction.

De plus, montrons que la suite $\{a_n\}_{n=1}^\infty$ est
croissante.   Cela se démontre aussi par induction.  En fait, nous  montrons
par induction que
\begin{equation}\label{sssqrt}
a_{n+1} = 2^{1/2^{(n+1)}}\,a_n
\end{equation}
pour tout $n$.  Puisque $2^{1/2^{(n+1)}} > 1$, nous obtenons alors que
$a_{n+1} > a_n$ pour tout $n$.  L'équation (\ref{sssqrt}) est vrai
pour $n=1$ car
\[
a_2 = \sqrt{2 \sqrt{2}} = \sqrt{2} \sqrt{\sqrt{2}} =
2^{1/2^2} \sqrt{2} = 2^{1/2^2} a_1 \; .
\]
Supposons que l'équation (\ref{sssqrt}) soit vrai pour $n$ et
montrons qu'elle est alors vrai pour $n+1$.  Nous avons
\[
a_{n+2} = \sqrt{2a_{n+1}} = \sqrt{2 \left(2^{1/2^{(n+1)}} \, a_n\right)}
= \sqrt{2 a_n} \sqrt{2^{1/2^{(n+1)}}}
= a_{n+1} 2^{1/2^{(n+2)}} \; .
\]
La deuxième égalité est une conséquence de l'hypothèse d'induction.
Ce qui démontre que (\ref{sssqrt}) est vrai pour $n+1$ et complète
la preuve par induction.

Donc $\{a_n\}_{n=1}^\infty$ est une suite croissante et
bornée supérieurement.  Il découle du théorème précédent que la suite
converge. Appelons $L$ la limite de la suite $\{a_n\}_{n=1}^\infty$.

La suite $\{ a_{n+1} \}_{n=1}^\infty$ tend aussi vers $L$ car c'est la
suite $\{ a_n\}_{n=1}^\infty$ auquel nous avons enlevé le premier terme.  De
plus, nous pouvons montrer à partir de la définition de limite d'une suite
que $\{ \sqrt{2a_n} \}_{n=1}^\infty$ tend vers $\sqrt{2L}$.  Puisque
$a_{n+1} = \sqrt{2 a_n}$ pour tout $n$, nous avons donc que
\[
L= \lim_{n\rightarrow \infty} a_{n+1} =
\lim_{n\rightarrow \infty} \sqrt{2 a_n} = \sqrt{2L} \; .
\]
La limite $L$ doit donc satisfaire l'équation $L^2-2L = L(L-2) = 0$.
Puisque $\{ a_n \}_{n=1}^\infty$ est une suite croissante de termes
positifs, nous devons avoir $L=2$.
\end{egg}

\section{Séries}

Si $a_1$, $a_2$, \ldots, $a_n$ sont $n$ nombres réels, le symbole
$\displaystyle \sum_{k=1}^n a_k$ dénote la somme de ces nombres.  En
d'autres mots,
\[
\sum_{k=1}^n a_k = a_1 + a_2 + \ldots + a_n \; .
\]
De façon générale, si $k_1 \leq k_2$ sont deux nombres entiers et
$a_k$ pour $k_1 \leq k \leq k_2$ sont des nombres réels, alors
\[
\sum_{k=k_1}^{k_2} a_k = a_{k_1} + a_{k_1+1} + a_{k_1+2} + \ldots + a_{k_2} \ .
\]
En générale, nous avons que $a_k = f(k)$ pour une fonction $f:\ZZ \to \RR$.
Ainsi,
\[
\sum_{k=k_1}^{k_2} a_k = \sum_{k=k_1}^{k_2} f(k) =
f(k_1) + f(k_1+1) + f(k_1+2) + \ldots + f(k_2) \ .
\]

\begin{egg}
Nous avons
\begin{align*}
\sum_{k=3}^8 \ \underbrace{k^2}_{=f(k)}
&= \underbrace{3^2}_{=f(3)}
+ \underbrace{4^2}_{=f(4)}
+ \underbrace{5^2}_{=f(5)} + \ldots
+ \underbrace{8^2}_{=f(8)} \\
&= 9 + 16 + 25 + \ldots + 64
\end{align*}
et
\begin{align*}
\sum_{k=1}^6 \underbrace{\frac{1}{k^2+1}}_{=f(k)}
&= \underbrace{\frac{1}{1^2+1}}_{=f(1)}
+ \underbrace{\frac{1}{2^2+1}}_{=f(2)}
+ \underbrace{\frac{1}{3^2+1}}_{=f(3)} + \ldots
+ \underbrace{\frac{1}{6^2+1}}_{=f(6)} \\
&= \frac{1}{2}+ \frac{1}{5} + \frac{1}{10} + \ldots + \frac{1}{37} \ .
\end{align*}
\end{egg}

Il découle de la commutativité de l'addition et de la distributivité du
produit sur l'addition que
\[
\sum_{n=k_1}^{k_2} c\,a_n = c \sum_{n=k_1}^{k_2} a_n \qquad \text{et}
\qquad
\sum_{n=k_1}^{k_2} (a_n+b_n) = \sum_{n=k_1}^{k_2} a_n + \sum_{n=k_1}^{k_2} b_n
\ ,
\]
où $c$, $a_n$ et $b_n$ pour $k_1\leq n \leq k_2$ sont des nombres
réels.

\begin{focus}[\theory]{\prp}
Certaines sommes sont fréquemment utilisées.
\begin{align*}
\sum_{k=1}^n 1 &= \underbrace{1+1+1+\ldots+1}_{n\text{ fois}} = n \ ,
& \sum_{k=0}^n k &= \frac{n(n+1)}{2} \ ,\\
\sum_{k=0}^n k^2 &= \frac{n(n+1)(2n+1)}{6} ,
& \sum_{k=0}^n k^3 &= \left(\frac{n(n+1)}{2}\right)^2
\intertext{et}
\sum_{k=0}^n k^4 &= \frac{n(n+1)(2n+1)(3n^3+3n-1)}{30} \ .
\end{align*}
\end{focus}

\begin{rmk}[\theory]
\subI{i} Démontrons premièrement la deuxième identité.  Posons
$\displaystyle S = \sum_{k=1}^n k$.  Nous avons
\[
\begin{array}{ll|c|c|c|c|c|c|c|c|c|r}
\cline{3-3}\cline{5-5}\cline{7-7}\cline{9-9}\cline{11-11}
& S = &1&+ &2&+ &3&+ \ldots + &(n-1)&+ &n& \\
\text{et\qquad} &&&&&&&&&&& \\
& S = &n&+ &(n-1)& + &(n-2)&+ \ldots + &2&+ &1& . \\
\cline{3-3}\cline{5-5}\cline{7-7}\cline{9-9}\cline{11-11}
\multicolumn{2}{c}{} &
\multicolumn{9}{c}{\underbrace{\rule{24em}{0em}}_{n\text{ fois}}} &
\end{array}
\]
La somme de ces deux équations donne $2S = n(n+1)$.  Donc
\[
\sum_{k=0}^n k = S = \frac{n(n+1)}{2} \ .
\]

\subI{ii} Nous donnons deux démonstrations de la troisième identité.

\subI{ii-1} La première démonstration ne fait pas appel au principe
d'induction.  Considérons la séries {\em télescopique}\quad
$\displaystyle \sum_{k=1}^n ((k+1)^3 -k^3)$.  Puisque
\begin{align*}
\sum_{k=1}^n ((k+1)^3 -k^3) &= (2^3 -1)+(3^3-2^3)+(4^3-3^3) + \ldots \\
&\qquad + (n^3-(n-1)^3) + ((n+1)^3-n^3) \\
&= -1 + (2^3 -2^3)+(3^3-3^3) + \ldots + (n^3-n^3) + (n+1)^3 \\
&= (n+1)^3 - 1
\end{align*}
et
\begin{align*}
\sum_{k=1}^n ((k+1)^3 -k^3) &= \sum_{k=1}^n ( 3k^2 + 3k + 1)
= 3 \sum_{k=1}^n k^2 + 3 \sum_{k=1}^n k + \sum_{k=1}^n 1 \\
&= 3 \sum_{k=1}^n k^2 + \frac{3n(n+1)}{2} + n \ ,
\end{align*}
nous obtenons que
\[
3 \sum_{k=1}^n k^2 + \frac{3n(n+1)}{2} + n = (n+1)^3 - 1 \ .
\]
Si nous résolvons pour $\displaystyle \sum_{k=1}^n k^2$, nous trouvons
\begin{align*}
\sum_{k=1}^n k^2 &= \frac{1}{3}
\left( (n+1)^3 - 1 - \frac{3n(n+1)}{2}- n \right) \\
&= \frac{1}{3} \left( n^3 +3n^2 + 3n -\frac{3}{2} n^2 - \frac{3}{2}n -
  n\right)\\
&=\frac{1}{3}\left( n^3 +\frac{3}{2}n^2 + \frac{1}{2}n\right)
= \frac{n}{6}(2n^2 + 3n+1) = \frac{n(2n+1)(n+1)}{6}
\end{align*}

\subI{ii-2} La deuxième démonstration utilise la principe d'induction.
L'hypothèse d'induction est
\[
P(n) : \qquad \sum_{k=1}^n k^2 = \frac{n(n+1)(2n+1)}{6} \; .
\]
$P(1)$ est vrai car, pour $n=1$, nous avons
\[
\sum_{k=1}^n k^2 = \sum_{k=1}^1 k^2 = 1^2 = 1 \quad \text{et} \quad
\frac{n(n+1)(2n+1)}{6} = \frac{1 \times 2 \times 3}{6} = 1 \; .
\]
Supposons que $P(n)$ soit vrai.  Alors
\begin{align*}
\sum_{k=0}^{n+1} k^2 &= \sum_{k=0}^n k^2 + (n+1)^2
= \frac{n(n+1)(2n+1)}{6} + (n+1)^2 \\
&= \left(\frac{n(2n+1)}{6} + (n+1)\right) (n+1)
= \frac{2n^2 + 7n + 6}{6}\,(n+1) \\
&= \frac{(2n+3)(n+2)}{6}\,(n+1) \; ,
\end{align*}
où la deuxième égalité provient de l'hypothèse d'induction.  Cette
dernière expression n'est nul autre que $P(n+1)$.  Par induction, nous
concluons que $P(n)$ est vrai pour tout $n$.

La démonstration des autres identités est laissée au lecteur. 
\end{rmk}

\begin{egg}[\theory]
Nous avons
\begin{align*}
\sum_{k=1}^{10} (2k^2 + 4 k + 1) &= 
2\sum_{k=1}^{10} k^2 + 4 \sum_{k=1}^{10} k + \sum_{k=1}^{10} 1 \\
&= 2 \frac{10(10+1)(2\times 10 +1)}{6} + 4 \frac{10(10+1)}{2} + 7
= 1000
\end{align*}
et
\begin{align*}
\sum_{k=1}^n (4k^2 + n k) &= 4\sum_{k=1}^n k^2 + n \sum_{k=1}^n k \\
&= 4 \frac{n(n+1)(2n+1)}{6} + n \frac{n(n+1)}{2}
= \frac{(11n+4)(n(n+1)}{6} \ .
\end{align*}
\end{egg}

\begin{focus}{\dfn}
Soit $\{ a_n\}_{n=1}^\infty$, \ldots une suite de nombres réels.  Posons
\[
S_k = \sum_{n=1}^k a_n = a_1 + a_2 + \ldots + a_k
\]
pour $k=1$, $2$, $3$, \ldots

L'expression $\displaystyle \sum_{n=1}^\infty a_n$ est définie comme
étant la suite $\{S_k\}_{k=1}^\infty$.

L'expression $\displaystyle \sum_{n=1}^\infty a_n$ est appelée une
{\bfseries série}\index{Série}, les $a_n$ sont les
{\bfseries termes}\index{Série!termes} de la série, et les termes
$S_k$ de la suite $\displaystyle \{S_k\}_{k=1}^\infty$ sont appelés
les {\bfseries sommes partielles}\index{Série!Somme partielle} de la
série.

Nous disons que la série {\bfseries converge}\index{Série!convergence}
si la suite $\displaystyle \{S_k\}_{k=1}^\infty$ des sommes partielles
converge.  Si la suite $\{S_k\}_{k=1}^\infty$ des sommes partielles
diverge, nous disons que la série {\bfseries diverge}\index{Série!divergence}.

Si la suite $\{S_k\}_{k=1}^\infty$ des sommes partielles tend vers
$S \in \RR$, nous écrivons
\[
\sum_{n=1}^\infty a_n = S
\]
et nous disons que la {\bfseries somme} de la série est $S$.  Par abus
de langage, nous désignons souvent la somme de la série par
$\displaystyle \sum_{n=1}^\infty a_n$.  C'est le contexte qui
détermine le sens donné à $\displaystyle \sum_{n=1}^\infty a_n$.
\end{focus}

\begin{egg}
La série $\displaystyle \sum_{n=1}^\infty \frac{1}{n}$ représente la
suite $\{S_k\}_{k=1}^\infty$ des sommes partielles
\[
S_k = \sum_{n=1}^k \frac{1}{n} = 1 + \frac{1}{2} + \frac{1}{3} + \ldots
  + \frac{1}{k} \; .
\]
La série $\displaystyle \sum_{n=1}^\infty \frac{1}{n}$ est appelée la
{\bfseries série harmonique}.\index{Série!harmonique}
Nous démontrerons plus loin que cette série ne converge pas.

La série $\displaystyle \sum_{n=1}^\infty \frac{\sin(n)}{n^2+n}$
représente la suite $\{S_k\}_{k=1}^\infty$ des sommes partielles
\[
S_k = \sum_{n=1}^k \frac{\sin(n)}{n^2+n}
= \frac{\sin(1)}{2} + \frac{\sin(2)}{6} + \frac{\sin(3)}{12} + \ldots
  + \frac{\sin(k)}{k^2+k} \; .
\]
Nous démontrerons plus loin que cette série converge.
\end{egg}

\begin{egg}[][Série géométrique]
Soit $r$ un nombre réel.  Montrons que la série
$\displaystyle \sum_{n=0}^\infty r^n$ converge si $|r|<1$ et diverge
si $|r|\geq 1$.

La série $\displaystyle \sum_{n=0}^\infty r^n$ est
très importante en mathématiques et, pour cette raison, nous lui donnons
un nom.  Nous appellons cette série la {\bfseries série géométrique}.
\index{Série!géométrique} La valeur $r$ est appelée la
{\bfseries raison} de la série géométrique.

Les sommes partielles de la série géométrique sont de la forme
\[
S_k = 1 + r + r^2 + \ldots + r^k
\]
pour $k=0$, $1$, $2$, \ldots \ Ainsi,
\begin{align*}
rS_k &= r + r^2 + r^3 + \ldots + r^{k+1}
\intertext{et}
(1-r)S_k &= S_k - r S_k = 1 - r^{k+1} \; .
\end{align*}
Donc
\[
S_k = \frac{1-r^{k+1}}{1-r}
\]
si $r\neq 1$.  Pour $|r|<1$, il découle du théorème~\ref{suite2}
que
\[
\lim_{k\rightarrow \infty} S_k
= \lim_{k\rightarrow \infty} \frac{1-r^{k+1}}{1-r}
= \frac{\displaystyle 1- \lim_{k\rightarrow \infty}r^{k+1}}{1-r}
= \frac{1}{1-r} \; .
\]
Si $r \leq -1$, la suite $\{r^{k+1}\}_{k=1}^\infty$ ne converge pas.
La suite des sommes partielles ne converge donc pas si $r\leq -1$.
Si $r>1$, la suite $\{r^{k+1}\}_{k=1}^\infty$ tend
vers plus l'infini.  Ce qui implique que la suite des sommes
partielles tend aussi vers plus l'infini.  De même, si $r=1$, nous
obtenons que $S_k = k+1$ pour tout $k$ et la suite des sommes
partielles tend donc vers plus l'infini.
\label{geoseries}
\end{egg}

En résumé, nous avons le résultat suivant.

\begin{focus}{\prp}
\[
\sum_{n=0}^\infty r^n = \frac{1}{1-r} \quad \text{pour} \quad  |r|<1 \; ,
\]
et la série $\displaystyle \sum_{n=0}^\infty r^n$ diverge si
$|r|\geq 1$.
\end{focus}

Pour déterminer si une série $\displaystyle \sum_{j=0}^\infty a_j$ est
une série géométrique, il faut vérifier que $a_{j+1}/a_j = r$, la
raison de la série, pour tout $j$.  Si c'est le cas, alors
$a_j = a_0 r^j$ pour tout $j$ et
$\displaystyle \sum_{j=0}^\infty a_j = a_0\sum_{j=0}^\infty r^j$.

\begin{egg}
Déterminons si la série
\[
\sum_{n=0}^\infty \left(\frac{1}{5}\right)^{2n}
\]
converge.  Si elle converge, nous voulons sa valeur.

Puisque
\[
\sum_{n=0}^\infty \left(\frac{1}{5}\right)^{2n}
= \sum_{n=0}^\infty \left( \left(\frac{1}{5}\right)^2 \right)^n
= \sum_{n=0}^\infty \left(\frac{1}{25}\right)^n
\]
est une série géométrique de raison $r = 1/25$, elle converge car
$|r|<1$.  En fait,
\[
\sum_{n=0}^\infty \left(\frac{1}{5}\right)^{2n} = \frac{1}{1-(1/25)}
= \frac{25}{24} \; .
\]
\end{egg}

\begin{egg}
Déterminons si la série
  $\displaystyle \sum_{n=1}^\infty \frac{7^{n+2}}{5^{2n-1}}$
converge ou diverge.

Les termes de la série sont
$\displaystyle \frac{7^{n+2}}{5^{2n-1}} = 5\times 7^2 \times
\left(\frac{7}{25}\right)^n$.  Ainsi, la série est une série
géométrique de raison $7/25$.  Comme la raison est plus petite que $1$
en valeur absolue, la série converge.
\end{egg}

\begin{focus}{\thm}
Si $\displaystyle \sum_{n=1}^\infty a_n$ et
$\displaystyle \sum_{n=1}^\infty b_n$ sont deux séries convergentes et
$c$ est un nombre réel, alors
$\displaystyle \sum_{n=1}^\infty (a_n+b_n)$ et
$\displaystyle \sum_{n=1}^\infty c\, a_n$ sont des séries convergentes.
De plus, si
\[
\sum_{n=1}^\infty a_n = A \quad \text{et} \quad \sum_{n=1}^\infty b_n = B
\]
alors
\[
\sum_{n=1}^\infty (a_n+b_n) = A+B \quad \text{et}
\quad \sum_{n=1}^\infty c\, a_n = cA \; .
\]
\label{series_linear}
\end{focus}

\begin{proof}[\theory]
Le théorème précédent est une conséquence du théorème~\ref{suite3}.
En effet, considérons les sommes partielles suivantes.
\[
A_k = \sum_{n=1}^k a_n \ , \ 
B_k = \sum_{n=1}^k b_n \ , \ C_k = \sum_{n=1}^k c\,a_n
\ \text{et} \ D_k = \sum_{n=1}^k (a_n+b_n) \; .
\]
Nous avons $C_k = c\,A_k$ et $D_k = A_k + B_k$.  Ainsi,
\[
\sum_{n=1}^\infty c\,a_n = \lim_{k\rightarrow \infty} C_k
= \lim_{k\rightarrow \infty} c\,A_k
= c\,\lim_{k\rightarrow \infty} A_k = c\, A
= c\, \sum_{n=1}^\infty a_n
\]
et
\[
\sum_{n=1}^\infty (a_n+b_n) = \lim_{k\rightarrow \infty} D_k
= \lim_{k\rightarrow \infty} (A_k + B_k)
= \lim_{k\rightarrow \infty} A_k + \lim_{k\rightarrow \infty} B_k
= A + B
= \sum_{n=1}^\infty a_n + \sum_{n=1}^\infty b_n
\; .
\]
\end{proof}

\begin{egg}[\eng]
Déterminons si la série
\[
\sum_{n=0}^\infty \frac{3-4^{n-1}}{5^n}
\]
converge.  Si elle converge, nous voulons aussi sa valeur?

Nous avons
\[
\frac{3-4^{n-1}}{5^n} =
3 \left( \frac{1}{5}\right)^n - \frac{1}{4} \left(\frac{4}{5}\right)^n \ .
\]
De plus,
\[
\sum_{n=0}^\infty \left( \frac{1}{5} \right)^n \quad \text{et}
\quad \sum_{n=0}^\infty \left( \frac{4}{5} \right)^n
\]
sont des séries géométriques de raisons $r = 1/5$ et $r = 4/5$
respectivement. Puisque $|r|<1$ pour les deux séries, elles
convergent.  Plus précisément,
\[
\sum_{n=0}^\infty \left( \frac{1}{5} \right)^n = \frac{1}{1-(1/5)} =
\frac{5}{4} \quad \text{et} \quad
\sum_{n=0}^\infty \left( \frac{4}{5} \right)^n = \frac{1}{1-(4/5)} = 5 \ .
\]
Ainsi, grâce au théorème précédent,
\[
\sum_{n=0}^\infty \frac{3-4^{n-1}}{5^n}
= 3 \sum_{n=0}^\infty \left( \frac{1}{5} \right)^n - \frac{1}{4}
\sum_{n=0}^\infty \left( \frac{4}{5} \right)^n
= \frac{15}{4} - \frac{5}{4} = \frac{5}{2} \ .
\]
\end{egg}

\begin{egg}[\eng]
Déterminons si la série
\[
-3+\frac{6}{5} -\frac{12}{25} + \frac{24}{125} - \ldots =
\sum_{n=0}^\infty (-3)\, \left(\frac{-2}{5}\right)^n
\]
converge.  Si elle converge, nous voulons sa valeur.  Que pouvons-nous
dire au sujet de la série
\begin{equation}\label{shiftseries}
\sum_{n=3}^\infty (-3)\left(\frac{-2}{5}\right)^n \ ?
\end{equation}

La série $\displaystyle \sum_{n=0}^\infty \left(\frac{-2}{5}\right)^n$ est une
série géométrique de raison $r = -2/5$.  Cette série converge car
$|r|<1$.  Puisque 
\[
\sum_{n=0}^\infty \left(\frac{-2}{5}\right)^n = \frac{1}{1-(-2/5)}
= \frac{5}{7} \; ,
\]
nous avons
\[
\sum_{n=0}^\infty (-3)\left(\frac{-2}{5}\right)^n
= -3 \sum_{n=0}^\infty \left(\frac{-2}{5}\right)^n
= -3 \, \frac{5}{7} = \frac{-15}{7} \; .
\]

Pour étudier la série (\ref{shiftseries}), posons
\[
S_k = \sum_{n=0}^k (-3)\left(\frac{-2}{5}\right)^n \quad \text{et}
\quad  T_k = \sum_{n=3}^k (-3)\left(\frac{-2}{5}\right)^n \; .
\]
Pour $k\geq 3$, nous avons
\[
T_k = S_k - \sum_{n=0}^2 (-3)\left(\frac{-2}{5}\right)^n
= S_k + 3 - \frac{6}{5} + \frac{12}{25} \; .
\]
Ainsi,
\begin{align*}
 \sum_{n=3}^\infty (-3)\left(\frac{-2}{5}\right)^n &=
\lim_{k\rightarrow \infty} T_k = \lim_{k\rightarrow \infty}
\left(S_k + 3 - \frac{6}{5} + \frac{12}{25} \right) \\
& = \left(\lim_{k\rightarrow \infty} S_k\right)
+ 3 - \frac{6}{5} + \frac{12}{25}
= \left( \sum_{n=0}^\infty (-3)\left(\frac{-2}{5}\right)^n \right)
+ 3 - \frac{6}{5} + \frac{12}{25} \\
& = \frac{-15}{7} + 3 - \frac{6}{5} + \frac{12}{25}
= \frac{24}{175} \; .
\end{align*}
\end{egg}

\begin{egg}[\eng][Série télescopique]
Déterminons si la série suivante converge et, si elle converge,
trouvons sa valeur.
\begin{equation}\label{telescope}
\sum_{n=3}^\infty \frac{2}{n(n-1)} \; .
\end{equation}

À l'aide de la méthode des {\em fractions partielles}, nous obtenons
\[
\frac{2}{n(n-1)} = \frac{2}{n-1} - \frac{2}{n}
\]
pour tout $n \geq 3$.  Ainsi, les sommes partielles de la série
(\ref{telescope}) sont
\begin{align*}
S_k &= \sum_{n=3}^k \frac{2}{n(n-1)}
= \sum_{n=3}^k \left( \frac{2}{n-1} - \frac{2}{n} \right) \\
&= \left( \frac{2}{2} - \frac{2}{3} \right) +
\left( \frac{2}{3} - \frac{2}{4} \right) +
\left( \frac{2}{4} - \frac{2}{5} \right) + \ldots
+\left( \frac{2}{k-1} - \frac{2}{k} \right) \\
&= 1 - \frac{2}{k}
\end{align*}
pour tout $k\geq 3$.  Le premier terme à l'intérieure d'une parenthèse est
annulé par le deuxième terme à l'intérieure de la parenthèse précédente sauf
pour le premier terme de la première parenthèse.  Nous avons donc
\[
\sum_{n=3}^\infty \frac{2}{n(n-1)} = \lim_{k\rightarrow \infty} S_k
= 1 - \lim_{k\rightarrow \infty} \frac{2}{k} = 1 \; .
\]

Les séries pour lesquelles les sommes partielles peuvent être réduites
grâce à une procédure permettant d'annuler une partie d'un terme à
l'aide d'une partie du terme suivant comme nous avons fait ci-dessus
(une procédure qui rappelle le mécanisme utilisé pour ranger les
télescopes rétractiles) sont appelées des {\bfseries séries télescopiques}.
\index{Série!télescopique}
\end{egg}

\begin{egg}[\theory\ \eng]
Montrons que la série harmonique
$\displaystyle \sum_{n=1}^\infty \frac{1}{n}$ diverge.

Les sommes partielles sont
$\displaystyle S_n = \sum_{j=1}^n \frac{1}{j}$.
Montrons par induction que
\[
\lim_{k\rightarrow \infty} S_{2^k} = +\infty \; .
\]
Ainsi, la suite des sommes partielles $\{S_n\}_{n=1}^\infty$ ne peut
pas tendre vers un nombre réel $S$ car c'est une suite de termes
croissants dont les termes $S_{2^k}$ tendent vers l'infini.

Montrons par induction que
\begin{equation}\label{harmInduct}
S_{2^k} \geq 1 + \frac{k}{2}
\end{equation}
pour $k=1$, $2$, $3$, \ldots.

Puisque $S_2 = 1 + 1/2$, (\ref{harmInduct}) est vrai pour
$k=1$.  Supposons que l'inégalité (\ref{harmInduct}) soit vraie et
montrons qu'elle est aussi vraie si nous remplaçons $k$ par $k+1$.  Nous avons
\begin{align*}
S_{2^{k+1}} &= S_{2^k} + \frac{1}{2^k+1} + \frac{1}{2^k+2} + \ldots
+ \frac{1}{2^{k+1}} \\
&= S_{2^k} + \underbrace{\frac{1}{2^k+1} + \frac{1}{2^k+2} + \ldots
+ \frac{1}{2^k+2^k}}_{2^k \text{ termes } \geq 1/2^{k+1}} \\
&\geq S_{2^k} + \frac{2^k}{2^{k+1}}
\geq \left(1+\frac{k}{2}\right) + \frac{1}{2}
= 1 + \frac{k+1}{2} \; .
\end{align*}
Ce qui donne (\ref{harmInduct}) si nous remplaçons $k$ par $k+1$.

Nous avons donc démontré par induction que $S_{2^k} \geq 1 + k/2 \geq 0$
pour $k=1$, $2$, $3$, \ldots.  Puisque la suite
$\{1+k/2\}_{k=1}^\infty$ tend vers plus l'infini, il en est de même
pour la suite $\{S_{2^k}\}_{k=1}^\infty$.
\end{egg}

\begin{focus}{\prp}
La série $\displaystyle \sum_{n=1}^\infty \frac{1}{n^p}$ converge si
$p>1$ et diverge si $p\leq 1$.
\label{Pseries}
\end{focus}

\begin{rmk}[\eng]
Une démonstration élémentaire de ce théorème existe.  Cette
démonstration ne sera pas donnée mais, à sa place, le théorème
précédent sera démontré à l'aide du {\em test de l'intégrale} qui sera
énoncé à la section~\ref{integralTest} du chapitre sur les
applications de l'intégrale.
\end{rmk}

\begin{focus}{\thm}
Si la série $\displaystyle \sum_{n=1}^\infty a_n$
converge, alors $\displaystyle \lim_{n\rightarrow \infty} a_n = 0$.
\label{condTC}
\end{focus}

\begin{proof}[\theory]
Pour démontrer le théorème précédent, posons
$\displaystyle S = \sum_{n=1}^\infty a_n$.  Puisque les suites
$\{S_n\}_{n=1}^\infty$ et $\{S_{n+1}\}_{n=1}^\infty$ de
sommes partielles tendent vers $S$, il découle du
théorème~\ref{suite3} que la suite
$\displaystyle \{S_{n+1}-S_n\}_{n=1}^\infty = \{ a_{n+1} \}_{n=1}^\infty
= \{ a_n \}_{n=2}^\infty$
tend vers $S-S=0$.
\end{proof}

\begin{rmk}
Il est {\bfseries faux} de dire que
$\displaystyle \lim_{n\rightarrow \infty} a_n = 0$ implique que la
série $\displaystyle \sum_{n=1}^\infty a_n$ converge.  La série
harmonique est la série $\displaystyle \sum_{n=1}^\infty a_n$ où
$a_n = 1/n$.  Il est vrai que
$\displaystyle \lim_{n\rightarrow \infty} a_n = 0$ mais la série
harmonique diverge.
\end{rmk}

\subsection{Tests de convergence \eng}\label{conv_tests}

Nous présentons des méthodes pour déterminer si une série converge ou
diverge.  Il faut démontrer qu'une série converge avant d'essayer
d'estimer la limite de ses sommes partielles (soit analytiquement ou
numériquement à l'aide d'un ordinateur).  Il y a peu de séries pour
lesquelles nous possédons une formule pour déterminer exactement la somme,
il faut donc bien souvent utiliser un ordinateur pour estimer la
valeur des séries convergentes.

Le critère de convergence suivant découle du théorème~\ref{condTC}.

\begin{focus}{\thm}
Si la suite des termes $\displaystyle \{a_n\}_{n=1}^\infty$ ne tend
pas vers $0$, alors la série $\displaystyle \sum_{n=1}^\infty a_n$
diverge.
\label{NCforCofS}
\end{focus}

\begin{egg}
La série $\displaystyle \sum_{n=1}^\infty \frac{\sqrt{n^2+1}}{n+1}$
diverge car
\[
\lim_{n\rightarrow \infty} \frac{\sqrt{n^2+1}}{n+1}
= \lim_{n\rightarrow \infty} \frac{\sqrt{1+1/n^2}}{1+1/n}
= \frac{\sqrt{\displaystyle 1+ \lim_{n\rightarrow \infty}1/n^2}}
{\displaystyle 1+ \lim_{n\rightarrow \infty}1/n} = 1 \neq 0 \; .
\]
\end{egg}

\begin{egg}
Démontrons que la série
$\displaystyle \sum_{n=1}^\infty \frac{7^{2n+2}}{5^{2n-1}}$
diverge.

Les termes de la série sont
$\displaystyle a_n = \frac{7^{2n+2}}{5^{2n-1}} = 5\times 7^2 \times
\left(\frac{49}{25}\right)^n$.   Ainsi,
$\lim_{n\rightarrow \infty} a_n \neq 0$ et la série diverge.

Nous aurions aussi pu dire que la série est une série géométrique de
raison $49/25 > 1$.  Comme la raison est plus grande que $1$ en valeur
absolue, la série diverge.
\end{egg}

\begin{egg}
Démontrons que la série
$\displaystyle \sum_{n=1}^\infty \arctan(n)$ diverge.

Les termes de la série sont $\displaystyle a_n = \arctan(n)$.
Puisque $\lim_{n\rightarrow \infty} a_n = \pi/2 \neq 0$, la série diverge.
\end{egg}

La théorème suivant est une conséquence du théorème~\ref{suiteBORN}.

\begin{focus}[][Test de comparaison]{\thm} \label{comp_theo}
Soit $\displaystyle \{a_n\}_{n=1}^\infty$ et
$\displaystyle \{b_n\}_{n=1}^\infty$, deux suites de termes positifs
ou nuls.
\begin{enumerate}
\item Si la série $\displaystyle \sum_{n=1}^\infty b_n$ converge et
$a_n\leq b_n$ pour tout $n$, alors $\displaystyle \sum_{n=1}^\infty a_n$
converge.
\item Si la série $\displaystyle \sum_{n=1}^\infty b_n$ diverge et
$a_n\geq b_n$ pour tout $n$, alors $\displaystyle \sum_{n=1}^\infty a_n$
diverge.
\end{enumerate}
\end{focus}

\begin{egg}
La série $\displaystyle \sum_{n=1}^\infty \frac{n}{\sqrt{n^5+1}}$
converge car
\[
\frac{n}{\sqrt{n^5+1}} < \frac{1}{n^{3/2}} \quad , \quad n=1,2,3,\ldots
\]
et la série $\displaystyle \sum_{n=1}^\infty \frac{1}{n^{3/2}}$
converge grâce à la proposition~\ref{Pseries} avec $p = 3/2 > 1$.

La série $\displaystyle \sum_{n=1}^\infty \frac{2}{1+n^{1/3}}$ diverge car
\[
\frac{2}{1+n^{1/3}} \geq \frac{2}{n^{1/3}+n^{1/3}} =\frac{1}{n^{1/3}}
\quad , \quad n=1, 2, 3,\ldots
\]
et la série $\displaystyle \sum_{n=1}^\infty \frac{1}{n^{1/3}}$ diverge grâce
à la proposition~\ref{Pseries} avec $p =1/3 < 1$.
\end{egg}

\begin{egg}
Déterminons si la séries $\displaystyle \sum_{n=1}^\infty \frac{1}{n 3^n}$
converge ou diverge.

Puisque $\displaystyle 0 < \frac{1}{n 3^n} \leq \frac{1}{3^n}$
pour tout $n>0$ et $\displaystyle \sum_{n=1}^\infty \frac{1}{3^n}$
converge (série géométrique de raison $1/3$), nous en déduisons à partir du
test de comparaison que $\displaystyle \sum_{n=1}^\infty \frac{1}{n 3^n}$
converge.
\end{egg}

\begin{egg}
Déterminons si la série $\displaystyle \sum_{n=1}^\infty \frac{1}{5n^4 + 4}$
converge ou diverge.

Puisque $\displaystyle 0 < \frac{1}{5n^4 + 4} \leq \frac{1}{5n^4}$ pour
tout $n>0$ et $\displaystyle \sum_{n=1}^\infty \frac{1}{5n^4}$
est une série convergente (proposition~\ref{Pseries} avec $p = 4>1$), nous
obtenons du test de comparaison que
$\displaystyle \sum_{n=1}^\infty \frac{1}{5n^4 + 4}$ converge.
\end{egg}

\begin{egg}
Déterminons si la série
$\displaystyle \sum_{n=1}^\infty \frac{\sqrt{n^3+2}}{(5n^5+n^4)^{1/3}}$
converge ou diverge.

Nous avons que
\[
\frac{\sqrt{n^3+2}}{(5n^5+n^4)^{1/3}}
= \frac{n^{3/2}(1+2/n^3)^{1/2}}{n^{5/3}(5+1/n)^{1/3}}
= \frac{(1+2/n^3)^{1/2}}{n^{1/6}(5+1/n)^{1/3}}
\geq \frac{1}{6^{1/3} n^{1/6}} \ .
\]
pour tout $n>0$ car $1 + 2/n^3 > 1$ et $5+1/n < 6$ pour tout $n>0$.
De plus, $\displaystyle \sum_{n=1}^\infty \frac{1}{6^{1/3} n^{1/6}}
= \frac{1}{6^{1/3}} \sum_{n=1}^\infty \frac{1}{n^{1/6}}$
est une série qui diverge grâce à la proposition~\ref{Pseries} avec
$p=1/6 <1$, nous en
déduisons à partir du test de comparaison que
$\displaystyle \sum_{n=1}^\infty \frac{\sqrt{n^3+2}}{(5n^5+n^4)^{1/3}}$
diverge.
\end{egg}

\begin{focus}[][Test de comparaison à la limite]{\thm}
Soit $\displaystyle \{a_n\}_{n=1}^\infty$ et
$\displaystyle \{b_n\}_{n=1}^\infty$, deux suites de termes positifs.
\begin{enumerate}
\item Si $\displaystyle \lim_{n\rightarrow \infty} \frac{a_n}{b_n} = c
\in ]0,\infty[$, alors la série $\displaystyle \sum_{n=1}^\infty a_n$
converge si et seulement si la séries 
$\displaystyle \sum_{n=1}^\infty b_n$ converge.
\item \label{compLT_lbl1}
Si $\displaystyle \lim_{n\rightarrow \infty} \frac{a_n}{b_n} = 0$, alors
$\displaystyle \sum_{n=1}^\infty b_n$ converge implique que
$\displaystyle \sum_{n=1}^\infty a_n$ converge.
\end{enumerate}
\label{comp_lim_theo}
\end{focus}

\begin{rmkList}
\begin{enumerate}
\item L'énoncer de l'item~\ref{compLT_lbl1} du
théorème~\ref{comp_lim_theo} peut être reformulé de la façon
suivante.  Si
$\displaystyle \lim_{n\rightarrow \infty} \frac{a_n}{b_n} = 0$, alors 
$\displaystyle \sum_{n=1}^\infty a_n$ diverge implique que
$\displaystyle \sum_{n=1}^\infty b_n$ diverge.
\item Si $\displaystyle \lim_{n\rightarrow \infty} \frac{a_n}{b_n} = \infty$,
il suffit d'inverser les rôles de $a_n$ et $b_n$ pour pouvoir utiliser
l'item~\ref{compLT_lbl1} du théorème~\ref{comp_lim_theo}.
\end{enumerate}
\end{rmkList}

\begin{egg}
La série $\displaystyle \sum_{n=1}^\infty \frac{1}{1+3^n}$ converge
car la série $\displaystyle \sum_{n=1}^\infty \frac{1}{3^n}$ converge
(c'est une série géométrique de raison $1/3$) et
\[
\lim_{n\rightarrow \infty} \frac{1/3^n}{1/(1+3^n)} 
= \lim_{n\rightarrow \infty} \left( 1 + \frac{1}{3^n}\right) 
= 1 + \lim_{n\rightarrow \infty}\frac{1}{3^n} = 1 \in ]0,\infty[ \; .
\]

Par contre, la série $\displaystyle \sum_{n=1}^\infty \frac{\ln(n)}{\sqrt{n}}$
diverge car la série $\displaystyle \sum_{n=1}^\infty \frac{1}{\sqrt{n}}$
diverge (le cas $p=1/2$ de la proposition~\ref{Pseries}) et
\[
\lim_{n\rightarrow \infty} \frac{1/\sqrt{n}}{\ln(n)/\sqrt{n}} 
= \lim_{n\rightarrow \infty} \frac{1}{\ln(n)} = 0
\]
où nous supposons que $n>1$ pour ne pas avoir de division par zéro.
\end{egg}

\begin{focus}[][Test de d'Alembert ou du quotient]{\thm} \label{Alembert}
Soit $\displaystyle \sum_{n=1}^\infty a_n$ une série de termes
positifs.  Si la suite $\displaystyle \left\{ \frac{a_{n+1}}{a_n}
\right\}_{n=1}^\infty$ converge et
$\displaystyle L = \lim_{n\rightarrow \infty} \frac{a_{n+1}}{a_n}$
($L$ peut être $+\infty$), alors
\begin{enumerate}
\item la série $\displaystyle \sum_{n=1}^\infty a_n$ converge si $L<1$ et
\item la série $\displaystyle \sum_{n=1}^\infty a_n$ diverge si $L>1$.
\end{enumerate}
\end{focus}

\begin{rmk}
Si $L=1$ au théorème précédent, nous ne pouvons rien dire au sujet de la
série $\displaystyle \sum_{n=1}^\infty a_n$.

Par exemple, les séries $\displaystyle \sum_{n=1}^\infty \frac{1}{n}$
et $\displaystyle \sum_{n=1}^\infty \frac{1}{n^2}$ satisfont
$\displaystyle \lim_{n\rightarrow \infty} \frac{a_{n+1}}{a_n} = 1$.
La première série diverge (c'est la série harmonique) alors que la
deuxième série converge (c'est le cas $p=2$ de la
proposition~\ref{Pseries}).
\label{dAlembert1}
\end{rmk}

\begin{egg}
La série $\displaystyle \sum_{n=1}^\infty a_n$ avec
$\displaystyle a_n = \frac{(n+2)5^n}{n 4^{3n}}$ converge car
\begin{align*}
\lim_{n\rightarrow \infty} \frac{a_{n+1}}{a_n} &=
\lim_{n\rightarrow \infty} \left( \frac{(n+3)5^{n+1}}{(n+1) 4^{3(n+1)}}
\right) \left( \frac{(n+2)5^n}{n 4^{3n}} \right)^{-1} \\
&=\lim_{n\rightarrow \infty} \frac{5(n^2+3n)}{4^3(n^2+3n+2)}
=\lim_{n\rightarrow \infty} \frac{5(1+3/n)}{4^3(1+3/n+2/n^2)}
= \frac{5}{4^3} < 1 \; .
\end{align*}
\end{egg}

\begin{egg}
Montrons que la série
$\displaystyle \sum_{n=1}^\infty \frac{n!}{(2n+1)!}$ converge.

Le nombre $n!$,
{\bfseries prononcé $\mathbf n$ factoriel}\index{Factoriel},
est le nombre défini par $n! = n(n-1)(n-2)\ldots 1$
si $n$ est un entier positif et $n! = 1$ si $n=0$.

Utilisons le test de d'Alembert pour démontrer que cette
série converge.   Nous avons la série $\displaystyle \sum_{n=1}^\infty a_n$ où
$\displaystyle a_n = \frac{n!}{(2n+1)!}$.  Puisque
\begin{align*}
\lim_{n\rightarrow \infty} \frac{a_{n+1}}{a_n}
&= \lim_{n\rightarrow \infty} \frac{(n+1)!}{(2(n+1)+1)!}
\left(\frac{n!}{(2n+1)!}\right)^{-1}
= \lim_{n\rightarrow \infty}
\frac{(n+1)!}{n!}\;\frac{(2n+1)!}{(2n+3)!} \\
&= \lim_{n\rightarrow \infty} \frac{n+1}{(2n+2)(2n+3)}
= \lim_{n\rightarrow \infty} \frac{1/n + 1/n^2}{4 + 10/n + 6/n^2} = 0
<1 \; ,
\end{align*}
la série converge.
\end{egg}

\begin{egg}
Déterminons si la série
$\displaystyle \sum_{n=1}^\infty \frac{n^n}{(2n)!}$ converge ou diverge.

Utilisons le test du quotient.  Puisque
\begin{align*}
\lim_{n\to\infty} \left(\frac{(n+1)^{n+1}}{(2(n+1))!}\right)
\left(\frac{n^n}{(2n)!}\right)^{-1}
& = \lim_{n\to\infty} \left( 1 + \frac{1}{n}\right)^n
\frac{n+1}{(2n+1)(2n+2)}  \\
& = \lim_{n\to\infty} \left( 1 + \frac{1}{n}\right)^n
\frac{1/n + 1/n^2}{(4 + 6/n + 2/n^2}  \\
& = e \times 0 = 0 < 1 \ ,
\end{align*}
la série $\displaystyle \sum_{n=1}^\infty \frac{n^n}{(2n)!}$ converge.

Le fait que
$\displaystyle \lim_{n\to\infty} \left( 1 + \frac{1}{n}\right)^n = e$
provient du théorème~\ref{limitDefe} que nous verrons prochainement.
\end{egg}

\begin{focus}[][Test de la racine]{\thm} \label{roottest}
Soit $\displaystyle \sum_{n=1}^\infty a_n$ une série de termes
positifs.  Si la suite
$\displaystyle \left\{ \sqrt[n]{a_n} \right\}_{n=1}^\infty$ converge
et $\displaystyle L = \lim_{n\rightarrow \infty} \sqrt[n]{a_n}$
($L$ peut être $+\infty$), alors
\begin{enumerate}
\item la série $\displaystyle \sum_{n=1}^\infty a_n$ converge si $L<1$ et
\item la série $\displaystyle \sum_{n=1}^\infty a_n$ diverge si $L>1$.
\end{enumerate}
\end{focus}

\begin{rmk}
Comme pour le Test de d'Alembert, nous ne pouvons rien dire au sujet
de la série
$\displaystyle \sum_{n=1}^\infty a_n$ si $L=1$ dans le théorème précédent. 

Par exemple, si nous utilisons le fait que la suite
$\{ \sqrt[n]{n} \}_{n=1}^\infty$ tend vers $1$, nous pouvons
facilement montrer que les séries
$\displaystyle \sum_{n=1}^\infty \frac{1}{n}$ et
$\displaystyle \sum_{n=1}^\infty \frac{1}{n^2}$ satisfont
$\displaystyle \lim_{n\rightarrow \infty} \sqrt[n]{a_n} = 1$.
Comme à la remarque~\ref{dAlembert1}, la première série diverge alors
que la deuxième série converge.
\end{rmk}

\begin{egg}
La série $\displaystyle \sum_{n=2}^\infty a_n$ avec
$\displaystyle a_n = \frac{1}{(\ln(n))^n}$ converge car
\[
\lim_{n\rightarrow \infty} \sqrt[n]{\frac{1}{(\ln(n))^n}} =
\lim_{n\rightarrow \infty} \sqrt[n]{\left(\frac{1}{\ln(n)}\right)^n} =
\lim_{n\rightarrow \infty} \frac{1}{\ln(n)} = 0 < 1 \; .
\]
\end{egg}

Un autre test pour déterminer si une série de termes positifs converge
est le {\em test de l'intégrale}.  Ce test sera présenté à la
section~\ref{integralTest} du chapitre sur les applications de
l'intégrale d'une fonction.

\subsection{Convergence absolue et séries alternées \eng}\label{serie_alt}

Jusqu'à maintenant, la majorité des séries que nous avons considérées
ne possédaient que des termes positifs.  Nous allons maintenant
étudier un type particulier de séries avec des termes positifs et
négatifs.

\begin{focus}{\dfn} \index{Série!alternée}
La série $\displaystyle \sum_{n=1}^\infty a_n$ est une
{\bfseries série alternée} si $a_n a_{n+1} < 0$
pour tout $n$.
\end{focus}

\begin{egg}
La série $\displaystyle \sum_{n=1}^\infty \frac{(-1)^n}{n}$ est une
série alternée.  C'est une série de la forme
$\displaystyle \sum_{n=1}^\infty a_n$ où
$\displaystyle a_n = \frac{(-1)^n}{n}$.  Ainsi,
$a_n >0$ pour $n$ pair et $a_n <0$ pour $n$ impair.  Nous avons
\[
\sum_{n=1}^\infty \frac{(-1)^n}{n} = -1 + \frac{1}{2} - \frac{1}{3} +
\frac{1}{4} - \ldots
\]
\end{egg}

\begin{focus}[][Test des séries alternées]{\thm}
Si la série $\displaystyle \sum_{n=1}^\infty (-1)^n a_n$ satisfait les
conditions suivantes alors elle converge.
\begin{enumerate}
\item $a_n >0$ pour tout $n$,
\item $a_{n+1} \leq a_n$ pour tout $n$ et
\item $\displaystyle \lim_{n\rightarrow \infty} a_n = 0$.
\end{enumerate}
De plus, les sommes partielles
$\displaystyle S_n = \sum_{k=1}^n (-1)^k a_k$ satisfont la relation
\begin{equation}\label{altError}
 | S - S_n | \leq a_{n+1}
\end{equation}
où $\displaystyle S = \lim_{n\rightarrow \infty} S_n
= \sum_{n=1}^\infty (-1)^n a_n$.
\label{altTest}
\end{focus}

\begin{proof}[\theory]
Nous démontrons ce théorème car la démonstration nous permet de mieux
comprendre la relation (\ref{altError}).

Puisque $S_{2(k+1)} - S_{2k} = a_{2k+2} - a_{2k-1} \leq 0$ pour tout
$k\geq 1$, nous obtenons
\begin{equation}\label{alt1}
S_{2(k+1)} \leq S_{2k} \quad , \quad k\geq 1 \ .
\end{equation}
Donc
\[
\ldots \leq S_6 \leq S_4 \leq S_2 \; .
\]
De même, puisque $S_{2k+1} - S_{2k-1} = a_{2k} - a_{2k+1} \geq 0$ pour
tout $k\geq 1$, nous obtenons
\begin{equation}\label{alt2}
S_{2k+1} \geq S_{2k-1} \quad , \quad  k\geq 1 \ .
\end{equation}
Donc
\[
S_1 \leq S_3 \leq S_5 \leq \ldots
\]
De plus
\begin{equation}\label{alt3}
S_{2k-1} \leq S_{2k-1} + a_{2k} = S_{2k} \quad , \quad k \geq 1 \ .
\end{equation}

Nous déduisons de (\ref{alt1}), (\ref{alt2}) et (\ref{alt3}) que
\begin{equation}\label{alt4}
S_q \leq S_p
\end{equation}
pour $p$ pair et $q$ impair.  En effet, si $q < p$, alors
\[
S_p \geq S_{p-1} \geq S_q \; .
\]
La première inégalité provient de (\ref{alt3}) et la deuxième de
(\ref{alt2}).  Si par contre $q>p$, alors
\[
S_q \leq S_{q+1} \leq S_p \; .
\]
La première inégalité provient de (\ref{alt3}) et la deuxième de
(\ref{alt1}).

Puisque la suite $\{S_{2k}\}_{k=1}^\infty$ est décroissante et bornée
inférieurement par $S_q$ où $q$ est impair, il découle du
théorème~\ref{suiteBORN} que cette suite converge.  Soit
\[
B = \lim_{k\rightarrow \infty} S_{2k} \; .
\]
De même, puisque la suite $\{S_{2k-1}\}_{k=1}^\infty$ est croissante et
bornée supérieurement par $S_p$ où $p$ est pair, il découle du
théorème~\ref{suiteBORN} que cette suite converge.  Soit
\[
A = \lim_{k\rightarrow \infty} S_{2k-1} \; .
\]
Puisque $B \geq S_q$ pour tout $q$ impair grâce à (\ref{alt4}), nous
déduisons que $B \geq A$.
% De même, nous avons $A \leq S_p$ pour tout $p$ pair grâce à (\ref{alt4}).

Puisque la suite $\{S_{2k-1}\}_{k=1}^\infty$ tend vers $A$ et la
suite $\{S_{2k}\}_{k=1}^\infty$ tend vers $B$, nous avons que
\[
B-A = \lim_{k\rightarrow \infty} S_{2k} - \lim_{k\rightarrow \infty} S_{2k-1}
= \lim_{k\rightarrow \infty} \left( S_{2k} - S_{2k-1} \right)
= \lim_{k\rightarrow \infty} a_{2k} =0 \; .
\]
grâce à la troisième hypothèse du théorème.

Pour résumé, si nous posons $S=A=B$, nous avons montré que
\[
S_1 \leq S_3 \leq S_5 \leq \ldots \leq S \leq \ldots \leq S_6 \leq S_4
\]
et
\[
\sum_{n=1}^\infty (-1)^n a_n = \lim_{n\rightarrow \infty} S_n = S \; .
\]

Nous avons maintenant tous les ingrédients nécessaires pour démontrer la
relation (\ref{altError}).  Il découle de $S_{2k-1} \leq S \leq S_{2k}$
que
\[
0 \leq S - S_{2k-1} \leq S_{2k} - S_{2k-1} = a_{2k} \; .
\]
Ainsi, $|S- S_n| \leq a_{n+1}$ lorsque $n$ est impair.  De même, il
découle de $S_{2k+1} \leq S \leq S_{2k}$ que
\[
0 \geq S - S_{2k} \geq S_{2k+1} - S_{2k} = -a_{2k+1} \; .
\]
Ainsi, $|S-S_n| \leq a_{n+1}$ lorsque $n$ est pair.  Nous avons donc démontré
(\ref{altError}).  Nous retrouvons à la figure~\ref{ALTERN} une
représentation graphique du raisonnement que nous venons de faire.
\end{proof}

\PDFfig{3_suites_series/alterned}{Représentation graphique de l'erreur
$|S_n-S|$ pour une série alternée}{Pour la série alternée du
théorème~\ref{altTest}, si nous utilisons $S_n$ pour estimer la valeur
$S$, l'erreur $|S_n - S|$ commisse est plus petite ou égale à
$a_{n+1}$}{ALTERN}

\begin{egg}
La série $\displaystyle \sum_{n=1}^\infty \frac{(-1)^n}{n}$ converge
car c'est une série alternée avec $a_n = 1/n$ qui satisfait les trois
hypothèses du test des séries alternées.
\end{egg}

\begin{egg}
Montrons que la série
$\displaystyle \sum_{n=1}^\infty (-1)^{n-1}\frac{n}{n^2+1}$ converge,
puis cherchons une petite valeur $N$ telle que
$|S_n - S|< 10^{-3}$ si $n\geq N$ où $S_n$ est la $n^e$ somme
partielle de la série et $S$ est la somme de la série.

Nous avons
\[
\sum_{n=1}^\infty (-1)^{n-1}\frac{n}{n^2+1} =
-\sum_{n=1}^\infty (-1)^n\frac{n}{n^2+1}
\]
où la série de droite est une série alternée avec $a_n = n/(n^2+1)$.
Puisque $a_n = n/(n^2+1) >0$,
\[
\lim_{n\rightarrow \infty} a_n =
\lim_{n\rightarrow \infty} \frac{n}{n^2+1} =
\lim_{n\rightarrow \infty} \frac{1/n}{1+1/n^2} = 0
\]
et
\begin{equation}\label{inequAlt}
a_{n+1} < a_n
\end{equation}
pour tout $n$, la série
$\displaystyle \sum_{n=1}^\infty (-1)^n\frac{n}{n^2+1}$ et
donc la série
$\displaystyle \sum_{n=1}^\infty (-1)^{n-1}\frac{n}{n^2+1}$ converge.
Notons que (\ref{inequAlt}) provient de la relation
\[
(n+1)(n^2+1) = n^3 + n^2 + n + 1 < n^3 +2n^2 + n + n = n( (n+1)^2+1)
\]
pour tout $n\geq 1$.

Pour déterminer $N$ tel que $|S_n - S |<10^{-3}$ pour tout $n\geq N$,
il suffit de trouver $n$ tel que $|S_n - S |< a_{n+1} < 10^{-3}$.
Pour simplifier les calculs, nous choisissons $n$ tel que
\[
a_{n+1} = \frac{n+1}{(n+1)^2+1} < \frac{1}{n+1} < 10^{-3}
\]
et résolvons pour $n$ pour trouver $n > 10^3-1$.  Il suffit de
prendre $N=10^3$.
\end{egg}

D'autres applications du test des séries alternées sont données à
l'exemple~\ref{deriveAlt} dans le chapitre sur les applications de la
dérivée d'une fonction.

\begin{focus}{\dfn}
Une série de termes réels $\displaystyle \sum_{n=1}^\infty a_n$
{\bfseries converge absolument}\index{Série!convergence absolue}
si la série de termes positifs $\displaystyle \sum_{n=1}^\infty |a_n|$
converge.

Si la série $\displaystyle \sum_{n=1}^\infty a_n$ converge mais la
série $\displaystyle \sum_{n=1}^\infty |a_n|$ diverge, nous disons que la
série $\displaystyle \sum_{n=1}^\infty a_n$
{\bfseries converge conditionnellement}.\index{Série!convergence conditionnelle}
\end{focus}

\begin{rmk}
Le Test de comparaison, théorème~\ref{comp_theo},
le Test de comparaison à la limite, théorème~\ref{comp_lim_theo},   
le Test de d'Alembert ou du quotient, théorème~\ref{Alembert}, et le
Test de la racine, théorème~\ref{roottest}, peuvent être utilisés pour
déterminer la convergence absolue d'une série.
\end{rmk}

\begin{egg}
La série $\displaystyle \sum_{n=1}^\infty \frac{(-1)^n}{n}$ converge
car elle satisfait le test des séries alternées.  Par contre, la série
\[
\sum_{n=1}^\infty \left|\frac{(-1)^n}{n}\right| =
\sum_{n=1}^\infty \frac{1}{n}
\]
est la série harmonique qui diverge.  La série
$\displaystyle \sum_{n=1}^\infty \frac{(-1)^n}{n}$ converge donc
conditionnellement.
\end{egg}

Comme nous venons de le voir à l'exemple précédent, nous pouvons avoir une
série qui converge mais qui ne converge pas absolument.  L'inverse
n'est pas possible.

\begin{focus}{\thm}
Si une série converge absolument alors elle converge.
\end{focus}

\begin{proof}[\theory]
Soit $\displaystyle \sum_{n=1}^\infty a_n$ une série qui converge
absolument.  Posons $b_n = a_n + |a_n|$ pour $n\geq 1$.
Remarquons que
\[
b_n = \begin{cases}
0 & \quad \text{pour}\quad  a_n<0 \\
2 a_n & \quad \text{pour}\quad  a_n \geq 0
\end{cases}
\]
et $b_n - |a_n| = a_n$ pour tout $n\geq 1$.

Puisque la série $\displaystyle \sum_{n=1}^\infty |a_n|$ converge, il
en est de même pour la
série $\displaystyle \sum_{n=1}^\infty 2 |a_n|$.  De 
plus, puisque $0 \leq b_n \leq 2 |a_n|$ pour tout $n\geq 1$ et la
série $\displaystyle \sum_{n=1}^\infty 2 |a_n|$ converge, nous avons par le
test de comparaison que la série $\displaystyle \sum_{n=1}^\infty b_n$
converge.

Finalement, puisque les séries $\displaystyle \sum_{n=1}^\infty |a_n|$
et $\displaystyle \sum_{n=1}^\infty b_n$ convergent, la série
$\displaystyle \sum_{n=1}^\infty (b_n - |a_n|) = \sum_{n=1}^\infty a_n$
converge.
\end{proof}

\begin{egg}
La série $\displaystyle \sum_{n=1}^\infty \frac{\sin(n)}{n^3}$
converge car elle converge absolument.  En effet,
\[
0 \leq \left| \frac{\sin(n)}{n^3} \right|
= \frac{|\sin(n)|}{n^3} \leq \frac{1}{n^3}
\]
et la série $\displaystyle \sum_{n=1}^\infty \frac{1}{n^3}$ converge
grâce à la proposition~\ref{Pseries}.  Il découle du test de comparaison
que la série
$\displaystyle \sum_{n=1}^\infty \left| \frac{\sin(n)}{n^3} \right|$
converge.
\end{egg}

\section{Le nombre $\mathbf{e}$ et la fonction $\mathbf{e^x}$}\label{nbrE}

% Bug in tcolorbox : The text before the box is needed.

Voici la première définition rigoureuse du nombre $e$.  Une autre
définition équivalente à celle-ci sera donné au chapitre sur les
séries entières.

\begin{focus}{\thm}
La suite
$\displaystyle \left\{\left(1+\frac{1}{n}\right)^n\right\}_{n=1}^\infty$ 
converge.  Sa limite est définie comme étant le {\bfseries nombre
d'Euler}\index{Nombre d'Euler}
\[
e=2.718281828459\ldots
\]
\label{limitDefe}
\end{focus}

\begin{proof}[\theory]
Pour démontrer que cette limite existe, nous utilisons la formule
suivante, connue sous le nom de
{\bfseries formule du binôme}\index{formule du binôme}, 
qui permet de développer une expression de la forme $(a+b)^n$ où $n$
est un entier positif.
\[
(a+b)^n = a^n + n a^{n-1}b + \frac{n(n-1)}{2}a^{n-2}b^2 + \ldots
+ n a b^{n-1} + b^n \\
= \sum_{k=0}^n \binom{n}{k} a^{n-k} b^k
\]
où $\displaystyle \binom{n}{k} = \frac{n!}{k!(n-k)!}$.

Par exemple, $\displaystyle (a+b)^2 = a^2 + 2ab + b^2$
et $\displaystyle (a+b)^3 = a^3 + 3a^2b + 3ab^2 + b^3$.

Montrons que la suite
$\displaystyle \left\{\left(1+\frac{1}{n}\right)^n\right\}_{n=1}^\infty$ 
converge.

Premièrement, nous montrons que la suite est bornée supérieurement.  Grâce
à la formule du binôme, nous avons
\[
\left( 1 + \frac{1}{n} \right)^n = \sum_{k=0}^n \binom{n}{k}
\left(\frac{1}{n}\right)^k \; .
\]
Puisque
\begin{align*}
\binom{n}{k} \left(\frac{1}{n}\right)^k
&= \frac{n(n-1)(n-2)\ldots(n-k+1)}{k!}\left(\frac{1}{n}\right)^k \\
&= \frac{1}{k!}\left(1-\frac{1}{n}\right)\left(1 - \frac{2}{n}\right)\ldots
\left(1-\frac{k-1}{n}\right)
< \frac{1}{k!}
\end{align*}
pour $1\leq k \leq n$ et
\[
\frac{1}{k!} = \frac{1}{1\times 2\times 3\times \ldots\times k} \leq
\frac{1}{\underbrace{2\times 2 \times 2 \ldots \times 2}_{k-1\text{ fois}}}
 = \frac{1}{2^{k-1}}
\]
pour $1\leq k \leq n$, nous avons que
\[
\left( 1 + \frac{1}{n} \right)^n 
= 1 + \sum_{k=1}^n \binom{n}{k} \left(\frac{1}{n}\right)^k
\leq 1 + \sum_{k=1}^n \frac{1}{2^{k-1}}
< 1 + \sum_{k=0}^\infty \left(\frac{1}{2}\right)^k
= 1 + \frac{1}{1-1/2} = 3 \; .
\]

Deuxièmement, montrons que la suite
$\displaystyle \left\{\left(1+\frac{1}{n}\right)^n\right\}_{n=1}^\infty$
est croissante.  Nous avons que
\begin{align*}
\left(1+\frac{1}{n+1}\right)^{n+1}\bigg/
\left(1+\frac{1}{n}\right)^n
&= \left( \left(1+\frac{1}{n+1}\right)\left(\frac{n}{n+1}\right) \right)^n
\left(1+\frac{1}{n+1}\right) \\
&= \left( \frac{n^2+2n}{(n+1)^2} \right)^n
\left(1+\frac{1}{n+1}\right) \\
&= \left(1- \frac{1}{(n+1)^2} \right)^n \left(1+\frac{1}{n+1}\right) \; .
\end{align*}
De plus, il découle de la formule du binôme que
\begin{align*}
\left(1- \frac{1}{(n+1)^2} \right)^n &=
1 - \frac{n}{(n+1)^2} + \underbrace{\frac{n(n-1)}{2(n-1)^4} -
\frac{n(n-1)(n-2)}{3!(n-1)^6}}_{>0} + \ldots \\
& + \underbrace{\frac{n}{(n-1)^{2(n-1)}} - \frac{1}{(n-1)^{2n}}}_{>0}
> 1 - \frac{n}{(n+1)^2}
\end{align*}
si $n$ est impair et
\begin{align*}
\left(1- \frac{1}{(n+1)^2} \right)^n &=
1 - \frac{n}{(n+1)^2} + \underbrace{\frac{n(n-1)}{2(n-1)^4} -
\frac{n(n-1)(n-2)}{3!(n-1)^6}}_{>0} + \ldots \\
&+ \underbrace{\frac{n(n-1)}{2(n-1)^{2(n-2)}} -
\frac{n}{(n-1)^{2(n-1)}}}_{>0} + \underbrace{\frac{1}{(n-1)^{2n}}}_{>0}
> 1 - \frac{n}{(n+1)^2}
\end{align*}
si $n$ est pair.  Ainsi,
\begin{align*}
\left(1+\frac{1}{n+1}\right)^{n+1}\bigg/
\left(1+\frac{1}{n}\right)^n
&> \left(1 - \frac{n}{(n+1)^2}\right)\left(1+\frac{1}{n+1}\right) \\
&= \frac{n^3+3n^3+3n+2}{n^3+3n^2+3n+1}>1 \; .
\end{align*}
La convergence de la série
$\displaystyle \left\{\left(1+\frac{1}{n}\right)^n\right\}_{n=1}^\infty$
est donc une conséquence du théorème~\ref{suiteBORN}
\end{proof}

\begin{rmk}
Une autre façon équivalente de définir le nombre $e$ est donnée
lors de l'étude de la dérivée de fonctions exponentielles.
\end{rmk}

\begin{rmk}[\theory]
Comme nous avons annoncé à la section~\ref{cerclevicieux},
une définition simple de $b^x$ pour $b\in]0,\infty[$ et $x\in \RR$, et
en particulier de $e^x$, sera donnée à la section~\ref{DEFofE}. 
Cette définition fera appel aux séries.
\end{rmk}

\begin{egg}
Évaluons la limite de la suite
$\displaystyle \left\{\frac{n\,e^{-n}}{n+1}\right\}_{n=1}^\infty$.

Pour ce faire, nous remarquons que la suite
$\displaystyle \{ e^{-n} \}_{n=1}^\infty$ tend vers $0$ car c'est le
cas $0< r = e^{-1}<1$ du théorème~\ref{suite1}.  Il est aussi facile
de voir à partir du graphe de la fonction exponentielle $f(x)=e^x$
à la figure~\ref{exp_N} que la limite de la suite
$\displaystyle \{ e^{-n}\}_{n=1}^\infty$ est $0$. 

Nous pouvons alors conclure que la limite de la suite
$\displaystyle \left\{\frac{n\,e^{-n}}{n+1}\right\}_{n=1}^\infty$ est
aussi $0$ car
\[
\lim_{n\rightarrow \infty} \frac{n\, e^{-n}}{n+1}
= \lim_{n\rightarrow \infty} \left( \frac{e^{-n}}{1+1/n} \right)
= \frac{\displaystyle \lim_{n\rightarrow \infty} e^{-n}}
{\displaystyle \lim_{n\rightarrow \infty} \left(1 + 1/n\right)}
= \frac{\displaystyle \lim_{n\rightarrow \infty} e^{-n}}
{\displaystyle 1 + \lim_{n\rightarrow \infty} (1/n)}
= \frac{0}{1+0} = 0 \; .
\]
\end{egg}

\PDFfig{3_suites_series/exp_n}{La suite $\{e^{-n}\}_{n=1}^\infty$}
{La suite $\{e^{-n}\}_{n=1}^\infty$ tend vers $0$}{exp_N}

\begin{egg}
Évaluons la limite
$\displaystyle \lim_{n\rightarrow \infty} \frac{e^{-n}}{n^2}$.

Comme à l'exemple précédent, nous avons que la suite
$\displaystyle \{ e^{-n} \}_{n=1}^\infty$ tend vers $0$.  De plus, la
suite $\displaystyle \{ 1/n^2 \}_{n=1}^\infty$ tend aussi vers $0$
(le cas $r=2>0$ du théorème~\ref{suite2}).  Ainsi, grâce au
théorème~\ref{suite3},
\[
\lim_{n\rightarrow \infty} \frac{e^{-n}}{n^2}
= \left(\lim_{n\rightarrow \infty} e^{-n} \right)
\left(\lim_{n\rightarrow \infty} \frac{1}{n^2}\right) = 0\times 0 = 0 \; .
\]
\end{egg}

\begin{rmk}[\theory]
En analyse réel, dont fait partie le calcul différentielle et intégral,
la fonction $e^z$ et les fonctions $\sin(z)$ et $\cos(z)$ font parties
de deux familles de fonctions que l'on étudie séparément.  Par contre,
en analyse complexe, la fonction $e^z$ et les fonctions $\sin(z)$ et
$\cos(z)$ sont intimement reliées.  Donc, les nombres $e$ et $\pi$
sont intimement reliés.  C'est un sujet fascinant pour les  
étudiant.e.s qui poursuivront leurs études en mathématique.
\end{rmk}

\subsection{Fonctions hyperboliques \eng}

Le but de cette section est de fournir une brève description des
fonctions hyperboliques pour ceux qui poursuivront des études plus
avancées des mathématiques pour l'ingénierie.  Ces fonctions ne
joueront aucun rôle dans le reste de ce manuel.

Les {\bfseries fonctions hyperboliques} \index{Fonctions hyperboliques}
sont:
\begin{enumerate}
\item le sinus hyperbolique $\displaystyle \sinh(x) = \frac{e^x - e^{-x}}{2}$
pour $x\in \RR$,
\item le cosinus hyperbolique $\displaystyle \cosh(x) = \frac{e^x + e^{-x}}{2}$
pour $x\in \RR$,
\item la tangente hyperbolique $\displaystyle
\tanh(x) = \frac{e^x - e^{-x}}{e^x + e^{-x}} = \frac{\sinh(x)}{\cosh(x)}$
pour $x\in \RR$,
\item la cotangente hyperbolique $\displaystyle
\coth(x) = \frac{e^x + e^{-x}}{e^x - e^{-x}} = \frac{\cosh(x)}{\sinh(x)}$
pour $x\in \RR\setminus \{0\}$,
\item la sécante hyperbolique $\displaystyle
\sech(x) = \frac{2}{e^x + e^{-x}} = \frac{1}{\cosh(x)}$ pour $x\in \RR$, et
\item la cosécante hyperbolique $\displaystyle
\csch(x) = \frac{2}{e^x - e^{-x}} = \frac{1}{\sinh(x)}$ pour
pour $x\in \RR\setminus\{0\}$.
\end{enumerate}
Ces fonctions sont fréquemment utilisées en ingénierie.  Le graphe du
sinus hyperbolique et celui du cosinus hyperbolique sont donnés à la
figure~\ref{trigHyper}

\PDFfig{3_suites_series/trigHyper}{Le graphe du sinus et du
cosinus hyperbolique}{Le graphe du sinus hyperbolique et du
cosinus hyperbolique}{trigHyper}

Les fonctions hyperboliques n'apportent rien de nouveau du point de
vue des mathématiques puisqu'elles proviennent de formules algébriques
en termes des fonctions exponentielles $e^x$ et $e^{-x}$.  Il n'en
reste pas moins qu'elle permettent de présenter plusieurs résultats de
façon élégante.

Les fonctions hyperboliques satisfont les identités suivantes:
$\displaystyle \sinh(-x) = -\sinh(x)$ (fonction impaire),
$\displaystyle \cosh(-x) = \cosh(x)$ (fonction paire),
$\displaystyle \cosh^2(x) - \sinh^2(x) = 1$,
$\displaystyle \sinh(x+y) = \sinh(x) \cosh(y) + \cosh(x)\sinh(y)$ et
biens d'autres.  Elles satisfont donc des identités semblables à
celles satisfaites par les fonctions trigonométriques.

Finalement, pour chaque fonction hyperbolique, nous pouvons définir une
fonction inverse.  Par exemple, la fonction $\sinh^{-1}$ est définie
par $\displaystyle \sinh^{-1}(x) = \ln(x+\sqrt{x^2+1})$ pour
$x\in \RR$.  En effet,
\[
y = \sinh^{-1}(x)  \Leftrightarrow y = \frac{e^x - e^{-x}}{2} 
\Leftrightarrow e^x- 2y - e^{-x} = 0 \Leftrightarrow
e^{2x}-2y e^x - 1 = 0 \ .
\]
C'est un polynôme de degré deux en $e^x$.  Donc, si nous utilisons la
formule pour trouver les racines d'une polynôme de degré deux, nous
obtenons
\[
e^x = \frac{2y + \sqrt{4y^2 +4}}{2} = y +\sqrt{y^2+1} \ .
\]
Nous ne pouvons pas utiliser la formule
$\displaystyle \frac{2y - \sqrt{4y^2 +4}}{2} = y -\sqrt{y^2+1}$ car elle
donne une valeurs négatives alors que $e^x >0$.  Nous avons donc
que $x = \ln(y+\sqrt{y^2+1})$ pour $y\in \RR$.  Pour respecter la
tradition qui veut que $y$ soit une fonction de $x$, nous échangeons
les rôles de $x$ et $y$ pour obtenir
$y = \sinh^{-1}(x) = \ln(x+\sqrt{x^2+1})$ pour $x\in \RR$.

La fonction $\cosh$ a un inverse si nous considérons seulement
$\cosh(x)$ pour $x\geq 0$.  Sur l'intervalle $[0,\infty[$, la fonction
$\cosh$ est injective et possède donc un inverse.  L'image de $\cosh$
est $[1,\infty[$.  L'inverse de $\cosh$ est donnée par
$\cosh^{-1}(x) = \ln(x+\sqrt{x^2-1})$ pour $x\geq 1$.

Finalement, La fonction $\tanh$ a un inverse si nous considérons.
La fonction $\tanh^{-1}$ est définie par
$\displaystyle \tanh^{-1}(x) = \frac{1}{2}\,\ln\left(\frac{1+x}{1-x}\right)$
pour $-1 < x < 1$.  Le domaine de $\tanh^{-1}$ est $]-1,1[$, l'image
de $\tanh$.

}  % End of theory

\section{Exercices}

\subsection{Suites}

\begin{question}
Déterminez si les suites suivantes convergent ou divergent.
Si elles convergent, calculez leurs limites.
\begin{center}
\begin{tabular}{*{2}{l@{\hspace{0.5em}}l@{\hspace{3em}}}l@{\hspace{0.5em}}l}
\subQ{a} & $\displaystyle \left\{\frac{2^n}{3^{n+1}} \right\}_{n=1}^\infty$ &
\subQ{b} & $\displaystyle \left\{ \ln(n+3) - \ln(n) \right\}_{n=1}^\infty$ &
\subQ{c} & $\displaystyle \left\{\frac{2^{n+1}}{5^{n-1}} \right\}_{n=1}^\infty$
\\[1em]
\subQ{d} & $\displaystyle \left\{ 3 + \sin\left(n\frac{\pi}{2}\right)
\right\}_{n=1}^\infty$ &
\subQ{e} & $\displaystyle \left\{ 2 + \cos(n\pi) \right\}_{n=1}^\infty$ &
\subQ{f} & $\displaystyle \left\{\frac{n^2 - n + 1}{2n^2 + 1}
\right\}_{n=0}^\infty$
\end{tabular}
\end{center}
\label{3Q1}
\end{question}

% $\displaystyle \left\{\frac{e^n}{n^2 + 1} \right\}_{n=1}^\infty$

\subsection{Séries}

\begin{question}
Nous laissons tomber une balle d'une hauteur de 10 mètres et elle
rebondit.  A chaque bond elle atteint $3/4$ de la hauteur du bond
précédent.  Ainsi, au premier bond, la balle atteint la hauteur de
$10\times(3/4)$ mètres, au deuxième bond, la balle atteint la
hauteur de $10\times(3/4)^2$ mètres, etc.\\
\subQ{a} Trouvez une expression pour la hauteur du $n^e$ bond.\\
\subQ{b} Trouvez une expression pour la distance verticale totale
parcourue par la balle lorsqu'elle frappe le sol pour la première,
deuxième, troisième et quatrième fois.\\
\subQ{c} Trouvez une expression pour la distance verticale totale
parcourue par la balle lorsqu'elle frappe le sol pour la $n^e$ fois.\\
\subQ{d} Quelle sera éventuellement la distance parcourue par la
balle?
\label{3Q2}
\end{question}

\begin{question}
Trouvez l'aire de la région en gris dans le dessin suivant.  La
région en gris est l'union d'un nombre infini de disques de rayon
décroissant.
\PDFgraph{3_suites_series/disques}
\label{3Q3}
\end{question}

\begin{question}[\eco]
Un fabricant de cerfs-volants vent $5000$ unités par année.  Chaque
année, $10$\% des cerfs-volants vendus depuis le début de la
production sont brisés par leur propriétaire.

\subQ{a} Combien y-aura-t-il de cerfs-volants après $n$ années?

\subQ{b} Quel est le niveau de stabilisation du marché pour ce type de
cerfs-volants?  C'est-à-dire, si $P_n$ est le nombre de cerfs-volants
après $n$ années, quelle est la limite
$\displaystyle \lim_{n\rightarrow \infty} P_n$ si cette limite existe.?
\label{3Q4}
\end{question}

\begin{question}[\eng]
Déterminez si les séries suivantes convergent ou divergent.  Si elles
convergent, calculez leurs sommes.
\begin{center}
\begin{tabular}{*{2}{l@{\hspace{0.5em}}l@{\hspace{3em}}}l@{\hspace{0.5em}}l}
\subQ{a} & $\displaystyle \sum_{n=1}^\infty \frac{5^{n+2}}{7^{n-2}}$ &
\subQ{b} & $\displaystyle \sum_{n=0}^\infty \frac{2^n+1}{6^n}$ &
\subQ{c} & $\displaystyle \sum_{n=0}^\infty
  \frac{2^n- 3^{n+2}}{5^{n+1}}$ \\[1em]
\subQ{d} & $\displaystyle\sum_{n=1}^\infty\frac{2}{n(n+1)}$ &
\subQ{e} & $\displaystyle \sum_{n=1}^\infty 5^n 2^{-2n}$ &
\subQ{f} & $\displaystyle \sum_{n=1}^\infty \frac{30}{n^2+3n+2}$ \\[1em]
\subQ{g} & $\displaystyle \sum_{n=1}^\infty \frac{1}{n(n+5)}$ &
& & &
\end{tabular}
\end{center}
\label{3Q5}
\end{question}

\subsection{Tests de convergence}

\begin{question}[\eng]
Déterminez si les séries suivantes sont convergentes ou divergentes.
\begin{center}
\begin{tabular}{*{2}{l@{\hspace{0.5em}}l@{\hspace{3em}}}l@{\hspace{0.5em}}l}
\subQ{a} & $\displaystyle \sum_{n=1}^\infty \frac{5^n}{(n+1)!}$ &
\subQ{b} & $\displaystyle \sum_{n=1}^\infty\left(1+\frac{2}{n}\right)^n$ &
\subQ{c} & $\displaystyle \sum_{n=0}^\infty
           \frac{n^2-1}{\sqrt[3]{n^7+1}}$ \\[1em] 
\subQ{d} & $\displaystyle \sum_{n=0}^\infty\;\frac{n^{1/3}}{n+4}$ &
\subQ{e} & $\displaystyle \sum_{n=1}^\infty \frac{2}{\sqrt[3]{n}}$ &
\subQ{f} & $\displaystyle \sum_{n=0}^\infty \frac{(n+1)^2}{n!}$ \\[1em]
\subQ{g} & $\displaystyle \sum_{n=1}^\infty \frac{1}{n(n^3+5)^{1/3}}$ &
\subQ{h} & $\displaystyle \sum_{n=1}^\infty
           \frac{\cos^2(n)}{n\sqrt{n}}$ &
\subQ{i} & $\displaystyle \sum_{n=1}^\infty\frac{n}{5^n}$ \\[1em]
\subQ{j} & $\displaystyle \sum_{n=0}^\infty\;\frac{\cos^2(n)}{3^n}$ &
\subQ{k} & $\displaystyle \sum_{n=1}^\infty \frac{n^2+1}{n^5+1}$ &
\subQ{l} & $\displaystyle \sum_{n=1}^{\infty}
\frac{(3n+1)^{1/2}}{(n^3+n)^{1/4}}$ \\[1em]
\subQ{m} & $\displaystyle \sum_{n=1}^\infty \frac{5+3\sin(n)}{3n^3+n+4}$ &
\subQ{n} & $\displaystyle \sum_{n=1}^\infty \frac{n^3+2}{\sqrt{n^7+8}}$ &
&
\end{tabular}
\end{center}
\label{3Q6}
\end{question}

\subsection{Convergence absolue et séries alternées}

\begin{question}[\eng]
Déterminez si les séries suivantes convergent absolument,
convergent conditionnellement ou divergent.
\begin{center}
\begin{tabular}{*{2}{l@{\hspace{0.5em}}l@{\hspace{3em}}}l@{\hspace{0.5em}}l}
\subQ{a} & $\displaystyle \sum_{n=1}^\infty (-1)^n\frac{1}{\sqrt{n^2+3}}$ &
\subQ{b} & $\displaystyle \sum_{n=2}^\infty \frac{(-1)^n}{\ln(n)}$ &
\subQ{c} & $\displaystyle \sum_{n=1}^\infty \frac{(-1)^n n!}{(2n+1)!}$
\\[1em]
\subQ{d} & $\displaystyle \sum_{n=1}^\infty \frac{(-1)^n}{\sqrt{n(n+2)}}$ &
\subQ{e} & $\displaystyle \sum_{n=1}^\infty (-1)^n\left(\frac{1}{n^{3/2}}
+ \frac{1}{n^2}\right)$ &
\subQ{f} & $\displaystyle \sum_{n=1}^\infty \frac{(-1)^n n^2}{(2n+1)(2n+3)}$
\\[1em]
\subQ{g} & $\displaystyle \sum_{n=1}^{\infty} \frac{\sin(n)}{n^2 +n-1}$ &
& & &
\end{tabular}
\end{center}
\label{3Q7}
\end{question}

\begin{question}[\eng]
Montrez que la série
$\displaystyle \sum_{n=1}^\infty \frac{(-1)^{n+1}}{3^n}$ converge et
trouvez un entier $N$ tel que la somme partielle $S_n$ de cette série
satisfasse $|S - S_n | < 10^{-3}$ pour $n \geq N$ où $S$ est la somme
de la série.
\label{3Q8}
\end{question}

\begin{question}[\eng]
Montrez que la série
$\displaystyle \sum_{n=1}^\infty \frac{(-1)^n \, n}{4^n}$ converge et
trouvez un entier $N$ tel que la somme partielle $S_n$ de cette série
satisfasse $|S - S_n | < 10^{-3}$ pour $n \geq N$ où $S$ est la somme
de la série.  Donnez une approximation de la somme $S$ de la série
avec une erreur inférieure à $10^{-3}$. 
\label{3Q9}
\end{question}

\begin{question}[\eng]
Soit la série
$\displaystyle \sum_{n=1}^\infty\;\frac{(-1)^n}{(2n+1)!}$.

\subQ{a} Montrez que cette série converge.\\
\subQ{b} Trouvez un petit entier $N$ tel que $|S - S_N | < 10^{-4}$ où
$S$ est la somme de la série et $S_N$ est la somme partielle des $N$
premiers termes de la séries.\\
\subQ{c} Donnez une approximation de la somme $S$ de la série avec une
erreur inférieure à $10^{-4}$.
\label{3Q10}
\end{question}


%%% Local Variables: 
%%% mode: latex
%%% TeX-master: "notes"
%%% End: 

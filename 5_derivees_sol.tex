\section{Dérivée}

\subsection{Taux de variation}

\compileSOL{\SOLUb}{\ref{5Q1}}{
\subQ{a} \\
\subQ{i}
$\displaystyle \frac{f(1+1)-f(1)}{1} = \frac{2\times 2^2 - 2}{1} = 6$,
\quad
$\displaystyle \frac{f(1+0.5)-f(1)}{0.5} = \frac{2 \times (1.5)^2 - 2}{0.5}
= 5$,\\
$\displaystyle \frac{f(1+0.1)-f(1)}{0.1} = \frac{2 \times (1.1)^2 - 2}{0.1}
= 4.2$ et\\
$\displaystyle \frac{f(1+0.01)-f(1)}{0.01} =
\frac{2 \times (1.01)^2 - 2}{0.01} = 4.02$ respectivement.

\subQ{ii} $y-2 = 6(t-1)$, $y-2 = 5(t-1)$, $y-2 = 4.2(t-1)$ et
$y-2 = 4.02(t-1)$ respectivement.

\subQ{iii} Voir la figure ci-dessous.

\subQ{iv} La pente de la tangente est la limite des pentes
\[
\frac{f(t_0 + \Delta t) - f(t_0)}{\Delta t} =
\frac{f(1 + \Delta t) - f(1)}{\Delta t}
\]
des sécantes qui passent par les points $(t_0,f(t_0))= (1,2)$ et
$(t_0+\Delta t, f(t_0+\Delta t)) = (1+\Delta t, f(1+\Delta t))$ lorsque 
$\Delta t$ tend vers $0$.  Ainsi, nous pouvons conclure à partir des calculs
faits en (i) que la pente de la tangente à la courbe $y=f(x)$ au point
$(1,2)$ est possiblement $4$.

\subQ{v} $y-2 = 4(t-1)$

\MATHgraph{5_derivees/limsec1_a}{8cm}

\subQ{b}\\
\subQ{i} $\displaystyle \frac{f(0+1)-f(0)}{1} = \frac{e^{2(1)} - 1}{1}
\approx 6.3891$, \quad
$\displaystyle \frac{f(0+0.5)-f(0)}{0.5} = \frac{e^{2(0.5)} - 1}{0.5}
\approx 3.4366$,\\
$\displaystyle \frac{f(0+0.1)-f(0)}{0.1} = \frac{2^{2(0.1)} - 1}{0.1}
\approx 2.2140$ et\\
$\displaystyle \frac{f(0+0.01)-f(0)}{0.01} =
\frac{e^{2(0.01)} - 1}{0.01} \approx 2.0201$ respectivement.

\subQ{ii} $y-1 \approx 6.3891(t-0)=6.3891 t$, $y-1 \approx 3.4366 t$,
$y-1 \approx 2.2140 t$ et $y-1 \approx 2.0201 t$ respectivement.

\subQ{iii} Voir la figure ci-dessous.

\subQ{iv} La pente de la tangente est la limite des pentes
\[
\frac{f(t_0 + \Delta t) - f(t_0)}{\Delta t} =
\frac{f(\Delta t) - f(0)}{\Delta t}
\]
des sécantes qui passent par les points $(t_0,f(t_0))= (0,1)$ et
$(t_0+\Delta t, f(t_0+\Delta t)) = (\Delta t, f(\Delta t))$ lorsque 
$\Delta t$ tend vers $0$.  Ainsi, nous pouvons conclure à partir des calculs
faits en (i) que la pente de la tangente à la courbe $y=f(x)$ au point
$(0,1)$ est possiblement $2$.

\subQ{v} $y-1 = 2t$

\MATHgraph{5_derivees/limsec1_b}{8cm}
}

\compileSOL{\SOLUa}{\ref{5Q2}}{
\subQ{a} $\displaystyle \frac{p(1)-p(0)}{1} = \frac{1.5^1 - 1}{1} = 0.5$

\subQ{b} $\displaystyle \frac{p(0.1)-p(0)}{0.1} = \frac{1.5^{0.1} - 1}{0.1}
\approx 0.4138$

\subQ{c} $\displaystyle \frac{p(0.01)-p(0)}{0.01} = \frac{1.5^{0.01} - 1}{0.01}
\approx 0.4063$

\subQ{d} $\displaystyle \frac{p(0.001)-p(0)}{0.001} =
\frac{1.5^{0.001} - 1}{0.001} \approx 0.4055$

\subQ{e} $\ln(1.5) \approx 0.40546510$

\subQ{f} $y-1 \approx 0.40546510\, t$
}

\compileSOL{\SOLUb}{\ref{5Q3}}{
\subQ{a}
À $t=0$, $\displaystyle H'(0) \approx \frac{H(1)-H(0)}{1-0} = 1.07$.
À $t=1$, puisque que
$\displaystyle \frac{H(0)-H(1)}{0-1} = 1.07$ et 
$\displaystyle \frac{H(2)-H(1)}{2-1} = 1.22$, nous estimons $H'(1)$ par
la moyenne
$\displaystyle H'(1) \approx \frac{1.22+1.07}{2} = 1.145$.
À $t=2$, puisque que
$\displaystyle \frac{H(1)-H(2)}{1-2} = 1.22$ et 
$\displaystyle \frac{H(3)-H(2)}{3-2} = 1.34$, nous estimons $H'(2)$ par la
moyenne
$\displaystyle H'(2) \approx \frac{1.22+1.34}{2} = 1.28$.
Nous procédons de la même façon pour trouver
$H'(3) \approx 1.305$, $H'(4) \approx 1.435$, $H'(5) \approx 1.63$,
$H'(6) \approx 1.78$, $H'(7) \approx 1.87$, $H'(8) \approx 2.14$ et
$H'(9) \approx 2.42$.  Finalement, 
$\displaystyle H'(10) \approx \frac{H(10)-H(9)}{10-9} = 2.4$\ .

\subQ{b}
\MATHgraph{5_derivees/tauxrel1_a}{8cm}

\subQ{c}
\[
\begin{array}{c|cccccc}
t & 0 & 1 & 2 & 3 & 4 & 5 \\
\hline
H'(t)/H(t) \approx & 0.105835 & 0.102415 & 0.103225 &
0.094978 & 0.095602 & 0.098133  \\
\multicolumn{7}{c}{} \\
\multicolumn{1}{c}{} & 6 & 7 & 8 & 9 & 10 & \\
\cline{2-7}
\multicolumn{1}{c}{} & 0.097427 & 0.092711 & 0.097228 & 0.098977 & 0.089385 &
\end{array}
\]

\subQ{d}
\MATHgraph{5_derivees/tauxrel1_b}{8cm}

\subQ{f} Le taux de croissance relatif n'est pas constant mais
augmente.  Alors qu'il ne semble pas y avoir de relation entre les
différents taux de croissance instantanés que nous avons calculés, nous pouvons
facilement imaginer en regardant le graphe des taux de croissance
relatifs en fonction du temps qu'ils semblent tracer une courbe
convexe.
}

\compileSOL{\SOLUa}{\ref{5Q4}}{
{\scriptsize    
\[
\begin{array}{c|c|c}
a & b &
\text{Taux de croissance moyen entre $a$ et $b$ heures (individus/heure)} \\ 
\hline
0 & 1 & \rule{0em}{2em} \displaystyle \frac{2^1-2^0}{1-0} = 1 \\[0.7em]
0 & 0.1 & \displaystyle \frac{2^{0.1}-2^0}{0.1-0} = 0.7177\ldots \\[0.7em]
0 & 0.01 & \displaystyle \frac{2^{0.01}-2^0}{0.01-0} = 0.6956\ldots \\[0.7em]
0 & 0.001 & \displaystyle \frac{2^{0.001}-2^0}{0.001-0} = 0.6956\ldots \\
\vdots & \vdots & \qquad \vdots \\
0 & 10^{-5} & \displaystyle \frac{2^{10^{-5}}-2^0}{10^{-5}-0} = 0.6931\ldots
\\[0.7em]
0 & 10^{-6} & \displaystyle \frac{2^{10^{-6}}-2^0}{10^{-6}-0} = 0.6931\ldots \\
\vdots & \vdots & \qquad \vdots \\
 & \downarrow & \downarrow \\
 & 0 & 0.6931\ldots
\end{array}
\]
}
Nous en déduisons que
\[
\lim_{b\rightarrow 0} \frac{2^b - 2^0}{b-0} \approx 0.6931\ldots
\]
C'est le taux de variation instantané de $2^t$ à $t=0$.

Nous retrouvons dans la figure ci-dessous les graphes de $p$ et de la
droite tangente au graphe de $p$ au point $(0,1)$.
\MATHgraph{5_derivees/der_ass1A}{8cm}

La pente $m$ de la droite tangente est le taux de variation instantané
de $2^t$ à $t=0$; c'est-à-dire, $m \approx 0.6931$.  L'équation de
cette droite tangente dans la forme point-pente est
\[
(p-1) = m (t-0) \Rightarrow p = 0.6931 t + 1 \ .
\]
}

\compileSOL{\SOLUb}{\ref{5Q5}}{
\subQ{a}
\[
\begin{array}{c|c|c}
a & b & \text{Taux de croissance moyen de la $a^e$ à la $b^e$ heure} \\
& & \text{(visites/heure)} \\
\hline
2 & 3 & \rule{0em}{2em} \displaystyle \frac{87-57}{3-2} = 30 \\[0.7em]
3 & 4 & \displaystyle \frac{151-87}{4-3} = 64 \\[0.7em]
3 & 5 & \displaystyle \frac{246-87}{5-3} =  \frac{159}{2} = 79.5
\end{array}
\]

\subQ{b}
Nous pouvons estimer le taux de croissance instantané après trois heures en
calculant la moyenne du taux de croissance moyen de la $2^e$ à la
$3^e$ heure et du taux de croissance moyen de la $3^e$ à la $4^e$
heure.  Nous obtenons $(30+64)/2 = 47$ visites/heure.  Nous utilisons les
taux de croissance moyens calculés sur les plus petits intervalles de
temps possibles.

\subQ{c}
Ce n'est pas réaliste de demander d'estimer le taux de croissance
instantané après trois heures sur la base des données que nous
possédons.  Il faudrait calculer le nombre de visites sur des
intervalles de temps beaucoup plus petits qu'une heure.
}

\subsection{Dérivée d'une fonction}

\compileSOL{\SOLUb}{\ref{5Q6}}{
\subQ{a} La fonction $f$ n'est pas continue en $x=4$.

\subQ{b} La fonction $f$ n'est pas différentiable en $x=4$ car elle
n'est pas continue à ce point.  De plus, la fonction $f$ n'est
pas différentiable en $x=5.5$ car
\[
\lim_{x\to 5.5^-} \frac{f(x)-f(5.5)}{x-5.5} = 2 \quad \text{et}
\quad \lim_{x\to 5.5^+} \frac{f(x)-f(5.5)}{x-5.5} = -2 \neq 2
\]
impliquent que
\[
\lim_{x\to 5.5} \frac{f(x)-f(5.5)}{x-5.5}
\]
n'existe pas.  Le graphe de la fonction a un \flqq coin\frqq\ au point
$(5.5,f(5.5))$.  Finalement, La fonction $f$ n'est pas différentiable
en $x=6.5$ car
\[
\lim_{x\to 6.5} \frac{f(x)-f(6.5)}{x-6.5} = -\infty \ .
\]
La tangente à la courbe $y=f(x)$ au point $(6.5,f(6.5))$ est verticale.

\subQ{c} La dérivée de la fonction est nulle aux points où la tangente
à la courbe est horizontale.  Nous trouvons donc $x=0.5$, $2$ et $3.5$.
}

\compileSOL{\SOLUa}{\ref{5Q7}}{
La valeur de la dérivée de $f$ en un point est donné par la pente de
la droite tangente à la courbe en ce point.

\subQ{a} La dérivée est positive sur les intervalles $]0.5,2[$,
$]3,4[$ et $]5.5,7[$.

\subQ{b} La dérivée est négative sur les intervalles $]0,0.5[$,
$]1,3[$ et $]4,5.5[$.

\subQ{c} La dérivée est nulle aux points $x=0.5$, $2$, $3$ et $5.5$.
}

\compileSOL{\SOLUb}{\ref{5Q8}}{
\subQ{a} \PDFgraph{5_derivees/der31}

\subQ{b} \PDFgraph{5_derivees/der32}

\subQ{c} \PDFgraph{5_derivees/der33}
}

\compileSOL{\SOLUa}{\ref{5Q9}}{
\subQ{a} La figure ci-dessous contient des graphes possibles pour la
position et la vitesse de la voiture et de la dépanneuse en fonction
du temps répondant aux conditions de la question.  La voiture et la
dépanneuse ont la même vitesses en tout temps.
\PDFgraph{5_derivees/towcars1}

\subQ{b} La voiture et de la dépanneuse ont la même vitesse.
Pour $0\leq t < t_1$, la voiture et la dépanneuse accélèrent.
Pour $t_1 \leq t \leq t_2$, ils gardent une vitesse constante.
Finalement, pour $t>t_2$, ils sont au repos.  Disons que le freinage a
été plutôt brusque.
\PDFgraph{5_derivees/towcars2}
}

\compileSOL{\SOLUb}{\ref{5Q10}}{
La fonction $g$ ne peut pas être la dérivée de $f$ car, entre $0$ et
$3.7$, $f$ est une fonction strictement décroissante.  Donc $g$
devrait être négatif entre $0$ et $3.7$.

Nous avons que $f$ est la dérivée de $g$.  Nous avons que $f$ est
positif quand $g$ est strictement croissante, $f$ est négatif quand $g$ est
strictement décroissante, et $f$ est zéro aux points où $g$ à un
extremum.
}

\compileSOL{\SOLUb}{\ref{5Q11}}{
Nous avons que $f$ est la dérivée de $g$.  Nous avons que $f$ est positif quand
$g$ est strictement croissante, $f$ est négatif quand $g$ est
strictement décroissante, et $f$ est zéro aux points où $g$ à un
extremum.
}

\compileSOL{\SOLUb}{\ref{5Q13}}{
\PDFgraph{5_derivees/der30}
La température augmente lorsque $T'(t)>0$ et diminue lorsque
$T'(t)<0$.
}

\compileSOL{\SOLUa}{\ref{5Q14}}{
\PDFgraph{5_derivees/ass3C_sol1}
Les réponses ci-dessous font références au graphe ci-dessus.

\subQ{a} Il suffit de choisir une valeur de $t$ pour laquelle la pente
de la droite tangente à la courbe $y=V(t)$ au point $(t,V(t))$ est
positive.  Nous avons choisie $t = A = 0.5$.

\subQ{b} Il suffit de choisir une valeur de $t$ pour laquelle la pente
de la droite tangente à la courbe $y=V(t)$ au point $(t,V(t))$ est
négative.  Nous avons choisie $t = B = 1.25$.

\subQ{c} La valeur de $t$ pour laquelle la pente de la droite
tangente à la courbe $y=V(t)$ au point $(t,V(t))$ semble maximale est
$t = C = 0.75$.

\subQ{d} La valeur de $t$ pour laquelle la pente de la droite tangente
à la courbe $y=V(t)$ au point $(t,V(t))$ semble minimale est
$t = D = 1.25$.  

\subQ{e} Finalement, les valeurs de $t$ pour lesquelles la
pente de la droite tangente à la courbe $y=V(t)$ au point $(t,V(t))$
est nulle sont $t = E = 1$ et $t = F = 2$.

\subQ{f} Le graphe de la dérivée $V'$ de $V$ est donné ci-dessous.
\PDFgraph{5_derivees/ass3C_sol2}
Ce graphe de la dérivée $V'$ de la fonction $V$ est naturellement une
approximation du graphe de $V'$ car il faudrait calculer la pente de la
droite tangente à la courbe $y=V(t)$ à tous les points $(t,V(t))$ pour
pouvoir tracer exactement le graphe de $V'$.
}

\compileSOL{\SOLUb}{\ref{5Q15}}{
\subQ{a} La fonction $f$ n'est pas continue au point $x=1$ car
\[
  \lim_{x\rightarrow 1^-} f(x) \approx \frac{11}{6} \neq
  2 = \lim_{x\rightarrow 1^+} f(x) \ .
\]

\subQ{b} La fonction $f$ n'est pas différentiable aux points $x=0.5$ et
$x=1$.  Puisque nous venons de montrer que la fonction $f$ n'est pas
continue au point $x=1$, elle n'est donc pas différentiable à ce
point. Que la dérivée de $f$ n'existe pas au point
$x=0.5$ est justifié par la figure suivante.
\PDFgraph{5_derivees/ass3B_sol}
La fonction $f$ est continue au point $x=0.5$ mais ne possède pas de
droite tangente au point $(0.5, f(0.5))$.  Nous pouvons voir que
\[
\lim_{x\to 0.5^-} \frac{f(x)-f(0.5)}{x-0.5} = 0 
\neq \frac{19}{6} \approx \lim_{x\to 0.5^+} \frac{f(x)-f(0.5)}{x-0.5} 
\]
La limite à gauche est $0$ alors que celle à droite est beaucoup plus
grande.   La fonction $f$ n'est donc pas différentiable au point
$x = 0.5$.

\subQ{c} La dérivée de $f$ est nulle aux points $x$ tels que la droite
tangente à la courbe $y=f(x)$ au point $(x, f(x))$ est horizontal
(i.e. de pente nulle).  Ainsi, la fonction $f$ possède une dérivée
nulle au point $x = 1.5$ seulement.
}

\compileSOL{\SOLUb}{\ref{5Q16}}{
\begin{align*}
f'(x) &= \lim_{h\to 0} \frac{f(x+h)-f(x)}{h}
= \lim_{h\to 0} \frac{\left( 4 -(x+h)^2\right) - \left( 4-x^2\right)}{h} \\
&= \lim_{h\to 0} \frac{\left(-x^2 -2xh -h^2 +4\right) - \left( 4-x^2\right)}{h}
= \lim_{h\to 0} \frac{-2xh -h^2}{h} \\
& = \lim_{h\to 0} \left( -2x - h\right) = -2x \ .
\end{align*}
\PDFgraph{5_derivees/der26}

Le seul point $x$ pour lequel $f'(x) = 0$ est $x=0$ car $f'(x) = -2x$.
Puisque $f'(x)=-2x>0$ pour $x<0$, la fonction $f$ est strictement
croissante pour $x<0$.  Puisque $f'(x)=-2x<0$ pour $x>0$, la fonction
$f$ est strictement décroissante pour $x>0$.
}

\compileSOL{\SOLUb}{\ref{5Q17}}{
la pente de la sécante à la courbe $y=g(x)$ entre les points $x$ et
$x+h$ est
\[
\frac{g(x+h)-g(x)}{h} = \frac{(x+h) + 2(x+h)^2 - x - 2x^2}{h}
= \frac{h + 4h x+ 2h^2}{h} = 1 + 4x + 2h \ .
\]
La pente de la tangente à la courbe $y=g(x)$ au point $x$ est
\[
g'(x) = \lim_{h\rightarrow 0} \frac{g(x+h)-g(x)}{h}
= \lim_{h\rightarrow 0} \left( 1 + 4x + 2h \right)
= 1 + 4x + \lim_{h\rightarrow 0} 2h = 1+ 4x \ .
\]
}

\compileSOL{\SOLUb}{\ref{5Q18}}{
\subQ{a} Puisque
\begin{align*}
\frac{f(x+h)-f(x)}{h} &= \frac{(x+h+1)^2-(x+1)^2}{h} \\
&= \frac{(x^2 + 2xh + 2x + h^2 + 2h + 1) - (x^2 + 2x + 1)}{h} \\
&= \frac{2xh + 2h + h^2}{h} = 2x + 2 + h \ ,
\end{align*}
nous avons que
\[
f'(x) = \lim_{h\rightarrow 0} \frac{f(x+h)-f(x)}{h}
= \lim_{h\rightarrow 0} (2x + 2 + h) = 2x + 2 \ .
\]

\subQ{b}
\begin{align*}
f'(x) &= \lim_{h\rightarrow 0} \frac{f(x+h) - f(x)}{h}
=\lim_{h\rightarrow 0} \frac{1/(2(x+h)+5) - 1/(2x+5)}{h} \\
&=\lim_{h\rightarrow 0} \frac{(2x+5) - (2(x+h)+5)}{h(2x+5)(2x+5+2h)}
= \lim_{h\rightarrow 0} \frac{-2h}{h(2x+5)(2x+5+2h)} \\
&= \lim_{h\rightarrow 0} \frac{-2}{(2x+5)(2x+5+2h)}
= \frac{-2}{(2x+5)^2} \ .
\end{align*}

\subQ{c}
\begin{align*}
f'(x) &= \lim_{h\rightarrow 0} \frac{f(x+h) - f(x)}{h}
=\lim_{h\rightarrow 0} \frac{\big((x+h)^2+(x+h)\big)-\big(x^2+x\big)}{h} \\
&=\lim_{h\rightarrow 0} \frac{2xh+h^2 +h}{h}
= \lim_{h\rightarrow 0} \left(2x+h +1\right) = 2x+1 \ .
\end{align*}

\subQ{e}
\begin{align*}
f'(x) &= \lim_{h\rightarrow 0} \frac{f(x+h) - f(x)}{h}
=\lim_{h\rightarrow 0} \frac{1/((x+h)(x+1+h)) - 1/(x(x+1))}{h} \\
&=\lim_{h\rightarrow 0} \frac{x(x+1) - (x+h)(x+1+h)}{hx(x+1)(x+h)(x+1+h)}
= \lim_{h\rightarrow 0} \frac{-2xh -h -h^2}{hx(x+1)(x+h)(x+1+h)} \\
&= \lim_{h\rightarrow 0} \frac{-2x -1 -h}{x(x+1)(x+h)(x+1+h)}
= \frac{-2x-1}{x^2(x-1)^2} \ .
\end{align*}
}

\compileSOL{\SOLUa}{\ref{5Q19}}{
Il est évident qu'il n'existe pas de valeur $c$ telle que $H(c) = 0.5$ car
$0.5$ n'est pas dans l'image de $H$.

La pente de la sécante entre les points $(-1,H(-1))$ et $(1,H(1))$ est
$\displaystyle \frac{H(1)-H(-1)}{2} = \frac{1}{2}$.  Puisque $H'(x)=0$ pour
tout $x\neq 0$ et $H'(x)$ n'est pas défini à $x=0$, il est alors clair que
nous ne pouvons pas trouver une valeur $c$ telle que $H'(c)= 1/2$.

Le problème avec la fonction de Heaviside est qu'elle n'est pas continue à
l'origine.
}

\subsection{Calcul des dérivées}

\compileSOL{\SOLUb}{\ref{5Q21}}{
Puisque $g'(y) = -3 < 0$ pour tout $y$, la fonction est strictement
décroissante sur $\RR$.  Ce n'est pas surprenant car $z=g(y)=-3y+5$
est l'équation d'une droite de pente $-3$.  L'intervalle où la
fonction est strictement croissante est l'intervalle vide.
}

\compileSOL{\SOLUa}{\ref{5Q22}}{
\subQ{a} $T(t) = p(t)M(t)= (2\times 10^6 + 10^3 t^2)(80 - 0.5 t)$.

\subQ{b}
\begin{align*}
T'(t) &= \left(\dfdx{(2\times 10^6 + 10^3 t^2)}{t}\right)(80 - 0.5 t)
+ (2\times 10^6 + 10^3 t^2)\left(\dfdx{(80 - 0.5 t)}{t}\right) \\
&= 2\times 10^3\, t(80 - 0.5 t) - 0.5(2\times 10^6 + 10^3 t^2) \\
&= -1.5\times 10^3\, t^2 + 1.6 \times 10^5 \, t - 10^6 \ .
\end{align*}

\subQ{c} Nous avons que
\[
T'(t) = -1.5\times 10^3\, t^2 + 1.6 \times 10^5\, t - 10^6 =
-1.5\times 10^3 \left( t^2 - 106.\overline{6} t + 666.\overline{6}\right)= 0
\]
pour
\begin{align*}
t = t_1 &= \frac{106.\overline{6}- \sqrt{106.\overline{6}^2 - 4 \times
666.\overline{6}}}{2} = 6.\overline{6}
\intertext{et}
t = t_2 &= \frac{106.\overline{6}+ \sqrt{106.\overline{6}^2 - 4 \times
666.\overline{6}}}{2} = 100
\end{align*}
À $t=t_1$, $T(t_1) = 153674074.\overline{074}$,
$M(t_1)= 76.\overline{6}$ et $p(t_1) = 2004444.\overline{4} \approx
2004445$.
À $t=t_2$, $T(t_2) = 3.6 \times 10^8$, $M(t_2)= 30$ et
$p(t_2) = 1.2\times 10^7$.
}

\compileSOL{\SOLUb}{\ref{5Q23}}{
Si $v(t)$ est la vitesse du train au temps $t$ et $u(t)$ est la
vitesse du passager par rapport au train au temps $t$, alors
$w(t) = v(t)+u(t)$ est la vitesse du passager par rapport au sol au
temps $t$.  Nous avons $v(t) = 110$ km/h et $u(t)=-3$ km/h au temps
$t$.  Alors la vitesse du passager est $w(t) = 110-3 = 107$ km/h au
temps $t$.  Notez que $u(t)<0$ car le passager se déplace dans la direction
opposée au mouvement du train.
}

\compileSOL{\SOLUb}{\ref{5Q24}}{
\subQ{c} Nous avons $H(G) = P(M(G))$ où $M(G)= 5G+2$ et $P(M)=0.5M$.
Puisque $M'(G) = 5$ et $P'(N) = 0.5$, nous obtenons
\[
\dydx{H}{G} = P'(M(G))M'(G) = 0.5\times 5 = 2.5 \ .
\]
\subQ{d} Nous avons $H(I) = F(V(I))$ où $F(V) = 37 + 0.4 V$ et $V(I) = 5 I^2$.
Puisque $F'(V) = 0.4$ et $V'(I) = 10 I$, nous obtenons
\[
  \dydx{H}{I} = F'(V(I)) V'(I) = 0.4 \times 10 I = 4I \ .
\]
}

\compileSOL{\SOLUb}{\ref{5Q25}}{
Nous avons comme information que $L'(t)$ est constant et donc égale
au taux de variation moyen
$\displaystyle \frac{\Delta L}{\Delta t} = \frac{5}{10} = 0.5$.

Utilisons
\begin{align*}
\dydx{H}{t} &= \dydx{B}{W}(W(L(t))\; \dydx{W}{L}(L(t))\; \dydx{L}{t}(t) \\
&= \left( \frac{2}{3} \times 0.007\, W^{-1/3}\right)
\left(2.53 \times 0.12 \, L^{1.53} \right) \left( 0.5 \right)
\end{align*}
pour calculer $\displaystyle \dydx{H}{t}$ lorsque $L=18$.
Puisque $W = 0.007 (18)^{2/3} = 0.04807799619\ldots$\ lorsque $L=18$,
nous obtenons
\begin{align*}
\dydx{H}{t} &=
\left( \frac{2}{3} \times 0.007 (0.04807799619\ldots)^{-1/3}\right)
\left(2.53 \times 0.12 \times 18^{1.53} \right)
\left( 0.5 \right) \\
&= 0.1622544808
\end{align*}
lorsque $L=18$.
}

\compileSOL{\SOLUb}{\ref{5Q26}}{
\subQ{a} $\displaystyle f'(x) = \frac{1}{5} x^{1/5 - 1} =
\frac{1}{5} x^{-4/5}$

\subQ{b} $\displaystyle h'(t) = \frac{1}{e}x^{1/e-1}
= \frac{1}{e}x^{(1-e)/e}$

\subQ{c} $\displaystyle g'(z) = 3 \left(3 z^{3-1}\right)
+ 2 \left( 2 z^{2-1} \right) = 9 z^{2} + 4z$

\subQ{e} Nous avons la fonction composée $f(x) = f_1(f_2(x))$ où
$f_2(x) = 2x+1$ et $f_1(y) = y^3$.  Puisque $f_1'(y) = 3 y^2$ et
$f_2'(x) = 2$, nous obtenons
\[
f'(x) = f_1'(f_2(x)) \; f_2'(x) = 3(2x+1)^2 \left( 2 \right)
= 6 (2x+1)^2 \ .
\]

\subQ{f} $\displaystyle h'(x) = 135x (x^2-5)^{133/2}$

\subQ{g} Si nous dérivons l'expression
\[
\ln| G(x) | = \ln\left(\frac{|1+x||2+x|}{|3+x|}\right)
= \ln|1+x| + \ln|2+x| - \ln|3+x| \ ,
\]
nous obtenons
\[
\frac{G'(x)}{G(x)} = \frac{1}{1+x} + \frac{1}{2+x} - \frac{1}{3+x} \ .
\]
Donc
\[
G'(x) = \left(\frac{1}{1+x} + \frac{1}{2+x} - \frac{1}{3+x}\right)G(x)
= \frac{2+x}{3+x} + \frac{1+x}{3+x} - \frac{(1+x)(2+x)}{(3+x)^2} \ .
\]

\subQ{h} $\displaystyle f'(t) = \frac{1}{2\sqrt{t}(1+\sqrt{t})^2}$

\subQ{i} $\displaystyle f'(x) = \frac{3}{(3x-1)\ln(2)}$

\subQ{j} Si nous complétons la factorisation, nous avons
$f(y) = (5y-3)^7(y-1)(y+1)$.
Donc
\[
\ln|f(y)| = \ln\left(|5y-3|^7|y-1||y+1|\right)
= 7 \ln|5y-3| + \ln|y - 1| + \ln|y+1| \ .
\]
Si nous dérivons des deux côtés de cette équation, nous obtenons
\[
\frac{f'(y)}{f(y)} = \frac{35}{5y-3} + \frac{1}{y-1} + \frac{1}{y+1} \ .
\]
Ainsi,
\begin{align*}
f'(y) &= \left(\frac{35}{5y-3} + \frac{1}{y-1} + \frac{1}{y+1}\right)f(y) \\
&= 35 (5y-3)^6(y^2-1) + (5y-3)^7(y+1) + (5y-3)^7(y-1) \\
&= (5y-3)^6\left( 35(y^2-1)+2y(5y-3)\right)
= (5y-3)^6(45y^2-6y-35) \ .
\end{align*}

\subQ{k}
\begin{align*}
h'(t) &= \frac{\displaystyle
\left(\dfdx{(1+t)}{t}\right)(2-t)-(1+t)\left(\dfdx{(2-t)}{t}\right)}
        {(2-t)^2} \\
&= \frac{(2-t)+(1+t)}{(2-t)^2} = \frac{3}{(2-t)^2}
\end{align*}

\subQ{l}
\begin{align*}
g'(z) &= \frac{\displaystyle
\left(\dfdx{(1+z^2)}{z}\right)(1+2z^3)-(1+z^2)\left(\dfdx{(1+2z^3)}{z}\right)}
{(1+2z^3)^2} \\
&= \frac{2z(1+2z^3)-6z^2(1+z^2)}{(1+2z^3)^2} = \frac{-2z^4-6z^2+2z}{(1+2z^3)^2}
\end{align*}

\subQ{m} Nous avons $f(x) = g(x)/h(x)$, où $g(x)=\ln(2x+e^x)$ et $h(x)= 2x+e^x$.
Nous allons utiliser la règle pour la dérivée d'un quotient.  Notons
que $g(x)$ est une composition de fonctions et que
\[
g'(x) = \left( \dfdx{\ln(y)}{y}\bigg|_{y=2x+e^x}\right)\dfdx{(2x + e^x)}{x}
= \frac{1}{y}\bigg|_{y=2x+e^x} (2+e^x) = \frac{2+e^x}{2x+e^x} \ .
\]
Nous avons
\begin{align*}
f'(x) &= \frac{g'(x) h(x) - g(x)h'(x)}{ h^2(x)}
= \frac{ \big((2+e^x)/(2x+e^x)\big)(2x+e^x)
- (2+e^x) \ln(2x+e^x)}{(2x+e^x)^2} \\
&= \frac{ (2+e^x) - (2+e^x) \ln(2x+e^x)}{(2x+e^x)^2}
= (2+e^x) \left( \frac{  1- \ln(2x+e^x)}{(2x+e^x)^2} \right) \ .
\end{align*}

\subQ{n} Si nous dérivons des deux côtés de l'équation
\[
\ln|h(x)| = \ln\left(\frac{|1+3x|^2}{|1+2x|^3}\right)
= 2 \ln|1+3x| - 3\ln|1+2x| \ ,
\]
nous obtenons
\[
\frac{h'(x)}{h(x)} = \frac{6}{1+3x} - \frac{6}{1+2x} \ .
\]
Ainsi,
\begin{align*}
h'(x) &= \left( \frac{6}{1+3x} - \frac{6}{1+2x}\right) h(x)
= \left(\frac{6}{1+3x} - \frac{6}{1+2x}\right)
\left(\frac{(1+3x)^2}{(1+2x)^3}\right) \\
&= 6 \left( (1+2x) - (1+3x) \right) \left(\frac{(1+3x)}{(1+2x)^4}\right)
= \frac{-6x (1+3x)}{(1+2x)^4} \ .
\end{align*}

Nous avons utilisé $\ln|1 +3x| = g_2(g_1(x))$ où
$g_1(x) = 1+3x$ et $g_2(z)=\ln|z|$ pour obtenir
\[
\dfdx{\ln|1+3x|}{x} = g_2'(g_1(x)) g_1'(x)
= \left(\frac{1}{z}\bigg|_{z=1+3x}\right) \times 3 = \frac{3}{1+3x} \ .
\]
La dérivée de $\ln|1+2x|$ est calculée de façon semblable.

\subQ{o} Nous avons $\displaystyle f(x) = e^{-7x} = f_1(f_2(x))$ où
$f_1(y)=e^y$ et
$f_2(x) = -7x$.  Puisque $f_1'(y) = e^y$ et $f_2'(x) = -7$, nous obtenons
\[
f'(x) = f_1'(f_2(x)) f_2'(x) = e^{-7x} (-7) = - 7 e^{-7x} \ .
\]

\subQ{p} Puisque
\[
g(z) = \left(1 + \frac{2}{1+z}\right)^7 = \left(\frac{3+z}{1+z}\right)^7
= (3+z)^7 (1+z)^{-7} \ ,
\]
nous avons
\[
\ln|g(z)| = \ln\left(|3+z|^7 |1+z|^{-7}\right)
= 7\ln|3+z| - 7 \ln|1+z| \ .
\]
Si nous dérivons des deux côtés de cette équation, nous obtenons
\[
\frac{g'(z)}{g(z)} = \frac{7}{3+z} - \frac{7}{1+z}
\]
Ainsi,
\begin{align*}
g'(z) &= \left( \frac{7}{3+z} - \frac{7}{1+z}\right) g(z)
= \left(\frac{7}{3+z} - \frac{7}{1+z}\right)
\left(\frac{(3+z)^7}{(1+z)^7}\right) \\
&= 7 \left( (1+z) - (3+z) \right) \left(\frac{(3+z)^6}{(1+z)^8}\right)
= \frac{-14 (3+z)^6}{(1+z)^8} \ .
\end{align*}

\subQ{q} Puisque $f(t) = f_1(f_2(t))$ où $f_1(y) = y^{33}$ et
$f_2(t) = 1+3t$, nous obtenons
\[
f'(t) = f_1'(f_2(t)) f_2'(t) = 33(1+3t)^{32} \left( 3\right)
= 99(1+3t)^{32} \ . 
\]

\subQ{r} Nous avons $\displaystyle h(x) = \ln|\ln(x)| = h_1(h_2(x))$ où
$h_1(y) = \ln|y|$ et $h_2(x) = \ln(x)$ pour $x>0$.  Puisque
$h_1'(y) = y^{-1}$ et $h_2'(x) = x^{-1}$, nous obtenons
\[
h'(x) = h_1'(h_2(x)) h_2'(x) = \frac{1}{y}\bigg|_{y=\ln(x)}
\left(\frac{1}{x}\right)
= \frac{1}{\ln(x)} \left(\frac{1}{x}\right)
= \frac{1}{x\ln(x)}
\]
pour $0<x<1$ et $x>1$.

\subQ{s} Puisque
\[
g(t) = \ln\left(\frac{t^2}{(t-2)^3}\right) = 2\ln(t) - 3 \ln(t-2) \ ,
\]
nous avons
\[
g'(t) = \frac{2}{t} - \frac{3}{t-2}  \ .
\]

\subQ{t} Puisque
\[
f(x) = x^{\ln(x)} = e^{\ln(x^{\ln(x)})} = e^{(\ln(x))^2} \ ,
\]
nous avons
\begin{align*}
f'(x) &= \dfdx{e^{(\ln(x))^2}}{x}
=\left(\dfdx{e^y}{y}\bigg|_{y=(\ln(x))^2}\right)
\left(\dfdx{ (\ln(x))^2}{x}\right) \\
& = e^{(\ln(x))^2} \dfdx{(\ln(x))^2}{x}
= 2\, e^{(\ln(x))^2} \, \ln(x)\,\dfdx{\ln(x)}{x} \\
&= 2 \, e^{(\ln(x))^2} \, \ln(x)\,\left(\frac{1}{x}\right)
= 2\, x^{\ln(x)} \, \ln(x)\,\left(\frac{1}{x}\right)
= 2\, x^{\ln(x)-1}\,\ln(x) \ .
\end{align*}

\subQ{u} Puisque
\[
g(x) = x^{x^2} = e^{\ln(x^{x^2})} = e^{x^2 \ln(x)} \ ,
\]
nous avons
\begin{align*}
g'(x) &= \dfdx{e^{x^2\ln(x)}}{x}
= \left(\dfdx{e^y}{y}\bigg|_{y=x^2\ln(x)}\right)
\left(\dfdx{x^2 \ln(x)}{x}\right) \\
&= e^{x^2\ln(x)} \dfdx{\left( x^2\ln(x) \right)}{x}
 = e^{x^2\ln(x)} \left(2x\ln(x) + x^2\, \frac{1}{x}\right) \\
 &= x \, e^{x^2\ln(x)} \left( 2\ln(x) + 1\right)
 = x\, x^{x^2} \left(2\ln(x) +1\right)
 = x^{x^2+1}\left(2\ln(x) + 1\right) \ .
\end{align*}

\subQ{v} Grâce aux propriétés du logarithme, nous avons
\[
g(t) = 7\ln(t) + 8\ln(t^2-5) - 5 \ln(t-2) \ .
\]
Donc
\[
g'(t) = \frac{7}{t} + \frac{16t}{t^2-5} - \frac{5}{t-2} \ .
\]
La règle de dérivation des fonctions composées a été utilisée pour
calculer la dérivée de $\ln(t^2-5)$.
}

\compileSOL{\SOLUb}{\ref{5Q27}}{
\subQ{a}
$\displaystyle h'(\theta) = \sec(\theta)(\sec^2(\theta)+\tan^2(\theta))$

\subQ{c} $\displaystyle f'(t) = \frac{\cos(x)}{4+\sin(x)}$

\subQ{d} $\displaystyle g(\theta) = \frac{\sin(\theta) + \theta\cos(\theta)}
{\sqrt{2\theta\, \sin(\theta)}}$

\subQ{e} Nous avons
$\displaystyle F(\theta)=f_1(f_2(f_3(\theta)))$ où
$f_1(z) = z^2$, $f_2(y)=\tan(y)$ et $f_3(\theta) = \sin(\theta)$.  Ainsi,
\[
F'(\theta) = f_1'(f_2(f_3(\theta)))\ f_2'(f_3(\theta))\ f_3'(\theta) \ .
\]
Or, $f_1'(z) = 2z$, $f_2'(y) = \sec^2(y)$ et $f_3'(\theta) = \cos(\theta)$.
Nous obtenons donc
\[
F'(\theta) = 2\tan(\sin(\theta)) \; \sec^2(\sin(\theta)) \; \cos(\theta) \ .
\]

\subQ{f} Nous avons $g(x) = f(x)\, h(x)$ où $f(x)=x$ et $h(x)= \arctan(x^2)$.
Nous allons utiliser la règle pour la dérivée d'un produit.  Notons
que $h(x)$ est une composition de fonctions et que
\[
h'(x) = \left( \dfdx{\arctan(y)}{y}\bigg|_{y=x^2}\right)\dfdx{x^2}{x}
= \frac{1}{1+y^2}\bigg|_{y=x^2} (2x) = \frac{2x}{1+x^4} \ .
\]
Nous obtenons
\[
g'(x) = f'(x) h(x) + f(x)h'(x) = \arctan(x^2) + x \left(\frac{2x}{1+x^4}\right)
= \arctan(x^2) + \frac{2x^2}{1+x^4} \ .
\]

\subQ{g}
\[
h'(\theta) = \left(\dfdx{\sin(\theta)}{\theta}\right)\cos(\theta)
+\sin(\theta)\left( \dfdx{\cos(\theta)}{\theta}\right) =
\cos^2(\theta)  - \sin^2(\theta)
\]

\subQ{h} Nous avons $\cos(2x-1) = f_1(f_2(x))$ où $f_1(y)=\cos(y)$ et
$f_2(x) = 2x-1$.  Donc
\[
\dfdx{\cos(2x-1)}{x} = f_1'(f_2(x)) f_2'(x) =
-\sin(2x-1)\, (2) = -2\sin(2x-1)
\]
et ainsi
\[
h'(x) = \dfdx{\left(3 + \cos(2x-1)\right)}{x}
= \dfdx{\cos(2x-1)}{x} = -2\sin(2x-1) \ .
\]

\subQ{i} Nous avons $e^{\cos(z)} = f_1(f_2(z))$ où $f_1(y) = e^y$ et
$f_2(z) = \cos(z)$.  Puisque $f_1'(y) = e^y$ et $f_2'(x) = -\sin(x)$,
nous obtenons
\[
g'(z) = \dfdx{\left( e^{\cos(z)}\right)}{z} = f_1'(f_2(z)) f_2'(z)
= e^{\cos(z)} (-\sin(z)) = - e^{\cos(z)}\sin(z) \ .
\]

\subQ{j} Si nous n'utilisons pas la formule pour calculer la dérivée de la
sécante, nous pouvons procéder comme suit.  Nous avons
\[
f(\theta) = \sec(\theta) = \left(\cos(\theta)\right)^{-1}
= f_1(f_2(\theta)) \ .
\]
où $f_1(y)=y^{-1}$ et $f_2(\theta) = \cos(\theta)$.  Puisque
$f_1'(y) = - y^{-2}$ et $f_2'(\theta) = -\sin(\theta)$, nous trouvons
\begin{align*}
f'(\theta) &= f_1'(f_2(\theta)) f_2'(\theta)
= -\left(\cos(\theta)\right)^{-2} \left(-\sin(\theta)\right)
= \frac{\sin(\theta)}{\left(\cos(\theta)\right)^2} \\
&= \left(\frac{\sin(\theta)}{\cos(\theta)}\right)
\left( \frac{1}{\cos(\theta)} \right)
= \tan(\theta) \sec(\theta) \ .
\end{align*}

\subQ{k} Nous avons $f(x) = g(h(k(x)))$ où
$k(x) = x^{100}+1$, $h(y) = \sin(y)$
et $g(z) = e^z$.  Donc $f'(x) = g'(h(k(x)))h'(k(x))k'(x)$.
Or $g'(z) = e^z$, $h'(y) = \cos(y)$ et $k'(x) = 100 x^{99}$.  Donc
\[
f'(x) = 100 \, x^{99} \, \cos(x^{100}+1) \, e^{\sin(x^{100}+1)} \ .
\]

\subQ{m} Puisque
$\displaystyle f(t) = \ln\left(2t-\cos(6t)\right)-\ln(t)$, nous trouvons
\[
f'(t) = \frac{2+6\sin(6t)}{2t-\cos(6t)} - \frac{1}{t} \ .
\]
}

\compileSOL{\SOLUb}{\ref{5Q28}}{
\subI{i} Avec la règle de la dérivée d'un quotient, nous avons
\begin{align*}
\dfdx{\left( \frac{1}{1+e^x} \right)}{x} &=
\frac{\displaystyle  \left(\dfdx{1}{x}\right) (1+e^x) -
(1) \left(\dfdx{(1+e^x)}{x}\right)}{(1+e^x)^2} \\
&= \frac{0\times  (1+e^x) - e^x}{(1+e^x)^2} = \frac{-e^x}{(1+e^x)^2} \ .
\end{align*}

\subI{ii} Pour utiliser la règle de la dérivée de fonctions composées,
notons que
\[
\frac{1}{1+e^x} = (1+e^x)^{-1} = f_1(f_2(x))
\]
où $f_1(y)=y^{-1}$ et $f_2(x)= 1+e^x$.  Puisque $f_1'(y) = - y^{-2}$
et $f_2'(x) = e^x$, nous obtenons
\[
\dfdx{\left( \frac{1}{1+e^x} \right)}{x} =
\dfdx{\left( (1+e^x)^{-1} \right)}{x} = f_1'(f_2(x))f_2'(x)
= -(1+e^x)^{-2} e^x = \frac{-e^x}{(1+e^x)^2} \ .
\]
}

\compileSOL{\SOLUa}{\ref{5Q29}}{
\subI{i}
Avec les identités pour les fonctions logarithmiques, nous avons
$f(x) = \ln(7x) = \ln(7) + \ln(x)$.  Donc
$f'(x) = 1/x$.

\subI{ii} Pour utiliser la règle des fonctions composées, notons que
$\displaystyle \ln(7x) = f_1(f_2(x))$ où
$f_1(y)=\ln(y)$ et $f_2(x)= 7x$.  Donc
\[
\dfdx{\left( \ln(7y) \right)}{x} = f_1'(f_2(x))f_2'(x)
= \frac{1}{y}\bigg|_{y=7x} \times 7
= \frac{1}{7x} \times 7 = \frac{1}{x} \ .
\]
}

\compileSOL{\SOLUa}{\ref{5Q30}}{
\subI{i} Nous avons la règle
$\displaystyle \dfdx{(a^x)}{x} = a^x \ln(a)$ pour $a>0$.
Donc $\displaystyle \dfdx{(7^x)}{x} = 7^x \ln(7)$.

\subI{ii} Pour utiliser la règle des fonctions composées, notons que
$\displaystyle 7^x = e^{\ln(7^x)} = e^{x\ln(7)} = f_1(f_2(x))$ où
$f_1(y)= e^y$ et $f_2(x)= x\ln(7)$.  Donc
\[
\dfdx{\left( 7^x \right)}{x} = f_1'(f_2(x))f_2'(x)
= e^y \big|_{y=x\ln(7)} \ln(7) 
= e^{x\ln(7)} \ln(7) = e^{\ln(7^x)}\ln(7) = 7^x\ln(7) \ .
\]
}

\compileSOL{\SOLUb}{\ref{5Q31}}{
\subI{i}
Puisque $\cos(2\theta) = \cos^2(\theta) - \sin^2(\theta)$, nous avons
\begin{align*}
\dfdx{\left(\cos(2\theta)\right)}{\theta}
&= \dfdx{\left(\cos^2(\theta)\right)}{\theta} -
\dfdx{\left(\sin^2(\theta)\right)}{\theta}
= -2\cos(\theta)\sin(\theta) - 2 \cos(\theta)\sin(\theta) \\
&= -4 \cos(\theta)\sin(\theta) = - 2\sin(2\theta) \ .
\end{align*}
Nous avons utilisé la règle de la dérivée d'un produit pour calculer
\begin{align*}
\dfdx{\left(\cos^2(\theta)\right)}{\theta} &=
2\cos(\theta) \; \dfdx{\cos(\theta)}{\theta}
= -2\cos(\theta)\sin(\theta)
\intertext{et}
\dfdx{\left(\sin^2(\theta)\right)}{\theta} &=
2\sin(\theta) \; \dfdx{\sin(\theta)}{\theta}
= 2\cos(\theta)\sin(\theta) \ .
\end{align*}
Nous aurions pu calculer la dérivée de $\cos^2(\theta)$ et $\sin^2(\theta)$ à
l'aide de la règle de la dérivée de fonctions composées.

\subI{ii} Puisque $\cos(2\theta) = f_1(f_2(\theta))$ où
$f_2(y) = \cos(y)$ et $f_2(\theta) = 2\theta$, nous avons
\[
f'(\theta) = f_2'(f_1(z)) f_1'(z) = - \sin(y)\big|_{y=2\theta} (2)
= -\sin(2\theta) \, (2) = -2\sin(2\theta) \ .
\]
}

\compileSOL{\SOLUb}{\ref{5Q32}}{
\subQ{a} Puisque $h'(x) = f'(g(x))g'(x)$, nous avons
$h'(1) = f'(g(1))g'(1) = f'(2)\times 6 = 5\times 6 = 30$.

\subQ{b} Puisque $h'(x) = g'(f(x))f'(x)$, nous avons
$h'(1) = g'(f(1))f'(1) = g'(3) \times 4 = 9 \times 4 = 36$.

\subQ{c} Puisque
$\displaystyle h'(x) = \frac{f'(x)g(x) - f(x)g'(x)}{g^2(x)}$, nous avons\\
$\displaystyle h'(1) = \frac{f'(1)g(1) - f(1)g'(1)}{g^2(1)}
= \frac{4 \times 2 - 3\times 6}{2^2} = -\frac{5}{2}$.

\subQ{d} Puisque $f(h(x)) = x$, nous avons que $f'(h(x))h'(x) = 1$.  Ainsi,
$\displaystyle h'(1) = \frac{1}{f'(h(1))}$.  Or $h(1) = f^{-1}(1)$ est
la valeur qu'il faut donner à $f$ pour obtenir $1$.  Donc, $h(1)=2$.
Nous avons que $\displaystyle h'(1)=\frac{1}{f'(2)} = \frac{1}{5}$.
}

\compileSOL{\SOLUa}{\ref{5Q33}}{
Nous avons $f'(x)>0$ pour tout $x$ car $f$ est une fonction strictement
croissante.  Ainsi, $\displaystyle g'(x) = \frac{-f'(x)}{f^2(x)} < 0$
pour tout $x$ car $f(x)>0$ pour tout $x$ par hypothèse.
}

\compileSOL{\SOLUb}{\ref{5Q34}}{
\subQ{a}\\
\subI{i} Puisque
\[
y=2+x^3 \Leftrightarrow x^3 = y-2 \Leftrightarrow x = (y-2)^{1/3} \ ,
\]
l'inverse de $f$ est $f^{-1}(x) = (x-2)^{1/3}$.  Ainsi, la règle des
fonctions composées donne
\[
\dfdx{\left( f^{-1}(x) \right)}{x} = \dfdx{(x-2)^{1/3}}{x}
= \frac{1}{3} (x-2)^{-2/3} \ .
\]

\subI{ii} Puisque $f(f^{-1}(x)) = x$ pour tout $x$, nous avons que
\[
1 = \dfdx{(x)}{x} = \dfdx{ \left( f(f^{-1}(x)) \right)}{x}
= f'(f^{-1}(x)) \dfdx{\left( f^{-1}(x)\right)}{x} \ .
\]
Donc
\[
\dfdx{\left( f^{-1}(x)\right)}{x} = \frac{1}{f'(f^{-1}(x))}
\]
où $f'(f^{-1}(x)) \neq 0$.  Puisque $f'(x) = 3x^2$, nous obtenons
\[
\dfdx{\left( f^{-1}(x)\right)}{x} = \frac{1}{3((x-2)^{1/3})^2}
= \frac{1}{3(x-1)^{2/3}} = \frac{1}{3} (x-2)^{-2/3} \ .
\]
}

\compileSOL{\SOLUb}{\ref{5Q35}}{
\subQ{a} Puisque $\displaystyle y'= \frac{3x^2}{x^3-7}$, nous avons que
$y'(2) = 12$.  L'équation de la droite tangente dans la forme
point-pente est $y - 0 = 12(x -2)$.  Donc $y = 12 x -24$.

\subQ{b} Puisque $\displaystyle y'= \cos(\sin(x))\cos(x)$, nous avons que
$y'(\pi) = -1$.  L'équation de la droite tangente dans la forme
point-pente est $y - 0 = -1(x - \pi)$.  Donc $y = -x +\pi$.
}

\compileSOL{\SOLUb}{\ref{5Q36}}{
\subQ{a} La courbe coupe l'axe des $x$ au point $x=-2$.  La pente est
donnée par $f'(-2) = 3 (-2)^2 = 12$.

\subQ{b} La droite tangente passe par le point $(-2,0)$ et a une pente
de $f'(-2) = 12$.  L'équation de la droite tangente est donnée par
$( y - 0 ) = f'(-2) ( x - (-2) )$; c'est-à-dire, $y = 12x + 24$.

\subQ{c} La droite perpendiculaire a une pente $m$ telle que
$m \times f'(-2) = -1$.  Donc $m=-1/12$.  La droite doit passer par
le point $(-2,0)$.  L'équation de cette droite est donc
$(y - 0) = (-1/12) ( x - (-2))$; c'est-à-dire, $y = -x/12 - 1/6$.
}

\compileSOL{\SOLUb}{\ref{5Q37}}{
Puisque $f'(x) = \sqrt{3} + 2 \cos(x)$, nous avons que $f'(x) = 0$ lorsque
$\displaystyle \cos(x) = -\frac{\sqrt{3}}{2}$.  Donc
$\displaystyle x = \pm \frac{2\pi}{3} + 2n \pi$ pour $n \in \ZZ$.
}

\compileSOL{\SOLUb}{\ref{5Q38}}{
Nous avons $A'(r) = 2\pi r$.
\PDFgraph{5_derivees/disk}

Par définition, nous avons
\begin{align*}
A'(r) &= \lim_{\Delta r\to 0} \frac{A(r+\Delta r)-A(r)}{\Delta r}
= \lim_{\Delta r\to 0} \frac{\pi(r+\Delta r)^2-\pi r^2}{\Delta r} \\
&= \lim_{\Delta r\to 0} \frac{2 \pi r \Delta r  + \pi (\Delta r)^2}
{\Delta r} = \lim_{\Delta r\to 0} \left( 2 \pi r + \pi \Delta r\right)
= 2 \pi r \ .
\end{align*}
Puisque nous divisons une aire en m$^2$ (i.e. $\pi(r+\Delta r)^2-\pi r^2$)
par une longueur en m (i.e. $\Delta r$), nous obtenons que $A'(r)$ est en
m.  Ce sont les unités de $2\pi r$.

Notez que lorsque $\Delta r \to 0$, l'anneau \lgm tend\rgm\ vers un
cercle de rayon $r$ qui a une circonférence de $2\pi r$.
}

\compileSOL{\SOLUa}{\ref{5Q39}}{
Nous cherchons $c \in [0,2]$ tel que
\[
2c = f'(c) = \frac{f(2)-f(0)}{2-0} = \frac{4-0}{2-0} = 2 \ .
\]
Donc $c=1$.  C'est la valeur $c$ qui est prédite par le Théorème de la
moyenne.
}

%%% Local Variables: 
%%% mode: latex
%%% TeX-master: "notes"
%%% End: 

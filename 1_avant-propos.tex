\nonumchapter{\AP}

Le contenu de ce manuel couvre la grande majorité des sujets présentés
dans les cours de calcul différentiel et intégral pour les étudiants
\footnote{Dans ce manuel, le genre non marqué, c'est-à-dire le
masculin, quand il est employé pour désigner des personnes, renvoie
aussi bien à des femmes qu'à des hommes.}
en sciences et génies.  Les seuls préalables sont les mathématiques
normalement enseignées au secondaire en Ontario.  En fait, le premier
chapitre revoit rapidement tous les concepts mathématiques que le
lecteur devrait connaître à la sortie du secondaire.  Certains
concepts qui ne sont pas présentés au secondaire ont aussi été ajoutés
au chapitre~\ref{ChapFonct}.  Il est donc fortement suggérer au
lecteur de revoir le chapitre~\ref{ChapFonct} même s'il est
confiants de bien connaître les mathématiques de base.

Afin que le lecteur puisse bien comprendre le matériel présenté dans
ce manuel, nous avons porté un très grand soin à fournir tous les
détails et calculs algébriques qui supportent les énoncés
mathématiques.  De cette façon, nous espérons que le lecteur ne sera pas
nuit dans son apprentissage de la matière par des détails techniques.
De plus, cela permet au lecteur de développer ses capacités à
exécuter correctement des calculs algébriques.

Nous avons aussi choisi d'inclure les solutions détaillés d'un grande
nombre de problèmes à la fin du manuel.  Le lecteur devrait
sérieusement essayer de résoudre les problèmes à la fin de chaque
chapitre avant de regarder les solutions.  Nous n'apprenons pas à
jouer de la guitare en regardant un guitariste jouer.  Il faut
sérieusement pratiquer au point d'avoir le bout des doigts sensibles.
C'est la même chose pour les mathématiques.  Ce n'est pas en lissant
simplement les solutions des problèmes produites par une autre
personne que nous apprenons à résoudre des problèmes.  Il faut en
résoudre soit même.

Une autre raison pour justifier l'inclusion d'un grande nombre de
solutions est qu'il est maintenant facile d'accéder à un logiciel qui
peut presque complètement résoudre les problèmes ou d'utiliser
certains sites Internet pour demander à d'autres personnes de
résoudre les problèmes.  C'est la meilleure façon pour ne rien
apprendre.  C'est sans compter que bien souvent les solutions
fournies sur ces sites sont erronées.  Donc, si vraiment vous ne
pouvez pas résoudre un problème après avoir essayé pour une bonne
période de temps, il est préférable de regarder la
solution à la fin du manuel pour obtenir une solution correcte; en
assumant que l'auteur n'ait pas fait lui-même d'erreurs. 

Ce manuel peut être utilisées pour trois des variantes des cours de
calcul différentiel et intégral que nous retrouvons dans la majorité des
universités en Ontario.

{\renewcommand{\labelitemi}{\textbullet}
\begin{itemize}
\item Les items marqués par le symbole \eng sont spécifiquement pour
les cours de {\bfseries Calcul différentiel et intégral pour les 
étudiants en génie}.
\item Les items marqués par le symbole \life sont spécifiquement pour
les cours de {\bfseries Calcul différentiel et intégral pour les
étudiants en sciences de la vie}.
\item Les items marqués par le symbole \eco sont spécifiquement pour
les cours de {\bfseries Calcul différentiel et intégral pour les
étudiants en administration}.
\end{itemize}
}

Les items qui n'ont aucun de ces symboles sont requis pour les trois
variantes des cours de calcul différentiel et intégral.  Le contenu
pour les étudiants en administration est probablement à un niveau plus
avancé que celui du contenu normalement présenté à ce groupe d'étudiants.

L'ordre dans lequel le matériel est présenté est standard.  Il y a
cependant certaines additions au matériel que nous ne retrouvons
traditionnellement pas dans un manuel de calcul différentiel et
intégral à ce niveau.  Une introduction à l'algèbre linéaire est
fournie au chapitre~\ref{chapALLin}.  Le matériel présenté dans ce
chapitre est tout ce que le lecteur a besoin de connaître en algèbre
linéaire pour l'étude des fonctions de plusieurs variables aux
chapitres~\ref{chapFonctPlusVar}, \ref{chapDerFonctMult}, \ref{chapIntMult}
et \ref{chapCalcVect}; et l'étude des systèmes d'équations
différentielles au chapitre~\ref{ChapSystEquDiff}.   Donc, aucun cours
d'algèbre linéaire n'est préalable aux derniers chapitres de ce
manuel.  Une application concrète de l'algèbre linéaire est fournie
par les chaînes de Markov à la section~\ref{sectChainseMarkov}.
Un chapitre comme le chapitre~\ref{ChapSystEquDiff} sur les
systèmes d'équations différentielles n'est généralement pas inclus
dans un manuel de calcul différentiel et intégral.  En particulier, lorsque
l'emphase de l'étude des systèmes d'équations différentielles est sur
l'aspect qualitatif de ces systèmes.

Les chapitres~\ref{ChapSuitesSeries} et \ref{chapLimites} couvrent
certains sujets présentés au secondaire.  Cependant, plusieurs sujets
qui ne sont pas abordés au secondaire font parties de ces chapitres.
La matière des cours de calcul différentiel et intégral au niveau
universitaire débute donc avec certaines sections de ces chapitres
selon le cours de calcul différentiel et intégral auxquels l'étudiant
est inscrit.

Comme les principaux utilisateurs de ces notes depuis plus de dix ans
ont été les étudiants inscrits aux cours de Calcul différentiel et
intégral pour les sciences de la vie à l'Université d'Ottawa, il n'est
pas surprenant qu'une emphase toute particulière a été apporté à ce
sujet.  Cela se traduit par plusieurs exemples en biologie et
médecine, en particulier dans les premiers chapitres.

Le cours de calcul différentiel et intégral pour les sciences de la
vie comporte deux principales parties.  La première partie
est dédiée au calcul différentiel pour les fonctions d'une variable et
culmine avec l'étude des \lgm systèmes dynamiques discrets\rgm\ à la
section~\ref{Sect_SDD}.  Les systèmes dynamiques discrets sont
utilisés entre autre pour décrire certaines caractéristiques des
populations animales mesurées à intervalles réguliers.
La deuxième partie du cours de calcul différentiel et intégral pour
les sciences de la vie est dédié au calcul intégral.  Le but ultime de
cette partie est l'étude des \lgm systèmes dynamiques (continus)\rgm\ 
au chapitre~\ref{ChapSystEquDiff}.  Les systèmes dynamiques continues
sont des systèmes d'équations différentielles.  Ils sont utilisés
entre autre pour modéliser les réactions chimiques, la croissance des
populations et leurs mouvements, etc.

\noindent {\bfseries Avertissement}: Les modèles biologiques utilisés
dans les exemples et les questions ne représentent pas toujours des
situations réelles.  Pour obtenir des modèles mathématiques qui soient
intéressants et utilisent la théorie présentée dans les notes, tout en
étant accessibles pour le niveau du cours, nous avons dû créer des
modèles qui ne sont pas basés sur des données scientifiques.  Nous
avons quand même essayé d'obtenir des modèles qui soient qualitativement
valables.

Alors que l'emphase est misse sur les applications à la biologie au
début du manuel, notre attention se tourne vers les applications à la
physique lorsque nous avançons dans la matière pour porter exclusivement
sur les applications à la physique lorsque nous abordons l'étude des
fonctions de plusieurs variables.  En fait, à partir du
chapitre~\ref{chapFonctPlusVar} qui introduit les fonctions de
plusieurs variables, nous nous attendons à ce que le lecteur est
une plus grande maturité mathématique.

Le manuel contient deux courts chapitres, le
chapitre~\ref{CHAPvecteurs} sur les vecteurs et le
chapitre~\ref{cha[ReprParam} sur les représentations paramétriques de
courbes dans le plan et dans l'espace.  Le
chapitre~\ref{CHAPvecteurs} sur les vecteurs contient beaucoup plus
d'information que nécessaire.  Cependant, ce chapitre est une bonne
préparation pour l'étude des fonctions de plusieurs variables.  En
particulier, il permet au lecteur de développer ses aptitudes à
visualiser les objets en trois dimensions; ce qui sera très utile lors
l'étude de l'intégral double et triple au chapitre~\ref{chapIntMult}.
Le chapitre~\ref{cha[ReprParam} est fondamental pour l'étude du calcul
vectoriel au chapitre~\ref{chapCalcVect} et des systèmes d'équations
différentielles au chapitre~\ref{ChapSystEquDiff}.

Le chapitre~\ref{chapSerEnt} sur les séries entières est probablement
le plus théorique des chapitres.  Cela n'est pas surprenant dû à la
nature du sujet.  Le but principal de ce chapitre est de déterminer la
convergence des séries entières.  Il n'en reste pas moins que ce
chapitre est fondamental pour ceux qui poursuivront l'étude des
équations différentielles comme c'est le cas pour plusieurs des
domaines du génie.

Nous avons limité au minimum le contenu du chapitre~\ref{chapCalcVect} sur
le calcul vectoriel.  Nous avons adopté une approche intuitive à ce sujet.
Il aurait été bénéfique d'aborder plus en détails les opérations sur
les courbes (additions et différences) pour créer de nouvelles
courbes, et l'interprétation de ces opérations en termes d'opérations
sur les intégrales le long de courbes.  Ce chapitre ouvre la porte au
vaste sujet de la géométrie différentielle.  Nous nous sommes retenu pour
ne pas déborder do contenu d'un cours de calcul différentiel et intégral
standard.

\section*{Théorie}

Les items marqués par le symbole \theory sont spécifiquement pour les
lecteurs intéressés à la théorie et la rigueur en mathématique.
Plusieurs de ces items demandent une connaissance du concept de \lgm
démonstration\rgm\ que la majorité des étudiants du secondaire
n'auront probablement pas vu.  Ces items sont généralement optionnels.
Ils sont pour les lecteurs curieux qui voudrait en savoir plus sur les
méthodes enseignées en classe.

Il est fort probable que la majorité des étudiants en science de la
vie et même en génie ne liront pas les sections plus théoriques.
Malheureusement, pour eux, les mathématiques se limitent
à une suite de méthodes et trucs pour résoudre certains problèmes.  Ce
manuel contient ces méthodes et trucs.  Cependant, les sections plus
théoriques donnent un avant goût de se que sont les mathématiques.  Nous
espérons que certains lecteurs seront intéressés par la rigueur
mathématiques et voudront poursuive cette direction dans leur études.

Il y a plusieurs mentions du nombre d'Euler $e$.  Nous donnons plusieurs
façons équivalentes de le définir.  Nous mentionnons indirectement le lien
entre le nombre $e$ et les fonctions trigonométriques, donc le nombre
$\pi$, lors d'une des démonstrations de la règle d'addition pour les
fonctions trigonométriques $\sin$ et $\cos$.  Nous espérons que les
étudiants seront fascinés par le fait que les nombres $e$ et $\pi$
sont reliés.

\section*{Notation}

\begin{defn*}
Les ensembles suivants seront fréquemment utilisés dans le présent
document.
\begin{itemize}
\item $\NN = \{0,1,2,3,\ldots \}$ est l'ensemble des nombres naturels.
\item $\NNp = \{1,2,3,\ldots \}$ est l'ensemble des nombres naturels
positifs.
\item $\ZZ = \{0,1,-1,2,-2,3,-3, \ldots \}$ est l'ensemble des nombres
entiers.
\item $\QQ$ est l'ensemble des nombres rationnels; c'est-à-dire, les
  nombres de la forme $n/m$ où $n,m \in \ZZ$ et $m \neq 0$.
\item $\RR$ est l'ensemble des nombres réels.
\end{itemize}
\end{defn*}

En français, la virgule est utilisée pour séparer la partie entière de
la partie décimale d'un nombre et nous utilisons un espace pour séparer
les multiples de $10^3$.  Ainsi,
\[
105\ 456\ 263,456 =
105\times 10^6+ 456\times 10^3 + 263 + \frac{456}{1000} \ .
\]
Cependant, dans le présent document, nous utiliserons la notation
anglaise.  Le point sépare la partie entière de la partie décimale d'un
nombre et la virgule sépare les multiples de $10^3$.  Nous écrivons donc
\[
105,456,263.456 = 105\times 10^6+ 456\times 10^3 + 263 + \frac{456}{1000} \ .
\]
La raison principale de ce choix est que la grande majorité des
données utilisées dans ce document proviennent de documents en anglais
qui utilisent cette notation.  De plus, les logiciels sont tous en
anglais et utilisent aussi cette notation.  Pour être consistent, nous
avons donc choisi d'utiliser la notation anglaise.

\section*{Avertissement}

Les notes de cours que vous avez devant vous représente un ouvrage
inachevé, qui est en constante évolution.  Il ne faut donc pas être
surpris d'y retrouver des fautes d'orthographe, des coquilles, etc.
Les corrections seront apportées au cours du temps suite aux
commentaires des lecteurs.  L'auteur prend entière responsabilité
des erreurs; comment pourrait-il faire autrement?

\section*{Remerciements}

J'aimerais remercier Yves Bourgault, Wadii Hajji et Monica Nevins,
des collègues de travail, dont les suggestions ont améliorées la
présentation de la matière.

De plus, j'aimerais remercier Paméla Touchette-Giroux pour son
excellent travail de révision des textes pour la première édition de
ces notes.  Il n'en reste pas moins que je prends toute la
responsabilité pour les fautes présentes dans le texte.

%%% Local Variables: 
%%% mode: latex
%%% TeX-master: "notes"
%%% End: 

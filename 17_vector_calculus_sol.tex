\section{Calcul vectoriel}

\subsection{Intégrale le long d'une courbe}

\compileSOL{\SOLUb}{\ref{17Q1}}{
Nous avons que $\sigma(0) = (1,1,1)$ et $\sigma(1) = (e, e^{-1}, e^2)$.
Ainsi, $0 \leq t \leq 1$ et
\begin{align*}
\int_C F \cdot \dx{\VEC{s}} &=
\int_0^1 F(\sigma(t)) \cdot \sigma'(t) \dx{t} \\
&= \int_0^1 (e^{3t}, e^{-t} + e^{2t}, e^t)\cdot
(e^t, -e^{-t}, 2e^{2t}) \dx{t}
= \int_0^1 \left( e^{4t} -e^{-2t} - e^t + 2e^{3t}\right) \dx{t} \\
&= \left( \frac{1}{4}e^{4t} +\frac{1}{2}e^{-2t}
- e^t + \frac{2}{3}e^{3t}\right)\bigg|_0^1
=  \frac{1}{4}e^4 +\frac{1}{2}e^{-2} - e + \frac{2}{3}e^{3}
- \frac{5}{12} \ .
\end{align*}
}

\compileSOL{\SOLUb}{\ref{17Q2}}{
Partageons le trajet en deux sections: $C_1$ est le quart de
cercle de rayon $2$ centré à l'origine de $(-2,0)$ et $(0,2)$, et
$C_2$ est la segment de droite de $(0,2)$ à $(3,1)$. 

Nous choisissons une représentation paramétrique pour chacune des sections.
Pour $C_1$,
\[
  (x,y) = (2\cos(\pi-\theta),2\sin(\pi-\theta))
  = (-2 \cos(\theta),2 \sin(\theta))
\]
pour $0\leq \theta \leq \pi/2$.  Pour $C_2$,
\[
  (x,y) = (2-t, 3t)
\]
pour $0 \leq t \leq 1$. Ainsi,
\begin{align*}
\int_{C_1} \sqrt{y} x \dx{x} + 2xy \dx{y}
&= - 4\sqrt{2} \int_0^{\pi/2} \cos(\theta) (\sin(\theta))^{3/2}
    \dx{\theta}
- 16 \int_0^{\pi/2} \sin(\theta) \cos^2(\theta) \dx{\theta} \\
&= -\frac{8\sqrt{2}}{5} (\sin(\theta))^{5/2}\bigg|_0^{\pi/2}
+ \frac{16}{3} \cos^3(\theta)\bigg|_0^{\pi/2}
= - \frac{16}{3} - \frac{8\sqrt{2}}{5}
\end{align*}
et
\begin{align*}
\int_{C_2} \sqrt{y} x \dx{x} + 2xy \dx{y} &=
-\sqrt{3} \int_0^1 \sqrt{t}(2-t)\dx{t}
+ 18 \int_0^1 (2-t)t\dx{t} \\
&= -\sqrt{3} \int_0^1 \left(2 t^{1/2} - t^{3/2}\right) \dx{t}
+ 18 \int_0^1 \left(2t-t^2\right) \dx{t} \\
&= -\sqrt{3} \left(\frac{4}{3} t^{3/2} - \frac{2}{5}t^{5/2}\right)\bigg|_0^1
+ 18 \left( t^2 - \frac{t^3}{3} \right)\bigg|_0^1 
= -\frac{14\sqrt{3}}{15} + 12 \ .
\end{align*}
Nous avons donc
\begin{align*}
\int_C \sqrt{y} x \dx{x} + 2xy \dx{y} &=
\int_{C_1} \sqrt{y} x \dx{x} + 2xy \dx{y}
+ \int_{C_2} \sqrt{y} x \dx{x} + 2xy \dx{y} \\
&= \frac{20}{3} -\frac{14\sqrt{3}}{15} - \frac{8\sqrt{2}}{5} \ .
\end{align*}
}

\compileSOL{\SOLUb}{\ref{17Q4}}{
Nous pouvons partager le trajet en trois segments de droite: $C_1$ de
$(0,0,0)$ à $(0,3,0)$, $C_2$ de $(0,3,0)$ à $(2,3,1)$, et $C_3$
de $(2,3,1)$ à $(2,0,0)$.

Nous choisissons une représentation paramétrique pour chacun des segments.
Pour $C_1$,
\[
  (x,y,z) = (0, t, 0)
\]
pour $0\leq t \leq 3$.  Pour $C_2$,
\[
  (x,y,z) = (2t, 3, t)
\]
pour $0 \leq t \leq 1$.  Finalement, pour $C_3$,
\[
  (x,yz) = (2,3-3t, 1-t)
\]
pour $0 \leq t \leq 1$.  Ainsi,
\[
\int_{C_1} F \cdot \dx{s}
= \int_{C_1} yz \dx{x} + xz \dx{y} + y \dx{z}
= \int_0^1 0 \dx{t} = 0 \ .
\]
Ceci n'est pas surprenant car la force $F(x,y,z) = (0,0,y)$ est
perpendiculaire à la direction du déplacement sur $C_1$.
\[
\int_{C_2} F \cdot \dx{s}
= \int_{C_2} yz \dx{x} + xz \dx{y} + y \dx{z}
= \int_0^1 (6t+3) \dx{t} = 6 \ .
\]
Finalement,
\[
\int_{C_3} F \cdot \dx{s}
= \int_{C_3} yz \dx{x} + xz \dx{y} + y \dx{z}
= -9 \int_0^1 (1-t) \dx{t} = -\frac{9}{2} \ .
\]
Nous avons donc
\[
\int_C F \cdot \dx{s} =
\int_{C_1} F \cdot \dx{s} + \int_{C_2} F \cdot \dx{s}
+ \int_{C_3} F \cdot \dx{s}
= \frac{3}{2} \ .
\]
}

\subsection{Rotationnel et divergence}

\compileSOL{\SOLUb}{\ref{17Q6}}{
\subQ{a} Nous avons $F = F_1 \ii + F_2 \jj + F_3 \kk$ où
$F_1(x,y,z) = e^x\cos(z)$, $F_2(x,y,z) = y$ et
$F_3(x,y,z) = -e^x\sin(z)$.  Donc
\begin{align*}
\curl\,F & = \left( \pdydx{F_3}{y} - \pdydx{F_2}{z} \right) \ii +
\left( \pdydx{F_1}{z} - \pdydx{F_3}{x} \right) \jj +
\left( \pdydx{F_2}{x}- \pdydx{F_1}{y} \right) \kk \\
&= \left( 0 - 0 \right) \ii + \left(-e^x\sin(z) + e^x \sin(z) \right) \jj +
\left(0  - 0\right) \kk = \VEC{0} \ .
\end{align*}
Ce qui confirme que $F$ est conservateur grâce à la
proposition~\ref{rotFzero}.  De plus, $F = \nabla f$ pour une fonction $f$.

\subQ{b} Si $f:\RR^3 \to \RR$ satisfasse $F = \nabla f$, alors
\[
  \pdydx{f}{x} = F_1(x,y,z) = e^x\cos(z) \Rightarrow
  f(x,y,z) = e^x\cos(z) + g(y,z)
\]
pour une certaine fonction $g:\RR^2 \to \RR$.  De plus,
\[
  \pdydx{f}{y} = F_2(x,y,z) = y \Rightarrow
  \pdydx{g}{y}(y,z) = y \Rightarrow g(y,z) = \frac{y^2}{2} + h(z)
\]
pour une certaine fonction $h:\RR \to \RR$.  Nous avons donc
$\displaystyle f(x,y,z) = e^x\cos(z) + \frac{y^2}{2} + h(z)$.
Finalement,
\[
  \pdydx{f}{z} = F_3(x,y,z) \Rightarrow
  -e^x \sin(z) + h'(z) = -e^x \sin(z) \Rightarrow
  h'(z) = 0 \Rightarrow h(z) = C
\]
où $C$ est une constante d'intégration.  La fonction cherchée est
$\displaystyle f(x,y,z) = e^x\cos(z) + \frac{y^2}{2} + C$.
}

\compileSOL{\SOLUb}{\ref{17Q7}}{
\subQ{a} Nous avons que $F = F_1 \ii + F_2\jj$ où $F_1(x,y) = e^{2y}$ et
$F_2(x,y) = 1 + 2x e^{2y}$.  Grâce à la proposition~\ref{rotFzero}
(exemple~\ref{rotFzeroD2}), il suffit de vérifier que
$\displaystyle \pdydx{F_2}{x} = \pdydx{F_1}{y}$ pour démontrer que $F$
est conservateur.  Puisque
\[
  \pdydx{F_2}{x} = \pdfdx{\left(1 + 2x e^{2y}\right)}{x} = 2 e^{2y}
\quad \text{et} \quad
  \pdydx{F_1}{y} = \pdfdx{\left( e^{2y} \right)}{y} = 2 e^{2y} \ ,
\]
nous avons bien que $\displaystyle \pdydx{F_2}{x} = \pdydx{F_1}{y}$ et
$F$ est conservateur.  Il découle du théorème~\ref{Fnablaf1}
que $F = \nabla f$ pour une fonction $f:\RR^2 \to \RR$.  En effet,
\[
  \pdydx{f}{x} = F_1(x,y) = e^{2y} \Rightarrow f(x,y) = x e^{2y} + g(y)
\]
pour une certaine fonction $g:\RR \to \RR$.  Ainsi,
\[
  \pdydx{f}{y} = F_2(x,y) = 1 + 2x e^{2y} \Rightarrow
  2x e^{2y} + g'(y) = 1 + 2 x e^{2y} \Rightarrow
  g'(y) = 1 \Rightarrow g(y) = y + C
\]
où $C$ est une constante d'intégration.  La fonction cherchée est
$f(x,y) = x e^{2y} + y + C$.  Comme pour les intégrales
définies, nous pouvons poser $C=0$.

Si $\Gamma$ est une courbe quelconque qui va du point
$\sigma(0) = (0,1)$ au point $\sigma(1) = (e,2)$, donc en particulier
la courbe $C$, nous avons que
\[
  \int_{\Gamma} F \cdot \dx{\VEC{s}}
  = \int_{\Gamma} \nabla f \cdot \dx{\VEC{s}}
  = f(e,2)-f(0,1) = e^5 - 1
\]
grâce au Théorème fondamental des intégrales le long de courbes.

\subQ{c} Nous avons que $F = F_1 \ii + F_2\jj$ où $F_1(x,y) = 2x+y^2+3x^2y$ et
$F_2(x,y) = 2xy + x^3 + 3y^2$.  Grâce à la proposition~\ref{rotFzero}
(exemple~\ref{rotFzeroD2}), il suffit de vérifier que
$\displaystyle \pdydx{F_2}{x} = \pdydx{F_1}{y}$ pour démontrer que $F$ est
conservateur.  Puisque
\[
  \pdydx{F_2}{x} = \pdfdx{\left(2xy + x^3 + 3y^2\right)}{x} = 2y +3x^2
\quad \text{et} \quad
  \pdydx{F_1}{y} = \pdfdx{\left(2x+y^2+3x^2y\right)}{y} = 2y + 3x^2 \ ,
\]
nous avons bien que $\displaystyle \pdydx{F_2}{x} = \pdydx{F_1}{y}$ et $F$
est conservateur.  Il découle du théorème~\ref{Fnablaf1}
que $F = \nabla f$ pour une fonction $f:\RR^2 \to \RR$.  En effet,
\[
\pdydx{f}{x} = F_1(x,y) = 2x+y^2+3x^2y
\Rightarrow f(x,y) = x^2 +xy^2 + x^3y + g(y)
\]
pour une certaine fonction $g:\RR \to \RR$.  Ainsi,
\begin{align*}
\pdydx{f}{y} = F_2(x,y) = 2xy + x^3 + 3y^2 &\Rightarrow
2xy + x^3 + g'(y) = 2xy + x^3 + 3y^2 \\
& \Rightarrow g'(y) = 3y^2 \Rightarrow g(y) = y^3 + C
\end{align*}
où $C$ est une constante d'intégration.  La fonction cherchée est donc
$f(x,y) = x^2 + xy^2 +x^3y + y^3 + C$.  Comme pour les intégrales
définies, nous pouvons prendre $C=0$.

Si $\Gamma$ est une courbe quelconque qui va du point $(0,0)$ au point
$(\pi,2)$, donc en particulier la courbe $C$, nous avons que
\[
  \int_{\Gamma} F \cdot \dx{\VEC{s}}
  = \int_{\Gamma} \nabla f \cdot \dx{\VEC{s}}
  = f(\pi,0)-f(0,0) = \pi^2
\]
grâce au Théorème fondamental des intégrales le long de courbes.

\subQ{e} 
Nous avons $F = F_1\; \ii + F_2\; \jj + F_3\; \kk$ où $F_1(x,y,z) = 2xyz$,
$F_2(x,y,z) = x^2z + z\cos(yz)$ et $F_3(x,y,z) = x^2y + y \cos(yz) + 2z$.
Grâce à la proposition~\ref{rotFzero}, il suffit de vérifier que
$\displaystyle \curl F = \VEC{0}$ pour démontrer que $F$ est
conservateur.  Or,
\begin{align*}
\curl\,F & = \left( \pdydx{F_3}{y} - \pdydx{F_2}{z} \right) \ii +
\left( \pdydx{F_1}{z} - \pdydx{F_3}{x} \right) \jj +
\left( \pdydx{F_2}{x}- \pdydx{F_1}{y} \right) \kk \\
&= \big( \left(x^2 +\cos(yz) - yz\sin(yz)\right) -
\left( x^2 + \cos(yz) - yz\sin(yz)\right) \big) \ii \\
&\qquad + \left( 2xy - 2xy \right) \jj +
\left(2xz  - 2xz\right) \kk = \VEC{0} \ .
\end{align*}
Donc $F$ est conservateur.  De plus, $F = \nabla f$ pour une fonction
$f:\RR^3 \to \RR$.

Si $f:\RR^3 \to \RR$ satisfasse $F = \nabla f$, alors
\[
  \pdydx{f}{x} = F_1(x,y,z) = 2xyz \Rightarrow
  f(x,y,z) = x^2yz + g(y,z)
\]
pour une certaine fonction $g:\RR^2 \to \RR$.  Donc
\begin{align*}
  \pdydx{f}{y} = F_2(x,y,z) = x^2z + z\cos(yz) &\Rightarrow
  x^2z + \pdydx{g}{y}(y,z) = x^2z + z\cos(yz) \\
 & \Rightarrow
  \pdydx{g}{y}(y,z) = z\cos(yz) \Rightarrow
  g(y,z) = \sin(yz) + h(z)
\end{align*}
pour une certaine fonction $h:\RR \to \RR$.  Nous avons donc
$\displaystyle f(x,y,z) = x^2yz  +\sin(yz) + h(z)$.  Finalement,
\begin{align*}
  \pdydx{f}{z} = F_3(x,y,z) &\Rightarrow
  x^2y + y \cos(yz) + h'(z) = x^2y + y \cos(yz) + 2z \\
&\Rightarrow  h'(z) = 2z \Rightarrow h(z) = z^2 + C
\end{align*}
où $C$ est une constante d'intégration.  La fonction cherchée est donc
$\displaystyle f(x,y,z) = x^2yz  +\sin(yz) + z^2 + C$.
Nous pouvons poser $C=0$.

Si $\Gamma$ est une courbe quelconque qui va du point $(0,0,0)$ au point
$(2,\pi,1/2)$, nous avons que
\[
  \int_{\Gamma} F \cdot \dx{\VEC{s}}
  = \int_{\Gamma} \nabla f \cdot \dx{\VEC{s}}
  = f(2,\pi,1/2)-f(0,0,0) = 2\pi + \frac{5}{4}
\]
grâce au Théorème fondamental des intégrales le long de courbes.
}

\compileSOL{\SOLUb}{\ref{17Q8}}{
Les champs de vecteurs qui sont de divergence nulle sont B, C et D. 
Le volume d'une petite région ne change pas sous l'influence du champ
de vecteurs.

Les champs de vecteurs qui sont irrotationel sont A et D.  Une petite
région ne tourne pas sur elle même sous l'influence du champ 
de vecteurs.
}

\subsection{Intégrale sur une surface}

\compileSOL{\SOLUb}{\ref{17Q9}}{
\PDFgraph{17_vector_calculus/surfQ6}
Une représentation paramétrique de la surface $S$ est donnée par
\[
  (x,y,z) = (2\cos(\theta), y, 2\sin(\theta))
\]
pour $(y,\theta) \in D = \left\{ (y,\theta) : 0\leq \theta < 2\pi \
\text{et} \ 0 \leq y \leq 4 -x = 4 - 2\cos(\theta) \right\}$.
Puisque
\begin{align*}
\dx{S} &= \left\| \left( \pdydx{x}{y} , \pdydx{y}{y} ,
\pdydx{z}{y} \right) \times
\left( \pdydx{x}{\theta} , \pdydx{y}{\theta} , \pdydx{z}{\theta} \right) 
\right\| \dx{A}
= \left\| (0, 1, 0) \times
\left( -2 \sin(\theta) , 0, 2 \cos(\theta) \right) \right\| \dx{A}\\
&= \left\| \left( 2 \cos(\theta) , 0 , 2 \sin(\theta)
 \right) \right\| \dx{A}
= \sqrt{ 4 \cos^2(\theta) + 4 \sin^2(\theta) } \dx{A}
= 2 \dx{A} \ ,
\end{align*}
nous avons
\begin{align*}
\iint_S xy \dx{S} &= \iint_D  4 y \cos(\theta) \dx{A}
= \int_0^{2\pi} \int_0^{4-2\cos(\theta)} 4 y \cos(\theta) \dx{y} \dx{\theta} \\
&= \int_0^{2\pi} \left( 2y^2 \cos(\theta)\right)\bigg|_{y=0}^{4 -2\cos(\theta)}
\dx{\theta}
= \int_0^{2\pi} 8(2-\cos(\theta))^2\cos(\theta) \dx{\theta} \\
&= 8\int_0^{2\pi} \left( 4\cos(\theta) -4 \cos^2(\theta) + \cos^3(\theta)
\right) \dx{\theta} \\
& = -16 \int_0^{2\pi}\left ( 1 + \cos(2\theta)\right) \dx{\theta}
= -16 \left(\theta + \frac{1}{2} \sin(2\theta)\right)\bigg|_0^{2\pi}
= -32 \pi
\end{align*}
car $\displaystyle \int_0^{2\pi} \cos(\theta) \dx{\theta} = 0$
et $\displaystyle \int_0^{2\pi} \cos^3(\theta) \dx{\theta} = 0$.
}

\compileSOL{\SOLUa}{\ref{17Q10}}{
L'intersection de la sphère et du cylindre est l'ensemble des points
$(x,y,z)$ qui satisfont
\[
x^2 + y^2 + z^2 = 4 \text{ et } x^2 + y^2 = 2 \Rightarrow
2 + z^2 = 4 \text{ et } x^2 + y^2 = 2 \Rightarrow z^2 = 2 \text{ et }
x^2 + y^2 = 2 \ .
\]
Puisque nous considérons la partie supérieure de la sphère, nous obtenons
$z = \sqrt{2}$.  L'intersection est donc le cercle $x^2 + y^2 = 2$ avec
$z= \sqrt{2}$.

Le dessin de la surface $S$ est donné ci-dessous
\PDFgraph{17_vector_calculus/surfQ2}
Une représentation paramétrique de la surface $S$ est donnée par
\[
  (x,y,z) = (2 \sin(\phi) \cos(\theta), 2 \sin(\phi) \sin(\theta),
  2 \cos(\phi))
\]
pour $(\theta, \phi) \in D = \{ (\theta,\phi) : 0 \leq \theta < 2\pi
\ \text{et} \ 0 \leq \phi \leq \pi/4\}$.
Puisque
\begin{align*}
\dx{S} &= \left\| \left( \pdydx{x}{\theta} , \pdydx{y}{\theta} ,
\pdydx{z}{\theta} \right) \times
\left( \pdydx{x}{\phi} , \pdydx{y}{\phi} , \pdydx{z}{\phi} \right) 
\right\| \dx{A}\\
&= \left\| \left(-2\sin(\phi)\sin(\theta), 2\sin(\phi)\cos(\theta), 0 \right)
\times
\left( 2\cos(\phi)\cos(\theta) , 2\cos(\phi)\sin(\theta) ,
-2 \sin(\phi) \right) \right\| \dx{A}\\
&= \left\| \left( -4 \sin^2(\phi)\cos(\theta) ,
-4 \sin^2(\phi)\sin(\theta) , -4 \sin(\phi)\cos(\phi) \right) \right\| \dx{A}\\
&= \sqrt{ 16 \sin^2(\phi)} \dx{A}
= 4 \sin(\phi) \dx{A}
\end{align*}
car $\sin(\phi) \geq 0$ pour $0 \leq \phi \leq \pi/4$, nous avons
\begin{align*}
\iint_S (x^2+y^2)z \dx{S} &= \iint_D  32 \sin^3(\phi)\cos(\phi) \dx{A}
= \int_0^{2\pi} \int_0^{\pi/4} 32 \sin^3(\phi)\cos(\phi) \dx{\phi} \dx{\theta} \\
&= \int_0^{2\pi} \left( 8 \sin^4(\phi) \bigg|_{\phi=0}^{\pi/4}\right) \dx{\theta}
= 2 \int_0^{2\pi} \dx{\theta}
= 4 \pi \ .
\end{align*}
}

\compileSOL{\SOLUb}{\ref{17Q11}}{
Le dessin de l'entonnoir est donné ci-dessous.
\PDFgraph{17_vector_calculus/MassQ1}
Une représentation paramétrique de $S$ est donnée par
\[
  (x,y,z) = (r \cos(\theta), r \sin(\theta), 2r)
\]
pour
$(r,\theta) \in D = \{ (r,\theta) : 1 \leq r \leq 3 \ \text{et}
\ 0 \leq \theta < 2\pi\}$.
Chaque tranche horizontal de $S$ est un cercle de rayon $r$.  Puisque
\begin{align*}
\dx{S} &= \left\| \left( \pdydx{x}{r} , \pdydx{y}{r} , \pdydx{z}{r} \right)
\times \left( \pdydx{x}{\theta} , \pdydx{y}{\theta} ,
\pdydx{z}{\theta} \right)\right\| \dx{A} \\
&= \left\|\left( \cos(\theta), \sin(\theta), 2 \right)
\times
\left( -r\sin(\theta) , r\cos(\theta) , 0 \right) \right\|\dx{A}
= \left\| \left( -2r\cos(\theta) , -2r \sin(\theta) , r\right)
\right\| \dx{A} \\
& = 3 r \dx{A}  \ ,
\end{align*}
la masse est donnée par 
\begin{align*}
\iint_S \rho \dx{S} &= \iint_D 3( 8 -2 r)r \dx{A}
= 6\int_0^{2\pi} \int_1^3 (4r - r^2) \dx{r} \dx{\theta} \\
&= 6\int_0^{2\pi} \left( 2r^2 - \frac{r^3}{3}\right)\bigg|_{r=1}^3 \dx{\theta}
= 44 \int_0^{2\pi} \dx{\theta} = 88 \pi \ .
\end{align*}
}

\compileSOL{\SOLUa}{\ref{17Q12}}{
\PDFgraph{17_vector_calculus/surfQ4}
Une représentation paramétrique de la surface $S$ est donnée par
\[
  (x,y,z) = (r\cos(\theta), r^2, r \sin(\theta))
\]
pour $(r,\theta) \in D = \{ (r,\theta) : 0 \leq r \leq 3 \ \text{et}
\ 0\leq \theta < 2\pi\}$.  Puisque
\begin{align*}
\dx{S} &= \left\| \left( \pdydx{x}{r} , \pdydx{y}{r} ,
\pdydx{z}{r} \right) \times
\left( \pdydx{x}{\theta} , \pdydx{y}{\theta} , \pdydx{z}{\theta} \right) 
\right\| \dx{A}\\
&= \left\| \left( cos(\theta), 2r, \sin(\theta) \right)
\times
\left( -r \sin(\theta), 0, r \cos(\theta) \right) \right\| \dx{A}\\
&= \left\| \left( 2 r^2 \cos(\theta), -r, 2 r^2 \sin(\theta)
 \right) \right\| \dx{A}\\
&= \sqrt{ r^2 + 4r^4 \cos^2(\theta) + 4r^4 \sin^2(\theta) }\ \dx{A}
= r \sqrt{1 + 4r^2}\ \dx{A}
\end{align*}
car $r \geq 0$, l'aire de $S$ est 
\begin{align*}
\iint_S \dx{S} &= \iint_D  r \sqrt{1 + 4r^2} \dx{A}
= \int_0^{2\pi} \int_0^3 r (1 + 4r^2)^{1/2} \dx{r} \dx{\theta} \\
&= \int_0^{2\pi} \left( \frac{ (1+4r^2)^{3/2}}{12}
 \bigg|_{r=0}^3\right) \dx{\theta}
= \frac{37^{3/2}-1}{12} \int_0^{2\pi} \dx{\theta}
= \frac{(37^{3/2}-1)\pi}{6} \ .
\end{align*}
}

\compileSOL{\SOLUb}{\ref{17Q14}}{
Le solide est représenté dans la figure ci-dessous.
\PDFgraph{17_vector_calculus/surfQ1}
Il faut premièrement trouver l'intersection de la sphère
$x^2 + y^2 + z^2 =8$ et du paraboloïde $z = (x^2 + y^2)/2$.  Après
avoir substitué l'expression pour $z$ donnée par le paraboloïde dans
l'équation de la sphère, nous obtenons
\begin{align*}
x^2 + y^2 + \left( \frac{x^2+y^2}{2}\right)^2 = 8 &\Rightarrow
(x^2 + y^2) + \frac{1}{4} (x^2 + y^2)^2 = 8 \\
&\Rightarrow (x^2 + y^2)^2 + 4(x^2 + y^2) - 32 = 0 \\ 
&\Rightarrow \left((x^2 + y^2)+8\right)\left((x^2 + y^2) -4\right) = 0
\ .
\end{align*}
Puisque $x^2 + y^2 \geq 0$, la seule solution possible est
$x^2 + y^2 = 4$.   Dans ce cas $z = (x^2 + y^2)/2 = 2$.
L'intersection est le cercle $x^2+y^2=4$ avec $z=2$.

Divisons la surface $S$ en deux sections: la surface supérieure du
solide, dénoté $S_1$, qui est formée par la partie de la sphère
$x^2 + y^2 + z^2 = 8$ et surface inférieure du solide, dénotée $S_2$,
qui est formée par la partie du paraboloïde $z = (x^2 + y^2)/2$.

Commençons par la surface supérieure $S_1$.  Une
représentation paramétrique est donnée par
\[
  (x,y,z) = (r\cos(\theta), r\sin(\theta), \sqrt{8 - r^2})
\]
pour $(t,\theta) \in D = \{ (r,\theta) : 0 \leq r \leq 2 \ \text{et}
\ 0 \leq \theta < 2\pi\}$.  Puisque
\begin{align*}
\dx{S} &= \left\| \left( \pdydx{x}{r} , \pdydx{y}{r} , \pdydx{z}{r} \right)
\times
\left( \pdydx{x}{\theta} , \pdydx{y}{\theta} , \pdydx{z}{\theta} \right) 
\right\| \dx{A}\\
&= \left\| \left(\cos(\theta) , \sin(\theta) , \frac{-r}{\sqrt{8-r^2}} \right)
\times
\left( -r\sin(\theta) , r \cos(\theta) , 0 \right) \right\| \dx{A}\\
&= \left\| \left( \frac{-r^2 \cos(\theta)}{\sqrt{8 - r^2}} ,
\frac{-r^2 \sin(\theta)}{\sqrt{8 - r^2}} , r \right) \right\| \dx{A}\\
&= \sqrt{ \frac{r^4 \cos^2(\theta)}{8 - r^2} +
\frac{r^4 \sin^2(\theta)}{8 - r^2} + r^2}\ \dx{A}
= \frac{2\sqrt{2}\ r}{\sqrt{8 - r^2}} \dx{A} \ ,
\end{align*}
l'aire de la surface supérieure $S_1$ est donnée par
\begin{align*}
\iint_{S_1} \dx{S} &= \iint_D \frac{2\sqrt{2}\ r}{\sqrt{8 - r^2}} \dx{A}
= \int_0^{2\pi} \int_0^2 \frac{2\sqrt{2}\ r}{\sqrt{8 - r^2}} \dx{r}
\dx{\theta} \\
& \int_0^{2\pi} \left( -2\sqrt{2} \sqrt{8 - r^2}\bigg|_{r=0}^2\right) \dx{\theta}
= 4(2 - \sqrt{2}) \int_0^{2\pi} \dx{\theta} = 8 \pi (2 - \sqrt{2}) \ .
\end{align*}

Pour la surface inférieure $S_2$, une représentation
paramétrique est donnée par
\[
  (x,y,z) = (r\cos(\theta), r\sin(\theta), r^2/2)
\]
pour $(r,\theta) \in D$ défini ci-dessus.  Puisque
\begin{align*}
\dx{S} &= \left\| \left( \pdydx{x}{r} , \pdydx{y}{r} , \pdydx{z}{r} \right)
\times
\left( \pdydx{x}{\theta} , \pdydx{y}{\theta} , \pdydx{z}{\theta} \right) 
\right\| \dx{A}\\
&= \left\| \left(\cos(\theta) , \sin(\theta) , r \right)
\times
\left( -r\sin(\theta) , r \cos(\theta) , 0 \right) \right\| \dx{A}\\
&= \left\| \left( -r^2 \cos(\theta) , -r^2 \sin(\theta) ,
  r \right) \right\| \dx{A}\\
&= \sqrt{ r^4 \cos^2(\theta) + r^4 \sin^2(\theta) + r^2}\ \dx{A}
= r\sqrt{r^2+1} \dx{A} \ ,
\end{align*}
l'aire de la surface inférieure $S_2$ est donnée par
\begin{align*}
\iint_{S_2} \dx{S} &= \iint_D r\sqrt{r^2+1} \dx{A}
= \int_0^{2\pi} \int_0^2 r\sqrt{r^2+1} \dx{r} \dx{\theta} \\
&= \int_0^{2\pi} \left( \frac{1}{3}
(r^2+ 1)^{3/2}\bigg|_{r=0}^2\right) \dx{\theta}
= \frac{1}{3}(5\sqrt{5}-1) \int_0^{2\pi} \dx{\theta}
= \frac{2\pi}{3}(5\sqrt{5}-1) \ .
\end{align*}
L'aire total de la surface $S$ du solide est
$\displaystyle 8 \pi (2 - \sqrt{2}) + \frac{2\pi}{3}(5\sqrt{5}-1)$.
}

\compileSOL{\SOLUa}{\ref{17Q15}}{
Le dessin de la partie de la surface $S$ qui se trouve dans la région
définie par $x,y,z \geq 0$ est donné ci-dessous.
\PDFgraph{17_vector_calculus/surfQ5}
La surface $S$ est formée de huit pièces de cette forme.
Une représentation paramétrique est donnée par
\[
  (x,y,z) = (a\cos(\theta), y, a\sin(\theta))
\]
pour $0 \leq \theta < 2\pi$ et
$0 \leq y \leq \sqrt{a^2 - x^2} = a\sin(\theta)$.  Puisque
\begin{align*}
\dx{S} &= \left\| \left( \pdydx{x}{\theta} , \pdydx{y}{\theta} ,
\pdydx{z}{\theta} \right) \times
\left( \pdydx{x}{y} , \pdydx{y}{y} , \pdydx{z}{y} \right) 
\right\| \dx{A}
= \left\| \left(-a \sin(\theta) , 0, a\cos(\theta) \right)
\times \left( 0, 1, 0 \right) \right\| \dx{A}\\
&= \left\| \left( -a \cos(\theta) , 0 , -a \sin(\theta) \right)
 \right\| \dx{A}
= a \dx{A} \ ,
\end{align*}
l'aire de la surface $S$ est donnée par
\begin{align*}
\iint_S \dx{S} &= 8 \int_0^{\pi/2} \int_0^{a\sin(\theta)} a \dx{y} \dx{\theta}
= 8\int_0^{\pi/2} \left( a y\bigg|_{y=0}^{a\sin(\theta)}\right) \dx{\theta} \\
&= 8a^2 \int_0^{2\pi} \sin(\theta)\dx{\theta}
= -8a^2 \cos(\theta)\bigg|_0^{\pi/2} = 8a^2 \ .
\end{align*}
}

\compileSOL{\SOLUb}{\ref{17Q16}}{
Le dessin de la surface de la sphère pour $z >0$ qui se
trouve à à l'intérieur du cylindre est donné ci-dessous.
\PDFgraph{17_vector_calculus/surfQ8}
Il y a une partie identique pour $z<0$.

L'équation $x^2 + y^2 = ax$ représente bien un cylindre car
\[
  x^2 + y^2 = ax \Leftrightarrow \left( x - \frac{a}{2}\right)^2 + y^2
  = \frac{a^2}{4} \  .
\]
C'est un cylindre de rayon $a/2$ dont l'axe est la droite parallèle à
l'axe des $z$ qui passe pas $(a/2,0,0)$.  La première représentation
paramétrique à laquelle nous pourrions pensée est celle donnée par les
coordonnées cylindriques centrées sur la droite parallèle à l'axe des
$z$ qui passe par le point $(a/2,0,0)$; c'est-à-dire,
$x = a/2 + r \cos(\theta)$, $y = r\sin(\theta)$ et
$z = (a^2 - (a/2+r\cos(\theta))^2 - r^2 \sin^2(\theta))^{1/2}$
pour $0\leq \theta \leq 2\pi$ et $0 \leq r \leq a/2$.
Malheureusement, ce n'est pas un bon choix.  Le calcul de l'intégrale
est beaucoup plus compliqué avec cette représentation qu'avec la
représentation paramétrique en coordonnées cylindrique centrées sur
l'axe des $z$.  C'est cette dernière représentation que nous allons
utiliser.  La représentation paramétrique est donnée par
\[
  (x,y,z) = \rho(\theta, r) = (r\cos(\theta), r\sin(\theta),  \sqrt{a^2 - r^2} )
\]
pour $(\theta,r) \in D = \{ (\theta,r) : -\pi/2 \leq \theta \leq \pi/2
\ \text{et} \ 0 \leq r \leq a \cos(\theta)\}$.
Pour déterminer les bornes pour $r$, notons que $x^2 + y^2 = ax$ avec
$x=r\cos(\theta)$ et $y=r\sin(\theta)$ donne $r^2 = a r\cos(\theta)$ et
donc $r=a\cos(\theta)$.  Puisque
\begin{align*}
\dx{S} &= \left\| \left( \pdydx{\rho_1}{\theta} , \pdydx{\rho_2}{\theta} ,
\pdydx{\rho_3}{\theta} \right) \times
\left( \pdydx{\rho_1}{y} , \pdydx{\rho_2}{y} , \pdydx{\rho_3}{y} \right) 
\right\| \dx{A} \\
&= \left\| \left(-r \sin(\theta) , r\cos(\theta), 0 \right)
\times \left( \cos(\theta), \sin(\theta), \frac{-r}{\sqrt{a^2-r^2}}
\right) \right\| \dx{A} \\
&= \left\| \left( -\frac{r^2\cos(\theta)}{\sqrt{a^2-r^2}},
-\frac{r^2\sin(\theta)}{\sqrt{a^2-r^2}}, -r \right)
\right\| \dx{A}
= \frac{ar}{\sqrt{a^2-r^2}} \dx{A} \ ,
\end{align*}
l'aire de la surface $S$ est donnée par
\begin{align*}
\iint_S \dx{S} &= 2 \int_{\pi/2}^{\pi/2} \int_0^{a\cos(\theta)}
\frac{ar}{\sqrt{a^2-r^2}} \dx{r} \dx{\theta}
= -2 \int_{\pi/2}^{\pi/2} \left(a\sqrt{a^2-r^2}\bigg|_0^{a\cos(\theta)}
\right) \dx{\theta} \\
&= -2 \int_{\pi/2}^{\pi/2} \left(a^2\sqrt{\sin^2(\theta)} - a^2
\right) \dx{\theta}
=  -4 a^2 \int_0^{\pi/2} \left(\sqrt{\sin^2(\theta)} - 1\right) \dx{\theta} \\
&=  -4 a^2\int_0^{\pi/2} \left(\sin(\theta) - 1\right) \dx{\theta}
= 4 a^2\left( \cos(\theta) + \theta \right) \bigg|_0^{\pi/2}
= 4 a^2 \left(\frac{\pi}{2} -1\right)
\end{align*}
car $a^2\sqrt{\sin^2(\theta)} - a^2$ est une fonction paire et
$\sin(\theta) \geq 0$ pour $0 \leq \theta \leq \pi/2$.
}

\compileSOL{\SOLUb}{\ref{17Q18}}{
Le dessin de la surface $S$ est donné ci-dessous.
\PDFgraph{17_vector_calculus/surfQ7}
Une représentation paramétrique est donné par
\[
(x,y,z) = (2\cos(\theta)\sin(\phi), 2\cos(\phi),
2\sin(\theta)\sin(\phi))
\]
pour $0 \leq \phi \leq \pi/2$ et $0 \leq \theta < 2\pi$.  Soit
\begin{align*}
\VEC{m} &= \left( \pdydx{x}{\theta} , \pdydx{y}{\theta} ,
\pdydx{z}{\theta} \right)
\times \left( \pdydx{x}{\phi} , \pdydx{y}{\phi} ,
\pdydx{z}{\phi} \right) \\
&= \left( -2 \sin(\theta)\sin(\phi), 0, 2\cos(\theta)\sin(\phi) \right)
\times
\left( 2\cos(\theta)\cos(\phi) , -2\sin(\phi) , 2\sin(\theta)\cos(\phi)
\right) \\
&= \left( 4\cos(\theta)\sin^2(\phi) , 4 \sin(\phi)\cos(\phi),
4 \sin(\theta)\sin^2(\phi) \right) \ .
\end{align*}
Puisque $4 \sin(\phi)\cos(\phi) >0$ pour $0\leq \phi \leq \pi/2$, le
vecteur $\displaystyle \VEC{m}$ est un vecteur normal à la surface $S$
qui pointe dans la direction de l'orientation de la surface $S$.
Notre représentation paramétrique est donc compatible avec
l'orientation sur $S$.  Le flux est
\begin{align*}
\iint_S F \cdot \dx{\VEC{S}}
&= \int_0^{2\pi}\int_0^{\pi/2} F\left(
2\cos(\theta)\sin(\phi),2\cos(\phi),2\sin(\theta)\sin(\phi)\right) \\
&\cdot \left( 4\cos(\theta)\sin^2(\phi) , 4 \sin(\phi)\cos(\phi) ,
4 \sin(\theta)\sin^2(\phi)  \right) \dx{\phi}\dx{\theta} \\
&= \int_0^{2\pi}\int_0^{\pi/2} \left(
4\cos(\theta)\sin(\phi)\cos(\phi), 4\cos^2(\phi),
4\sin(\theta)\cos(\phi)\sin(\phi)\right) \\
&\cdot \left( 4\cos(\theta)\sin^2(\phi) , 4 \sin(\phi)\cos(\phi) ,
4 \sin(\theta)\sin^2(\phi)  \right) \dx{\phi}\dx{\theta} \\
&= 16 \int_0^{2\pi}\int_0^{\pi/2} \left(
\sin^3(\phi)\cos(\phi) + \cos^3(\phi) \sin(\phi) \right)
\dx{\phi}\dx{\theta} \\
&= 4 \int_0^{2\pi} \left( \sin^4(\phi)
- \cos^4(\phi) \right)\bigg|_{\phi=0}^{\pi/2}  \dx{\theta}
= 8 \int_0^{2\pi} \dx{\theta} = 16 \pi \ .
\end{align*}
}

\compileSOL{\SOLUb}{\ref{17Q20}}{
Le dessin de la surface $S$ est donné ci-dessous.
\PDFgraph{17_vector_calculus/surfQ3}
Soit
\begin{align*}
\VEC{m} &= \pdydx{\rho}{u} \times \pdydx{\rho}{v}
= \left( \cos(v), \sin(v), 0 \right) \times
\left( -u\sin(v) , u\cos(v) , 1 \right) \\
& = \left( \sin(v) , -\cos(v) , u \right) \ .
\end{align*}
Puisque $u \geq 0$, le vecteur $\VEC{m}$
est un vecteur normal à la surface $S$ qui pointe dans la direction de
l'orientation de la surface $S$.  La représentation paramétrique est
compatible avec l'orientation sur la surface $S$.
Le flux est donc
\begin{align*}
\iint_S F \cdot \dx{\VEC{S}} &=  
\iint_D \left( u\sin(v) , u\cos(v), v\right) \cdot
\left( \sin(v) , -\cos(v), u\right) \dx{A} \\
&= \int_0^{2\pi} \int_0^1 \left( u\sin^2(v) - u\cos^2(v) + uv \right)
\dx{u} \dx{v}
= \int_0^{2\pi} \int_0^1 \left( -u\cos(2v) + uv \right)\dx{u} \dx{v} \\
&= \int_0^{2\pi} \left(-\cos(2v) + v \right)
\left(\frac{u^2}{2}\bigg|_{u=0}^1\right) \dx{v}
= \frac{1}{2} \int_0^{2\pi} \left(-\cos(2v) + v \right) \dx{v} \\
&= \frac{1}{2} \left( \frac{1}{2}\sin(2v) + \frac{v^2}{2}\right)\bigg|_0^{2\pi}
= \pi^2 \ .
\end{align*}
}

\compileSOL{\SOLUb}{\ref{17Q21}}{
Le dessin de la partie de la surface $S$ pour $x \geq 0$ est donné ci-dessous.
\PDFgraph{17_vector_calculus/surfQ9}
Il y a une partie semblable pour $x \leq 0$.

Une représentation paramétrique est donnée par
\[
  (x,y,z) = (a\cos(\theta), y, a\cos(\theta))
\]
pour $0 \leq \theta < 2\pi$ et
$-a |\cos(\theta)| \leq y \leq a |\cos(\theta)|$.  Pour déterminer
le domaine d'intégration de $y$, notons que l'intersection des
deux cylindres est donnée par
\[
  y^2 + z^2 = a^2 = x^2 + z^2 \Rightarrow y^2 = x^2 \Rightarrow |y| =
  |x| \ .
\]
% L'intersection des deux cylindre est donc formée des deux courbes
% fermées possédant les représentations paramétriques
% \[
% \sigma_1(\theta) = (a\cos(\theta), a|\cos(\theta)|, a\sin(\theta) )
% \quad \text{et} \quad
% \sigma_2(\theta) = (a\cos(\theta), -a|\cos(\theta)|, a\sin(\theta) )
% \]
% pour $0 \leq \theta < 2\pi$.

Soit
\begin{align*}
\VEC{q} &= \left( \pdydx{x}{\theta} , \pdydx{y}{\theta} ,
\pdydx{z}{\theta} \right)
\times \left( \pdydx{x}{y} , \pdydx{y}{y} , \pdydx{z}{y} \right)
= \left( -a \sin(\theta), 0, a\cos(\theta) \right)
\times \left(0 , 1, 0 \right) \\
&= \left( -a\cos(\theta) , 0, -a \sin(\theta) \right) \ .
\end{align*}
Le vecteur $\displaystyle \VEC{q}$ est un vecteur normal à la surface $S$
qui pointe dans la direction opposée à la direction de l'orientation
de la surface $S$.  Il faut donc utiliser $\VEC{m} = - \VEC{q}$ pour
respecter l'orientation sur la surface $S$ ou remplacer la
représentation paramétrique par
\[
  (x,y,z) = (a\cos(-\theta), y, a\cos(-\theta)) \ .
\]
Ce qui revient au même.

Puisque $\cos$ et $\sin$ sont des fonctions périodique de période
$2\pi$, le flux est donné par
\[
\iint_S F \cdot \dx{\VEC{S}}
= \int_{-\pi/2}^{3\pi/2}\int_{-a|\cos(\theta)|}^{a|\cos(\theta)|}
F\left(a\cos(\theta),y,a\sin(\theta)\right)
\cdot \left( a\cos(\theta), 0 , a \sin(\theta) \right)
\dx{y}\dx{\theta} \ .
\]
Pour calculer l'intégrale, nous devons diviser le
domaine d'intégration pour $\theta$ afin de traiter les bornes
d'intégration $\pm a|\cos(\theta)|$ pour $y$.
Puisque $|\cos(\theta)| = \cos(\theta)$
pour $-\pi/2 < \theta < \pi/2$ et $|\cos(\theta)| = -\cos(\theta)$ pour
$\pi/2 < \theta < 3\pi/2$, l'intégrale
pour $-\pi/2 < \theta \leq 3\pi/2$ sera la somme de
l'intégrale pour $-\pi/2 < \theta \leq \pi/2$ et de l'intégrale
pour $\pi/2 < \theta \leq 3\pi/2$.  Nous avons donc
\begin{align*}
\iint_S F \cdot \dx{\VEC{S}}
&=\int_{-\pi/2}^{\pi/2}\int_{-a\cos(\theta)}^{a\cos(\theta)}
F\left(a\cos(\theta),y,a\sin(\theta)\right)
\cdot \left(a\cos(\theta), 0 , a \sin(\theta) \right) \dx{y}\dx{\theta} \\
&\quad + \int_{\pi/2}^{3\pi/2}\int_{a\cos(\theta)}^{-a\cos(\theta)}
F\left(a\cos(\theta),y,a\sin(\theta)\right)
\cdot \left( a\cos(\theta), 0 , a \sin(\theta) \right) \dx{y}\dx{\theta} \ .
\end{align*}
La première intégrale est
\begin{align*}
&  \int_{-\pi/2}^{\pi/2}\int_{-a\cos(\theta)}^{a\cos(\theta)}
F\left(a\cos(\theta),y,a\sin(\theta)\right)
\cdot \left( a\cos(\theta), 0 , a \sin(\theta) \right) \dx{y}\dx{\theta} \\
& \qquad = \int_{-\pi/2}^{\pi/2}\int_{-a\cos(\theta)}^{a\cos(\theta)}
\left(a\cos(\theta),\cos(a\cos(\theta) y),a\sin(\theta)\right)
\cdot \left( a\cos(\theta), 0 , a \sin(\theta) \right) \dx{y}\dx{\theta} \\
& \qquad = a^2 \int_{-\pi/2}^{\pi/2}\int_{-a\cos(\theta)}^{a\cos(\theta)}
\dx{y}\dx{\theta}
= 2 a^3 \int_{-\pi/2}^{\pi/2}\cos(\theta)\dx{\theta}
= 2 a^3 \sin(\theta)\bigg|_{-\pi/2}^{\pi/2} = 4 a^3
\end{align*}
et la deuxième intégrale est
\begin{align*}
& \int_{\pi/2}^{3\pi/2}\int_{a\cos(\theta)}^{-a\cos(\theta)}
F\left(a\cos(\theta),y,a\sin(\theta)\right)
\cdot \left( a\cos(\theta), 0 , a \sin(\theta) \right) \dx{y}\dx{\theta} \\
& \qquad = \int_{\pi/2}^{3\pi/2}\int_{a\cos(\theta)}^{-a\cos(\theta)}
\left(a\cos(\theta),\cos(a\cos(\theta) y),a\sin(\theta)\right)
\cdot \left( a\cos(\theta), 0 , a \sin(\theta) \right) \dx{y}\dx{\theta} \\
& \qquad = a^2 \int_{\pi/2}^{3\pi/2}\int_{a\cos(\theta)}^{-a\cos(\theta)}
\dx{y}\dx{\theta}
= -2 a^3 \int_{\pi/2}^{3\pi/2}\cos(\theta)\dx{\theta}
= -2 a^3 \sin(\theta)\bigg|_{\pi/2}^{3\pi/2} = 4 a^3 \ .
\end{align*}
Notons que la deuxième intégrale peut être réduite à la première avec
la substitution $\theta = \phi - \pi$.  Le flux est donc
\[
  \iint_S F \cdot \dx{\VEC{S}} = 8 a^3 \ .
\]
}

\subsection{Théorèmes de Stokes et de Green}

\compileSOL{\SOLUb}{\ref{17Q23}}{
Le dessin de la région $D$ et de sa frontière $C$ est donné ci-dessous.
\PDFgraph{17_vector_calculus/GreenQ1}

Nous pourrions penser évaluer directement l'intégrale
\[
  \int_C(y+e^{x^2})\dx{x} + (2x + \cos(y^2))\dx{y}
\]
mais cela demanderait de longs et complexes calculs, sans compter le
fait que l'intégrale doit être calculée sur deux sections de la courbe
$C$, la section lorsque $y=x^2$ et celle lorsque $x=y^2$.

Une autre approche est d'utiliser le Théorème de Green.  Notons que
l'orientation de $C$ est l'orientation positive associée à la région
$D$ comme il est requis pour le Théorème de Green.  Ainsi,
\begin{align*}
\int_C(y+e^{x^2})\dx{x} + (2x + \cos(y^2))\dx{y} &=
\iint_D \left( \pdfdx{(2x + \cos(y^2))}{x} -
  \pdfdx{ (y+e^{x^2})}{y} \right) \dx{A} \\
&= \iint_D \dx{A}
= \int_0^1 \int_{x^2}^{\sqrt{x}} \dx{y}\dx{x}
= \int_0^1 \left( \sqrt{x} - x^2\right) \dx{x} \\
&= \left( \frac{2}{3} x^{3/2} - \frac{1}{3}x^3\right)\bigg|_0^1
= \frac{1}{3} \ .
\end{align*}
}

\compileSOL{\SOLUa}{\ref{17Q24}}{
Le dessin du parcoure $C$ de la particule incluant la direction de son
déplacement est donné ci-dessous.
\PDFgraph{17_vector_calculus/GreenQ2}

Le travail du champ de vecteur sur la particule est
\[
  W =  \int_C F \cdot \dx{\VEC{s}} =
  \int_Cx \dx{x} + (x^3 + 3xy^2)\dx{y} \ .
\]
Grâce au Théorème de Green, nous pouvons remplacer le calcul de cette
intégrale le long de la courbe $C$ par l'intégrale double suivante.
\[
W =  \iint_S \left( \pdfdx{\left(x^3 + 3 xy^2\right)}{x} -
  \pdfdx{ (x)}{y} \right) \dx{A}
= \iint_S \left( 3x^2 + 3y^2 \right) \dx{A} \ .
\]
Notons que l'orientation de $C$ est l'orientation positive associée à
la région $S$ comme il est requis pour le Théorème de Green.

Pour évaluer cette dernière intégrale, il est raisonnable d'utiliser
les coordonnées polaires
\[
  (x,y) = (r \cos(\theta), r \sin(\theta))
\]
pour $0 \leq r \leq 2$ et $0 \leq \theta \leq \pi$.  Puisque
$\displaystyle \frac{\partial(x,y)}{\partial(r,\theta)} = r $ et
$3x^2 + 3y^2 = 3r^2$, nous obtenons
\[
  W = 3 \int_0^{\pi} \int_0^2 r^3 \dx{r}\dx{\theta}
= 3 \int_0^{\pi} \left(\frac{r^4}{4}\right)\bigg|_0^2 \dx{\theta}
= 12 \int_0^{\pi} \dx{\theta} = 12 \pi \ .
\]
}

\compileSOL{\SOLUb}{\ref{17Q25}}{
Le dessin de la courbe $C$ est donné ci-dessous.
\PDFgraph{17_vector_calculus/StokesQ2}

Nous pourrions calculer directement cette intégrale.  Il faudrait
décomposer le rectangle $C$ en quatre segments qui connectent les
quatre points.  Nous laissons au lecteur le soin de faire ce calcul.
Il est simple.  Nous allons utiliser le Théorème de Stokes qui dit que
\[
  \int_C F \cdot \dx{\VEC{s}}
  = \iint_S \curl  F \cdot \dx{\VEC{S}}
\]
où $S$ est la région du plan $z=2$ délimitée par le rectangle $C$.
Nous avons $C = \partial S$.  L'orientation sur $S$ est dans la direction
positive de l'axe des $z$ pour qu'elle soit consistante avec la
direction de $C$.  Une représentation paramétrique de $S$ est
\[
  (x,y,z) = \rho(x,y) = (x,y,2)
\]
pour $0 \leq x,y \leq 1$.  Soit
\[
\VEC{m} = \pdydx{\rho}{x} \times \pdydx{\rho}{y}
= \left( 1, 0, 0 \right) \times \left( 0 , 1, 0 \right)
= \left( 0 , 0 , 1 \right) \ .
\]
Le vecteur $\VEC{m}$ est le vecteur normal unitaire à la surface $S$
qui pointe vers le haut, donc dans la direction de l'orientation de la
surface $S$.  Notre choix de représentation paramétrique pour $S$ est
donc consistant avec la direction de $S$.

Puisque
\[
\curl\,F = \nabla \times F
= \begin{vmatrix}
\ii & \jj & \kk \\
\displaystyle \pdydx{}{x} & \displaystyle \pdydx{}{y} &
\displaystyle \pdydx{}{z} \\
yz & xy & xz
\end{vmatrix}
= \left( 0, -z + y, y -z \right)
\]
si nous développons le déterminant selon la première ligne,
nous avons
\begin{align*}
\int_C F \cdot \dx{\VEC{s}}
&= \iint_S \curl F \cdot \dx{\VEC{S}}
= \int_0^1\int_0^1 \curl F(\rho(x,y)) \cdot
\left(\pdydx{\rho}{x} \times \pdydx{\rho}{y}\right) \dx{y}\dx{x} \\
&=\int_0^1\int_0^1 \left( 0, -2 + y, y - 2 \right) \cdot (0,0,1) \dx{y}\dx{x}
= \int_0^1\int_0^1 (y -2) \dx{y}\dx{x} \\
&= \int_0^1 \left(\frac{y^2}{2} - 2y\right)\bigg|_{y=0}^1\dx{x}
= -\frac{3}{2} \int_0^1 \dx{x}
= -\frac{3}{2} \ .
\end{align*}
}

\compileSOL{\SOLUb}{\ref{17Q26}}{
Il faut premièrement trouver l'équation du plan contenant les trois
points donnés.  Un vecteur normal $\VEC{v}$ au plan est donné par le
produit vectoriel de deux vecteurs non colinéaires $\VEC{v}_1$ et
$\VEC{v}_2$ qui sont parallèles au plan.  Un choix possible est
$\VEC{v}_1 = (1,1,1)- (0,2,0) = (1,-1,1)$ et
$\VEC{v}_2 = (-1,1,2) - (0,2,0) = (-1,-1,2)$.  Ainsi,
\[
  \VEC{v}= \VEC{v}_1 \times \VEC{v}_2
  = \det \begin{pmatrix}
\ii & \jj & \kk \\
1 & -1 & 1 \\
-1 & -1 & 2 \end{pmatrix}
= - \ii - 3\; \jj - 2\; \kk
\]
si nous développons le déterminant selon la première ligne.
L'équation du plan tangent est donc donnée par
\begin{align*}
  (-1,-3,-2)\cdot \left( (x,y,z) - (0,2,0)\right) = 0
  &\Rightarrow -x -3(y-2) -2z = 0
  \Rightarrow z = -\frac{x}{2} -\frac{3y}{2} + 3 \ .
\end{align*}

La courbe $C$ est le résultat de l'intersection du cylindre
$x^2 + y^2 = 4$ et du plane $z = -x/2 -3y/2 + 3$.   Le dessin de cette
courbe est donné ci-dessous.
\PDFgraph{17_vector_calculus/StokesQ4}
Nous pourrions calculer directement l'intégrale
$\displaystyle \int_C F \cdot \dx{\VEC{s}}$ en utilisant la
représentation paramétrique de la courbe $C$ donnée par
\[
  (x,y,z) = \rho(\theta) = (2 \cos(\theta), -2 \sin(\theta) ,
  -\frac{1}{2}\cos(\theta) + \frac{3}{2}\sin(\theta) + 3 )
\]
pour $0 \leq \theta \leq 2\pi$.  Nous laissons le soin au lecteur de
vérifier que cette représentation paramétrique est bien consistante
avec la direction sur $C$.  Nous allons utiliser le Théorème de Stokes
qui dit que
\[
  \int_C F \cdot \dx{\VEC{s}}
  = \iint_S \curl  F \cdot \dx{\VEC{S}}
\]
où $S$ est la région du plan $z= -x/2 -3y/2 + 3$ qui est contenue à
l'intérieur du cylindre $x^2 + y^2 = 4$.  L'orientation sur $S$ est
donnée par un vecteur normal qui pointe dans la direction de
$z$ négatif pour qu'elle soit consistante avec la direction sur
$C$.  Une représentation paramétrique de $S$ est
\[
  (x,y,z) = \rho(\theta,r) = (r\cos(\theta), r\sin(\theta),
-\frac{r}{2} \cos(\theta) - \frac{3r}{2} \sin(\theta) + 3)
\]
pour $0 \leq \theta < 2\pi$ et $0 \leq r \leq 2$.  Soit
\begin{align*}
\VEC{m} &= \pdydx{\rho}{\theta} \times \pdydx{\rho}{r} \\
&= \left( -r\sin(\theta), r\cos(\theta),
  \frac{r}{2}\sin(\theta) - \frac{3r}{2}\cos(\theta) \right) \times
\left( \cos(\theta) , \sin(\theta), -\frac{1}{2}\cos(\theta)
- \frac{3}{2} \sin(\theta) \right) \\
&= \left( -\frac{r}{2} , -\frac{3r}{2} , -r \right) \ .
\end{align*}
le vecteur $\VEC{m}$ est un vecteur normal à la surface $S$
qui pointe dans la direction de $z$ négatif car $r\geq 0$.
La représentation paramétrique pour $S$ est donc
consistant avec l'orientation sur $S$.

Puisque
\[
\curl F = \nabla \times F
= \begin{vmatrix}
\ii & \jj & \kk \\
\displaystyle \pdydx{}{x} & \displaystyle \pdydx{}{y} &
\displaystyle \pdydx{}{z} \\
y^2 & x & z
\end{vmatrix}
= \left( 0, 0, 1 - 2y \right)
\]
si nous développons le déterminant selon la première ligne, nous avons
\begin{align*}
\int_C F \cdot \dx{\VEC{s}}
&= \iint_S \curl  F \cdot \dx{\VEC{S}}
= \int_0^{2\pi}\int_0^2 \curl  F(\rho(\theta,r)) \cdot
\left(\pdydx{\rho}{\theta} \times \pdydx{\rho}{r}\right)
\dx{r}\dx{\theta} \\
&=\int_0^{2\pi}\int_0^2 \left( 0, 0, 1 - 2r\sin(\theta)\right)
 \cdot \left(-\frac{r}{2}, -\frac{3r}{2},-r\right) \dx{r}\dx{\theta} \\
&= \int_0^{2\pi}\int_0^2 \left(-r + 2r^2\sin(\theta)\right) \dx{r}\dx{\theta}
= \int_0^{2\pi} \left(-\frac{r^2}{2}
+ \frac{2r^3}{3}\sin(\theta) \right)\bigg|_{r=0}^2\dx{\theta} \\
& = \int_0^{2\pi} \left( -2 + \frac{16}{3} \sin(\theta)\right)\dx{\theta}
= \left( -2\theta - \frac{16}{3} \cos(\theta)\right)\bigg|_0^{2\pi} 
= - 4 \pi \ .
\end{align*}
}

\compileSOL{\SOLUa}{\ref{17Q27}}{
Le dessin de la surface $S$ avec sa frontière $C = \partial S$ est
donné ci-dessous.
\PDFgraph{17_vector_calculus/StokesQ3}
Le dessin inclus aussi le vecteur normal unitaire $\VEC{n}$ qui défini
l'orientation de la surface $S$ associée à la direction positive sur
le courbe $C$.

Nous pourrions calculer directement l'intégrale le long de la courbe $C$
en faisant la somme de l'intégrale le long de la courbe $C$ dans le
plan $z=0$, puis dans le plan $x=0$ et finalement
dans le plan $y=0$.  Ce n'est pas cette approche qui sera utilisée.
Grâce au Théorème de Stokes, nous avons que
\[
  \int_C F\cdot \dx{\VEC{s}} = \iint_S \curl\,F \cdot \dx{\VEC{S}} \ .
\]
C'est cette dernière intégrale que nous allons évaluer.

Une représentation paramétrique de $S$ est
\[
  (x,y,z) = \rho(r,\theta) = (r\cos(\theta) , r\sin(\theta), 4 - r^2)
\]
pour $0 \leq r \leq 2$ et $0 \leq \theta \leq \pi/2$.  Soit
\begin{align*}
\VEC{m} &= \pdydx{\rho}{r} \times \pdydx{\rho}{\theta}
= \left( \cos(\theta), \sin(\theta), -2r \right) \times
\left( -r\sin(\theta) , r\cos(\theta), 0 \right) \\
&= \left( 2r^2\cos(\theta) , 2r^2\sin(\theta) , r \right) \ .
\end{align*}
le vecteur $\VEC{m}$ est un vecteur normal à la surface $S$
qui pointe dans la direction de l'orientation de la
surface $S$.  Notre choix de représentation paramétrique pour $S$ est
donc consistant avec l'orientation de $S$.

Puisque
\[
\curl\,F = \nabla \times F
= \begin{vmatrix}
\ii & \jj & \kk \\
\displaystyle \pdydx{}{x} & \displaystyle \pdydx{}{y} &
\displaystyle \pdydx{}{z} \\
-yz(1-z) & xz & xyz
\end{vmatrix}
= \left( x(z-1), y(z-1), 2z -z^2 \right)
\]
si nous développons le déterminant selon la première ligne seulement,
nous avons
\begin{align*}
\int_C F \cdot \dx{\VEC{s}}
&= \iint_S \curl  F \cdot \dx{\VEC{S}}
= \iint_D \curl  F(\rho(r,\theta)) \cdot
\left(\pdydx{\rho}{r} \times \pdydx{\rho}{\theta}\right) \dx{A} \\
&=\int_0^{\pi/2}\int_0^2 \left( r\cos(\theta)(3-r^2), r\sin(\theta)
(3 -r^2) , -8 + 6r^2 -r^4 \right) \\
& \qquad \cdot (2r^2\cos(\theta), 2r^2\sin(\theta),r)
  \dx{r}\dx{\theta} \\
& = \int_0^{\pi/2}\int_0^2 (-3r^5  + 12r^3 -8r) \dx{r}\dx{\theta}
= \int_0^1 \left(-\frac{y^6}{2} + 3r^4 -4 r^2\right)\bigg|_{r=0}^2\dx{\theta}
= 0 \ .
\end{align*}
}

\compileSOL{\SOLUb}{\ref{17Q28}}{
Le dessin de la surface $S$ est donné ci-dessous.
\PDFgraph{17_vector_calculus/StokesQ5}
Grâce au Théorème de Stokes, nous avons que
\[
  \iint_S \curl  F \cdot \dx{\VEC{S}}
  = \int_C F \cdot \dx{\VEC{s}}
\]
où $C = \partial S$ est la frontière de $S$.  En fait, $C$ est le
cercle $x^2+y^2=4$ pour $z=4$.  L'orientation positive sur $C$ qui est
associée à l'orientation sur $S$ est dans le sens 
contraire aux aiguilles d'une montre quand nous regardons $C$ à partir
d'une point $(0,0,z)$ avec $z > 4$.

Une représentation paramétrique pour $C$ est
\[
  (x,y,z) = \sigma(\theta) = (2\cos(\theta), 2\sin(\theta), 4)
\]
pour $0 \leq \theta < 2\pi$.  Nous avons
\begin{align*}
\iint_S \curl  F \cdot \dx{\VEC{S}}
&= \int_C F \cdot \dx{\VEC{s}}
= \int_0^{2\pi} \sigma(\theta) \cdot \sigma'(\theta) \dx{\theta} \\
& = \int_0^{2\pi} \left(2\cos(\theta), 0, -2\sin(\theta)\right)\cdot
\left( -2\sin(\theta), 2\cos(\theta), 0\right) \dx{\theta} \\
& = -4 \int_0^{2\pi} \cos(\theta)\sin(\theta)\dx{\theta}
= -2 \sin^2(\theta)\bigg|_0^{2\pi}  = 0 \ .
\end{align*}
}

\compileSOL{\SOLUa}{\ref{17Q29}}{
Le dessin de la surface $S$ est donné ci-dessous.
\PDFgraph{17_vector_calculus/StokesQ1}

Nous pourrions calculer directement cette intégrale.  Cependant
\begin{align*}
\curl\,F &= \nabla \times F
= \begin{vmatrix}
\ii & \jj & \kk \\
\displaystyle \pdydx{}{x} & \displaystyle \pdydx{}{y} &
\displaystyle \pdydx{}{z} \\
z^3 e^{xz^2} & z^4 e^{yz^2} & xy
\end{vmatrix} \\
&= \left( x - (4z^3 +2 yz^5)e^{yz^2} ,   -y + (3z^2 +2 xz^4)e^{xz^2},
 0 \right)
\end{align*}
si nous développons le déterminant selon la première ligne seulement.
Ceci n'est pas très encourageant pour le calcul de l'intégrale.

Nous allons donc utiliser le Théorème de Stokes qui dit que
\[
  \iint_S \curl\,F \cdot \dx{\VEC{S}}
  = \int_C F\cdot \dx{\VEC{s}}
\]
où $C = \partial S$, la frontière de $S$.  Nous décomposons $C$ en deux
courbes: $C_1$ est le cercle $x^2+y^2 =4$ avec $z=2$ et $C_2$ est le
cercle $x^2+y^2 =4$ avec $z=4$.  L'orientation positive sur $C_1$ et
$C_2$ imposée par l'orientation de $S$ est indiquée dans la figure
ci-dessus; dans le sens contraire des aiguilles d'une montre sur
$C_1$ et dans le sens des aiguilles d'une montre sur $C_2$ lorsque que
vu d'un point $(0,0,z)$ avec $z > 4$.

Une représentation paramétrique pour $C_1$ est
\[
  (x,y,z) = \sigma_1(\theta) = (2 \cos(\theta), 2\sin(\theta), 2)
\]
pour $0 \leq \theta \leq 2\pi$.  Puisque
$\sigma_1'(\theta) = (-2\sin(\theta), 2\cos(\theta), 0)$,
la représentation paramétrique est consistante avec l'orientation de $C_1$.

Une représentation paramétrique pour $C_2$ est
\[
  (x,y,z) = \sigma_2(\theta) = (2 \cos(\theta), 2\sin(\theta), 4)
\]
pour $0 \leq \theta \leq 2\pi$.  Puisque
$\sigma'(\theta) = (-2\sin(\theta), 2\cos(\theta), 0)$,
cette représentation paramétrique est associée à
l'orientation inverse de $C_2$.  Il faudra donc prendre la valeur
négative de l'intégrale.

Nous avons donc
\begin{align*}
\int_C F\cdot \dx{\VEC{s}} &=
\int_{C_1} F\cdot \dx{\VEC{s}} + \int_{C_2} F\cdot
\dx{\VEC{s}} \\
&= \int_0^{2\pi} \left(
8 e^{8\cos(\theta)}, 16 e^{8\sin(\theta)}, 4\cos(\theta)\sin(\theta)\right)
\cdot (-2\sin(\theta), 2\cos(\theta), 0) \dx{\theta} \\
&\qquad - \int_0^{2\pi} \left(
64 e^{32\cos(\theta)}, 256 e^{32\sin(\theta)}, 16\cos(\theta)\sin(\theta)\right)
\cdot (-2\sin(\theta), 2\cos(\theta), 0) \dx{\theta} \\
&= -16 \int_0^{2\pi} \sin(\theta) e^{8\cos(\theta)} \dx{\theta}
+ 32 \int_0^{2\pi} \cos(\theta) e^{8\sin(\theta)} \dx{\theta} \\
& \qquad +128 \int_0^{2\pi} \sin(\theta) e^{32\cos(\theta)} \dx{\theta}
- 512 \int_0^{2\pi} \cos(\theta) e^{32\sin(\theta)}\dx{\theta} \\
&= \left(2e^{8\cos(\theta)} \right)\bigg|_0^{2\pi} 
+ \left(4 e^{8\sin(\theta)} \right)\bigg|_0^{2\pi} 
- \left( 4 e^{32\cos(\theta)} \right)\bigg|_0^{2\pi} 
- \left( 16 e^{32\sin(\theta)}\right)\bigg|_0^{2\pi}
= 0 \ .
\end{align*}
}

\subsection{Théorème de la divergence}

\compileSOL{\SOLUb}{\ref{17Q30}}{
Le dessin du tétraède $E$ est donné ci-dessous.
\PDFgraph{17_vector_calculus/DivQ2}

Le Théorème de la divergence nous donne
\[
  \iint_S F \cdot \dx{\VEC{S}} = \iiint_E \diV  F \dx{V}
\]
où $S = \partial E$.  Puisque $\diV  F(x,y,z) = 3$, nous obtenons
\begin{align*}
\iiint_E \diV F \dx{V} &=
3\int_0^1\int_0^{1-x}\int_{z=0}^{1-x-y} \dx{z}\dx{y}\dx{x}
= 3\int_0^1\int_0^{1-x} \left(z\bigg|_0^{1-x-y}\right) \dx{z}\dx{y}\dx{x} \\
& = 3\int_0^1\int_0^{1-x} (1-x-y) \dx{y}\dx{x} 
  = 3\int_0^1\left(y-xy-\frac{y^2}{2}\right)\bigg|_{y=0}^{1-x} \dx{x} \\
& = 3\int_0^1\left( \frac{x^2}{2} - x + \frac{1}{2}\right) \dx{x}
= 3\left( \frac{x^3}{6} - \frac{x^2}{2} + \frac{x}{2}\right)\bigg|_0^1
= \frac{1}{2} \ .
\end{align*}
}

\compileSOL{\SOLUa}{\ref{17Q31}}{
Le dessin du solide $E$ est dooné ci-dessous.
\PDFgraph{17_vector_calculus/DivQ1}

Grâce au Théorème de la divergence, le flux de $F$ qui
s'échappe de $E$ est donné par
\[
  \iint_S F \cdot \dx{\VEC{S}} = \iiint_E \diV  F \dx{V}
\]
où $S = \partial E$.  Pour calculer l'intégrale triple, utilisons les
coordonnées cylindriques
\[
  (x,y,z) = (r\cos(\theta), r\sin(\theta), z)
\]
pour $0 \leq \theta \leq 2\pi$, $0 \leq r \leq 1$ et
$0 \leq z \leq 2$.  Puisque
\[
\left| \frac{\partial(x,y,z)}{\partial(\theta,r,z)} \right| =
\left| \det \begin{pmatrix}
  \displaystyle \pdydx{x}{\theta} & \displaystyle \pdydx{x}{r}
  & \displaystyle \pdydx{x}{z} \\[0.8em]
  \displaystyle \pdydx{y}{\theta} & \displaystyle \pdydx{y}{r}
  & \displaystyle \pdydx{y}{z} \\[0.8em]
  \displaystyle \pdydx{z}{\theta} & \displaystyle \pdydx{z}{r}
  & \displaystyle \pdydx{z}{z}
\end{pmatrix} \right|
= 
\left| \det \begin{pmatrix}
  -r\sin(\theta) & \cos(\theta) & 0 \\
  r\cos(\theta) & \sin(\theta) & 0 \\
  0 & 0 & 1 \end{pmatrix} \right|
= r
\]
et
\[
  \diV F(x,y,z) = 3 x^2 + 3 y^2 + 3 z^2 = 3(x^2 + y^2 + z^2) \ ,
\]
nous obtenons
\begin{align*}
\iiint_E \diV  F \dx{V} &=
\int_0^{2\pi} \int_0^1 \int_0^2 3 (r^2 + z^2) r \dx{z} \dx{r} \dx{\theta}
= \int_0^{2\pi} \int_0^1 \left( 3r^3 z + z^3 r \right)\bigg|_{z=0}^2
\dx{r}\dx{\theta} \\
&= \int_0^{2\pi} \int_0^1 \left(6 r^3 + 8r \right) \dx{r} \dx{\theta}
= \int_0^{2\pi} \left(\frac{3}{2} r^4 + 4r^2 \right)\bigg|_{r=0}^1 \dx{\theta}
= \frac{11}{2} \int_0^{2\pi} \dx{\theta} = 11\pi \ .
\end{align*}
}

\compileSOL{\SOLUb}{\ref{17Q32}}{
Le dessin du solide $E$ est donné ci-dessous.
\PDFgraph{17_vector_calculus/DivQ3}

Grâce au Théorème de la divergence, Le flux de $F$ qui
s'échappe de $E$ est donné par
\[
  \iint_S F \cdot \dx{\VEC{S}} = \iiint_E \diV  F \dx{V}
\]
où $S = \partial E$.  Pour calculer l'intégrale triple, nous utilisons les
coordonnées cylindriques
\[
  (x,y,z) = (r\cos(\theta), r\sin(\theta), z)
\]
pour $0 \leq \theta \leq 2\pi$, $0 \leq r \leq 2$ et
$0 \leq z \leq 4-r^2$.  Puisque
\[
\left| \frac{\partial(x,y,z)}{\partial(\theta,r,z)} \right| =
\left| \det \begin{pmatrix}
  \displaystyle \pdydx{x}{\theta} & \displaystyle \pdydx{x}{r}
  & \displaystyle \pdydx{x}{z} \\[0.8em]
  \displaystyle \pdydx{y}{\theta} & \displaystyle \pdydx{y}{r}
  & \displaystyle \pdydx{y}{z} \\[0.8em]
  \displaystyle \pdydx{z}{\theta} & \displaystyle \pdydx{z}{r}
  & \displaystyle \pdydx{z}{z}
\end{pmatrix} \right|
= 
\left| \det \begin{pmatrix}
  -r\sin(\theta) & \cos(\theta) & 0 \\
  r\cos(\theta) & \sin(\theta) & 0 \\
  0 & 0 & 1 \end{pmatrix} \right|
= r
\]
et $\displaystyle \diV F(x,y,z) = x^2 + y^2$, nous obtenons
\begin{align*}
\iiint_E \diV  F \dx{V} &=
\int_0^{2\pi} \int_0^2 \int_0^{4-r^2}  r^3 \dx{z} \dx{r} \dx{\theta}
= \int_0^{2\pi} \int_0^2 \left( r^3 z \right)\bigg|_{z=0}^{4-r^2}
\dx{r}\dx{\theta} \\
&= \int_0^{2\pi} \int_0^2 \left(4 r^3 - r^5 \right) \dx{r} \dx{\theta}
= \int_0^{2\pi} \left( r^4 - \frac{r^6}{6} \right)\bigg|_{r=0}^2 \dx{\theta}
= \frac{16}{3} \int_0^{2\pi} \dx{\theta}
= \frac{32\pi}{3} \ .
\end{align*}
}

%%% Local Variables: 
%%% mode: latex
%%% TeX-master: "notes"
%%% End: 

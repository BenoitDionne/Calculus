\section{Limites et fonctions continues}

\subsection{Limites}

\compileSOL{\SOLUb}{\ref{4Q1}}{
Posons $\displaystyle f(x) = \frac{e^{2x} - 1}{x}$.  Si nous choisissons la
suite $\displaystyle \left\{ x_n \right\}_{n=1}^\infty$ où
$x_n = 1/n$, nous obtenons une suite qui tend vers $0$.  Les données du
tableau suivant suggèrent que
$\displaystyle \lim_{n\rightarrow \infty} f(x_n) = 2$.
\[
\begin{array}{c|c|c} 
n & x_n = 1/n & f(x_n) \\
\hline
1 & 1 & 6.389056\ldots \\
2 & 1/2 & 3.43656\ldots \\
3 & 1/3 & 2.8432\ldots \\
\vdots & \vdots & \vdots \\
100 & 1/100 & 2.020134\ldots \\
\vdots & \vdots & \vdots \\
1000 & 1/1000 & 2.002001 \\
\vdots & \vdots & \vdots \\
\downarrow & \downarrow & \downarrow \\
\infty & 0 & 2
\end{array}
\]
Comme $\displaystyle \lim_{n\rightarrow \infty} f(x_n) = 2$ quelle que
soit la suite $\displaystyle \left\{ x_n \right\}_{n=1}^\infty$ qui
tend vers $0$, nous pouvons écrire
\[
\lim_{x\rightarrow 0}\frac{e^{2x} - 1}{x} = \lim_{x\rightarrow 0}f(x) = 2 \ .
\]
}

\compileSOL{\SOLUb}{\ref{4Q2}}{
Nous pouvons facilement déduire du graphe de la fonction $p$ que
$\displaystyle \lim_{t\rightarrow 1^-} p(t) = 0$ car
$p(t) = 0$ pour tous $t<1$.  De même, nous avons que
$\displaystyle \lim_{t\rightarrow 1^+} p(t) = 2$.
Ainsi, la limite $\displaystyle \lim_{t\rightarrow 1} p(t)$ n'existe pas car
$\displaystyle \lim_{t\rightarrow 1^-} p(t) \neq \lim_{t\rightarrow 1^+} p(t)$.
}

\compileSOL{\SOLUb}{\ref{4Q3}}{
\subQ{a} Puisque
\begin{align*}
\frac{1-\cos(0.1)}{0.1} &\approx 0.04995834 \ ,
\frac{1-\cos(0.01)}{0.01}\approx 0.00499995 \ , \\
\frac{1-\cos(0.001)}{0.001} &\approx 4.9999995\times 10^{-4} \ ,
\frac{1-\cos(0.0001)}{0.0001}\approx 4.9999999 \times 10^{-5} \ , \ldots 
\to 0 \ ,
\end{align*}
nous pouvons assumer que
$\displaystyle \lim_{x\to 0} \frac{1-\cos(x)}{x} = 0$.

\subQ{b}
Puisque
\begin{align*}
\frac{\ln(1-0.1)}{0.1} &\approx -1.05360515 \ ,
\frac{\ln(1-0.01)}{0.01}\approx -1.00503358\ \ , \\
\frac{\ln(1-0.001)}{0.001} &\approx -1.00050033 \ ,
\frac{\ln(1-0.0001)}{0.0001}\approx -1.0000500 \ , \ldots 
\to -1 \ ,
\end{align*}
nous pouvons assumer que
$\displaystyle \lim_{x\to 0} \frac{1-\ln(x)}{x} = -1$.
}

\compileSOL{\SOLUb}{\ref{4Q4}}{
Nous avons $\sqrt{x} < 0.1$ si $0\leq x < 0.01$.
Nous avons $\sqrt{x} < 0.01$ si $0 \leq x < 0.0001$.  La convergence n'est
certainement pas rapide.  Il faut que $x$ soit très petit pour que
$\sqrt{x}$ soit petit.
}

\compileSOL{\SOLUa}{\ref{4Q5}}{
$\displaystyle \lim_{x\to 1^-}f(x) = 1.5$ et 
$\displaystyle \lim_{x\to 1^+}f(x) = 0.5$. Comme la limite à droite
est différente de la limite à gauche la limite,
$\displaystyle \lim_{x\to 1}f(x)$ n'existe pas.

$\displaystyle \lim_{x\to 2^-}f(x) = 6$ et 
$\displaystyle \lim_{x\to 2^+}f(x) = 5$. Comme la limite à droite est
différente de la limite à gauche la limite,
$\displaystyle \lim_{x\to 2}f(x)$ n'existe pas.
}

\compileSOL{\SOLUb}{\ref{4Q6}}{
Puisque $v$ est une fonction continue, 
$\displaystyle \lim_{t\rightarrow 0} v(t) = v(0) = 1$.

\begin{enumerate}
\item Nous cherchons $t$ tel que $|v(t) - 1| < 1$.
\[
|v(t) - 1| < 1 \Rightarrow t^2 < 1 \Rightarrow -1<t<1 \ .
\]
\item Nous cherchons $t$ tel que $|v(t) - 1| < 0.5$.
\[
|v(t) - 1| < 0.5 \Rightarrow t^2 < 0.5 \Rightarrow -\sqrt{0.5}
<t<\sqrt{0.5} \ .
\]
\item Nous cherchons $t$ tel que $|v(t) - 1| < 0.01$.
\[
|v(t) - 1| < 0.01 \Rightarrow t^2 < 0.01 \Rightarrow
-0.1 <t< 0.1 \ .
\]
\end{enumerate}
}

\compileSOL{\SOLUa}{\ref{4Q8}}{
Nous avons $\displaystyle \lim_{x\to 0^+} H(x) = 1$ car $H(x_n)=1$
pour toute suite $\displaystyle \{ x_n\}_{n=1}^\infty$ qui tend vers
$0$ avec $x_n \geq 0$.

Nous avons $\displaystyle \lim_{x\to 0^-} H(x) = 0$ car $H(x_n)=0$
pour toute suite $\displaystyle \{ x_n\}_{n=1}^\infty$ qui tend vers
$0$ avec $x_n < 0$.

Puisque $\displaystyle \lim_{x\to 0^+} H(x) \neq \lim_{x\to 0^-} H(x)$,
la limite $\displaystyle \lim_{x\to 0} H(x)$ n'existe pas.
}

\subsection{Fonctions continues}

\compileSOL{\SOLUb}{\ref{4Q9}}{
\subQ{a}
Puisque $\displaystyle f(t) = \frac{1+t+t^2}{1+t}$ est une fonction continue
à $t=0$, nous avons
\[
\lim_{t\rightarrow 0} \frac{1+t+t^2}{1+t} = 
\lim_{t\rightarrow 0} f(t) = f(0) = 1 \ .
\]

\subQ{b} Puisque $\displaystyle f(x) = \frac{e^x}{1+x}$ est une
fonction continue à $x =1$, nous avons
\[
\lim_{x\to 0} \frac{e^x}{1+x} = \lim_{x\to 0} f(x) = f(0) = \frac{e^0}{1+0} =
1 \ .
\]

\subQ{c} Puisque $\displaystyle f(x) = \frac{(x-2)}{2+\sqrt{2x^2-4}}$
est une fonction continue à $x=2$, nous avons
\[
\lim_{x\to 2} \frac{(x-2)}{2+\sqrt{2x^2-4}}
= \lim_{x\to 2} f(x) = f(2) = \frac{(2-2)}{2+\sqrt{2\times 2^2-4}}
= \frac{0}{4} = 0 \ .
\]

\subQ{d} Puisque $\displaystyle f(z) = \frac{3z}{1+\ln(1+z)}$ est une
fonction continue à $z=0$, nous avons
\[
\lim_{z\rightarrow 0} \frac{3z}{1+\ln(1+z)}
= \lim_{z\rightarrow 0} f(z) = f(0) = \frac{0}{1+0} = 0 \ .
\]

\subQ{e} Puisque $\displaystyle  f(x) = \frac{\cos(\pi/x)}{x^2-5}$ est
une fonction continue à $x=3$, nous avons
\[
\lim_{x\to 3} \frac{\cos(\pi/x)}{x^2-5} = \lim_{x\to 3} f(x) = f(3) 
= \frac{\cos(\pi/3)}{3^2-5} = \frac{1/2}{4} = \frac{1}{8} \ .
\]
}

\compileSOL{\SOLUb}{\ref{4Q10}}{
\subQ{a} La fonction
$\displaystyle f(x) = \frac{(x-3)}{2-\sqrt{x^2-5}}$ n'est pas définie
à $x=3$ car le dénominateur est nul à $x=3$.  La fonction n'est donc
pas continue à $x=3$.  Le numérateur est aussi nul à $x=3$ donc il y a
une possibilité que la limite existe à $x=3$.  Nous
éliminons la racine carrée au dénominateur en multipliant la fonction 
$f$ par l'expression
$\displaystyle \frac{2+\sqrt{x^2-5}}{2+\sqrt{x^2-5}}$.  Comme cette
expression est égale à $1$, nous ne changeons pas la valeur de $f(x)$ pour
$x \neq 3$.  Nous obtenons
\begin{align*}
\frac{(x-3)}{2-\sqrt{x^2-5}} &= \left(\frac{(x-3)}{2-\sqrt{x^2-5}}\right)
\left(\frac{2+\sqrt{x^2-5}}{2+\sqrt{x^2-5}}\right)
= \frac{(x-3)(2+\sqrt{x^2-5})}{4 - (x^2-5)} \\
& = \frac{(x-3)(2+\sqrt{x^2-5})}{9 - x^2}
= \frac{(x-3)(2+\sqrt{x^2-5})}{(3-x)(3+x)}
= -\frac{2+\sqrt{x^2-5}}{3+x} \ .
\end{align*}
Ainsi,
\[
\lim_{x\to 3} \frac{(x-3)}{2-\sqrt{x^2-5}}
= \lim_{x\to 3} -\frac{2+\sqrt{x^2-5}}{3+x}
= -\frac{2+\sqrt{3^2-5}}{3+3} = -\frac{2}{3}
\]
où nous avons utilisé le fait que
$\displaystyle h(x) = -\frac{2+\sqrt{x^2-5}}{3+x}$ est une fonction
continue à $x=3$ pour calculer la dernière limite.

\subQ{b} Pour déterminer si
$\displaystyle \lim_{x\to 5} \frac{|x-5|}{x^2-25}$ existe, nous devons
considérer les cas $x<5$ et $x>5$.

Pour $x<5$, $|x-5| = 5 - x$ et nous obtenons
\[
\lim_{x\to 5^-} \frac{|x-5|}{x^2-25}
= \lim_{x\to 5^-} \frac{5-x}{(x-5)(x+5)}
= \lim_{x\to 5^-} \frac{-1}{x+5} = -\frac{1}{10} \ .
\]

Pour $x>5$, $|x-5| = x- 5$ et nous obtenons
\[
\lim_{x\to 5^+} \frac{|x-5|}{x^2-25}
= \lim_{x\to 5^+} \frac{x-5}{(x-5)(x+5)}
= \lim_{x\to 5^+} \frac{1}{x+5} = \frac{1}{10} \ .
\]

Puisque
$\displaystyle \lim_{x\to 5^-} \frac{|x-5|}{x^2-25}
\neq \lim_{x\to 5^+} \frac{|x-5|}{x^2-25}$, nous devons conclure que
$\displaystyle \lim_{x\to 5} \frac{|x-5|}{x^2-25}$ n'existe pas.

\subQ{c} La fonction
$\displaystyle f(x) = \frac{3-\sqrt{9-3x}}{8x}$ n'est pas définie
à $x=0$ car le dénominateur est nul à $x=0$.  La fonction n'est donc
pas continue en $x=0$.  Puisque le numérateur est aussi
nul à $x=0$, il y a une possibilité que la limite existe à $x=0$.  Nous
éliminons la racine carrée au numérateur en multipliant la fonction
$f$ par l'expression
$\displaystyle \frac{3+\sqrt{9-3x}}{3+\sqrt{9-3x}}$.
Comme cette expression est égale à $1$, nous ne changeons pas la valeur de
$f(x)$.  Nous obtenons
\[
\frac{3-\sqrt{9-3x}}{8x} = \left( \frac{3-\sqrt{9-3x}}{8x} \right)
\left( \frac{3+\sqrt{9-3x}}{3+\sqrt{9-3x}}\right)
= \frac{9-(9-3x)}{ 8x(3+\sqrt{9-3x})}
= \frac{3}{8(3+\sqrt{9-3x})} \ .
\]
Ainsi,
\[
\lim_{x\to 0} \frac{3-\sqrt{9-3x}}{8x}
= \lim_{x\to 0} \frac{3}{8(3+\sqrt{9-3x})}
= \frac{3}{8(3+\sqrt{9})} = -\frac{1}{16}
\]
où nous avons utilisé le fait que
$\displaystyle h(x) = \frac{3}{8(3+\sqrt{9-3x})}$ est une fonction
continue à $x=0$ pour calculer la dernière limite.

\subQ{d} If faut noter que pour $x$ près de $1$ (e.g. entre $0$ et $2$)
nous avons $|x-2|= 2-x$.  Ainsi,
\[
\lim_{x\to 1} \frac{|x-2|-1}{x^2-1}
= \lim_{x\to 1} \frac{(2-x)-1}{(x-1)(x+1)}
= \lim_{x\to 1} \frac{1-x}{(x-1)(x+1)}
= \lim_{x\to 1} \frac{-1}{x+1} = -\frac{1}{2}
\]
où nous avons utilisé le fait que
$\displaystyle h(x) = \frac{-1}{x+1}$ est une fonction
continue à $x=1$ pour calculer la dernière limite.

\subQ{e} Pour $x<-3$, nous avons que $|x+3| = -(x+3)$.  Ainsi,
\[
\lim_{x\to -3^-} \frac{|x+3|(5+x)}{x+3}
= \lim_{x\to -3^-} \frac{-(x+3)(5+x)}{x+3}
= \lim_{x\to -3^-} -(5+x) = -(5-3) = -2 \ .
\]
Pour $x>-3$, nous avons que $|x+3| = x+3$.  Ainsi,
\[
\lim_{x\to -3^+} \frac{|x+3|(5+x)}{x+3}
= \lim_{x\to -3^+} \frac{(x+3)(5+x)}{x+3}
= \lim_{x\to -3^+} (5+x) = (5-3) = 2 \ .
\]
Puisque
$\displaystyle \lim_{x\to -3^-} \frac{|x+3|(5+x)}{x+3}
\neq \lim_{x\to -3^+} \frac{|x+3|(5+x)}{x+3}$, nous devons conclure que\\
$\displaystyle \lim_{x\to -3} \frac{|x+3|(5+x)}{x+3}$ n'existe pas.

\subQ{f} Puisque le numérateur et le dénominateur de
$\displaystyle \frac{5x^4 + 3x^2-8}{12x^2-11x-1}$ sont nuls lorsque
$x=1$, nous savons qu'ils ont au moins un facteur $x-1$ en commun.
Si nous divisons le numérateur et le dénominateur par $x-1$, nous trouvons que
$5x^4 + 3x^2-8  = (x-1)(5x^3+5x^2+8x+8)$ et
$12x^2-11x-1 = (x-1)(12x+1)$.  Donc
\[
\lim_{x\to 1} \frac{5x^4 + 3x^2-8}{12x^2-11x-1}
= \lim_{x\to 1} \frac{(x-1)(5x^3+5x^2+8x+8)}{(x-1)(12x+1)}
= \lim_{x\to 1} \frac{5x^3+5x^2+8x+8}{12x+1}
= 2
\]
où nous avons utilisé le fait que
$\displaystyle h(x) = \frac{5x^3+5x^2+8x+8}{12x+1}$ est une fonction
continue à $x=1$ pour calculer la dernière limite.
}

\compileSOL{\SOLUb}{\ref{4Q11}}{
Nous avons $\displaystyle V(t) = e^{\alpha t}$ car $V_0 = 1$ par hypothèses.
Pour déterminer $\alpha$, nous utilisons $V(1000) = 2.71828$.  Nous avons
\begin{align*}
e^{1000\; \alpha} = 2.71828 &
\Rightarrow \ln\left(e^{1000\; \alpha}\right) = \ln(2.71828)
\Rightarrow 1000\; \alpha = \ln(2.71828) \\
&\Rightarrow \alpha = \frac{\ln(2.71828)}{1000} =
9.999993273472820\ldots \times 10^{-4} \ .
\end{align*}
Donc, si nous supposons que les données du problème sont approximatives,
nous pouvons dire que $\alpha = 10^{-3}$.

Pour répondre à la deuxième partie de la question, nous devons trouver $t$ tel
que
\[
2.71828 -0.1 = 2.61828 < V(t) < 2.81828 = 2.71828 +0.1 \ .
\]
Nous avons
\begin{align*}
2.61828 < e^{0.001\; t} < 2.81828
&\Rightarrow \ln(2.61828) < 0.001\; t < \ln(2.81828) \\
&\Rightarrow 10^3\; \ln(2.61828) < t < 10^3 \; \ln(2.81828) \\
&\Rightarrow 962.51761\ldots < t < 1036.126769\ldots
\end{align*}
}

\compileSOL{\SOLUb}{\ref{4Q12}}{
L'équation de la droite qui joint les points $(-0.1,-1)$ et $(0.1,1)$ est
$y = 10 x$.  Donc
\[
f(x) =
\begin{cases}
-1 & \quad \text{si} \quad x < -0.1 \\
10x & \quad \text{si} \quad -0.1 \leq x < 0.1 \\
1 & \quad \text{si} \quad x \geq 0.1
\end{cases}
\]
}

\compileSOL{\SOLUa}{\ref{4Q13}}{
Soit $W(V)$ l'impulsion électrique produite par le neurone après qu'il
est reçu une impulsion électrique de $V$ volts.  On nous dit dans la
question que
\[
W(V) =
\begin{cases}
V_1 & \quad \text{si} \quad V < V_0 \\
2V & \quad \text{si} \quad V > V_0
\end{cases}
\]
De plus, on nous dit dans la question que $W$ est continue.  Donc $W$ doit
aussi être définie à $V = V_0$ et
\[
W(V_0) = \lim_{V\to V_0^-} W(V) = \lim_{V\to V_0^+} W(V) \ .
\]
Puisque
\[
\lim_{V\to V_0^-} W(V) = V_1  \quad \text{et} \quad
\lim_{V\to V_0^+} W(V) = 2V_0 \ ,
\]
nous avons donc $W(V_0) = V_1 = 2 V_0$.  La fonction $W$ est naturellement
continue pour $V \neq V_0$.  Finalement, la définition de $W$ est
\[
W(V) =
\begin{cases}
2V_0 & \quad \text{si} \quad V < V_0 \\
2V & \quad \text{si} \quad V \geq V_0
\end{cases}
\]
}

\compileSOL{\SOLUb}{\ref{4Q14}}{
Comme $f$ est défini par un polynôme pour $x > 3$ et un fonction
sinusoïdale pour $x<3$, nous avons que $f$ est continue partout sauf
possiblement à $x=3$.  Pour que $f$ soit continue en $x=3$, nous devons
avoir
\[
\lim_{x\to 3^-} f(x) = \lim_{x\to 3^+} f(x) = f(3) \ .
\]
C'est-à-dire,
\[
a \sin\left(\frac{3\pi}{2}\right) = \frac{3^2}{2} - 4
\Rightarrow -a = \frac{1}{2} \Rightarrow a = -\frac{1}{2} \ .
\]
}

\compileSOL{\SOLUb}{\ref{4Q15}}{
Comme $f$ est défini par des polynômes pour $x \neq 3$, nous avons que $f$
est continue partout sauf possiblement à $x=3$.  Pour que $f$ soit
continue à $x=3$, nous devons avoir
\[
\lim_{x\to 3^-} f(x) = \lim_{x\to 3^+} f(x) = f(3) \ .
\]
Puisque
$\displaystyle \lim_{x\to 3^-} f(x) = \lim_{x\to 3^-} (a x^2 + 4x) = 9a + 12$,
$\displaystyle \lim_{x\to 3^+} f(x) = \lim_{x\to 3^+} x^2 + a x = 9 + 3a$ et
$f(3) = 9 + 3a$, nous devons donc avoir $9a + 12 = 9 + 3a$.  Si nous
résolvons, nous trouvons $a= -1/2$.
}

\compileSOL{\SOLUa}{\ref{4Q16}}{
Puisque $f$ est définie par une fonction rationnelle sans division par
$0$ pour $<3$ et par des sommes de termes en $x^\alpha$ pour $x > 3$,
nous avons que $f$ est continue partout sauf possiblement en $x=3$ et
$x=4$.

Pour que $f$ soit continue en $x=3$, nous devons avoir
\[
\lim_{x\to 3^-} f(x) = \lim_{x\to 3^+} f(x) = f(3) \ .
\]
Puisque
\[
\lim_{x\to 3^-} f(x) = \lim_{x\to 3^-} \frac{x^2 - 2x -3}{x-3}
= \lim_{x\to 3^-} \frac{(x+1)(x-3)}{x-3}
= \lim_{x\to 3^-} (x+1) = 4 \; ,
\]
$\displaystyle \lim_{x\to 3^+} f(x) = \lim_{x\to 3^+} a x^2 + x + b = 9a +3 +b$
et $f(3) = 9a + 3 + b$, nous devons donc avoir $4 = 9a +3 + b$.  
Ce qui donne l'équation $9a + b = 1$.

Pour que $f$ soit continue en $x=4$, nous devons avoir
\[
\lim_{x\to 4^-} f(x) = \lim_{x\to 4^+} f(x) = f(4) \ .
\]
Puisque
$\displaystyle \lim_{x\to 4^-} f(x) = \lim_{x\to 4^-} a x^2 + x + b =
16 a + 4 +b$,
$\displaystyle \lim_{x\to 4^+} f(x) = \lim_{x\to 4^+} b \sqrt{x} +
\frac{5 a x}{2}  = 2b + 10 a$
et $f(4) = 2b + 10 a$, nous devons donc avoir $16 a + 4 + b = 2 b + 10 a$.
Ce qui donne l'équation $6a - b = -4$.

Il faut résoudre le système d'équations linéaires
\begin{align*}
9a + b &= 1 \\
6a - b &= -4
\end{align*}
pour trouver $a$ et $b$.  Si nous additionnons les deux équations, nous
trouvons $15 a = -3$.  Donc $a = -1/5$.  Si nous substituons cette valeur
de $a$ dans l'équation $6a - b = -4$, nous obtenons
$ -6/5 - b = -4$. Donc $b= 14/5$
}

\compileSOL{\SOLUa}{\ref{4Q17}}{
Posons $f(x) = e^x + x^2 -x - 2$.  $f$ est une fonction continue sur la
droite réelle.  De plus, $f(0) = 1-2 = -1 < 0$ et
$f(1) = e-2 \approx 0.7182818 >0$.  Grâce au Théorème des valeurs
intermédiaires, il existe donc $c$ entre $0$ et $1$ tel que
$f(c)=0$; c'est-à-dire, $e^c + c^2 -2 = c$.
}

\compileSOL{\SOLUb}{\ref{4Q18}}{
\subQ{a} Posons $f(x) = \tan(x) + 4x - x^2 - e^x$.  Nous avons que $f$ est
une fonction continue sur l'intervalle $[0,\pi/4]$ telle que
$f(0) = -1 <0$ et
\begin{align*}
f\left(\frac{\pi}{4}\right) &= \tan\left(\frac{\pi}{4}\right)
+4 \left(\frac{\pi}{4}\right) - \left(\frac{\pi}{4}\right)^2 - e^{\pi/4} \\
& = 1 + \pi - \pi^2/16 - e^{\pi/4} \approx 1.33146232777\ldots > 0 \ .
\end{align*}
Grâce au Théorème des valeurs intermédiaires, il existe donc $c$ entre $0$ et
$\pi/4$ tel que $f(c) = 0$.

\subQ{b} Posons $\displaystyle f(x) = e^{\cos(x/2)} - 2\sin(x/2)$.
Nous avons que $f$ est une fonction continue sur l'intervalle $[0, \pi]$ avec
$f(0)= e - 0 = e>0$ et
$\displaystyle f(\pi) = e^{\cos(\pi/2)}-2\sin(\pi/2) = e^0 - 2 = -1 <0$.
Grâce au Théorème des valeurs intermédiaire, il existe donc $c$ entre $0$ et
$\pi$ tel que $f(c)=0$.  C'est-à-dire,
$\displaystyle e^{\cos(c/2)} = 2\sin(c/2)$.
}

\compileSOL{\SOLUa}{\ref{4Q19}}{
Si nous supposons que le prix de l'essence varie de façon continue en
fonction du temps, alors nous pouvons utiliser le Théorème des valeurs
intermédiaires pour conclure que le prix a été de \$$2.25$ à un
certain moment au cours de la semaine.   Mais, est-ce vraiment
réaliste de croire que le prix de l'essence a varié de façon continue?
}

\subsection{Limites à l'infini et limites infinies}

\compileSOL{\SOLUb}{\ref{4Q20}}{
Pour une précision de $0.1$\%, nous avons que $0.1$\% de $440$ est $0.44$ et
le coût du diapason sera de
$\displaystyle \frac{5}{0.44} \approx 11.36$ dollars.
Pour une précision de $0.01$\%, nous avons que $0.01$\% de $440$ est $0.044$
et le coût du diapason sera de
$\displaystyle \frac{5}{0.044} \approx 113.64$ dollars.

Nous avons que $\displaystyle \lim_{x\to 440} \frac{5}{|x-440|} = +\infty$.
Le coût du diapason va devenir exorbitant si nous exigeons la
perfection.  En fait, nous n'atteindrons jamais la perfection.
}

\compileSOL{\SOLUb}{\ref{4Q21}}{
Nous avons
\[
\frac{1}{\sqrt{x}} >10 \Rightarrow 0 < \sqrt{x} < \frac{1}{10}
\Rightarrow 0 < x < \frac{1}{100}
\]
et
\[
\frac{1}{\sqrt{x}} >100 \Rightarrow 0 < \sqrt{x} < \frac{1}{100}
\Rightarrow 0 < x < \frac{1}{10^4} \ .
\]
La croissance n'est pas rapide.  Il faut que $x$ soit très petit pour que
$1/\sqrt{x}$ soit grand.
}

\compileSOL{\SOLUb}{\ref{4Q22}}{
Soit $h(t) = (1-t)^{-4}$.  Comme la fonction $h$ n'est pas
définie en $t=1$, elle n'est certainement pas continue.  Nous devons donc
utiliser les suites pour estimer la limite.
\begin{center}
\begin{tabular}{|c|c|c|}
\cline{1-3}
$      n      $ & $t_n = 1+1/n^2$ & $     f(t_n)  $ \\ 
\cline{1-3}
$1.000000$ & $2.000000$ & $1$ \\ 
$2.000000$ & $1.250000$ & $256$ \\ 
$3.000000$ & $1.111111$ & $6561$ \\ 
$4.000000$ & $1.062500$ & $65536$ \\ 
$5.000000$ & $1.040000$ & $390625$ \\ 
$6.000000$ & $1.027778$ & $1679616$ \\ 
$7.000000$ & $1.020408$ & $5764801$ \\ 
$8.000000$ & $1.015625$ & $16777216$ \\ 
$9.000000$ & $1.012346$ & $43046721$ \\ 
$10.000000$ & $1.010000$ & $10^8$ \\ 
$\vdots$ & $\vdots$ & $\vdots$ \\
$98.000000$ & $1.000104$ & $\approx 8.50763 \times 10^{15}$ \\ 
$99.000000$ & $1.000102$ & $\approx 9.22744 \times  10^{15}$ \\ 
$100.000000$ & $1.000100$ & $\approx 10^{16}$ \\ 
 & $\downarrow$ & $\downarrow$ \\
& $1$ & $+\infty$ \\
\cline{1-3}
\end{tabular}
\end{center}
% Voir suitelimitQ1.m pour créer ce tableau.
Pour toute autre suite $\displaystyle \{t_n\}_{n=1}^\infty$ qui tend
vers $1$, nous avons toujours
$\displaystyle (1-t_n)^{-4} \rightarrow +\infty$ lorsque $n\to \infty$.
Donc $\displaystyle \lim_{x\to 1} (1-t)^{-4} = +\infty$.

Nous aurions pu résonner à partir de la définition de la fonction $h$.
Soit $\displaystyle \{t_n\}_{n=1}^\infty$ une suite qui tend vers $0$.
C'est le cas pour $t_n = 1/n^2$ que nous avons utilisé dans le tableau
ci-dessus.  Si nous substituons $t=1+ t_n$ dans $h(t)$, nous obtenons
$\displaystyle h(1+t_n) = t_n^{-4} = \frac{1}{t_n^4}$.  Puisque
$t_n^4 \to 0$ lorsque $n\to \infty$, nous avons que $1/t_n^4 \to +\infty$
lorsque $n\to \infty$. Plus $t_n^4>0$ est petit, plus $1/t_n^4$ est
grand.
}

\compileSOL{\SOLUa}{\ref{4Q23}}{
Posons $\displaystyle f(y) = y^2 \ln(y-1)$.  Avant de répondre à la
question, notons que $f(y)$ n'est pas défini pour $y<1$.  Il est donc
impossible de considérer $\displaystyle \lim_{y\rightarrow 1^-} f(y)$
et $\displaystyle \lim_{y\rightarrow 1} f(y)$.

Si nous choisissons la suite $\displaystyle \left\{ x_n \right\}_{n=1}^\infty$
où $x_n = 1+1/n$, nous obtenons une suite de termes plus grand que $1$
qui tend vers $1$.  Nous avons les résultats suivants.
\[
\begin{array}{c|c|c} 
n & x_n = 1/n & f(x_n) \\
\hline
1 & 2 & 0 \\
2 & 3/2 & -1.55958\ldots \\
\vdots & \vdots & \vdots \\
10,000 & 1.0001 & -9.212182\ldots \\
\vdots & \vdots & \vdots \\
10^9 & 1+10^{-9} & -20.723265\ldots \\
\vdots & \vdots & \vdots \\
10^{15} & 1+10^{-15} & -34.4342154\ldots \\
\downarrow & \downarrow & \downarrow \\
\infty & 1 & ?
\end{array}
\]
Il semble que $\displaystyle \lim_{n\rightarrow \infty} f(x_n) = -\infty$
car $f(x_n)$ semble décroître sans borne inférieure lorsque $n$
augmente.  Malheureusement, ce n'est pas convaincant.
$\displaystyle f\left(1+10^{-15}\right) = -34.4342154\ldots$
n'est pas une très petite valeur négative même si $1+10^{-15}$ est très
près de $1$.  Peut-être que
$\displaystyle \lim_{n\rightarrow \infty} f(x_n) = -35$.

C'est ici qu'il est important d'utiliser notre connaissance de la
fonction $\ln$ et de mettre notre calculatrice de côté.  Nous savons que
$\ln(x)$ tend vers $-\infty$ lorsque $x>0$ tend vers $0$ (pensez au
graphe de $\ln$).  Donc $\ln(y-1)$ tend vers $-\infty$ lorsque $y>1$
tend vers $1$.  Puisque $y^2$ tend vers $1$ lorsque $y$ tend vers $1$,
alors $y^2\ln(y-1)$ tend vers $-\infty$ lorsque $y>1$ tend vers $1$.
C'est-à-dire,
\[
\lim_{x\rightarrow 1^+} y^2\ln(y-1) = -\infty \ .
\]

Le fait que $f$ ne soit pas défini en $y=1$ n'empêche pas l'existence
de la limite $\displaystyle \lim_{y\rightarrow 1^+} f(y)$ car la
limite d'une fonction à un point est indépendante du comportement de
la fonction à ce point.
}

\compileSOL{\SOLUb}{\ref{4Q24}}{
\subQ{a} Si nous divisons le numérateur et dénominateur par $x^2$,
nous avons
\[
\lim_{x\rightarrow \infty} \frac{x^2+5x-4}{3x^2 +1}
= \lim_{x\rightarrow \infty} \frac{ 1+ (5/x)-(4/x^2)}{3 + (1/x^2)}
= \frac{\displaystyle 1+ \lim_{x\rightarrow \infty}(5/x)
-\lim_{x\rightarrow \infty}(4/x^2)}
{\displaystyle 3 + \lim_{x\rightarrow \infty}(1/x^2)}
= \frac{1}{3}
\]
car $\displaystyle \lim_{x\to \infty}\frac{1}{x^r} = 0 $ pour $r>0$.

\subQ{b} Si nous factorisons $x^3$ à l'extérieur de la racine cubique, nous
obtenons
\begin{align*}
\lim_{x\rightarrow \infty} \frac{x-(8x^3+3)^{1/3}}{x}
&= \lim_{x\rightarrow \infty} \frac{x- x\left(8+(3/x^3)\right)^{1/3}}{x} 
= \lim_{x\rightarrow \infty} \left( 1- \left(8+(3/x^3)\right)^{1/3}\right) \\
&= \left( 1- \left(8+ \lim_{x\rightarrow \infty} (3/x^3)\right)^{1/3}\right)
= 1 - 8^{1/3} = -1 \ .
\end{align*}

\subQ{c}
Puisque $\displaystyle \lim_{x\to 0^-} \; \frac{1}{x} = -\infty$ et
$\displaystyle \lim_{x\to 0^+} \; \frac{1}{x} = +\infty$, la limite
$\displaystyle \lim_{x\to 0} \; \frac{1}{x}$ n'existe pas.

\subQ{d} Puisque
\begin{align*}
\sqrt{x^2+3x} - \sqrt{x^2 + 7x}
&= \left(\sqrt{x^2+3x} - \sqrt{x^2 + 7x}\right)
\left(\frac{\sqrt{x^2+3x} + \sqrt{x^2 + 7x}}
{\sqrt{x^2+3x} + \sqrt{x^2 + 7x}} \right) \\
&= \frac{(x^2+3x) - (x^2 + 7x)}
{\sqrt{x^2+3x} + \sqrt{x^2 + 7x}}
= \frac{-4x}{\sqrt{x^2+3x} + \sqrt{x^2 +7x}} \\
&= \frac{-4x}{x\sqrt{+ (3/x)} + x\sqrt{1 +(7/x)}}
= \frac{-4}{\sqrt{1+ (3/x)} + \sqrt{1 +(7/x)}} \ ,
\end{align*}
nous avons
\begin{align*}
\lim_{x\to \infty} \left(\sqrt{x^2+3x} - \sqrt{x^2 + 7x}\right)
&= \lim_{x\to \infty} \frac{-4}{\sqrt{1+ (3/x)} + \sqrt{1 +(7/x)}} \\
&= \frac{-4}{\sqrt{\displaystyle 1+ \lim_{x\to \infty}(3/x)}
+ \sqrt{\displaystyle 1 + \lim_{x\to \infty} (7/x)}} \\
&= \frac{-4}{\sqrt{1}+\sqrt{1}} = -2 \ .
\end{align*}

\subQ{e} Puisque
\[
\frac{2e^{2x} -e^{-3x}}{3e^{2x} - 4e^{-5x}}
= \frac{e^{2x}\left( 2 -e^{-5x}\right)}{e^{2x}\left(3 - 4e^{-7x}\right)}
= \frac{2 -e^{-5x}}{3 - 4e^{-7x}} \ ,
\]
Nous avons
\[
\lim_{x\rightarrow \infty} \frac{2e^{2x} -e^{-3x}}{3e^{2x} - 4e^{-5x}}
= \lim_{x\rightarrow \infty} \frac{2 -e^{-5x}}{3 - 4e^{-7x}}
=\frac{\displaystyle 2 - \lim_{x\rightarrow \infty} e^{-5x}}
{\displaystyle 3 - 4 \lim_{x\rightarrow \infty}e^{-7x}}
= \frac{2}{3}
\]
car $\displaystyle \lim_{x\rightarrow \infty} e^{rx} = 0$ pour $r<0$.
}

\compileSOL{\SOLUa}{\ref{4Q25}}{
Puisque
\[
  0 \leq \left| \frac{\sin^2(x)}{1+x^2} \right| \leq \frac{1}{1+x^2}
\]
et $\displaystyle \lim_{x\to \infty} \frac{1}{1+x^2} = 0$, nous avons grâce
au Théorème des gendarmes, théorème~\ref{gendarmeF}, que \\
$\displaystyle \lim_{x\to \infty} \left| \frac{\sin^2(x)}{1+x^2} \right| =0$.
Il s'en suit que
$\displaystyle \lim_{x\to \infty} \frac{\sin^2(x)}{1+x^2} =0$.
}

%%% Local Variables: 
%%% mode: latex
%%% TeX-master: "notes"
%%% End: 

\section{Fonctions}

\subsection{Algèbre}

\compileSOL{\SOLUb}{\ref{2Q1}}{
\subQ{a} $\displaystyle (3^4)^{0.5} = 3^2 = 9$

\subQ{b}
$\displaystyle 2^{2^3} \times 2^{2^2} = 2^{2^3 + 2^2} = 2^{8+4} = 2^{12}$

\subQ{c} $\displaystyle \log_3(1) = 0$ car $3^0 = 1$.

\subQ{d} $\displaystyle \log_{10}(5) + \log_{10}(20)
= \log_{10}(5\times 20) = \log_{10}(100) = 2$

\subQ{e} $\displaystyle \log_{10}(500) - \log_{10}(50)
= \log_{10}(500/50) = \log_{10}(10) = 1$

\subQ{f} $\displaystyle \log_{42.3}(42.3^7) = 7$
}

\compileSOL{\SOLUb}{\ref{2Q2}}{
\subQ{a}
\[
\frac{(x^4 y^{1/4})^{1/2}}{y^{1/2}} \Rightarrow
\frac{(x^4)^{1/2} (y^{1/4})^{1/2}}{y^{1/2}} \Rightarrow
\frac{x^{4/2} y^{1/8}}{y^{1/2}} \Rightarrow
x^2 y^{1/8- 1/2} \Rightarrow x^2 y^{-3/8} \ .
\]

\subQ{b}
\begin{align*}
\frac{x+1}{x-2} + \frac{x+1}{x-3} + \frac{-9x+21}{x^2-5x+6}
&= \frac{x+1}{x-2} + \frac{x+1}{x-3} + \frac{-9x+21}{(x-2)(x-3)} \\
&= \frac{(x+1)(x-3) + (x+1)(x-2) -9x+21}{(x-2)(x-3)} \\
&= \frac{2x^2 -12 x +16}{(x-2)(x-3)}
= \frac{2(x-4)(x-2)}{(x-2)(x-3)}
= \frac{2(x-4)}{(x-3)} \ .
\end{align*}
}

\compileSOL{\SOLUb}{\ref{2Q3}}{
\[
\begin{array}{lll}
\multicolumn{2}{l|}{x^3 + 2 x^2 + x + 3} & x+2 \\
\cline{3-3}\\[-0.8em]
-(x^3 + 2 x^2) & & x^2 +1 \\
\cline{1-2} \\[-0.8em]
 & x+3 & \\
 & -(x+2) & \\
\cline{2-2} \\[-0.8em]
 & 1 &
\end{array}
\]
Donc
\[
  \frac{x^3 + 2x^2 +x+3}{x+2} = x^1 + 1 + \frac{1}{x+2} \ .
\]
}

\compileSOL{\SOLUb}{\ref{2Q4}}{
Les racines du polynôme $a x^2 + b x + c$ sont données par la formule
\[
x_{\pm} = \frac{-b \pm \sqrt{b^2 - 4 a c }}{2 a}
\]
si $b^2 - 4 a c \geq 0$.  Dans ce cas, nous avons
$a x^2 + b x + c = a(x-x_-)(x-x_+)$.  Si $b^2 - 4 a c <0$, il n'y a
pas de racine réelle et nous ne pouvons pas factoriser le polynôme.

\subQ{a}
Les racines du polynôme $x^2+x-6$ sont
\[
x_+ = \frac{-1 + \sqrt{1 +24}}{2} = \frac{-1 + 5}{2} = 2
\quad \text{et} \quad x_- = \frac{-1 - \sqrt{1 +24}}{2} =
\frac{-1 - 5}{2} = -3 \ .
\]
Donc $x^2+x-6 = (x-2)(x+3)$.

\subQ{b}
Les racines du polynôme $3x^2 - 5 x -2$ sont
\[
x_+ = \frac{5 +\sqrt{25 + 24}}{6} = \frac{5 + 7}{6} = 2
\quad \text{et} \quad
x_- = \frac{5 - \sqrt{25 + 24}}{6} = \frac{5 - 7}{6} = -\frac{1}{3} \ .
\]
Donc $\displaystyle 3x^2 - 5 x -2 = 3
(x-2)\left(x+\frac{1}{3}\right)=(x-2)(3x + 1)$.

\subQ{c}  Nous avons
\[
x^{3/2} + x^{1/2}-12 x^{-1/2} = x^{-1/2} \left( x^2 + x -12\right)
= x^{-1/2} (x+4)(x-3)
\]
car les racines du polynôme $x^2 + x -12$ sont
\[
x_+ = \frac{-1 + \sqrt{1 + 48}}{2} = \frac{-1 + 7}{2} = 3
\quad \text{et} \quad
x_- = \frac{-1 -\sqrt{1 + 48}}{2} = -4 \ .
\]
}

\compileSOL{\SOLUb}{\ref{2Q5}}{
\subQ{a} Les racines d'une polynôme de la forme $ax^2 +bx + c$ sont
\[
x_\pm = \frac{-b \pm \sqrt{b^2-4ac}}{2a} \ .
\]
Donc les racines cherchées sont 
\begin{align*}
x_+ &= \frac{-2 + \sqrt{2^2-4(-2)}}{2} = \frac{-2 + \sqrt{12}}{2}
= \frac{-2 + 2\sqrt{3}}{2} = -1 + \sqrt{3}
\intertext{et}
x_- &= \frac{-2 - \sqrt{2^2-4(-2)}}{2} = \frac{-2 - \sqrt{12}}{2}
= \frac{-2 - 2\sqrt{3}}{2} = -1 - \sqrt{3} \ .
\end{align*}

\subQ{b} Puisque le coefficient de $x^2$ est positif, nous avons une
parabole ouverte vers le haut (i.e. convexe).  Cette parabole coupe
l'axe des $x$ aux points $(x,0)$ où $x^2+2x-2=0$; c'est-à-dire, aux
points $(x_-,0)$ et $(x_+,0)$.  La parabole coupe l'axe des $y$ au
point $(0,-2)$.  La parabole à un minimum lorsque
$\displaystyle x = \frac{x_-+x_+}{2} = -1$; le point milieu entre
$x_-$ et $x_+$.  Nous obtenons la figure ci-dessous.
\PDFgraph{2_fonctions/parabola}
}

\compileSOL{\SOLUb}{\ref{2Q7}}{
\subQ{a}
\[
7\times 5^{3x} = 21 \Leftrightarrow 5^{3x} = 3
\Leftrightarrow \ln\left(5^{3x}\right) = \ln(3)
\Leftrightarrow 3x \ln(5) = \ln(3) 
\Leftrightarrow x = \frac{\ln(3)}{3\ln(5)}
\]

\subQ{b}
\begin{align*}
4\times 3^{-2x+1} = 7\times 3^{3x} &
\Leftrightarrow \frac{3^{-2x+1}}{3^{3x}} = \frac{7}{4} 
\Leftrightarrow 3^{-5x+1} = \frac{7}{4} \\
& \Leftrightarrow \ln\left(3^{-5x+1}\right) = \ln\left(\frac{7}{4}\right)
\Leftrightarrow (-5x + 1)\ln(3) = \ln(7) - \ln(4) \\
& \Leftrightarrow x = -\frac{1}{5}
\left(-1 + \frac{\ln(7) - \ln(4)}{\ln(3)}\right)
\end{align*}

\subQ{c}
\[
\log_{10}(100^x) = 3 \Rightarrow \log_{10}(10^{2x}) = 3
\Rightarrow 2x = 3 \Rightarrow x = \frac{3}{2} \ .
\]
}

\compileSOL{\SOLUb}{\ref{2Q8}}{
Si nous multiplions des deux côtés par $e^x$, nous obtenons
\[
e^x + 2e^{-x} = 3 \Rightarrow e^{2x} + 2 = 3e^x
\Rightarrow e^{2x} - 3e^x + 2 = 0 \Rightarrow (e^x-2)(e^x-1) = 0 \ .
\]
Donc $e^x=1$ donne $x=0$ et $e^x =2$ donne $x=\ln(2)$.
}

\compileSOL{\SOLUb}{\ref{2Q9}}{
\subQ{a}
Pour $x \neq 0$ et $-1$,  
\[
  \frac{2x}{x+1} = \frac{2x-1}{x} \Leftrightarrow
  2x^2 = (2x-1)(x+1)=2x^2 +x - 1 \Leftrightarrow x = 1 \ .
\]

\subQ{b} Pour $x \neq \pm 1$,
\begin{align*}
\frac{1}{x+1} + \frac{3}{x-1} = 4
&\Leftrightarrow \frac{(x-1)+3(x+1)}{(x+1)(x-1)}
= \frac{4x + 2}{(x+1)(x-1)} = 4 \\
&\Leftrightarrow 4x + 2 = 4(x+1)(x-1) = 4x^2 -4 \\
&\Leftrightarrow 4x^2 - 4 x - 6 = 4 \left( x^2 - x -
\frac{3}{2}\right) = 0  \ .
\end{align*}
Ainsi, $\displaystyle x = \frac{1 \pm \sqrt{7}}{2}$.

\subQ{c} Nous devons avoir $x>5$ pour que les trois logarithmes soient
bien définis.  Donc
\begin{align*}
\ln(x-3)+\ln(x-5) = \ln(2x-6)
& \Leftrightarrow \ln( (x-3)(x-5) ) = \ln(x^2 - 8x +15) = \ln(2x-6) \\
& \Leftrightarrow x^2 - 8x +15 = 2x-6 \\
& \Leftrightarrow x^2 - 10x + 21 = (x-3)(x-7) = 0
\Leftrightarrow x = 7
\end{align*}
pour $x>5$.  La racine $3$ ne satisfait pas $x>5$.

\subQ{d}
$|x-2| = 5$ si $x - 2 = 5$ ou $x-2 = -5$; c'est-à-dire, si $x = 7$ ou $-3$.

\subQ{e} Ce cas n'est pas beaucoup plus difficile que le précédent.
Si nous assumons que $x \neq 5/2$, qui n'est pas une solution de toute
façon, alors
\[
|x - 2| = |2x-5| \Leftrightarrow \frac{|x-2|}{|2x -5|} =
\left|\frac{x-2}{2x-5}\right| = 1
\Leftrightarrow \frac{x-2}{2x-5} = 1 \text{ ou } \frac{x-2}{2x-5} = -1\; .
\]
Si $\displaystyle \frac{x-2}{2x-5} = 1$, nous obtenons
$x-2 = 2x-5$ et ainsi $x = 3$.
Si $\displaystyle \frac{x-2}{2x-5} = -1$, nous obtenons
$x-2 = -2x+5$ et ainsi $x = 7/3$.

\subQ{f}
\[
|x^2 - 4| = 4 \Leftrightarrow x^2 - 4 = 4 \text{ ou } x^2 - 4 = -4
\Leftrightarrow x^2 = 8 \text{ ou } x^2=0 \Leftrightarrow
x = \pm 2\sqrt{2} \text{ ou } x = 0 \ .
\]

\subQ{g} Nous devons avoir $0<x<3$ pour que les deux logarithmes soient
bien définis.  Donc
\begin{align*}
\ln(2) - \ln(x) &= \ln(3-x) \Leftrightarrow
\ln(2) - \ln(x) - \ln(3-x) = 0 \Leftrightarrow
\ln\left(\frac{2}{x(3-x)}\right) = 0 \\
&\Leftrightarrow \frac{2}{x(3-x)} = 1 \Leftrightarrow 2 = x(3-x)
\Leftrightarrow x^2 -3x + 2 = 0 \\
&\Leftrightarrow (x-2)(x-1) = 0 \Leftrightarrow x = 1 \text{ et } x=2
\end{align*}
pour $0<x<3$.
}

\compileSOL{\SOLUb}{\ref{2Q10}}{
\subQ{a} Nous avons
\[
  5 -\frac{6}{x} < 4 \Leftrightarrow 1 - \frac{6}{x} < 0
  \Leftrightarrow \frac{x-6}{x} < 0 \ .
\]
La fonction $f(x) = (x-6)/x$ peut changer de signe seulement aux
points $x$ où $f(x) = 0$ ou $f(x)$ n'est pas définie; c'est-à-dire, aux
points $0$ et $6$.  Il suffit donc de déterminer le signe de $f(x)$
dans les intervalles délimités par ces points pour trouver les
intervalles où $f(x) < 0$. 
\[
\begin{array}{c|c|c|c}
x & x<0 & 0<x<6 & x>6 \\
\hline
x-6 & - & - & + \\
\hline
x & - & + & + \\
\hline
f(x) = (x-6)/x & + & - & +
\end{array}
\]
Donc $5 - 6/x < 4$ pour $0<x<6$.
  
\subQ{b} Nous avons $x^2 -4x + 3 = (x-1)(x-3)$.  La fonction
$f(x) = (x-1)(x-3)$ peut changer de signe seulement aux points $x$ où
$f(x) = 0$; c'est-à-dire, aux points $1$ et $3$.  Il suffit donc
de déterminer le signe de $f(x)$ dans les intervalles délimités par
ces points pour trouver les intervalles où $f(x) > 0$.
\[
\begin{array}{c|c|c|c}
x & x<1 & 1<x<3 & x>3 \\
\hline
x-1 & - & + & + \\
\hline
x-3 & - & - & + \\
\hline
f(x) = (x-1)(x-3) & + & - & +
\end{array}
\]
Donc $x^2 -4x + 3 > 0$ pour $x<1$ ou $x>3$.

\subQ{c} Nous avons
\[ 
\frac{2}{x+1} >\frac{1}{x-3} \Leftrightarrow \frac{2}{x+1} -\frac{1}{x-3} > 0
\Leftrightarrow \frac{x-7}{(x+1)(x-3)} > 0 \ .
\]
La fonction $\displaystyle f(x) = \frac{x-7}{(x+1)(x-3)}$ peut changer de
signe seulement aux points $x$ où $f(x) = 0$ ou $f(x)$ n'est pas
définie.  Nous parlons des points $-1$, $3$ et $7$.  Il suffit donc
de déterminer le signe de $f(x)$ dans les intervalles délimités par
ces points pour trouver les intervalles où $f(x) > 0$.
\[
\begin{array}{c|c|c|c|c}
 & x<-1 & -1<x<3 & 3<x<7 & x>7 \\
\hline
x+1 & - & + & + & + \\
\hline
x-3 & - & - & + & + \\
\hline
x-7 & - & - & - & + \\
\hline
\rule{0em}{1.5em} \displaystyle f(x) = \frac{x-7}{(x+1)(x-3)} & - & + & - & +
\end{array}
\]
Donc $\displaystyle f(x) > 0$ si $-1<x<3$ ou $x>7$.

\subQ{d} Nous avons
\[
  \frac{x^2}{x+2}  < 1 \Leftrightarrow 0 < 1 - \frac{x^2}{x+2} 
= - \frac{x^2 -x - 2}{x+2} = -\frac{(x-2)(x+1)}{x+2} \ .
\]
La fonction $\displaystyle f(x) = -\frac{(x-2)(x+1)}{x+2}$ peut changer de
signe seulement aux points $x$ où $f(x) = 0$ ou $f(x)$ n'est pas
définie.  Nous parlons des points $-2$, $-1$ et $2$.  Il suffit donc
de déterminer le signe de $f(x)$ dans les intervalles délimités par
ces points pour trouver les intervalles où $f(x) > 0$.
\[
\begin{array}{c|c|c|c|c}
x & x < -2 & -2 < x <-1 & -1 < x < 2 & x > 2 \\
\hline
x-2 & - & - & - & + \\
\hline
x+1 & - & - & + & + \\
\hline
x+2 & - & + & + & + \\
\hline
f(x) & + & - & + & -
\end{array}
\]
L'inégalité est donc vrai pour $x < -2$ ou $-1<x < 2$.

\subQ{e} Nous avons
\[
\frac{x}{x-3}  < \frac{-6}{x+1} \Leftrightarrow 
0 < \frac{-6}{x+1} - \frac{x}{x-3}
= - \frac{x^2 + 7x - 18}{(x-3)(x+1)}
= - \frac{(x+9)(x-2)}{(x-3)(x+1)} \ .
\]
La fonction $\displaystyle f(x) = -\frac{(x+9)(x-2)}{(x-3)(x+1)}$
peut changer de signe seulement aux points $x$ où $f(x) = 0$ ou $f(x)$
n'est pas définie.  Nous parlons des points $-9$, $-1$, $2$ et $3$.  Il
suffit donc de déterminer le signe de $f(x)$ dans les intervalles
délimités par ces points pour trouver les intervalles où $f(x) > 0$.
\[
\begin{array}{c|c|c|c|c|c}
x & x < -9 & -9 < x <-1 & -1 < x < 2 & 2 < x < 3 & x > 3 \\
\hline
x+9 & - & + & + & + & + \\
\hline
x-2 & - & - & - & + & + \\
\hline
x-3 & - & - & - & - & + \\
\hline
x+1 & - & - & + & + & + \\
\hline
f(x) & - & + & - & + & - 
\end{array}
\]
L'inégalité est donc vrai pour $-9<x<-1$ ou $2<x<3$.

\subQ{f} Nous avons
\[
|x-2| < 5 \Leftrightarrow -5 < x-2 < 5
\Leftrightarrow -5 <x-2 \text{ et } x-2<5
\Leftrightarrow -3 <x \text{ et } x<7 \ .
\]
L'inégalité est donc vrai pour $-3 < x < 7$.

\subQ{g} Si nous assumons que $x \neq 1$, qui n'est pas une solution de toute
façon, alors
\[
|x^2-2x - 5| < |x-1| \Leftrightarrow
\frac{|x^2-2x - 5|}{|x-1|} = \left|\frac{x^2-2x - 5}{x-1}\right| < 1
\Leftrightarrow 
-1 < \frac{x^2-2x - 5}{x-1} < 1 \ .
\]
Nous avons deux inégalité à résoudre simultanément.

\subI{i} Nous commençons avec l'inégalité
\[
  -1 < \frac{x^2-2x - 5}{x-1} \Leftrightarrow
  0 < 1 + \frac{x^2-2x - 5}{x-1} = \frac{x^2 -x - 6}{x-1}
  = \frac{(x-3)(x+2)}{x-1} \ .
\]
La fonction $\displaystyle f(x) = \frac{(x-3)(x+2)}{x-1}$ peut
changer de signe seulement aux points $x$ où $f(x) = 0$ ou $f(x)$ n'est
pas définie.  Nous parlons des points $-2$, $1$ et $3$.  Il
suffit de déterminer le signe de $f(x)$ dans les intervalles
délimités par ces points pour trouver les intervalles où $f(x) > 0$.
\[
\begin{array}{c|c|c|c|c|c|c|c}
x & x < -2 & -2 & -2 < x <1 & 1 & 1 < x < 3 & 3 & x > 3 \\
\hline
f(x) & - & 0 & + & \text{N.D.} & - & 0 & + 
\end{array}
\]
L'inégalité est donc vrai pour $]-2, 1[ \cup ]3,\infty[$.

\subI{ii} Nous considérons maintenant
\[
 \frac{x^2-2x - 5}{x-1} < 1 \Leftrightarrow
  0 < 1 - \frac{x^2-2x - 5}{x-1} = -\frac{x^2 - 3x - 4}{x-1}
  = -\frac{(x-4)(x+1)}{x-1} \ .
\]
La fonction $\displaystyle f(x) = -\frac{(x-4)(x+1)}{x-1}$ peut
changer de signe seulement aux points $x$ où $f(x) = 0$ ou $f(x)$ n'est
pas définie.  Nous parlons des points $-1$, $1$ et $4$.  Il
suffit de déterminer le signe de $f(x)$ dans les intervalles
délimités par ces points pour trouver les intervalles où $f(x) > 0$.
\[
\begin{array}{c|c|c|c|c|c|c|c}
x & x < -1 & -1 & -1 < x <1 & 1 & 1 < x < 4 & 4 & x > 4 \\
\hline
f(x) & + & 0 & - & \text{N.D.} & + & 0 & - 
\end{array}
\]
L'inégalité est donc vrai pour $]-\infty, -1[ \cup ]1,4[$.

Pour répondre à la question initiale, il faut trouver les valeurs de
$x$ qui satisfont les deux cas simultanément.  Il faut donc trouver
l'intersection des deux ensembles de solutions que nous avons.
Nous obtenons $]-2, -1[ \cup ]3,4[$.
}

\subsection{Functions}

\compileSOL{\SOLUb}{\ref{2Q11}}{
Puisque $f(x) = 3 + 2x-x^2 = 4 - (x-1)^2$, nous avons que
$f(x)$ peut prendre toutes les valeurs plus petites ou égales à $4$
car $(x-1)^2$ peut prendre toutes les valeurs plus grandes ou égales
à $0$.
}

\compileSOL{\SOLUb}{\ref{2Q13}}{
\subQ{a} La fonction $g$ est définie pour tout $x$ réel.  Le
domaine de $g$ est donc $\RR$.  Puisque $x^3$ peut prendre toutes les
valeurs entre $-\infty$ et $+\infty$, on a que l'image de $g$ est
l'image de $5^x$ pour $x \in \RR$; c'est-à-dire que l'image de $g$
est $]0,\infty[$.

Puisque
\[
  g(x_1) = g(x_2) \Leftrightarrow 5^{x_1^3} = 5^{x_2^3}
  \Leftrightarrow x_1^3 = x_2^3 \Leftrightarrow x_1 = x_2 \ ,
\]
nous avons que $g$ est une fonction injective sur $\RR$, son domaine.
Nous pouvons donc inverser $g: \RR \rightarrow ]0,\infty[$.  L'inverse de $g$
est donné par
\[
y = g(x) \Leftrightarrow y = 5^{x^3} \Leftrightarrow \log_5(y) = x^3
\Leftrightarrow x = (\log_5(y))^{1/3} \ .
\]
Par convention, nous écrivons
$\displaystyle y = g^{-1}(x) = \sqrt[3]{\log_5(x)}$.
Le domaine de $g^{-1}$ est $]0,\infty[$, l'image de $g$, et l'image de
$g^{-1}$ est $\RR$, le domaine de $g$.

\subQ{b} Déterminons les valeurs de $x$ pour lesquelles
$(x+1)/(x-1) \geq 0$.  L'expression $(x+1)/(x-1)$ peut changer de
signe seulement aux points $x$ où elle est nul ou elle n'est pas
définie.  C'est-à-dire aux points $-1$ et $1$.  Il suffit donc de
déterminer le signe de $(x+1)/(x-1)$ dans les intervalles délimités
par ces points pour trouver les intervalles où $(x+1)/(x-1) > 0$.
\[
\begin{array}{c|c|c|c|c|c}
x & x < -1 & -1 & -1 < x <1 & 1 & x > 1 \\
\hline
x+1 & - & 0 & + & 2 & + \\
\hline
x-1 & - & 2 & - & 0 & + \\
\hline
(x+1)/(x-1) & + & 0 & - & \text{N.D} & + 
\end{array}
\]
Le domaine de $f$ est donc $]-\infty,-1] \cup ]1,\infty[$.  L'image
de $f$ est $[0,1[\cup ]1, \infty[$ car
\[
\frac{x+1}{x-1} = \frac{x-1 + 2}{x-1} = 1 + \frac{2}{x-1}
\]
où $2/(x-1)$ prend toutes les valeurs plus grande que $0$
lorsque $1 <x < \infty$ et $2/(x-1)$ prend toutes les valeurs
entre $-1$ et $0$ ($-1$ inclus) lorsque $-\infty < x \leq -1$.  Ne pas oublier
que la racine carrée est positive ou nulle.

Puisque
\begin{align*}
f(x_1) = f(x_2) &\Rightarrow
\sqrt{ \frac{x_1+1}{x_1-1} } = \sqrt{ \frac{x_2+1}{x_2-1} }
\Rightarrow \frac{x_1+1}{x_1-1} = \frac{x_2+1}{x_2-1} \\
&\Rightarrow (x_1+1)(x_2-1) = (x_2+1)(x_1-1) \\
&\Rightarrow x_1 x_2 + x_2 - x_1 - 1 = x_1 x_2 - x_2 + x_1 -1 \\
&\Rightarrow x_2 - x_1 = - x_2 + x_1 \Rightarrow 2 x_2 = 2 x_1
\Rightarrow x_2 = x_1 \ ,
\end{align*}
nous avons que $f$ est une fonction injective sur son domaine.
Nous pouvons donc inverser
\[
f: \DO f \rightarrow \IM f \ .
\]
L'inverse de $f$ est donné par
\begin{align*}
y = f(x) &\Leftrightarrow y = \sqrt{ \frac{x+1}{x-1} }
\Leftrightarrow y^2 = \frac{x+1}{x-1}
\Leftrightarrow y^2 (x-1) = x + 1 \\
&\Leftrightarrow y^2 x - x = y^2 + 1
\Leftrightarrow x( y^2 - 1) = y^2 + 1 \\
&\Leftrightarrow x = \frac{y^2+1}{y^2-1} \ .
\end{align*}
Notez que la deuxième équivalence ci-dessus est vrai seulement
parce que nous assumons que $y$ est dans l'image de $f$;
c'est-à-dire, $y \in [0,1[\cup ]1, \infty[$.

Par convention, nous écrivons
$\displaystyle y = f^{-1}(x) = \frac{x^2+1}{x^2-1}$.  Le domaine de
$f^{-1}$ est l'image de $f$ et l'image de $f^{-1}$ est le domaine de
$f$.

\subQ{c}  La fonction $h$ est définie pour tout $x$ sauf $x=-1$.
Le domaine de $h$ est donc $\RR \setminus \{-1\}$.  Puisque
$y^{10} \geq 0$ pour tout $y \in \RR$, et
$\displaystyle \frac{x}{x+1}$ peut prendre toutes les valeurs
réelles sauf $1$ (pourquoi?) en variant $x$, nous avons que l'image de $h$
est $[0,1[ \cup ]1,\infty[$. 

La fonction $h$ n'est pas injective car
\[
h(1) = \left( \frac{1}{2} \right)^{10} = \left( -\frac{1}{2} \right)^{10} = 
h\left(-\frac{1}{3}\right) \ .
\]
En fait, le lecteur peut vérifier que $h(x_1) = h(x_2)$ si
$x_1 + 2x_1 x_2 + x_2 = 0$.  Donc $h$ n'est pas inversible sur son
domaine.

\noindent {\bfseries Note}: Si nous restreignons la fonction $h$ à
$[0,\infty[$, alors $h:[0,\infty[ \rightarrow [0,1[$ est inversible.
Plus précisément, puisque
\begin{align*}
h(x_1) = h(x_2) &\Rightarrow \left( \frac{x_1}{x_1+1} \right)^{10} =
\left( \frac{x_2}{x_2+1} \right)^{10}
\Rightarrow \frac{x_1}{x_1+1} = \frac{x_2}{x_2+1} \\
&\Rightarrow x_1(x_2+1) = x_2(x_1+1)
\Rightarrow x_1 x_2 + x_1 = x_1 x_2 + x_2
\Rightarrow x_1 = x_2 \ ,
\end{align*}
nous avons que $h$ est injective sur $[0, \infty[$.  Nous avons utilisé le fait
que $\displaystyle \frac{x_1}{x_1+1} \geq 0$ et
$\displaystyle = \frac{x_2}{x_2+1} \geq 0$ pour obtenir la deuxième
implication ci-dessus.  L'inverse de $h$ est donné par
\begin{align*}
y = h(x) & \Leftrightarrow y = \left( \frac{x}{x+1} \right)^{10}
\Leftrightarrow y^{1/10} = \frac{x}{x+1} \Leftrightarrow 
y^{1/10} (x+1) = x\\
&\Leftrightarrow y^{1/10} x - x = -y^{1/10} \Leftrightarrow
x( y^{1/10} - 1) = -y^{1/10}
\Leftrightarrow x = \frac{y^{1/10}}{1 - y^{1/10}} \ .
\end{align*}
Nous avons utilisé le fait que $\displaystyle \frac{x}{x+1} \geq 0$ pour
obtenir la deuxième équivalence ci-dessus.  Par convention, nous écrivons
$\displaystyle y=h^{-1}(x)= \frac{x^{1/10}}{1 - x^{1/10}}$.  Le
domaine de $h^{-1}$ est l'image de $h$ où $h$ est restreint à
$[0,\infty[$.  C'est-à-dire que le domaine de $h^{-1}$ est $[0,1[$.
L'image de $h^{-1}$ est $[0,\infty[$, le domaine que nous avons assigné à
$h$.

Nous pourrions montrer que si nous restreignons la fonction $h$ à
$]-\infty,-1[$, alors $h:]-\infty,-1[ \rightarrow ]1,\infty[$ est
aussi inversible.  De même, si nous restreignons la fonction $h$ à $]-1,0]$,
alors $h:]-1,0] \rightarrow [0,\infty[$ est inversible.
}

\compileSOL{\SOLUb}{\ref{2Q14}}{
\[
(f\circ g)(x) = f(g(x)) = \sqrt{g^2(x) - 1}
= \sqrt{ (\sqrt{x^2+1})^2 - 1}
= \sqrt{ (x^2+1) - 1} = \sqrt{x^2} = |x| \ .
\]
}

\compileSOL{\SOLUb}{\ref{2Q15}}{
\subQ{a} Nous avons $\displaystyle f(x) = f_1(f_2(x))$ où $f_1(y) = 5 y^{-1}$
et $f_2(x) = 1+5^x$.

\subQ{b} Nous avons $\displaystyle h(t) = h_1(h_2(t))$ où $h_1(x) = x^{-4}$
et $h_2(t) = 1-t^2$.
}

\compileSOL{\SOLUb}{\ref{2Q16}}{
Nous avons que $\displaystyle g(x) = g_1(g_2(g_3(x)))$ où $g_1(x) = \cos(x)$,
$g_2(x) = \sqrt{x} = x^{1/2}$ et $g_3(x) = 1+x^2$.
}

\subsection{Trigonométrie}

\compileSOL{\SOLUb}{\ref{2Q18}}{
Puisque $\theta$ est un angle connu (i.e. $\theta = \pm \pi/3 + 2n\pi$),
nous utilisons le cercle unité pour répondre à cette question.
\PDFgraph{2_fonctions/unitCircle}
Nous trouvons donc $\sin(\theta) = \pm \sqrt{3}/{2}$.

Nous aurions pu utiliser l'identité $\cos^2(\theta) + \sin^2(\theta) = 1$
pour obtenir le même résultat.
}

\compileSOL{\SOLUb}{\ref{2Q19}}{
\subQ{a} La fonction est de la forme
\[
x = f(t) = M + A \cos\left(\frac{2\pi}{P}\, (t-T) \right)
\]
où l'amplitude est $A = (9-(-1))/2= 5$, la moyenne est $M=(9+(-1))/2= 4$,
la période est $P=2$ et la phase est $T=0$.  Donc
\[
x = f(t) = 4 + 5 \cos\left(\frac{2\pi}{2}(t-0)\right)
= 4 + 5 \cos\left(\pi t\right) \ .
\]

\subQ{b} La fonction est de la forme
\[
x = g(t) = M + A \cos\left(\frac{2\pi}{P}\, (t-T) \right)
\]
où l'amplitude est $A=(8-(-2))/2= 5$, la moyenne est $M=(8+(-2))/2= 3$,
la période est $P=2$ et la phase est $T=0.5$. Donc
\[
x = g(t) = 3 + 5 \cos\left(\frac{2\pi}{2}(t- 0.5)\right)
= 3 + 5 \cos\left(\pi (t-0.5) \right) \ .
\]

\subQ{c} La fonction est de la forme
\[
x = V(t) = M + A \cos\left(\frac{2\pi}{P}\, (t-T) \right)
\]
où la moyenne est $M = (2.2+1.8)/2 = 2$, l'amplitude
$A = (2.2-1.8)/2 = 0.2$, la période est $0.2$, et la phase est
$T = 0.1$.  Donc
\begin{align*}
x = V(t) &= M + A\;\cos\left( \frac{2\pi}{P} \left( t - T\right) \right)
= 2 + 0.2\;\cos\left( \frac{2\pi}{0.2} \left( t - 0.1\right) \right) \\
&= 2 + 0.2\;\cos\left(10\pi (t - 0.1)\right) \ .
\end{align*}

\subQ{d} La fonction est de la forme
\[
x = h(t) = M + A \cos\left(\frac{2\pi}{P}\, (t-T) \right)
\]
où la moyenne est $M=3$, l'amplitude est $A=2$, la période est $P=10$
et la phase est $T = 1$.  Donc
\[
x = h(t) = 3 + 2 \cos\left(\frac{2\pi}{10}\, (t-1) \right)
= 3 + 2 \cos\left(\frac{\pi}{5}\, (t-1) \right) \ .
\]
}

\compileSOL{\SOLUb}{\ref{2Q20}}{
\subQ{a} Puisque $\cos(t) = -\cos(t-\pi)$, nous pouvons écrire
\[
f(t) = 2 - \cos(t) = 2 + \cos(t-\pi) =
M + A \cos\left( \frac{2\pi}{P}\left( t- T\right)\right)
\]
avec $M=2$, $A=1$, $P=2\pi$ et $T = \pi$.

\subQ{b} Nous avons
\[
f(t) = 2 + \sin(t) = 2 + \cos\left( t - \frac{\pi}{2}\right)
= M + A\;\cos\left( \frac{2\pi}{P} \left( t - T\right) \right) \ ,
\]
où $M=2$ est la moyenne, $A=1$ est l'amplitude, $P=2\pi$ est la
période et $\displaystyle T = \frac{\pi}{2}$ est la phase.

\subQ{c} Puisque $\sin(\pi/2 - \theta) = \cos(\theta)$, nous avons que
\[
\sin\left(\frac{\pi}{2}(2-t)\right) =
\sin\left(\frac{\pi}{2} - \frac{\pi}{2}(t-1)\right) =
\cos\left(\frac{\pi}{2}(t - 1)\right) \ .
\]
Ainsi
\[
f(t) = -5 + 7\sin\left(\frac{\pi}{2}(2-t)\right)
= -5 + 7 \cos\left(\frac{\pi}{2}(t - 1)\right)
= M + A \cos\left( \frac{2\pi}{P} (t - T)\right)
\]
où $M= -5$ est la moyenne, $A=7$ est l'amplitude, $P=4$ est la
période et $T=1$ est la phase.
}

\compileSOL{\SOLUb}{\ref{2Q21}}{
\subQ{a}
\[
  h(z) = 1 + 5 \cos\left(\frac{\pi z}{2} - \frac{3\pi}{2}\right)
  = 1 + 5 \cos\left(\frac{2\pi}{4}(z -3)\right)
  = M + A \cos\left( \frac{2\pi}{P}(z - Z)\right)
\]
où $M=1$ est la moyenne, $A=5$ est l'amplitude, $P=4$ est la période
et $Z=3$ est la phase.
\PDFgraph{2_fonctions/sinusoid3}

\subQ{b}
\begin{align*}
f(x) &= 6 \sin(3x-6) -4 = -4 + 6 \cos\left((3x-6) - \frac{\pi}{2}\right) \\
& = -4 + 6 \cos\left(\frac{2\pi}{2\pi/3}
\left( x - \left(2 + \frac{\pi}{6}\right)\right)\right)
= M + A \cos\left( \frac{2\pi}{P} \left( x - Z\right)\right) \ .   
\end{align*}
L'amplitude est $A = 6$.  La période est $P = 2 \pi/3$, la moyenne est
$M = -4$ et la phase est $\displaystyle Z = 2 + \frac{\pi}{6}$.  
Remarque que $\displaystyle Z = 2 - \frac{\pi}{2}$ serait préférable
et acceptable car la période est $2\pi/3$.
\PDFgraph{2_fonctions/ass2Y}

\subQ{c}
\[
W(y) = -2.0 + 3.0\cos\left( \frac{2\pi}{0.5}(y + 0.1)\right) =
M + A\;\cos\left( \frac{2\pi}{P} ( y - Z) \right)
\]
où $M=-2$ est la moyenne, $A=3$ est l'amplitude, $P=0.5$ est la période et
$\displaystyle Z = 0.4$ est la phase.  Remarquons que $Z=-0.1$
serait aussi acceptable pour la phase car la période est $0.5$.

Le graphe de $W$ est donnée ci-dessous.
\PDFgraph{2_fonctions/ass2Z}
}

\compileSOL{\SOLUb}{\ref{2Q22}}{
La période $P$ est $24$ heures, la moyenne $M$ est $(37.2+36.6)/2 = 36.9$,
l'amplitude $A$ est $(37.2-36.6)/2 = 0.3$ et la phase $T$ est $13$.
Ainsi,
\[
T(t) = 36.9 + 0.3 \cos\left( \frac{2\pi}{24} (t - 13) \right)
= 36.9 + 0.3 \cos\left(\frac{\pi}{12}(t - 13)\right)
\]
Le graphe de cette fonction est
\MATHgraph{2_fonctions/sinusoidale3}{8cm}
}

\compileSOL{\SOLUa}{\ref{2Q23}}{
Nous avons la figure ci-dessous où le Théorème de Pythagore a été
utilisé pour trouver la longueur de la diagonale du rectangle.
\PDFgraph{2_fonctions/triangle2}
Nous trouvons donc que
$\sin(\theta/2) = 2/\sqrt{13}$ et $\cos(\theta/2) = 3/\sqrt{13}$.
Ainsi
\[
\cos(\theta) = \cos\left(\frac{\theta}{2} + \frac{\theta}{2}\right)
= \cos\left(\frac{\theta}{2}\right)\cos\left(\frac{\theta}{2}\right)
- \sin\left(\frac{\theta}{2}\right)\sin\left(\frac{\theta}{2}\right)
= \frac{9}{13} - \frac{4}{13} = \frac{5}{13} \ .
\]
}

\compileSOL{\SOLUb}{\ref{2Q24}}{
Nous pourrions toujours utiliser les identités trigonométriques pour
résoudre cette question.  Mais si vous êtes comme l'auteur de ce
manuel et avez de la difficulté à mémoriser toutes ces formules, un
retour à la définition du cosinus et du sinus à partir d'un triangle
peut résoudre cette question.

Nous obtenons le dessin suivant de $\displaystyle \sin(\theta)=\frac{1}{x}$.
\PDFgraph{2_fonctions/triangle3}

Ainsi, $\displaystyle \tan(\theta) = \frac{1}{\sqrt{x^2-1}}$.  La valeur
négative est donnée par l'angle obtus.
}

\compileSOL{\SOLUa}{\ref{2Q25}}{
Soit $\theta = \arcsin(x)$ où $-\pi/2 < \theta < \pi/2$.  Donc
$\sin(\theta) = x$ et nous obtenons la figure suivante.
\PDFgraph{2_fonctions/trig8}
Ainsi, $\tan(\arcsin(x)) = \tan(\theta) = x/\sqrt{1-x^2}$.
}

\subsection{Fonctions exponentielles}

\compileSOL{\SOLUb}{\ref{2Q26}}{
Si $f(x) = p b^x$, alors
\begin{align*}
b^2 &= \frac{p b^4}{p b^2} = \frac{f(4)}{f(2)}
 = \frac{0.1875}{0.75} \Rightarrow b = 0.5 \\
b^2 &= \frac{p b^6}{p b^4} = \frac{f(6)}{f(4)}
= \frac{0.046875}{0.1875} \Rightarrow b = 0.5
\intertext{et}
b^2 &= \frac{p b^8}{p b^6} = \frac{f(8)}{f(6)} 
= \frac{0.011719}{0.046875} \Rightarrow b \approx 0.50000533
\end{align*}
Nous pouvons supposer que $f(x) = p (1/2)^x$.  Puisque
$f(2) = p (1/2)^2 = 0.75$ par hypothèse, nous obtenons $p = 3$.
Ainsi, $f(x)= 3 (1/2)^x$.

Si $g(x) = m x + b$, alors
\begin{align*}
2m &= (4m+b) - (2m+b) = g(4) - g(2) = 10.3 - 5.7 = 4.6 \Rightarrow
m = 2.3 \\
2m &= (6m+b) - (4m+b) = g(6) - g(4) = 14.9 - 10.3 = 4.6 \Rightarrow
m = 2.3
\intertext{et}
2m &= (8m+b) - (6m+b) = g(8) - g(6) = 19.5 - 14.9 = 4.6 \Rightarrow
m = 2.3
\end{align*}
Nous pouvons supposer que $g(x) = 2.3 x + b$.  Puisque
$g(2) = 2.3 \times 2 + b = 5.7$, nous obtenons $b = 1.1$.  Ainsi,
$g(x)= 2.3 x + 1.1$ .
}

\compileSOL{\SOLUa}{\ref{2Q27}}{
Nous avons $\displaystyle S(t) = S_0 10^{0.693 t}$.  Pour répondre à la
première question, il faut trouer $t$ tel que $S(t) = 2 S(0)$.  Donc
\begin{align*}
S_0 10^{0.693 t} = 2 S_0 & \Rightarrow 10^{0.693 t} = 2
\Rightarrow \log_{10}\left(10^{0.693 t}\right) = \log_{10}(2)
\Rightarrow 0.693 \; t = \log_{10}(2) \\
&\Rightarrow t = \frac{\log_{10}(2)}{0.693} = 0.4343876\ldots
\end{align*}
Remarquons que la solution ne dépend pas de la valeur de $S_0$.

Pour répondre à la deuxième question, il faut trouer $t$ tel que
$S(t) = 10 S(0)$.  Donc
\[
S_0 10^{0.693 t} = 10 S_0 \Rightarrow 10^{0.693 t} = 10
\Rightarrow 0.693\; t = 1
\Rightarrow t = \frac{1}{0.693} = 1.443001\ldots
\]
}

\compileSOL{\SOLUb}{\ref{2Q28}}{
Puisque la population double tous les $24$ ans, nous avons que
$p(24) = 2 p(0)$.  Ainsi,
\begin{align*}
p_0 e^{24 \alpha} = 2 p_0 & \Rightarrow e^{24 \alpha} = 2
\Rightarrow \ln\left(e^{24 \alpha}\right) = \ln(2)
\Rightarrow 24 \alpha = \ln(2) \\
&\Rightarrow \alpha = \frac{\ln(2)}{24} = 0.02888\ldots
\end{align*}

Donc $\displaystyle p(t) \approx p_0 e^{0.02888\; t}$.  Puisque
$\displaystyle p(12) \approx 500 e^{0.02888\times 12} = 707.1067\ldots$, 
nous avons donc approximativement $707$ individus (par km$^2$) après
$12$ ans.
}

\compileSOL{\SOLUa}{\ref{2Q29}}{
Pour trouver les unités de $\alpha$, nous écrivons l'équation
$V(t) = V_0 e^{\alpha t}$ en termes des unités seulement.
\[
\text{cm}^3 = \text{cm}^3\ e^{(\text{unités de }\alpha) \times s} \ .
\]
Pour que l'équation soit satisfaite, il faut que
$\displaystyle e^{(\text{unités de }\alpha) \times s}$ n'est pas d'unités.  Il
faut donc que les unités de $\alpha$ soient $s^{-1}$.

La valeur initiale n'est pas nécessaire pour déterminer le temps qu'il
faut pour que le volume double, quadruple, etc.  Pour
déterminer le temps nécessaire pour que le volume double, nous cherchons
$t$ tel que $V_0 e^{1.1 t} = 2 V_0$; c'est-à-dire, $e^{1.1 t} = 2$.
Donc $\displaystyle t = \frac{\ln(2)}{1.1} \approx 0.63013380$ s.
Ainsi, le volume double à tous les $0.63013380$ s.  Le volume sera quatre
fois plus grand après $2 \times 0.63013380$ s, le volume sera huit
fois plus grand après $2 \times 2 \times 0.63013380$ s, etc.
}

\compileSOL{\SOLUb}{\ref{2Q30}}{
Nous cherchons $t$ tel que
$\displaystyle Q(t) = Q_0 e^{-0.000122\; t} = \frac{1}{2} Q_0$.
Comme $Q_0 \neq 0$, nous pouvons diviser des deux côtés pour obtenir
\begin{align*}
Q_0 e^{-0.000122\; t} = \frac{1}{2} Q_0 &\Rightarrow
e^{-0.000122\; t} = \frac{1}{2} = 2^{-1} \\
&\Rightarrow -0.000122\; t = \ln\left(2^{-1}\right) = - \ln(2) \\
&\Rightarrow t = \frac{-\ln(2)}{-0.000122} = 5,681.53426\ldots
\ \text{années.}
\end{align*}
}

\compileSOL{\SOLUb}{\ref{2Q31}}{
Puisque $\displaystyle 200 = \frac{1}{2^3} \times 1600$, nous pouvons
utiliser un raisonnement élémentaire pour répondre à la première
question.  Après $43$ ans, la population sera de $800$ individus
(par m$^2$).  Après un autre $43$ ans, la population sera de $400$
individus (par m$^2$).  Finalement, après un autre $43$ ans, la
population sera de $200$ individus (par m$^2$).  Il faut donc
$3\times  43 = 129$ ans pour que la population atteigne $200$
individus (par m$^2$). 

La méthode que nous venons d'utiliser ne peut pas déterminer le nombre
d'années pour atteindre une population de $437$ individus.  Il faut
trouver $P(t)$.

Nous savons que $P(t) = P_0 e^{\alpha t}$, où $\alpha$ est le taux de
croissance relatif et $P_0$ est le nombre initial d'individus (par
m$^2$) dans la population.  Pour déterminer $\alpha$, nous utilisons le
fait que la demie-vie est de $43$ ans. Donc
\[
P_0 e^{43 \alpha} = \frac{1}{2} P_0
\Rightarrow e^{43 \alpha} = \frac{1}{2}
\Rightarrow 43 \alpha = \ln\left(\frac{1}{2}\right) = - \ln(2)
\Rightarrow \alpha = \frac{- \ln(2)}{43} \approx
-0.0161197
\]

Nous cherchons $t$ tel que $\displaystyle 1600 e^{\alpha t} = 437$.  Donc
\begin{align*}
1600 e^{\alpha t} = 437
&\Rightarrow e^{\alpha t} = \frac{437}{1600}
\Rightarrow \alpha t = \ln\left(\frac{437}{1600}\right)
= \ln(437)-\ln(1600) \\
&\Rightarrow t = \frac{\ln(437)-\ln(1600)}{\alpha}
\approx \frac{\ln(437)-\ln(1600)}{-0.0161197}
\approx 80.511768 \ \text{ans}.
\end{align*}
}

\compileSOL{\SOLUb}{\ref{2Q32}}{
Nous pouvons \lgm imaginer\rgm \ que $g$ est la fonction $\cos(2 \pi t)$
avec une amplitude $e^t$ qui augmente lorsque $t \to \infty$ et
converge vers zéro lorsque $t \to -\infty$..
La fonction $\cos(2\pi t)$ est une fonction périodique de période
$1$.  De plus, $\cos(2\pi t) =0$ pour $t = (2n+1)/4$ avec
$n \in \ZZ$.  Le graphe de $g$ est donnée ci-dessous.
\MATHgraph{2_fonctions/expcos_a}{8cm}
}

\compileSOL{\SOLUa}{\ref{2Q33}}{
Nous pouvons \lgm imaginer\rgm \ que $W$ est la fonction $\cos(2 \pi t)$
avec une amplitude $e^{-t}$ qui augmente lorsque $t \to -\infty$ et
converge vers zéro lorsque $t \to \infty$.  La fonction
$\cos(2\pi t)$ est une fonction périodique de période $1$.  De plus,
$\cos(2\pi t) =0$ pour $t = (2n+1)/4$ avec $n \in \ZZ$.
Le graphe de $W$ est donnée ci-dessous.
\MATHgraph{2_fonctions/expcos_b}{8cm}
}

%%% Local Variables: 
%%% mode: latex
%%% TeX-master: "notes"
%%% End: 

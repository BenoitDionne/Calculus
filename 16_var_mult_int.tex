\chapter[Intégrales multiples \eng]{Intégrales multiples\\ \eng}
\label{chapIntMult}

\compileTHEO{

Dans ce chapitre, nous généralisons la notion d'intégration d'une fonction
définie sur un intervalle à l'intégration d'une fonction à valeurs
réels définie sur un domaine de $\RR^n$ où $n \geq 2$.  Grâce au
Théorème de Fubini, nous pourrons réduire le calcul d'intégrales de
fonctions de plusieurs variables au calcul d'intégrales
{\em emboîtées} de fonctions définies sur des intervalles.  Nous
terminons le chapitre avec une section sur la méthode de substitution
pour les intégrales de fonctions de plusieurs variables.  Chose
surprenante, le déterminant va jouer une rôle important dans la
méthode de substitution.

Ce chapitre exige de bonnes capacités à visualiser les objets dans
l'espace.

\section{Définition de l'Intégrale}\label{DefIntMulti}

La définition de l'intégrale double d'une fonction $f:D\rightarrow \RR$ où
$D \subset \RR^2$ est un rectangle est très semblable à celle de l'intégrale
définie d'une fonction d'une variable $f:I\rightarrow \RR$ où $I\subset \RR$
est un intervalle.  Au lieu de partager un intervalle en sous-intervalles,
nous partageons un rectangle en sous-rectangles.

Soit $f:D\rightarrow \RR$ un fonction continue (sauf possiblement à quelques
points) et bornée sur le rectangle
\[
D=\left\{(x,y): a\leq x \leq b \text{ et } c\leq y \leq d \right\} \; .
\]
Soit $a=x_0 < x_1 < x_2 < \ldots < x_p = b$ une partition de l'intervalle
$[a,b]$ et $c=y_0 < y_1 < y_2 < \ldots < y_{q} = b$ une partition de
l'intervalle $[c,d]$.  Pour $0 \leq i < p$ et $0 \leq j < q$, nous
définissons les rectangles
\[
A_{i,j} = \left\{ (x,y) : x_i\leq x < x_{i+1} \text{ et }
y_j\leq y < y_{j+1} \right\} \; .
\]
L'ensemble des rectangles $A_{i,j}$ forme une
{\bfseries partition}\index{Partition} $\cal P$
de $D$ en sous-rectangles (figure~\ref{VOLUME2}).

\PDFfig{16_var_mult_int/volume2}{Partition d'un rectangle en sous-rectangle}
{Partition d'un rectangle $D$ en $p\times q$ sous-rectangle $A_{i,j}$ pour
$i=0$,$1$, \ldots, $p-1$ et $q=0$, $1$, \ldots, $q-1$.}{VOLUME2}

L'aire d'un rectangle $A_{i,j}$ est
$\Delta A_{i,j} = \Delta x_i \, \Delta y_j$ où $\Delta x_i = x_{i+1}-x_i$ et
$\Delta y_j = y_{j+1}-y_j$.  Nous définissons une mesure de la grosseur de la
partition $\cal P$ par
\[
\| {\cal P} \| = \max \left\{ \Delta x_i + \Delta y_j : 0\leq i < p
\text{ et } 0 \leq j < q \right\} \; .
\]

Pour chaque rectangle $A_{i,j}$, nous choisissons
$\left(x_i^\ast, y_j^\ast\right)$ dans ce rectangle.  La somme
\[
S_{\cal P} = \sum_{A_{i,j}\in\cal P} f\left(x_i^\ast,y_j^\ast\right) \Delta A_{i,j}
= \sum_{\substack{0 \leq i < p\\ 0 \leq j < q}}
f\left(x_i^\ast,y_j^\ast\right) \Delta x_i \, \Delta y_j
\]
est une {\bfseries somme de Riemann}\index{Somme de Riemann} pour la
partition ${\cal P}$.
Une même partition a un nombre infini de sommes de
Riemann selon le choix des points
$\left(x_i^\ast, y_j^\ast\right) \in A_{i,j}$ pour
$0\leq i <p$ et $0\leq j < q$. 

\begin{defn}[+\theory] \label{defDI_rig}
Soit $f:D\rightarrow \RR$ un fonction continue presque
partout\footnotemark\ et bornée sur le rectangle
\[
D=\left\{(x,y): a\leq x \leq b \text{ et } c\leq y \leq d \right\}
\; .
\]
L'{\bfseries intégrale double}\index{Intégrale double} de $f$ sur $D$,
dénotés $\displaystyle \iint_D f(x,y) \dx{A}$, est le nombre $I$ tel
que, pour tout $\epsilon > 0$, il existe $\delta >0$ pour lequel
\[
\left| I - S_{\cal P} \right| < \epsilon
\]
quelle que soit la partition ${\cal P}$ avec $\|{\cal P}\|<\delta$
et le choix de points $(x_i^\ast,y_j^\ast)$ utilisés pour la somme de
Riemann $S_{\cal P}$.  Le rectangle $D$ est appelé le
{\bfseries domaine d'intégration}\index{Domaine d'intégration}.  La
fonction $f$ est appelée {\bfseries l'intégrande}\index{Intégrande}.
\end{defn}

\footnotetext{Dans les livres d'analyse, ils parlent d'un ensemble de
mesure zéro.  Pour nous, ce sont des ensembles qui contiennent
seulement des points ou des courbes isolées, dont l'aire est nul.}

Comme à la Section~\ref{Riemann_int}, nous pourrions donner une
définition plus générale de l'intégrale double.

Il découle de la définition de l'intégrale double que nous pouvons
estimer la valeur de l'intégrale
$\displaystyle \iint_D f(x,y) \dx{A}$ en choisissant
des sommes de Riemann pour des partitions ${\cal P}_n$ telles que
$\|{\cal P}_n\|$ approche $0$ lorsque $n$ augmente.  Nous avons
\begin{equation}\label{defDI_int_cpm}
\iint_D f(x,y) \dx{A} = \lim_{n\rightarrow 0}  S_{{\cal P}_n} \; .
\end{equation}
Si $f(x,y) \geq 0$ pour tout $(x,y) \in D$, alors
$f\left(x_i^\ast,y_j^\ast\right) \Delta x_i \, \Delta y_j$ est
le volume du solide définie par
\[
T_{i,j} = \left\{ (x,y,z) : (x,y) \in A_{i,j} \text{ et }
0 \leq z \leq f(x^\ast,y^\ast) \right\}
\]
(figure~\ref{VOLUME1}).
Nous avons alors que $\displaystyle \int_D f(x,y) \dx{A}$ est le volume du
solide définie par
\[
T = \left\{ (x,y,z) : (x,y) \in D \text{ et } 0 \leq z \leq f(x,y)
\right\} \; ,
\]
le solide dont la base est le rectangle $D$, les cotés sont les plans
verticaux $x=a$, $x=b$, $y=c$ et $y=d$, et qui est sous la surface
$y=f(x,y)$.

\PDFfig{16_var_mult_int/volume1}{Illustration de ce que représente un terme de
la somme de Riemann dans la définition de l'intégrale double} {Illustration
de ce que représente un terme de la somme de Riemann dans la définition de
l'intégrale double.}{VOLUME1}

\begin{defn}
Si $f:R\rightarrow \RR$ est une fonction continue presque partout et
bornée dont le domaine $R$ est un ensemble borné qui n'est pas un
rectangle, nous choisissons un rectangle $D$ qui contient $R$ et nous
posons
\[
g(x,y) = \begin{cases} f(x,y) & \quad \text{si} \quad (x,y) \in R \\
0 & \quad \text{si} \quad (x,y) \in D\setminus R
\end{cases} \; .
\]
L'intégrale de $f$ sur $R$ est alors définie par
\[
\iint_R f(x,y) \dx{A} = \iint_D g(x,y) \dx{A} \; .
\]
\end{defn}

Nous pourrions de la même façon définir l'intégrale triple d'une fonction
$f:\RR^3 \to \RR$ où nous utiliserions des boites dans $\RR^3$ au lieu
des rectangles de $\RR^2$ pour définir l'intégrale.  Nous utilisons la
notation $\displaystyle \iiint_D f(\VEC{x}) \dx{V}$ pour
désigner l'{\bfseries intégrale triple}\index{Intégrale triple}.
Toutes les propriétés que nous allons énoncer pour l'intégrale double
sont aussi vrais pour l'intégrale triple.

Comme la définition de l'intégrale double ressemble énormément à celle de
l'intégrale d'une fonction d'une variable, il n'est pas surprenant que
l'intégrale double possède des propriétés semblables à celle de l'intégrale
d'une fonction d'une variable.

\begin{prop}
\begin{enumerate}
\item Soit $D\subset \RR^2$ et $f,g:D \rightarrow \RR$ deux fonctions
continues presque partout et bornées sur $D$.  Alors
\[
\iint_D \left( \alpha f(x,y) + \beta g(x,y) \right) \dx{A} =
\alpha \iint_D f(x,y) \dx{A} + \beta \iint_D g(x,y) \dx{A}
\]
quels que soient $\alpha$ et $\beta$ dans $\RR$.
\item Si $D = D_1 \cup D_2 \subset \RR^2$ où $D_1 \cap D_2$ est un
ensemble négligeable de points\footnotemark, et $f:D\rightarrow \RR$
est une fonction continue presque partout et bornée, alors
\[
\iint_D d \dx{A} = \iint_{D_1} f \dx{A} + \iint_{D_2} f \dx{A} \; .
\]
\end{enumerate}
\end{prop}

\footnotetext{Un ensemble négligeable est ce que nous avons caractérisé
précédemment comme étant un ensemble de mesure zéro, soit un ensemble
qui contient seulement des points ou des courbes isolées,}

\section{Théorème de Fubini}

Il n'est pas toujours facile d'évaluer les intégrales doubles lorsque le
domaine d'intégra\-tion n'est pas un rectangle.  Le théorème suivant va nous
permette de calculer les intégrales doubles sur des domaines $D$ \flqq
simple\frqq\ à l'aide des techniques d'intégration de fonctions d'une
variable que nous avons étudiées précédemment.

\begin{theorem}[Théorème de Fubini]
Soit $f:D\rightarrow \RR$ un fonction continue presque partout et
bornée sur le rectangle
\[
D=\left\{(x,y): a\leq x \leq b \text{ et } c\leq y \leq d \right\} \; .
\]
Nous avons que
\[
\iint_D f(x,y) \dx{A} = \int_a^b \int_c^d f(x,y) \dx{y} \dx{x}
= \int_c^d \int_a^b f(x,y) \dx{x} \dx{y} \; .
\]
\end{theorem}

En d'autres mots, il y a des choix pour évaluer l'intégrale de $f$ sur
un rectangle $D$,
\begin{enumerate}
\item Nous évaluons l'intégrale $\displaystyle \int_c^d f(x,y) \dx{y}$ en
assumant que $x$ est constant pour obtenir une fonction $g(x)$ et par la
suite nous évaluons l'intégrale $\displaystyle \int_a^b g(x) \dx{x}$.
\item Nous évaluons l'intégrale $\displaystyle \int_a^b f(x,y) \dx{x}$ en
assumant que $y$ est constant pour obtenir une fonction de $h(y)$ et par la
suite nous évaluons l'intégrale $\displaystyle \int_c^d h(y) \dx{y}$.
\end{enumerate}

Soit
\[
R = \left\{ (x,y) : a\leq x \leq b \text{ et } g_1(x) \leq y \leq
g_2(x) \right\} \; ,
\]
le domaine d'intégration qui est représenté dans le dessin à gauche dans
la figure~\ref{VOLUME3}.  Il découle du théorème de Fubini et de la définition
d'une intégrale double sur un domaine qui n'est pas un rectangle que
\[
\iint_R f(x,y) \dx{A} = \int_a^b \int_{g_1(x)}^{g_2(x)} f(x,y) \dx{y} \dx{x}
\; .
\]
De même, soit
\[
R = \left\{ (x,y) : c\leq y \leq d \text{ et } h_1(y) \leq x \leq
h_2(y) \right\} \; ,
\]
le domaine d'intégration qui est représenté dans le dessin à droite dans la
figure~\ref{VOLUME3}.  Nous avons que
\[
\iint_R f(x,y) \dx{A} = \int_c^d \int_{h_1(y)}^{h_2(y)} f(x,y) \dx{x} \dx{y}
\; .
\]

\PDFfig{16_var_mult_int/volume3}{Intégrale double à l'aide du théorème de
Fubini}{Intégrale double pour des domaines qui se prête bien au théorème de
Fubini.}{VOLUME3}

\begin{egg}
Évaluons l'intégrale double $\displaystyle \iint_R (x-2y) \dx{A}$
où $R$ est le domaine définie par
\[
R = \left\{ (x,y) : 1 \leq x \leq 3 \text{ et }
1+x \leq y \leq 2x \right\} \; .
\]
Le dessin du domaine $R$ est donné ci-dessous.
\PDFgraph{16_var_mult_int/volume4}
Nous avons
\begin{align*}
\iint_R (x-2y) \dx{A} &= \int_1^3 \int_{x+1}^{2x} (x-2y) \dx{y}\dx{x}
= \int_1^3 \left(xy-y^2\right)\bigg|_{y=x+1}^{y=2x} \dx{x} \\
&= \int_1^3 \left( (2x^2-4x^2) - \left(x(x+1)-(x-1)^2\right) \right)\dx{x} \\
&= \int_1^3 \left( -2x^2 - 3 x + 1\right) \dx{x}
= \left( -\frac{2x^3}{3} - \frac{3x}{2} + x\right)\bigg|_1^3
= - \frac{82}{3} \; .
\end{align*}
\end{egg}

\begin{egg}
Évaluons l'intégrale double $\displaystyle \iint_R 3x \dx{A}$
où $R$ est le domaine bornée par $y=x$ et $y=x^2-4x+4$.

La valeur de $x$ pour les points d'intersection de ces deux courbes
est donné par
\[
  x = x^2-4x+4 \Rightarrow x^2-5x+4 = (x-4)(x-1) = 0
  \Rightarrow x=1 \quad \text{ou} \quad x = 4 \ .
\]
Il y a deux points d'intersection, $(1,1)$ et $(4,4)$.  Le dessin
du domaine $R$ est donné ci-dessous.
\PDFgraph{16_var_mult_int/volume5}

Nous déduisons de la forme du domaine qu'il est préférable d'intégrer par
rapport à $y$ en premier et par rapport à $x$ par la suite.  Ainsi,
\begin{align*}
\iint_R 3x \dx{A} &= \int_1^4 \int_{x^2-4x+4}^{x} 3x \dx{y}\dx{x}
= \int_1^4 \left( 3xy \right)\bigg|_{y=x^2-4x+4}^{x} \dx{x} \\
&= \int_1^4 \left( 3x^2 - 3x(x^2-4x+4) \right)\dx{x}
= \int_1^4 \left( -3x^3 +15 x^2 -12x \right) \dx{x} \\
&= \left( -\frac{3}{4}\, x^4 + 5\, x^3 - 6x^2\right)\bigg|_1^4
= \frac{135}{4} \; .
\end{align*}
\end{egg}

\begin{egg}
Évaluons l'intégrale double $\displaystyle \iint_R (y^2+ x) \dx{A}$ où
$R$ est le domaine bornée par les paraboles $x=y^2$ et $x=3-2y^2$.

La valeur de $x$ pour les points d'intersection de ces deux paraboles
est donné par
\[
  y^2 = 3-2y^2 \Rightarrow y^2 = 1 \Rightarrow x= \pm 1 \ .
\]
Il y a deux points d'intersection, $(1,1)$ et $(1,-1)$.  Le dessin du
domaine $R$ est donné ci-dessous.
\PDFgraph{16_var_mult_int/volume6}

Nous déduisons de la forme du domaine qu'il est préférable d'intégrer par
rapport à $x$ en premier et par rapport à $y$ par la suite.  Ainsi,
\begin{align*}
\iint_R (y^2+x) \dx{A} &= \int_{-1}^1 \int_{y^2}^{3-2y^2} (y^2-x) \dx{x}\dx{y}
= \int_{-1}^1 \left( xy^2 + \frac{1}{2}\, x^2 \right)
\bigg|_{x=y^2}^{3-2y^2} \dx{y} \\
&= \int_{-1}^1 \left( \left(y^2(3-2y^2) +\frac{1}{2}\,(3-2y^2)^2\right)
-\left( y^4  + \frac{1}{2}\, y^4\right) \right)\dx{y} \\
&= \int_{-1}^1 \left( -\frac{3}{2}\,y^4 -3\, y^2 +\frac{9}{2} \right) \dx{y}
= \left( -\frac{3}{10}\, y^5 - y^3 +\frac{9}{2}\,y\right)\bigg|_{-1}^1
= \frac{32}{5} \; .
\end{align*}
\end{egg}

\begin{egg}
Calculons le volume du solide borné par le cylindre $y^2+z^2 = 16$
et les plans $x=0$, $y=0$, $z=0$ et $y = 2x$.  Nous supposons que
$x,y,z \geq 0$.

Traçons un dessin du solide en question pour justifier le choix de
notre intégrale.  Le dessin du solide est donné ci-dessous.
\PDFgraph{16_var_mult_int/volume18}

Le volume de ce solide est donc
\[
V = \int_0^2 \int_{2x}^4 \sqrt{16-y^2} \dx{y}\dx{x} \ .
\]
Il serait possible de calculer cette intégrale à l'aide de la
substitution trigonométrique $y = 4\sin(\theta)$ mais nous pouvons éviter
ce trouble en changeant l'ordre d'intégration.
\[
V = \int_0^4 \int_0^{y/2} \sqrt{16-y^2} \dx{x}\dx{y}
= \int_0^4 \left( x \sqrt{16-y^2}\right)\bigg|_{x=0}^{x=y/2} \dx{y}
= \int_0^4 \frac{y}{2} \sqrt{16-y^2} \dx{y}
\]
Pour calculer cette intégrale, nous utilisons la substitution
$u= 16 - y^2$.  Donc $\dx{u} = -2 y \dx{y}$, $u=16$ lorsque $y=0$ et $u=0$
lorsque $y=4$.  Ainsi,
\[
V = -\frac{1}{4} \int_{16}^0 u^{1/2} \dx{u} 
= -\frac{1}{6} u^{3/2}\bigg|_{16}^0
= \frac{1}{6} (16)^{3/2} = \frac{32}{3} \ .
\]
\end{egg}

\section{Substitution}

Nous avons vu au Théorème~\ref{OneDSubstRule} la règle de substitution
suivante.  Si $g$ est une fonction différentiable (croissante ou
décroissante) et $f$ est un fonction intégrable sur l'image de $[a,b]$
par $g$, alors
\begin{equation}\label{SubtOriented}
\int_a^b f(g(x))g'(x)  \dx{x} = \int_{g(a)}^{g(b)} f(y) \dx{y} \; .
\end{equation}
Nous aimerions pouvoir généraliser cette méthode aux fonctions de
plusieurs variables.  Cependant, la formule précédente utilise le fait
qu'il y a une direction croissante évidente sur la l'axe des $x$.  Ce
n'est pas le cas si nous nous trouvons dans le plan ou dans l'espace.
l'intégrale de Riemann que nous avons introduite au
Chapitre~\ref{chapter_integr} et que nous utilisons dans la formule
(\ref{SubtOriented}) est orientée
(e.g. $\displaystyle \int_a^b = - \int_b^a$) alors que l'intégrale de
Riemann que nous avons introduite dans ce chapitre ne l'est pas.

Dans la formule ci-dessus, si $g'(x) < 0$ sur l'intervalle $[a,b]$,
alors $g(a) > g(b)$ et nous pouvons écrire
\begin{align*}
\int_a^b f(g(x)) |g'(x)|  \dx{x} &=
\int_a^b f(g(x)) (-g'(x))  \dx{x} = - \int_a^b f(g(x)) g'(x)  \dx{x} \\
&= -\int_{g(a)}^{g(b)} f(y) \dx{y} = \int_{g(b)}^{g(a)} f(y) \dx{y} \; ,
\end{align*}
où la borne inférieure est plus petite que la borne supérieure pour
cette dernière intégrale.  Si nous utilisons la définition de
l'intégrale que nous avons introduite au début du chapitre, nous
pouvons écrire
\begin{equation}\label{SubtNonOriented}
  \int_{[a,b]} f(g(x)) |g'(x)|  \dx{x} = \int_{g([a,b])} f(y) \dx{y} \; ,
\end{equation}
C'est donc la formule (\ref{SubtNonOriented}) que nous allons
généraliser.   Notons que
\[
  g([a,b]) \equiv \{ g(x) : x \in [a,b] \}
\]
est l'image de l'intervalle $[a,b]$ par la fonction $g$.

\begin{theorem}
Supposons que $g:\RR^2 \to \RR^2$ soit une fonction de classe $C^1$ et
$U$ est un ensemble ouvert de $\RR^2$ tel que $\diff g(\VEC{u})$ est
une matrice inversible pour tout $\VEC{u} \in U$.  Si $V = g(U)$ et
$f:V \to \RR$ est une fonction intégrable, alors
\[
  \iint_{U} f(g(\VEC{x})) |\det \diff g(\VEC{x})| \dx{A} =
  \iint_{V} f(\VEC{y}) \dx{A}
\]
(figure~\ref{substND}).
\index{Règle de substitution}\index{Changement de variables}
\label{NDSubstRule}
\end{theorem}

\PDFfig{16_var_mult_int/substND}{Changement de variable pour
intégrales multiples}{Représentation schématique du changement de
variable pour intégrales multiples}{substND}

Si nous posons $\VEC{y} = g(\VEC{x})$, alors $y_i = g_i(x_1,x_2)$ pour
$i=1$ et $2$, et
\[
\diff g(\VEC{x}) =
\begin{pmatrix}
  \displaystyle \pdydx{y_1}{x_1} & \displaystyle \pdydx{y_1}{x_2} \\[0.8em]
  \displaystyle \pdydx{y_2}{x_1} & \displaystyle \pdydx{y_2}{x_2}
\end{pmatrix}
\]
Le {\bfseries Jacobian}\index{Jacobian} de la fonction $g$, dénoté
$\displaystyle \frac{\partial(y_1,y_2)}{\partial(x_1,x_2)}$, est
définie par
\[
    \frac{\partial(y_1,y_2)}{\partial(x_1,x_2)} = \det \diff g(\VEC{x}) \ .
\]
C'est cette dernière expression pour $\det \diff g(\VEC{x})$
que nous utiliserons le plus souvent pour représenter la règle de
substitution précédente.
\[
  \iint_{U} f(g(\VEC{x})) \left|
    \frac{\partial(y_1,y_2)}{\partial(x_1,x_2)} \right|
  \dx{A} = \iint_{V} f(\VEC{y}) \dx{A}
\]

La règle de substitution a naturellement une extension à $\RR^n$ pour
$n>2$.  Par exemple, dans $\RR^3$, nous remplaçons l'intégrale double
par une intégrale triple pour obtenir une formule de la forme
\[
  \iiint_{U} f(g(\VEC{x})) |\det \diff g(\VEC{x})| \dx{V} =
  \iiint_{V} f(\VEC{y}) \dx{V}
\]
avec
\[
\diff g(\VEC{x}) =
\begin{pmatrix}
  \displaystyle \pdydx{y_1}{x_1} & \displaystyle \pdydx{y_1}{x_2}
  & \displaystyle \pdydx{y_1}{x_3} \\[0.8em]
  \displaystyle \pdydx{y_2}{x_1} & \displaystyle \pdydx{y_2}{x_2}
  & \displaystyle \pdydx{y_2}{x_3} \\[0.8em]
  \displaystyle \pdydx{y_3}{x_1} & \displaystyle \pdydx{y_3}{x_2}
  & \displaystyle \pdydx{y_3}{x_3}
\end{pmatrix} \ .
\]
Le {\bfseries Jacobian}\index{Jacobian} de la fonction $g$ est alors
définie par
\[
\frac{\partial(y_1,y_2,y_3)}{\partial(x_1,x_2,x_3)} = \det \diff g(\VEC{x}) \ .
\]

\begin{egg}
Afin d'obtenir une intégrale qui est simple à calculer, utilisons la
méthode de substitution pour évaluer l'intégrale
$\displaystyle \int_{-a}^a \int_0^{\sqrt{a^2-y^2}} (x^2 + y^2)^{3/2}
\dx{x} \dx{y}$.
Nous invitons le lecteur a essayer de calculer directement cette intégrale.

Il faut premièrement avoir une bonne idée du domaine d'intégration
$V$.  Le dessin du domaine d'intégration est donné ci-dessous.
\PDFgraph{16_var_mult_int/SubstNDEgg1}

Puisque $V$ est un demi-cercle de rayon $a$ centré à l'origine, nous
utilisons le changement de coordonnées en
{\bfseries coordonnées polaires}\index{Coordonnées polaires}.
La région $V$ est l'ensemble des points
\[
  (x,y) = g(r,\theta) = (r \cos(\theta),r \sin(\theta))
\]
pour
$(r,\theta) \in U = \{ (r,\theta) : 0 \leq r \leq a \ , \ -\pi/2 \leq \theta
\leq \pi/2 \}$.
Puisque
\[
\left| \frac{\partial(x,y)}{\partial(r,\theta)} \right| = 
  \left| \det \begin{pmatrix}
  \displaystyle \pdydx{x}{r} & \displaystyle \pdydx{x}{\theta}\\[0.8em]
  \displaystyle \pdydx{y}{r} & \displaystyle \pdydx{y}{\theta} 
  \end{pmatrix} \right|
= \left| \det \begin{pmatrix}
\cos(\theta) & -r \sin(\theta) \\
\sin(\theta) & r \cos(\theta)
  \end{pmatrix}\right|  = r \ ,
\]
nous obtenons
\begin{align*}
\int_{-a}^a \int_0^{\sqrt{a^2-y^2}} (x^2 + y^2)^{3/2} \dx{x} \dx{y}
&= \int_{-\pi/2}^{\pi/2} \int_{-2}^2 \left(r^2\cos^2(\theta) + r^2
   \sin^2(\theta)\right)^{3/2} r \dx{r}\dx{\theta} \\
&= \int_{-\pi/2}^{\pi/2} \int_{-2}^2 (r^2)^{3/2} r \dx{r} \dx{\theta}
=  \int_{-\pi/2}^{\pi/2} \int_{-2}^2 r^4 \dx{r} \dx{\theta} \\
&= \int_{-\pi/2}^{\pi/2} \frac{r^5}{5}\bigg|_{r=0}^{2} \dx{\theta}
= \frac{2^5}{5} \int_{-\pi/2}^{\pi/2} \dx{\theta}
= \frac{2^5}{5} \theta\, \bigg|_{-\pi/2}^{\pi/2} = \frac{32\pi}{5}
\end{align*}
\label{SubstNDEgg1Text}
\end{egg}

\begin{rmk}
À l'exemple précédent, $r$ représente la variable $x_1$ dans la
formule de la règle de substitution, théorème~\ref{NDSubstRule},
$\theta$ représente $x_2$, $x$ représente $y_1$ et $y$ représente
$y_2$.
\end{rmk}

\begin{egg}
Calculons la masse de la demi-sphère supérieure $S$ de rayon $2$ centrée à
l'origine dont la densité est donnée par $\rho(x,y,z) = z$.
\PDFgraph{16_var_mult_int/MassEgg1}

En coordonnées cartésienne, la masse est donnée par la formule
\[
  M = \int_{-2}^2 \int_{-\sqrt{4-x^2}}^{\sqrt{4-x^2}}
  \int_0^{\sqrt{4-x^2-y^2}} z \dx{z} \dx{y} \dx{x} \ .
\]
Il est certainement possible de calculer cette intégrale mais nous pouvons
simplifier l'intégrale à l'aide d'une substitution.  Nous allons
utilisé les {\bfseries coordonnées cylindriques}\index{Coordonnées
 cylindriques}.  La demi-sphère $S$ est l'ensemble des points
\[
  (x,y,z) = (r \cos(\theta), r\sin(\theta), z)
\]
pour $0 \leq \theta < 2\pi$, $0 \leq r \leq 2$ et
$ 0 \leq z \leq \sqrt{4-x^2-y^2} = \sqrt{4-r^2}$.
Puisque
\[
\left| \frac{\partial(x,y,z)}{\partial(r,\theta,z)} \right| = 
\left| \det \begin{pmatrix}
  \displaystyle \pdydx{x}{r} & \displaystyle \pdydx{x}{\theta}
  & \displaystyle \pdydx{x}{z} \\[0.8em]
  \displaystyle \pdydx{y}{r} & \displaystyle \pdydx{y}{\theta}
  & \displaystyle \pdydx{y}{z} \\[0.8em]
  \displaystyle \pdydx{z}{r} & \displaystyle \pdydx{z}{\theta}
  & \displaystyle \pdydx{z}{z}
\end{pmatrix} \right|
= \left| \det \begin{pmatrix}
  \cos(\theta) & -r \sin(\theta) & 0 \\
  \sin(\theta) & r \cos(\theta) & 0 \\
  0 & 0  & 1
\end{pmatrix} \right|
= r \ ,
\]
nous avons
\begin{align*}
M &= \int_0^{2\pi} \int_0^2 \int_0^{\sqrt{4-r^2}} z r \dx{z} \dx{r} \dx{\theta}
= \int_0^{2\pi} \int_0^2
\left( \frac{z^2}{2} \bigg|_{z=0}^{\sqrt{4-r^2}} \right) r \dx{r} \dx{\theta} \\
& = \int_0^{2\pi} \int_0^2 \frac{1}{2} (4-r^2) r \dx{r} \dx{\theta}
= \int_0^{2\pi} \left( r^2 - \frac{r^4}{8} \right)\bigg|_{r=0}^{2} \dx{\theta}
= 2 \int_0^{2\pi} \dx{\theta} = 4 \pi \ .
\end{align*}
\label{Mass3DEgg1}
\end{egg}

\PDFfig{16_var_mult_int/PolarCoords}{Coordonnées polaires}
{Représentation en coordonnées sphériques du point
$\VEC{p} = (r\cos(\theta)\sin(\phi), r\sin(\theta)\sin(\phi), r\cos(\phi))$
pour $0 \leq \phi \leq \pi$, $0 \leq \theta < 2\pi$ et $r \geq 0$.}
{PolarCoords}

\begin{egg}
Comme à l'exemple~\ref{Mass3DEgg1}, calculons la masse de la
demi-sphère supérieure $S$ de rayon $2$ centrée à l'origine.
Cependant, cette fois ci, la densité est donnée par
$\rho(x,y,z) = \sqrt{x^2+y^2+z^2}$.

En coordonnées cartésienne, la masse est donnée par la formule
\[
  M = \int_{-2}^2 \int_{-\sqrt{4-x^2}}^{\sqrt{4-x^2}}
  \int_0^{\sqrt{4-x^2-y^2}} x^2 + y^2 + z^2 \dx{z} \dx{y} \dx{x} \ .
\]
Nous pourrions utiliser les coordonnées cylindriques comme nous avons
fait à l'exemple précédent mais la forme de la formule pour la densité
suggère un autre changement de coordonnées.   Nous allons
utiliser les {\bfseries coordonnées sphériques}\index{Coordonnées
sphériques} (figure~\ref{PolarCoords}).
La demi-sphère $S$ est l'ensemble des points
\[
  (x,y,z) = (r \cos(\theta)\sin(\phi), r\sin(\theta)\sin(\phi),r \cos(\phi))
\]
pour
$0 \leq \theta < 2\pi$, $0 \leq \phi \leq \pi/2$ et $0 \leq r \leq 2$.

Puisque $x^2 + y^2 + z^2 = r^2$ et
\begin{align*}
\left| \frac{\partial(x,y,z)}{\partial(r,\theta,\phi)} \right| &= 
\left| \det \begin{pmatrix}
\displaystyle \pdydx{x}{r} & \displaystyle \pdydx{x}{\theta}
& \displaystyle \pdydx{x}{\phi} \\[0.8em]
\displaystyle \pdydx{y}{r} & \displaystyle \pdydx{y}{\theta}
& \displaystyle \pdydx{y}{\phi} \\[0.8em]
\displaystyle \pdydx{z}{r} & \displaystyle \pdydx{z}{\theta}
& \displaystyle \pdydx{z}{\phi}
\end{pmatrix} \right| \\
&= \left| \det \begin{pmatrix}
\cos(\theta)\sin(\phi) & -r \sin(\theta)\sin(\phi) &
  r\cos(\theta)\cos(\phi) \\
\sin(\theta)\sin(\phi) & r \cos(\theta)\sin(\phi) &
 r\sin(\theta)\cos(\phi) \\
\cos(\phi) & 0 & -r\sin(\phi)
\end{pmatrix} \right|
= r^2 \sin(\phi) \ ,
\end{align*}
l'intégrale pour calculer la masse devient
\begin{align*}
M &= \int_0^{2\pi} \int_0^2 \int_0^{\pi/2}  r^4 \sin(\phi) \dx{\phi}
\dx{r} \dx{\theta}
= \int_0^{2\pi} \int_0^2
\left(-\cos(\phi) \bigg|_{\phi=0}^{\pi/2}\right) r^4 \dx{r} \dx{\theta} \\
& = \int_0^{2\pi} \int_0^2 r^4 \dx{r} \dx{\theta}
= \int_0^{2\pi} \frac{r^5}{5}\bigg|_{r=0}^{2} \dx{\theta}
= \frac{32}{5} \int_0^{2\pi} \dx{\theta} = \frac{64\pi}{5} \ .
\end{align*}
\end{egg}

}  % End of theory

\section{Exercices}

\subsection{Théorème de Fubini}

\begin{question}
Évaluez les intégrales suivantes.

\subQ{a} $\displaystyle \iint_R \sin(2x+y) \dx{A}$ où
$R = \{ (x,y) : 0 \leq x \leq \pi/4 \ , \ 0 \leq y \leq \pi/2 \}$.

\subQ{b} $\displaystyle \iint_R \frac{2+y}{3+x} \dx{A}$ où
$R = \{ (x,y) : -1 \leq x \leq 1 \ , \ -1 \leq y \leq 2 \}$.
\label{16Q1}
\end{question}

\begin{question}
Évaluez $\displaystyle \iint_R (2x+3y)^2 \dx{A}$ où $R$ est la région
bornée qui est délimitée par $y=x+1$, $y=-x+1$ et $y=0$.
\label{16Q2}
\end{question}

\begin{question}
Évaluez chacune des intégrales ci-dessous en changeant l'ordre
d'intégration.

\begin{center}
\begin{tabular}{*{1}{l@{\hspace{0.5em}}l@{\hspace{6em}}}l@{\hspace{0.5em}}l}
\subQ{a} & $\displaystyle \int_0^1 \int_{y^{1/3}}^1 \sqrt{x^4+1} \dx{x}\dx{y}$
& \subQ{b} &
$\displaystyle \int_0^{\pi^{1/4}} \int_{y^2}^{\pi^{1/2}} y\sin(x^2)
\dx{x}\dx{y}$
\end{tabular}
\end{center}
\label{16Q3}
\end{question}

\begin{question}
Pour chacune des intégrales ci-dessous, dessinez le domaine
d'intégration $D$ et donnez une intégrale équivalente en changeant
l'ordre d'intégration.

\subQ{a} $\displaystyle \int_0^4 \int_{-\sqrt{x}}^{\sqrt{x}} f(x,y) \dx{y}\dx{x}
+ \int_4^9\int_{-\sqrt{x}}^{6-x} f(x,y) \dx{y} \dx{x}$ \\
\subQ{b} $\displaystyle \int_{-6}^{-3} \int_{-\sqrt{6-x}}^{\sqrt{6-x}}
f(x,y) \dx{y}\dx{x} + \int_{-3}^2 \int_{-\sqrt{6-x}}^{-x} f(x,y) \dx{y}\dx{x}$
\label{16Q4}
\end{question}

\begin{question}
Nous pourrions vérifier que le {\bfseries moment par rapport à l'axe des
$x$}\index{Moment par rapport à l'axe des $x$} d'une surface plane
$D$ de densité $\rho(x,y)$ est
$\displaystyle M_x = \iint_D y \rho(x,y) \dx{A}$.  De même,
le {\bfseries moment par rapport à l'axe des $y$}\index{Moment par
rapport à l'axe des $y$} d'une surface plane $D$ de densité
$\rho(x,y)$ est
$\displaystyle M_y = \iint_D x \rho(x,y) \dx{A}$.   Le
{\bfseries centre de masse}\index{Centre de masse} est le point
$\displaystyle \left(\frac{M_y}{M}, \frac{M_x}{M}\right)$ où $M$ est
la masse de la surface plane.

Trouvez la masse, les moments par rapport à l'axe des $x$ et à l'axe
des $y$, et le centre de masse de la surface plane $D$ bornée par
l'axe des $x$, et les droites $y=x$ et $y = -3x + 4$.  La densité de
la surface est donnée par $\rho(x,y) = x$.
\label{16Q5}
\end{question}

\begin{question}
Calculez le volume du solide dont la base est la région du plan $x,y$ bornée
par la droite $y=x$ et la parabole $x=y^2-2y$, et qui est bornée au-dessus
par $z=x^2+y^2$.
\label{16Q6}
\end{question}

\begin{question}
Calculez le volume du solide $S$ sous le paraboloïde $z = x^2 +y^2$ et
au dessous de la région $D$ du plan $x,y$ bornée par l'axe des $x$, l'axe
des $y$ et la droite $x+y=1$.
\label{16Q7}
\end{question}

\begin{question}
Calculez le volume du solide $S = \{ (x,y,z) : 1 \leq x \leq 2 \ , \
-1 \leq y \leq 1 \ \text{et} \ 0 \leq z \leq x^2-y^2 \}$.  Tracez un
dessin du solide en question pour justifier le choix de votre
intégrale.
\label{16Q8}
\end{question}

\begin{question}
Dessinez le solide borné par les surfaces suivantes:
$z = 4y^2$, $z = 4$, $x=0$ et $x = 2$.
Utilisez une intégrale pour calculer le volume de ce solide.
\label{16Q9}
\end{question}

\begin{question}
Calculez le volume du solide borné par le cylindre $x^2+z^2 = 9$ et les plans
$x=0$, $y=0$, $z=0$ et $x+2y = 2$.  Nous supposons que $x,y,z \geq 0$.
Tracez un dessin du solide en question pour justifier le choix de
votre intégrale.
\label{16Q10}
\end{question}

\begin{question}
Calculez le volume du solide $S$ formé de l'intersection des cylindres
$x^2+y^2=4$ et $x^2+z^2=4$.
Tracez un dessin du solide en question pour justifier le choix de
votre intégrale.
\label{16Q11}
\end{question}

\begin{question}
Calculez le volume du tétraèdre délimité par les plans $x=0$, $y=0$, $z=0$
et $\displaystyle 2x+y+ \frac{z}{3}=1$.
\label{16Q12}
\end{question}

\begin{question}
Calculez le volume du tétraèdre de sommets au points $(0,0,0)$,
$(0,0,1)$, $(0,2,0)$ et $(2,2,0)$.
\label{16Q13}
\end{question}

\begin{question}
Évaluez l'intégrale $\displaystyle \iint_S xy \dx{V}$ où $S$ est le tétraèdre
$S$ délimité par les plans $y=0$, $z=0$, $y=x$ et
$\displaystyle 3x+ 3y+ 2z=6$.
\label{16Q14}
\end{question}

\begin{question}
Évaluez l'intégrale $\displaystyle \iint_S xz \dx{V}$ où $S$ est le tétraèdre
$S$ dont les sommets sont $(0,0,0)$, $(0,1,0)$, $(1,1,0)$ et $(0,1,1)$.
\label{16Q15}
\end{question}

\begin{question}
Récrivez l'intégrale
$\displaystyle \int_0^2 \int_0^{y^3} \int_0^{y^2} f(x,y,z) \dx{z}\dx{x}\dx{y}$
en changeant l'ordre d'intégration: (1) dans l'ordre $\dx{z}\dx{y}\dx{x}$,
(2) dans l'ordre $\dx{x}\dx{y}\dx{z}$ et (3)  dans l'ordre
$\dx{y}\dx{z}\dx{x}$. 
\label{16Q16}
\end{question}

\begin{question}
Récrivez l'intégrale
$\displaystyle \int_{0}^{4} \int_{\sqrt{x}}^{2} \int_{0}^{2-y}
f(x,y,z) \dx{z}\dx{y}\dx{x}$ en changeant l'ordre d'intégration: (1)
dans l'ordre $\dx{x}\dx{z}\dx{y}$ et (2) dans l'ordre
$\dx{y}\dx{x}\dx{z}$.
\label{16Q17}
\end{question}

\subsection{Substitution}

\begin{question}
Calculez le volume du solide défini par $x^2 + y^2 \leq 1$ et
$z^2 \leq x^2 + y^2$.
\label{16Q18}
\end{question}

\begin{question}
Soit l'intégrale $\displaystyle \int_0^2 \int_{-\sqrt{x^2+y^2}}^0
\frac{xy}{\sqrt{x^2+y^2}} \dx{y} \dx{x}$.

\subQ{a} Dessinez le domaine d'intégration $D$.\\
\subQ{b} Récrivez l'intégrale en coordonnées polaires.\\
\subQ{c} Évaluez l'intégrale.
\label{16Q19}
\end{question}

\begin{question}
Soit $R$ la région à l'intérieure de la sphère $x^2 + y^2 + z^2 = 9$
et au dessus de la surface définit par $z = \sqrt{x^2+y^2}$.   Donnez
l'intégrale triple avec les bornes d'intégration qui va donner le
volume $V$ de cette région.

\subQ{a} Utilisez les coordonnées cartésiennes avec l'ordre
d'intégration $\dx{z}\dx{y}\dx{x}$. \\
\subQ{b} Utilisez les coordonnées cylindriques avec l'ordre
d'intégration $\dx{z}\dx{r}\dx{\theta}$. \\
\subQ{c} Utilisez les coordonnées sphériques avec l'ordre
d'intégration $\dx{r}\dx{\theta}\dx{\phi}$.
\label{16Q20}
\end{question}

\begin{question}
Dessinez la région déterminée par les bornes d'intégration de
l'intégrale suivante et utilisez les coordonnées cylindriques pour
évaluer cette intégrale.
\begin{center}
\begin{tabular}{*{1}{l@{\hspace{0.5em}}l@{\hspace{3em}}}l@{\hspace{0.5em}}l}
\subQ{a} & $ \displaystyle
\int_{-1}^1 \int_0^{\sqrt{1-x^2}}\int_{x^2+y^2}^1 z \dx{z}\dx{y}\dx{x}$ &
\subQ{b} & $\displaystyle
\int_0^{\sqrt{2}} \int_{-\sqrt{2-y^2}}^{\sqrt{2-y^2}}
\int_{1+x^2+y^2}^{5-x^2-y^2}z \dx{z}\dx{x}\dx{y}$
\end{tabular}
\end{center}
\label{16Q21}
\end{question}

\begin{question}
Dessinez la région déterminée par les bornes d'intégration de
l'intégrale suivante et utilisez les coordonnées sphériques pour
évaluer cette intégrale.
\begin{center}
\begin{tabular}{*{1}{l@{\hspace{0.5em}}l@{\hspace{3em}}}l@{\hspace{0.5em}}l}
\subQ{a} & $ \displaystyle
\int_{-1}^1 \int_0^{\sqrt{1-x^2}}\int_{\sqrt{x^2+y^2}}^1 z \dx{z}\dx{y}\dx{x}$ &
\subQ{b} & $\displaystyle \int_0^1 \int_x^{\sqrt{2-x^2}}
\int_{\sqrt{x^2+y^2}}^{\sqrt{4-x^-y^2}} \dx{z}\dx{y}\dx{x}$
\end{tabular}
\end{center}
\label{16Q22}
\end{question}

\begin{question}
Soit l'intégrale
\[
\int_0^{1/\sqrt{2}} \int_y^{\sqrt{1-y^2}}
\int_0^{\sqrt{1-x^2-y^2}} (x^2+y^2+z^2) \dx{z}\dx{x}\dx{y} \ .
\]
\subQ{a} Tracez le dessin de la région $E$ déterminée par les bornes
d'intégration de l'intégrale.\\
\subQ{b} Utilisez la substitution appropriée pour évaluer cette
intégrale.
\label{16Q23}
\end{question}

\begin{question}
Déterminez le volume du solide bornée par la surface
$z = 25 - x^2-y^2$ et le plan $z=16$. 
\label{16Q24}
\end{question}

\begin{question}
Calculez le volume de la région au dessus de la paraboloïde d'équation
$z = x^2 + y^2$ et au dessous du cône d'équation $z = \sqrt{x^2+y^2}$.
\label{16Q25}
\end{question}

\begin{question}
Calculez le volume du solide borné par les surfaces $z^2 = x^2 + y^2$
et $x^2+y^2 = 4$.
\label{16Q26}
\end{question}

\begin{question}
Calculez le volume du solide borné par le paraboloïde
$z = 4x^2 + 3y^2$ et le cylindre parabolique $y^2 + z = 4$.
\label{16Q27}
\end{question}

\begin{question}
Nous cherchons la masse du solide délimité par la partie supérieure ($z>0$)
de la sphère de rayon $16$ centrée à l'origine et le plan $z=2$.
La densité est donnée par
$\displaystyle \rho(x,y,z) = \frac{1}{\sqrt{x^2+y^2+z^2}}$ g/cm$^3$.

\subQ{a} Utilisez les coordonnées cylindrique pour trouver cette
masse.\\
\subQ{b} Utilisez les coordonnées sphériques pour trouver cette
masse.
\label{16Q28}
\end{question}

\begin{question}
Quelle est la masse du cornet de crème glacée bornée par la sphère
$x^2+y^2+z^2 =25$ et le cône $\displaystyle \frac{16}{9} z^2 = x^2 + y^2$ pour
$z\geq 0$ si la densité est donnée par $\rho(x,y,z) = x^2+y^2$ g/cm$^3$.
\label{16Q29}
\end{question}

\begin{question}
Considérez une sphère de rayon $R_1$ centrée à l'origine qui est
percée par un cylindre de rayon $R_2 < R_1$ dont l'axe est un diamètre
de la sphère.  Si la densité de la sphère est donnée par
$\rho(x,y,z) = x^2+y^2+z^2$, quelle est la masse de
la sphère percée par le cylindre?
\label{16Q30}
\end{question}


%%% Local Variables: 
%%% mode: latex
%%% TeX-master: "notes"
%%% End: 

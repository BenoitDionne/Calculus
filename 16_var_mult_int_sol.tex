\section{Intégrales multiples}

\subsection{Théorème de Fubini}

\compileSOL{\SOLUb}{\ref{16Q1}}{
\subQ{a} Grâce au Théorème de Fubini, nous avons
\begin{align*}
\iint_R \sin(2x+y) \dx{A} &= \int_0^{\pi/4} \int_0^{\pi/2} \sin(2x+y)
\dx{y}\dx{x}
= \int_0^{\pi/4} \left( -\cos(2x + y) \right)\bigg|_{y=0}^{\pi/2} \dx{x} \\
&= \int_0^{\pi/4} \left( -\cos(2x + \pi/2) + \cos(2x) \right) \dx{x} \\
&=  \frac{-1}{2}\sin(2x + \pi/2) \bigg|_0^{\pi/4}
    + \frac{1}{2} \sin(2x) \bigg|_0^{\pi/4} \\
&=  \frac{-1}{2}\sin(\pi) + \frac{1}{2} \sin(\pi/2)
  + \frac{1}{2} \sin(\pi/2) - \frac{1}{2} \sin(0) = 1 \ .
\end{align*}

\subQ{b} Grâce au Théorème de Fubini, nous avons
\begin{align*}
\iint_R \frac{2+y}{3+x} \dx{A} &= \int_{-1}^1 \int_{-1}^2 \frac{2+y}{3+x}
                             \dx{y}\dx{x}
= \int_{-1}^1 \left( \frac{2y + y^2/2}{3+x} \right)\bigg|_{y=-1}^2 \dx{x} \\
&= \int_{-1}^1 \left( \frac{6}{3+x} + \frac{3}{2(3+x)} \right) \dx{x} 
= \int_{-1}^1 \frac{15}{2(3+x)} \dx{x} \\
&=  \frac{15}{2} \ln(3+x) \bigg|_{-1}^1
=  \frac{15}{2} \left(\ln(4) - \ln(2) \right)
= \frac{15}{2} \ln(2) \ .
\end{align*}
}

\compileSOL{\SOLUb}{\ref{16Q2}}{
Le dessin de la région $R$ est donné ci-dessous.
\PDFgraph{16_var_mult_int/surf1}
\begin{align*}
\iint_R (2x+3y)^2 \dx{A} &= \int_0^1 \int_{y-1}^{1-y} (2x+3y)^2 \dx{x}\dx{y}
= \frac{1}{6} \int_0^1 \left(2x+3y\right)^3\bigg|_{x=y-1}^{1-y} \dx{y} \\
& = \frac{1}{6} \int_0^1 \left( (2+y)^3 - (5y-2)^3 \right) \dx{y}
= \frac{1}{6} \left( \frac{1}{4} (2+y)^4 - \frac{1}{20} (5y-2)^4
\right)\bigg|_0^1 \\
&= \frac{1}{24} \left( 3^4 - 2^4 \right) - \frac{1}{120}
\left(3^4 - (-2)^4 \right) = \frac{13}{6} \ .
\end{align*}
}

\compileSOL{\SOLUb}{\ref{16Q3}}{
\subQ{a} Le dessin du domaine d'intégration est est donné ci-dessous.
\PDFgraph{16_var_mult_int/surf2}
\begin{align*}
\int_0^1 \int_{y^{1/3}}^1 \sqrt{x^4+1} \dx{x}\dx{y} &=
\int_0^1 \int_0^{x^3} \sqrt{x^4+1} \dx{y}\dx{x} 
= \int_0^1 \left( y \sqrt{x^4+1}\right)\bigg|_{y=0}^{x^3} \dx{x} \\
&= \int_0^1 x^3 \sqrt{x^4+1} \dx{x} \ .
\end{align*}
Si $u=x^4+1$, alors $\dx{u} = 4 x^3 \dx{x}$, $u=1$
lorsque $x=0$ et $u=2$ lorsque $x=1$.  Donc
\[
\int_0^1 \int_{y^{1/3}}^1 \sqrt{x^4+1} \dx{x}\dx{y} =
\frac{1}{4} \int_1^2 u^{1/2} \dx{u}
= \frac{1}{6} u^{3/2}\bigg|_1^2 = \frac{\sqrt{2}}{3} - \frac{1}{6} \ .
\]

\subQ{b} Le dessin du domaine d'intégration est donné ci-dessous.
\PDFgraph{16_var_mult_int/surf3}
\begin{align*}
\int_0^{\pi^{1/4}} \int_{y^2}^{\pi^{1/2}} y \sin(x^2) \dx{x}\dx{y}
&= \int_0^{\pi^{1/2}} \int_0^{\sqrt{x}} y \sin(x^2) \dx{y}\dx{x} 
= \int_0^{\pi^{1/2}} \left( \frac{y^2}{2} \sin(x^2)
\right)\bigg|_{y=0}^{\sqrt{x}} \dx{x} \\
&= \frac{1}{2} \int_0^{\pi^{1/2}} x \sin{x^2} \dx{x}
= - \frac{1}{4} \cos(x^2)\bigg|_0^{\pi^{1/2}} = \frac{1}{2} \ .    
\end{align*}
}

\compileSOL{\SOLUb}{\ref{16Q4}}{
\subQ{a} Le dessin du domaine d'intégration est donné ci-dessous.
\PDFgraph{16_var_mult_int/orderQ1}
La valeur de $x$ aux points d'intersection de la courbe $x=y^2$ avec la
droite $y=6-x$ est donnée par
\[
  x = (6-x)^2 \Rightarrow x^2 -13 x + 36 = (x-9)(x-4) = 0
\Rightarrow x=4 \quad \text{ou} \quad x = 9 \ .
\]
Nous avons donc les points d'intersections $(4,2)$ et $(9,-3)$.
\[
  \int_0^4 \int_{-\sqrt{x}}^{\sqrt{x}} f(x,y) \dx{y}\dx{x}
  + \int_4^9\int_{-\sqrt{x}}^{6-x} f(x,y) \dx{y} \dx{x}
 = \int_{-3}^2 \int_{y^2}^{6-y} f(x,y) \dx{x}\dx{y} \ .
\]
}

\compileSOL{\SOLUb}{\ref{16Q5}}{
Le dessin de la surface plane $D$ est donné ci-dessous.
\PDFgraph{16_var_mult_int/MomentEgg1}
La valeur de $x$ au point d'intersection des deux droites est donnée par
\[
  x = -3x + 4 \Rightarrow 4x = 4 \Rightarrow x = 1 \ .
\]
Nous trouvons le point d'intersection $(1,1)$.

La masse de $S$ est
\begin{align*}
M &= \iint_D p(x,y) \dx{A} = \int_0^1 \int_y^{(4-y)/3} x \dx{x}\dx{y}
= \int_0^1 \left( \frac{x^2}{2}\right)\bigg|_{x=y}^{(4-y)/3} \dx{y} \\
&= \int_0^1 \left( \frac{(4-y)^2}{18} - \frac{y^2}{2}\right) \dx{y}
= -\frac{(4-y)^3}{54}\bigg|_0^1 - \frac{y^3}{6}\bigg|_0^1 = \frac{14}{27} \ .
\end{align*}
Cette ordre d'intégration a été choisie car elle nous évite a avoir à
découper l'intégrale en deux comme c'est le cas si l'intégration est
par rapport à $y$ en premier.

Le moment par rapport à l'axe des $x$ est
\begin{align*}
M_x &= \iint_D y p(x,y) \dx{A} = \int_0^1 \int_y^{(4-y)/3} yx \dx{x}\dx{y}
= \int_0^1 \left( y \frac{x^2}{2}\right)\bigg|_{x=y}^{(4-y)/3} \dx{y} \\
&= \int_0^1 \left( \frac{y(4-y)^2}{18} - \frac{y^3}{2}\right)\dx{y}
= -\frac{4}{9} \int_0^1 ( y^3 + y^2 - 2y) \dx{y}
= -\frac{4}{9} \left( \frac{y^4}{4} + \frac{y^3}{3} - y^2\right)\bigg|_0^1
= \frac{5}{27} \ .
\end{align*}

Le moment par rapport à l'axe des $y$ est
\begin{align*}
M_y &= \iint_D x p(x,y) \dx{A} = \int_0^1 \int_y^{(4-y)/3} x^2 \dx{x}\dx{y}
= \int_0^1 \left( \frac{x^3}{3}\right)\bigg|_{x=y}^{(4-y)/3} \dx{y} \\
&= \int_0^1 \left( \frac{(4-y)^3}{81} - \frac{y^3}{3} \right) \dx{y}
= -\frac{(4-y)^4}{324}\bigg|_0^1 - \frac{y^4}{12}\bigg|_0^1
= \frac{37}{81} \ .
\end{align*}

Ainsi, le centre de masse est
\[
  (\overline{x}, \overline{y}) = \left(\frac{M_y}{M} , \frac{M_x}{M} \right)
  = \left(\frac{37}{42}, \frac{5}{14} \right) \ .
\]
}

\compileSOL{\SOLUb}{\ref{16Q6}}{
Le dessin de la base du solide est donné ci-dessous.
\PDFgraph{16_var_mult_int/volume11}
La valeur de $y$ aux points d'intersection de la droite $y=x$ et de la
parabole $x=y^2-2y$ est donnée par
\[
y= y^2-2y \Rightarrow 0=y^2-3y=y(y-3) \Rightarrow y = 0 \text{ ou }
y = 3 \ .
\]
Nous trouvons donc les points d'intersection $(0,0)$ et $(3,3)$.  Nous
déduisons à partir du dessin ci-dessus qu'il est préférable d'intégrer par
rapport à $x$ en premier puis par rapport à $y$.

Le volume du solide est
\begin{align*}
V &= \int_0^3 \int_{y^2-2y}^y (x^2+y^2) \dx{x}\dx{y}
= \int_0^3 \left(\frac{x^3}{3}+y^2x\right)\bigg|_{x=y^2-2y}^y \dx{y} \\
&= \int_0^3 \left( -\frac{y^6}{3} + 2y^5 -5 y^4 + 6 y^3\right)
\dx{y}
= \left( -\frac{y^7}{21} + \frac{y^6}{3} -y^5 + \frac{3 y^4}{2}\right)
\bigg|_0^3 = \frac{3^5}{14} \ .
\end{align*}
}

\compileSOL{\SOLUa}{\ref{16Q7}}{
Le dessin du solide $D$ est donné ci-dessous.
\PDFgraph{16_var_mult_int/volume19}
Le volume du solide est
\begin{align*}  
V & =\iint_D (x^2 + y^2) \dx{A} = \int_0^1 \int_0^{1-x} (x^2 + y^2) \dx{y}\dx{x}
  = \int_0^1 \left( x^2 y + \frac{y^3}{3}\right)\bigg|_{y=0}^{1-x} \dx{x} \\
& = \int_0^1 \left( x^2 (1-x) + \frac{(1-x)^3}{3}\right) \dx{x}
 = \left( \frac{x^3}{3} - \frac{x^4}{4} -
   \frac{(1-x)^4}{12}\right)\bigg|_0^1 = \frac{1}{6} \ .
\end{align*}
}

\compileSOL{\SOLUb}{\ref{16Q8}}{
Le dessin du solide $S$ est donné ci-dessous.
\PDFgraph{16_var_mult_int/volume17}

Soit $R = \{ (x,y) : 1 \leq x \leq 2 \ , \ -1 \leq y \leq 1\}$.
Puisque $x^2 - y^2 \geq 0$ pour $(x,y)\in R$, nous pouvons calculer le
volume $V$ de $S$ avec la formule suivante. 
\begin{align*}
V & =\iint_R \left( x^2 - y^2 \right) \dx{A} = \int_1^2 \int_{-1}^1 (x^2 - y^2)
\dx{y}\dx{x}
= \int_1^2 \left( x^2 y - \frac{y^3}{3} \right)\bigg|_{y=-1}^1 \dx{x} \\
&= \int_1^2 \left( \left(x^2 - \frac{1}{3}\right) -\left( -x^2
+ \frac{1}{3}\right)\right) \dx{x} 
= \int_1^2 \left( 2x^2 - \frac{2}{3} \right) \dx{x}
=  \frac{2}{3} \left( x^3- x \right)\bigg|_1^2 = 4 \ .
\end{align*}
}

\compileSOL{\SOLUb}{\ref{16Q10}}{
Le dessin du solide est donné ci-dessous.
\PDFgraph{16_var_mult_int/volume12}
Le volume de ce solide est
\begin{align*}
V &= \int_0^2 \int_0^{-x/2 + 1} \sqrt{9-x^2} \dx{y}\dx{x}
= \int_0^2 \left( y \sqrt{9-x^2}\right)\bigg|_{y=0}^{-x/2+1} \dx{x} \\
&= \int_0^2 \left( -\frac{x}{2} +1 \right)\sqrt{9-x^2} \dx{x}
= -\frac{1}{2} \int_0^2 x \sqrt{9-x^2} \dx{x}  + \int_0^2 \sqrt{9-x^2} \dx{x}
\ .
\end{align*}
Pour calculer la première intégrale, nous utilisons la substitution
$u=9-x^2$.  Donc $\dx{u} = -2 x \dx{x}$, $u=9$ lorsque $x=0$ et $u=5$
lorsque $x=2$.  Pour calculer la deuxième intégrale, nous utilisons la
substitution trigonométrique $x=3\cos(\theta)$.  Donc
$\dx{x} = -3\sin(\theta) \dx{\theta}$,
$\theta =\pi/2$ lorsque $x=0$ et $\theta = \arccos(2/3)$ lorsque
$x=2$.  Ainsi,
\begin{align*}
V &= \frac{1}{4} \int_9^5 u^{1/2} \dx{u} 
-9 \int_{\pi/2}^{\arccos(2/3)} \sin^2(\theta) \dx{\theta}
= -\frac{1}{4} \int_5^9 u^{1/2} \dx{u} +\frac{9}{2}
\int_{\arccos(2/3)}^{\pi/2} \left( 1 - \cos(2\theta)\right)\dx{\theta} \\
&= -\frac{1}{6} u^{3/2}\bigg|_{u=5}^9 +\frac{9}{2}
\left( \theta - \frac{1}{2}\sin(2\theta)
\right)\bigg|_{\theta=\arccos(2/3)}^{\pi/2} \\
&= -\frac{1}{6} (27- 5^{3/2}) + \frac{9}{2}
\left( \frac{\pi}{2} - \arccos(2/3) + \frac{1}{2}\sin(2\arccos(2/3)) \right) \\
&= -\frac{1}{6} (27- 5^{3/2}) + \frac{9}{2}
\left( \frac{\pi}{2} - \arccos(2/3) + \frac{2\sqrt{5}}{9} \right)
= -\frac{9}{2} + \frac{19\sqrt{5}}{18} + \frac{9\pi}{4} -
\frac{9}{2}\arccos(2/3) \ . 
\end{align*}
où nous avons utilisé la relation
$\sin(2\theta) = 2\cos(\theta)\sin(\theta)$ pour 
évaluer $\sin(2\arccos(2/3))$.
}

\compileSOL{\SOLUa}{\ref{16Q11}}{
Le dessin de la partie du solide $S$ qui se trouve dans la région
définie par $x,y,z \geq 0$ est présenté ci-dessous.
\PDFgraph{16_var_mult_int/volume13}
Le solide $S$ est formé de huit pièces de cette forme.
Le volume du solide est
\begin{align*}
V &= 8 \int_0^2 \int_0^{\sqrt{4-x^2}} \int_0^{\sqrt{4-x^2}}
\dx{z}\dx{y}\dx{x}
= 8 \int_0^2 \int_0^{\sqrt{4-x^2}} \sqrt{4-x^2} \dx{y}\dx{x} \\
&= 8 \int_0^2 ( 4-x^2 ) \dx{x} = 8 \left( 4x - \frac{x^3}{3}
\right)\bigg|_{x=0}^2
= 8 \left( 8 - \frac{8}{3} \right) = \frac{128}{3} \ .
\end{align*}
}

\compileSOL{\SOLUb}{\ref{16Q12}}{
La figure ci-dessous contient le dessin du tétraèdre.
\PDFgraph{16_var_mult_int/volume8}
L'intersection du plan $\displaystyle 2x+y+ \frac{z}{3}=1$ avec le plan $z=0$
est la droite $\displaystyle 2x+y =1$.

Le volume du tétraèdre est
\begin{align*}
V &= \int_0^{1/2} \int_0^{1-2x} \int_0^{3(1-2x-y)} \dx{z}\dx{y}\dx{x}
= \int_0^{1/2} \int_0^{1-2x} z\bigg|_{z=0}^{3(1-2x-y)} \dx{y}\dx{x} \\
&= 3\int_0^{1/2} \int_0^{1-2x} (1-2x-y) \dx{y}\dx{x}
= 3\int_0^{1/2} \left(y-2xy-\frac{1}{2}y^2\right)\bigg|_{y=0}^{1-2x} \dx{x} \\
& = 3\int_0^{1/2} \left((1-2x)-2x(1-2x) -\frac{1}{2}(1-2x)^2\right)\dx{x} \\
& = 3\int_0^{1/2} \left( (1-2x)^2 -\frac{1}{2}(1-2x)^2\right)\dx{x}
= \frac{3}{2} \int_0^{1/2} (1-2x)^2 \dx{x}
= - \frac{1}{4} (1-2x)^3\bigg|_0^{1/2} = \frac{1}{4} \ .
\end{align*}
}

\compileSOL{\SOLUa}{\ref{16Q13}}{
La figure ci-dessous contient le dessin du tétraèdre $S$.
\PDFgraph{16_var_mult_int/volume21}
Le volume $V$ to tétraède est
\begin{align*}
V & = \int_0^2 \int_x^2 \int_0^{1-y/2} \dx{z} \dx{y}\dx{x}
= \int_0^2 \int_x^2 \left(1 - \frac{y}{2}\right) \dx{y}\dx{x}
= \int_0^2 \left(y - \frac{y^2}{4}\right)\bigg|_{y=x}^2  \dx{x} \\
& = \int_0^2 \left(1 - x + \frac{x^2}{4}\right) \dx{x}
= \left(x - \frac{x^2}{2} + \frac{x^3}{12}\right)\bigg|_0^2 
= \frac{2}{3} \ .
\end{align*}
}

\compileSOL{\SOLUb}{\ref{16Q14}}{
La figure ci-dessous contient le dessin du tétraèdre.
\PDFgraph{16_var_mult_int/volume14}
L'intersection du plan $\displaystyle 3x+ 3y+ 2z=6$ avec le plan $z=0$
est la droite $\displaystyle y=-x+2$.
Ainsi,
\begin{align*}
\int_S xy \dx{V} &
= \int_0^1 \int_y^{2-y} \int_0^{3-3x/2-3y/2} xy\dx{z}\dx{x}\dx{y}
= \int_0^1 \int_y^{2-y} xyz\bigg|_{z=0}^{3-3x/2-3y/2} \dx{x}\dx{y} \\
&= \int_0^1 \int_y^{2-y} \left(3xy- \frac{3x^2y}{2}-\frac{3xy^2}{2}\right)
\dx{x}\dx{y}
= \int_0^1 \left(\frac{3x^2y}{2}- \frac{x^3y}{2}-\frac{3x^2y^2}{4}\right)
\bigg|_{x=y}^{2-y}\dx{y} \\
&= \int_0^1 \left(\frac{3(2-y)^2y}{2}- \frac{(2-y)^3y}{2}-
\frac{3(2-y)^2y^2}{4}\right)
- \left(\frac{3y^3}{2}- \frac{y^4}{2}-\frac{3y^4}{4}\right) \dx{y} \\
&= \int_0^1 \left( y^4 - 3 y^2 + 2 y\right)
\dx{y}
= \left( \frac{y^5}{5} - y^3 + y^2 \right)\bigg|_0^1 = \frac{1}{5} \ .
\end{align*}
}

\compileSOL{\SOLUa}{\ref{16Q15}}{
La figure ci-dessous contient le dessin du tétraèdre $S$.
\PDFgraph{16_var_mult_int/volume15}
L'équation du plan qui contient les points $\VEC{0}$, $(1,1,0)$ et $(0,1,1)$
est donnée par $( (x,y,z)-(0,0,0) ) \cdot \VEC{n} = 0$ où $\VEC{n}$
est un vecteur perpendiculaire au plan.  Nous obtenons un vecteur
perpendiculaire au plan a l'aide du produit vectoriel des vecteurs
$(1,1,0)$ et $(0,1,1)$ qui sont deux vecteurs du plan.
Puisque
\[
  \VEC{n} = (1,1,0) \times (0,1,1) = (1,-1,1) \ ,
\]
l'équation du plan qui contient les trois points est donnée par
\[
(x,y,z) \cdot (-1,1,-1) = 0 \Rightarrow z = -x + y \ .
\]
Ainsi,
\begin{align*}
\int_S xz \dx{V} &
= \int_0^1 \int_x^1 \int_0^{-x+y} xz\dx{z}\dx{y}\dx{x}
= \int_0^1 \int_x^1 \frac{xz^2}{2} \bigg|_{z=0}^{-x+y} \dx{y}\dx{x}
= \int_0^1 \int_x^1 \frac{x(-x+y)^2}{2} \dx{y}\dx{x} \\
&= \int_0^1 \frac{x(-x+y)^3}{6}\bigg|_{y=x}^1\dx{x}
= \int_0^1 \frac{x(-x+1)^3}{6} \dx{x}
= \frac{1}{6} \int_0^1 \left( -x^4 +3 x^3 - 3x^2 +x\right) \dx{x} \\
&= \frac{1}{6} \left( -\frac{x^5}{5} + \frac{3 x^4}{4} - x^3
+ \frac{x^2}{2} \right)\bigg|_0^1 = \frac{1}{120} \ .
\end{align*}
}

\compileSOL{\SOLUb}{\ref{16Q16}}{
La figure ci-dessous contient un dessin du solide $S$ décrit par les
bornes d'intégration.
\PDFgraph{16_var_mult_int/volume16}
Nous avons
\[
\int_0^2 \int_0^{y^3} \int_0^{y^2} f(x,y,z) \dx{z}\dx{x}\dx{y} =
\int_0^8 \int_{x^{1/3}}^{2} \int_0^{y^2} f(x,y,z) \dx{z}\dx{y}\dx{x} \ ,
\]
\[
\int_0^2 \int_0^{y^3} \int_0^{y^2} f(x,y,z) \dx{z}\dx{x}\dx{y} =
\int_0^4 \int_{z^{1/2}}^{2} \int_0^{y^3} f(x,y,z) \dx{x}\dx{y}\dx{z}
\]
et
\begin{align*}
\int_0^2 \int_0^{y^3} \int_0^{y^2} f(x,y,z) \dx{z}\dx{x}\dx{y} &=
\int_0^8 \int_0^{x^{2/3}} \int_{x^{1/3}}^2 f(x,y,z) \dx{y}\dx{z}\dx{x} \\
&\qquad + \int_0^8 \int_{x^{2/3}}^4 \int_{z^{1/2}}^2 f(x,y,z) \dx{y}\dx{z}\dx{x}
\ .
\end{align*}
Pour la dernière intégrale, il faut trouver la projection dans le plan
$x$,$z$ de l'intersection des surfaces $z=y^2$ et $x=y^3$ pour $x,y,z\geq 0$.
L'intersection est une courbe donnée par l'équation
$z^{1/2} = x^{1/3}$; c'est-à-dire, $z= x^{2/3}$.
Pour $0\leq x \leq 8$ et $0 \leq z \leq x^{2/3}$, nous avons que
$x^{1/3} \leq y \leq 2$.  Pour $0\leq x \leq 8$ et
$\leq x^{2/3} \leq z \leq 4$, nous avons que $z^{1/2} \leq y \leq 2$.  
}

\subsection{Substitution}

\compileSOL{\SOLUb}{\ref{16Q19}}{
\subQ{a}
\PDFgraph{16_var_mult_int/coordPolQ1}

\subQ{b}
Le domaine $D$ est formé des points
$(x,y) = (r \cos(\theta), r \sin(\theta))$ pour $0 \leq r \leq 2$ et
$-\pi/2 \leq \theta \leq 0$.  Ainsi,
\[
\left| \frac{\partial(x,y)}{\partial(r,\theta)} \right| =
\left| \det \begin{pmatrix}
\cos(\theta) & -r\sin(\theta)\\
\sin(\theta) & r\cos(\theta) \\
\end{pmatrix} \right| = r
\]
et
\begin{align*}
\int_0^2 \int_{-\sqrt{x^2+y^2}}^0
\frac{xy}{\sqrt{x^2+y^2}} \dx{y} \dx{x}
&= \int_{-\pi/2}^0\int_0^2 \frac{r^2 \cos(\theta)\sin(\theta)}{r}
\; r\dx{r}\dx{\theta} \\
&= \int_{-\pi/2}^0\int_0^2 r^2 \cos(\theta)\sin(\theta) \dx{r}\dx{\theta} \ .
\end{align*}

\subQ{c}
\begin{align*}
  \int_{-\pi/2}^0\int_0^2 r^2 \cos(\theta)\sin(\theta) \dx{r}\dx{\theta}
&= \int_{-\pi/2}^0 \left(\frac{r^3}{3}
\cos(\theta)\sin(\theta) \right)\bigg|_{r=0}^2  \dx{\theta} \\
&= \frac{8}{3} \int_{-\pi/2}^0 \cos(\theta)\sin(\theta) \dx{\theta}
= \frac{8}{3} \left( \frac{1}{2} \sin^2(\theta) \right)\bigg|_{-\pi/2}^0
= -\frac{4}{3} \ .
\end{align*}
}

\compileSOL{\SOLUb}{\ref{16Q20}}{
Le dessin de la région $R$ est donné ci-dessous.
\PDFgraph{16_var_mult_int/volume20}
Si nous substituons $z^2 = x^2 + y^2$ pour $z \geq 0$ dans
$x^2 + y^2 + z^2 = 9$, nous obtenons $2z^2 = 9$.  Donc $z = 3/\sqrt{2}$.
L'intersection de la sphère $x^2 + y^2 + z^2 = 9$ avec la surface
$z = \sqrt{x^2 + y^2}$ est donc le cercle donné par
$x^2 + y^2 = (3/\sqrt{2})^2 = 9/2$  et $z=3/\sqrt{2}$.

\subQ{a} Avec les coordonnées cartésiennes, nous avons
\[
  V = \int_{-3/\sqrt{2}}^{3/\sqrt{2}} \int_{-\sqrt{9/2 - x^2}}^{\sqrt{9/2 - x^2}}
  \int_{\sqrt{x^2+y^2}}^{\sqrt{9-x^2-y^2}} \dx{z}\dx{y}\dx{x} \ .
\]
Le calcul de cette intégrale n'est pas à conseiller.

\subQ{b} En coordonnées cylindriques, la région $R$ est l'ensemble des
points
\[
  (x,y,z) = (r\cos(\theta), r\sin(\theta), z)
\]
pour $0 \leq \theta < 2\pi$, $0 \leq r \leq 3/\sqrt{2}$ et
$r = \sqrt{x^2+y^2} \leq z \leq \sqrt{9 - x^2 - y^2} = \sqrt{9 - r^2}$.
Puisque
\[
\left| \frac{\partial(x,y,z)}{\partial(r,\theta,z)} \right| =
\left| \det \begin{pmatrix}
\cos(\theta) & -r\sin(\theta) & 0 \\
\sin(\theta) & r\cos(\theta) & 0 \\
0 & 0 & 1
\end{pmatrix} \right| = r \ ,
\]
le volume de la région $R$ est
\[
  V = \int_0^{2\pi} \int_0^{3/\sqrt{2}}\int_r^{\sqrt{9-r^2}}
r  \dx{z}\dx{r}\dx{\theta} \ .
\]

\subQ{c} En coordonnées sphériques, la région $R$ est l'ensemble des
points
\[
(x,y,z) = (r\cos(\theta)\sin(\phi), r\sin(\theta)\sin(\phi), r\cos(\phi))
\]
pour $0 \leq \theta < 2\pi$, $0 \leq \phi \leq \pi/2$ et
$0 \leq r \leq 3$.
Remarquons que l'intersection du cône avec le plan $x=0$ donne
$|y|= |z|$.  L'angle entre l'axe des $z$ et le côté du cône est donc
$\pi/4$.  Puisque
\[
\left| \frac{\partial(x,y,z)}{\partial(r,\theta,\phi)} \right| =
\left| \det \begin{pmatrix}
\cos(\theta)\sin(\phi) & -r\sin(\theta)\sin(\phi) & r\cos(\theta)\cos(\phi) \\
\sin(\theta)\sin(\phi) & r\cos(\theta)\sin(\phi) & r\sin(\theta)\cos(\phi) \\
\cos(\phi) & 0 & -r\sin(\phi)
\end{pmatrix} \right| =
r^2\sin(\phi) \ ,
\]
le volume de la région $R$ est
\[
  V = \int_0^{2\pi} \int_0^{\pi/4}\int_0^3  r^2\sin(\phi)
  \dx{r}\dx{\phi}\dx{\theta} \ .
\]
}

\compileSOL{\SOLUb}{\ref{16Q21}}{
\subQ{a} Le dessin de la région $R$ est donné ci-dessous.
\PDFgraph{16_var_mult_int/cylcoordQ1}
En coordonnées cylindriques, la région $R$ est l'ensemble des points
\[
  (x,y,z) = (r\cos(\theta), r\sin(\theta), z)
\]
pour $0 \leq \theta \leq \pi$, $0 \leq r \leq 1$ et
$x^2+y^2 = r^2 \leq z \leq 1$.
Puisque
\[
\left| \frac{\partial(x,y,z)}{\partial(r,\theta,z)} \right| =
\left| \det \begin{pmatrix}
\cos(\theta) & -r\sin(\theta) & 0 \\
\sin(\theta) & r\cos(\theta) & 0 \\
0 & 0 & 1
\end{pmatrix} \right| = r \ ,
\]
nous avons
\begin{align*}
\iint_R z \dx{V} &= \int_0^{\pi} \int_0^1\int_{r^2}^1 zr
\dx{z}\dx{r}\dx{\theta}
= \int_0^{\pi} \int_0^1 \left( \frac{z^2}{2} \, r\right)\bigg|_{z=r^2}^1
\dx{r}\dx{\theta} \\
&= \int_0^{\pi} \int_0^1 \left( \frac{r}{2} - \frac{z^5}{2}\right)
\dx{r}\dx{\theta} 
= \int_0^{\pi} \left( \frac{r^2}{4} - \frac{z^6}{12}\right)\bigg|_{r=0}^1 
\dx{\theta}
= \frac{1}{6} \int_0^{\pi} \dx{\theta} = \frac{\pi}{6} \ .
\end{align*}
}

\compileSOL{\SOLUb}{\ref{16Q22}}{
\subQ{b} Nous avons que $0 \leq x \leq 1$ et
$x \leq y \leq \sqrt{2-x^2}$.  Donc, pour $x$ fixe entre $0$ et
$1$, le poins $(x,y)$ se trouve sur le segment qui va du
point $(x,x)$ au point $(x,\sqrt{2-x^2})$; un point sur le cercle
$x^2+y^2 = 2$.  L'ensemble de ces points forme la région $D$
représentée dans la figure ci-dessous.
\PDFgraph{16_var_mult_int/SphericCoordsQ2a}
Nous avons que $\sqrt{x^2+y^2} \leq z \leq \sqrt{4-x^2-y^2}$.  Donc, pour
$(x,y)$ dans la région $D$, le point $(x,y,z)$ se trouve sur le
segment qui va du point $(x,y,\sqrt{x^2+y^2})$ au point
$(x,y,\sqrt{4-x^2-y^2})$.  le point $(x,y,\sqrt{x^2+y^2})$ fait partie
de l'hyperboloïde $z^2 = x^2 + y^2$ alors que le point\\
$(x,y,\sqrt{4-x^2 -y^2})$ fait partie de la sphère $x^2 + y^2 + z^2 = 4$.

Pour trouver l'intersection de l'hyperboloïde avec la sphère, nous
substituons $x^2 + y^2 = z^2$ dans $x^2 + y^2 + z^2 = 4$ pour obtenir
$2z^2 = 4$.  Donc $z = \sqrt{2}$ puisque nous considérons la région
$z>0$.   L'intersection est le cercle $x^2+y^2 = 2$ avec
$z = \sqrt{2}$.

Le dessin de la région $E$ définie par les bornes d'intégration est
donné ci-dessous.
\PDFgraph{16_var_mult_int/SphericCoordsQ2b}

En coordonnées sphériques, la région $E$ est l'ensemble des points
\[
(x,y,z) = (r\cos(\theta)\sin(\phi), r\sin(\theta)\sin(\phi), r \cos(\phi))
\]
pour $0 \leq r \leq 2$, $\pi/4 \leq \theta \leq \pi/2$ et
$0 \leq \phi \leq \pi/4$.  Puisque
\begin{align*}
\left| \frac{\partial(x,y,z)}{\partial(r,\theta,\phi)} \right| &= 
\left| \det \begin{pmatrix}
  \displaystyle \pdydx{x}{r} & \displaystyle \pdydx{x}{\theta}
  & \displaystyle \pdydx{x}{\phi} \\[0.8em]
  \displaystyle \pdydx{y}{r} & \displaystyle \pdydx{y}{\theta}
  & \displaystyle \pdydx{y}{\phi} \\[0.8em]
  \displaystyle \pdydx{z}{r} & \displaystyle \pdydx{z}{\theta}
  & \displaystyle \pdydx{z}{\phi}
\end{pmatrix} \right| \\
&= \left| \det \begin{pmatrix}
\cos(\theta)\sin(\phi) & -r \sin(\theta)\sin(\phi) &
 r\cos(\theta)\cos(\phi) \\
 \sin(\theta)\sin(\phi) & r \cos(\theta)\sin(\phi) &
 r\sin(\theta)\cos(\phi) \\
 \cos(\phi) & 0 & -r\sin(\phi)
\end{pmatrix} \right|
= r^2 \sin(\phi) \ ,
\end{align*}
nous obtenons
\begin{align*}
& \int_0^1 \int_x^{\sqrt{2-x^2}}
\int_{\sqrt{x^2+y^2}}^{\sqrt{4-x^2-y^2}} \dx{z}\dx{y}\dx{x}
= \int_{\pi/4}^{\pi/2} \int_0^{\pi/4} \int_0^2  r^2 \sin(\phi) \dx{r} \dx{\phi}
\dx{\theta} \\
&\qquad = \int_{\pi/4}^{\pi/2} \int_0^{\pi/4} \left(
\frac{r^3}{3}\right)\bigg|_{r=0}^2 \dx{\phi} \dx{\theta}
= \frac{8}{3} \int_{\pi/4}^{\pi/2} \int_0^{\pi/4} \sin(\phi)
\dx{\phi}\dx{\theta} \\
&\qquad  = \frac{8}{3} \int_{\pi/4}^{\pi/2}
\left(-\cos(\phi)\right)\bigg|_{\phi=0}^{\pi/4} \dx{\theta}
= \frac{8}{3}\left(1 - \frac{\sqrt{2}}{2}\right)
\int_{\pi/4}^{\pi/2} \dx{\theta}
= \frac{2\pi}{3}\left(1 - \frac{\sqrt{2}}{2}\right) \ .
\end{align*}
}

\compileSOL{\SOLUa}{\ref{16Q23}}{
\subQ{a} Nous avons que $0 \leq y \leq 1/\sqrt{2}$ et
$y \leq x \leq \sqrt{1-y^2}$.  Donc, pour $y$ fixe entre $0$ et
$1/\sqrt{2}$, le point $(x,y)$ se trouve sur le segment qui va du
point $(y,y)$ au point $(\sqrt{1-y^2},y)$; un point sur le cercle
$x^2+y^2 = 1$.  L'ensemble de ces points forme la région $D$
représentée dans la figure ci-dessous.
\PDFgraph{16_var_mult_int/SphericCoordsQ1a}
Nous avons que $0 \leq z \leq \sqrt{1-x^2-y^2}$.  Donc, pour $(x,y)$ dans la
région $D$, le point $(x,y,z)$ se trouve sur le segment qui va du
point $(x,y,0)$ au point $(x,y,\sqrt{1-x^2 -y^2}$; un point sur la
sphère $x^2+y^2 + z^2= 1$.

Le dessin de la région $E$ définie par les bornes d'intégration est
donné ci-dessous.
\PDFgraph{16_var_mult_int/SphericCoordsQ1b}

Nous pourrions utiliser les coordonnées cylindriques à partir de la région
$D$ mais comme la hauteur est déterminée par la sphère de rayon $1$ et
l'integrande est $x^2 + y^2 + z^2$, le carré de la distance du point  
$(x,y,z)$ à l'origine, les coordonnées sphériques sont préférables.
La région $E$ est l'ensemble des points
\[
(x,y,z) = (r \cos(\theta)\sin(\phi), r\sin(\theta)\sin(\phi), r \cos(\phi))
\]
pour $0 \leq r \leq 1$, $0 \leq \theta \leq \pi/4$ et
$0 \leq \phi \leq \pi/2$.  Puisque
\begin{align*}
\left| \frac{\partial(x,y,z)}{\partial(r,\theta,\phi)} \right| &= 
\left| \det \begin{pmatrix}
  \displaystyle \pdydx{x}{r} & \displaystyle \pdydx{x}{\theta}
  & \displaystyle \pdydx{x}{\phi} \\[0.8em]
  \displaystyle \pdydx{y}{r} & \displaystyle \pdydx{y}{\theta}
  & \displaystyle \pdydx{y}{\phi} \\[0.8em]
  \displaystyle \pdydx{z}{r} & \displaystyle \pdydx{z}{\theta}
  & \displaystyle \pdydx{z}{\phi}
\end{pmatrix} \right| \\
&= \left| \det \begin{pmatrix}
\cos(\theta)\sin(\phi) & -r \sin(\theta)\sin(\phi) &
 r\cos(\theta)\cos(\phi) \\
 \sin(\theta)\sin(\phi) & r \cos(\theta)\sin(\phi) &
 r\sin(\theta)\cos(\phi) \\
 \cos(\phi) & 0 & -r\sin(\phi)
\end{pmatrix} \right|
= r^2 \sin(\phi)
\end{align*}
et $x^2 + y^2 + z^2 = r^2$, nous obtenons
\begin{align*}
& \int_0^{1/\sqrt{2}} \int_y^{\sqrt{1-y^2}}
\int_0^{\sqrt{1-x^2-y^2}} (x^2+y^2+z^2) \dx{z}\dx{x}\dx{y} 
= \int_0^{\pi/4} \int_0^{\pi/2} \int_0^1  r^4 \sin(\phi) \dx{r} \dx{\phi}
\dx{\theta} \\
&\qquad = \int_0^{\pi/4} \int_0^{\pi/2} \left(
\frac{r^5}{5} \sin(\phi) \right)\bigg|_{r=0}^1 \dx{\phi} \dx{\theta}
= \frac{1}{5} \int_0^{\pi/4} \int_0^{\pi/2} \sin(\phi) \dx{\phi} \dx{\theta} \\
&\qquad = \frac{1}{5} \int_0^{\pi/4}
\left(-\cos(\phi)\right)\bigg|_{\phi=0}^{\pi/2} \dx{\theta}
= \frac{1}{5} \int_0^{\pi/4} \dx{\theta} = \frac{\pi}{20} \ .
\end{align*}
}

\compileSOL{\SOLUb}{\ref{16Q24}}{
L'intersection entre la surface $z = 25 - x^2 - y^2$ et le
plan $z=16$ est donnée par $16 = 25 -x^2-y^2$.  Nous obtenons le cercle
$x^2+y^2=9$ avec $z=16$.
\PDFgraph{16_var_mult_int/volume7}

\subI{1$^e$ méthode}
Le volume du solide est
\begin{align*}
V &= \int_{-3}^3 \int_{-\sqrt{9-x^2}}^{\sqrt{9-x^2}} \int_{16}^{25-x^2-y^2}
\dx{z}\dx{y}\dx{x}
= \int_{-3}^3 \int_{-\sqrt{9-x^2}}^{\sqrt{9-x^2}}
z\bigg|_{z=16}^{25-x^2-y^2} \dx{y}\dx{x} \\
&= \int_{-3}^3 \int_{-\sqrt{9-x^2}}^{\sqrt{9-x^2}} \left( 9-x^2-y^2 \right)
\dx{y}\dx{x}
= \int_{-3}^3 \left( 9 y - x^2y - \frac{y^3}{3} \right)
\bigg|_{y=-\sqrt{9-x^2}}^{\sqrt{9-x^2}} \dx{x} \\
&= 2\int_{-3}^3 \left( 9 - x^2 - \frac{1}{3}(9-x^2)\right)\sqrt{9-x^2} \dx{x}
= 2\int_{-3}^3 \left( 6 - \frac{2}{3}x^2 \right)\sqrt{9-x^2} \dx{x} \ .
\end{align*}
Avec la substitution trigonométrique $x=3 \cos(\theta)$ pour
$0\leq \theta\leq \pi$, nous obtenons $\dx{x} = -3 \sin(\theta) \dx{\theta}$,
$\theta= \pi$ lorsque $x=-3$ et $\theta = 0$ lorsque $x=3$.  Ainsi,
\begin{align*}
V &= -18\int_{\pi}^0 \left( 6 - 6\cos^2(\theta) \right)\sin^2(\theta)
 \dx{\theta}
= 108\int_0^{\pi} \left( 1 - \cos^2(\theta) \right)\sin^2(\theta) \dx{\theta}\\
&= 108\int_0^{\pi} \sin^2(\theta) \dx{\theta}
- 108\int_0^{\pi} \cos^2(\theta) \sin^2(\theta) \dx{\theta} \\
&= 54\int_0^{\pi} (1-\cos(2\theta)) \dx{\theta}
- 27 \int_0^{\pi} (1-\cos^2(2\theta)) \dx{\theta} \\
&= 54\pi - \frac{27}{2}\int_0^{\pi} (1- \cos(4\theta)) \dx{\theta}
= 54\pi - \frac{27\pi}{2} = \frac{81\pi}{2}\ .
\end{align*}

\subI{2$^e$ méthode}
En coordonnées cylindriques, la région $E$ bornée par la surface
$z = 25 - x^2-y^2$ et le plan $z=16$ est formée des points
\[
  (x,y,z) = (r\cos(\theta), r\sin(\theta), z)
\]
pour $0 \leq \theta < 2\pi$, $0 \leq r \leq 3$ et
$16 \leq z \leq 25 - x^2 -y^2 = 25 - r^2$.  Puisque
\[
\left| \frac{\partial(x,y,z)}{\partial(r,\theta,z)} \right| =
\left| \det \begin{pmatrix}
\cos(\theta) & -r\sin(\theta) & 0 \\
\sin(\theta) & r\cos(\theta) & 0 \\
0 & 0 & 1
\end{pmatrix} \right| = r \ ,
\]
le volume du solide est
\begin{align*}
V &= \int_0^3 \int_0^{2\pi} \int_{16}^{25-r^2} r \dx{z}\dx{\theta}\dx{r}
= \int_0^3 \int_0^{2\pi} rz\bigg|_{z=16}^{25-r^2} \dx{\theta}\dx{r}
= \int_0^3 \int_0^{2\pi} r\left( 9-r^2 \right) \dx{\theta}\dx{r} \\
& = 2\pi \int_0^3 \left( 9 r - r^3 \right) \dx{r}
= 2\pi \left( \frac{9}{2} r^2 - \frac{r^4}{4}\right)\bigg|_0^3
= \frac{81\pi}{2} \ .
\end{align*}
}

\compileSOL{\SOLUa}{\ref{16Q25}}{
Soit $E$ la région au dessus de la paraboloïde d'équation
$z = x^2 + y^2$ et au dessous du cône d'équation $z = \sqrt{x^2+y^2}$.
La figure ci-dessous présente une coupe transversale de $E$ selon le
plan $x=0$.
\PDFgraph{16_var_mult_int/volume22}

L'intersection de la paraboloïde et du cône est donnée par la solution
de $x^2+y^2 = \sqrt{x^2+y^2}$.  Les deux solutions possibles sont $x^2+y^2=0$
avec $z=0$, et $x^2 + y^2 = 1$ avec $z=1$.

En coordonnées cylindriques, $E$ est l'ensemble des points
\[ 
  (x,y,z) = (r\cos(\theta), r\sin(\theta), z)
\]
pour $0 \leq r \leq 1$, $0 \leq \theta \leq 2\pi$ et
$r^2 = x^2 + y^2 \leq z \leq \sqrt{x^2+y^2} = r$.  Puisque
\[
\left| \frac{\partial(x,y,z)}{\partial(r,\theta,z)} \right| =
\left| \det \begin{pmatrix}
\cos(\theta) & -r\sin(\theta) & 0 \\
\sin(\theta) & r\cos(\theta) & 0 \\
0 & 0 & 1
\end{pmatrix} \right| = r \ ,
\]
le volume du solide $E$ est
\begin{align*}
\iiint_E \dx{V} &=
\int_0^{2\pi} \int_0^1 \int_{r^2}^{r} r \dx{z}\dx{r}\dx{\theta}
= \int_0^{2\pi} \int_0^1  rz\bigg|_{z=r^2}^{r} \dx{r}\dx{\theta} \\
&= \int_0^{2\pi}\int_0^1  \left(r^2 - r^3 \right) \dx{r}\dx{\theta}
= \int_0^{2\pi} \left(\frac{r^3}{3} - \frac{r^4}{4} \right)\bigg|_{r=0}^1
\dx{\theta}
= \frac{1}{12} \int_0^{2\pi}  \dx{\theta} = \frac{\pi}{6} \ .
\end{align*}
}

\compileSOL{\SOLUb}{\ref{16Q27}}{
Soit $E$ le solide borné par le paraboloïde $z = 4x^2 + 3y^2$ et le
cylindre parabolique $y^2 + z = 4$.  Le dessin du solide $E$ est
donné ci-dessous.
\PDFgraph{16_var_mult_int/volume23}

Les points $(x,y,z)$ de l'intersection des deux surfaces doivent
satisfaire $z = 4 - y^2 = 4x^2 + 3y^2$; c'est-à-dire,
$x^2 +y^2 = 1$ et $z = 4 - y^2$.   Cela décrit une courbe.

En coordonnées cylindriques, la région $E$ est l'ensemble des points
\[
  (x,y,z) = (r\cos(\theta), r\sin(\theta), z)
\]
pour $0 \leq r \leq 1$, $0 \leq \theta \leq 2\pi$ et
$3r^2 + r^2\cos^2(\theta)  \leq z \leq 4 - r^2\sin^2(\theta)$ car
$4x^2 + 3y^2 = 4r^2 \cos^2(\theta) + 3 r^2 \sin^2(\theta)
= 3r^2 + r^2\cos^2(\theta)$ et
$4 - y^2 = 4 - r^2\sin^2(\theta)$.  Puisque
\[
\left| \frac{\partial(x,y,z)}{\partial(r,\theta,z)} \right| =
\left| \det \begin{pmatrix}
\cos(\theta) & -r\sin(\theta) & 0 \\
\sin(\theta) & r\cos(\theta) & 0 \\
0 & 0 & 1
\end{pmatrix} \right| = r \ ,
\]
le volume du solide est
\begin{align*}
\iiint_E \dx{V} &=
\int_0^{2\pi} \int_0^1 \int_{3r^2+r^2\cos(\theta)}^{4-r^2\sin(\theta)}
r \dx{z}\dx{r}\dx{\theta}
= \int_0^{2\pi} \int_0^1 rz\bigg|_{z=3r^2+r^2\cos(\theta)}^{4-r^2\sin(\theta)}
 \dx{r}\dx{\theta} \\
&= \int_0^{2\pi}\int_0^1  \left(4 - 3r^2 -r^2\sin^2(\theta)
- r^2\cos^2(\theta) \right)r \dx{r}\dx{\theta} \\
&= 4 \int_0^{2\pi}\int_0^1  \left(r - r^3\right) \dx{r}\dx{\theta}
= 4 \int_0^{2\pi} \left(\frac{r^2}{2} - \frac{r^4}{4} \right)\bigg|_{r=0}^1
\dx{\theta}
= \int_0^{2\pi}  \dx{\theta} = 2\pi \ .
\end{align*}
}

\compileSOL{\SOLUb}{\ref{16Q29}}{
Soit $E$ la région bornée par la sphère $x^2+y^2+z^2 =25$ et le cône
$\displaystyle \frac{16}{9} z^2 = x^2 + y^2$ pour $z\geq 0$.

L'intersection de la sphère avec le cône est donnée par
\[
25-x^2-y^2 = z^2 = \frac{9}{16}\left( x^2 + y^2 \right) \ .
\]
Nous obtenons $16=x^2+y^2$ et $z=3$.  Le dessin du cornet est donné
ci-dessous.
\PDFgraph{16_var_mult_int/volume9}

Afin de calculer la masse du cône, nous utilisons les coordonnées
sphériques.  La région $E$ est l'ensemble des points
\[
(x,y,z) = (r\cos(\theta)\sin(\phi), r\sin(\theta)\sin(\phi), r\cos(\phi))
\]
pour $0 \leq \theta < 2\pi$, $0 \leq \phi \leq \arccos(3/5)$ et
$0 \leq r \leq 4$.  Puisque
\[
\left| \frac{\partial(x,y,z)}{\partial(r,\theta,\phi)} \right| =
\left| \det \begin{pmatrix}
\cos(\theta)\sin(\phi) & -r\sin(\theta)\sin(\phi) & r\cos(\theta)\cos(\phi) \\
\sin(\theta)\sin(\phi) & r\cos(\theta)\sin(\phi) & r\sin(\theta)\cos(\phi) \\
\cos(\phi) & 0 & -r\sin(\phi)
\end{pmatrix} \right| = r^2\sin(\phi) \ ,
\]
la masse du cône est
\begin{align*}
M &= \int_0^5 \int_0^{2\pi} \int_0^{\arccos(3/5)} r^4\sin^3(\phi)
\dx{\phi}\dx{\theta}\dx{r} \\
&= \int_0^5 \int_0^{2\pi} \int_0^{\arccos(3/5)} r^4(1-\cos^2(\phi))
\sin(\phi) \dx{\phi}\dx{\theta}\dx{r} \ .
\end{align*}
Si $u=\cos(\phi)$, alors $\dx{u} = -\sin(\phi)\dx{\phi}$, $u=1$
lorsque $\phi=0$ et $u = 3/5$ lorsque $\phi=\arccos(3/5)$.  Donc
\begin{align*}
M &= -\int_0^5 \int_0^{2\pi} \int_1^{3/5} r^4(1-u^2) \dx{u} \dx{\theta}\dx{r} 
= \int_0^5 \int_0^{2\pi} \int_{3/5}^1 r^4(1-u^2) \dx{u} \dx{\theta}\dx{r} \\
& = \int_0^5 \int_0^{2\pi} r^4\left( u - \frac{u^3}{3}\right)\bigg|_{u=3/5}^1
\dx{\theta}\dx{r} 
= \frac{52}{375} \int_0^5 \int_0^{2\pi} r^4 \dx{\theta}\dx{r} \\
&= \frac{104\pi}{375} \int_0^5 r^4 \dx{r}
= \frac{104\pi}{375} \left(\frac{r^5}{5}\bigg|_{r=0}^5 \right)
= \frac{520\pi}{3} \ \text{g.}
\end{align*}
}

\compileSOL{\SOLUb}{\ref{16Q30}}{
Soit $E$ la sphère de rayon $R_1$ centrée à l'origine qui est percée
par un cylindre de rayon $R_2 < R_1$ dont l'axe est un diamètre de la
sphère.  Nous pouvons supposer que l'axe du cylindre est l'axe des $z$
car la densité est une fonction de la distance à l'origine seulement.
L'équation du cylindre est donc alors $x^2+y^2=R_2^2$.

Le dessin du solide $E$ est donnée ci-dessous.
\PDFgraph{16_var_mult_int/volume10}
Nous avons enlevé une tranche du solide pour mieux voir la partie de
la sphère qui a été enlevée.  Pour calculer la masse du solide, nous
allons calculer la masse de la sphère (avant d'être percée par le
cylindre) puis soustraire de cette masse la masse de la partie de la
sphère qui a été enlevée par le passage du cylindre.

\subQ{i} La masse de la sphère peut être facilement calculée à l'aide des
coordonnée sphériques
\[
  (x,y,z) = (r\cos(\theta)\sin(\phi), r\sin(\theta)\sin(\phi), r\cos(\phi))
\]
pour $0 \leq \theta < 2\pi$, $0 \leq \phi \leq \pi$ et $0 \leq r \leq R_1$.
Puisque
\[
\left| \frac{\partial(x,y,z)}{\partial(r,\theta,\phi)} \right| =
r^2\sin(\phi) \ ,
\]
la masse de la sphère est
\begin{align*}
M_s &= \int_0^{R_1} \int_0^{2\pi} \int_0^\pi r^4
\sin(\phi)\dx{\phi}\dx{\theta} \dx{r}
= -\int_0^{R_1} \int_0^{2\pi} r^4 \cos(\phi)\bigg|_{\phi=0}^\pi
\dx{\theta} \dx{r} \\
& = 2\int_0^{R_1} \int_0^{2\pi} r^4 \dx{\theta} \dx{r}
= 4\pi \int_0^{R_1} r^4 \dx{r}
= \frac{4\pi r^5}{5}\bigg|_0^{R_1} = \frac{4\pi}{5} R_1^5 \ .
\end{align*}

\subQ{ii} Pour calculer la masse de la partie cylindrique qui a été
retirée de la sphère, nous utilisons les coordonnées cylindriques
\[
  (x,y,z) = (r\cos(\theta), r\sin(\theta), z)
\]
pour $0 \leq \theta < 2\pi$, $0 \leq r \leq R_2$ et
\[
-\sqrt{R_1^2-r^2} = -\sqrt{R_1^2 - x^2 - y^2} \leq z \leq
\sqrt{R_1^2 - x^2 - y^2} = \sqrt{R_1^2-r^2} \ .
\]
Puisque
\[
\left| \frac{\partial(x,y,z)}{\partial(r,\theta,z)} \right| = r \ ,
\]
la masse de cette partie cylindrique est
\begin{align*}
M_c &= \int_0^{R_2} \int_0^{2\pi} \int_{-\sqrt{R_1^2-r^2}}^{\sqrt{R_1^2-r^2}}
\left( z^2 + r^2 \right) r \dx{z}\dx{\theta}\dx{r}
= \int_0^{R_2} \int_0^{2\pi} 
\left( \left( \frac{z^3}{3} + r^2 z \right)r \right)
\bigg|_{z=-\sqrt{R_1^2-r^2}}^{\sqrt{R_1^2-r^2}} \dx{\theta}\dx{r} \\
&= \frac{2}{3} \int_0^{R_2} \int_0^{2\pi} 
\left( R_1^2+2r^2 \right) \sqrt{R_1^2-r^2}\ r \dx{\theta}\dx{r}
= \frac{4\pi}{3} \int_0^{R_2} \left( R_1^2+2r^2 \right)
\sqrt{R_1^2-r^2}\ r \dx{r} \ . 
\end{align*}
Pour calculer cette dernière intégrale, posons $u=R_1^2-r^2$.  Donc
$\dx{u} = -2 r\dx{r}$, $u= R_1^2$ lorsque $r=0$ et $u=R_1^2-R_2^2$ lorsque
$u=R_2$.  Ainsi,
\begin{align*}
M_c &= -\frac{2\pi}{3} \int_{R_1^2}^{R_1^2-R_2^2}
\left( R_1^2+2 (R_1^2-u) \right) u^{1/2} \dx{u}
= \frac{2\pi}{3} \int_{R_1^2-R_2^2}^{R_1^2}
\left( 3 R_1^2 u^{1/2} -2 u^{3/2} \right) \dx{u} \\
&= \frac{2\pi}{3} \left( 2 R_1^2 u^{3/2} - \frac{4}{5} u^{5/2})
\right)\bigg|_{R_1^2-R_2^2}^{R_1^2}
= \frac{4\pi}{5} R_1^5
- \frac{4\pi}{15}\left( 3 R_1^4 - R_1^2 R_2^2 - 2 R_1^4\right)
\sqrt{R_1^2-R_2^2} \ .
\end{align*}
La masse de notre solide est
\[
M_s-M_c = \frac{4\pi}{15}\left( 3 R_1^4 - R_1^2 R_2^2 - 2 R_1^4\right) \ .
\]
}

%%% Local Variables: 
%%% mode: latex
%%% TeX-master: "notes"
%%% End: 

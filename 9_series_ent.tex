\chapter[Séries entières \eng]{Séries entières\\ \eng}\label{chapSerEnt}

\compileTHEO{

Ce chapitre est d'une nature plus théorique que les chapitres
précédents.  Les séries entières vont nous permettre de définir
plusieurs fonctions qui sont solutions
{\em d'équations différentielles}, le sujet du 
chapitre~\ref{chap_equ_diff} et qui est normalement étudié plus
profondément après les premiers cours de calcul différentiel et
intégral.  Nous présentons dans ce chapitre plusieurs techniques pour
déterminer si une série entière converge.  Finalement, ceux pour qui
la théorie les intéresse pourront apprécier la définition rigoureuse
de la fonction $e^x$ donnée dans la dernière section du chapitre.
  
\section{Convergence des séries entières} 

\begin{focus}{\dfn}
Une {\bfseries série entière au voisinage de $c \in \RR$} (ou
{\bfseries près de $c$})\index{Série entière} est une séries de la forme
\[
\sum_{n=0}^\infty a_n (x-c)^n \; .\,
\]
où $\displaystyle \left\{a_n\right\}_{n=0}^\infty$ est une suite de
nombres réels.  Les nombres $a_n$ pour $n=0$, $1$, $2$, \ldots sont
appelés les {\bfseries coefficients} \index{Série entière!coefficient}
de la série et $a_n (x-c)^n$ pour $n=0$, $1$, $2$, \ldots\ sont les
{\bfseries termes de la série}. \index{Série entière!terme}
\end{focus}

Pour chaque valeur de $x$, nous avons une série de nombres réels pour
laquelle il faut déterminer la convergence.

\begin{focus}{\dfn}
Soit
\[
\sum_{n=0}^\infty a_n (x-c)^n
\]
une série entière au voisinage de $c$.  Nous disons que la série entière
{\bfseries converge} pour une valeur de $x$ si la limite
\[
\lim_{k\rightarrow \infty} \sum_{n=0}^k a_n (x-c)^n
\]
existe pour cette valeur de $x$.  Nous dénotons cette limite par
$\displaystyle \sum_{n=0}^\infty a_n (x-c)^n$.
\index{Série entière!convergence}  La somme
$\displaystyle \sum_{n=0}^k a_n (x-c)^n$ est une {\bfseries somme partielle}
de la série. \index{Série entière!somme partielle}

le contexte détermine si $\displaystyle \sum_{n=0}^\infty a_n (x-c)^n$ fait
référence à la série entière où la limite de la série entière pour une valeur
de $x$.
\end{focus}

Selon la valeur de $x$, la série peut converger (absolument ou
conditionnellement) ou diverger.

\begin{egg}
La série entière $\displaystyle \sum_{n=0}^\infty \frac{x^n}{n!}$ converge
absolument pour tout $x$.

Les termes de cette série sont $\displaystyle a_n = \frac{x^n}{n!}$.
Ainsi,
\[
\lim_{n\rightarrow \infty} \frac{|a_{n+1}|}{|a_n|}
= \lim_{n\rightarrow \infty} \frac{|x|^{n+1}/(n+1)!}{|x|^n/n!}
= \lim_{n\rightarrow \infty} \frac{|x|}{n+1} = 0 < 1
\]
pour tout $x$.  Donc, par le test du quotient, la série converge absolument
pour tout $x$.
\label{exp_serie_egg}
\end{egg}

\begin{egg}
Montrons que la série
$\displaystyle \sum_{i=0}^\infty \frac{(-5)^n(x-2)^n}{n^2+1}$ converge
absolument pour $9/5 < x < 11/5$.

Les termes de la série sont de la forme
$\displaystyle a_n = \frac{(-5)^n(x-2)^n}{n^2+1}$.
Ainsi,
\begin{align*}
\lim_{n\rightarrow \infty} \frac{|a_{n+1}|}{|a_n|}
&= \lim_{n\rightarrow \infty} \frac{5^{n+1}|x-2|^{n+1}/((n+1)^2+1)}
{5^n|x-2|^n/(n^2+1)} \\
&= \lim_{n\rightarrow \infty} \frac{5\,|x-2|(n^2+1)}{n^2+2n+2}
= \lim_{n\rightarrow \infty} \frac{5\,|x-2|(1+1/n^2)}{1+2/n+2/n^2}
 = 5\,|x-2| < 1
\end{align*}
pour $9/5 < x < 11/5$.  Donc, par le test du quotient, la
série converge absolument pour $9/5 < x < 11/5$.
De plus, le test du quotient dit que la série diverge pour $x<9/5$ et
$x>11/5$ car nous avons alors que
\[
\lim_{n\rightarrow \infty} \frac{|a_{n+1}|}{|a_n|} > 1 \; .
\]
\label{entiere_egg}
\end{egg}

À l'exemple précédent, nous avons montrer que la série convergeait
pour $|x-2|<1/5$ et divergeait pour $|x-2|>1/5$.  Le théorème suivant
confirme que c'est typique des séries entières.

\begin{focus}{\thm}
Soit $\displaystyle \left\{a_n\right\}_{n=0}^\infty$ une suite de nombres
réels et $c$ un nombre réelle.  Soit que la séries entière
$\displaystyle \sum_{n=0}^\infty a_n (x-c)^n$ diverge pour toute valeur
de $x \neq c$ ou il existe un nombre $R>0$ ($R=\infty$ est admissible)
tel que la série entière $\displaystyle \sum_{n=0}^\infty a_n (x-c)^n$
converge absolument pour $|x-c|<R$ et diverge pour $|x-c|>R$.
\end{focus}

\begin{rmk}
Le théorème précédent ne dit rien au sujet de la convergence de la série
$\displaystyle \sum_{n=0}^\infty a_n (x-c)^n$ pour $|x-c|=R$.  Il
faut utiliser les tests de convergence qui ont été présentés aux
sections~\ref{conv_tests} et \ref{serie_alt} pour déterminer la convergence
de $\displaystyle \sum_{n=0}^\infty a_n (x-c)^n$ pour ces valeurs de $x$.
\end{rmk}

\begin{egg}[][Suite de l'exemple~\ref{entiere_egg}] \label{entiere_eggS}
Nous avons montré que la série
$\displaystyle \sum_{i=0}^\infty \frac{(-5)^n(x-2)^n}{n^2+1}$ convergeait
pour $|x-2|<1/5$ et divergeait pour $|x-2|>1/5$.  Avons-nous convergence
pour $x=9/5$ et $x=11/5$?

Si $x=9/5$, nous obtenons la série de termes positifs
\[
\sum_{i=0}^\infty \frac{1}{n^2+1} = 1 + \sum_{i=1}^\infty \frac{1}{n^2+1} \; .
\]
Cette dernière série converge car
$\displaystyle \frac{1}{n^2+1} < \frac{1}{n^2}$ et la série
$\displaystyle \sum_{i=1}^\infty \frac{1}{n^2}$ converge d'après la
proposition~\ref{Pseries}.

Si $x=11/5$, nous obtenons la série alternée
\[
\sum_{i=0}^\infty \frac{(-1)^n}{n^2+1} \; .
\]
Cette série satisfait le test des séries alternées et donc converge.  En
effet, c'est une série de la forme
$\displaystyle \sum_{i=0}^\infty (-1)^n a_n$ avec $a_n = 1/(n^2+1)$.  Nous avons
$a_n>0$ et $a_{n+1} < a_n$ pour tout $n$.  De plus
$\displaystyle \lim_{n\rightarrow \infty} a_n = 0$.
\end{egg}

\begin{focus}{\dfn}
Le {\bfseries rayon de convergence} d'une série entière
$\displaystyle \sum_{n=0}^\infty a_n (x-c)^n$ est le nombre $R$ tel que la
série converge pour $|x-c|<R$ et diverge pour $|x-c|>R$.
\index{Série entière!rayon de convergence}

L'{\bfseries intervalle de convergence} est
$\displaystyle \left\{x : \sum_{n=0}^\infty a_n (x-c)^n
\text{ converge} \right\}$.
\index{Série entière!intervalle de convergence}
\end{focus}

\begin{egg}[][Suite de l'exemple~\ref{entiere_eggS}]
Le rayon de convergence de la série
$\displaystyle \sum_{i=0}^\infty \frac{(-5)^n(x-2)^n}{n^2+1}$ est
$1/5$ et l'intervalle de convergence est
$\displaystyle \left[\frac{9}{5} , \frac{11}{5}\right]$.
\end{egg}

Le théorème suivant est une conséquence du test du quotient pour les séries
de nombres.

\begin{focus}{\thm} \label{RadiusConvComp}
Le rayon de convergence $R$ de la série entière
$\displaystyle \sum_{n=0}^\infty a_n (x-c)^n$ est donnée par
\[
\frac{1}{R} = \lim_{n\rightarrow \infty} \left|\frac{a_{n+1}}{a_n}\right|
\]
si cette limite existe.  Si la limite est $0$, alors $R=\infty$.
\end{focus}

\begin{egg}
Trouvons le rayon de convergence et l'intervalle de convergence de la série
\begin{equation}\label{exSRSone}
\sum_{n=1}^\infty \frac{2\times 4 \times \ldots \times (2n)}
{1 \times 3 \times \ldots \times (2n-1)} \, x^n \; .
\end{equation}

C'est une série de la forme $\displaystyle \sum_{n=1}^\infty a_n x^n$ où
\[
a_n = \frac{2 \times 4 \times \ldots \times (2n)}
{1 \times 3 \times \ldots \times (2n-1)} \; .
\]
Puisque
\[
\lim_{n\rightarrow \infty} \left| \frac{a_{n+1}}{a_n}\right|
= \lim_{n\rightarrow \infty} \left|
\frac{\displaystyle \frac{2 \times 4 \times \ldots \times (2n+2)}
{\displaystyle 1 \times 3 \times \ldots \times (2n+1)}}
{\displaystyle \frac{2 \times 4 \times \ldots \times (2n)}
{\displaystyle 1 \times 3 \times \ldots \times (2n-1)}} \right|
= \lim_{n\rightarrow \infty} \left| \frac{2n-2}{2n+1} \right| = 1 \ ,
\]
le rayon de convergence de (\ref{exSRSone}) est $R=1$.

Pour $x=1$, nous obtenons la série
\begin{equation}\label{exSRSoneA}
\sum_{n=1}^\infty \frac{2\times 4 \times \ldots \times (2n)}
{1 \times 3 \times \ldots \times (2n-1)} \; .
\end{equation}
C'est une série de la forme $\displaystyle \sum_{n=1}^\infty a_n$ où
\[
a_n = \frac{2\times 4 \times \ldots \times (2n)}
{1 \times 3 \times \ldots \times (2n-1)}
 = \frac{2}{1}\times \frac{4}{3} \times \ldots \times \frac{2n}{2n-1}
>1
\]
pour tout $n$.  Puisque la suite
$\displaystyle \left\{ a_n \right\}_{n=1}^\infty$ des termes de la
série ne tend pas vers $0$, la série (\ref{exSRSoneA}) ne converge
pas.

Pour $x=-1$, nous obtenons la série
\begin{equation}\label{exSRSoneB}
\sum_{n=1}^\infty (-1)^n \,\frac{2\times 4 \times \ldots \times (2n)}
{1 \times 3 \times \ldots \times (2n-1)} \; .
\end{equation}
C'est une série de la forme
$\displaystyle \sum_{n=1}^\infty a_n$ où
\[
a_n = (-1)^n \, \frac{2\times 4 \times \ldots \times (2n)}
{1 \times 3 \times \ldots \times (2n-1)}
 = (-1)^n \,\frac{2}{1}\times \frac{4}{3} \times \ldots \times
\frac{2n}{2n-1} \; .
\]
Donc $a_n > 1$ pour $n$ pair et $a_n<-1$ pour $n$ impair.  Comme pour la
série précédente, la suite
$\displaystyle \left\{ a_n \right\}_{n=1}^\infty$ des termes de la
série ne converge pas vers $0$.  La série (\ref{exSRSoneB}) ne
converge donc pas.

L'intervalle de convergence est $]-1,1[$.
\end{egg}

\section{Fonctions définies par des séries}

\begin{focus}{\thm} \label{der_int_series}
Considérons la série entière $\displaystyle \sum_{n=0}^\infty a_n (x-c)^n$
dont le rayon de convergence est $R>0$, et la fonction
$f:]c-R,c+R[\rightarrow \RR$ définie par
\[
f(x) = \sum_{n=0}^\infty a_n (x-c)^n \quad , \quad c-R<x<c+R \; .
\]
Alors $f$ est différentiable sur l'intervalle $]c-R,c+R[$ et
\begin{equation}\label{derseries}
f'(x) = \sum_{n=1}^\infty n\,a_n (x-c)^{n-1} \quad , \quad c-R<x<c+R \; .
\end{equation}
Cette série converge au moins pour $c-R<x<c+R$.

De plus, si $c-R<a<b<c+R$, l'intégrale de $f$ entre $a$
à $b$ existe et est donnée par
\begin{equation}\label{intseries}
\int_a^b f(x) \dx{x} =
\sum_{n=1}^\infty \left(\frac{a_n}{n+1} (x-c)^{n+1}\right) \bigg|_{x=a}^b \; .
\end{equation}
Cette série numérique converge.
\end{focus}

\begin{rmk}
En d'autres mots, l'équation (\ref{derseries}) dit que pour dériver une
fonction donnée par une série entière, il suffit de dériver chacun des termes
de la série car
\[
\dfdx{a_n (x-c)^n}{x} = n\, a_n (x-c)^{n-1} \; .
\]
Alors que l'équation (\ref{intseries}) dit que pour intégrer
une fonction donnée par une série entière il suffit d'intégrer chacun des
termes de la série car
\[
\int_a^b a_n (x-c)^n \dx{x} 
= \frac{a_n}{n+1} (x-c)^{n+1} \bigg|_{x=a}^b \; .
\]
\end{rmk}

Si nous remplaçons $x$ par $t$, $a$ par $c$ et $b$ par $x$ dans l'équation
(\ref{intseries}), nous obtenons
\[
\int_c^x f(t) \dx{t} =
\sum_{n=1}^\infty \frac{a_n}{n+1} (t-c)^{n+1} \bigg|_{t=c}^x
= \sum_{n=1}^\infty \frac{a_n}{n+1} (x-c)^{n+1} \;.
\]
Puisque $\displaystyle F(x) = \int_c^x f(t) \dx{t}$ est une primitive
de $f$, nous pouvons énoncer le corollaire suivant.

\begin{focus}{\cor}
Considérons la série entière $\displaystyle \sum_{n=0}^\infty a_n (x-c)^n$
dont le rayon de convergence est $R>0$, et la fonction
$f:]c-R,c+R[\rightarrow \RR$ définie par
\[
f(x) = \sum_{n=0}^\infty a_n (x-c)^n \quad , \quad c-R<x<c+R \; .
\]
Une primitive de $f$ sur l'intervalle $]c-R,c+R[$ est donnée
par
\[
\int f(x) \dx{x} = \sum_{n=1}^\infty \frac{a_n}{n+1} (x-c)^{n+1} +C
\]
où $C$ est une constante.  La série précédente converge au moins pour
$c-R<x<c+R$.
\end{focus}

\begin{egg}
La série géométrique que nous avons vu à l'exemple~\ref{geoseries} est
\begin{equation}\label{derintgeo}
\frac{1}{1-x} = \sum_{n=0}^\infty x^n  \quad , \quad |x|<1 \; .
\end{equation}
En dérivant des deux cotés, nous obtenons
\[
\frac{1}{(1-x)^2} = \sum_{n=1}^\infty nx^{n-1}  \quad , \quad |x|<1 \; .
\]
C'est une série convergente pour $|x|<1$ qui représente la fonction
$1/(1-x)^2$ pour $|x|<1$.  En intégrant des deux cotés de
(\ref{derintgeo}), nous obtenons
\[
- \ln|1-x| = \sum_{n=0}^\infty \frac{x^{n+1}}{n+1} + C
  \quad , \quad |x|<1 \; .
\]
Pour $x=0$, cette équation donne $0 = -\ln(1) = C$.  Donc
\[
\ln|1-x| = - \sum_{n=0}^\infty \frac{x^{n+1}}{n+1}   \quad , \quad |x|<1 \; .
\]
C'est une série convergente pour $|x|<1$ qui représente la
fonction $\ln|1-x|$ pour $|x|<1$.
\end{egg}

\section{Séries de Taylor et de MacLaurin}

À la section~\ref{approx_local}, nous avons vu le théorème de Taylor
qui, pour une fonction donnée, donne un polynôme qui permet d'estimer
cette fonction.  Plus 
précisément, si $f:[a,b]\rightarrow \RR$ est une fonction continue sur
l'intervalle $[a,b]$, $f$ est $(k+1)$ fois différentiables sur $]a,b[$ où
$k\geq 0$, et $x$ et $c$ sont dans l'intervalle $]a,b[$, alors il existe
$\xi = \xi(k,c,x)$ entre $x$ et $c$ tel que
\begin{equation}\label{TPfourTS1}
f(x) = p_k(x) + r_k(x)  \ ,
\end{equation}
où
\begin{equation}\label{TPfourTS2}
p_k(x) = \sum_{n=0}^k \frac{1}{n!} f^{(n)}(c)(x-c)^n
\quad \text{et} \quad
r_k(x) = \frac{1}{(k+1)!} f^{(k+1)}(\xi)\, (x-c)^{k+1} \ . 
\end{equation}

Si $f$ est infiniment différentiable, est-il vrai que
\begin{equation}\label{problTaylor}
f(x) = \sum_{n=0}^\infty \frac{1}{n!} f^{(n)}(c)(x-c)^n
\end{equation}
pour $x \in ]a,b[$?  Non, ce n'est pas toujours vrai.  L'expression en
(\ref{problTaylor}) peut être fausse de plusieurs façons.
\begin{enumerate}
\item La série en (\ref{problTaylor}) peut ne pas converger pour $x\neq c$.
\item La série en (\ref{problTaylor}) peut converger pour certaines valeurs
de $x$ mais la limite de la série peut être différente de $f(x)$ à ces
valeurs de $x$.
\end{enumerate}

\begin{egg}
Pour illustrer la deuxième cas où (\ref{problTaylor}) peut être faux,
considérons la fonction
\[
f(x) = \begin{cases}
e^{-1/x^2} & \quad \text{pour} \quad x\neq 0 \\
0 & \quad \text{pour}\quad x = 0
\end{cases}
\]
A partir de la définition de la dérivée d'une fonction, nous pouvons montrer que
$\displaystyle \dydxn{f}{x}{n}(0) = 0$ pour tout $n>0$.
Donc
\[
\sum_{n=0}^\infty \frac{1}{n!} f^{(n)}(0)x^n = 0 \quad , \quad
x \in \RR \; .
\]
C'est une série qui converge pour tout $x\in \RR$.  En fait, cette série
converge vers $0$ pour tout $x\in \RR$.  Malheureusement, pour
$x \neq 0$, cette série ne converge pas vers $f(x) \neq 0$.
\end{egg}

% \begin{egg}
% L'exemple suivant illustre le premier cas où (\ref{problTaylor}) peut
% être faux.

% À venir.
% \end{egg}

\begin{egg}
Bien souvent l'intervalle de convergence de la série de Taylor est une
fraction du domaine de la fonction.  Considérons la série
\[
\sum_{n=0}^\infty \frac{1}{n!} f^{(n)}(0)x^n
\]
pour $\displaystyle f(x) = \frac{1}{1-x}$.  Puisque
\[
f^{(n)}(x) = \frac{n!}{(1-x)^{n+1}} \quad \text{pour} \quad n=0, 1, 2,
\ldots
\]
nous avons que $\displaystyle f^{(n)}(0) = n!$ pour $n=0$, $1$, $2$,
\ldots\quad  Donc
\[
\sum_{n=0}^\infty \frac{1}{n!} f^{(n)}(0)x^n = \sum_{n=0}^\infty x^n
\; .
\]
C'est la série géométrique qui converge seulement pour $|x| <1$
(l'exemple~\ref{geoseries}).  Pour $|x|< 1$, la série converge bien
vers $f(x)$ mais la fonction $f$ est définie pour tout $x\neq 1$.
\label{eggtaylorgeom}
\end{egg}

Malgré les points négatifs que nous venons de soulever, l'expression
(\ref{problTaylor}) est très importante.

\begin{focus}{\dfn} \index{Série de Taylor}\index{Série de MacLaurin}
Soit $f;]a,b[\rightarrow \RR$ une fonction infiniment différentiable.  La
{\bfseries série de Taylor de $f$ près de $c \in ]a,b[$} est la série
\begin{equation}\label{TaylorSeries}
f(x) \sim \sum_{n=0}^\infty \frac{1}{n!} f^{(n)}(c)(x-c)^n \; .
\end{equation}
Lorsque $c=0$, cette série est aussi appelée la
{\bfseries série de MacLaurin de $f$}.

Le symbole $\sim$ indique seulement que la série est associée à la fonction
$f$.  Nous ne savons pas si la série converge ou si elle converge vers $f(x)$
pour $x\in]a,b[$ près de $c$.
\end{focus}

\begin{focus}{\mth}
Soit $f:]a,b[\rightarrow \RR$ une fonction infiniment différentiable et
$c \in ]a,b[$.  Pour déterminer si la série de Taylor de $f$ près de $c$
converge et si elle converge vers $f(x)$ pour $x\in]a,b[$ près de $c$, il
faut démontrer que le reste $r_k(x)$ définie en (\ref{TPfourTS2}) satisfait
\[
\lim_{k\rightarrow \infty} r_k(x) =
\lim_{k\rightarrow \infty}
\frac{1}{(k+1)!} f^{(k+1)}(\xi(k,c,x)) (x-c)^{k+1} = 0
\]
pour $x$ près de $c$.
\end{focus}

\begin{rmk}[\theory]
Le reste $r_k$ pour le Théorème de Taylor, théorème~\ref{theoTaylor}, est
aussi donné par la formule suivante.
\begin{equation}\label{TRest}
r_k(x) = \frac{1}{k!} \int_c^x (x-t)^k f^{(k+1)}(t) \dx{t} \; .
\end{equation}
Le Théorème de Taylor est parfois présenté avec cette formule pour le 
reste.

La démonstration de la formule (\ref{TRest}) est une suite d'intégrations par
parties.  Si $f$ est suffisamment différentiable, alors
\[
r_k(x) = \frac{1}{k!} \int_c^x (x-t)^k f^{(k+1)}(t) \dx{t} 
= \int_c^x g(t)h'(t) \dx{t}
\]
où $\displaystyle g(t) = \frac{(x-t)^k}{k!}$ et $h'(t) = f^{(k+1)}(t)$.
Donc $\displaystyle g'(t) = -\frac{(x-t)^{k-1}}{(k-1)!}$, 
$h(t) = f^{(k)}(t)$ et
\begin{align}
r_k(x) &= g(t)h(t)\bigg|_c^x - \int_c^x g'(t)h(t)\dx{t} =
\left(\frac{(x-t)^k}{k!} f^{(k)}(t)\right)\bigg|_c^x +
\int_c^x \frac{(x-t)^{k-1}}{(k-1)!} f^{(k)}(t) \dx{t} \nonumber \\
&= -\frac{(x-c)^k}{k!} f^{(k)}(c)
+\int_c^x \frac{(x-t)^{k-1}}{(k-1)!} f^{(k)}(t) \dx{t} \ . \label{TRestA}
\end{align}
Nous utilisons encore une fois la méthode d'intégration par parties.
Nous avons
\[
\int_c^x \frac{(x-t)^{k-1}}{(k-1)!} f^{(k)}(t) \dx{t}
= \int_c^x g(t)h'(t) \dx{t}
\]
où $\displaystyle g(t) = \frac{(x-t)^{k-1}}{(k-1)!}$ et
$h'(t) = f^{(k)}(t)$.  Donc
$\displaystyle g'(t) = -\frac{(x-t)^{k-2}}{(k-2)!}$, 
$h(t) = f^{(k-1)}(t)$ et
\begin{align*}
\int_c^x \frac{(x-t)^{k-1}}{(k-1)!} f^{(k)}(t) \dx{t}
&= g(t)h(t)\bigg|_0^x - \int_0^x g'(t)h(t)\dx{t} \\
&= \left(\frac{(x-t)^{k-1}}{(k-1)!} f^{(k-1)}(t)\right)\bigg|_c^x +
\int_c^x \frac{(x-t)^{k-2}}{(k-2)!} f^{(k-1)}(t) \dx{t} \\
&= -\frac{(x-c)^{k-1}}{(k-1)!} f^{(k-1)}(c) 
+\int_c^x \frac{(x-t)^{k-2}}{(k-2)!} f^{(k-1)}(t) \dx{t} \ .
\end{align*}
En combinant ce dernier résultat avec (\ref{TRestA}), nous obtenons
\[
r_k(x) = -\frac{(x-c)^k}{k!} f^{(k)}(c)
-\frac{(x-c)^{k-1}}{(k-1)!} f^{(k-1)}(c) 
+\int_c^x \frac{(x-t)^{k-2}}{(k-2)!} f^{(k-1)}(t) \dx{t} \ .
\]
Après $k$ intégrations par parties, nous obtenons
\[
r_k(x) = -\frac{(x-c)^k}{k!} f^{(k)}(c)
-\frac{(x-c)^{k-1}}{(k-1)!} f^{(k-1)}(c) -
\ldots - (x-c) f'(c) +\int_c^x f'(t) \dx{t} \ .
\]
Puisque $\displaystyle \int_c^x f'(t)\dx{t} = f(x)-f(c)$, nous obtenons
finalement
\begin{align*}
f(x) &= f(c) +  f'(c) (x-c) + \frac{f''(c)}{2!}(x-c)^2
+ \frac{f^{(3)}(c)}{3!} (x-c)^3 + \ldots \\
&\qquad
+ \frac{(x-c)^{k-1}}{(k-1)!} f^{(k-1)}(c) + \frac{(x-c)^k}{k!} f^{(k)}(c)
+ r_k(x) \ .
\end{align*}

Pour la formule du reste donnée au théorème~\ref{theoTaylor},
supposons que $c<x$.  La démonstration dans le cas $c>x$ est
identique.  Soit
\[
m = \min \{ f^{k+1}(t) : c \leq t \leq x \} \quad \text{et} \quad
M = \max \{ f^{k+1}(t) : c \leq t \leq x \} \ .
\]
Nous avons donc
\[
\frac{m}{k!} \int_c^x (x-t)^k \dx{t} \leq
r_k(x) = \frac{1}{k!} \int_c^x (x-t)^k f^{(k+1)}(t) \dx{t} \leq
\frac{M}{k!} \int_c^x (x-t)^k \dx{t} \ .
\]
Ce qui donne
\[
-\frac{m}{(k+1)!}\; (x-t)^{k+1}\bigg|_c^x \leq r_k(x) \leq
-\frac{M}{(k+1)!}\; (x-t)^{k+1}\bigg|_c^x
\]
et, plus précisément,
\[
\frac{m}{(k+1)!}\; (x-c)^{k+1} \leq r_k(x) \leq
\frac{M}{(k+1)!}\; (x-c)^{k+1} \ .
\]
Finalement, nous avons
\[
m \leq \frac{(k+1)!}{(x-c)^{k+1}}\; r_k(x) \leq M \ .
\]
Le théorème des valeurs intermédiaires peut être utilisé pour trouver
$\xi = \xi(k,c,x)$ entre $c$ et $x$ tel que
\[
f^{(k+1)}(\xi) = \frac{(k+1)!}{(x-c)^{k+1}}\; r_k(x) \ .
\]
Ainsi,
\[
r_k(x) = \frac{f^{(k+1)}(\xi)}{(k+1)!}\; (x-c)^{k+1}  \ .
\]
\end{rmk}

\begin{egg}
À l'exemple~\ref{eggtaylorgeom}, nous avons vu que la série de MacLaurin de
$\displaystyle f(x) = \frac{1}{1-x}$ est la série géométrique
$\displaystyle \sum_{n=0}^\infty x^n$.  À l'exemple~\ref{geoseries}, 
nous avons aussi vu que cette série tend vers $f(x)$ pour $-1< x < 1$.

Utilisons la formule du reste $r_k$ donnée en (\ref{TPfourTS2}) pour
démontrer d'une autre façon que la série géométrique tend vers $f(x)$
pour $x$ près de $0$.  En effet, puisque
\[
f^{(n)}(x) = \frac{n!}{(1-x)^{n+1}} \quad \text{pour} \quad n=0, 1, 2, \ldots
\]
nous avons que
\[
|r_k(x)| =
\left| \frac{1}{(k+1)!} f^{(k+1)}(\xi) x^{k+1} \right|
= \left| \frac{1}{(1-\xi)^{k+2}} x^{k+1} \right|
= \frac{|x|^{k+1}}{|1-\xi|^{k+2}}  \; .
\]
Pour $-1<x<0$, nous avons que $\xi <0$ car $\xi$ est entre $x$ et $0$
pour tout $k$.  Ainsi,
\[
0 \leq |r_k(x)|
= \frac{|x|^{k+1}}{|1-\xi|^{k+2}} < |x|^{k+1} \rightarrow 0
\quad \text{lorsque} \quad k \rightarrow \infty
\]
car $|x|<1$.  Par le théorème des gendarmes,
\[
\lim_{k\rightarrow \infty} |r_k(x)| = 0 \quad \text{pour} \quad -1<x<0 \; .
\]
Pour $0\leq x <1/2$, toujours en utilisant le fait que $\xi$ est entre
$0$ et $x$, nous avons que
\[
0 \leq \frac{|x|^{k+1}}{|1-\xi|^{k+2}} < \frac{|x|^{k+1}}{|1-x|^{k+2}}
= \left(\frac{x}{1-x}\right)^{k+1} \frac{1}{1-x} \rightarrow 0
\quad \text{lorsque} \quad k \rightarrow \infty
\]
car
\[
0 \leq \frac{x}{1-x} < 1 \quad \text{pour} \quad 0\leq x <1/2 \; .
\]
Par le théorème des gendarmes,
\[
\lim_{k\rightarrow \infty} |r_k(x)| = 0 \quad \text{pour} \quad 0\leq x< 1/2 \; .
\]

Démontrer à l'aide de la formule du reste que la série converge aussi pour
$1/2 \leq x <1$ demande une analyse plus poussée que nous ne ferons pas.
\end{egg}

\begin{egg}
Trouvons la séries de MacLaurin de $\cos(x)$ et montrons qu'elle
converge vers $\cos(x)$ pour tout $x$.  Nous allons faire de même pour
$\sin(x)$.

\subQ{i} Soit $f(x) = \cos(x)$.  Puisque
\[
f'(x) = -\sin(x) \ , \ f''(x) = -\cos(x) \ , \ f^{(3)}(x) =
\sin(x) \ \text{et} \ f^{(4)}(x) = \cos(x) = f(x) \ ,
\]
nous avons que
\[
f^{(4m)}(x) = \cos(x) \ , \ f^{(4m+1)}(x) = -\sin(x) \ , \ 
f^{(4m+2)}(x) = -\cos(x) \ \text{et} \ f^{(4m+3)}(x) = \sin(x)
\]
pour $m=0$, $1$, $2$, \ldots\quad  Donc
\begin{align*}
f^{(4m)}(0) &= \cos(0) = 1 \ , \ f^{(4m+1)}(0) = -\sin(0) = 0 \ , \ 
f^{(4m+2)}(0) = -\cos(0) = -1 \\
&\text{et} \ f^{(4m+3)}(0) = \sin(0) = 0
\end{align*}
pour $m=0$, $1$, $2$, \ldots\quad  En résumé, $f^{(n)}(0) = 0$ lorsque $n$ est
impaire, $f^{(n)}(0) = 1$ lorsque $n=0$, $4$, $8$, \ldots \  et
$f^{(n)}(0) = -1$ lorsque $n=2$, $6$, $10$, \ldots \  La série de MacLaurin
de $\cos(x)$ est donc
\[
\cos(x) \sim \sum_{n=0}^\infty \frac{1}{n!}\,f^{(n)}(0) \, x^n
= \sum_{n=0}^\infty \frac{(-1)^n}{(2n)!} \, x^{2n}
\; .
\]

Montrons que cette série converge vers $\cos(x)$ pour tout $x$ en
montrant que
\[
\lim_{k\rightarrow \infty} r_k(x) =
\lim_{k\rightarrow \infty}
\frac{1}{(k+1)!} f^{(k+1)}(\xi) x^{k+1} = 0
\]
pour tout $x$.  Puisque
$\displaystyle \left| f^{(k+1)}(\xi) \right|
= \left| (-1)^{(k+1)/2} \cos(\xi) \right| \leq 1$ pour $k$ impaire et
$\displaystyle \left| f^{(k+1)}(\xi) \right|
= \left| (-1)^{k/2+1} \sin(\xi) \right| \leq 1$ pour $k$ pair, nous avons
\[
0 \leq \left| r_k(x) \right| \leq \left| \frac{x^{k+1}}{(k+1)!} \right|
\rightarrow 0 \quad \text{lorsque} \quad k \rightarrow \infty
\]
quel que soit $x$ (remarque~\ref{r_sin_cos} ci-dessous).  Il
découle du théorème des gendarmes que
$\displaystyle \lim_{k\rightarrow \infty} r_k(x) = 0$ pour tout $x$.

\subQ{ii} Soit $g(x) = \sin(x)$.  Puisque
\[
g'(x) = \cos(x) \ , \ g''(x) = -\sin(x) \ , \ g^{(3)}(x) = -\cos(x) \ 
\text{et} \ g^{(4)}(x) = \sin(x) = g(x) \ , 
\]
nous avons que
\[
g^{(4m)}(x) = \sin(x) \ , \ g^{(4m+1)}(x) = \cos(x) \ , \ g^{(4m+2)}(x) =
-\sin(x) \ \text{et} \ g^{(4m+3)}(x) = -\cos(x)
\]
pour $m=0$, $1$, $2$, \ldots\quad  Ainsi,
\begin{align*}
g^{(4m)}(0) &= \sin(0) = 0 \ , \ g^{(4m+1)}(0) = \cos(0) = 1 \ , \ 
g^{(4m+2)}(0) = -\sin(0) = 0 \\ 
&\text{et} \quad
g^{(4m+3)}(0) = -\cos(0) = -1
\end{align*}
pour $m=0$, $1$, $2$, \ldots\quad  En résumé, $g^{(n)}(0) = 0$ lorsque $n$ est
paire, $g^{(n)}(0) = 1$ lorsque $n=1$, $5$, $9$, \ldots \ et
$g^{(n)}(0) = -1$ lorsque $n=3$, $7$, $11$, \ldots \  La série de MacLaurin
de $\sin(x)$ est donc
\[
\sin(x) \sim \sum_{n=0}^\infty \frac{1}{n!}\,g^{(n)}(0) \, x^n
= \sum_{n=0}^\infty \frac{(-1)^n}{(2n+1)!}\, x^{2n+1}
\; .
\]

Montrons que cette série converge vers $\sin(x)$ pour tout $x$ en
montrant que
\[
\lim_{k\rightarrow \infty} r_k(x) =
\lim_{k\rightarrow \infty}
\frac{1}{(k+1)!} g^{(k+1)}(\xi) x^{k+1} = 0
\]
pour tout $x$.  Puisque $\displaystyle \left| g^{(k+1)}(\xi) \right|
= \left| (-1)^{(k+1)/2} \sin(\xi) \right| \leq 1$ pour $k$ impaire et
$\displaystyle \left| g^{(k+1)}(\xi) \right|
= \left| (-1)^{k/2} \cos(\xi) \right| \leq 1$ pour $k$ pair, nous avons
\[
0 \leq \left| r_k(x) \right| \leq \left| \frac{x^{k+1}}{(k+1)!} \right|
\rightarrow 0 \quad \text{lorsque} \quad k \rightarrow \infty
\]
quel que soit $x$ (remarque~\ref{r_sin_cos} ci-dessous).  Il
découle du théorème des gendarmes que
$\displaystyle \lim_{k\rightarrow \infty} r_k(x) = 0$ pour tout $x$.
\label{sin_cos_series}
\end{egg}

\begin{rmk}
Dans l'exemple précédent, nous avons utilisé le fait que
\[
\lim_{k\rightarrow \infty}
\left| \frac{1}{(k+1)!} x^{k+1}\right| = 0
\]
pour tout $x$.  Pour démontrer ce résultat, il faut faire référence
à l'exemple~\ref{exp_serie_egg} où il a été démontré que la série
$\displaystyle \sum_{n=0}^\infty \frac{x^n}{n!}$ convergeait pour
tout $x$.  Il découle du théorème~\ref{NCforCofS} que le terme général
$\displaystyle \frac{x^n}{n!}$ de la série tend vers $0$ pour tout $x$
lorsque $n$ tend vers plus l'infini; c'est-à-dire,
\[
\lim_{k\rightarrow \infty} \frac{x^n}{n!} = 0 \quad \text{pour} \quad x\in\RR \; .
\]
\label{r_sin_cos}
\end{rmk}

\begin{egg}
Trouvons la série de Taylor de $f(x)=\sin(x)$ près de $\pi/4$ et
montrons qu'elle converge vers $f(x)$ pour tout $x$.

Nous avons vu à l'exemple précédent que
\[
f^{(4m)}(x) = \sin(x) \ , \ f^{(4m+1)}(x) = \cos(x) \ , \ 
f^{(4m+2)}(x) = -\sin(x) \ \text{et} \ f^{(4m+3)}(x) = -\cos(x)
\]
pour $m=0$, $1$, $2$, \ldots.  Ainsi,
\begin{align*}
f^{(4m)}\left(\frac{\pi}{4}\right) &= \sin\left(\frac{\pi}{4}\right) =
\frac{\sqrt{2}}{2} \ , \
f^{(4m+1)}\left(\frac{\pi}{4}\right) = \cos\left(\frac{\pi}{4}\right)
= \frac{\sqrt{2}}{2} \ , \\
f^{(4m+2)}\left(\frac{\pi}{4}\right) &= -\sin\left(\frac{\pi}{4}\right)
= -\frac{\sqrt{2}}{2}  \ \text{et} \ 
f^{(4m+3)}\left(\frac{\pi}{4}\right) = -\cos\left(\frac{\pi}{4}\right)
= -\frac{\sqrt{2}}{2}
\end{align*}
pour $m=0$, $1$, $2$, \ldots.  Nous remarquons de plus que
\begin{align*}
n=4m & \Rightarrow \frac{n(n-1)}{2} = \frac{4m(4m-1)}{2} = 2m(4m-1)
\quad \text{est pair},\\
n=4m+1 & \Rightarrow \frac{n(n-1)}{2} = \frac{(4m+1)\,4m}{2} =
2m(4m+1) \quad \text{est pair},\\
n=4m+2 & \Rightarrow \frac{n(n-1)}{2} = \frac{(4m+2)(4m+1)}{2} =
(2m+1)(4m-1) \quad \text{est impair et}\\
n=4m+3 & \Rightarrow \frac{n(n-1)}{2} = \frac{(4m+3)(4m+2)}{2} =
(4m+3)(2m+1) \quad \text{est impair}.
\end{align*}
Ceci nous permet d'écrire
\[
\sin(x) \sim \sum_{n=0}^\infty \frac{1}{n!} \,
f^{(n)}\left(\frac{\pi}{4}\right) \, \left(x-\frac{\pi}{4}\right)^n
= \sum_{n=0}^\infty (-1)^{n(n-1)/2} \frac{\sqrt{2}}{2} \,
\frac{1}{n!} \, \left(x-\frac{\pi}{4}\right)^n \; .
\]
Pour démontrer la convergence, nous montrons que le reste $r_k(x)$ dans le
Théorème de Taylor tend vers $0$ lorsque $k$ tend vers plus l'infini
quel que soit le choix de $x$; c'est-à-dire,
\[
\lim_{k\rightarrow \infty} r_k(x) =
\lim_{k\rightarrow \infty}
\frac{1}{(k+1)!} f^{(k+1)}(\xi) x^{k+1} = 0
\]
pour tout $x$.  Puisque
$\displaystyle \left| f^{(k+1)}(\xi) \right|
= \left| (-1)^{(k+1)/2} \sin(\xi) \right| \leq 1$ pour $k$ impaire et
$\displaystyle \left| f^{(k+1)}(\xi) \right|
= \left| (-1)^{k/2} \cos(\xi) \right| \leq 1$ pour $k$ pair, nous avons
\[
0 \leq \left| r_k(x) \right| \leq \left| \frac{1}{(k+1)!}\,
\left(x-\frac{\pi}{4}\right)^{k+1} \right|
\rightarrow 0 \quad \text{lorsque} \quad k \rightarrow \infty
\]
quel que soit $x$ (remarque~\ref{r_sin_cos}).
\end{egg}

\begin{egg}
Trouvons la série de McLaurin de $f(x)=\ln(1-x)$ et montrons
qu'elle converge vers $f(x)$ pour $|x|<1/2$.

Nous avons
\[
f'(x) = -\frac{1}{1-x} \ , \ f''(x) = -\frac{1}{(1-x)^2} \ , \ 
f^{(3)}(x) = -\frac{2}{(1-x)^3} \ , \  f^{(4)}(x) = -\frac{6}{(1-x)^4}
\ , \ldots
\]
Par induction, nous pouvons montrer que
\[
f^{(n)}(x) = -\frac{(n-1)!}{(1-x)^n} \quad , \quad n>0 \ .
\]
Puisque $f(0)=0$ et $f^{(n)}(0)= -(n-1)!$ pour $x=0$, nous avons
\[
\ln(1-x) \sim \sum_{n=1}^\infty \left(-\frac{(n-1)!}{n!}\right) x^n 
= - \sum_{n=1}^\infty \frac{x^n}{n} \ .
\]
Pour démontrer la convergence, il faut montrer que le reste $r_k(x)$ dans le
Théorème de Taylor tend vers $0$ lorsque $k$ tend vers plus l'infini
quel que soit $x \in ]-1/2,1/2[$; c'est-à-dire, 
\[
\lim_{k\rightarrow \infty} r_k(x) =
\lim_{k\rightarrow \infty} \frac{1}{(k+1)!} f^{(k+1)}(\xi) x^{k+1} = 0
\]
pour tout $x \in ]-1/2,1/2[$.  C'est une conséquence de la relation
suivante.
\begin{equation} \label{lnconV}
0 \leq \left| r_k(x) \right|
= \left|\frac{1}{k+1} \, \left(\frac{x}{1-\xi}\right)^{k+1} \right|
\leq \frac{1}{k+1} \rightarrow 0 \quad \text{lorsque} \quad
k \rightarrow \infty
\end{equation}
quel que soit $x \in ]-1/2,1/2[$.  Pour obtenir l'inégalité en
(\ref{lnconV}), nous remarquons que $|x| < 1/2 < 1-\xi$ pour tout
$x \in ]-1/2,1/2[$ car $\xi$ est entre $0$ et $x$.
\end{egg}

Considérons une fonction $f$ définie par
\[
f(x) = \sum_{n=0}^\infty a_n \, (x-c)^n \quad \text{pour} \quad |x-c|<R
\]
où $R>0$.  Est-ce que cette série est la série de Taylor de $f$ près de $c$?
C'est à dire, est-ce que $a_n = f^{(n)}(c) \big/n!$ pour $n=0$, $1$, $2$,
\ldots

Nous avons
\[
f(c) = \sum_{k=0}^\infty a_k \, (x-c)^k \bigg|_{x=c}
= \lim_{n\rightarrow \infty}
\left( \sum_{k=0}^n a_k (x-c)^k \bigg|_{x=c} \right)
= \lim_{n\rightarrow \infty} \left( a_0 \right) = a_0 \; .
\]
Grâce à la règle de dérivation des séries, nous avons que
\begin{align*}
f'(c) &= \sum_{k=1}^\infty k\,a_k (x-c)^{k-1} \bigg|_{x=c}
= \lim_{n\rightarrow \infty}
\left( \sum_{k=1}^n k a_k (x-c)^{k-1} \bigg|_{x=c}\right)
= \lim_{n\rightarrow \infty} \left( a_1 \right) = a_1 \; , \\
f''(c) &= \sum_{k=2}^\infty k(k-1)\,a_k (x-c)^{k-2} \bigg|_{x=c}
= \lim_{n\rightarrow \infty}
\left( \sum_{k=2}^n k(k-1)\,a_k (x-c)^{k-2} \bigg|_{x=c} \right) \\
&= \lim_{n\rightarrow \infty} \left( 2\, a_2 \right) = 2\, a_2 \; , \\
f^{(3)}(c) &= \sum_{k=3}^\infty k(k-1)(k-2)\,a_k (x-c)^{k-3} \bigg|_{x=c} \\
&= \lim_{n\rightarrow \infty}
\left( \sum_{k=3}^n k(k-1)(k-2)\, a_k (x-c)^{k-3} \bigg|_{x=c} \right)
= \lim_{n\rightarrow \infty} \left( 3\times 2 \,a_3 \right) = 3!\, a_3
\end{align*}
et, en générale,
\begin{align*}
f^{(m)}(c) &= \sum_{k=m}^\infty k(k-1)(k-2)\ldots(k-m+1)\,a_k
(x-c)^{k-m} \bigg|_{x=c} \\
&= \lim_{n\rightarrow \infty}
\left( \sum_{k=m}^n k(k-1)(k-2)\ldots(k-m+1)\,a_k (x-c)^{k-m}
\bigg|_{x=c} \right) \\
&= \lim_{n\rightarrow \infty} \left( m! \,a_m \right)
= m! \, a_m \; .
\end{align*}

Nous avons donc démontré le résultat suivant.

\begin{focus}{\prop}
Soit $f$ une fonction définie par
\[
f(x) = \sum_{n=0}^\infty a_n \, (x-c)^n \quad \text{pour} \quad |x-c|<R
\]
où $R>0$.  Alors
\[
a_n = \frac{1}{n!}\, f^{(n)}(c) \quad , \quad n=0,1,2,3,\ldots
\]
\end{focus}

En d'autres mots, la série qui définit la fonction $f$ est la série de Taylor
de $f$ près de $c$.  Un corollaire de la proposition précédente est le
suivant.

\begin{focus}{\cor} \label{uniqueSeries}
Si
\[
\sum_{n=0}^\infty a_n \, (x-c)^n = \sum_{n=0}^\infty b_n \, (x-c)^n
\quad \text{pour} \quad |x-c|<R
\]
où $R>0$, alors,
\[
a_n = b_n  \quad , \quad n=0,1,2,3,\ldots
\]
En fait, si
\[
f(x) = \sum_{n=0}^\infty a_n \, (x-c)^n = \sum_{n=0}^\infty b_n \, (x-c)^n
\quad \text{pour} \quad |x-c|<R \; ,
\]
alors
\[
a_n = b_n = \frac{1}{n!}\,f^{(n)}(c) \quad , \quad n=0,1,2,3,\ldots
\]
En d'autres mots, une fonction $f:\RR \rightarrow \RR$ peut avoir au
plus une représentation en série entière près de $c$.
\end{focus}

Ce dernier corollaire peut nous aider à trouver la série de Taylor de
certaines fonction sans avoir à calculer une dérivée.

\begin{egg}
Trouvons la série de MacLaurin de la fonction
$\displaystyle f(x) = \frac{1}{1+x^3}$.

La série géométrique est
\[
\frac{1}{1-u} = \sum_{n=0}^\infty u^n \quad , \quad |u|<1 \; .
\]
Si $u=-x^3$, alors
\[
f(x) = \frac{1}{1+x^3} = \frac{1}{1-\left(-x^3\right)}
= \sum_{n=0}^\infty \left(-x^3\right)^n
= \sum_{n=0}^\infty (-1)^n x^{3n} \quad , \quad |x|<1 \; .
\]
C'est la série cherchée.
\end{egg}

\begin{egg}
Trouvons la série de MacLaurin de la fonction
$\displaystyle f(x) = \frac{1}{7+2x}$.

Remarquons que
\[
f(x) = \frac{1}{7+2x} = \frac{1}{7}
\left( \frac{1}{1-\left(-\frac{2x}{7}\right)} \right) \; .
\]
Si nous substituons $u= -2x/7$ dans la série géométrique
\[
\frac{1}{1-u} = \sum_{n=0}^\infty u^n \quad , \quad |u|<1 \; ,
\]
nous trouvons
\[
f(x) = \frac{1}{7} \sum_{n=0}^\infty \left(-\frac{2x}{7}\right)^n
= \sum_{n=0}^\infty (-1)^n \frac{2^n}{7^{n+1}} \, x^n
\quad , \quad \left|\frac{2x}{7}\right| <1 \; .
\]
C'est la série cherchée.  Elle converge vers $f(x)$ pour $|x| < 7/2$.
\end{egg}

\begin{egg}
Soit $f(x) = \sin(x/3)$.  Trouvons la série de MacLaurin de
cette fonction et montrons que cette série converge vers $f(x)$ pour tout
$x\in \RR$.

Si nous substituons $u=x/3$ dans la série de MacLaurin
\[
\sin(u) = \sum_{n=0}^\infty \frac{(-1)^n}{(2n+1)!}\, u^{2n+1} \quad
\text{pour} \quad u \in \RR \; ,
\]
nous trouvons
\[
f(x) = \sin\left(\frac{x}{3}\right)
= \sum_{n=0}^\infty \frac{(-1)^n}{(2n+1)!}\,
\left(\frac{x}{3}\right)^{2n+1}
= \sum_{n=0}^\infty \frac{(-1)^n}{3^{2n+1}\,(2n+1)!}\,x^{2n+1}
\quad , \quad x\in \RR \; ,
\]
\end{egg}

\begin{egg}
Trouvons la série de MacLaurin de la fonction
$\displaystyle f(x) = \frac{1}{(x-1)(1-2x)}$.

La méthode des fractions partielles nous permet d'exprimer $f$ de la façon
suivante.
\[
f(x) = \frac{1}{(x-1)(1-2x)} = \frac{A}{x-1} + \frac{B}{1-2x}
\]
où $A$ et $B$ sont des constantes.  Pour déterminer $A$ et $B$, nous exprimons
les fractions sur un dénominateur commun.
\[
\frac{1}{(x-1)(1-2x)} = \frac{A(1-2x)}{(x-1)(1-2x)} +
\frac{B(x-1)}{(1-2x)(x-1)} \; .
\]
Les numérateurs de chaque coté de l'égalité doivent donc être égaux;
c'est-à-dire,
\[
1 = A(1-2x) + B(x-1) \; .
\]
Si nous substituons $x=1/2$ dans cette équation, nous trouvons $1= -B/2$.  Donc
$B=-2$.  De même, si nous substituons $x=1$ dans cette équation, nous trouvons
$1= -A$.  Donc $A=-1$.  Nous avons ainsi que
\[
f(x) = -\frac{1}{x-1} - \frac{2}{1-2x} \; .
\]
Le premier terme peut être représenté par la série géométrique.
\[
-\frac{1}{x-1} = \frac{1}{1-x} = \sum_{n=0}^\infty x^n \quad , \quad
|x|<1 \; .
\]
Pour trouver la série représentant le deuxième terme, nous utilisons encore la
série géométrique
\[
\frac{1}{1-u} = \sum_{n=0}^\infty u^n \quad , \quad |u|<1
\]
avec $u=2x$ pour obtenir
\[
\frac{2}{1-2x} = 2 \left(\frac{1}{1-2x}\right) = 2 \sum_{n=0}^\infty (2x)^n
= \sum_{n=0}^\infty 2^{n+1} x^n \quad , \quad |2x|<1 \; .
\]
Cette dernière série converge pour $|x|<1/2$ alors que la série pour le
premier terme de $f$ converge pour $|x|<1$.  Donc
\[
f(x) = \sum_{n=0}^\infty x^n - \sum_{n=0}^\infty 2^{n+1} x^n
= \sum_{n=0}^\infty \left( 1 - 2^{n+1}\right) x^n \quad ,
\quad |x|<1/2 \; .
\]
Nous avons additionné les séries en additionnant les coefficients de
$x^n$ pour $n=0$, $1$, $2$, \ldots.
\end{egg}

\begin{egg}
Si $\displaystyle f(x) = \frac{1}{(x-1)(1-2x)}$, quelle est la valeur
de $f^{(100)}(0)$?

Nous avons vu à l'exemple précédent que
\[
f(x) = \sum_{n=0}^\infty \left( 1 - 2^{n+1}\right) x^n \quad ,
\quad |x|<1/2 \; .
\]
Or, d'après le corollaire~\ref{uniqueSeries}, le coefficient de $x^{100}$ est
\[
\frac{1}{100!}\,f^{(100)}(0) = \left( 1 - 2^{101}\right) \; .
\]
Donc
\[
f^{(100)}(0) = 100!\,\left( 1 - 2^{101}\right) \; .
\]
\end{egg}

Il est aussi possible d'utiliser les séries entières pour évaluer des
intégrales.

\begin{egg}
Évaluons l'intégrale $\displaystyle \int_0^{0.5} \cos(x^2) \dx{x}$ avec une
précision de $10^{-8}$.

La série de MacLaurin de $\cos(u)$ est
\[
\cos(u) = \sum_{n=0}^\infty (-1)^n \frac{1}{(2n)!}\,u^{2n} \quad , \quad
u \in \RR \; .
\]
Si nous substituons $u=x^2$, nous trouvons
\[
\cos\left(x^2\right) = \sum_{n=0}^\infty (-1)^n
\frac{1}{(2n)!}\,\left(x^2\right)^{2n}
= \sum_{n=0}^\infty (-1)^n \frac{1}{(2n)!}\,x^{4n} \quad , \quad
x \in \RR \; .
\]
Ainsi,
\begin{align*}
\int_0^{0.5} \cos\left(x^2\right) \dx{x} 
&= \sum_{n=0}^\infty (-1)^n \frac{1}{(2n)!}\,\int_0^{0.5} x^{4n} \dx{x}
= \sum_{n=0}^\infty (-1)^n \frac{1}{(2n)!}\,
\frac{x^{4n+1}}{4n+1}\bigg|_0^{0.5} \\
&= \sum_{n=0}^\infty (-1)^n \frac{1}{2^{4n+1}\,(4n+1)\,(2n)!} \; .
\end{align*}
C'est une série alternée de la forme
$\displaystyle \sum_{n=0}^\infty (-1)^n a_n$ où
$\displaystyle a_n = \frac{1}{2^{4n+1}\,(4n+1)\,(2n)!}$.  Notez que cette
série satisfait le Test des séries alternées (i.e. le théorème~\ref{altTest}),
ce qui prouve qu'elle converge en accord avec la conclusion du
théorème~\ref{der_int_series}.  De plus, si
$\displaystyle S = \sum_{n=0}^\infty (-1)^n a_n$ et
$\displaystyle S_N = \sum_{n=0}^N (-1)^n a_n$ est la $N^{e}$ somme
partiel, alors $\displaystyle |S-S_N| \leq a_{N+1}$.

La somme partiel $S_N$ donnera une approximation de
$\displaystyle \int_0^{0.5} \cos(x^2) \dx{x}$ avec
une précision de $10^{-8}$ si nous choisissons $N$ tel que
$\displaystyle |S-S_N| \leq a_{N+1} < 10^{-8}$.
Nous avons
\[
a_{N+1} = a_3 = \frac{1}{13\times 2^{13}\times 6!} \approx 1.3042
\times 10^{-8} \not< 10^{-8}
\]
pour $N=2$ et
\[
a_{N+1} = a_4 = \frac{1}{17\times 2^{17}\times 8!} \approx 1.1
\times 10^{-11} < 10^{-8}
\]
pour $N=3$.  La somme partiel
\begin{align*}
S_3 &= \sum_{n=0}^3 (-1)^n \frac{1}{2^{4n+1}\,(4n+1)\,(2n)!}
= \frac{1}{2} - \frac{1}{5\times 2^5\times 2!}
+\frac{1}{9\times 2^9\times 4!} - \frac{1}{13\times 2^{13}\times 6!} \\
&\approx 0.4968840292 \; .
\end{align*}
est donc l'approximation cherchée.
\end{egg}

\section{Série du binôme}

Pour $k \in \ZZ$, posons
\begin{align*}
\binom{k}{n} &= \frac{k(k-1)(k-2)\ldots(k-n+1)}{n!} \quad \text{pour}
\quad n=1,2,3,\ldots
\intertext{et}
\binom{k}{0} &= 1 \; .
\end{align*}

Si $k$ est un entier non négatif, la {\bfseries formule du binôme}
\index{Binôme!Formule du binôme} est
\begin{equation}\label{finitebin}
(1+x)^k = 1 + k\,x + \frac{k(k-1)}{2}\,x^2 + \ldots + k\,x^{k-1} + x^k
= \sum_{n=0}^k \binom{k}{n} x^n \; .
\end{equation}
Pour cette raison, $\displaystyle \binom{k}{n}$ est appelé un
{\bfseries coefficient du binôme}\index{Binôme!Coefficient du binôme}.
Ceux qui auraient déjà étudié la combinatoire reconnaissent
certainement le coefficient du binôme puisqu'il représente le nombre
de possibilités de choisir $n$ éléments d'un ensemble de 
$k$ éléments distincts en ignorant l'ordre.

De plus, la série géométrique
\[
\frac{1}{1-u} = \sum_{n=0}^\infty u^n \quad , \quad |u|<1 \; ,
\]
avec $u=-x$ donne
\[
(1+x)^{-1} = \sum_{n=0}^\infty \binom{-1}{n} x^n \quad , \quad |x|<1\; ,
\]
car
\begin{align*}
\binom{-1}{n} &= \frac{(-1)(-1-1)(-1-2)\ldots(-1-n+1)}{n!} \\
&= \frac{(-1)(-2)(-3)\ldots(-n)}{n!} = \frac{(-1)^n n!}{n!} = (-1)^n
\quad , \quad n>0 \;.
\end{align*}

Qu'arrive-t-il si $k$ est un nombre réel?  Existe-t-il une formule semblable
à la formule du binôme si $k$ est un nombre réel quelconque?  Le théorème
suivant répond à ces questions.

\begin{focus}[][Série du binôme]{\thm}\label{bin_series}
\index{Série du binôme}
Soit $k\in \RR$.  Alors
\begin{equation}\label{infinitebin}
(1+x)^k = \sum_{n=0}^\infty \binom{k}{n} \, x^n \quad , \quad |x|<1 \; .
\end{equation}
\end{focus}

\begin{rmk}
La formule (\ref{finitebin}) où $k$ est un entier non négatif est
incluse dans la formule (\ref{infinitebin}).  En effet, si $k$ est un
entier positif,
\[
\binom{k}{n} = \frac{k(k-1)(k-2)\ldots(k-n+1)}{n!} = 0
\quad \text{pour} \quad n > k
\]
car un des facteurs au numérateur est nul.
\end{rmk}

\begin{proof}[\theory][Théorème~\ref{bin_series}]
Soit $f(x)=(1+x)^k$ où $k\in \RR$.  Nous avons
\begin{align*}
f'(x) &= k\,(1+x)^{k-1} \; ,\\
f''(x) &= k(k-1)\,(1+x)^{k-2} \; , \\
f^{(3)}(x) &= k(k-1)(k-2)\,(1+x)^{k-3} \; , \\
\vdots & \qquad \vdots
\end{align*}
Par induction, nous trouvons
\[
f^{(n)}(x) = k(k-1)(k-2)\ldots(k-n+1)\,(1+x)^{k-n} \quad , \quad
n=1,2,3,\ldots
\]
La série de MacLaurin de $f$ est donc
\[
f(x) \sim \sum_{n=0}^\infty \frac{1}{n!}\,f^{(n)}(0)\, x^n
= \sum_{n=0}^\infty \frac{1}{n!}\,k(k-1)(k-2)\ldots(k-n+1)\, x^n
= \sum_{n=0}^\infty \binom{k}{n} \, x^n \; .
\]

Montrons que cette série converge pour $|x|<1$.  Pour ce faire, nous
calculons le rayon de convergence $R$.  Nous avons une série de la forme
$\displaystyle \sum_{n=0}^\infty a_n x^n$ où
$\displaystyle a_n = \binom{k}{n}$.  Donc 
\[
\frac{1}{R} = \lim_{n\rightarrow \infty} \left|\frac{a_{n+1}}{a_n}\right|
= \lim_{n\rightarrow \infty} \left|
\frac{\displaystyle \frac{k(k-1)(k-2)\ldots(k-n)}{(n+1)!}}
{\displaystyle \frac{k(k-1)(k-2)\ldots(k-n+1)}{n!}} \right|
= \lim_{n\rightarrow \infty} \left| \frac{k-n}{n+1} \right| = 1
\]
et ainsi $R=1$.

Il reste à montrer que la série du binôme converge bien vers
$(1+x)^k$.  Pour ce faire, nous ne prouverons pas que le reste
$r_k(x)$ définie en (\ref{TPfourTS2}) tend vers $0$ lorsque $k$ tend
vers plus l'infini pour $x \in ]-1,1[$.  Nous allons utiliser le fait
que l'équation différentielle
\begin{equation} \label{binODE}
(1+x) y'= k\ y \quad , \quad y(0)=1
\end{equation}
à une unique solution près de l'origine.

Remarquons que $f(x) = (1+x)^k$ est une solution de (\ref{binODE})
pour $|x|<1$.  De plus, si nous posons
\[
g(x) = \sum_{n=0}^\infty \binom{k}{n} \, x^n \quad , \quad |x|<1 \; ,
\]
nous avons que
\[
g'(x) = \sum_{n=1}^\infty \binom{k}{n}\,n \, x^{n-1}
\quad , \quad |x|<1 \; .
\]
Ainsi,
\[
(1+x) g'(x) = (1+x) \sum_{n=1}^\infty \binom{k}{n}\,n \, x^{n-1} \\
= \sum_{n=1}^\infty \binom{k}{n}\,n \, x^{n-1} +
\sum_{n=1}^\infty \binom{k}{n}\,n \, x^n \quad , \quad |x|<1 \; .
\]
Si nous remplaçons $n$ par $m+1$ dans la première série, nous obtenons
\begin{align*}
(1+x) g'(x)
&= \sum_{m=0}^\infty \binom{k}{m+1}\,(m+1) \, x^m +
\sum_{n=1}^\infty \binom{k}{n}\,n \, x^n \\
&= k + \sum_{m=1}^\infty \binom{k}{m+1}\,(m+1) \, x^m +
\sum_{n=1}^\infty \binom{k}{n}\,n \, x^n \quad , \quad |x|<1 \; .
\end{align*}
Comme le nom $m$ donné à l'indice de la première série est arbitraire, nous
pouvons remplacer $m$ par $n$ dans cette série.  Nous obtenons
\begin{align*}
(1+x) g'(x)
&= k + \sum_{n=1}^\infty \binom{k}{n+1}\,(n+1) \, x^n +
\sum_{n=1}^\infty \binom{k}{n}\,n \, x^n \\
&= k + \sum_{n=1}^\infty \left( \binom{k}{n+1}\,(n+1)
+ \binom{k}{n}\,n \right) \, x^n \quad , \quad |x|<1 \; .
\end{align*}
Or
\begin{align*}
& \binom{k}{n+1}\,(n+1) + \binom{k}{n}\,n \\
& \qquad = \frac{k(k-1)(k-2)\ldots(k-n+1)(k-n)}{(n+1)!} \, (n+1)
+ \frac{k(k-1)(k-2)\ldots(k-n+1)}{n!} \, n \\
& \qquad = \frac{k(k-1)(k-2)\ldots(k-n+1)}{n!} \, (k-n) +
\frac{k(k-1)(k-2)\ldots(k-n+1)}{n!} \, n \\
& \qquad = k\,\frac{k(k-1)(k-2)\ldots(k-n+1)}{n!} = k\, \binom{k}{n}
\quad , \quad n=1,2,3,\ldots
\end{align*}
Ainsi,
\[
(1+x) g'(x)
= k + k \,\sum_{n=1}^\infty \binom{k}{n} \, x^n
= k\, \sum_{n=0}^\infty \binom{k}{n} \, x^n
= k\, g(x) \quad , \quad |x|<1 \; .
\]
Puisque $g(0) =1$, nous avons que $g$ est une solution de (\ref{binODE}) pour
$|x|<1$.  Par unicité des solutions $f(x)=g(x)$ pour $|x|<1$; c'est-à-dire,
\[
(1+x)^k = \sum_{n=0}^\infty \binom{k}{n} \, x^n \quad , \quad |x|<1 
\; .
\]
\end{proof}

\begin{egg}
Trouvons la série de MacLaurin de la fonction
$\displaystyle \frac{x^2}{(1-x^3)^{1/2}}$.

Si nous substituons $u=-x^3$ et $k=-1/2$ dans la série du binôme
\[
(1+u)^k = \sum_{n=0}^\infty \binom{k}{n} \, u^n \quad , \quad |u|<1 \; ,
\]
nous trouvons
\[
(1-x^3)^{-1/2} = \sum_{n=0}^\infty \binom{-1/2}{n} \, (-x^3)^n
= \sum_{n=0}^\infty (-1)^n\,\binom{-1/2}{n} \, x^{3n} \quad ,
\quad |x|<1 \; .
\]
Or
\begin{align*}
(-1)^n\,\binom{-1/2}{n} &=
(-1)^n\,\frac{-\frac{1}{2}\left(-\frac{1}{2}-1\right)
\left(-\frac{1}{2}-2\right)\ldots\left(-\frac{1}{2}-n+1\right)}{n!} \\
&= \frac{\frac{1}{2}\left(\frac{1}{2}+1\right)\left(\frac{1}{2}+2\right)\ldots
\left(\frac{1}{2}+n-1\right)}{n!} \\
&= \frac{\frac{1}{2}\left(\frac{3}{2}\right)\left(\frac{5}{2}\right)\ldots
\left(\frac{2n-1}{2}\right)}{n!}
= \frac{3\times 5 \times 7 \times \ldots \times (2n-1)}{2^n\,n!} \; .
\end{align*}
Donc
\[
(1-x^3)^{-1/2} = \sum_{n=0}^\infty
\frac{3\times 5 \times 7 \times \ldots \times (2n-1)}{2^n\,n!} \, x^{3n} \; ,
\quad |x|<1 \; ,
\]
et
\[
\frac{x^2}{(1-x^3)^{1/2}} = \sum_{n=0}^\infty
\frac{3\times 5 \times 7 \times \ldots \times (2n-1)}{2^n\,n!} \, x^{3n+2}
\quad , \quad |x|<1 \; .
\]
\end{egg}

\section{Définition de l'exponentiel et de son inverse \theory}\label{DEFofE}

Donnons une définition rigoureuse de $b^x$ pour $b>0$ et
$x\in \RR$ dans le but d'illustrer le niveau de sophistication mathématique
associé à $b^x$.  Cette définition a l'avantage d'être mieux adaptée au
calcul algébrique que la Définition~\ref{pr_def_of_exp}. 

La définition de $b^x$ pour $b>0$ et $x\in \RR$ requière la série entière de
la fonction exponentielle $e^x$.

\begin{focus}{\prop}
Soit $E: \RR \rightarrow ]0,+\infty[$, la fonction définie par
\[
E(x) \equiv \sum_{n=0}^\infty \frac{x^n}{n!}
= \lim_{N\rightarrow \infty} \sum_{n=0}^N \frac{x^n}{n!}
\]
pour $x\in \RR$.  Cette série converge pour tout $x\in \RR$ comme nous
avons démontré à l'aide du test de d'Alembert à
l'exemple~\ref{exp_serie_egg}.

La fonction $E$ possède les propriétés suivantes.
\begin{enumerate}
\item $E(0)=1$.
\item $E(a+b) = E(a)\, E(b)$ pour tous nombres réels $a$ et $b$.
\item $\displaystyle \dydx{E}{x}(x) = E(x)$ pour tout $x$.
\end{enumerate}
\end{focus}

\begin{focus}{\dfn}
Le nombre $e$ est définie par
\[
e \equiv E(1) = \sum_{n=0}^\infty \frac{1}{n!}
= \lim_{N\rightarrow \infty} \sum_{n=0}^N \frac{1}{n!} \; .
\]
\end{focus}

Nous pouvons montrer que c'est le même nombre $e$ que nous avons
défini précédemment par la formule
\[
e = \lim_{n\rightarrow \infty} \left(1+\frac{1}{n}\right)^n \; .
\]
Le nombre $e$ est irrationnel; c'est-à-dire que $e$ ne peut être écrit comme
une fraction $m/n$ avec $m$ et $n$ des entiers.

À l'aide des deux premières propriétés de la proposition précédente,
nous pouvons vérifier que $e^x = E(x)$ pour $x$ un nombre rationnel,
Pour cette raison,
et puisque $E$ est continue et croissante, nous pouvons montrer que la
définition suivante de $e^x$ correspond à la Définition~\ref{pr_def_of_exp}.

\begin{focus}{\dfn}
\[
e^x \equiv E(x) = \sum_{n=0}^\infty \frac{x^n}{n!} \quad , \quad x \in \RR \; .
\]
\end{focus}

Il découle de cette définition que $e^x$ satisfait les propriétés des
exposants: $e^{x+y} = e^x\, e^y$, $e^{xy} = (e^x)^y$, etc.

\begin{focus}{\dfn}
La fonction inverse de $e^x$ est alors définie par
\[
y = \ln (x) \quad \Leftrightarrow \quad e^y=x
\]
pour $x>0$.
\end{focus}

Nous pouvons montrer avec cette définition que $\ln(x)$ satisfait les
propriétés des logarithmes:
$\ln(xy) = \ln(x)+\ln(y)$, $\ln(x^y) = y\ln(x)$, etc.

\begin{focus}{\dfn}
Pour $x\in \RR$ et $b>0$, nous définissons $b^x$ par
\[
b^x \equiv e^{x\ln(b)}
\]
Pour $x > 0$ et $b>0$, nous définissons $\log_b(x)$ par
\[
\log_b{x} \equiv \frac{\ln(x)}{\ln(b)} \; .
\]
\end{focus}

Ces définitions coïncident avec les définitions traditionnelles de $b^x$ et
$\log_b(x)$ respectivement.  Par exemple, si $x$ est un entier positif,
\[
e^{x\ln(b)} = \underbrace{e^{\ln(b)} \times e^{\ln(b)} \times
  \ldots \times e^{\ln(b)}}_{x \text{ fois}}
= \underbrace{b \times b \times \ldots \times b}_{x \text{ fois}}
= b^x \; .
\]

De plus, puisque $\displaystyle b^{1/\ln(b)} = e$ car
\[
\ln( b^{1/\ln(b)} ) = \frac{1}{\ln(b)} \, \ln(b) = 1 = \ln(e) \;,
\]
nous obtenons que
\[
b^{\ln(x)/\ln(b)} = \left(b^{1/\ln(b)}\right)^{\ln(x)}
= e^{\ln(x)} = x \; .
\]
$\ln(x)/\ln(b)$ est donc bien l'exposant qu'il faut donner à $b$ pour obtenir
$x$.

Comme nous avons vu lors de l'étude de la dérivée des fonctions
exponentielles, le nombre $e$ joue un rôle très important en
mathématiques.  Comme le nombre $\pi$, le nombre $e$ est une des
valeurs fondamentales en mathématiques.

\begin{egg}
Trouvons la série de MacLaurin de la fonction $f(x) = (1+x)e^{-2x}$.

Si nous substituons $u=-2x$ dans la série
\[
e^u= \sum_{n=0}^\infty \frac{1}{n!}\, u^n \quad , \quad u \in \RR
\; ,
\]
nous obtenons
\[
e^{-2x} = \sum_{n=0}^\infty \frac{1}{n!}\, (-2x)^n
= \sum_{n=0}^\infty \frac{(-2)^n}{n!}\, x^n \quad , \quad x \in \RR
\; .
\]
Ainsi,
\begin{align*}
(1+x)e^{-2x} &= (1+x) \sum_{n=0}^\infty \frac{(-2)^n}{n!}\, x^n
= \sum_{n=0}^\infty \frac{(-2)^n}{n!}\, x^n +
\sum_{n=0}^\infty \frac{(-2)^n}{n!}\, x^{n+1} \quad , \quad x \in \RR
\; .
\end{align*}
Si nous posons $m=n+1$ dans la deuxième série, nous obtenons
\begin{align*}
(1+x)e^{-2x} &= \sum_{n=0}^\infty \frac{(-2)^n}{n!}\, x^n +
\sum_{m=1}^\infty \frac{(-2)^{m-1}}{(m-1)!}\, x^m  \quad , \quad x \in
\RR \; .
\end{align*}
Notons que $m=1$ lorsque $n=0$.  Comme le nom $m$ de l'indice utilisé dans
la deuxième série est arbitraire, nous pouvons remplacer $m$ par $n$ dans cette
série.  Nous obtenons
\begin{align*}
(1+x)e^{-2x} &= \sum_{n=0}^\infty \frac{(-2)^n}{n!}\, x^n +
\sum_{n=1}^\infty \frac{(-2)^{n-1}}{(n-1)!}\, x^n
=  1 + \sum_{n=1}^\infty \frac{(-2)^n}{n!}\, x^n +
\sum_{n=1}^\infty \frac{(-2)^{n-1}}{(n-1)!}\, x^n \\
&=  1 + \sum_{n=1}^\infty \left( \frac{(-2)^n}{n!} +
\frac{(-2)^{n-1}}{(n-1)!} \right)\, x^n
=  1 + \sum_{n=1}^\infty \frac{(-2)^{n-1}}{(n-1)!} \left(
\frac{n-2}{n} \right)\, x^n
\end{align*}
pour $x \in \RR$.
\end{egg}

}  % End of theory

\section{Exercices}

\subsection{Convergence des séries entières}

\begin{question}
Déterminez le rayon et l'intervalle de convergence des séries suivantes.
\begin{center}
\begin{tabular}{*{2}{l@{\hspace{0.5em}}l@{\hspace{3em}}}l@{\hspace{0.5em}}l}
\subQ{a} & $\displaystyle \sum_{n=1}^\infty \frac{x^n}{n^2}$ &
\subQ{b} & $\displaystyle \sum_{n=0}^\infty \frac{n^2 x^n}{10^n}$ &
\subQ{c} & $\displaystyle \sum_{n=0}^\infty \frac{x^n}{n!}$ \\[1em]
\subQ{d} & $\displaystyle \sum_{n=0}^\infty \sqrt{n}(x+2)^n$ &
\subQ{e} & $\displaystyle \sum_{n=1}^\infty (-1)^n \,\frac{(x-1)^n}{n\,4^n}$ &
\subQ{f} & $\displaystyle \sum_{n=1}^\infty
\frac{n x^n}{ 1 \cdot 3 \cdot 5 \cdot \ldots \cdot (2n-1)}$  \\[1em]
\subQ{g} & $\displaystyle \sum_{n=0}^\infty (-1)^n \frac{(5x-1)^n}{\sqrt{n+1}}$
&
\subQ{h} & $\displaystyle   \sum_{n=1}^\infty\;(-1)^n\frac{(5x-1)^n}{3^n\,n^2}$
&
\subQ{i} & $\displaystyle \sum_{n=1}^\infty \frac{(-5)^n (x+3)^n}{n}$  \\[1em]
\subQ{j} & $\displaystyle \sum_{n=1}^\infty \frac{3^n(x-2)^n}{n^2}$ &
\subQ{k} & $\displaystyle \sum_{n=1}^\infty \frac{(x-2)^n}{3^n\sqrt{n}}$ &
&
\end{tabular}
\end{center}
\label{9Q1}
\end{question}

\begin{question}
Déterminez le rayon et l'intervalle de convergence des séries suivantes:
\begin{center}
\begin{tabular}{*{2}{l@{\hspace{0.5em}}l@{\hspace{3em}}}l@{\hspace{0.5em}}l}
\subQ{a} & $\displaystyle \sum_{n=1}^\infty \frac{(-1)^n x^{2n-1}}{(2n-1)!}$ &
\subQ{b} & $\displaystyle \sum_{n=0}^\infty (-1)^n \,\frac{x^{2n+1}}{2n+1}$ &
\subQ{c} & $\displaystyle \sum_{n=1}^\infty\frac{4^nx^{2n}}{n^2}$
\end{tabular}
\end{center}
\noindent Note: Nous ne pouvons pas utiliser directement le
théorème~\ref{RadiusConvComp}.  Il faut retourner à la source.
\label{9Q2}
\end{question}

\subsection{Fonctions définies par des séries}

\begin{question}
Utilisez la série géométrique pour déterminer le rayon de convergence,
l'intervalle de convergence, et la somme de chacune des séries
suivantes.
\begin{center}
\begin{tabular}{*{1}{l@{\hspace{0.5em}}l@{\hspace{6em}}}l@{\hspace{0.5em}}l}
\subQ{a} & $\displaystyle \sum_{n=0}^\infty x^{n+1}$ &
\subQ{b} & $\displaystyle \sum_{n=1}^\infty x^{2n}$
\end{tabular}
\end{center}
\label{9Q3}
\end{question}

\begin{question}
Utilisez la série géométrique pour déterminer la somme de la série
$\displaystyle \sum_{n=1}^\infty n 2^{n-1} x^{n+1}$ sur son intervalle de
convergence?  Expliquez pourquoi nous ne pouvons pas utiliser la série
géométrique pour déterminer le rayon de convergence et l'intervalle de
convergence de cette série?  Déterminez le rayon de convergence et
l'intervalle de convergence de cette série.
\label{9Q4}
\end{question}

\begin{question}
La fonction $g(x)$ est définie par la série suivante.
\begin{equation} \label{questDefg}
  g(x) = \sum_{n=0}^\infty \frac{n x^n}{3^n} \ .
\end{equation}

\subQ{a} Quelle est le domaine de cette fonction?  En d'autres mots,
qu'elle est l'intervalle de convergence de la série?\\
\subQ{b} Pouvons-nous calculer $g'(2)$?  Si oui, calculez la valeur de
$g'(2)$.\\
\subQ{c} Trouvez la série qui représente
$\displaystyle \int_0^2 g(x) \dx{x}$ si cela est possible.  Justifiez
si c'est possible.
\label{9Q5}
\end{question}

\begin{question}
Donnez la série entière de la fonction $\displaystyle f(x) = \ln (1 + x^2)$
pour $x$ près de l'origine.  Quel est le rayon de convergence de cette
série?  Combien de termes de la série entière doivent être utilisés
pour obtenir une approximation de $\ln(5/4)$ avec une précision d'au
moins $5 \times 10^{-5}$ (i.e. avec une erreur plus petite que
$5 \times 10^{-5}$)?\\
\noindent Note: Vous ne devez pas calculer $\ln(5/4)$ avec votre
calculatrice pour comparer avec les sommes partielles.
\label{9Q6}
\end{question}

\begin{question}
Utilisez la série de MacLaurin de chacune des fonctions suivantes pour
calculer $\displaystyle \dydxn{f}{x}{9}(0)$.
\begin{center}
\begin{tabular}{*{2}{l@{\hspace{0.5em}}l@{\hspace{4em}}}l@{\hspace{0.5em}}l}
\subQ{a} & $\displaystyle f(x) = e^{x^3}$ &
\subQ{b} & $\displaystyle f(x) = x^3\,e^{x^2}$ &
\subQ{c} & $\displaystyle f(x) = \frac{x}{4+x^2}$
\end{tabular}
\end{center}
\label{9Q7}
\end{question}

\begin{question}
Trouvez les trois premiers termes non nuls de la série de MacLaurin de
$f(x) = e^{x^2}\cos(x)$.
\label{9Q8}
\end{question}

\begin{question}
Pour chacune des pairs de fonctions $g(x)$, $f(x)$ ci-dessous,
utilisez la série de Maclaurin de de la fonction $g(x)$ pour obtenir
la série de Maclaurin de la fonction $f(x)$.
\begin{center}
\begin{tabular}{*{1}{l@{\hspace{0.5em}}l@{\hspace{3em}}}l@{\hspace{0.5em}}l}
\subQ{a} & $g(x) = \sin(x)$ et $f(x) = \sin(2x^3)$ &
\subQ{b} & $g(x) = \cos(x)$ et $f(x) = \sin^2(x)$
\end{tabular}
\end{center}
\label{9Q9}
\end{question}

\begin{question}
Trouvez les séries de MacLauren de
$\displaystyle f(x) = \frac{x^2}{1+x^2}$ et
$\displaystyle \int f(x)\dx{x}$.  Pour chaque série, donnez
l'intervalle de convergence.
\label{9Q10}
\end{question}

\begin{question}
Utilisez la série de MacLaurin de la fonction $\arctan(x)$ pour
trouver la série de MacLaurin de la fonction $x^2 \arctan(x^2)$.
Utilisez cette série pour obtenir une approximation de
\[
\displaystyle \int_0^{0.5} x^2 \arctan(x^2) \dx{x}
\]
avec une erreur inférieure à $10^{-6}$.
\label{9Q11}
\end{question}

\begin{question}
Exprimez sous la forme d'une série de MacLauren la function
$\displaystyle f(x) = \frac{x}{(x^2-4)^2}$ pour $|x| <2$..
\label{9Q12}
\end{question}

\begin{question}
Utilisez les série de Taylor près de l'origine pour mettre en
ordre croissance les fonctions suivantes.
\[
\ln(1+x^2) \quad , \quad \sin(x^2) \quad \text{et} \quad 1-\cos(x)
\]
\label{9Q13}
\end{question}

\begin{question}
La figure ci-dessous contient le graphe de plusieurs fonctions.
Utiliser les séries de Taylor près de l'origine pour associer chacune
des fonctions suivantes à son graphe.\\
Suggestion: Il est possible de répondre à cette question avec
l'approximation linéaire.\\
\begin{center}
\begin{tabular}{*{3}{l@{\hspace{0.5em}}l@{\hspace{4em}}}l@{\hspace{0.5em}}l}
\subQ{a} & $\displaystyle \frac{1}{1+x}$ &
\subQ{b} & $\displaystyle (1+x)^{1/8}$ &
\subQ{c} & $\displaystyle \sqrt{1+\frac{x}{2}}$ &
\subQ{d} & $\displaystyle \frac{1}{\sqrt{1+x}}$
\end{tabular}
\end{center}
\MATHgraph{9_series_ent/ordering}{8cm}
\label{9Q14}
\end{question}

\subsection{Série du binôme}

\begin{question}
Si $\displaystyle (1+x^3)^{3/5} = \sum_{n=0}^\infty b_n x^n$, quelle
est la valeur de $b_6$?
\label{9Q15}
\end{question}

\begin{question}
Utilisez la série du binôme pour obtenir la série de MacLaurin de
$\displaystyle f(x) = \sqrt{8+x^3}$.
\label{9Q16}
\end{question}

\begin{question}
Utilisez la série du binôme pour obtenir la série de
Maclaurin de $f(x) = \left(1+x^2\right)^{1/3}$.
\label{9Q17}
\end{question}

\begin{question}
Soit la fonction $\displaystyle f(x) = \frac{1}{\sqrt{1+x^3}}$.

\subQ{a} Utilisez la série du binôme pour obtenir la série de
MacLaurin de $f$.

\subQ{b} Utilisez la série que vous avez trouvée en (a) pour
estimer $\displaystyle \int_0^{0.1} f(x) \dx{x}$ avec une précision de
$10^{-8}$ (i.e.\ l'erreur est plus petite que $10^{-8}$).
\label{9Q18}
\end{question}


%%% Local Variables: 
%%% mode: latex
%%% TeX-master: "notes"
%%% End: 

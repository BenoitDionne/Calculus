\section{Dérivée de fonctions de plusieurs variables}

\subsection{Dérivées partielles}

\compileSOL{\SOLUb}{\ref{15Q1}}{
\subQ{a}
$\displaystyle f_x(x,y,z,t) = y^2z^3t^4$ car $y$, $z$ et $t$ se
comportent comme des constantes.

$\displaystyle f_y(x,y,z,t) = 2xyz^3t^4$ car $x$, $z$ et $t$ se
comportent comme des constantes.

$\displaystyle f_z(x,y,z,t) = 3xy^2z^2t^4$ car $x$, $y$ et $t$ se
comportent comme des constantes.

$\displaystyle f_t(x,y,z,t) = 4xy^2z^3t^3$ car $x$, $y$ et $z$ se
comportent comme des constantes.

\subQ{b}
$\displaystyle f_x(x,y) = \frac{2 (x-y) - (2x+y)}{(x-y)^2} =
\frac{-3y}{(x-y)^2}$ car $y$ se comporte comme une constante.

$\displaystyle f_y(x,y) = \frac{(x-y) + (2x+y)}{(x-y)^2} =
\frac{3x}{(x-y)^2}$ car $x$ se comporte comme une constante.

\subQ{c}
$\displaystyle \pdydx{h}{x} = \pdfdx{\big(f(x)g(y)\big)}{x} = f'(x)g(y)$ car
$y$ (et donc $g(y)$) se comporte comme une constante.

$\displaystyle \pdydx{h}{y} = \pdfdx{\big(f(x)g(y)\big)}{y} = f(x)g'(y)$ car
$x$ (et donc $f(x)$) se comporte comme une constante.

\subQ{d}
$\displaystyle \pdydx{h}{x} = \pdfdx{ f(xy) }{x} = f'(xy)
\pdfdx{\big(xy\big)}{x} = f'(xy)\, y$ car $y$ se comporte comme une constante.

$\displaystyle \pdydx{h}{y} = \pdfdx{ f(xy) }{y} = f'(xy)
\pdfdx{\big(xy\big)}{y} = f'(xy)\, x$ car $x$ se comporte comme une constante.

\subQ{e}
\[
\pdydx{h}{x} = \pdfdx{f\left(\frac{x}{y}\right)}{x} 
= f'\left(\frac{x}{y}\right) \pdfdx{\left(\frac{x}{y}\right)}{x}
= f'\left(\frac{x}{y}\right) \left(\frac{1}{y}\right)
\]
car $y$ se comporte comme une constante.
\[
\pdydx{h}{y} = \pdfdx{f\left(\frac{x}{y}\right)}{y} 
= f'\left(\frac{x}{y}\right) \pdfdx{\left(\frac{x}{y}\right)}{y}
= f'\left(\frac{x}{y}\right) \left(-\frac{x}{y^2}\right)
\]
car $x$ se comporte comme une constante.
}

\compileSOL{\SOLUb}{\ref{15Q4}}{
Si $g:\RR^2 \to \RR^3$ est la fonction définie par
\[
g(s,t) = \begin{pmatrix} s+t^2 \\ st \\ s^2+t \end{pmatrix} \ ,
\]
alors $w(s,t) = (f\circ g)(s,t) = f(g(s,t))$.  Ainsi,
\begin{align*}
\pdydx{w}{s}(s,t) &= \pdydx{f}{x}(g(s,t)) \pdydx{g_1}{s}(s,t)
+ \pdydx{f}{y}(g(s,t)) \pdydx{g_2}{s}(s,t)
+ \pdydx{f}{z}(g(s,t)) \pdydx{g_3}{s}(s,t) \\
&= 4 g_1(s,t) + 2t g_2(s,t) + 12s g_3^2(s,t)
= 4(s+t^2) + 2st^2 + 12s (s^2+t)^2 \ .
\end{align*}
De même,
\begin{align*}
\pdydx{w}{t}(s,t) &= \pdydx{f}{x}(g(s,t)) \pdydx{g_1}{t}(s,t)
+ \pdydx{f}{y}(g(s,t)) \pdydx{g_2}{t}(s,t)
+ \pdydx{f}{z}(g(s,t)) \pdydx{g_3}{t}(s,t) \\
&= 8t g_1(s,t) + 2s g_2(s,t) + 6 g_3^2(s,t)
= 4t(s+t^2) + 2s^2t + 6(s^2+t)^2 \ .
\end{align*}
Nous obtenons donc
\[
\pdydx{w}{s}(1,2) = 136 \quad \text{et} \quad \pdydx{w}{t}(1,2) = 98 \ .
\]
}

\compileSOL{\SOLUa}{\ref{15Q5}}{
Puisque $f_x(x,y,z) = 4x y^3 z^2 + 9x^2 y^2 z^4$ et
$f_{xy}(x,y,z) = 12 x y^2 z^2 + 18 x^2 y z^4$, nous obtenons
$f_{xyy}(x,y,z) = 24 xy z^2 + 18 x^2 z^4$ et
$f_{xyz}(x,y,z) = 24 x y^2 z + 72 x^2 y z^3$.
}

\compileSOL{\SOLUb}{\ref{15Q6}}{
Comme nous avons toujours fait, nous supposons que les coordonnées du
point $\VEC{p}$ sont $(p_1,p_2)$, les coordonnées du point $\VEC{a}$
sont $(a_1,a_2)$, etc.

\subQ{a} Puisque
\[
f_x(\VEC{p}) \cong \frac{f(\VEC{b}) - f(\VEC{p})}{b_1-p_1} =
\frac{8 - 6}{b_1-p_1} < 0
\]
car $b_1<p_1$, nous pouvons assumer que $f_x(\VEC{p}) < 0$.
Remarquons que $f(x,p_2)$ est donc une fonction décroissante pour $x$ au
voisinage de $p_1$.

\subQ{b} Puisque
\[
f_y(\VEC{p}) \cong \frac{f(\VEC{f}) - f(\VEC{p})}{f_1-p_1} =
\frac{8 - 6}{f_1-p_1} > 0
\]
car $f_1>p_1$, nous pouvons assumer que $f_y(\VEC{p}) > 0$.
Remarquons que $f(p_1,y)$ est donc une fonction croissante pour $y$ près de
$p_2$.

\subQ{c} Puisque
\begin{align*}
\frac{f(\VEC{a}) - f(\VEC{b})}{a_1-b_1} &= \frac{2}{a_1-b_1}
> \frac{f(\VEC{b}) - f(\VEC{p})}{b_1-p_1} = \frac{2}{b_1-p_1} \\
& > \frac{f(\VEC{p}) - f(\VEC{c})}{p_1-c_1} = \frac{2}{p_1-c_1}
> \frac{f(\VEC{c}) - f(\VEC{d})}{c_1-d_1} = \frac{2}{c_1-d_1}
\end{align*}
car $a_1-b_1 < b_1-p_1 < p_1-c_1 < c_1-d_1 <0$, nous pouvons supposer que
$0> f_x(\VEC{b}) > f_x(\VEC{p}) > f_x(\VEC{c}) > f_x(\VEC{d})$ et donc que
$f_x(x,p_2)$ est décroisant lorsque $x$ est près de $p_1$.  Donc
$f_{xx}(\VEC{p}) < 0$.

\subQ{d} Puisque
\begin{align*}
\frac{f(\VEC{g}) - f(\VEC{j})}{g_1-j_1} &= \frac{2}{g_1-j_1}
< \frac{f(\VEC{f}) - f(\VEC{i})}{f_1-i_1} = \frac{2}{f_1-i_1} \\
&< \frac{f(\VEC{p}) - f(\VEC{c})}{p_1-c_1} = \frac{2}{p_1-c_1}
< \frac{f(\VEC{e}) - f(\VEC{h})}{e_1-h_1} = \frac{2}{e_1-h_1} < 0
\end{align*}
car $0 > g_1-j_1 > f_1-i_1 > p_1-c_1 > e_1-h_1$, nous pouvons supposer que
$f_x(\VEC{g}) < f_x(\VEC{f}) < f_x(\VEC{p}) < f_x(\VEC{e}) < 0 $ et donc que
$f_x(p_1,y)$ est décroisant lorsque $y$ est près de $p_2$.  Donc
$\displaystyle f_{xy}(\VEC{p}) =
\pdfdx{\left( \pdydx{f}{x} \right)}{y}(\VEC{p}) < 0$.

\subQ{e} Puisque
\begin{align*}
\frac{f(\VEC{g}) - f(\VEC{f})}{g_2-f_2} &= \frac{2}{g_2-f_2}
> \frac{f(\VEC{f}) - f(\VEC{p})}{f_2-p_2} = \frac{2}{f_2-p_2} \\
&> \frac{f(\VEC{p}) - f(\VEC{e})}{p_2-e_2} = \frac{2}{p_2-e_2}
\end{align*}
car $0 < g_2-f_2 < f_2-p_2 < p_2-e_2$, nous pouvons supposer que
$f_y(\VEC{f}) > f_y(\VEC{p}) > f_y(\VEC{e}) > 0$ et donc que $f_y(p_1,y)$
est croissante lorsque $y$ est près de $p_2$.  Donc
$f_{yy}(\VEC{p}) > 0$.
}

\subsection{Plan tangent à une surface}

\compileSOL{\SOLUb}{\ref{15Q7}}{
\subQ{a} L'approximation linéaire de $z=f(x,y)$ au point $(x,y) = (40,20)$
est donnée par
\[
z = f(40,20) + f_x(40,20) (x-40) + f_y(40,20) (y-20) \ .
\]
C'est aussi l'équation du plan tangent à la courbe $z=f(x,y)$ au point
$(x,y,f(x,y)) = (40,20,f(40,20)) = (40,20,28)$.

Commençons par estimer les dérivées partielles.
\begin{equation} \label{ApprTG}
f_x(40,20) = \pdydx{f}{x}(40,20)
\approx \frac{f(45,20) - f(40,20)}{45-40} = \frac{40-28}{45-40}
= 2.4 \ .
\end{equation}
Nous pouvons aussi utiliser
\[
f_x(40,20) = \pdydx{f}{x}(40,20)
\approx \frac{f(35,20) - f(40,20)}{35-40} 
\approx \frac{25-28}{35-40} = 0.6
\]
et faire la moyenne de cette pente avec la pente obtenue en (\ref{ApprTG})
pour obtenir une nouvelle approximation $f_x(40,20) \approx 1.5$.  Il
n'y a rien qui garantit que cette moyenne donne une meilleure approximation
de $f_x(40,20)$ mais nous pouvons espérer que c'est le cas.  C'est donc
l'approximation que nous allons utiliser par la suite.

De même.
\[
f_y(40,20) = \pdydx{f}{y}(40,20)
\approx \frac{f(40,25) - f(40,20)}{25-20} = \frac{25-28}{25-20}
= -0.6 \ .
\]
Nous pourrions utiliser d'autres points pour estimer les dérivées
partielles mais nous allons nous contenter des valeurs que nous avons
trouvées.

Ainsi, l'équation du plan tangent est approximativement
\begin{equation} \label{appr_appro_lin}
z = 28 + 1.5 (x-40) - 0.6 (y-20) \ .
\end{equation}

\subQ{b}
\[
f(43,24) \approx 28 + 1.5 (43-40) - 0.6 (24-20) =  30.1 \ .
\]
Les courbes de niveau de $f$ semblent indiquer que $f(43,24)$ est une
estimation raisonnable de $f(43,24)$.  Comme notre méthode donne
seulement une approximation de l'équation du plan tangent à la courbe
$z=f(x,y)$ au point $(40,20,28)$, il est toujours prudent de vérifier
à partir de nos données que notre réponse est raisonnable.
}

\compileSOL{\SOLUb}{\ref{15Q8}}{
\subQ{a} La forme générale de l'équation du plan tangent à une surface
$z=f(x,y)$ au point $(x_0,y_0)$ est
\[
z = f(x_0,y_0) + \pdydx{f}{x}(x_0,y_0)\,(x-x_0) +
\pdydx{f}{y}(x_0,y_0)\,(y-y_0) \ .
\]
Dans le cas présent, $f(x,y) = \sin(xy)$ et $(x_0,y_0)=(1,0)$.  Ainsi,
\[
\pdydx{f}{x} = y\cos(xy) \qquad \text{et} \qquad \pdydx{f}{y} = x\cos(xy)
\]
donne
\[
\pdydx{f}{x}(1,0) = 0 \qquad \text{et} \qquad \pdydx{f}{y}(1,0) = 1 \ .
\]
De plus $f(1,0)=0$.  L'équation du plan tangent est donc
\[
z = 0 + 0\,(x-1) + 1 \, (y - 0) = y \ .
\]
}

\compileSOL{\SOLUb}{\ref{15Q9}}{
Nous avons la représentation paramétrique $\rho(u,v) = ( u - v, u + v, u^2)$.

Il faut trouver $(u,v)$ tel que $\rho(u,v) = (0,2,1)$.  Nous obtenons
de $u - v = 0$ que $u=v$.  Si nous substituons $v=u$ dans
$u+v =2$, nous obtenons $2u=2$ et donc $u = v =1$.  La relation
$u^2=1$ est bien satisfaite.

Pour trouver un vecteur normal $\VEC{m}$ à la surface $S$ au point
$(0,2,1)$, nous utilisons les deux vecteurs suivants qui sont tangent
à la surface $S$ au point $(0,2,1)$
\begin{align*}
\VEC{w}_1 &= \pdydx{\rho}{u}(1,1) = (1, 1 , 2u )\bigg|_{(u,v) = (1,1)}
= (1, 1, 2) \\
\intertext{et}
\VEC{w}_2 &= \pdydx{\rho}{v}(1,1)
= ( -1, 1, 0)\bigg|_{(u,v) = (\pi,\pi)} = (-1,1,0) \ .
\end{align*}
Ainsi,
\[
  \VEC{m} = \VEC{w}_1 \times \VEC{w}_2
= \det \begin{pmatrix}
\ii & \jj & \kk \\
1 & 1 & 2 \\
-1 & 1 & 0
\end{pmatrix} =  -2\ii - 2 \jj +2 \kk = (-2,-2, 2)
\]
où le déterminant a été développé selon la première ligne.

Ainsi, l'équation du plan tangent à la surface $S$ au point $(0,1,2)$
est donnée par
\[
  \ps{\VEC{m}}{(x,y-2,z-1)} = -2x -2(y-2) + 2(z-1) =
  -2x -2y +2z +2 = 0
\]
ou simplement $x+y-z-1=0$.
}

\subsection{Dérivées selon un direction donnée}

\compileSOL{\SOLUb}{\ref{15Q10}}{
\subQ{a} Premièrement, pour calculer la dérivée dans la direction de
$\VEC{v} = (1,\sqrt{3})$, il nous faut le vecteur unitaire $\VEC{n}$ dans
cette direction.
\[
\VEC{n} = \frac{1}{\| \VEC{v} \|} \VEC{v}
= \left(\frac{1}{2}, \frac{\sqrt{3}}{2} \right) \ .
\]
Puisque
\[
\nabla f(2,1) = (2xy, x^2 + 8y)\big|_{(2,1)}
= ( 4 , 12 ) \ ,
\]
nous obtenons
\[
\pdydx{f}{\VEC{u}}(2,1) = \nabla f(2,1) \cdot \VEC{n}
= 2 + 6\sqrt{3} \ .
\]

\subQ{b} Avant de calculer la dérivée dans la direction du vecteur
$\VEC{v}$, il faut normaliser ce vecteur.  Le vecteur $\VEC{n}$ de
longueur $1$ qui pointe dans la direction de $\VEC{v}$ est
\[
\VEC{n} = \frac{1}{\|\VEC{v}\|} \, \VEC{v} = \frac{1}{\sqrt{6}} \VEC{v}
= \left( \frac{2}{\sqrt{6}} , \frac{1}{\sqrt{6}} , -\frac{1}{\sqrt{6}}
\right) \ .
\]
Puisque
\begin{align*}
\nabla f(1,1,2) &=
\left( yz + \frac{2x}{x^2+y^2+z^2} , xz + \frac{2y}{x^2+y^2+z^2} ,
xy + \frac{2z}{x^2+y^2+z^2} \right) \bigg|_{(1,1,2)} \\
&= \left( \frac{7}{3} , \frac{7}{3} , \frac{5}{3}\right) \ .
\end{align*}
nous avons
\[
\pdydx{f}{\VEC{u}}(1,1,2) = \nabla f(1,1,2) \cdot \VEC{n} 
= \left( \frac{7}{3} , \frac{7}{3} , \frac{5}{3}\right) \cdot
\left( \frac{2}{\sqrt{6}} , \frac{1}{\sqrt{6}} , - \frac{1}{\sqrt{6}} \right)
= \frac{16}{3\sqrt{6}} \ .
\]

\subQ{c} Premièrement, pour calculer la dérivée dans la direction de
$\VEC{v} = (2,1,-2)$, il nous faut le vecteur unitaire $\VEC{n}$ dans cette
direction.
\[
\VEC{n} = \frac{1}{\| \VEC{v} \|} \VEC{v}
= \left(\frac{2}{3}, \frac{1}{3}, -\frac{2}{3} \right)
\]
Puisque
\begin{align*}
\nabla f(1,0,3) 
&= e^{xyz} \left( 1+(x+y+z)yz, 1+(x+y+z)xz, 1 +(x+y+z)xy
\right)\bigg|_{(1,0,3)}\\
&= (1, 13, 1) \quad ,
\end{align*}
nous avons
\[
\pdydx{f}{\VEC{u}} (1,0,3)
= \nabla f(1,0,3) \cdot \VEC{n} = \frac{13}{3} \ .
\]
}

\subsection{Propriétés du gradient}

\compileSOL{\SOLUb}{\ref{15Q14}}{
La surface est décrite par l'équation
$\displaystyle F(x,y,z) = x^2 -4xy + y^2 + z^2 = 2$.

Un vecteur perpendiculaire au plan tangent à cette surface au
point $(1,1,2)$ est donné par
\begin{align*}
  \nabla F(1,1,2) &= \left( \pdydx{F}{x}(x,y,z) , \pdydx{F}{y}(x,y,z) ,
\pdydx{F}{z}(x,y,z) \right) \bigg|_{(1,1,2)} \\
& = (2x-4y , -4x +2y, 2z ) \big|_{(1,1,2)} = (-2, -2, 4) \ .
\end{align*}
Ainsi, l'équation du plan tangent au point $(1,1,2)$ est donnée par
\[
0 = \ps{\nabla F(1,1,2)}{(x-1,y-1,z-2)} = -2(x-1) -2(y-1) +4(z-2)
\]
ou, plus simplement, par $2z -x -y -2=0$.
}

\compileSOL{\SOLUb}{\ref{15Q16}}{
Il faut trouver les points $(x_0,y_0,z_0)$ de l'ellipsoïde
$x^2+2y^2+3z^2 = 1$ où le plan tangent à l'ellipsoïde au point
$(x_0,y_0,z_0)$ est parallèle au plan $3x - y + 3z = 1$.  Ce qui
revient à trouver les points $(x_0,y_0,z_0)$ de l'ellipsoïde
$x^2+2y^2+ 3z^2 = 1$ où la perpendiculaire du plan tangent à
l'ellipsoïde au point $(x_0,y_0,z_0)$ est parallèle à la
perpendiculaire au plan $3x - y + 3z = 1$.

Soit $(3,-1,3)$ le vecteur formé des coefficients de $x$, $y$ et $z$
dans l'équation du plan $3x - y + 3z = 1$.  Ce vecteur
est perpendiculaire au plan $3x - y + 3z = 1$.
De plus, si $F(x,y,z) = x^2 + 2y^2 + 3z^2$, alors
$\nabla F(x_0,y_0,z_0)$ est un vecteur perpendiculaire au plan tangent
de la surface de niveau $F(x,y,z) = 1$ au point $(x_0,y_0,z_0)$ de
cette surface, donc perpendiculaire au plan tangent de l'ellipsoïde
$x^2+2y^2+ 3z^2 = 1$ au point $(x_0,y_0,z_0)$.

Il faut donc trouver les points $(x_0,y_0,z_0)$ tels que
$F(x_0,y_0,z_0) = 1$ (les points sont sur l'ellipsoïde $x^2+2y^2+3z^2 = 1$) 
et
\begin{align*}
\nabla F(x_0,y_0,z_0) &= \left(\pdydx{F}{x}(x,y,z), \pdydx{F}{y}(x,y,z),
\pdydx{F}{z}(x,y,z) \right) \bigg|_{(x,y,z)=(x_0,y_0,z_0)} \\
& = (2x , 4y, 6z) \bigg|_{(x,y,z)=(x_0,y_0,z_0)}
= (2x_0,4y_0,6z_0)
\end{align*}
est parallèle au vecteur $(3,-1,3)$.

Si $(2x_0,4y_0,6z_0)$ est parallèle au vecteur $(3,-1,3)$, alors
$(2x_0,4y_0,6z_0) = \lambda (3,-1,3)$ pour $\lambda \in \RR$.
Donc $\displaystyle x_0 = \frac{3}{2}\lambda$,
$\displaystyle y_0 = -\frac{1}{4}\lambda$ et
$\displaystyle z_0 = \frac{1}{2}\lambda$.
Si nous substituons $x=x_0$, $y=y_0$ et $z=z_0$ dans $F(x,y,z)=1$,
nous obtenons
\[
\left(\frac{3}{2}\lambda\right)^2
+ 2\left(\frac{1}{4}\lambda\right) +
+ 3\left(\frac{1}{2}\lambda\right)^2 =
\left(\frac{9}{4} + \frac{1}{8} + \frac{3}{4}\right) \lambda^2
= \frac{25}{8} \lambda^2 = 1 \ .
\]
Donc $\displaystyle \lambda = \pm \frac{2\sqrt{2}}{5}$.

Les points cherchés sont
\[
(x_0,y_0,z_0) = \pm \left( \frac{3}{2}\left( \frac{2\sqrt{2}}{5}\right),
-\frac{1}{4} \left(\frac{2\sqrt{2}}{5}\right) ,
\frac{1}{2} \left( \frac{2\sqrt{2}}{5}\right) \right)
= \pm \left( \frac{3\sqrt{2}}{5}, -\frac{\sqrt{2}}{10},
\frac{\sqrt{2}}{5} \right) \ .
\]
}

\subsection{Approximation locale des fonctions}

\compileSOL{\SOLUb}{\ref{15Q18}}{
La forme générale de l'approximation linéaire d'une fonction
$f(x,y)$ près de $(x_0,y_0)$ est
\[
p(x,y) = f(x_0,y_0) + \pdydx{f}{x}(x_0,y_0)\,(x-x_0) +
\pdydx{f}{y}(x_0,y_0)\,(y-y_0) \ .
\]

\subQ{a} L'approximation linéaire de $f$ près du point $(1,1)$ est
\begin{align*}
p(x,y) &= f(1,1) + \pdydx{f}{x}(1,1) (x-1) + \pdydx{f}{y}(1,1) (y-1) \\
&= \left(2x^2y^2+3xy+x\right)\big|_{(1,1)}
+\left(4xy^2 + 3y + 1\right)\big|_{(1,1)} (x-1) \\
& \qquad + \left(4x^2y +3x\right)\big|_{(1,1)} (y-1)
= 6 + 8(x-1) + 7(y-1) \ .
\end{align*}
Ainsi, $f(0.9, 1.1) \approx p(0.9, 1.1) = 5.9$.

\subQ{b} L'approximation linéaire de $f$ près du point $(1,1)$ est
\begin{align*}
p(x,y) &= f(1,1) + \pdydx{f}{x}(1,1) (x-1) + \pdydx{f}{y}(1,1) (y-1) \\
&= \left(\ln\left(\frac{2x^2+5y^2}{7}\right)\right)\big|_{(1,1)}
 + \left(\frac{4x}{2x^2+5y^2}\right)\big|_{(1,1)} (x-1) \\
& \qquad + \left(\frac{10y}{2x^2+5y^2}\right)\big|_{(1,1)} (y-1)
= 0 + \frac{4}{7}(x-1) + \frac{10}{7} (y-1) \ .
\end{align*}
Ainsi, $f(0.98,1.01) \approx p(0.98,1.01) \approx 0.13142857$.

\subQ{c} L'approximation linéaire de $f(x,y)$ près du point $(3,1)$ est
\begin{align*}
p(x,y) &= f(3,1) + \pdydx{f}{x}(3,1) (x - 3)
+ \pdydx{f}{x}(3,1) (y - 1) \\
&= \left((x^2-y^5)^{4/3}\right)\bigg|_{(3,1)}
+ \left(\frac{8x}{3}(x^2-y^5)^{1/3}\right)\bigg|_{(3,1)} (x-3) \\
& \qquad + \left( -\frac{20y^4}{3}(x^2-y^5)^{1/3}\right)\bigg|_{(3,1)} (y-1)
= 16 + 16 (x-3) - \frac{40}{3} (y-1) \ .
\end{align*}
Ainsi, $f(3.2,1.2) \approx p(3.2,1.2) = 16 + 16 (0.2) - (40/3) (0.2)
= 16 + 8/15$.

\subQ{d} L'approximation linéaire de $f(x,y)$ près du point $(0,1)$ est
\begin{align*}
p(x,y) &= f(0,1) + \pdydx{f}{x}(0,1) ( x- 0) +
  \pdydx{f}{y}(0,1) (y-1) \\
&= \left(1- \frac{2x}{y}+3y-4xy^2+e^{3x}\right)\bigg|_{(0,1)}
+ \left(-\frac{2}{y} - 4 y^2 + 3 e^{3x}\right)\bigg|_{(0,1)} x \\
& \qquad + \left(\frac{2x}{y^2} +3 - 8xy\right)\bigg|_{(0,1)} (y-1)
= 5 - 3 x + 3(y-1) \ .
\end{align*}
Ainsi, $f(-0.1,0.9) \approx p(-0.1,0.9) = 5 -3 (-0.1)+ 3(-0.1)
= 4.4$.
}

\subsection{Points critiques et valeurs extrêmes}

\compileSOL{\SOLUb}{\ref{15Q19}}{
Pour déterminer s'il y a un maximum local, un minimum local ou un col
en un point, il faut utiliser la matrice Hessian.  Puisque
$\displaystyle \pdydx{f}{x} = y -2xy -y^2$ et
$\displaystyle \pdydx{f}{y} = x -x^2 -2xy$, nous obtenons
\[
H(x,y) = \begin{pmatrix}
\displaystyle \pdydxn{f}{x}{2}(x,y) &
\displaystyle \pdydxnm{f}{y}{x}{2}{}{}(x,y) \\[0.9em]
\displaystyle \pdydxnm{f}{x}{y}{2}{}{}(x,y) &
\displaystyle \pdydxn{f}{y}{2}(x,y)
\end{pmatrix}
= \begin{pmatrix}
  -2y & 1-2x-2y \\
  1-2x-2y & -2x
\end{pmatrix} \ .
\]

Puisque $\displaystyle \det H(0,1)
= \det \begin{pmatrix} -2 & -1 \\ -1 & 0 \end{pmatrix} = - 1 < 0$,
il y a un col au point $(0,1)$.

Puisque $\displaystyle \det H(1/3,1/3)
= \det \begin{pmatrix} -2/3 & -1/3 \\ -1/3 & -2/3 \end{pmatrix} = 1/3 > 0$
et $\displaystyle \pdydxn{f}{x}{2}(1/3,1/3) = -2 < 0$, il y a un
minimum local au point $(1/3,1/3)$.
}

\compileSOL{\SOLUb}{\ref{15Q20}}{
\subQ{a} Comme la dérivée de la fonction $f$ existe en tout point,
les points critiques sont les racines de $\nabla f(x,y) = (0,0)$.
Puisque
\begin{align*}
\pdydx{f}{x}(x,y) = 0 &\Rightarrow 3 x^2 -3 = 0 \Rightarrow x = \pm 1
\intertext{et}
\pdydx{f}{y}(x,y) = 0 &\Rightarrow 3 y^2 -12 = 0 \Rightarrow y = \pm 2 \ ,
\end{align*}
les points critiques sont $(\pm 1,\pm 2)$.  Pour déterminer si un
point critique est associé à un maximum local, un minimum local ou un
col, nous évaluons la matrice Hessian à ce point.
\[
H(x,y) = \begin{pmatrix}
\displaystyle \pdydxn{f}{x}{2}(x,y) &
\displaystyle \pdydxnm{f}{y}{x}{2}{}{}(x,y) \\[0.9em]
\displaystyle \pdydxnm{f}{x}{y}{2}{}{}(x,y) &
\displaystyle \pdydxn{f}{y}{2}(x,y)
\end{pmatrix}
= \begin{pmatrix}
  6 x & 0 \\
  0 & 6 y
\end{pmatrix} \ .
\]

Puisque $\displaystyle \det H(1,2)
= \det \begin{pmatrix} 6 & 0 \\ 0 & 12 \end{pmatrix} = 72 > 0$
et $\displaystyle \pdydxn{f}{x}{2}(1,2) = 6 > 0$, il y a un 
un minimum local au point $(1,2)$.

Puisque $\displaystyle \det H(1,-2)
= \det \begin{pmatrix} 6 & 0 \\ 0 & -12 \end{pmatrix} = -72 < 0$,
i y a un col au point $(1,-2)$.

Puisque $\displaystyle \det H(-1,2)
= \det \begin{pmatrix} -6 & 0 \\ 0 & 12 \end{pmatrix} = -72 < 0$,
i y a un col au point $(-1,2)$.

Puisque $\displaystyle \det H(-1,-2)
= \det \begin{pmatrix} -6 & 0 \\ 0 & -12 \end{pmatrix} = 72 > 0$
et $\displaystyle \pdydxn{f}{x}{2}(-1,-2) = -6 < 0$, il y a un
maximum local au point $(-1,-2)$.

\subQ{c} Comme la dérivée de la fonction $f$ existe en tout point,
les points critiques sont les racines de $\nabla f(x,y) = (0,0)$.
Nous avons
\begin{align}
\pdydx{f}{x}(x,y) = 0 &\Rightarrow 3 x^2 + 3 y = 0
\Rightarrow y = - x^2  \label{QuestMMCQ1a}
\intertext{et}
\pdydx{f}{y}(x,y) = 0 &\Rightarrow 3 x - 3 y^2 = 0
  \Rightarrow x = y^2 \ . \label{QuestMMCQ1b}
\end{align}
Si nous substituons (\ref{QuestMMCQ1b}) dans (\ref{QuestMMCQ1a}), nous
obtenons $y = -y^4$.  Donc $y=0$ ou $1$.  Il y a deux points
critiques $(x,y) = (0,0)$ et $(1,-1)$.
Pour déterminer si un point critique est associé à un maximum local,
un minimum local ou un col, nous évaluons la matrice Hessian à ce
point.
\[
H(x,y) = \begin{pmatrix}
\displaystyle \pdydxn{f}{x}{2}(x,y) &
\displaystyle \pdydxnm{f}{y}{x}{2}{}{}(x,y) \\[0.9em]
\displaystyle \pdydxnm{f}{x}{y}{2}{}{}(x,y) &
\displaystyle \pdydxn{f}{y}{2}(x,y)
\end{pmatrix}
= \begin{pmatrix}
  6x & 3 \\
  3 & -6y
\end{pmatrix} \ .
\]

Puisque $\displaystyle \det H(0,0)
= \det \begin{pmatrix} 0 & 3 \\ 3 & 0 \end{pmatrix}  = -9 < 0$,
il y a un col au point $(0,0)$.

Puisque $\displaystyle \det H(1,-1)
= \det \begin{pmatrix} 6 & 3 \\ 3 & 6 \end{pmatrix} = 27 > 0$ et
$\displaystyle \pdydxn{f}{x}{2}(1,-1) = 6 >0$, il y a un
minimum local au point $(1,-1)$.

\subQ{e} Comme la dérivée de la fonction $f$ existe en tout point,
les points critiques sont les racines de $\nabla f(x,y) = (0,0)$.
Nous avons
\begin{align}
\pdydx{f}{x}(x,y) = 0 &\Rightarrow 6 x^2 + y^2 + 10 x = 0 \label{QuestEL1}
\intertext{et}
\pdydx{f}{y}(x,y) = 0 &\Rightarrow 2 xy + 2 y = 0
  \Rightarrow 2(x+1) y = 0 \ . \label{QuestEL2}
\end{align}
Nous obtenons $x = -1$ ou $y=0$ de (\ref{QuestEL2}).
Si $x = -1$, (\ref{QuestEL1}) donne $y^2 = 4$ et donc $y = \pm 2$.
Nous obtenons les points critiques $(-1,\pm 2)$.  Si $y = 0$,
(\ref{QuestEL1}) donne $6x(x + 5/3) = 0$ et donc $x=0$ ou
$x=-5/3$.  Nous obtenons les points critiques $(0,0)$ et $(-5/3,0)$.   
Pour déterminer si un point critique est associé à un maximum local,
un minimum local ou un col, nous évaluons la matrice Hessian à ce
point.
\[
H(x,y) = \begin{pmatrix}
\displaystyle \pdydxn{f}{x}{2}(x,y) &
\displaystyle \pdydxnm{f}{y}{x}{2}{}{}(x,y) \\[0.9em]
\displaystyle \pdydxnm{f}{x}{y}{2}{}{}(x,y) &
\displaystyle \pdydxn{f}{y}{2}(x,y)
\end{pmatrix}
= \begin{pmatrix}
  12 x + 10 & 2 y \\
  2 y & 2 x + 2
\end{pmatrix} \ .
\]

Puisque $\displaystyle \det H(-1,2)
= \det \begin{pmatrix} -2 & 4 \\ 4 & 0 \end{pmatrix} = - 16 < 0$,
il y a un col au point $(-1,2)$.

Puisque $\displaystyle \det H(-1,-2)
= \det \begin{pmatrix} -2 & -4 \\ -4 & 0 \end{pmatrix} = - 16 < 0$,
il y a un col au point $(-1,-2)$.

Puisque $\displaystyle \det H(0,0)
= \det \begin{pmatrix} 10 & 0 \\ 0 & 2 \end{pmatrix} = 20 > 0$ et
$\displaystyle \pdydxn{f}{x}{2}(0,0) = 10 >0$, il y a un
minimum local au point $(0,0)$.

Puisque $\displaystyle \det H(0,0)
= \det \begin{pmatrix} 10 & 0 \\ 0 & 2 \end{pmatrix} = 40/3 > 0$ et
$\displaystyle \pdydxn{f}{x}{2}(-5/3,0) = -10 < 0$, il y a un maximum
local au point $(-5/3,0)$.
}

\compileSOL{\SOLUb}{\ref{15Q21}}{
Puisque $f$ est une fonction continue sur l'ensemble fermé et borné
$S$, nous avons que $f$ possède un maximum absolu grâce au Théorème des
valeurs extrêmes, théorème~~\ref{TVEnD}.  

\PDFgraph{15_var_mult_der/questLM2}

Comme le maximum absolu est aussi un maximum local, le maximum absolu
de $f$ sera atteint soit à un maximum local de $f$ sur la frontière de
la région $S$ ou soit à un maximum local de $f$ à l'intérieur de la
région $S$.

Commençons par trouver les points critiques de $f$ à l'intérieur de
$S$.  Comme la dérivée de la fonction $f$ existe en tout point,
il faut résoudre $\nabla f(x,y) = (0,0)$; c'est-à-dire,
\begin{align*}
\pdydx{f}{x}(x,y) =0 &\Rightarrow 4y^2 - 2xy^2 - y^3 = 0 \Rightarrow
(4 -2x -y)y^2 = 0
\intertext{et}
\pdydx{f}{y}(x,y) = 0 &\Rightarrow 8xy -2x^2 y -3xy^2 = 0 \Rightarrow
(8-2x -3y)xy = 0 \ .                        
\end{align*}
Puisque $x\neq 0$ et $y\neq 0$ à l'intérieur de $S$, nous devons avoir
$2x + y = 4$ et $2x + 3y = 8$.  La solution de ce système d'équations
linéaires est $(x,y) = (1,2)$.  C'est bien un point à
l'intérieur de la région $S$.

Nous cherchons maintenant les points sur la frontière de la région $S$ où
$f$ pourrait atteint un maximum local.

Pour $x=0$ et $0 \leq y \leq 6$, nous avons $f(x,y) = 0$.  De même,
pour $y=0$ et $0 \leq x \leq 6$, nous avons $f(x,y) = 0$.  Si nous
restreignons $f$ au segment de droite défini par
$y = 6 - x$ et $0 \leq x \leq 6$, nous obtenons la fonction
\[
  g(x) = f(x,6-x) = 4x(6-x)^2 - x^2(6-x)^2 - x(6-x)^3
  = -2x (6-x)^2
\]
pour $0 \leq x \leq 6$.  Les points critique de $g$ sont les solutions
de $g'(x) = -2 (6-x)^2 + 4x(6-x) = 6(6-x)(x-2) = 0$.  Les deux points
critiques, $x=2$ et $x=6$, sont dans l'intervalle $[0,6]$.
Selon le Théorème des valeurs extrêmes pour $g$ sur l'intervalle
$[0,6]$, le maximum de $g$ est atteint à un des points critiques où à
une des extrémités de l'intervalle $[0,6]$.

Finalement, selon le Théorème des valeurs extrêmes pour $f$ définie
sur $S$, le maximum absolu de $f$ sur $S$ est atteint à un des points
suivants: $(0,6)$, $(2,4)$, $(6,0)$ et $(1,2)$.   Puisque $f(0,6) =
f(6,0) = 0$, $f(2,4) = -24$ et $f(1,2) = 4$, le maximum global est $4$
lorsque $(x,y) = (1,2)$.
}

\subsection{Multiplicateurs de Lagrange}

\compileSOL{\SOLUb}{\ref{15Q23}}{
\subQ{a} Utilisons la méthode des multiplicateurs de Lagrange avec
$g(x,y) = x^2 + y^2 - 2$.  L'équation
$\displaystyle \nabla f(\VEC{x}) = \lambda \nabla g(\VEC{x})$
donne
\begin{align}
\pdydx{f}{x}(x,y) = \lambda \pdydx{g}{x}(x,y) &
\Rightarrow y = 2 \lambda x \label{questLM3a}
\intertext{et}                                                
\pdydx{f}{y}(x,y) = \lambda \pdydx{g}{y}(x,y) &
\Rightarrow x = 2 \lambda y \ . \label{questLM3b}
\end{align}
Si nous substituons (\ref{questLM3b}) dans (\ref{questLM3a}), nous obtenons
\[
  y = 4 \lambda^2 y = 0  \Rightarrow (1-4\lambda^2) y = 0 \ .
\]
Il y a trois possibilités: $y=0$ , $\lambda = 1/2$ ou
$\lambda = -1/2$.

Si $y=0$, alors (\ref{questLM3b}) donne $x=0$.  Puisque $g(0,0) \neq 0$,
nous avons que $(0,0)$ n'est pas un point de notre domaine.

Si $\lambda = 1/2$, alors (\ref{questLM3b}) donne $x=y$.  Avec $x=y$,
$g(x,y)=0$ donne $2y^2 = 2$ et ainsi $y = \pm 1$.  Nous avons deux points:
$(1,1)$ et $(-1,-1)$.

Si $\lambda = -1/2$, alors (\ref{questLM3b}) donne $x=-y$.  Avec $x=-y$,
$g(x,y)=0$ donne $2y^2 = 2$ et ainsi $y = \pm 1$.  Nous avons deux points:
$(1,-1)$ et $(-1,1)$.

Puisque $f(1,1) = f(-1,-1) = 1$ et $f(1,-1) = f(-1,1) = -1$, le minimum
absolu est $-1$ aux points $(1,-1)$ et $(-1,1)$, et le maximum absolu
est $1$ au point $(1,1)$ et $(-1,-1)$.

\subQ{b} Utilisons la méthode des multiplicateurs de Lagrange avec
$g(x,y) = x^2 + y^2 - 1$.  L'équation
$\displaystyle \nabla f(\VEC{x}) = \lambda \nabla g(\VEC{x})$
donne
\begin{align}
\pdydx{f}{x}(x,y) = \lambda \pdydx{g}{x}(x,y) &
\Rightarrow 8x = 2 \lambda x \label{questLM4a}
\intertext{et}
\pdydx{f}{y}(x,y) = \lambda \pdydx{g}{y}(x,y) &
\Rightarrow 18y = 2 \lambda y \ . \label{questLM4b}
\end{align}
L'équation (\ref{questLM4a}) est satisfaite pour $x=0$.  Si nous
substituons $x=0$ dans $x^2 + y^2 =1$, nous obtenons $y = \pm 1$.
Nous avons donc deux points: $(0,1)$ et $(0,-1)$.  Pour accepter ces
deux points, il faut que (\ref{questLM4b}) soit aussi satisfait pour
ces deux points.  C'est le cas si $\lambda = 9$.

De même, l'équation (\ref{questLM4b}) est satisfaite pour $y=0$.  Si
nous substituons $y=0$ dans $x^2 + y^2 =1$, nous obtenons $x = \pm 1$.
Nous avons donc deux points: $(1,0)$ et $(-1,0)$.  Pour accepter ces
deux points, il faute que (\ref{questLM4a}) soit aussi satisfait pour
ces deux points.  C'est le cas si $\lambda = 4$.

Comme il est impossible de satisfaire (\ref{questLM4a}) et
(\ref{questLM4b}) simultanément si $x$ et $y$ sont non nuls.  Nous avons
donc trouvé tous les points d'intérêt.

Puisque $f(0, \pm 1) = 4$ et $f(\pm 1,0) = 9$, le maximum
absolu est $9$ aux points $(\pm 1,1)$ et le minimum absolu est $4$ aux
points $(0, \pm 1)$.

\subQ{c} Nous utilisons la méthode des multiplicateurs de Lagrange avec
$g(x,y,z) = x^2 + 4 y^2 + z^2 -17$.  L'équation
$\displaystyle \nabla f(\VEC{x}) = \lambda \nabla g(\VEC{x})$
donne
\begin{align*}
\pdydx{f}{x}(x,y,z) = \lambda \pdydx{g}{x}(x,y,z) &
\Rightarrow 1 = 2 \lambda x \Rightarrow x = \frac{1}{2\lambda} \ , \\
\pdydx{f}{y}(x,y,z) = \lambda \pdydx{g}{y}(x,y,z) &
\Rightarrow 3 = 8 \lambda y \Rightarrow y = \frac{3}{8\lambda}
\intertext{et}
\pdydx{f}{z}(x,y,z) = \lambda \pdydx{g}{z}(x,y,z) &
\Rightarrow -1 = 2 \lambda z \Rightarrow y = \frac{-1}{2\lambda}
\end{align*}
Si nous substituons ces valeurs de $x$, $y$ et $z$ dans $g(x,y,z) = 0$,
nous obtenons
\[
  \left(\frac{1}{2\lambda}\right)^2
  + 4 \left(\frac{3}{8\lambda}\right)^2
  + \left(\frac{-1}{2\lambda}\right)^2 = 17
  \Rightarrow \frac{17}{16\lambda^2} = 17
  \Rightarrow \lambda = \pm \frac{1}{4} \ .
\]
Nous obtenons $(x,y,z) = (2, 3/2, -2)$. pour $\lambda = 1/4$ et
$(x,y,z) = (-2, -3/2, 2)$ pour $\lambda = -1/4$.  Puisque
$f(2, 3/2, -2) = 17/2$ et $f(-2, -3/2, 2) = -17/2$, 
le maximum absolu est $17/2$ et le minimum absolu est $-17/2$.
}

\compileSOL{\SOLUb}{\ref{15Q24}}{
Soit $(x,y,z)$ un coin de la boite qui se trouve naturellement
sur l'ellipsoïde puisque nous cherchons la boite de volume maximal
connue dans l'ellipsoïde.  Nous pouvons assumer que $x,y,z \geq 0$.

\PDFgraph{15_var_mult_der/questLM4}

Il faut maximiser la fonction $f(x,y,z) = 8x y z$ sous la contrainte
que $\displaystyle g(x,y,z) = x^2 + \frac{y^2}{4} + \frac{z^2}{9} - 1 = 0$.

Utilisons la méthode des multiplicateurs de Lagrange.  L'équation
$\displaystyle \nabla f(\VEC{x}) = \lambda \nabla g(\VEC{x})$
donne
\begin{align}
\pdydx{f}{x}(x,y,z) = \lambda \pdydx{g}{x}(x,y,z) &
\Rightarrow 8yz = 2 \lambda x \label{LMQ4a} \ , \\
\pdydx{f}{y}(x,y,z) = \lambda \pdydx{g}{y}(x,y,z) &
\Rightarrow 8xz = \frac{1}{2} \lambda y \label{LMQ4b}
\intertext{et}
\pdydx{f}{z}(x,y,z) = \lambda \pdydx{g}{z}(x,y,z) &
\Rightarrow 8xy = \frac{2}{9} \lambda z \label{LMQ4c} \ .
\end{align}
Si nous soustrayons $y$ fois (\ref{LMQ4b}) de $x$ fois (\ref{LMQ4a}), nous
obtenons
\[
  2\lambda x^2 - \frac{1}{2} \lambda y^2 = 0 \Rightarrow
  y^2 = 4 x^2 \Rightarrow y = 2 x
\]
si $\lambda \neq 0$.  Nous reviendrons sur le cas $\lambda = 0$ par la
suite.
Si nous soustrayons $z$ fois (\ref{LMQ4c}) de $x$ fois (\ref{LMQ4a}), nous
obtenons
\[
  2\lambda x^2 - \frac{2}{9} \lambda z^2 = 0 \Rightarrow
  z^2 = 9 x^2 \Rightarrow z = 3 x
\]
si $\lambda \neq 0$.  Si nous substituons $y = 2 x$ et $z = 3 x$
dans l'équation de l'ellipsoïde 
$\displaystyle x^2 + \frac{y^2}{4} + \frac{z^2}{9} = 1$, nous obtenons
$3x^2 = 1$.  Donc $x = 1/\sqrt{3}$.  Cela donne le point
$\displaystyle \left( \frac{1}{\sqrt{3}}, 
\frac{2}{\sqrt{3}}, \frac{3}{\sqrt{3}} \right)$.
C'est un des huit coins d'une boite contenue dans l'ellipsoïde.

Si $\lambda = 0$, alors $(0,0,3)$, $(0,2, 0)$ et $(1, 0 ,0)$
sont les seules points de l'ellipsoïde avec des coordonnées non
négatives qui satisfont (\ref{LMQ4a}), (\ref{LMQ4b}) et (\ref{LMQ4c}).
Ils sont associés aux cas extrêmes lorsque la boite devient un des
axes de l'ellipsoïde.

Le volume maximal de $f$ est donc $\displaystyle \frac{16}{\sqrt{3}}$
aux points $\displaystyle \left( \frac{1}{\sqrt{3}}, 
\frac{2}{\sqrt{3}}, \frac{3}{\sqrt{3}} \right)$,
$\displaystyle \left( -\frac{1}{\sqrt{3}}, 
-\frac{2}{\sqrt{3}}, \frac{3}{\sqrt{3}} \right)$
$\displaystyle -\left( \frac{1}{\sqrt{3}}, 
  \frac{2}{\sqrt{3}}, -\frac{3}{\sqrt{3}} \right)$
et
$\displaystyle \left( \frac{1}{\sqrt{3}}, 
  -\frac{2}{\sqrt{3}}, -\frac{3}{\sqrt{3}} \right)$.
Le minimum absolu de $f$ est $\displaystyle -\frac{16}{\sqrt{3}}$
et est atteint aux autres coins de la boite.
}

\compileSOL{\SOLUb}{\ref{15Q25}}{
Si la base de l'aquarium est de $x$ unités par $y$ unités, et sa
hauteur est de $z$ unité, alors le coût de l'aquarium
est $C(x,y,z) = 5 xy + 2xz + 2 yz$.  Il faut minimiser $C$ sous la
contrainte que $g(x,y,z) = xyz -V = 0$, le volume de l'aquarium est
$V$.

Utilisons la méthode des multiplicateurs de Lagrange.  L'équation
$\displaystyle \nabla C(\VEC{x}) = \lambda \nabla g(\VEC{x})$
donne
\begin{align}
\pdydx{f}{x}(x,y,z) = \lambda \pdydx{g}{x}(x,y,z) &
\Rightarrow 5y + 2z = \lambda yz \Rightarrow 5y + 2z - \lambda yz = 0
\ , \label{LMAqua} \\
\pdydx{f}{y}(x,y,z) = \lambda \pdydx{g}{y}(x,y,z) &
\Rightarrow 5x + 2z = \lambda xz \Rightarrow 5x + 2z - \lambda xz = 0
\label{LMAqub}
\intertext{et}
\pdydx{f}{z}(x,y,z) = \lambda \pdydx{g}{z}(x,y,z) &
\Rightarrow 2x + 2y = \lambda xy \Rightarrow 2x + 2y - \lambda xy = 0 \ .
\label{LMAquc}
\end{align}
Si nous soustrayons $x$ fois (\ref{LMAqua}) de $y$ fois
(\ref{LMAqub}), nous obtenons $2z(y-x)=0$.  Puisque $z \neq 0$ car
$xyz=V$, nous avons $x=y$.  Si $y=x$ dans (\ref{LMAquc}), alors
$(4-\lambda x)x = 0$.  Comme précédemment, puisque $x\neq 0$
car $xyz=V$, nous avons $x = y = 4/\lambda$.  Si nous substituons cela dans
(\ref{LMAqua}) ou (\ref{LMAqub}), nous obtenons
\[
  5\, \frac{4}{\lambda} + 2 z - \lambda z \, \frac{4}{\lambda} = 0
  \Rightarrow z = \frac{10}{\lambda} \ .
\]
Finalement, si nous substituons $x = y = 4/\lambda$ et
$z = 10/\lambda$ dans $xyz=V$, nous trouvons
\[
  \left(\frac{4}{\lambda}\right)^2 \frac{10}{\lambda} = V
  \Rightarrow \lambda = \frac{2^{5/3} 5^{1/3}}{V^{1/3}} \ .
\]
Il y a donc un seul point qui satisfait
$\displaystyle \nabla C(\VEC{x}) = \lambda \nabla g(\VEC{x})$ et c'est
\[
(\tilde{x},\tilde{y},\tilde{z})
= \left( \frac{4}{\lambda}, \frac{4}{\lambda},\frac{10}{\lambda} \right)
= \left( \frac{2^{1/3} V^{1/3}}{5^{1/3}}, \frac{2^{1/3} V^{1/3}}{5^{1/3}},
\frac{5^{2/3} V^{1/3}}{2^{2/3}} \right) \ .
\]

Soit $S = \{ (x,y,z) : xyz = V \text{ pour } x,y,z >0 \}$.  Comme
l'ensemble $S$ sur lequel il faut maximiser la fonction $C$ n'est pas
un ensemble fermé et borné, le Théorème des valeurs extrêmes ne peu
pas être utilisé.  Pour déterminer si $C$ a un minimum ou un minimum
absolu au point que l'on vient de trouver, nous considérons
la fonction
\[
f(x,y) = C\left(x,y,\frac{V}{xy}\right) = 5xy + \frac{2V}{y} + \frac{2V}{x} \ .
\]
Puisque $f \to \infty$ lorsque $\|(x,y)\| \to 0$ ou lorsque
$\|(x,y)\|\to \infty$ pour $(x,y,V/(xy)) \in S$, la fonction $C$
doit donc avoir un minimum absolu au point 
$(\tilde{x},\tilde{y},\tilde{z})$.   Le coût minimal est donc
\[
C\left( \frac{2^{1/3} V^{1/3}}{5^{1/3}}, \frac{2^{1/3} V^{1/3}}{5^{1/3}},
    \frac{5^{2/3} V^{1/3}}{2^{2/3}} \right)
  = 2^{5/3} 5^{1/3} V^{1/3} \ .
\]
}

\compileSOL{\SOLUa}{\ref{15Q26}}{
Nous cherchons les points $(x,y)$ qui satisfont
$\displaystyle g(x,y) = \frac{1}{x^2} + \frac{1}{y^2} - 1 = 0$ et qui
donnent le maximum de $\displaystyle f(x,y) = \frac{1}{x} + \frac{1}{y}$.

Utilisons la méthode des multiplicateurs de Lagrange.
L'équation $\displaystyle \nabla f(\VEC{x}) = \lambda \nabla g(\VEC{x})$
donne
\begin{align}
\pdydx{f}{x}(x,y) = \lambda \pdydx{g}{x}(x,y) &
\Rightarrow -\frac{1}{x^2} = -2 \lambda \frac{1}{x^3}
\Rightarrow x = 2 \lambda
\label{questLM1a}
\intertext{et}  
\pdydx{f}{y}(x,y) = \lambda \pdydx{g}{y}(x,y) &
\Rightarrow -\frac{1}{y^2} = -2\lambda \frac{1}{x^3}
\Rightarrow  y = 2\lambda \ .
\label{questLM1b}
\end{align}
Si nous substituons (\ref{questLM1b}) et (\ref{questLM1b}) dans
$g(x,y) = 0$, nous obtenons $\displaystyle \lambda^2 = 1/2$.
Ainsi, nous avons $(x,y) = (2/\sqrt{2}, 2/\sqrt{2})$ lorsque
$\lambda = 1/\sqrt{2}$ et $(x,y) = (-2/\sqrt{2}, -2/\sqrt{2})$
lorsque $\lambda = -1/\sqrt{2}$.

Comme $f(2/\sqrt{2}, 2/\sqrt{2}) = \sqrt{2}$ et
$f(-2/\sqrt{2}, -2/\sqrt{2}) = -\sqrt{2}$, le maximum absolu
devrait être $\sqrt{2}$ au point $(2/\sqrt{2}, 2/\sqrt{2})$.
Malheureusement, l'ensemble
$E = \{ (x,y) : g(x,y) = 0$ n'est pas borné.  Nous ne pouvons donc pas
utiliser le Théorème des valeurs extrêmes, théorème~\ref{TVEnD},
pour conclure que nous avons trouvé le maximum absolu.

Nous avons dessiné l'ensemble $E$ dans la figure ci-dessous.  Ce sont
les quatre paraboles représentées par une ligne noire continue.
\PDFgraph{15_var_mult_der/questLM1}

Remarquons que $f(x,y) \to 1$ sur $E$ lorsque
$(x,y) \to (\pm \infty,1)$ avec $y>1$ ou $(x,y) \to (1,\pm\infty)$
avec $x >1$, et $f(x,y) \to -1$ sur $E$ lorsque
$(x,y) \to (\pm \infty,-1)$ avec $y<-1$ ou
$(x,y) \to (-1,\pm\infty)$ avec $x<-1$.   Nous pouvons donc utiliser le
théorème~\ref{TVEnD} sur l'ensemble fermé et borné défini
par $\displaystyle E \cap \{(x,y) : \|(x,y)\| \leq M \}$ pour $M$
assez grand.  Par exemple, nous pouvons prendre $M$ tel que
$|f(x,y)| < 1.1$ pour $(x,y) \in E$ et $\|(x,y)\| > M$.
}

\compileSOL{\SOLUb}{\ref{15Q28}}{
Puisque $f$ est une fonction continue sur l'ensemble fermé et borné
$D$, alors $f$ possède un maximum absolu grâce au Théorème des
valeurs extrêmes, théorème~\ref{TVEnD}.  

Comme le maximum absolu est aussi un maximum local, le maximum absolu
de $f$ sera atteint soit à un maximum absolu de $f$ sur la frontière du
disque $D$ ou soit à un maximum local de $f$ à l'intérieur du disque $D$. 

Commençons par trouver les points critiques de $f$ à l'intérieur de
$D$.  L'intérieur de $D$ est donné par $x^2 + y^2 < 4$.  Il faut donc
résoudre $\nabla f(x,y) = (0,0)$ pour $x^2 + y^2 < 4$.  Nous avons
\[
\pdydx{f}{x}(x,y) =0 \Rightarrow 2x = 0 \Rightarrow x = 0
\qquad \text{et} \qquad
\pdydx{f}{y}(x,y) = 0 \Rightarrow 4y +1 = 0 \Rightarrow
y = \frac{-1}{4} \ .                        
\]
Le point critique est $(0, -1/4)$ qui est bien un point à
l'intérieur du disque.

Cherchons maintenant le maximum absolu de $f$ sur la frontière du
disque $D$.  La frontière de $D$ est le cercle $x^2 + y^2 = 4$.
Posons $g(x,y) = x^2 + y^2 - 4$.  Il faut donc trouver le maximum
absolu de $f$ sous la contrainte que $g(x,y) = 0$.  Pour cela,
utilisons la méthode des multiplicateurs de Lagrange.

L'équation $\displaystyle \nabla f(\VEC{x}) = \lambda \nabla g(\VEC{x})$
donne
\begin{align}
\pdydx{f}{x}(x,y) = \lambda \pdfdx{g}{x}(x,y) &
\Rightarrow 2x = 2 \lambda x \Rightarrow 2x(1 - \lambda) = 0
\label{questLM5a}
\intertext{et}
  \pdydx{f}{y}(x,y) = \lambda \pdfdx{g}{y}(x,y) &
\Rightarrow 4y + 1 = 2 \lambda y \Rightarrow 2y(\lambda - 2) = 1 \ .
\label{questLM5b}
\end{align}
Si $\lambda \neq 1$, l'équation (\ref{questLM5a}) donne $x = 0$
L'équation $g(x,y) = 0$ avec $x=0$ donne $y^2 = 4$ et ainsi $y = \pm 2$.
L'équation (\ref{questLM5b}) est satisfaite avec $y = 1$ si
$\lambda = 1/2$, et avec $y=-1$ si $\lambda = 3/2$.  Dans les deux cas,
$\lambda$ est différente de $1$ comme il est requis.   Nous obtenons
donc deux points: $(0, \pm 2)$.

Si $\lambda = 1$, l'équation (\ref{questLM5a}) est naturellement
satisfaite et (\ref{questLM5b}) donne $y = -1/2$.
L'équation $g(x,y) = 0$ avec $y = -1/2$ donne $x^2 = 15/4$ et ainsi
$x = \pm \sqrt{15}/2$.  Nous obtenons deux autres points:
$(\pm \sqrt{15}/2, -1/2)$.

Puisque $f(0, -1/4) = 7/8$, $f(0,2) = 11$, $f(0,-2) = 7$,
$f(\sqrt{15}/2,-1/2) = 35/8$ et $f(-\sqrt{15}/2,-1/2) = 35/8$, la
maximum absolu est $11$ au point $(x,y) = (0,2)$.
}

\subsection{Dérivées des fonctions de $\mathbf{\RR^n}$ dans
 $\mathbf{\RR^m}$}

\compileSOL{\SOLUb}{\ref{15Q29}}{
\subQ{a}\\
\subQ{I} Nous avons
\[
Df(x,y) =
\begin{pmatrix}
\displaystyle \pdydx{f_1}{x}(x,y) & \displaystyle \pdydx{f_1}{y}(x,y) \\[1em]
\displaystyle \pdydx{f_2}{x}(x,y) & \displaystyle \pdydx{f_2}{y}(x,y)
\end{pmatrix}
\]
où $f_1(x,y) = x/y$ et $f_2(x,y) = 2xy$.  Ainsi,
\[
Df(x,y) =
\begin{pmatrix} 1/y & -x/y^2 \\ 2y & 2x \end{pmatrix}
\qquad \text{et} \qquad
Df(-1,1) =
\begin{pmatrix} 1 & 1 \\ 2 & -2 \end{pmatrix} \ .
\]

\subI{II} La formule générale pour l'approximation linéaire est
\[
f(x,y) \approx p(x,y) = f(x_0, y_0)
+ Df(x_0,y_0) \, \begin{pmatrix} x-x_0 \\ y-y_0 \end{pmatrix} \ .
\]
Pour $(x_0,y_0) = (-1,1)$, nous avons
\begin{align*}
f(x,y) \approx p(x,y) &= f(-1, 1)
+ Df(-1,1) \, \begin{pmatrix} x + 1 \\ y-1 \end{pmatrix} \\
&= \begin{pmatrix} -1 \\ -2 \end{pmatrix}
+ \begin{pmatrix} 1 & 1 \\ 2 & -2 \\ \end{pmatrix}
\begin{pmatrix} x+1 \\ y-1 \end{pmatrix}
= \begin{pmatrix}
-1 + x + y \\ 2 + 2x - 2y
\end{pmatrix} \ .
\end{align*}

\subI{III} La valeur exacte de $f$ au point $(-0.9, 1.05)$ est
\[
f(-0.9, 1.05) =
\begin{pmatrix}
-0.9/1.05 \\ 2(-0.9)(1.05)
\end{pmatrix}
=
\begin{pmatrix}
-0,\overline{857142} \\ -1.89
\end{pmatrix}
\]
et la valeur donnée par l'approximation linéaire de $f$
au point $(-0.9, 1.05)$ est
\[
f(-0.9, 1.05) \approx p(-0.9, 1.05) =
\begin{pmatrix}
-1 -0.9 +1.05 \\ 2 + 2(-0.9) - 2(1.05)
\end{pmatrix}
=
\begin{pmatrix}
-0.85 \\ -1.9
\end{pmatrix} \ .
\]
}

%%% Local Variables: 
%%% mode: latex
%%% TeX-master: "notes"
%%% End: 
